%%===============================================================================
%
%%xiv
\changefontsize{0.99\defaultfontsize}


In transcribing old music without bars, it is necessary to know that the ends of 
phrases and of lines of poetry are commonly expressed by notes of longer duration
than their relative value. Much of the music in Stafford Smith's \textit{Musica Antiqua} is
wrongly barred, and the rhythm destroyed by the non-observance of this rule. As
one of many instances, see “Tell me, dearest, what is love,” taken from a manuscript
of James the First’s time (\textit{Mus. Antiq}., i.~55). By carrying half the semibreve at
the end of the second bar into the third, he begins the second line of poetry (“\,'Tis
a lightning from above”) on the half-bar instead of at the commencement, and thus
falsifies the accent of that line and of all that follows. The antiquarian way would have
been, either to print the semibreve within the bar, or, which is far better, a minim with
a pause over it. In modernizing the notation, even the pause is unnecessary. Webbe
also bars incorrectly in the \textit{Convito Armonico}. For instance, in “We be three poor
mariners,” the tune is right the first time, but at the recurrence (on “Shall we go
dance the Round, the Round, the Round?”) he commences on the half-bar, because
he has given too much time to the word “ease” in the bar immediately preceding.

\bigskip
Plate 3.—“\textsc{Green Sleeves},” a tune mentioned by Shakespeare, from “Will\-iam
Ballet’s Lute Book,” described in note \textsuperscript{b} at p.~86. This is the version I have printed
at p.~230, but an exact translation of the copy will be found in my “National English
Airs,” i. 118. It is only necessary to remark that, in lute-music of the sixteenth
century, bars are placed rather to guide the eye than to divide the tune equally. The
time marked over the lines is the only sure guide for modern barring.

\bigskip
Plate 4.—“\textsc{Sellenger’s Round},” from a manuscript in the Fitzwilliam Museum,
at Cambridge, commonly known as “Queen Elizabeth’s Virginal Book.” See also
p.~71.

Dr.~Burney speaks of this manuscript first as “going under the name of Queen
Elizabeth’s Virginal Book,” and afterwards quotes it as if it had really been so.
I am surprised that he should not have discovered the error, considering that he had it
long enough in his possession to extract one of the pieces, and to give a full description
of the contents, (iii. 86, et seq.) It is now so generally known by that name,
that, for brevity’s sake, I have employed it throughout the work. Nevertheless, it
can never have been the property of Queen Elizabeth. It is written throughout in
one handwriting, and in that writing are dates of 1603, 1605, and 1612.

It is a small-sized folio volume, in red morocco binding of the time of James~I.,
elaborately tooled and ornamented with fleurs de lis, \&c., gilt edges, and the pages
are numbered to 419, of which 418 are written.

The manuscript was purchased at the sale of Dr.~Pepusch’s collection, in 1762, by
R. Bremner, the music-publisher, at the price of ten guineas, and by him given to
Lord Fitzwilliam.

Ward gives an account of Dr.~Bull’s pieces included in this virginal book, in his
\textit{Lives of the Gresham Professors}, fol., 1740, p.~203, but does not say a word of the
volume having belonged to Queen Elizabeth. We first hear of it in Dr.~Pepusch’s
possession, and, as he purchased many of his manuscripts in Holland (especially those
including Dr.~Bull’s compositions), it is by no means improbable that this English
manuscript may also have been obtained there. I am led to the conjecture by finding 
the only composer’s name invariably abbreviated is that of “Tregian.” At the commencement \pagebreak 
of Verstegan’s \textit{Restitution of decayed Intelligence}, Antwerp,~1605, is a  
%xiv
%%===============================================================================
%
%%xv
“sonnet concerning this work,” signed “Fr. Tregian,” shewing the connection of
the family with Holland, and in the virginal book one piece (No.~105, p.~196) has
only three letters of the author’s name, “Fre.” No. 60, p.~111, is “Treg. Ground;”
No. 80, p.~152, is “Pavana dolorosa, Treg.;” but No.~213, p.~315, is “Pavana
Chromatica, Mrs. Katherin Tregian’s Paven, by William Tisdall.” In the margin of
p.~312, is written, in a later hand, “R. Rysd silas.”

English music was so much in request in Holland in the early part of the seventeenth
century, that this collection of two hundred and ninety-six pieces of virginal
music may, not improbably, have been made for, or by, an English resident there,
and possibly designed as a present.

\bigskip
Plate 5.—“\textsc{The Hunt’s up},” from \textit{Musick's Delight on the Cithren}, 1666, and
“\textsc{Parth\-enia},” from a flageolet book, printed in 1682.

These are only given as specimens of musical notation. The curious will find exact
translations in \textit{National English Airs}, i. 118.
\normalsize

\vfill
{\hspace*{\fill}\rule{4em}{0.4pt}\raisebox{-3.15pt}{\rotatebox{45}{\rule{5pt}{5pt}}}\rule{4em}{0.4pt}\hspace*{\fill}}
\vfill
%xv
\pagebreak