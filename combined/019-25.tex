%===============================================================================



%025

\changefontsize{1.05\defaultfontsize}


\settowidth{\versewidth}{Bulluc sterteth, bucke verteth}
\indentpattern{0101300}
\begin{dcverse}
\begin{patverse}
Awe bleteth after lomb\\
Lhouth after calve cu;\\
Bulluc sterteth, bucke verteth\\
Murie sing Cuccu,\\
Cuccu, Cuccu.\\
Wel singes thu Cuccu\\
Ne swik thu naver nu.
\end{patverse}

\begin{patverse}
Ewe bleateth after lamb,\\
Loweth after calf [the] cow.\\
Bullock starteth,\myfootnote{\centering Jumps.} buck verteth\myfootnote{\centering Frequents the green fern} \\
Murie sing Cuccu,\\
Cuckoo, Cuckoo!\\
Well sing’st thou Cuckoo\\
Nor cease thou never now.
\end{patverse}
\end{dcverse}

In the original, the “Foot,” or Burden, is sung, as an under part by two 
voices, to the words, “Sing Cuccu, nu, sing Cuccu,” making a rude base to it.

Two other songs of the thirteenth century on the approach of Summer are
printed in Reliquiæ Antiquæ (8vo. Bond. 1841), but without music. The first
is taken from MSS. Egerton, No. 613, Brit. Mus., and begins thus:—

\settowidth{\versewidth}{“Somer is comen, and winter is gon, this day beginniz to longe [lengthen],}

\begin{scverse}
\vleftofline{“}Somer is comen, and winter is gon, this day beginniz to longe [lengthen],\\
And this foules everichon [birds every one] joy [t]hem wit[h] songe.”
\end{scverse}

The other from MSS. Digby, No. 86, Oxford, of the Thrush and the Nightingale:

\settowidth{\versewidth}{With blostme [blossom], and with brides roune [birds’ songs]}

\begin{scverse}
\indentpattern{001}
\begin{patverse}
\vleftofline{“}Somer is comen with love to toune\\
With blostme [blossom], and with brides roune [birds’ songs]\\
The note [nut] of hasel springeth,’' \&c.
\end{patverse}
\end{scverse}

In the Douce Collection (Bod. Lib., Ox., MS. No. 139), there is an English
song with music, beginning—
\settowidth{\versewidth}{Foweles in the frith, the fisses in the flod.}
\begin{scverse}
“Foweles in the frith, the fisses in the flod.”
\end{scverse}
and the MS., which contains it, is of the thirteenth century, but it is only in
two parts; and in Harl. MSS. No. 1717, is a French or Anglo-Norman song,
“Parti de Mal,” which seems to have been cut from an older manuscript to form
the cover of a Chronicle of the Dukes of Normandy, written by order of Henry II.
It is only for one voice, and a sort of hymn, but a tolerable melody. Both these
may be seen in Stafford Smith’s Musica Antiqua, Vol. 1.

Another very early English song, with music, is contained in a manu\-script,
“Liber de Antiquis Legibus,” now in the Record Room, Town Clerk’s Office,
Guildhall. It contains a Chronicle of Mayors and Sheriffs of London, and of the
events that occurred in their times, from the year 1188 to the month of August,
1274, at which time the manuscript seems to have been completed. It is the
Song of a Prisoner. The first four lines are more Saxon than modern English:—
%\medskip

\settowidth{\versewidth}{Geltles ihc sholye muchele schame}
\begin{dcverse}
\vin\vin\textsc{Original Words.}

Ar ne kuthe ich sorghe non\\
Nu ich mot manen min mon\\
Karful wel sore ich syche\\
Geltles ihc sholye muchele schame\\
Help, God, for thin swete name\\
\vin Kyng of Hevene riche.

\vin\vin\textsc{Words Modernized.}

Ere [this] knew I sorrow none\\
Now I must utter my moan\\
Full of care well sore I sigh\\
Guiltless I suffer much shame\\
Help, God, for thy sweet name,\\
\vin King of Heaven-Kingdom. \\
\end{dcverse}

%\end{fixedpage}%025
\pagebreak

