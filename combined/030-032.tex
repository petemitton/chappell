%===============================================================================
%030
\renewcommand\versoheader{english minstrelsy.}
\renewcommand\rectoheader{edward ii.}

\changefontsize{1.03\defaultfontsize}


The total sum expended was about 200\textit{l}., which according to the preceding
estimate would be equal to about 3,000\textit{l}. of our money.

The minstrels seem to have been in many respects upon the same footing as the
heralds; and the King of the Heralds, like the King at Arms, was, both here and
on the Continent, an usual officer in the courts of princes. Heralds seem even to
have been included with minstrels in the preceding account, for Carletone, who
occupies a fair position among them, receiving 1\textit{l}. as a payment, and 5\textit{s}. as a
gratuity, is in the latter case described as Carleton “Haralde.”

In the reign of Edward II., besides other grants to “King Robert,” before
mentioned, there is one in the sixteenth year of his reign to William de Morlee,
“The king’s minstrel, styled \textit{Roy de North}” of houses that had belonged to
John le Boteler, called Roy Brunhaud. So, among heralds, \textit{Norroy} was usually
styled \textit{Roy d'Armes de North} (Anstis. ii. 300), and the Kings at Arms in general
were originally called Reges Heraldorum, as these were Reges Minstrallorum.\myfootnote{\scriptsize %a
Heralds and minstrels seem to have been on nearly
the same footing abroad. For instance, Froissart tells us
“The same day th’ Erle of Foix gave to \textit{Heraudes} and
\textit{Minstrelles} the somme of fyve hundred frankes: and
gave to the Duke of Tourayn’s Minstrelles gowns of
Cloth of Gold, furred with Ermyns, valued at two hundred
franks.”—\textit{Chronicle Ed}. 1525, book 3, ch. 31.
} %end footnote
—\textit{Percy's Essay}.

The proverbially lengthy pedigrees of the Welsh were registered by their bards,
who were also heralds.\myfootnote{ %b
“The Welshman’s pedigree was his title-deed, by
which he claimed his birthright in the country. Every
one was obliged to shew his descent through nine generations,
in order to be acknowledged a free native, and by
which right he claimed his portion of land in the community. 
Among a people, where surnames were not in
use, and where the right of property depended on descent,
an attention to pedigree was indispensable. Hence arose
the second order of Bards, who were the \textit{Arwyddvierdd}, or
Bard-Heralds, whose duty it was to register arms and
pedigrees, as well as undertake the embassies of state.
The \textit{Arwyddvardd}, in early Cambrian history, was an
officer of national appointment, who, at a later period,
was succeeded by the \textit{Prydydd}, or Poet. One of these was
to attend at the birth, marriage, and death of any man of
high descent, and to enter the facts in his genealogy.
The \textit{Marwnad}, or Elegy, composed at the decease of such
a person, was required to contain truly and at length his
genealogy and descent; and to commemorate the survivor,
wife or husband, with his or her descent and progeny.
The particulars were registered in the books of the
\textit{Arwyddvardd}, and a true copy therefrom delivered to the
heir, to be placed among the authentic documents of the
family. The bard’s fee, or recompense, was a stipend
out of every plough land in the district; and he made a
triennial Bardic circuit to correct and arrange genealogical
entries.”—\textit{Extruded from Meyrick's Introduction to his
edition of Lewis Durm's Heraldic Visitations of Wales
2 vols. 4to. Llandovery}. 1846.
} %end footnote

In the reign of Edward II., \ad 1309, at the feast of the installation of Ralph,
Abbot of St. Augustin’s, at Canterbury, seventy shillings was expended on
minstrels, who accompanied their songs with the harp.— \textit{Warton}, vol. i., p.~89.

In this reign such extensive privileges were claimed by these men, and by dissolute
persons assuming their character, that it became a matter of public grievance, 
and a royal decree was issued in 1315 to put an end to it, of which the
following is an extract:—

“Edward by the grace of God, \&c. to sheriffes, \&c. greetyng, Forasmuch as\ldots\   many
idle persons, under colour of Mynstrelsie, and going in messages, and other faigned
business, have ben and yet be receaved in other mens houses to meate and drynke, and
be not therwith contented yf they be not largely consydered with gyftes of the lordes
of the houses: \&c\ldots\  We wyllyng to restrayne suche outrageous enterprises and idleness,
\&c. have ordeyned\ldots\  that to the houses of prelates, earles, and barons, none
resort to meate and drynke, unlesse he be a Mynstrel, and of these Minstrels that there
come none except it be three or four \textsc{Minstrels of honour} at the most in one day,
unlesse he be desired of the lorde of the house. \pagebreak
And to the houses of meaner men 
%\end{fixedpage}%030
%===============================================================================
%031
that none come unlesse he be desired, and that such as shall come so, holde themselves
contented with meate and drynke, and with such curtesie as the maister of the house
wyl shewe unto them of his owne good wyll, without their askyng of any thyng.
And yf any one do agaynst this Ordinaunce, at the firste tyme he to lose his \textit{Minstrelsie},
and at the second tyme to forsweare his craft, and never to be receaved for
a Minstrel in any house\ldots\  .Geven at Langley the vi. day of August, in the ix yere of
our reigne.”—\textit{Hearne’s Append. ad Leland Collect}., vol. vi., p.~36.

Stow, in his Survey of London, in an estimate of the annual expenses of
the Earl of Lancaster about this time, mentions a large disbursement for the
liveries of the minstrels. That they received vast quantities of money and costly
habiliments from the nobles, we learn from many authorities; and in a poem on
the times of Edward II., knights are recommended to adhere to their proper
costume lest they be mistaken for minstrels.

\settowidth{\versewidth}{“Kny[gh]tes schuld weare clothes}
\indentpattern{010101012003}
\begin{dcverse}
\begin{patverse}
\vleftofline{“}Kny[gh]tes schuld weare clothes\\
I-schape in dewe manere,\\
As his order wo[u]ld aske,\\
As wel as schuld a frere [friar]:\\
Now thei beth [are] disgysed,\\
So diverselych i-digt [bedight],\\
That no man may knowe\\
A mynstrel from a knyg[h]t\\
Well ny:\\
So is mekenes[s] falt adown\\
And pride aryse an hye."\\
\textit{Percy Soc., No}. 82, \textit{p}. 23.
\end{patverse}
\end{dcverse}

That minstrels were usually known by their dress, is shown by the following
anecdote, which is related by Stowe:—“When Edward II. this year (1316)
solemnized the feast of Pentecost, and sat at table in the great hall of Westminster,
attended by the peers of the realm, a certain woman, \textit{dressed in the habit
of a Minstrel, riding on a great horse, trapped in the Minstrel fashion}, entered the
hall, and going round the several tables, acting the part of a Minstrel, at length
mounted the steps to the royal table, on which she deposited a letter. Having
done this, she turned her horse, and, saluting all the company, she departed.”
The subject of this letter was a remonstrance to the king on the favors heaped
by him on his minions to the neglect of his faithful servants. The door-keepers
being called, and threatened for admitting such a woman, readily replied, “that
it never was the custom of the king’s palace to deny admission to Minstrels,
especially on such high solemnities and feast days.”

On the capital of a column in Beverley Minster, is the inscription, “Thys
pillor made the meynstyrls.” Five men are thereon represented, four in short
coats, reaching to the knee, and one with an overcoat, all having chains round
their necks and tolerably large purses. The building is assigned to the reign of
Henry VI., 1422 to 1460, when minstrelsy had greatly declined, and it cannot
therefore be considered as representing minstrels in the height of their prosperity.
They are probably only instrumental performers (with the exception, perhaps, of
the lute player); but as one holds a pipe and tabor, used only for rustic dances,
another a crowd or treble viol, a third what appears to be a bass flute, and a
fourth either a treble flute or perhaps that kind of hautboy called a wayght, or
wait, and there is no harper among them—I do not suppose any to have been of
that class called minstrels of honour, \pagebreak
who rode on horseback, with their servants 
%\end{fixedpage}%031
%===============================================================================
%32
 to attend them, and who could enter freely into a king’s palace. Such distinctions
among minstrels are frequently drawn in the old romances. For instance, in the
romance of Launfel we are told, “They had menstralles of moche honours,” and
also that they had “Fydelers, sytolyrs (citolers), and trompoteres.” It is not,
however, surprising that they should be rich enough to build a column of a
Minster, considering the excessive devotion to, and encouragement of, music which
characterised the English in that and the two following centuries.

No poets of any country make such frequent and enthusiastic mention of minstrelsy
as the English. There is scarcely an old poem but abounds with the
praises of music. Adam Davy, or Davie, of Stratford-le-Bow, near London,
flourished about 1312. In his Life of Alexander, we have several passages like~this:—
\settowidth{\versewidth}{The mynstrall synge, the jogelour carpe” (recite).}
\begin{scverse}
\vleftofline{“}Mer[r]y it is in halle to he[a]re the harpe,\\
The mynstrall synge, the jogelour carpe” (recite).
\end{scverse}
And again,—
\settowidth{\versewidth}{“Mery is the twynkelyng of the harpour.”}
\begin{scverse}
“Mery is the twynkelyng of the harpour.”
\end{scverse}
The fondness of even the most illiterate, to hear tales and rhymes; is much
dwelt on by Robert de Brunne, or Robert Mannyng, “the first of our vernacular
poets who is at all readable now.” All rhymes were then sung with accompaniment, 
and generally to the harp. So in 1338, when Adam de Orleton, bishop of
Winchester, visited his Cathedral Priory of St. Swithin, in that city, a minstrel
named Herbert was introduced, who sang the \textit{Song of Colbrond}, a Danish Giant,
and the tale of \textit{Queen Emma delivered from the plough-shares}, or trial by fire, in
the hall of the Prior. A similar festival was held in this Priory in 1374, when
similar gestes or tales were sung. Chaucer’s Troilus and Cresseide, though almost
as long as the Æneid, was to be “redde, or else songe,” and Warton has printed
a portion of the Life of St. Swithin from a manuscript, with points and accents
inserted, both over the words and dividing the line, evidently for the purposes of
singing or recitation (\textit{History of English Poetry}, vol. i., p.~15. 1840). We have
probably by far more tunes that are fitted for the recitation of such lengthy stories
than exist in any other country.

In the year 1362, an Act of Parliament passed, that “all pleas in the court
of the king, or of any other lord, shall be pleaded and adjudged in the English
tongue” (stat. 36 Edw. III., cap.~15); and the reason, which is recited in the
preamble, was, that the French tongue was so unknown in England that the
parties to the law-suits had no knowledge or understanding of what was said for
or against them, because the counsel spoke French. This was the era of Chaucer,
and of the author of Pierce Plowman—two poets whose language is as different as
if they had been born a century apart. Longland, instead of availing himself of the
rising and rapid improvements of the English language, prefers and adopts the style
of the Anglo-Saxon poets, even prefering their perpetual alliteration to rhyme.
His subject—a satire on the vices of the age, but particularly on the corruptions
of the clergy and the absurdities of superstition—does not lead him to say much
of music, but he speaks of ignorance of the art as a just subject of reproach.
\settowidth{\versewidth}{“They kennen [know] no more mynstralcy, ne musik, men to gladde,}
\begin{scverse}
\vleftofline{“}They kennen [know] no more mynstralcy, ne musik, men to gladde,\\
Than Mundy the muller [miller], of \textit{multa fecit Deus}!’’ 
\end{scverse}

\pagebreak

