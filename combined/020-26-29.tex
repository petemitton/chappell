%26
\renewcommand\versoheader{national songs not on church scales.}
\renewcommand\rectoheader{church music always in arrear.}
\changefontsize{1.05\defaultfontsize}



In the Arundel Collection (No. 292), there is a song in “a handwriting of
the time of Edward II.,” beginning—
\settowidth{\versewidth}{“Uncomly in cloystre I coure [cower] ful of care,”}
\begin{scverse}
“Uncomly in cloystre I coure [cower] ful of care,”
\end{scverse}
which is on the comparative difficulties of learning secular and church music,
but, except in the line, “Thou bitest asunder bequarre for bemol” (B natural
for B flat), there is no reference to the practice of music.

Secular music must have made considerable progress before the end of the
thirteenth century, for even Franco had spoken of a sort of composition called
“Conductus,” in which, instead of merely adding parts to a plain song, the student
was first to compose as pretty a tune as he could, and then to make descant
upon it;\myfootnote{ %a
“In Couductis aliter est operandum, quia qui vult
facere Conductum, primum cantum invenire debet pulchriorem
quam potest, deinde uti debet illo, ut de tenore,
faciendo discantura.”
} %end footnote
and he further says, that in every other case, some melody already made
is chosen, which is called the tenor, and governs the descant originating from it:
but it is different in the Conductus, where the cantus (or melody) and the descant
(or harmony) are both to be produced. This was evidently applied to secular
composition, since, about 1250, Odo, Archbishop of Rheims, speaks of Conducti et
Motuli as “jocose and scurrilous songs.”

Accidental sharps, discords and their resolutions, and even chromatic counterpoint,
are treated on by Marchetto of Padua (in his Romerium Artis Musicæ
Mensurabilis) in 1274, and the Dominican Monk, Peter Herp, mentions in
Chronicle of Frankfort, under the year 1300, that new singers, composers, and
harmonists had arisen, who used other scales or modes than those of the Church.\myfootnote{ %b
“Novi cantores surrexere, et componistæ, et figuristæ.
qui inceperunt alios modos assuere.” When music deviated
from the Church scales, it was called by the old
writers generally, \textit{Musica falsa}, and by Franchinus,
\textit{Musica ficta, seu colorata}, from the chromatic semitones
used in it.
} %end footnote
Pope John XXII. (in his decree given at Avignon in 1322) reproves those who,
“attending to the \textit{new notes and new measures of the disciples of the new school},
would rather have their ears tickled with semibreves and minims, and \textit{such frivolous}
inventions, than hear the ancient ecclesiastical chant.” White minims, with tails,
to distinguish them from semibreves, seem first to have been used by John de
Muris, about 1330, retaining the lozenge-shaped head to the note. He also used
signs to distinguish triple from common time. These points should be borne in
mind in judging of the age of manuscripts.

It will be observed that “Sumer is icumen in” is not within the compass of
any Church scale. It extends over the octave of F, and ends by descending to the
seventh below the key note for the close, which, indeed, is one of the most common
and characteristic terminations of English airs. The dance tune which follows
next in order has the same termination, and extends over a still greater compass
of notes. I shall therefore quit the subject of Church scales, relying on the
practical refutation which a further examination of the tunes will afford. Burney
has remarked that at any given period secular music has always been at least a
century in advance of Church music. And notwithstanding the improvements
in musical \textit{notation} made by monks, the Church still adhered to her imperfect 
system, as well as to bad harmony, for \pagebreak
 centuries after better had become general.  
%\end{fixedpage}%026
 %===============================================================================
%027
Even in the sixteenth century, modulation being still confined to the ecclesiastical
modes, precluded the use of the most agreeable keys in music. Zarlino, who
approved of the four modes added by Glareanus, speaks of himself, and a few
others, having composed in the eleventh mode, or key of C natural (which was not
one of the original eight), to which they were led \textit{by the vulgar musicians of the
streets and villages}, who generally accompanied rustic dances with tunes in this
key, and which was then called, \textit{Il modo lascivo}—The wanton key. I suppose it
acquired this name, because, like the “sweet Lydian measure” of old, the interval
from the seventh to the octave is only a semitone.

\musictitle{Dance Tune.}

\musicinfo{}{About 1300.}

\lilypondfile[staffsize=18]{lilypond/027-dance-tune}


The above dance tune is taken from the Musica Antiqua by John Stafford
Smith. He transcribed it from a manuscript then in the possession of Francis
Douce, Esq. (who bequeathed the whole of his manuscripts to the Bodleian
Library), and calls it, “a dance tune of the reign of Edward II., or earlier.”
The notation of the MS. is the same \pagebreak
as in that which contains \textit{Sumer is icumen in}, 
%\end{fixedpage}%027
%===============================================================================
%028
and I do not think it can be dated later than 1300. Dr: Crotch remarks:—
“The abundance of appoggiaturas in so ancient, a melody, and the number of bars
in the phrases, four in one and five in another—nine in each part, are its most
striking peculiarities. It is formed on an excellent design, similar to that of
several fine airs of different nations. It consists of three parts, resembling each
other excepting in the commencement of their phrases, in which they tower above
each other with increasing energy, and is altogether a curious and very favorable
specimen of the state of music at this very early period.”

\renewcommand\versoheader{english minstrelsy resumed.}
\renewcommand\rectoheader{edward i.}
\DFNsingle

The omission of the eighth bar in each phrase would make it strictly in modern~rhythm.


\section*{CHAPTER III.}

\subsection{English Minstrelsy from 1270 to 1480, and the gradual extinction
of the old Minstrel.}

%%\changefontsize{0.86\defaultfontsize}
Edward the First, according to the Chronicle of Walter Hemmingford, about the
year 1271, a short time before he ascended the throne, took his harper with him
to the Holy Land, who must have been a close and constant attendant on his
master, for when Edward was wounded at Ptolemais, the harper (Citharæda
suus), hearing the struggle, rushed into the royal apartment, and, striking the
assassin on the head with a tripod or trestle, beat out his brains.

“That Edward ordered a massacre of the Welsh bards,” says Sharon Turner,
“seems rather a vindictive tradition of an irritated nation than an historical fact.
The destruction of the independent sovereignties of Wales abolished the patronage
of the bards, and in the cessation of internal warfare, and of external ravages,
they lost their favorite subjects, and most familiar imagery. They declined
because they were no longer encouraged.” The Hon. Daines Barrington could
find no instances of severity against the Welsh in the laws, \&c. of this monarch,\myfootnote{ %a
\centering See his observations on the statutes, 4to. 4th Ed.
} %end footnote
and that they were not extirpated is proved by the severe law which we find in
the Statute Book, 4 Henry IV. (1402), c. 27, passed against them during the
resentment occasioned by the outrages committed under Owen Glendour. In that
act they are described as Rymours and Ministralx, proving that our ancestors
could not distinguish between them and our own minstrels.

\DFNdouble

In May, 1290, was celebrated the marriage of Queen Eleanor’s daughter Joan,
surnamed of Acre, to the Earl of Gloucester, and in the following July, that of
Margaret, her fifth daughter, to John, son of the Duke of Brabant. Both ceremonies
were conducted with much splendour, and a multitude of minstrels flocked
from all parts to Westminster: to the first came King Grey of England, King
Caupenny from Scotland, and Poveret, the minstrel of the Mareschal of Champagne.
The nuptials of Margaret, however, seem to have eclipsed those of her sister. 
Walter de Storton, the king’s harper, \pagebreak
distributed a hundred pounds, the gift of 
%\end{fixedpage}%028
%===============================================================================
%029
the bridegroom, among 426 minstrels, as well English as others.\myfootnote{ %
 Pages lxix. and lxx. Introduction to Manners and
Household Expenses of England in the 13th and 15th
centuries, illustrated by original records. 4to. London.
Printed for the Roxburghe Club, 1841, and quoted from
Wardrobe Book, 18 Edward I. Rot. Miscell. in Turr.
Lond. No. 56.
} %end footnote
 In 1291, in the
accounts of the executors of Queen Eleanor, there is an entry of a payment of
39\textit{s}., for a cup purchased to be given to one of the king’s minstrels.



The highly valuable roll, preserved among the records in the custody of the
Queen’s Remembrancer, which has been printed for the Roxburghe Club, marks
the gradations of rank among the minstrels, and the corresponding rewards
bestowed upon them. It contains the names of those who attended the \textit{cour
plenière} held by King Edward at the Feast of Whitsuntide, 1306, at Westminster,
and also at the New Temple, London; because “the royal palace,
although large, was nevertheless small for the crowd of comers.” Edward then
conferred the honor of knighthood upon his son, Prince Edward, and a great
number of the young nobility and military tenants of the crown, who were summoned
to receive it, preparatory to the King’s expedition to Scotland to avenge
the murder of John Comyn, and the revolt of the Scotch.

On this occasion there were six kings of the minstrels, five of whom, viz.,
Le Roy de Champaigne, Le Roy Capenny, Le Roy Boisescue, Le Roy Marchis,
and Le Roy Robert, received each five marks, or 3\textit{l}. 6\textit{s}. 8\textit{d}., the mark being
13\textit{s}. 4\textit{d}. It is calculated that a shilling in those days was equivalent to fifteen
shillings of the present time; according to which computation, they received 50\textit{l}.
each. The sixth, Le Roy Druet, received only 2\textit{l}. The list of money \textit{given} to
minstrels is principally in Latin; but that of \textit{payments} made to them being in
Norman French, it is difficult to distinguish English minstrels from others. Le
Roy de Champaigne was probably “Poveret, the minstrel of the Mareschal of
Champagne,” of 1290, Le Roy Capenny, “King Caupenny from Scotland,” and
Le Roy Robert, whom we know to have been the English king of the minstrels
by other payments made to him by the crown (see Anstis’ Register of the Order
of the Garter, vol. ii. p.~303), was probably the “King Grey of England” of
the former date. Among the names we find, Northfolke, Carletone, Ricard de
Haleford, Adam de Werintone (Warrington?), Adam de Grimmeshawe, Merlin,
Lambyn Clay, Fairfax, Hanecocke de Blithe, Richard Wheatacre, \&c. The
harpers are generally mentioned only by their Christian names, as Laurence,
Mathew, Richard, John, Robert, and Geoffrey, but there are also Richard de
Quitacre, Richard de Leylonde, William de Grimesar, William de Duffelde, John
de Trenham, \&c., as well as Adekyn, harper to the Prince, who was probably
a Welsh bard. In these lists only the principal minstrels are named, the remaining
sum being divided, by the kings and few others, among the \textit{menestraus de la
commune}. Harpers are in the majority where the particular branch of minstrelsy
is specified. Some minstrels are locally described, as Robert “de Colecestria,”
John “de Salopia,” and Robert “de Scardeburghe;” others are distinguished
as the harpers of the Bishop of Durham, Abbot of Abyngdon, Earls of Warrenne,
Gloucester, \&c.; one is Guillaume sans manière; another, Reginald le menteur;
a third is called Makejoye; and a fourth, Perle in the eghe. 
%\end{fixedpage}%029
\pagebreak


