%===============================================================================
%037
\changefontsize{1.03\defaultfontsize}

\renewcommand\rectoheader{gower.--richard ii.}
I shall conclude these numerous extracts with one of the song of nature, from
the Knighte’s Tale, (line 1493 to 98):—

\settowidth{\versewidth}{And with his stremès dryeth in the greves [groves]}
\begin{scverse}
\vleftofline{“}The busy larkè, messager of daye,\\
Salueth in hire song the morwe [morning] gray;\\
And fyry Phebus ryseth up so bright,\\
That al the orient laugheth of the light,\\
And with his stremès dryeth in the greves [groves]\\
The silver dropès, hongyng on the leeves.”
\end{scverse}
Having quoted so largely from Chaucer, whose portraiture of character and
persons has never been excelled, it will be unnecessary to refer to his contemporary, Gower, further than to say that in his \textit{Confessio Amantis}, Venus greets
Chaucer as her disciple and poet, who had filled the land in his youth with
dittees and “songès glade,” which he had made for her sake; and Gower says of
himself:—
\settowidth{\versewidth}{For her on whom myn hert laie.”}
\begin{scverse}
\vleftofline{“}And also I have ofte assaide\\
Roundel, Balades, and Virelaie\\
For her on whom myn hert laie.”
\end{scverse}
But about the same time, in the Burlesque Romance, The T[o]urnament of
Tottenham (written in ridicule of chivalry), we find a notice of songs in six parts
which demands attention. In the last verse:—
\settowidth{\versewidth}{Mekyl mirth was them among;}
\begin{scverse}
\vleftofline{“}Mekyl mirth was them among;\\
In every corner of the hous\\
Was melody delycyous\\
For to he[a]re precyus\\
\vin\vin Of six menys song.”
\end{scverse}

It has been supposed that this is an allusion to \textit{Sumer is icumen in}, which
requires six performers, but in all probability there were many such songs,
although but one of so early a date has descended to us. We find in the Statutes
of New College, Oxford (which was founded about 1380), that William of
Wykeham ordered his scholars to recreate themselves on festival days with songs
in the hall, both after dinner and supper; and as part-music was then in common
use, it is reasonable to suppose that the founder intended the students thereby to
combine improvement and recreation, instead of each singing a different~song.

In the fourth year of king Richard II. (1381), John of Gaunt erected at
Tutbury, in Staffordshire, a \textit{Court of Minstrels} similar to that annually kept at
Chester; and which, like a court-leet, or court-baron, had a legal jurisdiction,
with full power to receive suit and service from the men of this profession within
five neighbouring counties, to determine their controversies and enact laws; also
to apprehend and arrest such of them as should refuse to appear at the said court,
annually held on the 16th of August. For this they had a charter, by which
they were empowered to appoint a King of the Minstrels, with four officers to
preside over them. They were every year elected with great ceremony; the
whole form of which, as observed in 1680, is described by Dr.~Plot in his History
of Staffordshire. That the barbarous diversion of bull-running was no part of the
original institution, is fully proved by the Rev. Dr.~Pegge, in Archæologia, vol. ii.,
No. xiii., p.~86. The bull-running tune, however, is still popular in~Staffordshire. 

\pagebreak
