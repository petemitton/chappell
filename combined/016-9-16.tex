%===============================================================================

%009
\renewcommand\rectoheader{william i. to richard i.}
\changefontsize{1.01\defaultfontsize}

After the Conquest, the first notice we have relating to the Minstrels is the 
founding of the Priory and Hospital of St. Bartholo\-mew,\myfootnote{ %a
Vide the \textit{Monasticon}, tom. ii. pp.~166-67, for a curious
history of this priory and its founder. Also \textit{Stowe's Survey}. 
In the \textit{Pleasaunt History of Thomas of Reading}, 4to.
1662, he is likewise mentioned. His monument, in good
preservation, may yet be seen in the parish church of
St. Bartholomew, in Smithfield, London.
} %end footnote 
in Smithfield, by
Royer, or Raherus, the King’s Minstrel, in the the third year of King Henry I.,
\ad 1102. Henry’s conduct to a luckless Norman minstrel who fell into his power,
tells how keenly the minstrel’s sarcasms were felt, as well as the ferocity of Henry’s
revenge. “Luke de Barre,” said the king, “has never done me homage, but he has
fought against me. He has composed facetiously indecent songs upon me; he has
sung them openly to my prejudice, and often raised the horse-laughs of my malignant
enemies against me.” Henry then ordered his eyes to be pulled out. The
wretched minstrel rushed from his tormentors, and dashed his brains against
the wall.\myfootnote{ %b
Quoted from Ordericus Vitalis. Hist. Eccles. in Sharon
Turner's Hist. England.
} %end footnote


In the reign of King Henry II., Galfrid or Jeffrey, a harper, received in 1180
an annuity from the Abbey of Hide, near Winchester; and as every harper was
expected to sing,\myfootnote{ %
So in Horn-Child, K. Allof orders his steward,
Althebrus to “teche him of harpe and song.” And
Chaucer, in his description of the Limitour or Mendicant
Friar, speaks of harping as inseparable from singing—“in
his harping, when that he had sung.” Also in 1481, see
Lord Howard's agreement with William Wastell, Harper
of London, to teach a boy named Colet “to harp and to sing.”
} %end footnote
we cannot doubt that this reward was bestowed for his music
and his songs, which, as Percy says, if they were for the solace of the monks there,
we may conclude would be in the English language. The more rigid monks,
however, both here and abroad, were greatly offended at the honours and rewards
lavished on Minstrels. John of Salisbury, who lived in this reign, thus declaims
against the extravagant favour shown to them: “For \textit{you} do not, like the fools of
this age, pour out rewards to Minstrels (Histriones et Mimos\myfootnote{ %
Histrio, Mimus, Joculator, and Ministrallus, are all
nearly equivalent terms for Minstrels in Mediaeval Latin.
“Incepit \textit{more Histrionico}, fabulas dicere, et plerumque
cantare.” “Super quo \textit{Histriones cantabant}, sicut modo
cantatur de Rolando et Oliverio.” “Dat sex \textit{Mimis}
Domini Clynton, \textit{cantantibus}, \textit{citharisantibus}, ludentibus,” 
\&c. 4 s. Geoffrey of Monmouth uses \textit{Joculator} as
equivalent to \textit{Citharista}, in one place, and to \textit{Cantor} in
another. See Notes to Percy’s Essay.
} %end footnote
)and monsters of
that sort, for the ransom of your fame, and the enlargement of your name.”
—(\textit{Epist}. 247.)

“Minstrels and Poets abounded under Henry’s patronage: they spread the love
of poetry and literature among his barons and people, and the influence of the
royal taste soon became visible in the improved education of the great, in the
increasing number of the studious, and in the multiplicity of authors, who wrote
during his reign and the next.”—\textit{Sharon Turner’s Hist. Eng}.

In the reign of Richard I. (1189.) minstrelsy flourished with peculiar splendour.
His romantic temper, and moreover his own proficiency in the art, led him to be
not only the patron of chivalry, but also of those who celebrated its exploits.
Some of his poems are still extant. The romantic release of this king from the
castle of Durrenstein, on the Danube, by the stratagem and fidelity of his Minstrel
Blondel, is a story so well known, that it is needless to repeat it here.\myfootnote{ %
The best authority for this story, which has frequently
been doubted, is the Chronique de  Rains, written in the 13th Century.—See 
\textit{Wright's Biograph,Brit., Anglo Norman  p.~325.}
%TODO lost word -- Period
} %end footnote



Another circumstance which proves how easily Minstrels could always gain
admittance even into enemies’ camps and prisons, occurred in this reign. The
young heiress of D’Evreux, Earl of Salisbury, \pagebreak
“was carried abroad, and secreted
%===============================================================================
%010 
by her French relations in Normandy. To discover the place of her concealment, 
a knight of the Talbot family spent two years in exploring that province, at first
under the disguise of a pilgrim; till having found where she was confined, in
order to gain admittance he assumed the dress and character of a harper, and
being a \textit{jocose} person, exceedingly skilled in ‘the Gests of the Ancients,’—so they
called the romances and stories which were the delight of that age,—he was gladly
received into the family, whence he took an opportunity to carry off the young
lady, whom he presented to the king; and he bestowed her on his natural brother,
William Longespee (son of fair Rosamond), who became, in \textit{her} right, Earl of~Salisbury.

In the reign of king John (\ad 1212) the English Minstrels did good service
to Ranulph, or Randal, Earl of Chester. He, being beseiged in his Castle of
Rothelan (or Rhuydland), sent for help to De Lacy, Constable of Chester, who,
“making use of the Minstrels of all sorts, then met at Chester fair, by the allurements
of their music, assembled such a vast number of people, who went forth
under the conduct of a gallant youth, named Dutton (his steward and son-in-law)
that he intimidated the Welsh, who supposed them to be a regular body of armed
and disciplined soldiers, so that they instantly raised the siege and retired.”

For this deed of service to Ranulph, both De Lacy and Dutton had, by
respective charters, patronage and authority over the Minstrels and others, who,
under the descendants of the latter, enjoyed certain privileges and protection for
many~ages.

Even so late as the reign of Elizabeth, when this profession had fallen into such
discredit that it was considered in law a nuisance, the Minstrels under the jurisdiction
of the family of Dutton are expressly excepted out of all acts of Parliament
made for their suppression; and have continued to be so excepted ever since.\myfootnote{ %
See the statute of Eliz. anno. 39. cap.~iv. entitled an
Act for punishment of rogues, vagabonds, \&c.; also a renewal
of the same clauses in the last act on this subject,
passed in the reign of George~III. The ceremonies
attending the exercise of this jurisdiction are described
by Dugdale (Bar~i..~p.~101), and from him, by Percy.
} %end footnote

“We have innumerable particulars of the good cheer and great rewards given to
the Minstrels in many of the convents, which are collected by Warton and others.
But one instance, quoted from Wood’s Hist. Antiq. Ox., vol. i. p.~67, during the reign
of king Henry III. (sub. an. 1224), deserves particular mention. Two itinerant
priests, on the supposition of their being Minstrels, gained admittance. But the
cellarer, sacrist, and others of the brethren, who had hoped to have been entertained
by their diverting arts, \&c., when they found them to be only two indigent ecclesiastics,
and were consequently disappointed of their mirth, beat them, and turned
them out of the monastery.”

In the same reign (\ad 1252) we have mention of Master Richard, the king’s
Harper, to whom that monarch gave not only forty shillings and a pipe of wine,
but also a pipe of wine to Beatrice, his wife. Percy remarks, that the title of
Magister, or Master, given to this Minstrel, deserves notice, and shows his
respectable~situation.



“The learned and pious Grosteste, bishop \pagebreak 
of Lincoln who died in 1253, is said, 
%===============================================================================
%011
\renewcommand\rectoheader{king john to edward i.}
in some verses of Robert de Brunne, who flourished about the beginning of the 
next century, to have been very fond of the metre and music of the Minstrels.
The good prelate had written a poem in the Romanse language, called \textit{Manuel
Peche}, the translation of which into English, Robert de Brunne commenced in~1302, 
with a design, as he tells us himself, that it should be sung to the harp at
public entertainments.”

\DFNsingle

\settowidth{\versewidth}{Hys Harper’s chaumbre was fast ther}
\begin{dcverse} For lewde [unlearned] men I undertoke\\
In Englysshe tunge to make thys boke,\\
For many ben of swyche manere\\

That talys and rymys wyl blithly here,\\
Yn gamys and festys, and at the ale\\
Love men to listene trotevale. [triviality]
\end{dcverse}


The following anecdote concerning the love which his author, bishop
Gros\-teste, had for music, seems to merit a place here, though related in rude~rhymes.


\settowidth{\versewidth}{Hys Harper’s chaumbre was fast ther}

\begin{dcverse} I shall yow telle as I have herde\\
Of the bysshope Seynt Roberde,\\
Hys toname [surname] is Grostest\\
Of Lynkolne, so seyth the gest,\\
He loved moche to here the Harpe,\\
For mannes wytte it makyth sharpe.\\
Next hys chaumbre, besyde his study,\\
Hys Harper’s chaumbre was fast therby.\\
Many tymes, by nightes and dayes,\\
He had solace of notes and layes,\\
One askede hym the resun why\\
He hadde delyte in Mynstralsy?\\
He answerde hym on thys manere\\
Why he helde the Harpe so dere:\\
\columnbreak

\vleftofline{“}The vertu of the Harpe, thurgh\\
\vin\vin\vin [through] skylle and ryght,\\
\vleftofline{“}Wyll destrye the fendys [fiends] myght;\\
\vleftofline{“}And to the Cros by gode skylle\\
\vleftofline{“}Is the Harpe ylykened weyl.\\
\vleftofline{“}Tharefore, gode men, ye shall lere, [learn]\\
\vleftofline{“}Whan ye any Gleman here,\\
\vleftofline{“}To wurschep God at your powere,\\
\vleftofline{“}As Davyd seyth in the Santere. [Psalter]\\
\vleftofline{“}In harpe and tabour and symphan\myfootnote{ %a
\centering Either part-singing, or the instrument called the symphony.
} %end footnote
gle\\
\vleftofline{“}Wurschep God: in trumpes and sautre,\\
\vleftofline{“}In cordes, in organes, and bells ringyng:\\
\vleftofline{“}In all these wurschepe the hevene Kyng, \&c.”
\end{dcverse}\normalsize


Before entering on the reign of Edward I., I quit the Minstrels for awhile, to
endeavour to trace the progress of music up to that period. It will be necessary
to begin with the old Church Scales, it having been asserted that all national
music is constructed upon them—an assertion that I shall presently endeavour
to confute; and by avoiding, as far as possible, all obsolete technical, as well
as Greek terms, which render the old treatises on Music so troublesome a study,
I hope to convey such a knowledge of those scales as will answer the purpose of
such general readers as possess only a slight knowledge of music.

\section*{CHAPTER II.}

\subsection{Music of the Middle Ages.—Music in England to the end of
the Thirteenth Century.}

During the middle ages Music was always ranked, as now, among the seven
liberal arts, these forming the \textit{Trivium} and \textit{Quadrivium}, and studied by all
those in Europe who aspired at reputation for learning. The Trivium comprised
Grammar, Rhetoric, and Logic; the Quadrivium comprehended Music, 
\pagebreak
%===============================================================================
%012
Arithmetic, Geometry, and Astronomy. Sharon Turner remarks, that these 
comprised not only all that the Romans knew, cultivated, or taught, but
embodied “the whole encyclopaedia of ancient knowledge.” If we may trust
the following jargon hexameters, which he quotes as “defining the subjects
they comprised,” Music was treated as an art rather than as a science, and
a practical knowledge of it was all that was required:—

\renewcommand\versoheader{music of the middle ages.}
\renewcommand\rectoheader{gregorian tones.}
\DFNdouble
\settowidth{\versewidth}{\textit{Mus. canit}; Ar. numerat; Geo. ponderat; Ast. colit astra.}
\begin{quotation}\small
\noindent Gramm, loquitur; Dia. vera docet; Rhet. verba colorat\\
\textit{Mus. canit}; Ar. numerat; Geo. ponderat; Ast. colit astra.
\end{quotation}

\noindent But the methods of teaching both the theory and the practice of music were so
dark, difficult, and tedious, before its notation, measure, and harmonial laws were
settled, that we cannot wonder when we hear of youth having spent nine or ten
years in the study of scholastic music, and apparently to very little purpose.

In the latter part of the fourth century (\ad 374 to 397), Ambrose, bishop of
Milan, introduced a model of Church melody, in which he chose four series
or successions of notes, and called them simply the first, second, third, and fourth
tones, laying aside, as inapplicable, the Greek names of Doric, Phrygian, Lydian,
Æolic, Ionic, \&c. These successions distinguished themselves only by the position
of the semitones in the degrees of the scale, and are said to be as~follows:

\begin{center}
\begin{tabular}{llllllllllllll}
1st tone, & d & e & f & g & a & b & c & d \\
2nd tone, &\raisebox{1.5pt}{\rule{1em}{1pt}} & e & f & g & a & b & c & d & e \\
3rd tone, &\multicolumn{2}{c}{\raisebox{2pt}{\rule{3em}{1pt}}}& f & g & a & b & c & d & e & f \\
4th tone, &\multicolumn{3}{c}{\raisebox{2pt}{\rule{5em}{1pt}}} & g & a & b & c & d & e & f & g \\
\end{tabular}
\end{center}

These, Pope Gregory the Great (whose pontificate extended from 590 to 604)
increased to eight. He retained the four above-mentioned of Ambrose, adding to
them four others, which were produced by transposing those of Ambrose a fourth
lower; so that the principal note (or key-note, as it may be called) which formerly
appeared as the first in that scale, now appeared in the middle, or strictly
speaking, as the fourth note of the succession, the four additional scales being
called the \textit{plagal}, to distinguish them from the four more ancient, which received
the name of \textit{authentic}.

In this manner their order would of course be disarranged, and, instead of being
the first, second, third, and fourth tones, they became the first, third, fifth, and~seventh.

The following are the eight ecclesiastical tones (or scales) which still exist as such
in the music of the Romish church, and are called Gregorian, after their~founder:

\medskip
\small
\noindent \begin{tabular}{lcllrlrrclllll}
1st&tone&Authentic,&\multicolumn{2}{l}{\raisebox{2pt}{\rule{5em}{1pt}}}&D&e\tie f&g&A&b\tie c&D\\
2d&do.&Plagal,&A&b\tie c&D&e\tie f&g&A\\
3rd&do.&Authentic,&\multicolumn{3}{l}{\raisebox{2pt}{\rule{7em}{1pt}}}&E\tie f&g&a&B\tie c&d&E\\
4th&do.&Plagal,&\raisebox{1.5pt}{\rule{1em}{1pt}}&B\tie c&d&E\tie f&g&a&B&\\
5th&do.&Authentic,&\multicolumn{3}{l}{\raisebox{2pt}{\rule{7em}{1pt}}}&F&g&a&b\tie C&d&e\tie F\\
6th&do.&Plagal,&\raisebox{1.5pt}{\rule{1em}{1pt}}&C&d&e\tie F&g&a&b\tie C\\
7th&do.&Authentic,&\multicolumn{4}{l}{\raisebox{2pt}{\rule{10em}{1pt}}}&G&a&b\tie c&D&e\tie f&G\\
8th&do.&Plagal,&\multicolumn{2}{l}{\raisebox{2pt}{\rule{5em}{1pt}}}&D&e\tie f&G&a&b\tie c&D
\end{tabular}
\medskip
\normalsize



\noindent It will be perceived at the first glance,
that these Gregorian tones have only  
\pagebreak
%===============================================================================
%013
the intervals of the diatonic scale of C, such as are the white keys of the pianoforte,
without any sharps or flats. The only allowable accidental note in the Canto
fermo or plain song of the Romish church is B flat, the date of the introduction
of which has not been correctly ascertained.\myfootnote{ %a
It was probably derived from the tetrachords of the
Greek scale, which admitted both \textit{b} flat and \textit{b} natural, but
which it is not necessary to discuss here.
} %end footnote 
No sharp occurs in genuine chants
of high antiquity. In some modern books the flat is placed at the clef upon \textit{b}, for
the fifth and sixth modes, but the strict adherents to antiquity do not admit this
innovation. These tones only differ from one another in the position of the half
notes or semitones, as from \textit{b} to \textit{c}, and from \textit{e} to \textit{f}. In the four plagal modes, the
final or key note remains the same as in the relative authentic; thus, although in the
sixth mode we have the \textit{notes} of the scale of C, we have not in reality the key of
C, for the fundamental or key note is \textit{f}; and although the first and eighth tones
contain exactly the same notes and in the same position, the fundamental note of
the first is \textit{d}, and of the eighth \textit{g}. There is no other difference than that the
melodies in the four authentic or principal modes are generally (and should
strictly speaking be) confined within the compass of the eight notes above the key
note, while the four plagal go down to a fourth below the key note, and only
extend to a fifth above it.

No scale or key of the eight ecclesiastical modes is to us complete. The first
and second of these modes being regarded, according to the modern rules of
modulation, as in the key of D minor, want a flat upon \textit{b}; the third and fourth
modes having their termination in E, want a sharp upon \textit{f}; the fifth and sixth
modes being in F, want a flat upon \textit{b}; and the seventh and eighth, generally
beginning and ending in G major, want an \textit{f} sharp.

The names of Dorian, Phrygian, Lydian, Mixolydian, \&c., have been applied to
them with equal impropriety (more particularly since Glareanus, who flourished
in the sixteenth century); they bear no more resemblance to the Greek scales than
to the modern keys above cited.

Pope Gregory made an important improvement by discarding the thoroughly
groundless system of the tetrachord, adopted by the ancient Greeks,\myfootnote{ %b
In the old Greek notation there were 1620 tone characters, 
with which Musicians were compelled to burthen
their memories, and 990 marks actually different from
each other.
} %end footnote
and by
founding in its place that of the octave, the only one which nature indicates. And
another improvement no less important, in connexion with his system of the
octave, was the introduction of a most simple nomenclature of the seven sounds of
the scale, by means of the first seven letters of the alphabet. Burney says that the
Roman letters were first used as musical characters between the time of Boethius,\myfootnote{ %c
It appears from Burney, that Boethius used the first
fifteen letters of the alphabet, but only as marks of
reference in the divisions of the monochord, not as
musical notes or characters.
} %end footnote
who died in 526, and St. Gregory; but Kiesewetter\myfootnote{ %d
“History of the Modern Music of Western Europe,
from the first century of the Christian era, to the present
day,” \&c., by R. G. Kiesewetter, translated by Robert
Müller, 8vo., 1848. It is a very clearly and concisely
written history, and contains in an appendix within the
compass of a few pages, as much of the Greek music as
any modern can require to know.
} %end footnote 
attributes this improvement
in notation entirely to Gregory, in whose time the scale consisted only of two \pagebreak
octaves, the notes of the lower octave being expressed by capital letters, and the 
%===============================================================================
%014
higher by small letters. Eventually a third octave was added to the scale, four 
notes of which are attributed to Guido, and one to his pupils; the two remaining 
notes still later. The highest octave was then expressed by double letters; as, \textit{aa},
\textit{bb}, \&c. These three octaves in modern notes would constitute the following scale:

\bigskip
\lilypondfile[indent, staffsize=18]{lilypond/three-octaves}
\bigskip



This is the alphabetical system of names for the notes which we, in England,
still retain for every purpose but that of exercising the voice, for which solfaing
on vowels is preferred.

Gregory’s alphabetical system of notation was, however, only partially adoptaed.
Some wrote on lines varying from seven to fifteen in number, placing dots, like
modern crotchet-heads, upon them, but making no use of the spaces. Others used
spaces only, and instead of the dots wrote the words themselves in the spaces, disjointing
each syllable to place it in the position the note should occupy. A third
system was by points, accents, hooks, and strokes, written over the words, and they
were intended to represent to the singer, by their position, the height of the note,
and by their upward or downward tendency, the rising or falling of the voice. It
was, however, scarcely possible for the writer to put down a mark so correctly,
that the singer could tell exactly which note to take. It might be one or two
higher or lower. To remedy this, a red line was drawn over, and parallel to the
words of the text, and the marks were written above and below it. A further
improvement was the use of two lines, one red and the other yellow, the red for F,
the yellow for C, as it only left three notes (G, A, and B) to be inserted between~them.\myfootnote{ %a
Specimens of this notation, with red and yellow lines,
will be found in Martini’s Storia della Musica, vol. i.
p.~184; in Burney’s History, vol. ii. p.~37; in Hawkins’s
History, p.~947 (8vo. edition); and in Kiesewetter’s
p.~280. Also of other systems mentioned~above.
} %end footnote

Such was the notation before the time of Guido, a monk of Arezzo, in Tuscany,
who flourished about 1020. He extended the number of lines by drawing one
line under F, and another between F and C, and thus obtained four lines and
spaces, a number, which in the Rituals of the Romish Church has never been
exceeded.

The clefs were originally the letters F and C, used as substitutes for those red
and yellow lines. The Base clef still marks the position of F, and the Tenor
clef of C, although the forms have been changed.



Guido, in his Antiphonarium, gives the hymn
\textit{Ut queant laxis}\myfootnote{ %b
Hymn for St. John the Baptist’s day, written by
Paul the Deacon, about 774.

\settowidth{\versewidth}{UT queant laxis}
\begin{fnverse}\scriptsize
UT queant laxis\\
REsonare fibris,\\
MIra gestorum\\
FAmuli tuorum:\\
SOLve polluti,\\
LAbii reatum,\\
\vin\vin Sancte Johannes.
\end{fnverse}

SI was not the settled name for B until nearly the end
of the seventeenth century; and, although it was proposed
in 1547, Butler in his Principles of Musick, 1636, gives 
the names of the notes as Ut, re, mi, fa, sol, la, \textit{pha}. In
1673, Gio. Maria Bononcini, father of Handel's pseudorival,
used Do in place of Ut, but the French still retain Ut.
} %end footnote
\pagebreak
(from the 
%014
%===============================================================================
%015
initial lines of which the names of the notes, Ut, re, mi, fa, sol, la, were taken), in 
old ecclesiastical notation, and in the Chronicle of Tours, under the year 1033, he
is mentioned as the first who applied those names to the notes. He did not add
the Greek gamma (our G) at the bottom of the scale,\myfootnote{ %a
To distinguish G on the lowest line of the Base from
the G in the fifth space, the former was marked with the
Greek \Gamma, and hence the word gammut, applied to the
whole scale.
} %end footnote
as was long supposed,
for Odo, Abbot of Cluny, in Burgundy, had used it as the lowest note, in his
Enchiridion, a century before.

\renewcommand\rectoheader{scales, notation, clefs, and descant.}
To Franco, of Cologne (who, by the testimony of Sigebert, his cotemporary,
had acquired great reputation for his learning in 1047, and lived at least till 1083,
when he filled the office of Preceptor of the Cathedral of Liege), is to be ascribed
the invention of characters for \textit{time}.\myfootnote{ %b
John de Muris, who flourished in 1330, in giving a list
of anterior musicians, who had merited the title of
inventors, names Guido, who constructed the gammut, or
scale, for the monochord, and placed notes upon lines and
spaces; after whom came Magister Franco, who invented
the figures, or notes, of the Cantus mensurabilis (qui
invenit in cantu mensuram figurarum). Marchetto da
Padova, who wrote in 1274, calls Franco the inventor of the
four first musical characters; and Franchinus Gaffurius
twice quotes him as the author of the time-table.
} %end footnote
By this he conferred the most important
benefit on music, for, till then, \textit{written} melody was entirely subservient to syllabic
laws, and music in parts must have consisted of simple counterpoint, such, says
Burney, as is still practised in our parochial psalmody, consisting of note against
note, or sounds of equal length.

The first ecclesiastical harmony was called Descant, and by the Italians, Mental
Counterpoint (Contrappunto alia mente). It consisted of extemporaneous singing
in fourths, fifths, and octaves, above and below the plain song of the Church; and
although in its original sense, it implied only singing in two parts, it had made
considerable advances in the ninth century, towards the end of which we find
specimens, still existing, of harmony in three and four parts. When Descant was
reduced to writing, it was called Counterpoint, from \textit{punctum contra punctum},
point against point, or written notes placed one against the other.

Hubald, Hucbald, or Hugbald, as he is variously named, and who died in~930,
at nearly ninety years of age, has left us a treatise, called Musica Enchiriadis,
which has been printed by the Abbé Gerbert, in his Scriptores Ecclesiastici. In
chapters X. to XIV., De Symphoniis, he says: “There are three kinds of
symphony (harmony), in the fourth, fifth, and octave, and as the combination of
some letters and syllables is more pleasing to the ear than others, so is it with
sounds in music. All mixtures are not equally sweet.” In the fifteenth chapter
he uses a transient second and third, both major and minor; and in the eighteenth
he employs four thirds in succession. Burney says: “Hubald’s idea that one
voice might wander at pleasure through the scale, while the other remains fixed,
shows him to have been a man of genius and enlarged views, who, disregarding
rules, could penetrate heyond the miserable practice of his time, into our Points
d’Orgue, Pedale, and multifarious harmony upon a holding note, or single base,
and suggests the principal, at least, of the boldest modern harmony.” It is in
this last sense of amplifying a point, that we still retain the verb to \textit{descant} in
common use. Guido describes the Descant \pagebreak
existing in his time, as consisting of 
%===============================================================================
%016
fourths, fifths, and octaves under the plain-song or chant, and of octaves (either 
to the plain song or to this base) above it. He suggests what he terms a smoother
and more pleasing method of under-singing a plain-song, in admitting, besides the
fourth and the tone, the major and minor thirds; rejecting the semitone and the
fifth. “No advances or attempts at variety seem to have been made in counterpoint,
from the time of Hubald, to that of Guido, a period of more than a hundred
years; for with all its faults and crudities, the counterpoint of Hubald is at least
equal to the best combinations of Guido;” but the monk, Engelbert, who wrote in
the latter end of the thirteenth century, tells us that all “regular descant” consists
of the union of fourths, fifths, and octaves, so that these uncouth and barbarous
harmonies, in that regular succession which has been since prohibited,
continued in the Church for four centuries.



Before the use of lines, there were no characters or signs for more than two
kinds of notes in the Church; nor since ecclesiastical chants have been written
upon four lines and four spaces, have any but the square and lozenge characters,
commonly called Gregorian notes, been used in Canto fermo: and, although the
invention of the time-table extended the limits of ingenuity and contrivance to
the utmost verge of imagination, and became all-important to secular music,
the Church made no use whatever of this discovery.

That melody received no great improvement from the monks, need excite
no wonder, as change and addition were alike forbidden; but not to have
improved harmony more than they did for many centuries after its use was
allowed, is a matter of just surprise, especially since the cultivation of music
was a necessary part of their profession.

We have occasional glimpses of secular music through their writings; for
instance, Guido, who gives a fair definition of harmony in the sense it is now
understood (Armonia est diversarum vocum apta coadunatio), says that he
merely writes for the Church, where the pure Diatonic genus was first used, but
he was aware of the deficiency as regards other music. “Sunt prœterea et alia
musicorum genera aliis mensuris aptata.” Franco (about 1050) just mentions
Discantum in Cantilenis Rondellis—“Descant to Rounds or Roundelays,”—but
no more.

When Franco writes in four parts, he sometimes gives five lines to each part,
the five lowest for the Tenor or plain song, the next five for the Medius, five for
the Triplum Discantus, and the highest for the Quadruplum. Each has a clef
allotted to it. Although many changes in the form of musical notes have been
made since his time, the lines and spaces have remained without augmentation or
diminution, four for the plain song of the Romish Church, and five for secular
music.

He devotes one chapter to characters for measuring silence, and therein gives
examples of rests for Longs, Breves, Semibreves, and final pauses. He also
suggests dots, or points of augmentation. Bars are placed in the musical examples,
as pauses for the singers to take breath at the end of a sentence, verse, or phrase
of melody. And this is the only use made of bars in Canto~fermo.
\pagebreak