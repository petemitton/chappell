%%===============================================================================
%vi
\renewcommand\versoheader{introduction.}
\renewcommand\rectoheader{introduction.}
\changefontsize{1.03\defaultfontsize}

It might be supposed that the registers of the Company of Stationers would 
furnish a complete list of ballads and ballad-printers, but, having seen all the
entries from 1577 to 1799, I should say that not more than one out of every
hundred ballads was registered. The names of some of the printers are not to
be found in the registers.

It appears from an entry referring to the “white book” of the Company
(which is not now existing), that seven hundred and ninety-six ballads were left
in the council-chamber of the Company at the end of the year 1560, to be handed
over to the new Wardens, and at the same time but forty-four books.

Webbe, in a \textit{Discourse of English Poetrie}, printed in 1586, speaks of “the
\textit{un-countable} rabble of ryming ballet-makers and compylers of senseless sonnets,”
and adds, “there is not anie tune or stroke which may be sung or plaide on
instruments, which hath not some poetical ditties framed according to the numbers
thereof: some to \textit{Rogero}, some to \textit{Trenchmore}, to \textit{Downright Squire}, to galliardes,
to pavines, to jygges, to brawles, to all manner of tunes; which every fidler knows
better than myself, and therefore I will let them passe.” Here the class of music
is named with which old English ditties were usually coupled—dance and ballad
tunes. The great musicians of Elizabeth’s reign did not often compose airs of
the short and rhythmical character required for ballads. These were chiefly the
productions of older musicians, or of those of lower grade, and some of ordinary
fiddlers and pipers. The \textit{Frog Galliard} is the only instance I know of a popular
ballad-tune to be traced to a celebrated composer of the latter half of the sixteenth
century. The scholastic music then in vogue was of a wholly different character.
Point and counterpoint, fugue and the ingenious working of parts, were the great
objects of study, and rhythmical melody was but lightly esteemed.

In the reigns of James I. and Charles I., we find a few “new court tunes”
employed for ballads, but it was not until Charles II. ascended the throne that
composers of \textit{high} repute commenced, or re-commenced, the writing of simple
airs, and then but sparingly. Matthew Locke’s “\textit{The delights of the bottle}” is
perhaps the first song composed for the stage, that supplied a tune to ballads.

My former publication contained two hundred and forty-five airs; the present
number exceeds four hundred. Of these, two hundred are contained in the first
volume, which extends no further than the reign of Charles I. This portion of
the work may be considered as a \textit{collection}; but the number of airs extant of later
date is so much larger than of the earlier period, that the second volume can be
viewed only in the light of a \textit{selection}. To have made it upon the same scale as the
first would have occupied at least three volumes instead of one. My endeavour
has therefore been, to give as much variety of character as possible, but especially
to include those airs which were popular as ballad-tunes. Some of those contained
in the old collection have now given place to others of more general interest but
more than two hundred are retained. Every air has been re-harmonized, upon a
simple and consistent plan,—the introductions to the various reigns have been
added,—and nearly every line in the book has been rewritten.


I have been at some trouble to trace \pagebreak to its origin the assertion that the English 
%%===============================================================================
%vii
have no national music. It is extraordinary that such a report 
should have 
obtained credence, for England may safely challenge any nation not only to produce
as much, but also to give the same satisfactory proofs of antiquity. The
report seems to have gained ground from the unsatisfactory selection of English
airs in Dr.~Crotch’s \textit{Specimens of various Styles of Music}; but the national music
in that work was supplied by Malchair, a Spanish violin-player at Oxford, whose
authority Crotch therein quotes. It is perhaps not generally known that at the
time of the publication Dr.~Crotch was but nineteen years of age. No collection
of English airs had at that time been made to guide Malchair, and he followed
the dictum of Dr.~Burney in such passages as the following:—

“It is related by Giovanni Battista Donado that the Turks have a limited
number of tunes, to which the poets of their country have continued to write for
ages; and the vocal music of our own country seems long to have been equally
circumscribed: for, till the last century, it seems as if the number of our secular
and popular melodies did not greatly exceed that of the Turks.” In a note, he
adds, that the tunes of the Turks were in all twenty-four, which were to depict
melancholy, joy, or fury,—to be mellifluous or amorous. (\textit{History},~ii.~553.)

Again, in Shakespeare’s \textit{Midsummer Night’s Dream}, when Bottom has been
turned into an ass, and says “I have a reasonable good ear in music; let me have
tongs and bones,” the stage direction is “Musick tongs, Rural Music.” Burney
inverts the stage direction, and adds “Poker and tongs, marrowbones and cleavers,
salt-box, hurdy-gurdy, \&c., are the old national instruments of our island.”
(iii.~335.)

Jean Jacques Rousseau published a letter on French music, which he summed
up by telling his countrymen that “their harmony was abominable; their airs
were not airs; their recitative was not recitative; that they had no music, and
could not have any.” (Rousseau, \textit{Ecrits sur la Musique}, Paris, edit.~1823,
p.~312.) Dr.~Burney seems to have improved upon this model, for Rousseau did
not resort to misquotation to prove his case, but Dr.~Burney’s History is one
continuous misrepresentation of English music and musicians, only rendered
plausible by misquotation of every kind.

The effect of the misquotation is that he has been believed; and passages as
absurd as the following have been copied by writers who have relied upon his~authority:—

\begin{quotation}“The low state of our regal music in the time of Henry VIII., 1530, may be
gathered from the accounts given in Hall’s and Hollinshed’s Chronicles, of a masque
at Cardinal Wolsey’s palace, Whitehall, where the King was entertained with
‘a concert of drums and fifes.’ But this was soft music compared with that of
his heroic daughter Elizabeth, who, according to Hentzner, used to be regaled
during dinner ‘with twelve trumpets and two kettle-drums; which, together with
fifes, cornets, and side-drums, made the hall ring for half an hour together.’”
(\textit{History}, iii. 143.)
\end{quotation}

 
There is nothing of the kind in the 
books Dr.~Burney pretends to quote. The 
account of the chroniclers is of \pagebreak
Henry the Eighth’s landing at Wolsey’s palace, 
%%===============================================================================
%%viii 
where by a preconcerted arrangement, “divers chambers” (short cannon that 
made a loud report) were let off, and he was conducted into the hall with “such
a noise of drums and flutes as seldom had been heard the like,” for the purpose
of \textit{surprising} the Cardinal and the masquers. Not a word of the music of the
masque.



As to Queen Elizabeth, Hentzner describes only the military music to give notice
in the palace that dinner was being carried in. Music then answered the purpose
of the dinner-bell. He says “the queen dines and sups alone.”

Burney carries his depreciation of English authors systematically throughout
his work. It might be supposed that he would have allowed an author of so early
a date as John Cotton, who flourished soon after Guido, to pass unchallenged, but
he first misrepresents, and then contradicts him. Burney tells us that Cotton
ascribes the invention of neumæ erroneously to Guido (ii.~144). Now Cotton
speaks of various modes of writing music by the musical signs called neumæ, and
attributes the last only to Guido. It is certain that Burney read no more of the
treatise than the heading of a chapter (\textit{Quid utilitatis afferant neumæ a Guidone
inventæ}), for he proves by a note upon neumæ, that he only half understood what
they were. To any reader of Cotton’s treatise, such misapprehension would have
been impossible. (See Gerbert’s \textit{Scriptores Ecclesiastici de Musicâ}, ii. 257.)

It is not always easy to prove that a writer reviewed works without reading
them, but I doubt if any musician can now be found who believes that Burney
had examined “all the works he could find” of Henry Lawes, with the “care
and candour” that he professes; while in the case of Morley’s \textit{Concert-Lessons},
it is certain that he passed his facetious judgment upon them after scoring only
a portion of two parts, the work being in six. This is proved by his own manuscript
(Addit. MSS. 11,587, Brit. Mus.), and there was no perfect copy of the
work extant at the~time.

When Burney tells us that the Catch Club sang old compositions “better than
the authors \textit{intended}” (iii. 123),—that “our secular vocal music, during the first,
years of Elizabeth’s reign, seems to have been much inferior to that of the Church,”
and has no better proof of it than a book of songs \textit{composed by an amateur musician},
“Thomas Wythorne, Gent.,” in 1571 (iii.~119),—when he says that, in
the same reign, “the violin was hardly known to the English in shape or in
name!” (iii. 143),—and that Playford was the \textit{first} who published music in the
seventeenth century, yet commenced in 1653! (iii. 417 and~418),—he shews not
only a desire to underrate, but also a deficiency of knowledge, that must weaken
all confidence in him as an~historian.

 
In his review of the music in Elizabeth’s reign, he tells us that “the art of
singing, further than was necessary to keep a performer in tune and time, must
have been unknown\ldots\  \textit{solo songs, anthem}s, and cantatas, being productions of
later times” (iii. 114). A more strange misconception could scarcely have been
penned. No songs to the lute? No ballads? If so, Miles Coverdale might have
spared himself the trouble of telling the courtier “not to rejoice in his ballads,” 
and Chaucer should have represented \pagebreak
at least three persons as serenading the
%%===============================================================================%
%%ix
carpenter's wife, and not  one. As to the art of singing, Dr.~Burney has himself 
quoted the description of John of Salisbury, written four hundred years before
Queen Elizabeth’s reign, and that is quite enough to refute the opinion above
expressed; but, if more be required, the reader will find it here in the long note
at p.~404.

There was a proverb, of French origin, current both in Latin and English in
the fourteenth and fifteenth centuries, respecting the manner of singing by different
nations. The Latin version was “Galli cantant, Angli jubilant, Hispani
plangunt, Germani ulutant, Itali caprizant:” the English was “The French
sing,” or “The French pipe, the English carol [rejoice, or sing merrily], the
Spaniards wail, the Germans howl, the Italians caper.” (The allusion to the
Italians is rather as to their unsteady holding of notes than to their facility in
florid singing; \textit{caper} signifying “a goat.”) Burney, without any authority,
renders it “the English \textit{shout}” (iii. 182). Now, although we have no modern
English verb that is an exact translation of “jubilare,” the Italian “giubilare”
has precisely the same signification; and Pasqualigo, the Venetian ambassador
to Henry VIII., describing the singing of the English choristers in the King’s
chapel, says “their voices are really rather divine than human—non cantavano
ma jubilavano,” which can be understood only in a highly complimentary sense.

It is sufficient for my present purpose to say that Dr.~Burney’s \textit{History} is
written throughout in this strain. What with mistake, and what with misrepresentation, it can but mislead the reader as to English music or musicians; and
from the slight search I have made into his early Italian authorities, I doubt
whether even that portion is very reliable. The public has now forgotten
the contention between the rival histories of music of Hawkins and Burney, and
a few words should be placed upon record. Hawkins’s entire work was published
in 1776, and Burney’s first volume in the same year, his second in 1782, and his
third and fourth in 1789. Burney obtained a great reputation by his first volume,
which is upon the music of the ancients. In that he was assisted by the researches
of the Rev. Thomas Twining, the translator of Aristotle’s Poetics, who relinquished
his own projected, and partly-written history, in Burney’s favour.
Hawkins’s work is of great original research, and he is a far more reliable
authority for fact than Burney: still the history is by no means so well digested.
It is an analysis of book after book and life after life, fitted rather for supplying
materials to those who will dig them out, than to be read as a whole. Burney’s
is a very agreeably written book, but he made history pleasant by such lively
sallies as those I have quoted: he took his authorities at second hand, when the
originals were accessible; and copied especially from Hawkins, without acknowledgment,
and disguised the plagiarism by altering the language. Many of his appropriations
are to be traced by errors which it is impossible that two men reading
independently could commit. Burney had but one love,—the Italian school,—and
he thought the most minute particulars of the Italian opera of his day worthy
of being chronicled. The madrigal with him was a “many-headed monster” (iii. 385): 
French music was \pagebreak “displeasing to all ears but those of France,” and 
%%===============================================================================
%%x
Rousseau’s letter upon it “an excellent piece of musical criticism,” combining 
“good sense, taste, and reason” (iv. 615): he dismisses Sebastian Bach in half
a dozen lines; and, although he devotes much space to Handel’s operas, his
oratorios are often dismissed with the barest record of their existence by a line in
a note. \textit{Israel in Egypt}, \textit{Acis and Galatea}, \&c., are unnoticed.


The present collection will sufficiently prove that “the number of our secular
and popular melodies” was not quite as “circumscribed” as Dr.~Burney has
represented; but, indeed, he had a book in his library which alone gave a complete
refutation to his limited estimate. I have now before me one of the editions
of \textit{The Dancing Master}, a collection of Country Dances, published by Playford,
which was formerly in Burney’s possession. It contains more than two hundred
tunes, the names of which must convince an ordinary reader that at least a considerable
number among them are song and ballad tunes, while a musician will as
readily perceive many others to be of the same class, from the construction of
the melody. If a doubt should remain as to the character of the airs in collections
of this kind, further evidence is by no means wanting. Sir Thomas Elyot, writing
in 1531, and describing many ancient modes of dancing, says (in The \textit{Governour}),
“As for the special names [of the dances], they were taken as \textit{they be now}, either
of the names of the first inventour, or of the measure and number they do conteine,
or of the \textit{first words of the ditties} which the song comprehendeth, \textit{whereoff}
the daunce was made;” and, to approach nearer to the time of the publication in
question, Charles Butler, in 1636, speaks of “the infinite multitude of ballads
set to sundry pleasant and delightful tunes by cunning and witty composers, with
country dances \textit{fitted unto them}” See his \textit{Principles of~Musick}.

The eighteen editions of \textit{The Dancing Master} are of great assistance in the
chronological arrangement of our popular tunes from 1650 to 1728;\myfootnote{ %a
The first edition of this collection is entitled “The
English Dancing Master: or Plaine and easie rules for
the dancing of Country Dances, with the tune to each
dance (104 pages of music). Printed by Thomas Harper,
and are to be sold by John Playford, at his shop in the
Inner Temple, neere the Church doore.” The date is 1651,
but it was entered at Stationers’ Hall on 7th Nov., 1650.
This edition is on larger paper than any of the subsequent.
The next is \textit{The Dancing Master},\ldots\  with the tune to
each dance, to he play’d on the treble Violin: the \textit{second
edition}, enlarged and corrected from many grosse errors
which were in the former edition.” This was “Printed
for John Playford,” in 1652 (112 pages of music). The
two next editions, those of 1657 and 1665, each contain
132 country dances, and are counted by Playford as one
edition. To both were added “the tunes of the most
usual French dances, and also other new and pleasant
English tunes for the treble Violin.” That of 1665 was
“Printed by W. G., and sold by J. Playford and Z. Watkins,
at their shop in the Temple.” It has 88 tunes for
the violin at the end. (The tunes for the violin
were afterwards printed separately as Apollo's Banquet,
and are not included in any other edition of The
Dancing Master.) The date of the fourth edition is
1670 (155 pages of music). Fifth edition, 1675, and 160
pages of music. (The contents of the sixth edition are
ascertained to be almost identical with the fifth, by the
new tunes added to the seventh being marked with *, but
I have not seen a copy. From advertisements in Playford’s
other publications, it appears to have been printed
in 1680.) The seventh edition bears date 1686 (208 pages),
but to this “an additional sheet,” containing 32 tunes,
was first added, then “a new additional sheet” of 12
pages,” and lastly “a new addition” of 6 more. The
eighth edition was “Printed by E. Jones for H. Playford,”
and great changes made in the airs. It has 220 pages,—date,
1690. The ninth edition, 196 pages,—date, 1695.
“The second part of the Dancing Master,“24 pages,—date,
1696. The tenth edition, 215 pages,—date, 1698;
also the second edition of the second part, ending on p.~48
(irregularly paged), 1698. The eleventh is the first edition
in the new tied note, 312 pages,—date, 1701. The twelfth
edition goes back to the old note, 354 pages,—date, 1703.
The later editions are well known, but the above are
scarce.
} %end footnote
for, although
some airs run through every edition, we may tell by the omission of others, when
they fell into desuetude, as well as the airs by which their places were supplied. 

\pagebreak