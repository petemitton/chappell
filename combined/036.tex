%===============================================================================
%36
\changefontsize{1.04\defaultfontsize}

\settowidth{\versewidth}{That bin at feastes with the brede [bread]:}
\begin{scverse}
And in many an othir pipe,\\
That craftely began to pipe\\
Bothe in \textit{Douced} and eke in \textit{Rede},\myfootnote{ %a
Tyrwhitt thinks \textit{Doucete} an Instrument, and quotes
Lydgate—
\settowidth{\versewidth}{“Ther were trumpes and trumpetes,}
\begin{fnverse}
\vleftofline{“}Ther were trumpes and trumpetes,\\
Lowde shall[m]ys and doucetes.”
\end{fnverse}
but it seems to me only to mean soft pipes in opposition
to loud shalms. By the distinction Chaucer draws, “both
in douced and in reed” (the shalm being played on by
a reed), I infer by “douced” that flutes are intended; the
tone of which, especially the large flute, is extremely soft.
I had a collection of English flutes, of which one was
nearly a yard and a half long. All had mouth-pieces like
the flageolet, and were blown in the same manner; the
tone very pleasing, but less powerful and brilliant than
the modern or “German” flute.
} \\
That bin at feastes with the brede [bread]:\\
And many a \textit{Floite} and litlyng \textit{Horne}\\
And \textit{Pipes made of grenè corne}.\\
As have these little Herdègroomes\\
That kepin Beastes [keep oxen] in the broomes.”
\end{scverse}

As to the songs of his time, see the Frankeleyne’s Tale (line 11,254 to 60):—
\settowidth{\versewidth}{Sauf [save] in his songès somewhat wolde he wreye [betray]}
\begin{scverse}
\vleftofline{“}He was dispeired, nothing dorst he seye\\
Sauf [save] in his songès somewhat wolde he wreye [betray]\\
His woo, as in a general compleyning;\\
He said he loved, and was beloved nothing.\\
Of suche matier made he many \textit{Layes},\\
\textit{Songes, Compleyntes, Roundelets, Virelayes}:\\
How that he dorstè not his sorwe [sorrow] telle,\\
But languisheth as doth a fuyr in helle.”
\end{scverse}
and he speaks elsewhere of \textit{Ditees, Rondils, Balades}, \&c.

The following passages relate to minstrelsy, and to the manner of playing the
harp, pointing and performing with the nails, as the Spaniards do now with the
guitar. The first is from the House of Fame (Urry, line 105 to 112):—
\settowidth{\versewidth}{\ldots\  “Stoden\ldots\  the castell all aboutin}
\begin{scverse}
\ldots\  “Stoden\ldots\  the castell all aboutin\\
Of all manir of Minstralis\\
And gestours that tellen tales\\
Both of wepyng and of game,\\
And all that ’longeth unto fame;\\
There herde I playin on an \textit{Harpe}\\
\textit{That ysounid bothe well and sharpe}”
\end{scverse}
and from Troylus, lib. 2, 1030:—
\settowidth{\versewidth}{Touch aie o (one) string, or aie o warble harpe,}
\begin{scverse}
“For though that the best barper upon live\\
Would on the bestè sounid jolly harpe\\
That evir was, with all his fingers five\\
Touch aie o (one) string, or aie o warble harpe,\\
\textit{Were his nailes poincted nevir so sharpe}\\
It shoulde makin every wight to[o] dull\\
To heare [h]is Glee, and of his strokes ful.”
\end{scverse}

Even the musical gamut is mentioned by Chaucer. In the supplementary tale
he makes the host give “an hid[e]ouse cry in ge-sol-re-ut the haut,” and there is
scarcely a subject connected with the art as practised in his day, that may not be
illustrated by quotation from his works;
\begin{scverse}
“For, gif he have nought sayd hem, leeve [dear] brother,\\
In o bo[o]k, he hath seyd hem in another.” 
\end{scverse}


\pagebreak