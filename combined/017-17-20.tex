%===============================================================================



%017
\renewcommand\rectoheader{anglo-saxon music.}
\changefontsize{1.02\defaultfontsize}



Turning to England, Milton tells us, from the Saxon annals, that in 668, 
Pope Vitalian sent singers into Kent, and in 680, according to the Venerable
Bede,\myfootnote{ %a
As a proof of the veneration in which Bede was held,
and the absurd legends relating to him, I quote from
a song of the fifteenth century:—

\settowidth{\versewidth}{\vin\textit{Songs and Carols. Percy Soc}. No. 73, p.~31.}

\begin{fnverse}\scriptsize
\vleftofline{“}When Bede had prechd to the stonys dry\\
The my[gh]t of God made [t]hem to cry\\
Amen:--certys this no ly[e]!”\\
\vin\textit{Songs and Carols. Percy Soc}. No. 73, p.~31.
\end{fnverse}
} %end footnote
Pope Agatho sent John, the Præcentor of St. Peter’s at Rome, to
instruct the monks of Weremouth in the manner of performing the ritual, and he
opened schools for teaching music in other parts of the kingdom of Northumberland.
Bede was also an able musician, and is the reputed author of a short
musical tract in two parts, \textit{de Musica theorica}, and \textit{de Musica practica, seu mensurata};
but Burney says, although the first may have been written by him, the
second is manifestly the work of a much more modern author, and he considers it
to have been produced about the twelfth century, \ie, between the time of Guido
and the English John de Muris. There must always be a difficulty in identifying
the works of an author who lived at so remote a period, without the aid of
contemporary authority, or of allusions to them of an approximate date; and when
he has written largely, such difficulties must be proportionably increased. But,
rejecting both the treatises on music, if he be the author of the Commentary on
the Psalms, which is included in the collected editions of his works of 1563
and 1688, sufficient evidence will remain to prove, not only his knowledge of
music, but of all that constituted the “regular” descant of the church from the
ninth to the thirteenth century. I select one passage from his Commentary on
the 52nd Psalm. “As a skilful harper in drawing up the cords of his instrument,
tunes them to such pitches, that the higher may agree in harmony with the lower,
some differing by a semitone, a tone, or two tones, others yielding the consonance
of the fourth, fifth, or octave; so the omnipotent God, holding all men predestined
to the harmony of heavenly life in His hand like a well-strung harp, raises some
to the high pitch of a contemplative life, and lowers others to the gravity of active
life.” And he thus continues:—“Giving the consonance of the octave, which
consists of eight strings;”\ldots\  “the consonance of the fifth, consisting of five
strings; of the fourth, consisting of four strings, and then of the smaller vocal
intervals, consisting of two tones, one tone, or a semitone, and of there being
semitones in the high as well as the low strings.”\myfootnote{ %b
“Sient peritus citharæda chordas plures tendens in
cithara, temperat eas acumine et gravitate tali, ut
superiores inferioribus conveniant in melodia, quædam
semitonii, quædam unius toni, quædam duorum tonorum
differentiam gerentes, aliæ vero diatesseron, aliæ autem
diapente, vel etiam diapason consonantiam reddentes: ita
et Deus omnipotens omnes homines ad cœlestis vitæ
harmoniam prædestinatos in manu sua, quasi eitharam
quandam, chordis convenientibus ordinatam, habens,
quosdam quidem ad acutum contemplativæ vitæ sonum
intendit, alios verò ad activæ vitæ gravitatem temperando
remittit.”—“ut ad alios comparati quasi diapason consonantiam,
quæ oeto chordis constat, reddant..Quos
autem ad diapente consonantiam, quinque chordis constantem,
eligit, illi possunt intelligere qui tantæ jam perfectionis
sunt\ldots\  Diatesseron quatuor chordis constans.
\ldots\  Per minora vero vocum intervalla quæ duos tonos
aut unum, vel semitonium sonant\ldots\  Sed quia tam in
altisonis quam in grandisonis chordis habetur semitonium.”
\&c.—\textit{Bedæ Presbyteri Opera}, vol. 8, p.~1070, fol.
\textit{Basilæ}, 1563, or \textit{Coloniæ Agrippinæ}, vol. 8, p.~908,
fol. 1688.
} %end footnote
Our great king, Alfred,
according to Sir John Spelman, “provided himself of musitians, not common, or
such as knew but the practick part, but men skilful in the art itself;” and in 866, 
according to the annals of the Church of \pagebreak
Winchester, and the testimony of many 
%===============================================================================
%018
ancient writers, he founded a Professorship at Oxford,\myfootnote{ %a
The earliest express mention of the University of
Oxford, after the foundation of the schools there by
Alfred, is from the historian Ingulphus, whose youth
coincided with the early part of the reign of Edward the
Confessor. He tells us that, having been born in the City
of London, he was first sent to school at Westminster,
and that from Westminster he proceeded to Oxford,
where he studied the Aristotelian Philosophy, and the
rhetoritical writings of Cicero.
} %end footnote
for the cultivation of music 
as a science. The first who filled the chair was Friar John, of St. David’s, who
read not only lectures on Music, but also on Logic and Arithmetic. Academical
honors in the faculty of music have only been traced back to the year 1463, when
Henry Habington was admitted to the degree of Bachelor of Music, at Cambridge,
and Thomas Saintwix, Doctor of Music, was made Master of King’s College, in
the same university; but it is remarkable that music was the only one of the
seven sciences that conferred degrees upon its students, and England the only
country in which those degrees were, and are still conferred.

\renewcommand\versoheader{music in england, time of henry ii.}
\renewcommand\rectoheader{giraldus cambrensis’ account.}
%%%\changefontsize{1.0\defaultfontsize}

About 1159, when Thomas à Becket conducted the negociations for the
marriage of Henry the Second’s eldest son with the daughter of Louis VII., and
went to Paris, as chancellor of the English Monarch, he entered the French towns,
his retinue being displayed with the most solicitous ostentation, “preceded by two
hundred and fifty boys on foot, in groups of six, ten, or more together, singing
English songs, according to the custom of their country.”\myfootnote{ %b
“In ingressu Gallicanarum villarum et castrorum,
primi veuiebant garciones pedites quasi ducenti quinquaginta, 
gregatim euntes sex vel deni, vel plures simul,
aliquid lingua sua pro more patriæ suæ cantantes.”—
\textit{Stephanides, Vita S. Thomæ Cantuar}, pp.~20,~21.
} %end footnote 
This singing in groups
resembled the “turba canentium,” of which Giraldus afterwards speaks; and the
following passage from John of Salisbury, about 1170, shows at least the delight
the people had in listening to part-singing, or descant. “The rites of religion
are now profaned by music; and it seems as if no other use were made of it than
to corrupt the mind by wanton modulations, effeminate inflexions, and frittered
notes and periods, even in the \textit{Penetralia}, or sanctuary, itself. The senseless
crowd, delighted with all these vagaries, imagine they hear a concert of sirens,
in which the performers strive to imitate the notes of nightingales and parrots,
not those of men, sometimes descending to the bottom of the scale, sometimes
mounting to the summit; now softening, and now enforcing the tones, repeating
passages, mixing in such a manner the grave sounds with the more grave, and
the acute with the most acute, that the astonished and bewildered ear is unable
to distinguish one voice from another.”\myfootnote{ %
Musica cultum religionis incestat, quod ante conspectum
Domini, in ipsis penetralibus sanctuarii, lascivientis
vocis luxu, quadam ostentatione sui, muliebribus
modis notularum articulorumque cæsuris, stupentes
animulas emollire nituntur. Cum præcinentium, et succinentium, 
canentium, et decinentium, intercinentium,
et occinentium, præmolles modulationes audieris, Sirenarum
concentus credas esse, non hominum et de vocum
facilitate miraberis, quibus philomela vel psittacus, aut
si quid sonorius est, modos suos nequeunt coæquare. Ea
siquidem est, ascendeudi descendendique facilitas; ea
sectio vel geminatio notularum, ea replicatio articulorum,
singulorumque consolidatio; sic acuta vel acutissima,
gravibus et subgravibus temperantur, ut auribus sui
indicii fere subtrahetur auturitas.—\textit{Policraticus, sive de
Nugis Curialium}, lib. i., c.~6.
} %end footnote
It was probably this abuse of descant
that excited John’s opposition to music, and his censures on the minstrels, as
shown in the passage before quoted. It proves also, that descant in England did
not then consist merely of singing in two parts, but included the licenses and
ornaments of florid song. Even singing in canon seems to be comprised in the
words, “præcinentium et succinentium, canentium et decinentium.”

%%%\changefontsize{1.0\defaultfontsize}
About 1185, Gerald Barry, or Giraldus Cambrensis, 
archdeacon, and afterwards \pagebreak
%\end{fixedpage}%018
%===============================================================================
%019
bishop, of St. David’s, gave the following description of the peculiar manner 
of singing of the Welsh, and the inhabitants of the North of England: “The
Britons do not sing their tunes in unison, like the inhabitants of other countries,
but in different parts. So that when a company of singers meets to sing, as is
usual in this country, as many different parts are heard as there are singers, who
all finally unite in consonance and organic melody, under the softness of B flat.\myfootnote{ %a
“Uniting under the softness of B flat,” is not very
intelligible, but one thing may be inferred from it, that
they sang in the natural scale, such as the fifth mode
became by the use of B flat in the scale of F, and not in
the modes that were peculiar to the church. B flat was
only used in the fifth mode and its plagal.
} %end footnote
In the Northern parts of Britain, beyond the Humber, and on the borders of
Yorkshire, the inhabitants make use of a similar kind of symphonious harmony
in singing, but with only two differences or varieties of tone and voice, the one
murmuring the under part, the other singing the upper in a manner equally soft
and pleasing. This they do, not so much by art, as by a habit peculiar to themselves,
which long practice has rendered almost natural, and this method of singing
has taken such deep root among this people, that hardly any melody is accustomed
to be uttered simply, or otherwise than in many parts by the former, and in two
parts by the latter. And what is more astonishing, their children, as soon as they
begin to sing, adopt the same manner. But as not all the English, but only those
of the North sing in this manner, I believe they had this art at first, like their
language, from the Danes and Norwegians, who were more frequently accustomed
to occupy, as well as longer to retain, possession of those parts of the island.”\myfootnote{ %b
In musico modulamine non uniformiter ut alibi,
sed multipliciter multisque modis et modulis cantilenas
emittunt, adeò ut in turba canentium, sicut huic genti
mos est, quot videas capita tot audias carmina discriminàque
vocum varia, in unam denique sub B mollis
dulcedine blanda consonantiam etorganicam convenientia
melodiam. In borealibus quoque majoris Britanniæ partibus
trans Humbrum, Eboracique finibus Anglorum
populi qui partes illas inhabitant simili canendo symphonica
utuntur harmonia: binis tamen solummodo
tonorum differentiis et vocum modulando varietatibus,
una inferius sub murmurante altera verò supernè demulcente
pariter et delectante. Nec arte tantum sed usu
longævo et quasi in naturam mora diutina jam converso,
hæc vel illa sibi gens hanc specialitatem comparavit.
Qui adeò apud utramque invaluit et altas jam radices
posuit, ut nihil hic simpliciter, ubi multipliciter ut apud
priores, vel saltem dupliciter ut apud sequentes, mellitè
proferri consuaverit. Pueris etiam (quòd magis admirandum) 
et ferè infantibus, (cum prinum à fletibus in
cantus erumpunt) eandem modulationem observantibus,
Angli verò quoniam non generaliter omnes sed boreales
solùm hujusmodi vocum utuntur modulationibus, credo
quòd a Dacis ct Norwagiensibus qui partes illas insulæ
frequentiùs occupare ac diutiùs obtinere solebant, sicut
loqueudi affinitatem, sic canendi proprietatem contraxerunt.—\textit{Cambriæ
Descriptio}, cap. xiii.
} %end footnote
Now, allowing a little for the hyperbolic style so common with writers of that age,
this may fairly be taken as evidence that part-singing was common in Wales, or
that at least they made descant to their tunes, in the same way that singers did
to the plain-song or Canto fermo of the Church at the same period; also that
singing in two parts was common in the North of England, and that children tried
to imitate it. Burney and Hawkins think that what Giraldus says of the singing
of the people of Northumberland, in two parts, is reconcileable to probability,
because of the schools established there in the time of Bede, but Burney doubts
his account of the Welsh singing in many parts, and makes this “turba
canentium” to be \textit{of the common people}, adding, “we can have no exalted idea
of the harmony of an \textit{untaught} crowd.” These, however, are his own inferences;
Giraldus does not say that the singers were untaught, or that they were of
the common people. As he is describing \pagebreak
what was the custom in his own time, 
%===============================================================================
%020
not what had taken place a century before, there seems no sufficient ground 
for disbelieving his statement,\myfootnote{ %a
Dr.~Percy says, “The credit of Giraldus, which hath
been attacked by some partial and bigoted antiquaries,
the reader will find defended in that learned and curious
work, ‘Antiquities of Ireland,’ by Edward Ledwich,
LL.D. Dublin, 1790, 4to., p.~207. et seq.”
} %end footnote
and least of all, should they who are of the opinion
that all musical knowledge was derived from the monasteries, call it in question,
since, as already shown, part music had then existed in the Church, in the form
of descant, for three centuries.

\renewcommand\versoheader{harpers not taught by monks.}
\renewcommand\rectoheader{character of tunes often derived from instruments.}
%%%\changefontsize{1.06\defaultfontsize}

“If, however,” says Burney, “incredulity could be vanquished with respect to
the account which Giraldus Cambrensis gives of the state of music in Wales
during the twelfth century, it would be a Welsh MS. in the possession of Richard
Morris, Esq., of the Tower, which contains pieces for the harp, that are in \textit{full
harmony} or \textit{counterpoint}; they are written in a peculiar notation, and supposed
to be as old as the year 1100; at least, such is the known antiquity of many of
the songs mentioned in the collection,” \&c. It is not necessary here to enter
into the defence of Welsh music, but the specimens Dr.~Burney has printed from
that manuscript, which he describes as in full harmony and counterpoint, are
really nothing more than the few simple chords which must fall naturally under
the hand of any one holding the instrument, and such as would form a child’s
first lessons. First the chord, G C E, and then that of B D F, form the entire
bass of the only two lessons he has translated; and though from B to F is
a “false fifth,” it must be shown that the harper derived his knowledge of
the instrument from the Church, before the assertion that it is more modern
harmony than then in use can have any weight. In England, at least, not
only the evidence of Giraldus, but all other that I can find, is against such a
supposition. I have before alluded to the Romance of Horn-Child, (note \textit{c}, to
page 9), and here give the passage, to prove that such knowledge was not
derived from the Church, as well as to show what formed a necessary part of
education for a knight or warrior. It is from that part of the story where
Prince Horn appears at the court of the King of Westnesse.

\settowidth{\versewidth}{His steward, and [to] him said thus}
\begin{dcverse}
\vin\vin\textsc{Original Words.}

“The kyng com in to halle,\\
Among his knyhtes alle,\\
Forth he clepeth Athelbrus,\\
His stiward, and him seide thus:\\
‘Stiward, tac thou here\\
My fundling, for to lere\\
Of thine, mestere\\
Of wode and of ryuere,\\
\textit{Ant toggen o the harpe}\\
\textit{With is nayles sharpe}.\\
Ant tech him alle the listes\\
That thou euer wystest,\\
Byfore me to keruen.\\
And of my coupe to seruen:

\vin\vin\textsc{Words Modernized.}

The king came into [the] hall\\
Among his knights all,\\
Forth he calleth Athelbrus,\\
His steward, and [to] him said thus\\
“Steward, take thou here\\
My foundling, for to teach\\
Of thy mystery\\
Of wood and of river,\\
And to play on the harp\\
With his nails sharp.\\
And teach him all thou listest,\\
That thou ever knewest,\\
Before me to carve\\
And my cup to serve: 
\end{dcverse}

\pagebreak