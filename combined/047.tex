\renewcommand\versoheader{richard sheale.---extinction of minstrelsy.}


Sheale was a Minstrel in the service of Edward, Earl of Derby, who died in
1574, celebrated for his bounty and hospitality, of whom Sheale speaks most
gratefully, as well as of his eldest son, Lord Strange. The same MS. contains an
“Epilogue” on the Countess of Derby, who died in January, 1558, and his
version of Chevy Chace must have been written at \textit{least} ten years before the
latter date, if it be the one mentioned in the Complaynte of Scotland, which was
written in 1548.

In the thirty-ninth year of Elizabeth, an act was passed by which “Minstrels,
wandering abroad” were held to be “rogues, vagabonds, and sturdy beggars,”
and were to be punished as such. This act seems to have extinguished the profession
of the Minstrels, who so long had basked in the sunshine of prosperity.
The name, however, remained, and was applied to itinerant harpers, fiddlers,
and other strolling musicians, who are thus described by Puttenham, in his \textit{Arte
of English Poesie}, printed in 1589. Speaking of ballad music, he says, “The
over busy and too speedy return of one manner of tune, doth too much annoy,
and, as it were, glut the ear, unless it be in small and popular musicks sung by
these \textit{Cantabanqui} upon benches and barrels’ heads, where they have none other
audience than boys or country fellows that pass by them in the street; or else by
blind harpers, or such like tavern minstrels, that give a fit of mirth for a groat;
and their matter being for the most part stories of old time, as the Tale of Sir
Topas, Bevis of Southampton, Guy of Warwick, Adam Bell and Clym of the
Clough, and such other old romances or historical rhimes, \textit{made purposely for} the
recreation of the common people at Christmas dinners and bride-ales, and in
taverns and alehouses, and such other places of base resort. Also they” [these
short times] “be used in Carols and Rounds, and such like light and lascivious
poems, which are commonly more commodiously uttered by these buffons, or vices
in plays than by another~person.”

Ritson, whose animosity to Percy and Warton seems to have extended itself
to the whole minstrel race, quotes, with great glee, the following lines on their
downfall, which were written by Dr.~Bull, a rival musician:—

\settowidth{\versewidth}{He turned the Minstrels out of doors,}
\begin{scverse}
“When Jesus went to Jairus’ house,\\
(Whose daughter was about to die)\\
He turned the Minstrels out of doors,\\
Among the rascal company:\\
Beggars they are with one consent,—\\
And rogues, \textit{by act of Parliament.}” 
\end{scverse}

%\vspace{2\baselineskip}
\centerrule
\pagebreak