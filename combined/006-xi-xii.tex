%%===============================================================================
%%xi
\changefontsize{1.08\defaultfontsize}

Many of our ballad-tunes were not fitted for dancing, and therefore were not 
included in \textit{The Dancing Master}; but a considerable number of these is supplied
by the ballad-operas which were printed after the extraordinary success of \textit{The
Beggars' Opera} in 1728.

I might name many other books which have contributed their quota, especially
\textit{Wit and Mirth, or Pills to purge Melancholy}, with its numerous editions from 1699
to 1720,—but all are indicated in the work. I cannot, however, refrain from
some notice of the numerous foreign publications in which our national airs are
included. Sometimes they are in the form of country dances,—at others, as
songs, or as tunes for the lute. I have before me three sets of country dances
printed in Paris during the last century, and as one of these is the “5\textsuperscript{ême} Recueil
d’Anglaises telle qu’elles se dansent ché la Reine,” there must have been at least
four more of that series. Many of my readers may not know that the “Quadrille
de Contredanses” in which they join under the name of “a set of Quadrilles,”
is but our old “Square Country Dance” come back to us again. The
new designation commenced no longer ago than 1815,—just after the war.

Horace Walpole tells us in his letters, that our country dances were all the rage
in Italy at the time he wrote, and, as collections were printed at Manheim, Munich,
in various towns of the Netherlands, and even as far North as Denmark, it is
clear that they travelled over the greater part of Europe. The Danish collection
now before me consists of 296 pages, with a volume of nearly equal thickness to
describe the figures.

Some of the works printed in Holland during the seventeenth century, which
contain English airs, have materially assisted in the chronological arrangement.
Of these, Vallet’s \textit{Tablature de Luth, entitulé Le Secret des Muses}, was published
at Amsterdam iu 1615. \textit{Bellerophon, of Lust tot Wysheit}, in 1620, and other
editions at later dates. Valerius’s \textit{Nederlandtsche Gedenck-Clanck}, at Haerlem,
in 1626. Starter’s \textit{Friesche Lust-Hof}, and his \textit{Boertigheden}, in 1634, and other
editions without date. Camphuysen’s \textit{Stichtelycke Rymen}, 1647, 1652, and
without date. Pers’s \textit{Gesangh der Zeeden}, 1662, and without date. \textit{Urania},
1648, and without date.

It is only necessary to remark upon the chronological arrangement, that, in
order to ascertain what airs or ballads were popular in any particular reign, the
reader will have occasion to refer also to those which precede it. Without endless
repetition, it could not have been otherwise.

Facsimiles of a few of the manuscripts will be found in the following~pages.

I have now the pleasing duty of returning thanks to those who have assisted
me in this collection; and first to Edward F. Rimbault, LL.D., and Mr. G. A.
Macfarren. Dr.~Rimbault has been the largest contributor to my work, and a
contributor in every form. To him I am indebted for pointing out many airs
which would have escaped me, and for adding largely to my collection of notices
of others; for the loan of rare books; and for assisting throughout with his extensive
musical and bibliographical knowledge. To Mr. G. A. Macfarren for 
having volunteered to re-arrange the \pagebreak
airs which were to be taken from my former 
%%===============================================================================
%%xii
collection, as well as to harmonize the new upon a simple and consistent plan 
throughout. In my former work, some had too much harmony, and others even
too little, or such as was not in accordance with the spirit of the words. The
musician will best understand the amount of thought required to find characteristic
harmonies to melodies of irregular construction, and how much a simple air
will sometimes gain by being well fitted.

To the Right Hon. the Earl of Abergavenny I am indebted for the loan of
“Lady Nevell’s Virginal Book,” a manuscript collection of music for the virginals,
transcribed in 1591. To the late Lord Braybrooke I owe the means of
access to Pepys’s collection of ballads, which was indispensable for the due
prosecution of the work.

To Mr. J. Payne Collier, F.S.A., I am indebted for the loan of a valuable
manuscript of poetry, transcribed in the reign of James I., containing much of
still earlier date; and for free access to his collection of ballads and of rare books:
to Mr. George Daniel, of Canonbury, for copies of several Elizabethan ballads,
which are to be found only in his unique collection; and to Mr. David Laing,
F.S.A. Scot., for the loan of several rare books.

To Sir Frederick Madden, K.H., Keeper of the Manuscripts in the British
Museum, I am indebted for much information about manuscripts, readily given,
and with such uniform courtesy, that it becomes an especial pleasure to
acknowledge it.

\medskip
\hfill W.C.\hspace*{4em}
\medskip

3, \textit{Harley Place (N. W.),}

\qquad \textit{or} 201, \textit{Regent Street. (W.)} 

\pagebreak
%

