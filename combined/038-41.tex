%===============================================================================
%038

\changefontsize{1.06\defaultfontsize}

Du Fresne in his Glossary (art. Ministrelli), speaking of the King of the
Minstrels, says, “His office and power are defined in a French charter of
Henry~IV., king of England, in the Monasticon Anglicanum, vol. i., p.~355;”
but though I have searched through Dugdale’s Monasticon, I find no such
charter.


In 1402, we find the before-mentioned statute against the Welsh bards,%\linebreak
(4 Henry~IV., c.~27).\myfootnote{ %a
It runs in these terms: “Item, pour eschuir plusieurs
diseases et mischiefs qont advenuz devaunt ces heures en
la terre de Gales par plusieurs Westours Rymours,
Minstralx et autres Vacabondes, ordeignez est, et
establiz, que nul Westour, Rymour Minstral, ne Vacabond
soit aucunemeut sustenuz en la terre de Gales pur
faire kymorthas ou coillage sur la commune poeple
iliocques.”
} %end footnote
As they had excited their countrymen to rebellion
against the English government, it is not to he wondered (says Percy) that the
Act is conceived in terms of the utmost indignation and contempt against this
class of men, who are described as Rymours, Ministralx, which are apparently
here used as only synonymous terms to express the Welsh bards, with the usual
exuberance of our Acts of Parliament; for if their Ministralx had been mere
musicians, they would not have required the vigilance of the English legislature
to suppress them. It was their songs, exciting their countrymen to insurrection,
which produced “les diseases et mischiefs en la terre de Gales.”

At the coronation of Henry V., which took place in Westminster Hall (1413),
we are told by Thomas de Elmham, that “the number of harpers was exceedingly
great; and that the sweet strings of their harps soothed the souls of the guests
by their soft melody.” He also speaks of the dulcet sounds of the united
music of other instruments, in which no discord interrupted the harmony,
as “inviting the royal banqueters to the full enjoyment of the festival.”—
(Vit. et. Gest. Henr. V., c. 12, p.~23.) Minstrelsy seems still to have
flourished in England, although it had declined so greatly abroad; the Provençals
had ceased writing during the preceding century. When Henry was preparing
for his great voyage to France in 1415, an express order was given for his
minstrels to attend him.—(Rymer, ix.,~255.) Monstrelet speaks of the English
camp resounding with the national music (170) the day preceding the battle of
Agincourt, but this must have been before the king “gave the order for silence,
which was afterwards strictly observed.”

When he entered the City of London in triumph after the battle, the gates and
streets were hung-with tapestry representing the histories of ancient heroes; and
boys with pleasing voices were placed in artificial turrets, singing verses in his
praise. But Henry ordered this part of the pageantry to cease, and commanded
that for the future no “ditties should be made and sung by Minstrels\myfootnote{ %b
Hollinshed, quoting from Thomas de Elmham, whose
words are, “Quod cantus de suo triumpho fieri, sen per
\textit{Citharistas} vel alios quoscunque cantari penitus prohibebat.”
It will be observed that Hollinshed translates
Citharistas (literally harpers) minstrels.
} or others,”
in praise of the recent victory; “for that he would whollie have the praise
and thankes altogether given to God.”

Nevertheless, among many others, a minstrel-piece soon appeared on the
\textit{Seyge of Harflett} (Harfleur), and the \textit{Battayle of Agynkourte}, “evidently,” says
Warton, “adapted to the harp,” and of 
which he has printed some portions. 
\pagebreak
%===============================================================================
%39
(Hist. Eng. Poet., vol. ii. p.~257.) Also the following song, which Percy has
printed in his Reliques of Ancient Poetry, from a M.S. in the Pepysian Library,
and Stafford Smith, in his Collection of English Songs, 1779 fol., in fac-simile of
the old notation, as well as in modern score, and with a chorus in three parts to
the words, “Deo gratias, Anglia, redde pro victoria.” The tune is here given
with the first verse of the words,\myfootnote{ %a
I do not intend to reprint songs or Ballads that are
contained in Percy's Reliques of Ancient Poetry, without
some particular motive, for that delightful book can be
purchased in many shapes and at a small cost. As a
general rule, the versions given by Percy are best suited
to music, because more metrical than others, although
they maybe less exactly and minutely in accordance
with old copies, which are often very carelessly printed
or transcribed.
} %end footnote
for although the original is a regular composition
in three parts, it serves to shew the state of melody at an early period, and
the subject is certainly a national one.

\renewcommand\rectoheader{henry v.}

\musictitle{Song on the Victory of Agincourt.}

\musicinfo{Slowly and Majestically.}{1415.}

\lilypondfile[staffsize=13]{lilypond/039-song-on-the-victory-of-agincourt}

There are also two well-known ballads on the Battle of Agincourt; the one
commencing “A council grave our king did hold;” the other “As our king lay
musing in his bed,” which will be noticed under later dates; and a three-men’s
song, which was sung by the tanner and his fellows, to amuse the guests, in
Heywood’s play, \textit{King Edward IV}., beginning—

\settowidth{\versewidth}{Agincourt! Agincourt! know ye not Agincourt?}
\begin{scverse}\vleftofline{“}Agincourt! Agincourt! know ye not Agincourt?\\
Where the English slew or hurt\\
\vin All the French foemen?” \&c.
\end{scverse}

Although Henry had forbidden the minstrels to celebrate his victory, the order
evidently did not proceed from any disregard for the professors of music or of
song, for at the Feast of Pentecost, which he celebrated in 1416, having the
Emperor and the Duke of Holland as his guests, he ordered rich gowns for sixteen 
of his minstrels. And having before his \pagebreak 
 death orally granted an annuity of an 
%===============================================================================
%40
hundred shillings to each of his minstrels, the grant was confirmed in the first
year of his son, Henry VI. (\ad 1423), and payment ordered out of the exchequer.
Both the biographers of Henry declare his love for music.\myfootnote{ %a
“Musicis delectabatur.”—Tit. Liv., p. 5. “Instr\-umentis
organicis pluri\-mum deditus”—Elmham.
} %end footnote
Lydgate
and Occleve, the poets whom he patronized, attest also his love of literature, and
the encouragement he gave to it.

John Lydgate, Monk of Bury St. Edmunds, describes the minstrelsy of his
time less completely, but in nearly the same terms as Chaucer.

Lydgate was a very voluminous writer. Ritson enumerates 251 of his pieces,
and the list is far from being complete. Among his minor pieces are many songs
and ballads, chiefly satirical, such as “On the forked head-dresses of the ladies,”
on “Thievish Millers and Bakers,” \&c. A selection from these has been recently
printed by the Percy Society.

Among the devices at the coronation banquet of Henry VI. (1429), were, in
the first course, a “sotiltie” (subtlety) of St. Edward and St. Lewis, in coat
armour, holding between them a figure like King Henry, similarly armed, and
standing with a \textit{ballad under his feet}. “In the second, a device of the Emperor
Sigismund and King Henry V., arrayed in mantles of garter, and a figure like
Henry VI. kneeling before them with \textit{a ballad against the Lollards};\myfootnote{ %b
Ritson has printed one of these ballads against the
Lollards, in his Ancient Songs, p.~63, 1790, taken, from
\textit{MS. Cotton, Vespasian, B. 16. Brit. Mus}.
} %end footnote
and in the
third, one of our Lady, sitting with her child in her lap, and holding a crown in
her hand, St. George and St. Denis kneeling on either side, presenting to her
King Henry \textit{with a ballad in his hand}.\myfootnote{ %c
Quoted by Sharon Turner, from Fab. 419.
} %end footnote
These subtleties were probably devised
by the clergy, who strove to smother the odium which, as a body, their vices had
excited, by turning public attention to the further persecution of the Lollards.\myfootnote{ %
Sir John Oldcastle, Lord Cohham, bad been put to
death in the preceding reign.
} %end footnote
In a discourse which was prepared to be delivered at the Convocation of the
Clergy, ten days after the death of Edward IV., and which still exists in MS.
(MS. Cotton Cleopatra, E. 3), exhorting the clergy to amendment, the writer
complains that “The people laugh at us, and make us their songs all the day
long.” Vicious persons of every description had been induced to enter the church
on account of the protection it afforded against the secular power, and the facilities
it provided for continued indulgence in their vices.

In that age, as in more enlightened times, the people loved better to be pleased
than instructed, and the minstrels were often more amply paid than the clergy.
During many of the years of Henry VI., particularly in the year 1430, at the
annual feast of the fraternity of the \textsc{Holie Crosse}, at Abingdon, a town in
Berkshire, twelve priests each received four pence for singing a dirge: and the
same number of minstrels were rewarded each with two shillings and four pence,
besides diet and horse-meat. Some of these minstrels came only from Maydenhithe, 
or Maidenhead, a town at no great distance, in the same county. (Liber
Niger, p.~598.) In the year 1441, eight priests were hired from Coventry,
to assist in celebrating a yearly obit in the church \pagebreak 
of the neighbouring priory of 
%===============================================================================
%41
Maxtoke; as were six minstrels (\textsc{Mimi}) belonging to the family of Lord Clinton,
who lived in the adjoining Castle of Maxtoke, to sing, harp, and play in the hall
of the monastery, during the extraordinary refection allowed to the monks on that
anniversary. Two shillings were given to the priests, and four to the minstrels:
and the latter are said to have supped in \textit{camera picta}, or the painted chamber of
the convent, with the sub-prior, on which occasion the chamberlain furnished
eight massive tapers of wax. (Warton, vol. ii., p.~309.) However, on this occasion,
the priests seem to have been better paid than usual, for in the same year
(1441) the prior gave no more than sixpence to a preaching friar.

\renewcommand\rectoheader{henry vi.}

As late as in the early part of the reign of Elizabeth, we find an entry in the
books of the Stationers’ Company (1560) of a similar character: Item, payd to
the preacher, 6\textit{s}. 2\textit{d}. Item, payd to the minstrell, 12\textit{s}.; so that even in the
decline of minstrelsy, the scale of remuneration was relatively the same.

A curious collection of the songs and Christmas carols of this reign (Henry~VI.) 
have been printed recently by the Percy Society. (Songs and Carols, No. 73.)

The manuscript book from which they are taken, had, in all probability, belonged
to a country minstrel who sang at festivals and merry makings, and it has been,
most judiciously, printed entire, as giving a general view of the classes of poetry
then popular. A proportion of its contents consists of carols and religious songs,
such as were sung at Christmas, and perhaps at other festivals of the Church.
Another class, in which the MS. is, for its date, peculiarly rich, consists of
drinking songs. It also contains a number of those satirical songs against the
fair sex, and especially against shrews, which were so common in the middle ages,
and have a certain degree of importance as showing the condition of private
society among our forefathers. The larger number of the songs, including some
of the most interesting and curious, appear to be unique, and the others
are in general much better and more complete copies than those previously
known (viz. in MS. Sloane, No. 2593, Brit. Mus). The editor of the MS.
(Mr. T. Wright) observes that “The great variations in the different copies of
the same song, show that they were taken down from oral recitation, and had
often been preserved by memory among minstrels, who were not unskilful at
composing, and who were not only in the habit of, voluntarily or involuntarily,
modifying the songs as they passed through their hands, and adding or omitting
stanzas, but of making up new songs by stringing together phrases and lines, and
even whole stanzas from the different compositions which were imprinted on their
memories.” But what renders the manuscript peculiarly interesting, is, that it
contains the melodies of some of the songs as well as the words. From this it
appears that the same tune was used for different words. At page 62 is a note,
which in modern spelling is as follows: “This is the tune for the song following;
if so be that ye will have another tune, it may be at your pleasure, for I have set
all the song.” The words of the carol, “Nowell, Nowell,” (Noel) are written
under the notes, but the wassail song that follows, and for which the tune was also
intended, is of a very opposite character, “Bryng us in good ale.” I have
printed the first verse of each under the tune, but it requires to be sung more
quickly for the wassail song than for the carol. 

\pagebreak