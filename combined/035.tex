%===============================================================================
%035

\changefontsize{1.01\defaultfontsize}

\renewcommand\rectoheader{notices of music by chaucer.}
\settowidth{\versewidth}{\ldots\  . “some, for they can synge and daunce,}

The Wife of Bath says (lines 5481 and 2, and 6039 and 40), that wives were
chosen—
\begin{scverse}\ldots\  . “	some, for they \textit{can synge and daunce},\\
And some for gentilesse or daliaunce\ldots\  \\
How couthe I dannce to an harpe smale,\\
And synge y-wys as eny nightyngale.”
\end{scverse}
I shall conclude Chaucer’s inimitable descriptions of character with that of his
Oxford Clerk, who was so fond of books and study, that he loved Aristotle better

\begin{scverse}\vleftofline{“}Than robès riche, or fidel or sautrie\ldots\  \\
Souning in moral virtue was his speech,\\
And gladly would he lerne and gladly teche.”
\end{scverse}
We learn from the preceding quotations, that country squires in the fourteenth
century could pass the day in singing, or playing the flute, and that some could
“Songès well make and indite:” that the most attractive accomplishment in
a young lady was to be able to sing well, and that it afforded the best chance of
her obtaining an eligible husband; also that the cultivation of music extended
to every class. The Miller, of whose education Pierce Plowman speaks so slightingly,
could play upon the bagpipe; and the apprentice both on the ribible and
gittern. The musical instruments that have been named are the harp, psaltry,
fiddle, bagpipe, flute, trumpet, rote, rebec, and gittern. There remain the lute,
organ, shalm (or shawm), and citole, the hautboy (or wayte), the horn, and
shepherd’s pipe, and the catalogue will be nearly complete, for the cittern or
cithren differed chiefly from the gittern, in being strung with wire instead of gut,
or other material. The sackbut was a bass trumpet with a slide,\myfootnote{%a
“As he that plaies upon a Sagbut, by pulling it up
and down alters his tones and tunes.”—\textit{Burton’s Anatomy
of Melancholy}, 8vo. Edit, of 1800, p.~379.
} %end footnote 
like the modern
trombone; and the dulcimer differed chiefly from the psaltry in the wires being
struck, instead of being twitted by a plectrum, or quill, and therefore requiring
both hands to perform on it.

In the commencement of the Pardoner’s Tale he mentions lutes, harps, and
gitterns for dancing, as well as singers with harps; in the Knight’s Tale he represents
Venus with a citole in her right hand, and the organ is alluded to both in
the History of St. Cecilia, and in the tale of the Cock and the Fox.

In the House of Fame (Urry’s Edit., line 127 to 136), he says:
\settowidth{\versewidth}{How that he dorstè not his sorwe [sorrow] telle,}
\begin{scverse}
\vleftofline{“}That madin loude Minstralsies\\
In Cornmuse [bagpipe] and eke in \textit{Shalmies},\myfootnote{%b
A very early 1414 drawing of the Shalm, or Shawm, is in
one of the illustrations to a copy of Froissart, in the Brit.
Mus.—\textit{Royal} MSS. 18, E. Another in Commenius’
Visible World, translated by Hoole, 1650, (he translates
the Latin word \textit{gingras}, shawm,) from which it is copied
into Cavendish’s Life of Wolsey, edited by Singer, vol. i.
p.~114., Ed. 1825. The modern clarionet is an improvement
upon the shawm, which was played with a reed,
like the wayte, or hautboy, but being a bass instrument,
with about the compass of an octave, had probably more
the tone of a bassoon. It was used on occasions of state.
“What \textit{stately} music have you? You have shawms?
Ralph plays a stately part, and he must needs have
shawms.”—\textit{Knight of the Burning Pestle}. Drayton speaks
of it as shrill-toned: “E’en from the \textit{shrillest shawm}, unto
the cornamute."—\textit{Polyolbion}, vol. iv., p.~376. I conceive
the shrillness to have arisen from over-blowing, or else
the following quotation will appear contradictory:—
\settowidth{\versewidth}{It mountithe not to hye, but \textit{kepithe rule and space}.}
\begin{fnverse}
“A \textit{Shawme} maketh a \textit{swete} sounde, \\
\hspace{\vgap} for he \textit{tunythe the basse},\\
It mountithe not to hye, but \textit{kepithe rule and space}.\\
Yet \textit{yf it be blowne withe to vehement a wynde,}\\
It \textit{makithe it to mysgoverne out of his kynde}.”
\end{fnverse}
This is one of the “proverbis” that were written about
the time of Henry VII., on the walls of the Manor House
at Leckingfield, near Beverley, Yorkshire, anciently belonging
to the Percys, Earls of Northumberland, but uow
destroyed. There were many others relating to music,
and musical instruments (harp, lute, recorder, claricorde,
clarysymballis, virgynalls, clarion, organ, singing, and
musical notation,) and the inscribing them on the walls
adds another to the numberless proofs of the estimation
in which the art was held. A manuscript copy of them
is preserved in Bib, Reg. 18. D. 11. Brit. Mus.
}  \\
\end{scverse}


\pagebreak