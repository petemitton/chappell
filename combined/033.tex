%===============================================================================
%033

 \changefontsize{1.01\defaultfontsize}

\renewcommand\rectoheader{pierce plowman.--chaucer.}
\noindent He says, however, of himself, in allusion to the minstrels:—

\settowidth{\versewidth}{Ich can nat tabre, ne trompe, ne telle faire gestes,”}
\begin{scverse}\vleftofline{“}Ich can nat tabre, ne trompe, ne telle faire gestes,\\
Ne fithelyn, at fe[a]stes, ne harpen:\\
Japen ne jagelyn, ne gentilliche pipe;\\
Nother sailen [leap or dance], ne sautrien, ne singe with the giterne.”
\end{scverse}
He also describes his Friar as much better acquainted with the “\textit{Rimes of
Robinhode} and of \textit{Randal, erle of Chester},” than with his Paternoster.

Chaucer, throughout his works, never loses an opportunity of describing or
alluding to the general use of music, and of bestowing it as an accomplishment
upon the pilgrims, heroes, and heroines of his several tales or poems, whenever
propriety admits. We may learn as much from Chaucer of the music of his day,
and of the estimation in which the art was then held in England, as if a treatise
had been written on the subject.

Firstly, from the Canterbury Tales, in his description of the Squire (line 91
to~96), he says:—

\begin{scverse}\vleftofline{“}\textit{Syngynge he mas, or flowtynge} [fluting] \textit{al the day};\\
He was as fresh as is the moneth of May:\\
Short was his gonne, with sleevès long and wyde;\\
Well cowde he sitte on hors, and faire ryde.\\
\textit{He cowde songes wel make and endite,}\\
Juste (fence) and eke daunce, and wel p[o]urtray and write.”
\end{scverse}

Of the Nun, a Prioress (line 122 to 126), he says:—
\begin{scverse}\vleftofline{“}\textit{Ful wel sche sang the servise devyne},\\
\textit{Entuned in hire nose ful seemyly};\\
And Frensch sche spak ful faire and fetysly [neatly],\\
Aftur the schole of Stratford attè Bowe,\\
For Frensch of Parys was to hire unknowe” [unknown].
\end{scverse}

The Monk, a jolly fellow, and great sportsman, seems to have had a passion for
no music but that of hounds, and the bells on his horse’s bridle (line 169 to~171):
\begin{scverse}\vleftofline{“}And whan he rood [rode], men might his bridel heere\\
Gyngle in a whistlyng wynd so cleere,\\
And eke as lowde as doth the chapel belle.”
\end{scverse}

Of his Mendicant Friar, whose study was only to please (lines 235-270),
he says:
\begin{scverse}\vleftofline{“}And certayn he hadde a mery note;\\
\textit{Wel couthe he synge and playe on a rote }[hurdy-gurdy]\ldots\  \\
Somewhat he lipsede [lisped] for wantounesse,\\
To make his Englissch swete upon his tunge;\\
And \textit{in his harpyng, whan that he had sunge},\\
His eyghen [eyes] twynkeled in his he[a]d aright,\\
As don the sterrès [do the stars] in the frosty night.”
\end{scverse}

Of the Miller (line 564 to 568), he says:—
%\settowidth{\versewidth}{Was never trompe [trumpet] of half so gre[a]t a soun” (sound).}
\begin{scverse}\vleftofline{“}Wel cowde he ste[a]le corn, and tollen thries [take toll thrice];\\
And yet he had a thombe of gold,\footnotemark\ pardé,\\
A whight cote and blewe hood we[a]red he; \\
\end{scverse}

\footnotetext{ %
Tyrwhitt says there is an old proverb—“Every \textit{honest}
miller has a thumb of gold.” Perhaps it means that
nevertheless he was as honest as his brethren. There are
many early songs on thievish millers and bakers.
} %end footnotetext


\pagebreak