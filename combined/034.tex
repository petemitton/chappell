%===============================================================================
%034
\changefontsize{1.02\defaultfontsize}
 
\settowidth{\versewidth}{\textit{A baggepipe cowde he blowe and sowne} [sound],}
\begin{scverse}
\textit{A baggepipe cowde he blowe and sowne} [sound],\\
And therewithal he brought us out of towne.”\myfootnote{ %a
A curious reason for the use of the Bagpipe in Pilgrimages
will he found in State Trials—Trial of William
Thorpe. Henry IV., an. 8, shortly after Chaucer’s death.
“I say to thee that it is right well done, that Pilgremys
have with them both Syngers, and also Pipers, that whan
one of them, that goeth bar[e]fo[o]te, striketh his too upon
a stone, and hurteth hym sore, and maketh hym to blede;
it is well done that he or his fel[l]ow begyn than a Songe,
or else take out of his bosome a \textit{Baggepype} for to drive
away with soche myrthe the hurte of his felow.”
} %end footnote
\end{scverse}
Of the Pardoner (line 674 to 676):—
\begin{scverse}\vleftofline{“}\textit{Ful lowde he sang, ‘Come hider, love, to me}.’\\
This Sompnour bar[e] to him a \textit{stif burdoun},\footnotemark \\
Was never trompe [trumpet] of half so gre[a]t a soun” (sound).
\end{scverse}
\footnotetext{\scriptsizer %b
This Sompnour (Sumner or Summoner to the Ecclesiastical
Courts, now called Apparitor) supported him by
singing the \textit{burden}, or \textit{bass}, to his song in a deep loud
voice. \textit{Bourdon} is the French for \textit{Drone}; and \textit{Foot},
\textit{Under-song}, and \textit{Burden} mean the same thing, although
Burden was afterwards used in the sense of Ditty, or
any line often recurring in a song, as will be seen here-
after.
} %end footnotetext
Of the poor scholar, Nicholas (line 3213 to 3219):—
\begin{scverse}\vleftofline{“}And al above ther lay a gay \textit{sawtrye} [psaltry],\\
On which he made, a-nightes, melodye\\
So swetely, that al the chambur rang:\\
And \textit{Angelus ad Virginem} he sang.\\
And after that he sang \textit{The Kynge’s note};\\
Ful often blessed was his mery throte.”
\end{scverse}
Of the Carpenter’s Wife (lines 3257 and 8):--
\begin{scverse}\vleftofline{“}But of her song, it was as lowde and yerne [brisk]\\
As eny swalwe [swallow] chiteryng on a berne” [barn].
\end{scverse}
Of the Parish Clerk, Absolon (lines 3328 to 3335):—
\begin{scverse}\vleftofline{“}In twenty manners he coude skip and daunce,\\
After the schole of Oxenfordè tho,\\
And with his leggès casten to and fro;\\
\textit{And pleyen songes on a small Rubible}\myfootnote{Ribible (the diminutive of Ribibe or Rebec) is a small
fiddle with three strings.} [Rebec],\\
\textit{Ther-to he sang som tyme a lowde quynyble};\footnotemark\\
\textit{And as wel coude he pleye on a giterne}:\\
In al the toun nas [nor was] brewhous ne taverne\\
That he ne visited with his solas” [solace].
\end{scverse}
\footnotetext{\scriptsizer %d
To sing a “quinible” means to descant by singing
fifths on a plain-song, and to sing a “quatrible” to descant
by fourths. The latter term is used by Cornish in
his Treatise between Trowthe and Enformacion. 1528.
} %end footnotetext
He serenades the Carpenter’s Wife, and we have part of his song (lines 3352—64):%
\begin{scverse}\vleftofline{“}The moone at night ful cleer and brightè schoon,\\
And Absolon his giterne hath i-take,\\
For paramours he seyde he wold awake\ldots\  \\
He syngeth in hys voys gentil and smal—\\
Now, deere lady, if thi wille be,\\
I pray you that ye wol rewe [have compassion] on me.’\\
Full wel acordyng to his gyternyng,\\
This carpenter awook, and herde him syng.”
\end{scverse}
Of the Apprentice in the Cook’s Tale, who plays both on the ribible and gitterne:
\begin{scverse}\vleftofline{“}At every brideale wold he synge and hoppe;\\
He loved bet [better] the taverne than the schoppe.” 
\end{scverse}


\pagebreak