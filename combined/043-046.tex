%===============================================================================
%043
\renewcommand\rectoheader{christmas carol and wassail song.}
\changefontsize{1.01\defaultfontsize}

The notation of the original is in semibreves, minims, and crotchets, which
are diminished to crotchets, quavers, and semiquavers, as became necessary in
modernizing the notation; for the quickest note then in use was the crotchet.\myfootnote{ %a
After the Percy Society had printed the Songs, I was
to have had the opportunity of transcribing \textit{all} the Music;
but, in the mean time, the bookbinder to whom this rare
MS. was entrusted, disappeared, and with him the manuscript, 
which is, perhaps, already in some library in the
United States.
} %end footnote
The Christmas carol partakes so much of the character of sacred music, that it is
not surprising it should be in an old scale. If there were not the flat at the signature,
which takes off a little of the barbarity, it would be exactly in the eighth
Gregorian tone.

There are seven verses to the carol, but as they are not particularly interesting,
perhaps the words of the wassail song will be preferred, although we should not
now sing of “our blessed lady,” as was common in those days.
\settowidth{\versewidth}{Bring us in no brown bread, for that is made of bran,}
\begin{scverse}
Bring us in no brown bread, for that is made of bran,\\
Nor bring us in no white bread, for therein is no gain,\\
\hspace{\vgap}But bring us in good ale, and bring us in good ale;\\
\hspace{\vgap}For our blessed Lady’s sake, bring us in good ale.

Bring us in no beef, for there is many bones,\\
But bring us in good ale, for that go’th down at once. And bring, \&c.

Bring us in no bacon, for that is passing fat,\\
But bring us in good ale, and give us enough of that. And bring, \&c.

Bring us in no mutton, for that is passing lean,\\
Nor bring us in no tripes, for they be seldom clean. But bring, \&c.

Bring us in no eggs, for there are many shells,\\
But bring us in good ale, and give us nothing else. But bring, \&c.

Bring us in no butter, for therein are many hairs,\\
Nor bring us in no pig’s flesh, for that will make us bears. But bring, \&c.

Bring us in no puddings, for therein is all God’s good,\\
Nor bring us in no venison, that is not for our blood. But bring, \&c.

Bring us in no capon’s flesh, for that is often dear,\\
Nor bring us in no duck’s flesh, for they slobber in the mere, [mire]\\
\hspace{\vgap}But bring us in good ale, and bring us in good ale,\\
\hspace{\vgap}For our blessed lady’s sake, bring us in good ale.
\end{scverse}

An inferior copy of this song, without music, is in Harl. M.S., No. 541, from
which it has been printed in Ritson’s Ancient Songs, p. xxxiv. and xxxv.

With the reign of Edward IV. we may conclude the history of the \textit{old wandering}
minstrel. In 1469, on a complaint that persons had collected money in different
parts of the kingdom by assuming the title and livery of the king’s minstrels, he
granted to Walter Halliday, \textit{Marshal}, and to seven others whom he names,
a charter of incorporation. They were to be governed by a marshal appointed for
life, and two wardens to be chosen annually, who were authorized to admit members; 
also to examine the pretensions of all who exercised the minstrel profession, \pagebreak
and to regulate, govern, and punish them 
throughout the realm (those of Chester 
%===============================================================================
%44
excepted). “This,” says Percy, “seems to have some resemblance to the Earl
Marshal’s court among the heralds, and is another proof of the great affinity and
resemblance which the minstrels bore to the College of Arms.” Walter Halliday,
above mentioned, had been retained in the service of the two preceding monarchs,
and Edward had granted him an annuity of ten marks for life, in~1464.

%%\changefontsize{1.06\defaultfontsize}
In this reign we find also mention of a \textit{Serjeant} of the minstrels, who upon
one occasion did his royal master a singular service, and by which his ready access
to the king at all hours is very apparent: for “as he [K. Edward IV.] was in
the north contray, in the Monneth of Septembre, \textit{as he lay in his bedde}, one
named Alexander Carlile, that was Sarjaunt of the Mynstrellis, cam to him
in grete hast, and badde hym aryse, for he hadde enemyes cumming for to take
him, the which were within six or seven miles,” \&c.

Edward seems to have been very liberal to his minstrels. He gave to several
annuities of ten marks a year (6 Parl. Rolls, p.~89), and, besides their
regular pay, with clothing and lodging for themselves and \textit{their horses}, they had
two servants to carry their instruments, four gallons of ale per night, wax candles,
and other indulgences. The charter is printed in Rymer, xi. 642, by Sir
J. Hawkins, vol. iv., p.~366, and Burney, vol. ii., p.~429. All the minstrels
have English names.

When Elizabeth, his queen, went to Westminster Abbey to be church\-ed (1466),
she was preceded by troops of choristers, chanting hymns, and to these succeeded
long lines of the noblest and fairest women of London and its vicinity, attended by
bands of musicians and trumpeters, and forty-two royal singers. After the banquet
and state ball, a state concert commenced, at which the Bohemian ambassadors
were present, and in their opinion as well as that of Tetzel, the German who accompanied
them, and who has also recounted their visit to England, no better
singers could be found in the whole world,\myfootnote{ %a
Tetzel says, “Nach dem Tantz do muosten des
Kunigs Cantores kumen und muosten singen\ldots\  ich
mein das, in der Welt, nit besser Cantores sein.” “\textit{Des
böhmischen Herrn Leo’s von Rozmital Ritter,—Hof und
Pilger—Reise, 1465-1467,” \&c., 8vo., Stuttgart}, 1844, p.~157.

Again Tetzel says, “Do hörten wir das aller kostlichst
Korgesang, das alls gesatzt was, das lieblich zu hören
was.”—\textit{Ib}. p.~158.

Leo Von Rozmital, brother of the Queen of Bohemia
says "Musicos nullo uspiam in loco jucundiores et
suaviores audivimus, quam ibi: eorum chorus sexaginta,
circiter cantoribus constat,” —\textit{Ib}. p.~42.
} %end footnote
than those of the English king.
These ambassadors travelled through France, Belgium, Spain, Portugal, Italy,
and parts of Germany, as well as England, affording them, therefore, the widest
field for comparison with the singers of other countries.

At this time every great family had its establishment of musicians, and among
them the harper held a prominent position. Some who were less wealthy retained
a harper only, as did many bishops and abbots. In Sir John Howard’s expenses
(1464) there is an entry of a payment as a new year’s gift to Lady Howard’s
grandmother’s harper, “that dwellyth in Chestre.” When he became Lord
Howard he retained in his service, Nicholas Stapylton, William Lyndsey, and
“little Richard,” as singers, besides “Thomas, the harperd,” (whom he provided
with a “lyard,” or grey “gown”), and children of the chapel, who were successively
four, five, and six in number at different dates. Mr. Payne Collier, who
edited his Household Book from 1481 to \pagebreak
 1485 for the Roxburghe Club, remarks 
%===============================================================================
%45
on “the great variety of entries in connection with music and musical performers,”
as forming “a prominent feature” of the hook. “Not only were the musicians
attached to noblemen, or to private individuals, liberally rewarded, but also those
who were attached to particular towns, and who seem to have been generally
required to perform before Lord Howard on his various journies. On the 14th of
October, 1841, he entered into an agreement with William Wastell, harper of
London, that he should teach the son of John Colet, of Colchester, harper, for
a year, in order, probably, to render him competent afterwards to fill the post of
one of the family musicians.”

\renewcommand\rectoheader{edward iv.} %045


Here also a part of the stipulation was that, at the end of the year, Lord
Howard should give Wastell a \textit{gown}, which seems to have been the distinguishing
feature of a harper’s dress. In Laneham’s letter from Kenilworth (1575),
describing the “device of an \textit{ancient} minstrel and his song,” which was to have
been proffered for the amusement of queen Elizabeth, this “Squire minstrel, of
Middlesex, who travelled the country this summer season, unto worshipful men’s
houses,” is represented as a harper with a long gown of Kendal green, gathered
at the neck with a narrow gorget, and fastened before with a white clasp; his
gown having long sleeves down to mid-leg, but slit from the shoulders to the
hand, and lined with white. His harp was to be “in good grace dependent before
him,” and his “wrest,” or tuning-key, “tied to a green lace, and hanging by.”
He wore a red Cadiz girdle, and the corner of his handkerchief, edged with blue
lace hung from his bosom. Under the gorget of his gown hung a chain, “resplendent
upon his breast, of the ancient arms of Islington.” The acts of king
Arthur were the subject of his song.

The Romances which still remained popular [1480] are mentioned by William
of Nassyngton [in a MS. which Warton saw in the library of Lincoln Cathedral],
who gives his readers fair notice that \textit{he} does not intend to amuse them.


\settowidth{\versewidth}{“I warne you first at the begynnynge}

\begin{dcverse}
“I warne you first at the begynnynge\\
That I will make no vayne carpynge,\\
Of dedes of armes, ne of amours,\\
As does Mynstrellis and Gestours,\\
That maketh carpynge in many a place\\
Of \textsc{Octaviane} and \textsc{Isenbrace},

And of many other \textit{Gestes},\\
As namely, when they come to festes;\\
Ne of the lyf of \textsc{Bevys of Hamptoune},\\
That was a Knyght of grete renowne;\\
Ne of \textsc{Syr Gye of Warwyke}, \&c.\\
\hfill\textit{Warton}, vol. iv., p.~368.
\end{dcverse}


The invention of printing, coupled with the increased cultivation of poetry and
music by men of genius and learning, accelerated the downfall of the Minstrels.
They could not long withstand the superior standard of excellence in the sister
arts, on the one hand, and the competition of the ballad-singer (who sang without
asking remuneration, and sold his songs for a penny) on the other. In little more
than fifty years from this time they seem to have fallen into utter contempt. We
have a melancholy picture of their condition, in the person of Richard Sheale,
which it is impossible to read without sympathy, if we consider that to him we
are indebted for the preservation of the celebrated heroic ballad of \textit{Chevy Chace},\pagebreak
at which Sir Philip Sidney’s heart was 
 wont to beat, “as at the sound of a  
%===============================================================================
%46
trumpet;”\myfootnote{ %
“I never heard the old song of Percy and Douglas, that 
I found not my heart moved more than with a trumpet: and
yet it is sung but by some blind crowder, with no rougher
voice than rude style; which being so evil aparelled
in the dust and cobweb of that uncivil age, what would it
work, trimmed in the gorgeous eloquence of Pindare!”—
\textit{Sir Philip Sidney’s Defence of Poetry}.
} %end footnote
and of which Ben Jonson declared he would rather have been the 
author, than of all he had ever written. This luckless Minstrel had been robbed
on Dunsmore Heath, and, shame to tell, he was unable to persuade the public
that a son of the Muses had ever been possessed of sixty pounds, which he
averred he had lost on the occasion. The account he gives of the effect upon his
spirits is melancholy, and yet ridiculous enough. [As the preservation of the
old spelling is no longer essential to the rhyme or metre, I venture to give it in
modern orthography.]

\settowidth{\versewidth}{\textit{From the “Chant of Richard Sheale,”—British Bibliographer},}
\begin{scverse}
\vleftofline{“}After my robbery my memory was so decay’d\\
That I could neither sing, nor talk, my wits were so dismay’d.\\
My audacity was gone, and all my merry talk,\\
There are some here have seen me as merry as a hawk;\\
But now I am so troubled with fancies in my mind,\\
I cannot play the merry knave, according to my kind.\\
Yet to take thought, I perceive, is not the next way\\
To bring me out of debt,—my creditors to pay.\\
I may well say that I had but evil hap\\
For to lose about threescore pounds at a clap.\\
The loss of my money did not grieve me so sore,\\
But the talk of the people did grieve me much more.\\
Some said I was not robb’d, I was but a lying knave,\\
\textit{It was not possible for a Minstrel so much money to have}.\\
Indeed, to say the truth, it is right well known\\
That I never had so much money of my own,\\
But I had friends in London, whose names I can declare,\\
That at all times would lend me two hundred pounds of ware,\\
And with some again such friendship I found,\\
That they would lend me in money nine or ten pound.\\
The occasion why I came in debt I shall make relation—\\
My wife, indeed, is a silk-woman, by her occupation;\\
In linen cloths, most chiefly, was her greatest trade,\\
And at fairs and markets she sold sale-ware that she made,\\
As shirts, smocks, and partlets, head-clothes, and other things,\\
As silk thread and edgings, skirts, bands, and strings.\\
At Lichfield market, and Atherston, good customers she found,\\
Also at Tamworth, where I dwell, she took many a pound.\\
When I had got my money together, my debts to have paid,\\
This sad mischance on me did fall, that cannot be denay’d; [denied]\\
I thought to have paid all my debts and to have set me clear,\\
And then what evil did ensue, ye shall hereafter hear:\\
Because my carriage should be light I put my money into gold,\\
And without company I rode alone—thus was I foolish bold;\\
\textit{I thought by reason of my harp no man would me suspect,\\
For Minstrels oft with money, they be not much infect."\\
\hfill From the “Chant of Richard Sheale,”—British Bibliographer}, vol. iv., p.~100. 
\end{scverse}
\pagebreak