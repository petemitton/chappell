%===============================================================================

%021
\changefontsize{1.03\defaultfontsize}

\settowidth{\versewidth}{Horn-Child, thou vnderstond}
\begin{dcverse}
Ant his feren deuyse\\
With ous other seruise;\\
Horn-Child, thou vnderstond\\
\textit{Tech him of harpe and of song}.’”

And devise for his fellows\\
With us other service;\\
Horn-Child, thou understand\\
Teach him of harp and of song.”
\end{dcverse}

\noindent In another part of the poem he is introduced playing on his harp. 

\begin{dcverse}
Horn sette him abenche,\\
Is harpe he gan clenche\\
He made Rymenild a lay\\
Ant hue seide weylaway, \&c.\myfootnote{ %a
Warton’s History of English Poetry, vol. i., p.~38, 8vo., 1840.
}

Horn seated himself on a bench,\\
His harp he began to clench;\\
He made Rymenild a lay\\
And he said wellaway! \&c.
\end{dcverse}

In searching into the early history of the music of any country, the first
subject of inquiry should be the nature and character, as well as the peculiarities
of scale, of the musical instruments they possessed. If the musical instruments
in general use had an imperfect scale, the national music would generally, if not
universally, have retained the peculiarities of that scale. Hence the characteristics
of Scottish music, and of some of the tunes of the North of England, which resemble
it. In the following collection many can he pointed out as bagpipe tunes,
such as “Who liveth so merry in all this land, as doth the poor widow that selleth
the sand,” and “By the border’s side as I did pass,” both of which seem to
require the accompaniment of the drone, while others, like “Mall (or Moll)
Sims,” strictly retain the character of harp music. Where, however, the harp
was in general use, the scale would be more perfect than if some other instruments
were employed, and hence the melodics would exhibit fewer peculiarities,
unless, indeed, the harp was tuned to some particular scale, which, judging by
the passage above quoted from Bede, does not seem to have been the case in
England.

About 1250 we have the song, \textit{Sumer is icumen in}, the earliest secular composition, 
in parts, known to exist in any country. Sir John Hawkins supposed that
it could not he earlier than the fifteenth century, because John of Dunstable, to
whom the invention of figurative music has been attributed, died in 1455. But
Dr.~Burney remarks that Dunstable could not have been the inventor of that art,
concerning which several treatises were written before John was born, and shows
that mistake to have originated in a passage from Proportionales Musices, by John
Tinctor, a native of Flanders, and the “most ancient composer and theorist of
that country, whose name is upon record.” It is as follows: “Of which new art,
as I may call it (counterpoint), the fountain and source is said to have been among
the English, of whom Dunstable was the \textit{chief}.”\myfootnote{ %b
“Cujus, ut ita dicam, novæ artis (Contrapunctis), fons
et origo apud Anglos, quorum caput Dunstaple extitit,
fuisse perhibetur.” From Proportionale Musices, dedicated
to Ferdinand, king of Sicily, Jerusalem, and
Hungary (who reigned from 1458 to 1494), by John
Tinctor, Chaplain and Maestro di Capella to that Prince.
} %end footnote 
“Caput,” literally meaning
“head,” had been understood in its secondary sense of “originator or beginner.”

%%%\changefontsize{1.05\defaultfontsize}
Dr.~Burney’s opinion with respect to the age of this canon seems to have been
very unsettled (if indeed he can be said to have formed one at all). He first
presents it as a specimen of the harmony \pagebreak
in our country, “about the fourteenth 
%\end{fixedpage}%021
%===============================================================================
%022
and fifteenth century.” On the same page he tells us that the notes of the 
MS. resemble those of Walter Odington’s Treatise\myfootnote{ %a
The Best summary of the state of music in England,
about 1230, is contained in Walter Odington’s Treatise,
which is fully described in Burney’s History of Music,
vol. ii., p.~155, et seq. Burney considers it the most
complete of all the early treatises, whether written here
or abroad.
} %end footnote
(1230), and seem to be of the
thirteenth or fourteenth century, and he can hardly imagine the canon much
more modern. Then he is “sometimes inclined to imagine” it to have been
the production of the Northumbrians, (who, according to Giraldus Cambrensis,
used a kind of natural symphonious harmony,) \textit{but with additional parts}, and a
second drone-base of later times. By “additional parts” I suppose Burney
to mean \textit{adding to the length} of the tune, and so continuing the canon. Next
in reviewing “the most ancient musical tract that has been preserved in our
vernacular tongue” (by Lyonel Power), he says, this rule (a prohibition of
taking fifths and octaves in succession) seems to have been so much unknown
or disregarded by the composer of the canon, “Sumer is icumen in,” as to
excite a suspicion that it is “much more ancient than has been imagined.”
And finally, “It has been already shown that counterpoint, in the Church,
began by adding parts to plain chant; and in secular music, by harmonizing
old tunes, as florid melody did by variations on these tunes. It was long
before men had the courage to invent new melodies. It is a matter of surprise
that so little plain counterpoint is to be found, and of this little, none
correct, previous to attempts at imitation, fugue, and canon; contrivances to which
there was a very early tendency, in all probability, during times of extemporary
descant, before there was any such thing as written harmony: for we find in the
most ancient music in parts that has come down to us, that fugue and canon had
made considerable progress at the time it was composed. The song, or round,
‘Sumer is icumen in,’ is a very early proof of the cultivation of this art.” He
then proceeds to show how, according to Martini, from the constant habit of
descanting in successive intervals, new melodies would be formed in harmony with
the original, and whence imitations would naturally arise.

\renewcommand\versoheader{manuscripts---thirteenth century.}
\renewcommand\rectoheader{sumer is icumen in.}

Ritson, who knew more of the age of manuscripts than of musical history, is
of opinion that Burney and Hawkins were restrained by fear from giving their
opinion of its date, and says it may be referred to as early a period (\textit{at least}) as
the year 1250. Sir Frederick Madden,\myfootnote{ %b
Keeper of the Manuscripts in the British Museum.
} %end footnote
in a note to the last edition of Warton’s
English Poetry, says: “Ritson justly exclaims against the ignorance of those who
refer the song to the fifteenth century, when the MS. itself is certainly of the
middle of the thirteenth.” Mr. T. Wright, who has devoted his attention
almost exclusively to editing Anglo-Norman, Anglo-Saxon, and early English
manuscripts, says: “The latter part of this manuscript, containing, among others,
the long political song printed in my Pol. Songs, p.~72, was certainly written
during the interval between the battle of Lewes, in May, 1264, and that of
Evesham, in the year following, and most probably immediately after the firstmentioned
event. The earlier part of the MS., 
which contains the music, was evidently written at an earlier \pagebreak period—perhaps by twenty or 
thirty years—and 
%\end{fixedpage}%022
%===============================================================================
%023
the song with its music must therefore be given to the first half of the thirteenth 
century, at latest.” I have thus entered into detail concerning this song
(though all the judges of manuscripts, whom I have been enabled to consult, are
of the same opinion as to its antiquity), because it is not only one of the first
English songs with or without music, but the first example of counterpoint in six
parts, as well as of fugue, catch, and canon; and at least a century, if not two
hundred years, earlier than any composition of the kind produced out of
England.\myfootnote{ %a
The earliest specimen of secular part-music that has
yet been discovered on the Continent, is an old French
song, for three voices, the supposed production of a singer
and poet, by name Adam de la Hale, called Le Boiteux
d'Arras, who was in the service of the Comte de Provence.
The discovery has been recently made and communicated
by M. Fétis, in his Revue Musicale. “It may be placed
about the year~1280, if a dilettante of the discantus of \textit{the
following age} has not experimentalised on the melody left
by De la Hale, as on a tenor or Canto fermo; since the
other songs, in similar notation, are not in counterpoint;
and the manuscript may be assigned to the \textit{fourteenth}
century.” It is given in Kiesewetter’s History of Music.
} %end footnote

%%%\changefontsize{0.98\defaultfontsize}
The antiquity of the words has not been denied, the progress of our language
having been much more studied than our music, but the manuscript deserves much
more attention from musicians than it has yet received.\myfootnote{ %b
The Musical Notation in this MS, (Harl. 978) is
throughout the same. Only two forms of note are used
with occasional ligatures. “Sumer is icumen in” is on the
back of page 9, and just after it is an Antiphon in praise
of Thomas à Becket. At page 12 we have the musical
scale in letters, exactly corresponding with the scale of
Guido, with the ut, re, mi, fa, \&c., but only extending to
two octaves and four notes, without even the “\textit{e e},” said
to have been added by his pupils. At the back of that
page is an explanation of the intervals set to music, to
impress them on the memory by singing, and examples of
the ligatures used in the notation of the manuscript. At
page 8 is a hymn, “Ave gloriosa mater Salvatoris,” with
Latin and Norman French words, in score in three parts,
on fifteen red lines undivided, and with three clefs for the
voices. The remainder of the musical portion of the
manuscript consists of hymns, \&c., in one or two parts.
} %end footnote
It is not in Gregorian
notation, which might have been a bar to all improvement, but very much resembles
that of Walter Odington, in 1230. All the notes are black. It has neither
marks for time, the red note, nor the white open note, all of which were in use in
the following century.

The chief merit of this song is the airy and pastoral correspondence between
the words and music, and I believe its superiority to be owing to its having been
a national song and tune, selected, according to the custom of the time, as a basis
for harmony, and that it is not entirely a scholastic composition. The fact of its
having a natural drone bass would tend rather to confirm this view than otherwise.
The bagpipe, the true parent of the organ, was then in use as a rustic instrument
throughout Europe. The rote, too, which was in somewhat better estimation, had
a drone, like the modern hurdy-gurdy, from the turning of its wheel. When the
canon is sung, the key note may be sustained \textit{throughout}, and it will be in accordance
with the rules of modern harmony. But the foot, or burden, as it stands
in the ancient copy, will produce a very indifferent effect on a modern ear,\myfootnote{ %
We ought, perhaps, to except the lover of Scotch
Reels.
} %end footnote
from its
constantly making fifths and octaves with the voices, although such progressions
were not forbidden by the laws of music in that age. No subject would be more
natural for a pastoral song than the approach of Summer; and, curiously enough,
the late Mr. Bunting noted down an Irish song from tradition, the title of
which he translated “Summer is coming,” and the tune begins in the same way.
That is the air to which Moore adapted the words, “Rich and rare were the gems 
she wore.” Having given a fac-simile of \pagebreak
“Sumer is icumen in,” taken from the 
%\end{fixedpage}%023
%===============================================================================
%024
manuscript, and as it may be seen in score in Burney and Hawkins’ Histories, 
the tune is here printed, harmonized by Mr. Macfarren, as the first of National
English Airs. A few obsolete words have been changed, but the original are
given below.

\renewcommand\versoheader{sumer is icumen in.}
\renewcommand\rectoheader{songs with music---thirteenth century.}
%%%\changefontsize{1.1\defaultfontsize}


\musictitle{Sumer is icumen in.}
\musicinfo{Rather slow and smoothly.}{About 1250.}
\lilypondfile[staffsize=14]{lilypond/024-sumer-is-icumen-in}


\settowidth{\versewidth}{Groweth sed, and bloweth med}
\indentpattern{01013}
\begin{dcverse}
\vin\vin\textsc{Original Words.}

\begin{patverse}
Sumer is icumen\myfootnote{ %a
“\textit{icumen}” come (from the Saxon verb, \textit{cuman}, to
come); so in Robert of Gloucester, \textit{i\/}paied for paid.
} in,\\
Lhude\myfootnote{ %
Lhude, wde, awe, and calve, are all to be pronounced as
of two syllables.
} sing Cuccu,\\
Groweth sed, and bloweth med\\
And springth the wde nu\\
Sing Cuccu!
\end{patverse}

\vin\vin\textsc{Words Modernized.}

\begin{patverse}
Summer is come in,\\
Loud sing Cuckoo!\\
Groweth seed, and bloweth mead\\
And spring’th the wood now\\
Sing Cuckoo. 
\end{patverse}
\end{dcverse}


%\end{fixedpage}%024
\pagebreak