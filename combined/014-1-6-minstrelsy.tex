
\mainmatter

\changefontsize{1.05\defaultfontsize}

%===============================================================================
%
%001

\renewcommand\versoheader{english minstrelsy.}
\renewcommand\rectoheader{gleemen, scalds, bards.}


%001
\chapter{ON ENGLISH MINSTRELSY,\par
SONGS AND BALLADS.}


\section*{CHAPTER I.}


\subsection*{Minstrelsy from the Saxon Period to the Reign of Edward I.}

\textsc{Music and Poetry} are, in every country, so closely connected, during the 
infancy of their cultivation, that it is scarcely possible to speak of the one without
the other. The industry and learning that have been devoted to the subject of
English Minstrelsy, and more especially in relation to its Poetry, by Percy,
Warton, and Ritson, have left an almost exhausted field to their successors.
But, while endeavouring to combine in a compressed form the various curious
and interesting notices that have been collected by their researches, or which
the labours of more recent writers have placed within my reach, I hope I may
not prove altogether unsuccessful in my endeavour to throw a few additional rays
of light upon the subject, when contemplated, chiefly, in a musical point of view.

“The Minstrels,” says Percy, “were the successors of the ancient Bards, who
under different names were admired and revered, from the earliest ages, among
the people of Gaul, Britain, Ireland, and the North; and indeed by almost all
the first inhabitants of Europe, whether of Celtic or Gothic race; but by none
more than by our own Teutonic ancestors, particularly by all the Danish tribes.
Among these, they were distinguished by the name of \textit{Scalds}, a word which
denotes ‘smoothers and polishers of language.’ The origin of their art was
attributed to Odin or Wodin, the father of their Gods; and the professors of it
were held in the highest estimation. Their skill was considered as something
divine; their persons were deemed sacred; their attendance was solicited by kings;
and they were everywhere loaded with honours and rewards\ldots\  As these
honours were paid to Poetry and Song, from the earliest times, in those countries
which our Anglo-Saxon ancestors inhabited before their removal into Britain, we
may reasonably conclude that they would not lay aside all their regard for men
of this sort, immediately on quitting their German forests. At least, so long as
they retained their ancient manners and opinions, they would still hold them in
high estimation. But as the Saxons, soon after their establishment in this
island, were converted to Christianity, in proportion as literature prevailed among 
them, this rude admiration would begin to \pagebreak
abate, and poetry would no longer be a 
%001
%%===============================================================================
%002 
peculiar profession. Thus the poet and the minstrel early with us became two 
persons. Poetry was cultivated by men of letters indiscriminately; and many of
the most popular rhymes were composed amidst the leisure and retirement of
monasteries. But the Minstrels continued a distinct order of men for many ages
after the Norman conquest; and got their livelihood by singing verses to the
harp, principally at the houses of the great. There they were still hospitably
and respectfully received, and retained many of the honours shown to their predecessors, 
the bards and scalds. And though, as their art declined, many of
them only recited the compositions of others, some of them still composed songs
themselves, and all of them could probably invent a few stanzas on occasion.
I have no doubt but most of the old heroic ballads\ldots\  were composed by this
order of men.”



The term Minstrel, however, comprehended eventually not mere\-ly those who
sang to the harp or other instrument, romances and ballads, but also such as
were distinguished by their skill in instrumental music only. Of this abundant
proof will be given in the following pages. Warton says, “As literature, the
certain attendant, as it is the parent, of true religion and civility, gained ground
among the Saxons, poetry no longer remained a separate science, and the profession
of bard seems gradually to have declined among them: I mean the bard
under those appropriated characteristics, and that peculiar appointment, which he
sustained among the Scandinavian pagans. Yet their natural love of verse and
music still so strongly predominated, that in the place of their old Scalders, a new
rank of poets arose, called \textsc{Gleemen}, or Harpers.\myfootnote{ %a
\textsc{Gleemen}, or Harpers. Fabyan, speaking of Blagebride,
an ancient British king, famous for his skill in
poetry and music, calls him “a conynge musicyan, called
of the Britons god of \textit{Gleemen}.” The learned Percy says:
“This word \textit{glee} is derived from the Anglo-Saxon {\saxon ʒliʒʒ}
(gligg), \textit{musica}, music, minstrelsy (Somner). This is,
the common radix, whence arises such a variety of terms
and phrases relating to the minstrel art, as affords the
strongest internal proof that this profession was extremely
common and popular here before the Norman conquest.
\ldots\ The Anglo-Saxon harpers and gleemen were the
immediate successors and imitators of the Scandinavian
Scalds.” We have also the authority of Bede for the
practice of social and domestic singing to the harp, in
the Saxon language, upon this island, at the beginning of
the eighth century.
} 
These probably gave rise to
the order of English Minstrels, who flourished till the sixteenth century.”

Ritson, in his Dissertation on Romance and Minstrelsy (prefixed to his Collection
of Ancient English Metrical Romances), denies the resemblance between
the Scalds and the Minstrels, and attacks Percy with great acrimony for ascribing
with too great liberality, the composition of our ancient heroic songs
and metrical legends, to those by whom they were generally recited. Percy,
in the earlier editions of his Reliques of Ancient Poetry, said: “The Minstrels
seem to have been the genuine successors of the ancient Bards, who united the
arts of poetry and music, and sung verses to the harp, \textit{of their own composing},”
which he afterwards modified into “\textit{composed by themselves or others}.” With this
qualification there appears to be no essential difference between their systems, as
the following quotation from Ritson will show: “That the different professors of
minstrelsy were, in ancient times, distinguished by names appropriated to their
respective pursuits, cannot reasonably be disputed, though it may be difficult to
prove. The \textit{Trouveur}, \textit{Trouverre}, or \textit{Rymour},
 was he who composed \textit{romans}, \pagebreak
%002
%===============================================================================
%003 
\textit{contes}, \textit{fabliaux}, \textit{chansons} and \textit{lais}; and those who confined themselves to the 
composition of \textit{contes} and \textit{fabliaux} obtained the appellation of \textit{contours}, \textit{conteours}, or
\textit{fabliers}. The \textit{Menetrier}, \textit{Menestrel}, or \textit{Minstrel}, was he who accompanied his song
by a musical instrument, both the words and the melody being occasionally furnished
by himself, and occasionally by others.”


Le Grand says: “This profession which misery, libertinism, and the vaga\-bond
life of this sort of people, have much decried, required, however, a multiplicity of
attainments, and of talents, which one would, at this day, have some difficulty to
find reunited, and we have more reason to be astonished at them in those days of
ignorance; for besides all the songs, old and new,—besides the current anecdotes, 
the tales and fabliaux, which they piqued themselves on knowing,—besides
the romances of the time which it behoved them to know and to possess in part, they
could declaim, sing, compose music, play on several instruments, and accompany
them. Frequently even were they authors, and made themselves the pieces
they uttered.”—\textit{Ritson’s Dissertation}, p. clxiii.

The spirit of chivalry which pervades the early metrical romances could not
have been imparted to this country by the Romans. As Warton observes,
“There is no peculiarity which more strongly discriminates the manners of the
Greeks and Romans from those of modern times, than that small degree of attention
and respect with which those nations treated the fair sex, and the inconsiderable
share which they were permitted to take in conversation, and the general
commerce of life. For the truth of this observation, we need only appeal to the
classic writings: from which it appears that their women were devoted to a state
of seclusion and obscurity. One is surprised that barbarians should be greater
masters of complaisance than the most polished people that ever existed. No
sooner was the Roman empire overthrown, and the Goths had overpowered
Europe, than we find the female character assuming an unusual importance and
authority, and distinguished with new privileges, in all the European governments
established by the northern conquerors. Even amidst the confusions of
savage war, and among the almost incredible enormities committed by the Goths
at their invasion of the empire, they forbore to offer any violence to the women.”

That the people of England have in all ages delighted in secular or social
music, can be proved by numerous testimonies. The Scalds and Minstrels were
held in great repute for many ages, and it is but fair to infer that the reverence
shown to them arose from the love and esteem in which their art was held. The
Romans, on their first invasion of this island, found three orders of priesthood
established here from a period long anterior. The first and most influential were
the Druids; the second the Bards, whose business it was to celebrate the praises
of their heroes in \textit{verses and songs}, which they sang to their harps; and the third
were the Eubates, or those who applied themselves to the study of philosophy.



The Northern annals abound with pompous accounts of the honors conferred
on music by princes who were themselves proficients in the art; for music had
become a regal accomplishment, as we find by all the ancient metrical romances 
and heroic narrations,—and to sing to the \pagebreak
harp was the necessary accomplishment 
%===============================================================================
%004
of a perfect prince, or a complete hero. The harp seems to have been, for many 
ages, the favorite instrument of the inhabitants of this island, whether under
British, Saxon, Danish, or Norman kings. Even so early as the first invasion of
Britain by the Saxons, we have an incident which records the use of it, and which
shows that the Minstrel or Bard was well-known among this people; and that their
princes themselves could, upon occasion, assume that character. Colgrin, son of that
Ella who was elected king or leader of the Saxons, in the room of Hengist, was
shut up in York, and closely besieged by Arthur and his Britons. Baldulph,
brother of Colgrin, wanted to gain access to him, and to apprize him of a reinforcement
which was coming from Germany. He had no other way to accomplish
his design, but by assuming the character of a Minstrel. He therefore
shaved his head and beard, and dressing himself in the habit of that profession,
took his harp in his hand. In this disguise he walked up and down the trenches
without suspicion, playing all the while upon his instrument as a harper. By
little and little he advanced near to the walls of the city, and making himself
known to the sentinels, was in the night drawn up by a rope. Rapin places the
incident here related under the year 495. The story of King Alfred entering
and exploring the Danish camp under the disguise of a Minstrel, is related by
Ingulph, Henry of Huntingdon, Speed, William of Malmesbury, and almost all
the best modern historians; but we are also told that before he was twelve years
old, he could repeat a variety of Saxon songs, which he had learned from hearing
them sung by others, who had themselves, perhaps, only acquired them by tradition,
and that his genius was first roused by this species of erudition.

Bale asserts that Alfred’s knowledge of music was perfect; and it is evident
that he was an enthusiast in the art, from his paraphrase of Bede’s description of
the sacred poet Cædmon’s embarrassment when the harp was presented to him in
turn, that he might sing to it, “be hearpan singan;” Bede’s words are simply
“Surgebat a mediâ cænâ, et egressus, ad suum domum repedabat:” but Alfred
adds, that he arose for \textit{shame} (aras he for sceome); implying that it was a disgrace
to be found ignorant of the art.

We may also judge of the Anglo-Saxon love for song, from the course pursued
by St. Aldhelme, Abbot of Malmesbury, who died in 709. Being desirous of
instructing his then semi-barbarous countrymen, he was in the daily habit of
taking his station on the bridges and high roads, as if a Gleeman or Minstrel
by profession, and of enticing them to listen to him, by intermixing more serious
subjects with minstrel ballads.—\textit{Gul. Malms. de Pontificalibus. Lib}. 5. And
in the ancient life of St. Dunstan (whose feat of taking the evil one by the nose
with a pair-of red-hot pincers, was so favorite a sign for inns and taverns) he is
said, not only to have learnt “the vain songs of his nation,” but also “to have
constructed an organ with brass pipes, and filled with air from bellows.”
The Saint was a monk of Glastonbury, and born about 925.



That the harp was the common musical instrument of the Anglo-Saxons, may
also be inferred from the word itself, which is not derived from the British, or 
any other Celtic language, but of genuine \pagebreak
Gothic original, and current among 
%004
%===============================================================================
%005
every branch of that people, viz.:Ang. Sax. \textit{hearpe} and \textit{hearpe}; Iceland, \textit{harpa} 
and \textit{haurpa}; Dan. and Belg. \textit{harpe}; German, \textit{harpffe} and \textit{harpffa}; Gal. \textit{harpe};
Span, \textit{harpa}; Ital. \textit{arpa}. The Welsh, or Cambro-Britons, call their harp \textit{teylin},
a word for which no etymon is to be found in their language. In the Erse its
name is \textit{crwth}. That it was also the favorite musical instrument of the Britons
and other Northern nations in the middle ages, is evident from their laws,
and various passages in their history. By the laws of Wales (Leges Wallicæ), a
harp was one of the three things that were necessary to constitute a gentleman,
or a freeman; and none could pretend to that character who had not one of these
favorite instruments, or could not play upon it. To prevent slaves from pretending
to be gentlemen, it was expressly forbidden to teach, or to permit, them
to play upon the harp; and none but the king, the king’s musicians, and
gentlemen, were allowed to have harps in their possession. A gentleman’s harp
was not liable to be seized for debt; because the want of it would have degraded
him from his rank, and reduced him to that of a slave.

Alfred entered the Danish camp \ad 878; and about sixty years after, a
Danish king made use of the same disguise to explore the camp of our king
Athelstan. With his harp in his hand, and dressed like a minstrel, Aulaff, king
of the Danes, went among the Saxon tents; and taking his stand by the king’s
pavilion, began to play, and was immediately admitted. There he entertained
Athelstan and his lords with his singing and his music, and was at length dismissed
with an honorable reward, though his songs might have disclosed the fact
that he was a Dane. Athelstan was saved from the consequences of this stratagem
by a soldier, who had observed Aulaff bury the money which had been given him,
either from some scruple of honor or superstitious feeling. This occasioned
a discovery.

Now if the Saxons had not been accustomed to have Minstrels of their own,
Alfred’s assuming so new and unusual a character would have excited suspicions
among the Danes. On the other hand, if it had not been customary with the
Saxons to show favor and respect to the Danish Scalds, Aulaff would not have
ventured himself among them, especially on the eve of a battle. From the
uniform procedure of both these kings, we may fairly conclude that the same
mode of entertainment prevailed among both people, and that the Minstrel was
a privileged character with~each.

May it not be further said,—what a devotion to the art of music must have
existed in those rude times, when the vigilance of war was lulled into sleep and
false security, and the enmities of two detesting nations were forgotten for
awhile, in the enjoyment of sweet sounds!

That the Gleeman or Minstrel held a stated and continued office in the court
of our Anglo-Saxon kings, can be proved satisfactorily. We have but to turn to
the Doomsday Book, and find under the head: Glowecesterscire, fol.~162, col.~1.—“Berdic, 
Joculator Regis, habet iii villas,” \&c. That the word Joculator (at
this early period) meant Harper or Minstrel, is sufficiently evident from Geoffrey 
of Monmouth, of whom Dr.~Percy observes \pagebreak
very justly, “that whatever credit is 
%005
%===============================================================================
%006
due to him as a relator of
\textit{facts}, he is certainly as good authority as any for the 
signification of words.”



The musical instruments \textit{principally} in use among the Anglo-Saxons, were the
Harp, the Psaltry, the Fiðele, and a sort of Horn called in Saxon “Pip” or
Pipe. The Harp, however, was the national instrument. In the Anglo-Saxon
Poem of Beowulf it is repeatedly mentioned.

“There was the noise of the harp, the clear song of the poet.”\ldots\ “There
was song and sound altogether, before Healfdene’s Chieftains; the wood of joy
(harp) was touched, the song was often sung.”\ldots\  “The beast of war (warrior)
touched the joy of the harp, the wood of pleasure,” \&c.

The Fiðele (from which our words fiddler and fiddle are derived) was a sort of
viol, played on by a bow. The Psaltry, or Sawtrie, was strung with~wire.\myfootnote{ %a
Representations of Anglo-Saxon harps and pipes will
be found in Harl. MSS. 603, which also contains a
psaltry, in shape like the lyre of Apollo, but with more
strings, and having a concave hack. It agrees with that
which Augustine describes as carried in the hand of the
player, which had a shell or concave piece of wood on it,
that caused the strings to resound, and is much more
elegant in shape than those in Sir John Hawkins’s History,
copied from Kircher’s Musurgia. A representation
of the Fithele will he found in the Cotton Collection,
Tiberius, c. vi., and in Strutt’s Sports and Pastimes.
Both the manuscripts cited are of the tenth century.
} %end footnote

The Normans were a colony from Norway and Denmark, where the Scalds
had arrived at high renown before Rollo’s expedition into France. Many
of those men no doubt accompanied him to the duchy of Normandy, and left
behind them successors in their art; so that when his descendant William
invaded this kingdom, \ad 1066, he and his followers were sure to favor the
establishment of the minstrel profession here, rather than suppress it; indeed,
we read that at the battle of Hastings, there was in William’s army a valiant
warrior, named Taillefer, distinguished no less for the minstrel arts, than for his
courage and intrepidity. This man, who performed the office of Herald-minstrel
(Menestrier huchier), advanced at the head of the army, and with a loud voice
animated his countrymen, singing a war-song of Roland, \ie, “Hrolfr or Rollo,”
says our Anglo-Saxon historian, Sharon Turner;—then rushing among the
thickest of the English, and valiantly fighting, lost his~life.

The success of his ancestor Rollo, was one of the topics of the speech in which
William addressed his army before the battle, to excite in them the emulation of
establishing themselves in England as he had done in Normandy. A “Chanson
de Roland” continued in favor with the French soldiers as late as the battle of
Poictiers, in the time of their king John, for, upon his reproaching one of them with
singing it at a time when there were no Rolands left, he was answered that
Rolands would still be found if they had a Charlemagne at their head. This was
in 1356.

Dr.~Burney conjectured that the song, “L’homme armée,” which was so popular
in the fifteenth century, was \textit{the} Chanson de Roland; but M. Bottée de Toulmon
has quoted the first four lines of “L’homme armée” from the Proportionales
Musices of John Tinctor, and proved it to be only a love-song. He has also
printed the tune, which he extracted from one of the many Masses in which it
was used as a subject to make Descant on.\myfootnote{ %b
Annuaire Historique pour l’année, 1837. Publié par
la Sociéte de l’Histoire de France.
} %end footnote
%006
\pagebreak

