%===============================================================================
\renewcommand\rectoheader{normans.---battle of hastings.}
\changefontsize{1.05\defaultfontsize}



%007
Robert Wace, in the Roman de Rou, says that Taillefer sang with a loud voice 
(chanta à haute voix) the songs of \textit{Charlemagne}, Roland, \&c., and M. de
Toulmon considers the song of Roland to have been a Chanson de Geste, or
metrical romance; and that Taillefer merely \textit{declaimed} parts of such poems, holding
up those heroes as models to the assembled soldiers. The Chanson de Roland,
that was printed in Paris in 1837-8 (edited by M. Michel) from a copy in the
Bodleian Library, is a metrical romance in praise of the French hero, the \textit{Orlando
Innamorato}, and \textit{Furioso} of Boiardo, Berni, and Ariosto, but apparently of no such
antiquity,\myfootnote{ %a
It contains, also, about 4,000 verses; and it seems still
more improbable that so lengthy a composition should
have been generally and popularly known. It is more
likely to have originated in the favor with which an earlier
song was received.
} %end footnote
and it seems improbable that \textit{he} should have been the subject of the
Norman minstrel's song. All metrical romances, however, were originally recited
or chanted with an accompaniment; and Dr.~Crotch has printed a tune in the
third edition of his Specimens of Various Styles of Music, vol. 1, p.~133, as the
“\textsc{Chanson Roland} sung by the Normans as they advanced to the battle of
Hastings, 1066,” which I give as a curiosity, but without vouching for its
authenticity.

\musictitle{Chanson Roland.}

\lilypondfile[staffsize=18]{lilypond/007-chanson-roland}
\smallskip

Dr.~Crotch does not name the source from which he obtained this air, nor
have I been successful in tracing it.\myfootnote{ %b
The Chanson de Roland that has been printed recently, 
edited by Sir Henry Bishop, is a Composition by
the Marquis de Paulmy, taken from Burney's History of
Music, vol ii. p.~276, but Dr.~Burney does not give it as
an ancient song or tune. The tune, indeed, is not even
in imitation of antiquity.
} %end footnote
The story of Taillefer may, however, be
altogether apochryphal, as it is not mentioned by any \textit{contemporary} historian.



The English, according to Fordun, \pagebreak 
spent the night preceding the battle in 
%007
%%===============================================================================
%008
singing and drinking. “Illam noctem Angli totam in cantibus et potibus 
insomnem duxerunt.”—c.~13.

Ingulphus, a contemporary of William the Conqueror, speaks of the popular
ballads of the English in praise of their heroes; and William of Malmesbury,
in the twelfth century, mentions them also. Three parishes in Gloucestershire
were appropriated by William to the support of his minstrel; and although his
Norman followers would incline only to such of their own countrymen as excelled
in the art, and would listen to no other songs but those composed in their own
Norman-French, yet as the great mass of the original inhabitants were not extirpated,
these could only understand their own native Gleemen or Minstrels; and
accordingly, they fostered their compatriot Minstrels with a spirit of emulation
that served to maintain and encourage them and their productions for a considerable
period after the invasion. That they continued devoted to their AngloSaxon
tongue,\myfootnote{ %a
“The dialect of our Alfred, of the ninth century, in his
Saxon translation of Boethius and Bede, is more clear
and intelligible than the vulgar language, \textit{equally ancient},
of any other country in Europe. For I am acquainted
with no other language, which, like our own, can mount
in a regular and intelligible series, from the dialect now in
use to the ninth century: that is, from pure English to
pure Saxon, such as was spoken and written by King
Alfred, unmixed with Latin, Welch, or Norman.”—
\textit{Burney's History of Music}, vol. ii. p.~209.
} %end footnote
notwithstanding the opposition of their tyrannical conquerors, is
sufficiently plain.

“Of this,” says Percy, “we have proof positive in the old metrical romance
of Horn-Child, which, although from the mention of Sarazens, \&c., must have
been written at least after the first crusade in 1096, yet, from its Anglo-Saxon
language, or idiom, can scarcely be dated later than within a century after the
Conquest. This, as appears from its very exordium, was intended to be sung to a
popular audience, whether it was composed by or for a Gleeman, or Minstrel. But
it carries all the internal marks of being the work of such a composer. It appears
of \textit{genuine English growth}; for, after a careful examination, I cannot discover any
allusion to French or Norman customs, manners, composition, or phraseology: no
quotation, ‘as the romance sayeth:’ not a name or local reference, which was
likely to occur to a French rimeur. The proper names are all of northern
extraction. Child Horn is the son of Allof (\ie, Olaf or Olave), king of Sudenne
(I suppose Sweden), by his queen Godylde, or Godylt. Athulf and Fykenyld are
the names of subjects. Eylmer, or Aylmere, is king of Westnesse (a part of
Ireland); Rymenyld is his daughter; as Erminyld is of another king, Thurstan;
whose sons are Athyld and Beryld. Athelbrus is steward of king Aylmer, \&c. \&c.
All these savour only of a northern origin, and the whole piece is exactly such a
performance as one would expect from a Gleeman or Minstrel of the north of
England, who had derived his art and his ideas from his Scaldic predecessors
there.”

Although Ritson disputed the English origin of this romance, Sir Frederick
Madden, in a note to the last edition of Warton’s English Poetry, has proved
Percy to be right, and that the French Romance, Dan Horn (on the same subject
as Child Horn), \textit{is} a translation from the English. In the Prologue to another
Romance, King Atla, it is expressly stated that the stories of Aelof (Allof),
Tristan, and others, had been translated into French from the English. 

%008
\pagebreak