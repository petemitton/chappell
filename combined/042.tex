
%===============================================================================
%042
% \noindent\begin{minipage}{\textwidth}
%\musictitle{Christmas Carol.\textsuperscript{a}}
%\musicinfo{The Burden or Chorus}{About 1460.}
%\vspace{\baselineskip}
%
\changefontsize{1.06\defaultfontsize}

\musictitle{christmas carol.}
\musicinfo{The Burden or Chorus}{About 1460.}

%\fbox{
\begin{picture}(300pt,130pt)(0,0)
\drawline(200,0)(200,120)(360,120,)(360,0)
%\drawline(0,100)(65,100)(65,10)
\end{picture}
%} %end fbox
\vspace{-120pt}


\lilypondfile[staffsize=15]{lilypond/042-christmas-carol}

\footnotetext[1]{\scriptsizer The two bars marked off by a line are added, because
there would not otherwise be music enough for the \textit{Wassail
Song}. They are a mere repetition of the preceding,
and can be omitted at pleasure. The only way in
which the latter could have been sung to the music as
written in the manuscript, would be by omitting the line
“And bring us in good ale;” but, as it is \textit{merely} a repetition,
it \textit{could} be omitted.}

\pagebreak