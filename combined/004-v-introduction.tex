%===============================================================================

\intentionalemptypage

%%===============================================================================
%v
\changefontsize{\defaultfontsize}

\chapter{INTRODUCTION.}
\vspace*{\fill}

\hspace*{\parindent}\textsc{It} is now nearly twenty years since the publication of my collection of 
\textit{National English Airs} (the first of the kind), and about fourteen since the edition
was exhausted. In the interval, I found such numerous notices of music and
ballads in old English books, that nearly every volume supplied some fresh
illustration of my subject. If “Sternhold and Hopkins” was at hand--the
title-page told that the psalms were penned for the “laying apart of all
ungodly songs and ballads,” and the translation furnished a list of musical
instruments in use at the time it was made: if Myles Coverdale’s \textit{Ghostly
Psalms}--in the preface he alludes to the ballads of our courtiers, to the
whistling of our carters and ploughmen, and recommends young women at the
distaff and spinning-wheel to forsake their “\textit{hey, nonny, nonny---hey, trolly, lolly},
and such like fantasies;” thus shewing what were the usual burdens of their
songs. Even in the twelfth century, Abbot Ailred’s, or Ethelred’s, reprehension
of the singers gives so lively a picture of their airs and graces, as to resemble an
exaggerated description of opera-singing at the present day; and, if still receding
in point of date, in the life of St. Aldhelm, or Oldham, we find that, in order to
ingratiate himself with the lower orders, and induce them to listen to serious
subjects, he adopted the expedient of dressing himself like a minstrel, and first
sang to them their popular songs.

If something was to be gleaned from works of this order, how much more from
the comedies and other pictures of English life in the sixteenth and seventeenth
centuries! I resolved, therefore, to defer the re-publication for a few years, and
then found the increase of materials so great, that it became easier to re-write than
to make additions. Hence the change of title to the work.

Since my former publication, also, I have been favoured with access to the
ballads collected by Pepys, the well-known diarist; and the nearly equally celebrated
“Roxburghe Collection” (formed by Robert, Earl of Oxford, and increased
by subsequent possessors) has been added to the library of the British Museum.
These and other advantages, such as the permission to examine and make extracts
from the registers of the Stationers’ Company (through the liberality of the
governing body), have induced me to attempt a chronological arrangement of the
airs. Such an arrangement is necessarily imperfect, on account of the impossibility
of tracing the exact dates of tunes by unknown authors; but in every case
the reader has before him the evidence upon which the classification has been
founded.  

\pagebreak