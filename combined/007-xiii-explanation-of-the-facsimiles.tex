%%===============================================================================
\renewcommand\versoheader{explanation of the facsimiles.}
\renewcommand\rectoheader{explanation of the facsimiles.}
\changefontsize{1.05\defaultfontsize}
%xiii

\chapter{EXPLANATION OF THE FACSIMILES.}

\vfill

\quad Plate 1 (facing the title-page).—“\textsc{Sumer is icumen in},” from one of the Harleian 
Manuscripts in the British Museum, No. 978. It is literally a “six men’s song,”
such as is alluded to in the burlesque romance of The \textit{Turnament of Tottenham},
and, being of the middle of the thirteenth century, is perhaps the greatest musical
curiosity extant. The directions for singing it are in Latin: “Hanc rotam cantare
possunt quatuor socii. A paucioribus autem quam a tribus aut saltem duobus non
debet dici, preter eos qui dicunt pedem. Canitur autem sic. Tacentibus ceteris, unus
inchoat cum hiis qui tenent pedem. Et cum venerit ad primam notam post crucem,
inchoat alius, et sic de ceteris. Singuli vero repausent ad pausaciones scriptas, et non
alibi, spacio unius longæ notæ.” [Four companions can sing this Round. It should
not, however, be sung by less than three, or at least two, besides those who sing the
burden. It is to be sung thus:—One begins with those who sing the burden, the
others remaining silent; but when he arrives at the first note after the cross, another
begins. The rest follow' in the same order. Each singer must pause at the written
pauses for the time of one long note, but not elsewhere.] The directions for singing
the “Pes,” or Burden, are, to the first voice, “Hoc repetit unus quociens opus est,
faciens pausacionem in fine” [One voice repeats this as often as necessary, pausing at
the end]; and, to the second, “Hoc dicit alius, pausans in medio, et non in fine, sed
immediate repetens principium.” [Another sings this, pausing in the middle, and
not at the end, but immediately re-commencing.]

The melody of this Round is printed in modern notation at p.~24, and in the pages
which precede it (21 to 24) the reader will find some account of the manuscript from
which it is taken. It only remains to add that the composition is in what was called
“perfect time,” and therefore every long note must be treated as dotted, unless it is
immediately followed by a short note (here of diamond shape) to fill the time of the
dot. The music is on six lines, and if the lowest line were taken away, the remaining
would be the five now employed in part-music where the C clef is used on the third
line for a counter-tenor voice.

The composition will be seen in score in Hawkins’s and Burney’s Histories of
Music. The Round has been recently sung in public, and gave so much satisfaction,
even to modern hearers, that a repetition was demanded. It is published in a detached
form for four voices.

\bigskip
Plate 2.—“\textsc{Ah, the syghes that come fro’ my heart},” from a manu\-script of
the time of Henry VIII., in the British Museum (MSS. Reg., Append., 58). For
the melody in modern notation, see p.~57. 
%xiii
\pagebreak