\thispagestyle{empty}
%    \changefontsize{9.8}
\begin{center}

%    \vspace*{\baselineskip}

\Large POPULAR MUSIC \titlespace

\small OF THE

\titlespace

\Large OLDEN TIME:

\titlespace

\small A COLLECTION OF

\titlespace

\Large ANCIENT SONGS, BALLADS,

\titlespace

\small AND

\titlespace

\Large DANCE TUNES,

\titlespace

\small ILLUSTRATIVE OF THE

\titlespace

\Large NATIONAL MUSIC OF ENGLAND.

\titlespace

\normalsize WITH SHORT INTRODUCTIONS TO THE DIFFERENT REIGNS,

AND NOTICES OF THE AIRS FROM WRITERS OF THE

SIXTEENTH AND SEVENTEENTH CENTURIES.

\titlespace

\small ALSO

\titlespace

\large A SHORT ACCOUNT OF THE MINSTRELS.

\titlespace

\small BY

\titlespace

\Large W. CHAPPELL, F.S.A.

\titlespace

\normalsize THE WHOLE OF THE AIRS HARMONIZED BY G. A. MACFARREN.

\titlespace

\large VOL. II.

\titlespace

\footnotesize “Prout sunt illi Anglicani concentus suavissimi
quidem, ac elegantes.”

\textit{Thesaurus Harmonicus} \textsc{Laurencini},
\textit{Romani}, 1603.

\titlespace

\small 1855 Edition LONDON:

\textsc{Cramer, Beale \& Chappell, 201 Regent Street}

\titlespace

2021 Edition MILTON KEYNES

\textsc{Catrah Press}

\end{center}

\intentionalemptypage

\setmainfont{Baskerville10Pro}

\vspace*{4\baselineskip} \headingthree{CONTENTS OF VOL. II.}
\thispagestyle{empty}

\noindent\textsc{\hfill page}

\tocstyle{Conjectures as to Robin Hood}{387}

\tocstyle{Ballads relating to the adventures of Robin Hood.}{391
to 400}.

\tocstyle{Puritanism in its effects upon Music and its
accessories, and Introduction to the Commonwealth period.}{401 to
424}

\tocstyle{Songs and ballads of the civil war, and of the time of
Cromwell.}{425-466}

\tocstyle{Introduction to the reign of Charles II.}{467}

\tocstyle{Songs and ballads from Charles II. to William and
Mary.}{491 to 608}

\tocstyle{Remarks on Anglo-Scottish songs.}{609 to 616}

\tocstyle{Specimens of ditto.}{617 to 620}

\tocstyle{Introduction to the reigns of Queen Anne, George I.,
and George II.}{621 to 632}

\tocstyle{Songs and ballads of ditto.}{633 to 726}

\tocstyle{Traditional songs of uncertain date.}{727 to 749}

\tocstyle{Religious Christmas Carols.}{750 to 758}

\tocstyle{Appendix, consisting of additions to the Introductions,
and of further remarks upon the tunes included in both volumes.}{759
to 788}

\tocstyle{Characteristics of National English Airs, and
summary.}{789 to 797}

\vspace{2\baselineskip}
\begin{center}\rule{4em}{0.4pt}\end{center} \vspace*{4\baselineskip}
\vfill
%    \pagebreak


\intentionalemptypage \mainmatter \setcounter{page}{387}
\thispagestyle{empty} 

\setlength\fixedpagewidth{5in}
\begin{fixedpage}% page387

\vspace*{2\baselineskip}

\begin{center} \Large{POPULAR MUSIC}

\medskip

\normalsize\textsc{OF THE OLDEN TIME.} \end{center}


\centerrule



\headingfour{ROBIN HOOD.}



\centerrule



\textsc{Of} all the sources from which the fertile muse of the English
ballad-maker has derived its subjects, no one has proved more
inexhaustible, or more universally acceptable to the hearers, than
the life and adventures of Robin Hood; and it is indeed singular that
an outlaw of so early a time “should continue traditionally popular,
be chanted in ballads, have given rise to numerous proverbs, and
still be ‘familiar in our mouth as household words,’ in the
nineteenth century.”—

\settowidth{\versewidth}{Of Scarlock, George a Green, and Much,
the miller’s son;} \begin{scverse} \vleftofline{“}In this our spacious isle, I
think there is not one\\ But he hath heard some talk of him and
Little John;\\ And, to the end of time, the tales shall ne'er be
done\\ Of Scarlock, George a Green, and Much, the miller’s son;\\ Of
Tuck, the merry friar, which many a sermon made\\ In praise of Robin
Hood, his outlaws, and their trade.”\\ \attribution{Drayton’s
\textit{Polyolbion}, Song 26.} \end{scverse}

The theories, relative to the time in which he lived, vary
greatly. According to Ritson, he was born in the reign of Henry II.,
about the year 1160, and his true name was Robert Fitzooth, “which
vulgar pronunciation easily corrupted into Robin Hood.” M. Thierry
looks upon him as the chief of a band of Saxons resisting their
Norman oppressors. Mr. Wright considers him as a mere creature of the
imagination—a Robin Goodfellow\textsuperscript{a} —“one amongst the
personages of the early mythology of the Teutonic
people.”\textsuperscript{b} A writer in \textit{The Westminster
Review}\textsuperscript{c} believes him to have been one of the
\textit{Exheredati}, adherents of Simon de Montfort, who were reduced
to the greatest extremities after the battle of Evesham. The Rev.
Joseph Hunter,\textsuperscript{d} the last writer on the subject,
adopts the account given of him in the earliest ballads, and has
brought forward much curious historical evidence to confirm that
account. In his view, Robin Hood lived in the reign of Edward II.,
and was in all probability one of the “Contrariantes,” supporters of
the Earl of Lancaster, who was defeated at the battle of
Borough-bridge, in the month of March, 1321-2.

\begin{dcfootnote} a. The idea that Robin Hood is only a
corruption of Robin \textit{o’th’wood} was started by a correspondent of The
\textit{Gentleman's Magazine} for March, 1793.

b. \textit{Essays on the Literature, \&c., of the Middle Ages.}
By Thomas Wright, 2 vols., 8vo., 1850.

c. March,  1840.

d. \textit{Critical and Historical Tracts}, No. 4,“The
Ballad-Hero, Robin Hood.” By the Rev. Joseph Hunter, Vice Pres. Soc.
Ants., 8vo., 1852. \end{dcfootnote}
\end{fixedpage}%387
\pagebreak

\begin{fixedpage}%page388
\versoheader

Neither Mr. Wright nor Mr. Hunter place any reliance upon the
passage so often quoted from the \textit{Scoti-Chronicon}, concerning
Robin Hood. They regard it as part of the addition made to the
genuine Fordun in the fifteenth century. The earliest notice,
therefore, in our literature is contained in Longland’s poem,
\textit{The Vision of Pierce Ploughman}, where one of the characters,
representing Sloth, says:—

\settowidth{\versewidth}{But I kan rymes of Robyn Hode, and
Randolf, Earl of Chester.”} \begin{scverse} “I kan not perfitly my
paternoster as the Preist it singeth,\\ But I kan rymes of Robyn
Hode, and Randolf, Earl of Chester.” \end{scverse}

The date of this poem is between 1355 and 1365, and proves the
popularity of the ballads among the common people, in the reign of
Edward III. “It seems also to prove,” says Mr. Hunter, “that, in that
reign, the outlaw was regarded as an actual person, who had a
veritable existence, just as Randolph, Earl of Chester, was a real
person.”

Three of the ballads of Robin Hood are contained in manuscripts
which cannot be of later date than the fourteenth century. They are
\textit{The Tale of Robin Hood and the Monk}; \textit{Robin Hood and
the Potter}; and \textit{Robin Hood and Gandeleyn}. But, “far above
these in importance, is the poem—for it can hardly he called a
ballad—which was printed by Winkyn de Worde in or about 1495. It is
entitled \textit{The Lytel Geste of Robyn Hood}; and is a kind of
life of him, or rather a small collection of the ballads strung
together, so as to give a continuity to the story, and with a few
stanzas here and there, which appear to be the work of the person
who, in this manner, dealt with such of the ballads as were known to
him.” The language of the ballads thus incorporated is the same as of
the three ballads above cited, that is, of the fourteenth century.
Mr. Hunter takes \textit{The Lytel Geste} as a guide, and, comparing
it with historical evidence, worked out by his own researches, has
produced an account so probable and so confirmatory, as to leave
scarcely a doubt as to its general accuracy.

Many writers, like Grafton, Stow, and Camden, have referred to,
or quoted, Major’s account of Robin Hood, in his history, which was
first published in Paris, in 1521; but, when Major assigns him to the
reign of Richard the First, he writes only from conjecture. His words
are, “Circa hæc tempora, \textit{ut auguror}, Robertas Hudus Anglus
et Parvus Joannes, latrones famatissimi, in nemoribus latuerunt,”
\&c. (\textit{Historia Majoris Britanniæ, per Joannem Majorem}, 1521,
fol.~lv.,~v°.)

We may therefore revert to the history of Robin Hood, as it was
published in 1495 from materials of the preceding century; and,
although derived from ballads, Bayle has truly said, that “a
collection of ballads is not an unprofitable companion to an
historian;” while Selden has gone so far as to say that they are
often truer than history.

Without entering far into detail, I may mention a few of the
points adduced by Mr. Hunter, in corroboration, of the ballad
account, and refer the reader, for the life of Robin Hood, to his
excellent little book.

\textit{The Lytel Geste} lays the scene in the
reign of one of the Edwards, who is distinguished throughout by no
other epithet than that of “Edward our \textit{comely}
\end{fixedpage}%388
\pagebreak


\begin{fixedpage}%page389 
\rectoheader

\noindent king,” and who makes a
progress in Lancashire. Edward I. was never in Lancashire after he
became king, nor Edward III. in the early years of his reign, (to
which only could the ballads refer), and probably never at all. But
Edward II., to whom the term “our comely king,” so often applied,
would certainly be more appropriate than either to his father or his
son, made one progress in Lancashire, and only one; this was in the
autumn of the seventeenth year of his reign, \ad 1323.

The ballad represents the king at this time as especially intent
on the state of his forests, which were greatly wasted by the
depredations of such men as Robin Hood; and we have historical
evidence of Edward having then visited several of his forests, and of
his endeavour to reform the existing abuses.

In the ballad we are told that the king pardons Robin Hood, and
takes him into his service; that he remains at court a year and three
months; at which time, his money being nearly exhausted, and his men
having left him, except Little John and Scathelock, he becomes moody
and melancholy, and resolves to leave the court. He obtains
permission from the king for a short time, under the plea of making a
pilgrimage to a chapel he had dedicated to Mary Magdalene in
Barnsdale; he returns to the forest and there passes the remainder of
his life.

The date of the king’s progress to Lancashire being the autumn of
1323, would fix the period of Robin’s reception into his service a
little before the Christmas of that year; and in the “Jornal de la
Chambre,” from the 16th April to the 7th of July, 1324, Mr. Hunter
finds, for the first time, the name of “Robyn Hode” in the list of
persons who received wages as “vadlets” or porters of the chamber.
The entry is a payment to nineteen persons, whose names are
specified, from the 24th of March, at the rate of 3d. per day. In the
account which immediately precedes this, the names of those receiving
payment are not specified, and that of Robin Hood has not been
observed \textit{in any document bearing an earlier date}, and the
\textit{last} payment to him is on the 22nd of November, in the
following year.

Further, the Court Rolls of the Manor of Wakefield, of the ninth
year of Edward II., shew that, \textit{before} the Earl of
Lancaster’s rebellion, there was a \textit{Robertus} Hood (familiarly
\textit{Robin} Hood), a person of some consideration, living at or
near Wakefield, which is at no great distance from Barnsdale, and
some of the family continued there till 1407.

\centerrule

The three principal reasons for the excessive popularity of Robin
Hood were, firstly, his free, manly, warm-hearted, and merry
character—his protection of the oppressed, and hatred of all
oppressors, whether clerical or lay; secondly, the encouragement
given to archery, which kept his name alive among the people; and,
thirdly, the incorporation of characters representing Robin Hood and
his companions with the May-day games of the people.

On the first point Grafton says, “And one thing
was much commended in him, that he would suffer no woman to be
oppressed or otherwise abused. The poorer sort of people he favoured,
and would in no wise suffer their goods to 
\end{fixedpage} %389
\pagebreak
\begin{fixedpage}%page390 
\versoheader

\noindent 
be touched or spoiled, but
relieved and aided them with such goods as he gat from the rich,
which he spared not, namely, the rich priests, fat abbots, and the
houses of rich carles: and although his theft and rapine was to be
contemned, yet the aforesaid author [Major] praiseth him and saith,
that among the number of thieves he was worthy the name of the most
gentle thief.” (\textit{Chronicle}, p. 84.) As to the zeal with which
Robin Hood’s day was kept, Bishop Latimer complains, in his sixth
sermon before King Edward VI., that having sent overnight to a town,
that he would preach there in the morning, when he arrived he found
the church door locked, and after waiting half an hour and more for
the keys, one of the parish came to him, and said, “Sir, this a busy
day with us, we cannot hear you; it is Robin Hood’s day:” and he was
obliged to give place to Robin Hood.

Although there are so many songs about Robin Hood, I have found
but few tunes peculiarly appropriated to them. Many of the ballads
were sung to one air; and some to airs which have already been
printed in this collection under other names.

Dr. Rimbault, in his Musical Illustrations of Robin Hood,
appended to Mr. Gutch’s edition of the ballads, has printed the air
of \textit{The Bailiff's Daughter} (ante p.~203), as one of the tunes
to which “Robin Hood and the Pinder of Wakefield” was sung. His
“Robin Hood and Queen Katherine” is the tune of \textit{The Three
Ravens} (ante p. 59). “Robin Hood rescuing the Widow’s Son” is
another version of \textit{Lord Thomas and Fair Ellinor} (ante p.
145). “Robin Hood and Allan-a-Dale” is the first half of
\textit{Drive the cold winter away} (ante p. 193). “Robin Hood and
the Duke of Lancaster” (a satire upon Sir Robert Walpole) is to the
tune of \textit{The Abbot of Canterbury} (p. 350).

When Ophelia sings the line, “For bonny sweet Robin is all my
joy,” she probably quotes from a ballad of Robin Hood, now lost;
because the tune in one part of William Ballet’s Lute Book is
entitled \textit{Robin Hood is to the greenwood gone}, and in another
part, \textit{Bonny sweet Robin}. This has already been printed among
Ophelia’s songs (ante p. 233.)

The ballad of \textit{The Friar in the Well}, of which I have
found the tune, but not the original words (ante p. 273), was, in all
probability, a tale of Robin Hood’s fat friar. Anthony Munday, in his
play, \textit{The Downfall of Robert, Earl of Huntington}, refers to
it as one of the merry jests that had formed the subject of some
previous play about Robin Hood. At the end of act iv., where Little
John expresses his doubts as to the king’s approval, because the play
contains no “jests of Robin Hood; no merry morrices of Friar Tuck,”
\&c., the friar, personating the author, answers—

\settowidth{\versewidth}{How Greenleaf robb’d the shrieve of
Nottingham,} \begin{dcverse} “I promised him a play of Robin Hood,\\
His honourable life in merry Sherwood.\\ His majesty himself survey’d
the plot,\\ And bade me boldly write it, it was good.\\ For merry
jests they have been shewn before,

As how the friar fell into the well,\\ For love of Jenny, that
fair bonny belle;\\ How Greenleaf robb’d the shrieve of Nottingham,\\
And other mirthful matter full of game.\\ Our play expresses noble
Robert’s\\ \vin\vin\vin wrong.” \end{dcverse}

“How Greenleaf robb’d the sheriff of Nottingham,” is told in the
\textit{Lytel Geste of Robin Hood}, where Little John assumes the
name. 
\end{fixedpage}%390

\pagebreak

\begin{fixedpage}%page391
\rectoheader

\musictitle{Robin Hood And Arthur-A-Bland.}

Although a greater number of the Robin Hood ballads were probably
sung to this tune than to any other, I have not found earlier
authority for it than the ballad-operas which were published from
1728 to 1750. It does not appear in \textit{The Dancing Master},
being unfitted for dancing by its peculiar metre.

In \textit{The Jovial Crew}, 1731, the following song is adapted
to the tune:—

\settowidth{\versewidth}{With a hey down, down, and a down,}
\begin{dcverse} \indentpattern{01001101100110} \begin{patverse} \vin
\vleftofline{“}In Nottinghamshire\\ Let them boast of their beer\\
With a hey down, down, and a down,\\ I’ll sing in the praise of good
sack;\\ Old sack and old sherry\\ Will make your heart merry\\
Without e’er a rag to your back. \end{patverse}

\begin{patverse} Then cast away care,\\ Bid adieu to despair,\\
With a down, down, down, and a down,\\ Like fools, our own sorrows we
make,\\ In spite of dull thinking,\\ While sack we are drinking,\\
Our hearts are too easy to ache.’’ \end{patverse} \end{dcverse}

From the burden, the tune is sometimes entitled \textit{Hey down,
a down}; it is also referred to under the names of
\textit{Arthur-a-Bland}, \textit{Robin Hood}, \textit{Robin Hood
revived}, \textit{Robin Hood and the Stranger}, \&c.

Among the Robin Hood ballads sung to it, besides those which the
above names indicate, are “Robin Hood and the Beggar,” “Robin Hood
and the four Beggars,” “Robin Hood and the Bishop” (not the Bishop of
\textit{Hereford}), “Robin Hood’s Chase,” “Robin Hood and Little
John,” “Robin Hood and the Butcher,” “Robin Hood and the Ranger,” and
“Robin Hood and Maid Marian.”

Among the King’s Pamphlets (Brit. Mus., vol. xv., fol.) is one to
this air, dated Jan. 17, 1659, “To the tune of \textit{Robin Hood}”
It is entitled “The Gang: or the nine worthies and champions,
Lambert,” \&c., and is a political ballad on the nine leading members
of the Committee of Safety, who were deprived of their commissions
and ordered away from London by the Rump Parliament, after the
depression of Lambert’s party, and their own return to power.
(Reprinted in \textit{Political Ballads}, edited by Mr. Wright, for
the Percy Society, p. 188.) It commences thus:—

\settowidth{\versewidth}{Johnnie Lambert was first, a dapper
squire,} \indentpattern{001221} \begin{dcverse} \begin{patverse} “It
was at the hirth of a winter’s morn,\\ With a hey down, down, a-down,
down,\\ Before the crow had hist,\\ That nine, heroes in scorn,\\ Of
a Parliament forlorn,\\ Walk’d out with sword in fist. \end{patverse}

\indentpattern{031221} \begin{patverse} Johnnie Lambert was
first, a dapper squire,\\ With a hey down, \&c.,\\ A mickler man of
might\\ Was ne’er in Yorkshire,\\ And he did conspire\\ With Vane Sir
Harry, a knight,” \&c. \end{patverse} \end{dcverse}

Pepys says in his Diary, on the 9th of January, 1659, “I heard
Sir H. Vane was this day voted out of the House, and to sit no more
there,” \&c.

The black-letter copy of the ballad of “Robin Hood and
Arthur-a-Bland,” in the Collection of Anthony à Wood, is entitled
“Robin Hood and the Tanner; or, Robin Hood met with his match: A
merry and pleasant song, relating the gallant and fierce combat
fought between Arthur Bland, a tanner of Nottingham, and Robin Hood,
the greatest and most noblest archer in England. Tune is
\textit{Robin Hood and the Stranger}.” As it consists of thirty-seven
stanzas, it is too long to reprint. I therefore refer the reader to
Ritson’s \textit{Robin Hood}, ii. 31; to Evans’ Old Ballads, ii. 113;
or any other collection of songs of this celebrated outlaw.
\end{fixedpage} %391
\pagebreak

\begin{fixedpage}%page392
\versoheader

\musicinfo{Jovially.}{}

\includemusic{chappellV2001.pdf}

\musictitle{Robin Hood And The Curtal Friar.}

This chant was found by Dr. Rimbault, written in a contemporary
hand, on the fly-leaf of a copy of \textit{Parthenia}, which was
printed in 1611. The copies of the ballad in Anthony à Wood’s and in
the Pepys Collections (vol. i., No. 37) are entitled “The famous
battle between Robin Hood and the curtall Fryer. To a \textit{new
Northern tune}.”

The ballad of “The noble Fisherman; or, Robin Hood’s preferment,”
is directed to be sung to the tune of \textit{In Summer time}, with
which line this ballad begins; and perhaps both derive the name from
the ballad of “King Edward the Fourth and the Tanner of Tamworth,”
which commences in a similar way. The last was entered on the books
of the Stationers’ Company, to William Griffith, in 1564--5. Percy
reprints from a copy in the Bodleian Library, dated 1596, and the
\textit{tune} is mentioned in “Noctes Templariæ,” written in the year
1599 (Harl. MSS.):—“This night Stradilax, in great pomp, miscalled
himself a Lord\dots Poet Natazonius saluted him to the tune of
\textit{The Tanner and the King}.” The ballad begins thus:—

\settowidth{\versewidth}{In summer time, when leaves grow
greene,} \begin{scverse} \begin{altverse} \vleftofline{“}In summer
time, when leaves grow greene,\\ And blossoms bedecke the tree,\\
King Edward wolde a hunting ryde,\\ Some pastime for to see.”
\end{altverse} \end{scverse} Another copy will be found in the
Roxburghe Collection, i. 176. \end{fixedpage} %392
\pagebreak

\begin{fixedpage}%page393
\rectoheader

In the Pepys Collection, i. 463, there' is a ballad to the tune
of \textit{In Summer time}, but in quite a different metre, and
therefore to another tune. It is “The Rimer’s new Trimming. To the
tune of \textit{In Sommer time};” beginning—

\settowidth{\versewidth}{Being minded with mirth, until his turn
came,} \begin{scverse} \vleftofline{“}A rimer of late in a barber’s
shop\\ Sate by for a trimming to take his lot,\\ Being minded with
mirth, until his turn came,\\ To drive away time he thus began;”
\end{scverse} in stanzas of four lines, and “imprinted at London by
T. Langley.”

The ballad of “Robin Hood and the curtal Friar” is reprinted in
Ritson’s \textit{Robin Hood}, ii. 59; in Evans’ \textit{Old Ballads},
ii. 152; \&c.

Douce explains “curtal” to mean “curtailed,” or Franciscan friar;
because, conformably to the injunction of their founder, they wore
short habits. He quotes Staveley’s \textit{Romish Horseleech} to
prove that Franciscans were so called. \textit{Illustrations of
Shakspeare}, i. 60, 8vo., 1807.

\musicinfo{Jovially.}{}

\includemusic{chappellV2002.pdf}

\musictitle{Robin Hood and the Pinder of Wakefield.}

This ballad was entered at the Stationers’ Hall to Mr. John
Wallye and Mrs. Toye, in the first year of the registers, 1557--8. It
was so popular as to be twice alluded to by Shakespeare, in his
\textit{Henry IV}., Part II., act v., sc. 3; and in \textit{The Merry
Wives of Windsor}, act i., sc. 1. Also in Beaumont and Fletcher’s
\textit{Philaster}, act v., sc. 4; and quoted in Munday’s
\textit{Downfall of Robert, Rari of Huntington}, and Munday and
Chettle’s \textit{Death of Robert, Earl of Huntington}; both printed
in 1601.

\begin{lastpar}It is sometimes quoted as “Robin Hood, Scarlet,
and John;” sometimes as “The Pinder of Wakefield” (a “pinder” being
the pen or pound-keeper for impounding stray cattle); and the tune
occasionally entitled \textit{Wakefield on a green}, from the ditty.
Two copies are to be found, under that name, among the lute
manuscripts (said to be Dowland’s) in the Public Library, Cambridge
(D. d. ii. 11, and D. d. iii. 18); a third is contained in a
manuscript \end{lastpar} \end{fixedpage}%393
\pagebreak


\begin{fixedpage}%page394 
\versoheader

\noindent volume of virginal music of
the time of Queen Elizabeth, now in the possession of
\hbox{Dr.~Rimbault.}

The two lute copies seem, like many others in the same
manuscripts, to have no tune in them. They are probably pieces
constructed upon the ground or base of the air, to shew off the
execution of florid passages on the lute. I have constantly found
melody sacrificed in that way, both in lute and virginal music. In
virginal music, the skeleton of the tune can generally be found
running through the piece, sometimes in the base, and sometimes in an
inner part; although the arranger occasionally constructs a wholly
different treble. The tune, in this instance, is to be found in the
base, and in the inner parts; and I am indebted to Dr. Rimbault for
extracting it. Such versions are never very satisfactory, but must be
accepted when no better are to be had.

Drayton, in his \textit{Polyolbion}, Song 28, speaking of Robin
Hood, says:—

\settowidth{\versewidth}{Of Wakefield, George-a-Green, whose
fames so far are blown} \begin{scverse} \vleftofline{“}But of his
merry man, the Pindar of the town\\ Of Wakefield, George-a-Green,
whose fames so far are blown\\ For their so valiant fight, that
\textit{every Freeman's Song}\\ Can tell you of the same.—so be ye
talk’d on long,\\ For ye were merry lads, and those were merry days.”
\end{scverse}



If this be one of the Freemen’s Songs, to which Drayton alludes,
I suppose some of the voices sang the burden.

The ballad is contained in Ritson’s \textit{Robin Hood}, ii, 16;
Evans’ \textit{Old Ballads}, ii.~100; \&c.

\musicinfo{Moderate time.}{}

\includemusic{chappellV2003.pdf}

\end{fixedpage}%394
\pagebreak

\begin{fixedpage}%page395
\rectoheader

\musictitle{ROBIN HOOD AND THE BISHOP OF HEREFORD.}


This, now the most popular of the Robin Hood Ballads, is taken
from a broadside, with music, “printed for Daniel Wright, next the
Sun Tavern in Holborn.”

\settowidth{\versewidth}{“These byshoppes and these
archebyshoppes,} \begin{scverse} \begin{altverse} “These byshoppes
and these archebyshoppes,\\ Ye shall them bete and bynde,”
\end{altverse} \end{scverse} was an injunction carefully impressed by
Robin Hood upon his followers, and many are the tales of tricks he
played upon them, and upon the wealthy abbots. In Ritson’s opinion,
“the pride, avarice, and hypocrisy of the clergy of that age afforded
him ample justification;” but Ritson’s pen was equally dipped in gall
against the clergy of every age, and I verily believe it was the
outlaw’s injunction to his followers, rather than any other motive,
that induced Ritson to make him his hero. Drayton, in his
\textit{Polyolbion}, in the 26th Song, says of Robin Hood—

\settowidth{\versewidth}{From wealthy abbots’ chests, and churls’
abundant store,} \begin{scverse} \vleftofline{“}From wealthy abbots’
chests, and churls’ abundant store,\\ Which oftentimes he took, he
shared among the poor;\\ No lordly Bishop came in lusty Robin’s
way,\\ To him before he went, but for his feast must pay;\\ The widow
in distress he graciously reliev’d,\\ And remedied the wrongs of many
a virgin griev’d.” \end{scverse}

The title of the ballad is, “The Bishop of Hereford’s
entertainment by Robin Hood and Little John, \&c., in merry
Barnsdale.”


\musicinfo{Pompously.}{}

\includemusic{chappellV2004.pdf}

\backskip{2} \settowidth{\versewidth}{The Bishop of Hereford’s to
dine with me to-day,} \begin{dcverse}\footnotesize \begin{altverse}
As it befel in merry Barnsdale,\\ All under the greenwood tree,\\ The
bishop of Hereford was to come by,\\ With all his company.
\end{altverse}

\begin{altverse} Come, kill me a ven’son, said bold Robin Hood,
\\ Come, kill me a good fat deer, \\ The Bishop of Hereford’s to dine
with me to-day,\\ And he shall pay well for his cheer. \end{altverse}

\begin{altverse} We’ll kill a fat ven’son, said bold Robin Hood,
\\ And dress it by the highway side;\\ And we will watch the bishop
narrowly,\\ Lest some other way he should ride. \end{altverse}

\begin{altverse} Robin Hood dress’d himself in shepherd’s attire.
\\ With six of his men also;\\ And, when the Bishop of Hereford came
by, \\ They about the fire did go. \end{altverse}

\begin{altverse} O what is the matter? then said the bishop, \\
Or for whom do you make this ado?\\ Or why do you kill the king’s
venison,\\ When your company is so few? \end{altverse}

\begin{altverse} We are shepherds, said bold Robin Hood, \\ And
we keep sheep all the year,\\ And we are disposed to be merry this
day, \\ And to kill of the king’s fat deer. . \end{altverse}
\end{dcverse}

\end{fixedpage}%end395 
\pagebreak

\begin{fixedpage}%396
\versoheader

\settowidth{\versewidth}{The Bishop of Hereford’s to dine with me
to-day,} \begin{dcverse}\footnotesize \begin{altverse} You are brave
fellows! said the bishop,\\ And the king of your doings shall know:\\
Therefore make haste, and come along with me, \\ For before the king
you shall go. \end{altverse}

\begin{altverse} O pardon, O pardon, said bold Robin Hood,\\ O
pardon, I thee pray;\\ For it becomes not your lordship’s coat \\ To
take so many lives away. \end{altverse}

\begin{altverse} No pardon, no pardon, said the bishop,\\ No
pardon I thee owe;\\ Therefore make haste and come along with me, \\
For before the king you shall go. \end{altverse}

\begin{altverse} Then Robin set his back against a tree,\\ And
his foot against a thorn,\\ And from underneath his shepherd’s coat
\\ He pull’d out a bugle horn. \end{altverse}

\begin{altverse} He put the little end to his mouth.\\ And a loud
blast did he blow,\\ Till threescore and ten of bold Robin’s men \\
Came running all on a row: \end{altverse}

\begin{altverse} All making obeysance to bold Robin Hood; \\
’Twas a comely sight for to see.\\ What is the matter, master, said
Little John, \\ That you blow so hastily? \end{altverse}

\begin{altverse} O here is the Bishop of Hereford,\\ And no
pardon we shall have.\\ Cut off his head, master, said Little John,
\\ And throw him into his grave. \end{altverse}

\begin{altverse} O pardon, O pardon, said the bishop,\\ O pardon,
I thee pray;\\ For if I had known it had been you,\\ I’d have gone
some other way. \end{altverse}

\begin{altverse} No pardon, no pardon, said bold Robin Hood,\\ No
pardon I thee-owe;\\ Therefore make haste and come along with me, \\
For to merry Barnsdale you shall go. \end{altverse}

\begin{altverse} Then Robin he took the bishop by the hand, \\
And led him to merry Barnsdale; \\ He made him to stay and sup with
him that night,\\ And to drink wine, beer, and ale. \end{altverse}

\begin{altverse} Call in a reckoning, said the bishop,\\ For
methinks it grows wondrous high;\\ Lend me your purse, master, said
Little John, \\ And I’ll tell you bye and bye. \end{altverse}

\begin{altverse} Then Little John took the bishop’s cloak,\\ And
spread it upon the ground,\\ And out of the bishop’s portmantua \\ He
told three hundred pound. \end{altverse}

\begin{altverse} Here’s money enough, master, said Little John,
\\ And a comely sight ’tis to see;\\ It makes me in charity with the
bishop,\\ Tho’ he heartily loveth not me. \end{altverse}

\begin{altverse} Robin Hood took the bishop by the hand, \\ And
he caused the music to play;\\ And he made the old bishop to dance in
his boots,\\ And glad he could so get away. \end{altverse}
\end{dcverse}

\musictitle{ROBIN HOOD AND GUY OF GISBORNE.}

This tune is included among the English airs in
\textit{Nederlandtsche Gedenck-Clanck}, 1626; but the English name is
not given. In \textit{The Dancing Master}, from 1650 to 1690, it is
entitled “The chirping of the Lark;” and in Playford’s
\textit{Introduction to the Skill of Music}, “The Lark.”

It is evidently a ballad-tune; but I have not found any ballad
having particular reference to the song of the lark, and of suitable
metre,\textsuperscript{a} except “Robin Hood and Guy of Gisborne.” In
that, the story hangs upon Robin Hood’s being awakened from a dream
by the song of the woodweele, or woodlark;\textsuperscript{b} and I
have therefore coupled them.

\begin{dcfootnote} a. The measure of the ballad \textit{alone}
would not give any indication: it is too common. Any ballads like
“The Child of Elle;” any to the tune of \textit{Chevy Chace}, or to
\textit{Black and yellow} (which I have not succeeded in
indentifying) might be sung to it.

b. “Wodewall,” and “woodweele,” are explained by Jamieson, in his
\textit{Scottish Dictionary}, as synonimous words—“a bird of the
thrush kind; rather, perhaps, a woodlark:” but then, quoting
Sibbald’s \textit{Chronicle of Scottish Poetry}, he adds, “It appears
to be the green woodpecker.” I imagine the first to be the
“wood-\textit{pecker} and the second the wood\textit{lark} In
Adrianus Junius's \textit{Nomenclator}, translated by John Higins,
8vo., 1585, p. 58, he renders “Galgulus, galbula, ales luridus,” by
“the bird that we call a witwal or woodwall;” and according to Ray
(Syn. Av., p. 43) our witwall is a sort of woodpecker. But the “wood
\textit{weele}” of the ballad, and the “wood\textit{wale}” of
Chaucer, are certainly singing-birds. See the following lines from
\textit{The Romaunt of the Rose}, in the folio Chaucer of 1542;—

\settowidth{\versewidth}{In many places were nyghtyngales,}
\begin{fnverse} \vleftofline{“}In many places were nyghtyngales,\\
Alpes, fynches, and wodwales.\\ That in her swete songe delyten \\ In
thylke places as they habyten.” \end{fnverse} \end{dcfootnote}

\end{fixedpage}%end396 
\pagebreak


\begin{fixedpage}%397
\rectoheader

The ballad, which is as long as Chevy Chace, has only hitherto
been discovered in Dr. Percy’s folio manuscript, and the name of the
tune is not given. It is printed in the \textit{Reliques of Ancient
Poetry}; in Ritson’s, and other collections of songs of Robin Hood.

\musicinfo{Cheerfully.}{}

\includemusic{chappellV2005.pdf}

\backskip{2}

\musictitle{Robin Hood, Robin Hood, Said Little John.}

An ancient dance-tune of “Roben Hude” is mentioned in
Wedderburn’s \textit{Complainte of Scotland}, 1549, and again in “The
pityfull Historie of two loving Italians, Gaulfrido and Barnado,”
\&c., “translated out of Italian into Englishe meeter by John Drout.
Imprinted by Henry Binneman, 1570”—

\settowidth{\versewidth}{Then \textit{Robin Hood} was called for,
and \textit{Malkin} ere they went;} \begin{scverse}
\vleftofline{“}The minstrell he was called in some pretty jest to
play, \\ Then \textit{Robin Hood} was called for, and \textit{Malkin}
ere they went;\\ But Barnard ever to the mayde a loving look he
lent,\\ And he would very fayne have daunct with hir,” \&c.
\end{scverse} \begin{lastpar} \noindent This may be the dance in
question. It is arranged in Pammelia (1609) as one of three
\end{lastpar} \end{fixedpage}%end397 
\pagebreak

\begin{fixedpage}%page398 
\versoheader

\noindent country-dances, with words,
to be sung together, and entitled A Round for three country-dances in
one.”


\musicinfo{Gaily.}{}

\includemusic{chappellV2006.pdf}

Another dance of Robin Hood is printed by Dr. Rimbault, from one
of the lute manuscripts at Cambridge, but the same tune bears the
name of \textit{Robin Reddock} in William Ballet’s Lute Book.

\musictitle{The Lady Frances Nevill’s Delight.}

At the end of the edition of \textit{The Dancing Master} printed
in 1665, Playford added some “new and pleasant English tunes for the
treble-violin,” which he afterwards published in a separate form,
with others, under the title of \textit{Apollo’s Banquet for the
Treble Violin}. \textit{The Lady Frances Nevill’s Delight} is to be
found in both collections; in \textit{Musick's Delight on the
Cithren}, 1666; and in sundry manuscripts.


Some copies differ in the second part of the tune, therefore the
two versions are here printed.

The title of The \textit{Lady Frances Nevill’s Delight} gives no
clue to the original words; and, in default of them, Mr. Oxenford has
written the following song of Robin Hood. There is a great similarity
of \textit{character} between this air and that of \textit{The Hunter
in his career} (ante p. 256); and in it the reader will probably find
a similar resemblance in a modern popular song.

\musicinfo{Boldly, and in moderate time.}{}

\includemusic{chappellV2007.pdf}

\end{fixedpage}%end398 
\pagebreak

\begin{fixedpage}%page399
\rectoheader

\includemusic{chappellV2008.pdf}

\settowidth{\versewidth}{Till he made them drop their gear;}

\begin{dcverse} \indentpattern{110110} \begin{patverse} \vin Good
Robin oft gave chace \\ To the monks with sullen face,\\ Till he made
them drop their gear; \\ And their hearts would quake,\\ And their
lusty limbs would shake, \\ If gallant Robin Hood was near.
\end{patverse}

\settowidth{\versewidth}{And we drink to him with three times
three.} \begin{patverse} \vin Like that yeoman brave,\\ We hate a
canting knave,\\ As the very worst of companiè:\\ So, though -bold
Robin’s gone,\\ Still his heart lives on,\\ And we drink to him with
three times three. \end{patverse} \end{dcverse}

\end{fixedpage}%end399 
\pagebreak

\begin{fixedpage}%page400
\versoheader

\settowidth{\versewidth}{And we drink to him with three times
three.}

\begin{dcverse} \indentpattern{110110} \begin{patverse} \vin
Whene’er he filled his can,\\ He would drink to Marian,\\ To that
kind and lovely maid;\\ And he vow'd her smile \\ Would the worst of
cares beguile, \\ While tippling in the greenwood shade;
\end{patverse}

\begin{patverse} \vin As the bowl we pass,\\ Each quaffs it to
his lass,\\ Vowing none to be so fair as she:\\ So, though bold
Robin’s gone,\\ Still his heart lives on,\\ And we drink to him with
three times three. \end{patverse} \end{dcverse}

The following is another second part to the preceding tune:—

\includemusic{chappellV2009.pdf}

\centerrule

\end{fixedpage}%end400 
\pagebreak

\thispagestyle{empty} 
\begin{fixedpage}%page401

\begin{center}

\Large\bigskip\textsc{PURITANISM,}

\medskip

\normalsize\textsc{IN ITS EFFECTS UPON MUSIC AND ITS
ACCESSORIES.} \end{center}

\centerrule

\textsc{Puritanism}, which so long exercised a pernicious
influence upon music in this country, has been traced to a division
and separation between the exiles in Queen Mary’s reign: one party
being for retaining the whole order of service, as set forth in the
reign of Edward VI.; and the other for using only a part. According
to Neal, such of the clergy as refused to subscribe to the Liturgy,
ceremonies, and discipline of the Church of England in 1564, were
then first called Puritans.\textsuperscript{a} “Like the Church of
Geneva,” says Hentzner, “they reject all ceremonies anciently held,
and admit neither organs nor tombs in their places of worship, and
entirely abhor all difference in rank among churchmen, such as
bishops, deans,” \&c.

This, with their objections to the Liturgy, to surplices, copes,
and square caps, was an early stage of that puritanism which, having
once gained the ascendancy, aimed not only at the vices and follies
of the age, but also at the innocent amusements, the harmless
gaieties, and the elegancies, of life.

Queen Elizabeth shewed her desire for the retention of cathedral
service in the first year of her reign. Among the injunctions issued
to the clergy and laity in 1559, the forty-ninth was for the
continuance and maintenance of singing in the
church.\textsuperscript{b} It recites, also, that “because in divers
collegiate and some parish churches, there have been livings
appointed for the maintenance of men and children, to use singing in
the church, by means whereof the laudable science of music hath been
had in estimation, and preserved in knowledge;” therefore the Queen’s
Majesty, not “meaning in any wise the decay of any thing that might
tend to the use and continuance of the said science,” commands that
“no alteration be made of such assignments of living as have been
appointed either to the use of singing or music in the church, but
that the same do remain.”

In her own chapel the service was not only sung with the organ
and voices, but also “with the artificial music of cornets, sackbuts,
\&c., on festival days.”

\begin{dcfootnote} a. According to Neal, “Puritan is a name of
reproach, derived from the Cathari, or Puritani, of the third century
after Christ, but proper enough to express their desires of a more
pure form of worship and discipline in the Church.” He gives no
authority for this derivation, and if, as Hentzner says (1598), they
were first called Puritana by the Jesuit Sandys, it may be doubted
whether he sought in so remote a period for a name. \textit{In The
Travels of Cosmo III., Grand Duke of Tuscany, in Engtand in 1669},
the writer says, “They are called Puritans from considering
themselves pure and free from all sin, leaving out, in the Lord's
prayer, \textit{Et dimitte nobis debita nostra},” “And forgive us our
trespasses.” This is a probable derivation, as some at least, were
ultra-Calvinists. The more vehement Puritans in Elizabeth's reign
were called “Barrowists,” or “Brownists.” They maintained “that it is
not lawful to use the Lord's prayer publicly in the church for a set
form of prayer, and that all set and stinted, prayers are mere
babbling in the sight of the Lord, and not to be used in public
Christian assemblies.'' See the paper drawn up by the Lord Keeper
Puckring, printed by Strype (iv. 202, 8vo., Oxford, 1824). This was
the sect that afterwards prevailed.

b. This injunction is imperfectly printed in Neal's
\textit{History of the Puritans} (i. 152, 8vo,, 1732). It will be
found in Hawkins' \textit{History}, ii. 548, 8vo.; and Burney, iii.
18. \end{dcfootnote} \end{fixedpage}%end401

\pagebreak

\renewcommand\versoheadertext{puritanism, etc.}
\begin{fixedpage}%page402 
\versoheader

In 1582, she revoked all commissions
for penal statutes against concealments (except where suits were
pending); because those commissions had been abused by persons
endeavouring to obtain the property of churches and corporations. In
a letter from Lord Burghley, in 1586, we find that “Hir majestie is
pleased to confirme unto the vicars-choral of the Churche of Hereford
the graunt of their landes, which hath been sowght by divers greedie
persons to have been gotten from them as concealed.” (\textit{Egerton
Papers}, p. 119, 4to., Camden Soc., 1840.) Nevertheless, when she
gave the control of the lands and benefactions intended for singing
men and children, together with other church property, into the hands
of deans and chapters, she did more injury to the cause she desired
to advocate than all that puritanism could effect. Puritanism
triumphed for a time,—but the grasp of deans and chapters has never
been removed.

It was not long before the seed thus sown produced its fruits.
During the Queen’s life, the injunctions she had issued had the
effect of restraining, in some measure, the misappropriation of the
funds devoted to the musical service; but her injunctions died with
her, and the trusts remained.

The misappropriation of these funds was brought before the notice
of James I., in a paper entitled “The Occasions of the decay of Music
in Cathedrall and Colledge Churches at this time.” It is therein
stated that, “whereas, in former tymes of poperye, divers
benifactions have been given to singing men which have faine within
the danger of concealement, and have been againe restored to Deanes
and Canons by newe grauntes by the late Queene, with intencion that
the same should be imploied as before; contrariwise the same is
swallowed up by the Deanes and Canons, because they only are the body
of that incorporation, and the singing men are but inferior members.”
Among the means resorted to, were—Firstly, the giving the actual sum
at which the lands were formerly valued, “so as whereas 20
nobles\textsuperscript{a} a yeare, thirty yeares agone, would at this
day have equalled the worth of twenty markes a yeare in the
maintenance of a man, the same hath lost its value the one halfe, by
reason of the dearness of the tyme present.” Secondly, the places of
singing men were “bestowed upon Taylors, and Shoomakers, and
Tradesmen, which can singe only so muche as hath bene taught them”
[not \textit{read} music]; “and divers of the said places are
bestowed upon their owne men, the most of which can only read in the
church, and serve their master with a trencher at dynner, to the end
that , the founder may pay the Deanes or Prebends man his wages, and
save the hyre of a servant in the master’s purse.” Thirdly, “All
indeavour for teachinge of musick, or the forminge of voices by good
teachers was altogether neglected, as well in men as children;” and
“many that go under the name of choristers, have that same small
maintenance, not for singing, but beinge dumbe choristers, the said
wages being by ill governors bestowed upon them to keepe and
maintaine them for some other instruction, which the founder never
meant; so that in Colledges where there are founded sixteen, twelve,
or ten choristers, scarce four of them can singe a note.”

\begin{dcfootnote} a. The value of a noble was 6\textit{s}.
8\textit{d}., and of a mark accounts in marks and nobles in the
lawyers’ bills of 13\textit{s}. 4d. We have a vestige of the old
method of keeping accounts in marks and nobles in the lawyers bills
of the present day. \end{dcfootnote}

\end{fixedpage}%end402
\pagebreak


\renewcommand\rectoheadertext{in its effects upon music, etc}
\begin{fixedpage}%page403 
\rectoheader

Fourthly, that the number of singers
had already been halved in many places, and the money went into
prebendaries’ purses; that half the lodgings or chambers appointed by
the founders for the singing men, had either been kept by
prebendaries, or let at a yearly rent, they pocketing the money; and
that places were left open a year and a half, under pretence of not
having found competent persons. If, therefore, says the writer, in
cathedrals, where the original number of singers was forty, “now
diminished to twenty,” they be again “lessened to ten, how absurd
will it be that such large and stately buildings should be supplied
by so few, whose voices will only sound but as a little clapper in a
great~bell!”

It ends with a recommendation that the statutes of every
foundation may be examined; for, although deans lived like deans, and
prebendaries and canons lived like prebendaries and canons, “the poor
singing men do live like miserable beggars;” and “if the said lands
be not employed to the true use and intention of the founder, as the
members are sworn to preserve them, the aforesaid oath is violated
and broken, and the abuse needeth reformation.”\textsuperscript{a}

As these abuses were not reformed, it may be inferred that the
deans and chapters were too powerful for the singing men, as they
were in the late ecclesiastical commission, which has perpetuated the
misappropriation of the trusts intended for their benefit by the
founders. Well might the poet exclaim that—\looseness=-1

\settowidth{\versewidth}{Have very often but lean little souls.}
\begin{scverse} \vin\vin\vin\vin“fat Cathedral bodies\\ Have very
often but lean little souls.”\textsuperscript{b} \end{scverse}

As to the Puritans, many of the clergy who were raised to
preferments in Queen Elizabeth’s reign, spent the time of their exile
in such churches as followed the Genevan form of worship, and
returned much disaffected to the rites and ceremonies that were
re-established, and especially to cathedral service. The dislike to
cathedral service was not exclusively acquired in exile, for Thomas
Becon, who was afterwards made Prebendary of Canterbury by Queen
Elizabeth, had printed his \textit{Authorized Reliques of Rome} in
the last year of the reign of Edward VI. In that work he says, “As
for the Divine Service and Common Prayer, it is so chaunted and
minced and mangled of our costly, hired, curious, and nice musitions
(not to instruct the audience withall, nor to stirre up men’s minds
unto devotion, but with a lascivious harmony to tickle their ears),
that it may justly seeme, not to be a noyse made of men, but rather a
bleating of brute beasts; whiles the choristers neigh a descant as it
were a sort of colts; others bellow a tenour as it were a company of
oxen; others bark a counterpoint as it were a kennell of dogs; others
roar out a treble like a sort of bulls; others grunt out base as it
were a number of hogs.”\textsuperscript{c}

\begin{lastpar}In 1572, Thomas Cartwright, a violent Puritan, and
Margaret  Professor of \end{lastpar}

\begin{dcfootnote} a The manuscript from which these extracts are
made is in the British Museum (MSS. Reg. 18, B. 19), bound up with
James the First’s versification of the Psalms in his own handwriting.

b See on this subject, \textit{An Apology for Cathedral Service},
8vo., 1839, \textit{The Choral Service of the United Church of
England and Ireland}, by the Rev. John Jebb, 8vo,, 1843. Miss
Hackett’s three privately-printed books, viz., \textit{Brief Account
of Cathedral and Collegiate Schools, with an abstract of their
Statutes and Endowments}, 4to., 1827; \textit{Registrum Eleemosynariæ
D. Pauli Londinensis}, 4to., 1827; and \textit{A Correspondence and
Evidence respecting the ancient Collegiate School attached to St.
Paul's Cathedral}, 4to., 1832. Also the various publications of
Pring, the organist of Bangor. The case of the Minor Canons of
Canterbury, \&c., \&c. The same tale of violated trusts is told
in~all.

\begin{lastpar} c This passage is quoted by Prynne, in his
\textit{Histriomastix, the Player's Scourge}, 4to., 1633, as well as
an extract already printed here (Note C, p. 18), \end{lastpar}
\end{dcfootnote} 
\end{fixedpage}%end403
\pagebreak
\renewcommand\versoheadertext{puritanism,}

\setlength\fixedpagewidth{5.1in}
\begin{fixedpage}%404
\versoheader

Divinity at Cambridge, attacked cathedral music, and even
the service of the Queen’s own Chapel, in a similar spirit. “In all
their order of service, said he, “there is no edification, according
to the rule of the Apostle, but confusion. They toss the Psalms, in
most places, like tennis-balls.” This is in allusion to the verses
being sung alternately by the choir on the two sides of the dean and
precentor. “As for organs and curious singing, though they be proper
to Popish dens (I mean to cathedral churches), yet some others also
must have them. The Queen’s Chapel, and these churches, which should
be spectacles of Christian reformation, are rather patterns and
precedents to the people of all superstition.”

\begin{dcfootnote}\scriptsizerr
\noindent from John of Salisbury (and which I have
verified by a contemporary manuscript, written for Symon, Abbot of
St. Alban’s, who was installed \ad 1167, and died in 1188. See MSS.
Reg. 13, D. 4, British Museum); also the following, equally curious
for the early history of music in England, from Aelredus, Abbot of
Rivaulx, in Yorkshire, who died a.d. 1166. Prynne prints the original
Latin in a note, and quotes from \textit{Speculum Charitatis}, lib.
ii., cap. 23, \textit{Bibl. Patrum}, vol. xiii., p. 111, “Let me
speake now of those who, under the shew of religion, doe obpalliate
the businesse of pleasure: who usurpe those things for the service of
their vanity, which the ancient Fathers did profitably exercise in
their types of future things. Whence then, I pray, all types and
figures now ceasing, whence hath the Church so many Organs and
Musicall Instruments? To what purpose, I demand, is that terrible
blowing of Belloes, expressing rather the crackes of Thunder, than
the sweetnesse of a voyce? To what purpose serves that contraction
and inflection of the voice? This man singe a base, this a small
meane, another a treble, a fourth divides and cuts asunder, as it
were, certaine middle notes, One while the voyce is strained, anon it
is remitted, now againe it is dashed, and then againe it is inlarged
with a lowder sound. Sometimes, which is a shame to speake, it is
enforced into an horse’s neighings; sometimes, the masculine vigor
being laid aside, it is sharpened into the shrilnesse of a woman’s
voyce: now and then it is writhed, and retorted with a certaine
artificiall circumvolution. Sometimes thou mayst see a man with an
open mouth, not to sing, but, as it were, to breath out his last
gaspe, by shutting in his breath, and by a certaine ridiculous
interception of his voyce, as it were to threaten silence, and now
againe to imitate the agonies of a dying man, or the extasies of such
as suffer. In the mean time, the whole body is stirred up and downe
with certaine histrionical gestures; the lips are wreathed, the eyes
turne round, the shoulders play, and the bending of the fingers doth
answer every note. And this ridiculous dissolution is called
religion; and where these things are most frequently done, it is
proclaimed abroad that God is there more honourably served. In the
meane time, the common people standing by, trembling and astonished,
admire the sound of the Organs, the noyse of the Cymbals and musicall
Instruments, the harmony of the Pipes and Cornets: but yet looke upon
the lascivious gesticulations of the Singers, the meretricious
alternations, interchanges, and infractions of the voyces, not
without derision and laughter; so that a man may thinke that they
came, not to an Oratory, or house of prayer, but to a Theatre; not to
pray, but to gaze about them: neither is that dreadfull majesty
feared before whom they stand, etc. Thus, this Church singing, which
the holy Fathers have ordained that the weake might he stirred up to
piety, is perverted to the use of milawfull pleasure,” etc. The above
passage is so descriptive of the state of church music in England in
the middle of the twelfth century, that I regret not having seen it
in time for insertion in the text, in its proper place. It
corroborates Dr. Rimbault’s account, in his \textit{History of the
Organ}, that at that time organs had but one stop, and that Pipes,
Cornets, and Cymbals (of a small description, tuned in sets) were
used with them. Among the early improvements in the construction,
were the imitations of those instruments by stops. The description of
the singing in four parts, and of the airs and graces, and the
singers, have so modern an appearance, that they might almost have
been written yesterday. Prynne prints the original Latin, from \textit{Bibl.
Patrum}, but to ensure that no interpolations have been made, I have
collated that copy with a manuscript of the \textit{Speculum Charitatis},
written in the thirteenth century, and now in the British Museum. It
is MSS. Reg. 5, \textsc{B}. 9, and belonged to the Monastery of St. Mary, at
Coggeshall, in Essex. The name of the author is variously latinized,
Aelredus, Ailredus, Ealredus, \&c., his English name being Ethelred.
The passage in question, at fol. 191 of the Manuscript, is as
follows:—“De his nunc sermo sit, qui specie religionis negotium
voluptatis obpalliant: qui ea quæ antiqui patres in typis futororum
salubriter exercebant, in usum vanitatia usurpant. Unde quæso,
ceasantibus jam typis et figuris, unde in Ecclesia,tot Organa tot
Cymbala? Ad quid rogo terribilis ille follium flatus, tonitrui potius
fragorem quam vocis exprimens suavitatem? Ad quid illa vocia
contractio et infractio? Hic auccinit, ille discinit alter
supercinit, alter medias quasdam notas dividit et incidit. Nunc vox
stringitur, nunc frangitur, nunc impingitur, nunc diffusiori sonitu
dilatatur. Aliquando, quod pudet dicere, in equinos hinnitus cogitur,
aliquando, virili vigore deposito, in fæminiæ vocis gracilitate
acuitur: nonunquam artificiosa quadam circumvolutione torquetur et
retorquetur. Videas aliquando hominem aperto ore, quasi intercluso
halitu expirare, non cantare, ac ridiculosa quadam vocis
interceptione, quasi minitari silentium, nunc agones morientium, vel
extasim patientium imitari. Interim histrionicis quibusdam gestibus
totum corpus agitatur; torquentur labia, rotant oculi, ludunt humeri
et ad singulas quasque notas digitorum flexus respondet. Et haec
ridiculosa dissolutio vocatur religio; et ubi hæc frequentius
agitantur, ibi Deo honorabilius serviri clamatur. Stans intere
vulgus, sonitum Follium, crepitum Cymbalorum, harmoniam Fistularum,
tremens attonitusque miratur: sed lascivas Cantantium gesticulationes,
meretricias vocum alternationes et infractiones, non sine cachinno,
risuque intuetur; ut eos non ad Oratorium sed ad Theatrum, nec ad
orandum sed ad spectandum æstimes convenisse: nec timetur illa
tremenda majestas cui assistitur,” \&c. “Sic quod sancti Patres
instituerunt ut infirmi excitarentur ad affectum pietatis, in usum
assumitur illecitæ voluptatis.” 
\end{dcfootnote} 
\end{fixedpage}%404
\pagebreak

\setlength\fixedpagewidth{5in}
\begin{fixedpage}%page405
\rectoheader

Even in Convocation, it was proposed “that the use of organs be
abolished,” as early as 1562.

In 1586, while Parliament was sitting, another virulent Puritan
pamphlet was printed and industriously circulated. It was entitled “A
request of all true Christians to the Honourable House of
Parliament.” It prays “that all cathedral churches may he put down,
where the service of God is grievously abused by piping with organs,
singing, ringing, and trowling of Psalms, from one side of the choir
to another, with the squeaking of chanting choristers, disguised (as
are all the rest) in white surplices; some in corner caps and filthy
copes, imitating the fashion and manner of Antichrist the Pope, that
Man of Sin and Child of Perdition, with his other Rabble of
Miscreants and Shavelings.” In this book, deans and canons are
described as “unprofitable drones, or rather caterpillars of the
world,” who “consume yearly, some 2500\textit{l}., some 3000\textit{l}., some more,
some less, wherein no profit cometh to the Church of God.” Cathedrals
“are the dens of idle loitering lubbards; the harbours of
time-serving hypocrites, whose prebends and livings belong, some to
gentlemen, some to boys, and some to serving men and others.” While
such were the invectives of Puritans against church music, even in
Queen Elizabeth’s reign, it could not be expected that secular music,
or any but their own “psalms to hornpipes,” should escape similar
animadversion. Accordingly, Stephen Gosson, in his \textit{Schoole of Abuse}
(1579), comparing the music of his time with that of the ancients,
says, “Homer with his musick cured the sick soldiers in the Grecian
camp, and purged every man’s tent of the plague;” but “thinke you
that those miracles could be wrought with playing of dances, dumps,
pavans, galliards, measures, fancies, or new strains? They never
came where this grew, nor knew what it meant\ldots The Argives
appointed by their laws great punishments for such as placed above
seven strings upon any instrument: Pythagoras commanded that no
musician should go beyond his diapason” [octave]. “Were the Argives
and Pythagoras now alive, and saw how many strings, how many stops,
how many keys, how many clefs, how many moods, flats, sharps, rules,
spaces, notes, and rests; how many quirks and corners; what chopping
and changing, what tossing and turning, what wresting and wringing,
is among our musicians; I verily believe that they would cry out with
the countryman, \textit{Alas! here is fat feeding and lean beasts}; or, as
one said at the shearing of hogs, \textit{Great cry and little wool, Much ado
and small help}.” A passage from this author “against unprofitable
pipers and fiddlers,” and one from Thomas Lovell, against “dauncing
and minstralsye,” have already been quoted under Queen Elizabeth’s
reign (ante pp. 107, 108); but even Thomas Lodge, who replied to
Gosson “in defence of poetry, musick, and stage plays,” would not
defend the merry-making pipers and fiddlers. He says, “I admit not of
those that deprave music: your pipers are as odious to me as
yourself; neither allow I your harping merry beggars;” but “correct
not music when it is praiseworthy, lest your worthless misliking
bewray your madness.”

Philip Stubbes, in his \textit{Anatomy of Abuses}, first printed in 1583,
(and so popular with the Puritans that four editions of it were
printed within twelve years), devotes an entire chapter against
music. He says that from “a certain kind of
\end{fixedpage}%405
\pagebreak

\begin{fixedpage}%406
\versoheader

smooth sweetness in it, it is like unto honey, alluring the
auditory to effeminacy, pusillanimity, and loathsomeness of life\ldots
And right as good edges are not sharpened, but obtused, by being
whetted upon soft stones, so good wits, by hearing of soft music, are
rather dulled than sharpened, and made apt to all wantonness and
sin.” He complains of music “being used in public assemblies and
private conventicles as a directory to \textit{filthy} dancing;” and that
“through the sweet harmony and smooth melody thereof, it estrangeth
the mind, stirreth up lust, womanisheth the mind, and ravisheth the
heart.” Speaking of the minstrels who had licenses from the justices
of the peace, and lived upon their art, he says, “I think all \textit{good}
minstrels, sober and chaste musicians (I mean such as range the
country, riming and singing songs in taverns, ale-houses, inns, and
other public assemblies), may dance the wild morris through a
needle’s eye. There is no ship so balanced with massive matter as
their heads are fraught with all kinds of lascivious songs, filthy
ballads, and scurvy rhimes, serving for every purpose and every
company.”

These specimens of the Puritan spirit with regard to music may
suffice; but the curious will find similar passages in nearly all
their writings. The \textit{arguments} against cathedral music were ably
answered by Hooker in Book v. of his \textit{Ecclesiastical Polity}, and by
others. At the Restoration, the Rev. Joseph Brookbank published a
book in favour of church music, entitled “The well-tuned Organ; or,
an Exercitation: wherein this Question is fully and largely
discussed, whether or no Instrumental and Organical Musick be lawful
in Holy Publick Assemblies.” 4to., 1660. There is little argument in
the Puritan books against church music, they consist almost entirely
of bitter invective or vulgar abuse. Music, however, was not the only
subject of their attacks.

When James I. was making a progress through Lancashire in 1617,
he rebuked the Puritan magistrates for having prohibited and
unlawfully punished the people for using their “lawful recreations
and honest exercises upon Sundays and other holidays, after the
afternoon sermon or service;” and in the following year, he published
a declaration concerning such sports as were lawful. These were,
“dancing, either men or women; archery, for men; leaping, vaulting,
or any other such harmless recreation; May-games, Whitsun-ales,
Morris-dances, and the setting up of Maypoles, and other sports
therewith used, so as the same be had in due and convenient time,
without impediment or neglect of divine service.” Such recreations
were prohibited to “any that, though conform in religion, are not
present in the Church at the service of God, before going to the said
recreations;” and all were to be sharply punished who abused this
liberty by using these exercises before the end of all divine
services for that day; and each parish was to use the said recreation
by itself. The Puritan magistrates had forbidden these sports, under
the plea of taking away abuses; but such amusements had always been
held lawful, and “if,” said he, “these times he taken away from the
meaner sort, who labour hard all the week, they will have no
recreations at all to refresh their spirits; and, in place thereof,
it will set up filthy tipplings and drunkenness, and breed a number
of idle and discontented speeches in their ale-houses.” Also it will
“hinder the conversion of many, whom their priests
\end{fixedpage}%406
\pagebreak

\begin{fixedpage}%407
\rectoheader

will take occasion hereby to vex, persuading them that no honest
mirth or recreation is lawfully tolerable in our religion.” Such
sports as “bear and bullbaiting, and interludes,” were still held to
be unlawful on Sundays.\textsuperscript{a}

A similar “Declaration to his Subjects, concerning lawful sports
to be used,” was published by Charles I, in 1633.

These sports, except, perhaps, archery, leaping, and vaulting,
were condemned by the Puritans, not only as unlawful on Sundays, but
as altogether abominable. I have quoted Philip Stubbes on the
abomination of May-games (ante p. 133), and subjoin an extract from
Prynne’s \textit{Histriomastix}, on dancing.

“Dancing is for the most part attended with many amorous smiles,
wanton compliments, unchaste kisses, scurrilous songs and sonnets,
effeminate music, lust-provoking attire, ridiculous love-pranks; all
which savour only of sensuality, of raging fleshly lusts. Therefore
it is wholly to be abandoned of all good Christians. Dancing serves
no necessary use, no profitable, laudable, or pious end at all: it
issues only from the inbred pravity, vanity, wantonness,
incontinency, pride, profaneness, or madness of men’s depraved
natures. Therefore it must needs be unlawful unto Christians. The way
to heaven is too steep, too narrow, for men to dance in and keep
revel-rout: No way is large or smooth enough for capering roisters,
for jumping, skipping, dancing dames, but that broad, beaten,
pleasant road that leads to hell. The gate of heaven is too narrow
for whole rounds, whole troops, of dancers to march in together: Men
never went as yet by multitudes, much less by morrice-dancing troops,
to heaven: Alas, they scarce go two together; and these few, what are
they? Not dancers, but mourners, whose tune is \textit{Lachrymæ}; whose
music is sighs for sin; who know no other Cinque-pace but this to
heaven; to go mourning all the day long for their iniquities; to
mourn in secret like doves; to chatter like cranes for their own and
others sins.”—(p. 253.)

Another custom to which the Puritans had a real or pretended
aversion was that of kissing. Prynne alludes to it in the above
extract. It was not only customary to salute a partner at the
commencement and end of a dance (and there were many dances in which
there was much more kissing), but also on first meeting a fair friend
in the morning, or on taking leave of her.

“Kiss in the ring” still holds a place among the pastimes of the
lower orders; but, until the Puritans gained the upper hand, the
custom of kissing was universal, and (at least, for two centuries
before) peculiarly English.

Without entering upon the question as to whether it originated,
like the custom of drinking healths, from the introduction of Rowena
to Vortigern, when she “pressed the beaker with her little lips, and
saluted the amorous Vortigern with a little kiss,” it can, at least,
be shewn to have been general in Chaucer’s time. He alludes to the
custom frequently, and in the picture of the friar, in the Sompnour’s
Tale, he touches on the zeal and activity with which the holy father
performed this act of gallantry. As soon as the mistress of the house
enters the
room,--

\settowidth{\versewidth}{And kisseth her sweet, and chirketh as a sparrow}
\begin{scverse}
--“he riseth up full courtisly\\
And her embraceth in his armes narrow,\\
And kisseth her sweet, and chirketh as a sparrow\\
With his lippes.”
\end{scverse}
\begin{dcfootnote}
\textsuperscript{a} A copy of theproclamation.of James I. is in the library of the Society of
Antiquaries. It was also reprinted in 1817,
by G. Smeeton. That of Charles I. is reprinted in Harleian Miscellany, vol. 5, p.
70, 4to.
\end{dcfootnote}
\end{fixedpage}%407
\pagebreak

\begin{fixedpage}%408
\versoheader

Cavendish, in his life of Cardinal Wolsey, gives an account of
going to the castle of M. de Crequi, a French nobleman, “and very
nigh of blood to King Louis XII.,” where, he says, “I being in a fair
great dining chamber, where the table was covered for dinner, I
attended my lady’s coming; and, after she came thither out of her own
chamber, she received me most gently, like one of noble estate,
having a train of twelve gentlewomen. And when she with her train
came all out, she said to me, ‘For as much as ye be an Englishman,
whose custom is in your country to kiss all ladies and gentlewomen
without offence, and \textit{although it le not so here in this realm} (of
France), yet will I be so bold as to kiss you, and so shall all my
maidens.’ By means whereof I kissed my lady, and all her women. Then
went she to her dinner, being as nobly served as I have seen any of
her estate here in England.”—(p. 171, ed. 1827.)

In the same reign, Erasmus writes to a friend, describing the
beauty, the courtesy, and gentleness of the English ladies in glowing
terms, and this custom as one never sufficiently to be praised. He
tells him that if he were to come to England he would never be
satisfied with remaining for ten years, but must wish to live and die
here.\textsuperscript{a}

A Spanish pamphlet in the library of the British Museum (4to.,
dated 1604) gives an account of the ceremonies observed during the
residence of the Duke de Frias (Ambassador Plenipotentiary from the
Spanish Court) in England, on the accession of James I. In that the
writer says, “The Ambassador kissed her Majesty’s hands, craving at
the same time permission to salute the ladies present, a custom of
which the non-observance on such occasions is deeply resented by the
fair sex of this country,” and leave was accordingly given. (Ellis’s
\textit{Letters on English History}, v. iii., s. 2, p. 211.)

Again, when the celebrated Bulstrode Whitelock was at the court
of Christina, Queen of Sweden, as Ambassador from Cromwell, he waited
on her on Mayday, to invite her “to take the air, and some little
collation which he had provided as her humble servant.” Having
obtained her consent, she, with several ladies of her court,
accompanied him; and her Majesty, “both in supper time and
afterwards,” being “full of pleasantness and gaiety of spirits, among
other frolics, commanded him to teach her ladies the English mode of
salutation; which after some pretty defences, their lips obeyed, and
Whitelock most readily.” (\textit{Gent’s. Mag}.,
v. xcii., part i., p. 325.) “From these passages, it is evident
that the custom was as much admired by the ladies of other countries
as it was \textit{peculiar} to this.”

Whytford’s \textit{Pype of Perfection} has been quoted to prove that
objection was taken to the custom of kissing at the time of the
Reformation; but Whytford objected not only to kissing, but also to
every sort of salutation, even to shaking

\begin{dcfootnote}
\textsuperscript{a} “Quanquam si Britanniæ dotes satis pernosses, Fauste, næ tu
alatis pedibus, huc accurreres; et si podagra tua non sineret,
Dædalum te fieri optares. Nam ut e pluribus unum quiddam attingam.
Sunt hic nymphæ divinis vultibus, blandæ, faciles, et quas tu tuis
camænis facile anteponas. Est præterea mos nunquam satis laudatus:
Sive quo venias omnium osculis exciperis; sive discedas aliquo,
osculis demitteris: redis? redduntur suavia: 
venitur ad te? propinantur suavia: disceditur abs te
dividunter hasia: occuritur alicubi? basiatur affatim denique,
quocunque te moveas. Suaviorum plena sunt omnia. Quæ, si tu, Fauste,
gustasses semel quam aint mollicula quam fragrantia, profecto
cuperes non decennium solum, ut Solon fecit, sed ad mortem usque in
Anglia peregrinari.”—Eraami Epistol, Fausto Andrelino, p. 315, edit.
1642.
\end{dcfootnote}
\end{fixedpage}%408
\pagebreak

\begin{fixedpage}%409
\rectoheader

of hands, among \textit{religious} persons. He says, “It becometh not,
therefore, the persones religions to folow \textit{the maner of secular
persones}, that in theyr congresses, or common meetyngs or departyngs,
do use to kisse, take hands, or such other touchings.” (Fol. 213, b,
1532.) John Bunyan gives an amusing account of his scruples on the
subject, in his \textit{Grace Abounding}: “When I have seen good men salute
those women that they have visited, or that have visited them, I have
made my objections against it; and when they have answered that it
was but a piece of civility, I have told them that it was not a
comely sight. Some, indeed, have urged the holy kiss; but then I have
asked them why they made balks? why did they salute the most
handsome, and let the ill-favoured go?” This last question was, no
doubt, rather perplexing to the good men to answer; but here Bunyan
proves that very few were troubled by his scruples.

The abandonment of the custom is said to have been “a part of
that French code of politeness, which Charles II. introduced on his
restoration.” The last traces of its existence are perhaps in one or
two letters from country gentlemen, in \textit{The Spectator}; one of which
occurs in No. 240. The writer relates of himself, that he had
always been in the habit, even in great assemblies, of saluting all
the ladies round; but a town-bred gentleman had lately come into the
neighbourhood, and introduced his “fine reserved airs.” “Whenever,”
says the writer, “he came into a room, he made a profound bow, and
fell back, then recovered with a soft air, and made a bow to the
next, and so on. This is taken for the present fashion; and there is
no young gentlewoman within several miles of this place who has been
kissed ever since his first appearance among us.”

Another custom, to which the Puritans objected violently, was
that of men wearing long hair. Prynne wrote a book called The
\textit{Unlovelinesse of Lovelockes}, in which he quotes a hundred authorities
against it. Of these, one will suffice, from Purchas’s \textit{Pilgrim}: “Long
hair is an ornament to the female sex, a token of subjection, an
ensign of modesty: but modesty grows short in men as their hair grows
long; and a neat, perfumed, frizzled, powdered bush hangs but as a
token of \textit{vini non vendibilis}, of much wine, little wit, of men weary
of manhood, of civility, of Christianity, which would fain imitate
American savages, infidels, barbarians, or women at the least and
best.”—(c. li., p. 490.)

To this, Butler, the author of \textit{Hudibras}, retorted by a song upon
the Round-heads. “Among other affected habits,” says Mrs. Hutchinson
in her \textit{Memoirs of Colonel Hutchinson}, “few of the Puritans, what
degree soever they were of, wore their hair long enough to cover
their ears; and the ministers and many others cut it close round
their heads, with so many little peaks, as was something ridiculous
to behold. From this custom of wearing their hair, that name of
\textit{Roundhead} became the scornful term given to the whole Parliament
party, whose army indeed marched out as if they had been only sent
out till their hair was grown.” \textit{In A full and Complete Answer to A
Tale in a Tub}, 4to., 1642, the author says, “Some say we are so
termed (Roundheads), because we do cut our hair shorter than our
ears, and the reason is because long hair hinders the sound of the
Word from entering into the heart.” The following is Butler’s song:—
\end{fixedpage}%409
\pagebreak

\renewcommand\versoheadertext{cavaliers and roundheads.}
\begin{fixedpage}%410
\versoheader

\settowidth{\versewidth}{‘What creature’s that, with his short hairs,}
\begin{dcverse}
\indentpattern{001001}
\begin{patverse}
“What creature’s that, with his short hairs,\\
His little hand and huge long ears,\\
That this new faith hath founded?\\
The saints themselves were never such, \\
The prelates ne’er rul’d half so much; \\
Oh! such a rogue’s a Roundhead.
\end{patverse}

\begin{patverse}
What’s he that doth the bishops hate, \\
And counts their calling reprobate, \\
’Cause by the Pope propounded;\\
And thinks a zealous cobbler better \\
Than learned Usher in ev’ry letter?\\
Oh! such a rogue’s a Roundhead.

\end{patverse}

\begin{patverse}
What’s he that doth high treason say. \\
As often as his \textit{yea} and \textit{nay},\\
And wish the King confounded; \\
And dares maintain that Mr. Pim \\
Is fitter for a crown than him?\\
Oh! such a rogue’s a Roundhead.
\end{patverse}

\begin{patverse}
What’s he that if he chance to hear \\
A little piece of \textit{Common Prayer}, \\
Doth think his conscience wounded; \\
Will go five miles to preach and pray, \\
And meet a sister by the way?\\
Oh I such a rogue’s a Roundhead.”
\end{patverse}
\end{dcverse}

This is printed in Butler’s Posthumous Works, 1732, p. 105, and a
copy is among Ashmole’s MSS., No. 36, 37. The manuscript contains a
similar song on the Cavaliers, beginning “What monster’s that, that
thinks it good.”

The closely cut crown was the badge of all the lower order of
Puritans. Wood says, “the generality of Puritans had mortified
countenances, puling voices, and eyes commonly (when in discourse)
lifted up, with hands lying on their breasts. They mostly had short
hair, which at this time was commonly called \textit{the Committee cut}.”
(\textit{Fasti Oxon}., ii. 61.) It was not a new practice, for, according to
Aubrey, in 1619, when Milton the poet was ten years of age, “his
schoolmaster was a Puritan in Essex, who cut his hair short.” This
carries it back to the reign of James I. Although Milton was Latin
Secretary to the Commonwealth, he preserved his own “clustering
locks” throughout the rule of the Roundheads. Aubrey, in his
manuscript \textit{Collections for the Life of Milton}, tells us that “he had
a delicate, tuneable voice, and good skill in music.” After dinner it
was his habit to “play on the organ, and either he or his wife sang.
He made his nephews songsters, teaching them to sing from the time
they were with him; and although, towards his latter end, he was
visited with the gout, he would be cheerful, even in his gout fits,
and sing.” (Aubrey MSS., No. 10, Ashm. Mus.) In his \textit{Tractate on
Education}, Milton says, that after athletic exercise, “the interval
of unsweating, and that of a convenient rest before meat, may, both
with profit and delight, be taken up in recreating and composing the
travailed spirits with the solemn and divine harmonies of music,
heard or learned. Either while the skilful organist plies his grave
and fancied descant on lofty fugues, or with artful touches adorns
and graces the well-studied chords of some choice composer; sometimes
the lute or soft organ-stop waiting on elegant voices, either to
religious, martial, or civil ditties; which, if wise men and prophets
be not extremely out, have a great power over dispositions and
manners, to smooth and make them gentle from rustic harshness, and
distempered passions. The like also would not be unexpedient after
meat, to assist and cherish nature in her first concoction; and send
the mind back to study in good tune and satisfaction.” Milton imbibed
his love of music, in all probability, from his father,\textsuperscript{a} who made

\begin{dcfootnote}
\textsuperscript{a} Edward Phillips (nephew of the poet) says Milton’s 
father was disinherited “for embracing, when young, the
Protestant faith, and abjuring the Popish tenets: that 
he followed
the vocation of scrivener for many years, at 
his house in Bread Street, with
success suitable to his industry
and prudent conduct of his affairs. Yet he did
\end{dcfootnote}
\end{fixedpage}%410
\pagebreak

\renewcommand\rectoheadertext{the commonwealth.}

\begin{fixedpage}%411
\rectoheader

it the relaxation of his leisure hours, and was an excellent
amateur composer. In his time, the habit of singing part-music after
meals was general, especially after supper, the hour of which
corresponded with that of our present dinner. Although now more
common in Germany than in England, it is a practice that might be
revived with great advantage, for, while assisting digestion, there
is no time at which music is more thoroughly enjoyable to those who
can take a part.

It was said by A[llan] C[unningham], in the \textit{Penny Magazine} (No.
391, May 6, 1838), that the ballads “were on the side of the
parliament in the struggle with Charles.” I think this can only apply
to the early part of the contest, for after the fall of Archbishop
Laud, I doubt whether any more were written on their side. Laud had
rendered himself extremely unpopular by his intemperate zeal, and by
his rigorous prosecutions of all separatists, in the Star
Chamber—imprisoning some, and cutting off the ears of others.
Moreover, there was a general impression that he was endeavouring to
lead the country back to Popery. It is said of one of the daughters
of William, Earl of Devonshire, that having turned Catholic, she was
questioned by Laud as to the motives of her conversion. She replied
that her principal reason was a dislike to travel in a crowd. The
meaning being obscure, the Archbishop asked her what she meant. “I
perceive,” said she, “your Grace and many others are making haste to
Rome, and therefore, to prevent being crowded, I have gone before
you.” It is an undoubted fact that the Pope sent him a serious offer
of a Cardinal’s hat; indeed, Laud tells us as much in his diary. The
dissolution of the Parliament, in 1640, was generally attributed to
his instigation; and two thousand persons entered St. Paul’s at one
time, exclaiming, “No Bishop! No high Commission!” The most
scurrilous libels were affixed to the walls in every quarter of the
town; ballads, of which he was the subject, were composed and sung in
the streets; and pictures, in which he was exhibited in the most
undignified postures, were publicly displayed. The ale-houses teemed
with songs in which he was held up to derision. When this was told to
the Archbishop, “His lot,” he said, “was not worse than that of
David;” at the same time quoting the 69th Psalm, “They that sat in
the gate spake against me, and I was the song of the drunkards.”

It is reported of Archibald Armstrong, Charles the First’s jester
or fool, that he once asked permission of the King to say grace when
Laud was present; which being granted, he said, “All praise to the
Lord, and little \textit{laud} to the devil.” In one of the many lampoons of
the time, he is styled—

\settowidth{\versewidth}{‘One of Rome’s calves, far better fed than taught.’”}
\begin{scverse}
‘One of Rome’s calves, far better fed than taught.’”
\end{scverse}

There are still many ballads extant concerning Archbishop Laud.
Besides those which are to be found among the King’s Pamphlets in the
British Museum, a collection, partly in print and partly in
manuscript, was a few years ago in the

\begin{dcfootnote}
not so far quit his own generous and ingenious inclinations as to
make himself wholly a slave to the world; for he sometimes found
vacant hours for the study (which he made his recreation) of the
noble science of music, in which he advanced to that perfection,
that, as I have been told, and as I take it, by our author himself,
he composed an \textit{In Nomine} of forty parts, for which he was rewarded
with a gold medal and chain by a \textit{Polish Prince} (Aubrey says, by the
Landgrave of Hesse), to whom he presented it. However, this is a
truth not to be denied, that for several
songs of his composition, after the way of these times (three or
four of which are still to be seen in old Wilby’s \textit{set of Ayres},
besides some compositions of his in Ravenscroft’s \textit{Psalms}), he gained
the reputation of a considerable master in this most charming of all
the liberal sciences.” One of the madrigals in \textit{The Triumphs of
Oriana}, 1601, and several in Sir Christopher Leighton’s \textit{Tears and
Lamentations of a Sorrowful Soute}, were also composed by Milton, who
bore the same Christian name as his celebrated son.
\end{dcfootnote}
\end{fixedpage}%411
\pagebreak

\begin{fixedpage}%412
\versoheader

possession of Mr. Willis, the bookseller, who printed the
following in his \textit{Current Notes} for December, 1852. A copy is also in
MSS. Ashmole 39 and 37.

“A prognostication on W. Laud, late Archbishop of Canterbury,
written \ad., 1641, which accordingly is come to pass.—Sold at the
Black Ball in Cornhill, near the Exchange.” (With a woodcut of an
execution, the body stretched on the scaffold, and the executioner
holding up a bleeding head.)

\settowidth{\versewidth}{Your staff would strike his sceptre down,}
\begin{dcverse}
\indentpattern{002002}
\begin{patverse}
My little lord, methinks ’tis strange\\
That you should suffer such a change\\
In such a little space.\\
You, that so proudly t’other day\\
Did rule the King, and country sway,\\
Must trudge to ’nother place.
\end{patverse}

\begin{patverse}
Remember now from whence you came,\\
And that your grandsires of your name \\
Were dressers of old cloth;\textsuperscript{a}\\
Go, bid the dead men bring their shears,\\
And dress your coat to save your ears,\\
Or pawn your head for both. '
\end{patverse}

\begin{patverse}
The wind shakes cedars that are tall,\\
An haughty mind must have a fall,\\
You are but low I see;\\
And good it had been for you still,\\
If both your body, mind, and will,\\
In equal shape should be.
\end{patverse}

\begin{patverse}
The King, by hearkening to your charms\\
Hugg’d our destruction in his arms,\\
And gates to foes did ope;\\
Your staff would strike his sceptre down,\\
Your mitre would o’ertop the crown,\\
If you should be a Pope.
\end{patverse}

\begin{patverse}
But you that did so firmly stand,\\
To bring in Popery in this land,\\
Have miss’d your hellish aim;\\
Your saints fall down, your angels fly,\\
Your crosses on yourself do lie,\\
Your craft will be your shame.
\end{patverse}

\begin{patverse}
We scorn that Popes with crozier staves, \\
Mitres or keys, should make us slaves;\\
And to their feet to bend:\\
The Pope and his malicious crew.\\
We hope to handle all, like yon.\\
And bring them to an end.
\end{patverse}

\begin{patverse}
The silenc’d clergy, void of fear.\\
In your damnation will have share,\\
And speak their mind at large:\\
Your cheesecake cap and magpie gown,\\
That made such strife in every town,\\
Must now defray your charge.
\end{patverse}

\begin{patverse}
Within this six years, six ears have \\
Been cropt off worthy men and grave,\\
For speaking what was true;\\
But if your subtle head and ears \\
Can satisfy those six of theirs,\\
Expect but what’s your due.
\end{patverse}

\begin{patverse}
Poor people that have felt your rod\\
Yield \textit{Laud} to the devil, praise to God,\\
For freeing them from thrall;\\
Your little “Grace,” for want of grace,\\
Must lose your patriarchal place,\\
And have no grace at all.
\end{patverse}

\begin{patverse}
Your white lawn sleeves that were the wings \\
Whereon you soar’d to lofty things,\\
Must be your fins to swim;\\
Th’ Archbishop ’s \textit{see} by Thames must go,\\
With him unto the Tower below,\\
There to be rackt like him.
\end{patverse}

\begin{patverse}
Your oath cuts deep, your lies hurt sore, \\
Your \textit{canons} made Scot’s cannons roar,\\
But now I hope you’ll find\\
That there are cannons in the Tower\\
Will quickly batter down your power,\\
And sink your haughty mind.
\end{patverse}

\begin{patverse}
The Commonalty have made a vow,\\
No oath, no canons to allow,\\
No bishops’ \textit{Common Prayer};\\
No lazy prelates that shall spend\\
Such great revenues to no end\\
But virtue to impair.
\end{patverse}

\begin{patverse}
Dumb dogs that wallow in such store,\\
That would suffice above a score\\
Pastors of upright will;\\
Now they’ll make all the bishops teach,\\
And you must in the pulpit preach\\
That stands on Tower Hill.
\end{patverse}
\end{dcverse}

\begin{scfootnote}
\textsuperscript{a} Laud's father was a clothier, of Reading.
\end{scfootnote}
\end{fixedpage}%412
\pagebreak

\begin{fixedpage}%413
\rectoheader

\settowidth{\versewidth}{Your staff would strike his sceptre down,}
\begin{dcverse}
\indentpattern{002002}
\begin{patverse}
When the young lads to you did come \\
You knew their meaning by the drum, \\
You had better yielded then;\textsuperscript{a}\\
Your head and body then might have\\
One death, one burial, and one grave\\
By boys,—but two by men.
\end{patverse}

\begin{patverse}
But you that by your judgments clear,\\
Will make five quarters in a year,\\
And hang them on the gates;\\
That head shall stand upon the bridge,\\
When yours shall under traitor’s trudge,\\
And smile on your miss’d fates.
\end{patverse}

\begin{patverse}
The little \textit{Wren} that soar'd so high,\\
Thought on his wings away to fly,\\
Like \textit{Finch}, I know not whither;\\
But now the subtle whirly-wind,\\
\textit{Debauck}, hath left the bird behind,\\
You two must flock together.
\end{patverse}

\begin{patverse}
A bishop’s head, a deputy’s breast,\\
A \textit{Finch’s} tongue, a \textit{Wren} from’s nest,\\
Will set the devil on foot;\\
He's like to have a dainty dish,\\
At once both flesh, and fowl, and fish,\\
And \textit{Duck} and \textit{Lamb} to boot.
\end{patverse}

\begin{patverse}
But this I say; that your lewd life\\
Did fill both Church and State with strife,\\
And trample on the crown;\\
Like a bless’d martyr you will die\\
For Church’s good; she rises high\\
When such as you fall down.
\end{patverse}
\end{dcverse}

Another of the ballads against Laud is named “The Organ’s Echo,
to the tune of The Cathedral Service.” A third, “The Bishop’s last
Good-night:—

\settowidth{\versewidth}{“Where Popery and innovation do begin,}
\begin{scverse}
“Where Popery and innovation do begin,\\
There treason will by degrees come in.”
\end{scverse}

Laud was beheaded in 1644; and in the same year, Sir Edward
Dering brought a bill unto the House of Commons for the abolition of
Episcopacy. In his “Declaration and Petition to the House of
Commons,” printed in that year, he asserted, in the true spirit of
his party, that “one single groan in the spirit is worth the diapason
of all the Church music in the world.”

“Two ordinances of the Lords and Commons assembled in Parliament
for the speedy demolishing of all \textit{Organs}, images, and all matters of
superstitious monuments in all Cathedral and Collegiate or
Parish-Churches and Chapels throughout the kingdom,” were published
on the 9th of May, 1644, but their demolition had been nearly
accomplished two years before; for, as said by a writer of the time,—

\settowidth{\versewidth}{There’s new Church music found instead of those,}
\begin{scverse}
No organ-idols with pure ears agree,\\
Nor anthems—why? nay ask of them, not me;\\
There’s new Church music found instead of those,\\
The women’s sighs tuned to the Preacher’s nose.”
\end{scverse}

The account of their destruction will be found in “Mercurius
Rusticus; or the Country’s Complaint of the barbarous outrages
committed by the Sectaries of this flourishing kingdom;” in Culmer’s
“Cathedral News from Canterbury;” \&c. At Rochester, Sir John Seaton,
“that false traiterous Scot,” coming towards the church and hearing
the organs, started back, and “in the usual blessing of some of his
country, cried \textit{A Devil on those Bagpipes},” At Chichester, in 1642,
the rebels, under the command of Sir William Waller, “brake down the
organs, and dashing the pipes with their pole-axes, scoffingly said,
\textit{Hark! how the organs go};” and Sir Arthur Haslerig, being told where
the church plate was concealed, commanded his servants to break down
the wainscot round the room, and while

\begin{dcfootnote}
\textsuperscript{a} Five thousand London apprentices went to Lambeth 
to take him, but Laud was prepared, and they could not 
effect their purpose. One was secured, a tailor, who was hung for the attempt.
\end{dcfootnote}
\end{fixedpage}%413
\pagebreak

\begin{fixedpage}%414
\versoheader

they were doing it, danced and skipped, crying, “There, boys,
there, hoys, hark! it rattles, it rattles:” upon which, says the
writer, “Pray, mark what musick that is to which it is lawful for a
Puritan to dance.” In Westminster Abbey, “they brake down the organ
and pawned the pipes at several ale-houses for pots of ale. They put
on some of the singing men’s surplices, and in contempt of that
canonical habit, ran up and down the church; he that wore the
surplice being the hare and the rest the hounds.” At Exeter, they
“brake down the organs, and taking two or three hundred pipes with
them, in a most scornful and contemptuous manner, went up and down
the street, piping with them, and meeting some of the choristers of
the church, whose surplices they had stolen before, scoffingly told
them, ‘Boys; we have spoiled your trade, you must go and sing Hot
pudding pyes.’” At Peterborough, under Cromwell, after defacing the
tombs of Queen Catherine and Mary, Queen of Scots, “when their
unhallowed toilings had made them out of wind, they took breath
afresh on two pair of organs, piping with the very same about the
market place lascivious jigs, whilst their comrades danced after
them, some in copes, others with the surplices, and they brake down
the bellows to blow the coals of a bonfire to burn them.” On their
first visit to Canterbury, they slashed the service books, surplices,
\&c., and “began to play the tune of \textit{The Zealous Soldier} on the
organs or case of whistles, which never were in tune since.” But on
this occasion, some ran to the Commander-in-Chief, who called off the
soldiers, “who afterwards sung cathedral prick-song as they rode over
Barham Down towards Dover, with pricking leaves in their hands, and
lighted their tobacco pipes with them; and such pipes and cathedral
music,” in the opinion of Culmer, “did consort well together.” St.
Paul’s Cathedral was turned into horse-quarters for the soldiers of
the Parliament, except the choir, which was separated by a brick wall
from the nave, and converted into a preaching place. The entrance to
it was by a door which had formerly been a window. The Corinthian
portico at the west end was leased out to a man who built in it a
number of small shops, which he let to haberdashers, glovers, and
sempsters or milliners, and this was called \textit{Paul’s Change}.

Charles the First’s love of music is mentioned by Playford, in
his \textit{Introduction to the skill of Musick}, edit. 1760. He says that he
was not “behind any of his predecessors in his skill and love of this
divine art, especially in the service of Almighty God and that he
“often appointed the service and anthems; being, by his knowledge in
musick, a competent judge therein, and much delighted to hear that
excellent service composed by Dr. William Child, called his Sharp
Service. And for instrumental music, none pleased him like those
incomparable Fantasies for one Violin and Base-viol to the Organ,
composed by Mr. Coperario”\textsuperscript{a} (Cooper). In the British Museum (Addit.
MSS., 11,608, fol. 59) is a song the music of which was composed by
Charles I., the poetry by Thomas Carew. It commences:— 

\settowidth{\versewidth}{Mark how the blushful morn, in vain,}
\begin{scverse}
\vleftofline{“}Mark how the blushful morn, in vain,\\
Courts the am’rous marigold.”\textsuperscript{b}
\end{scverse}

“During the prosperous state of the King’s affairs” (says Lord
Orford,

\begin{dcfootnote}
\textsuperscript{a} The only known manuscript of these Fancies by 
Coperario is now in the possession of Dr. Rimbault.

\textsuperscript{b} The manuscript which contains this song is one of the
last that were purchased from Thorpe, the celebrated  bookseller,
for the British Museum. It is an important
manuscript in several respects.
\end{dcfootnote}
\end{fixedpage}%414
\pagebreak

\begin{fixedpage}%415
\rectoheader

Hist. Paint., ii. 147) “the pleasures of the Court were carried
on with much taste and magnificence. Poetry, painting, music, and
architecture, were all called in to make them rational amusements;
and I have no doubt but the celebrated festivals of Louis XIV. were
copied from the shows exhibited at Whitehall, in its time the most
polite court in Europe. Ben Jonson was the laureate; Inigo Jones the
inventor of the decorations; Laniere and Ferabosco” [Dr. Campion, Dr.
Giles, W. and H. Lawes, Simon Ives, Dr. Coleman, \&c.] “composed the
symphonies; the King, the Queen, and the young nobility, danced in
the interludes.”

Oliver Cromwell was also a great lover of music, and “entertained
the most skilful in that science in his pay and family.” Heath
compares him in his love for music to “wicked Saul, who, when the
evil spirit was upon him, thought to lay and still him with those
harmonious charms;” but he adds, that “generally he respected or at
least pretended to love, all ingenious or eximious persons in any
arts, whom he procured to be sent or brought to him.” (\textit{Flagellum}, p.
160, 4th edit., 1669). He engaged John Hingston, a celebrated
musician of the time, who had been in the service of Charles, to
instruct his daughters in music, and gave him à pension of 100\textit{l}. a
year. Hingston gave concerts at his own house, at which Cromwell
would often be present. At one of these, Sir Roger L’Estrange
happened to be a performer, and Sir Roger not leaving the room upon
Cromwell’s coming into it, the Cavaliers gave him the name of
Oliver’s Fiddler. In a pamphlet entitled \textit{Truth and Loyalty
vindicated}, 4to., 1662, Sir Roger thus tells the story:—

\begin{quotation}
“Mr. Edward Bagshaw will have it that I frequently solicited a
private conference with Oliver; and that I often brought my fiddle
under my cloak to facilitate my entry. Surely this Edward Bagshaw has
been pastor to a Gravesend boat; he has the vein so right. A fiddle
under my cloak? Truly my fiddle is a base viol, and that’s somewhat
a troublesome instrument under a cloak. ’Twas a great oversight he
did not tell my lord to what company (of fiddlers) I belonged.
Concerning the story of the fiddle, this I suppose might be the rise
of it. Being in St. James’ Park, I heard an organ touched in a little
low room of one Mr. Hickson’s. I went in, and found a private company
of some five or six persons. They desired me to take up a viol, and
bear a part. I did so, and that a part, too, not much to advance the
reputation of my cunning. By and by, without the least colour of a
design or expectation, in comes Cromwell. He found us playing, and,
as I remember, so he left us.’’
\end{quotation}

Sir Roger never lost the name, for as late as 1683 a pamphlet was
printed about him under the title of “The Loyal Observator; or
Historical Memoirs of the Life and Actions of Roger the Fidler.”

Anthony à Wood also tells a story of Cromwell’s love of music. He
says, “A. W. had some acquaintance with James Quin, M.A., one of the
senior students of Christ-Church, and had several times heard him
sing with great admiration. His voice was a base, and he had a great
command of it; ’twas very strong, and exceeding trouling. He had been
turn’d out of his place by the visitors, but being well acquainted
with some great men of those times that loved music, they introduced
him into the company of Oliver Cromwell the
\end{fixedpage}%415
\pagebreak

\begin{fixedpage}%416
\versoheader

Protector, who loved a good voice and instrumental music well. He
heard him sing with great delight, liquored him with sack, and in
conclusion, said, ‘Mr. Quin, you have done well, what shall I do for
you?’ To which Quin made answer, ‘That your Highness would be
pleased to restore me to my student’s place;’ which he did
accordingly.” (\textit{Life of Anthony à Wood}. Oxford, 1772, p. 139.)

Cromwell treated Oxford much better than Cambridge, and it seems
to have been a place of almost peaceable retirement for musicians,
during the Protectorate. Anthony à Wood gives a glowing account of
the delight he experienced in the weekly music parties there, and
relates some other freaks, such as joining in a disguise of country
fiddlers and going to Farringdon Fair. His companions in this were W.
Bull, who like himself played on the violin; E. Gregory, B.A. and
Gentleman Commoner of Merton College, who played on the base-viol; J.
Nap, of Trinity, on the citerne; and G. Mason, of the same College,
on another wire instrument. They got on very well, played to the
dancing on the green, and received a sufficiency of money and drink;
but, in returning home, they were overtaken by some soldiers, who
made them play in the open field, and left them without giving them a
penny. He says, “Most of my companions would afterwards glory in
this, but I was ashamed, and never could endure to hear of it.” (p.
81.) Wood’s accounts of the music parties, and of the musicians who
were then in Oxford, have been copied into Hawkins’ \textit{History of Music};
I will, therefore, only add what he says of the instruments:—“The
gentlemen in these private meetings played three, four, and five
parts, with viols, (as treble-viol, tenor, counter-tenor, and base,)
with an organ, virginal, or harpsicon, joyn’d with them: they
esteemed a violin to be an instrument only belonging to a common
fidler, and could not endure that it should come among them, for fear
of making their meetings to be vain and fidling. But before the
restoration of King Charles II. (and especially after), viols began
to be out of fashion, and only violins were used, as treble-violin,
tenor, and base-violin; and the King, according to the French mode,
would have twenty-four violins playing before him, while he was at
meals, as being more airy and brisk than viols.” (p. 97, 8vo., Oxford
Edit., 1772.) Hence the song of \textit{Four-and-twenty Fiddlers all of a
row}.

As to ballads, it was said, in 1641, that “there hath been such a
number of ballad-makers and pamphlet-writers employed this yeare,
that it is a wonder that there was any room for that which was made
in Queen Elizabeth’s time, upon the Northerne Rebellion, now
reprinted.” (\textit{Vox Borealis}.) In 1642, ballads respecting “the great
deeds of Oliver Cromwell at Worcester and Edgehill,” were gravely
proposed to Parliament to be sung at Christmas in place of
Christmas-carols. (See No. 6, of “Certaine Propositions offered to
the consideration of the Honourable houses of Parliament,” reprinted
in \textit{Antiquarian Repertory}, iii. 34, 4to., 1808.)

The ballads written against Cromwell personally were principally
aimed at his fanaticism, at his red nose, at his having been a brewer
(which is not the fact), and at his having been run away with by some
German horses, which they do not fail to wish had broken his neck.
The accident is thus related by Heath,
\end{fixedpage}%416
\pagebreak

\begin{fixedpage}%417
\rectoheader

in his Flagellum (4th edit., 1669):—“Cromwell would shew his
skill in driving six great German horses in Hyde Park (sent him as a
present by the court of Oldenburgh), but they no sooner heard the
lash of the whip, but away they ran, with Thurloe sitting trembling
in it for fear of his neck, over hill, over dale, and at last threw
down their inexpert governor from the box into the traces. Of this
some ingenious songs were made, and one called \textit{The Jolt}, by Sir John
Birkenhead, which being in print in a history, in the \textit{Rump Songs},
though the author is mistaken, is purposely forborn.” (p. 152.)

In 1642, the first ordinances were issued for the suppression of
stage plays; and in 1643, a tract was printed, with the title of “The
Actor’s Remonstrance or Complaint for the silencing of their
Profession;” which shews, among other things, the distress to which
the musicians of the theatres were thereby reduced. The writer says,
“Our musike that was held so delectable and precious that they
scorned to come to a tavern under twenty shillings salary for two
houres, now wander with their instruments under their cloaks (I mean
such as have any), to all houses of good fellowship, saluting every
room where there is company with, \textit{Will you have any musike, gentlemen?}”
(Note to Dodsley’s \textit{Old Plays}, v. 432.) Some of the shops in London
were kept open on Christmas-day in 1643, the people being fearful of
“a popish observance of the day.” The Puritans gradually prevailed,
and in 1647 some of the parish officers of St. Margaret’s,
Westminster, were committed to prison for permitting ministers to
preach on Christmas-day, and for adorning the church. On the 3rd of
June, 1647, it was ordained by Lords and Commons in Parliament that
the Feast of the Nativity of Christ should no longer be observed.

The final ordinance for suppressing all stage plays and
interludes, as “condemned by ancient heathens, and by no means to be
tolerated among professors of the Christian religion,” was enacted
Feb. 13, 1647--8; and on Dec. 13, 1648, Captain Betham was appointed
Provost-Martial, “with power to \textit{seize upon all ballad-singers}, and to
suppress stage plays.” (Whitelock’s \textit{Memorials}, p. 332.) From this
time we may safely assume that no more ballads were written in their
favour, and that the majority, at least, had long been against them.
Loyal songs were printed secretly, in spite of this ordinance; and,
in one by Sir Francis Wortley, Bart, (to the tune of \textit{Tom of Bedlam}),
printed \ad 1648, are the following concluding lines:—

\settowidth{\versewidth}{Bless Tom from the slash; from Bride-well’s lash,}
\begin{dcverse}
\begin{altverse}
\vleftofline{“}Bless the printer from the searcher \\
And from the Houses’ takers.\\
Bless Tom from the slash; from Bride-well’s lash, \\
Bless all poor ballad-makers.
\end{altverse}

\begin{altverse}
Those who have writ for the King, for the good King,\\
Be it rhime or reason,\\
If they please but to look through Jenkins his book,” (\textit{Lex Terræ}, 1647)\\
“They’ll hardly find it treason.”
\end{altverse}
\end{dcverse}

In 1649, while the King was still in prison, Marchamont Needham
wrote these lines, but did not then dare to print them:—

\settowidth{\versewidth}{Here’s a health to the King in sack,}
\begin{dcverse}
\vleftofline{“}Here’s a health to the King in sack, \\
To the Houses in small beer,\\
In vinegar to the crabbed pack  \\
Of priests at Westminster."
\end{dcverse}

The last is an allusion to the “synod of divines.”
\end{fixedpage}%417
\pagebreak

\begin{fixedpage}%418
\versoheader

Ad extraordinary collection of the political songs and ballads
from the commencement of the Long Parliament (Nov., 1640), to the
restoration of Charles II., is contained in what are termed the
King’s Pamphlets, now in the British Museum. These Pamphlets were
secretly collected by a bookseller, named George Thomason, and were
intended for the use of Charles I. They were presented to the
national library by George III., who is said to have purchased them
for three or four hundred Pounds, although the original collector
refused 4,000\textit{l}. for them. They consist of about 30,000 pieces
uniformly bound in 2,000 volumes, and the day of the month and year
in which each was issued are noted upon them. One of the volumes was
borrowed by Charles I., while at Hampton Court, and he dropped it in
the mud in his flight to the Isle of Wight. The accident is
commemorated by a memorandum in the book (vol. 100, small 4to.), and
the edges still show the stains of dirt—some to more than an inch in
depth.

The collections of songs which were printed at the Restoration,
are, as might be supposed, wholly on the side of the King. “Rats
rhymed to death; or, The Rump Parliament hang’d up in the Shambles,”
was one of the first. This was printed in 1660, and in the same year
“The Rump; a collection of Songs and Ballads, made upon those who
would be a Parliament, and were but the Rump of an House of Commons,
five times dissolv’d.” This was enlarged in 1662, and printed as
“Rump; or an exact collection of the choicest poems and songs
relating to the late times, by the most eminent wits from anno 1639
to anno 1661.” The last includes all in \textit{Rats rhymed to death}, except
two at the close of the volume. The most voluminous writer of \textit{songs}
on the King’s side was Alexander Brome; but by far the most useful
and important to the Royal cause was Martin Parker, of ballad fame.
His “The King shall enjoy his own again,” did more to support the
failing spirits of the Cavaliers throughout their trials than the
songs of all other writers put together, and contributed in no small
degree to the restoration of Charles II. Monk, the general who
brought him back, was a mere follower of the times.

Martin Parker is a writer who has certainly been under-valued.
Ritson pronounces him “a Grub-street scribbler, and great
ballad-monger of Charles the First’s time,” but he did not know that
he was the author of the poem, “The Nightingale warbling forth her
own disaster; or, the Rape of Philomela,”— of “Robin Conscience,”\textsuperscript{a}—or
of this song which he eulogises so highly.

In \textit{Vox Borealis}, 1641, he is described as “one Parker, the
Prelates’ Poet, who made many base ballads against the Scots,” for
which he was “like to have tasted of Justice Long’s liberality, and
hardly he escaped his Powdering-Tub, which the vulgar people call a
prison.” In an anti-episcopal pamphlet, called “Laws and Ordinances,
forced to be agreed upon by the Pope and his Shavelings for the
disposing of his adherents and the Popish Rites he sent into
England,” he

\begin{dcfootnote}
\textsuperscript{a} Mr. Gutch, in his account of Martin Parker (Robin 
Hood, ii. 84), does not mention his \textit{Robin Conscience}, a
copy of which is in the Bodleian Library (1635).
In Sam. Holland’s \textit{Romancio-Mastix, or a Romance on Romances},
mention is made Of “Martin Parker’s Heroic Poem called
\textit{Valentine and Orson}.” He was also the author of \textit{A
true Tale of Robin Hood}, printed for T. Cotes, 1631 (a copy in
the Ashmolean Library, dated 1686, and another in the  Bodleian, 
without date); of \textit{A Garland of Withered Roses}, 
1656; of \textit{The Poet’s Blind-Man’s Bough} [buff], or \textit{Have
among you, my blind Harpers}, 1641; of \textit{The King and a
 poor Northern man}, 1640 (the story of
which seems to have been taken from an old play); and of many of the ballads in this collection.
See Index.
\end{dcfootnote}
\end{fixedpage}%418
\pagebreak

\begin{fixedpage}%419
\rectoheader

is mentioned with two others,—Taylor, the Water-poet, and Thomas
Herbert.\textsuperscript{a} “Article 2.—We appoint John Taylor, Martin Parker, and
Herbert, all three English poetical, papistical, atheistical
ballad-makers, to put in print rhyme-doggery from the river of Styx,
against the truest Protestants, in railing lines; and, in the end,
young Gregory” [Gregory Brandon, the common hangman] “shall be their
paymaster.”\textsuperscript{b}

Martin Parker was probably at one time an alehouse-keeper, for
the author of \textit{Vox Borealis} says, “But now he swears he will never put
pen to paper for the prelates again, but betake himself to his
pitcht” [spouted] “can and tobacco pipe, and learn to sell his frothy
pots again, and give over poetry.”

In the “Actor’s Remonstrance or Complaint for the silencing of
their profession, and banishment from their several Play-houses,”
1643, the author expresses his fear that “some of our ablest ordinary
Play-Poets, instead of their annual stipends and beneficial
second-days, being for mere necessity compelled to get a living,\dots
 will shortly (if they have not been forced to do it already) be
incited to enter themselves into Martin Parker’s Society, and write
ballads.” This sounds like a covert threat to the puritan
magistrates, or, at least, as intended to let them understand that
their pens would be employed in a manner which might be less
agreeable to them. Martin Parker’s ballad-writing society is again
mentioned in “The Downefall of Temporizing Poets, unlicenst Printers,
upstart Booksellers, trotting Mercuries, and bawling Hawkers,”
1641.—“You [ballad-writers] are very religious men; rather than you
will lose half-a-crown, you will write against your own fathers. You
will make men’s wills before they be sicke, hang them before they are
in prison, and cut off heads before you know why or wherefore. You
have an indifferent strong corporation; twenty-three of you
sufficient writers, besides Martin Parker!” Twenty-four able
ballad-writers! and yet all their productions are now so scarce as to
be marketably worth their weight in gold.

“Inspired with the spirit of \textit{ballating}” says Flecknoe, in a
whimsey printed at the end of his \textit{Miscellanea}, 1653, “I shall sing in
Martin Parker’s vein:—

\settowidth{\versewidth}{With thy ballets did make all England roar,’” \&c.}
\begin{scverse}
\vleftofline{‘}O Smithfield, thou that in times of yore.\\
With thy ballets did make all England roar,’” \&c.
\end{scverse}

\begin{dcfootnote}
\textsuperscript{a} In Brand’s Sale-Catalogue, Part 2, No. 2923, is “Mercurie's
Message defended against the vain, foolish, simple, and absurd cavils
of Thomas Herbert, ridiculous Ballad-maker. Portraits, 4to., 1641.”
Several of Herbert's productions are mentioned by Lowndes.

\textsuperscript{b} As this pamphlet is very scarce., and exhibits an attempt at
humour, not usual in puritanical pamphlets, a few specimens are
subjoined.

“Article 1. We leave the great Archbishop’s cause” [Laud’s] “to
the mercy of the parliament, because it is not in our power to help
him,”

“3. I counsel the English Bishops to send their Mitres to the
book-binders’ shops, and bespeak them bibles well bossed therewith,
because we apprehend no means to keep them longer from their
studies.”

“4. We advise them to send their crosier staves to the joiners,
to be translated into crutches; for we see that (with great sorrow)
they must be forced to stoop.”

“5. We advise them to send their lawn sleeves to the
sempsters, that they may have handkerchers in readiness to wipe
their eyes when they shall weep for their just-deserved downfall.”

“6. Whereas the English Prelates and prestigious”
[juggling]“Priests, being well affected to Popish rites, vested their
black insides with white Rochets and Surplices, if they can procure
them, let them be turned into shirts for them; we counsel them
henceforth to vest themselves outwardly in mourning black.”

“7. We advise the Bishops to stuff their Cater-caps with
feathers, to serve them for cushions in their closets, that they may
sit at ease after they are driven to study thither.”

“8. It is our provident care that their scarlet robes he given to
their eldest daughter, wife, or nearest kinswoman, to be worn in a
petticoat for posterities, as an emblem of the predecessor’s crimes.”

“14. We censure the Organ-pipes to be burned in the founder’s
melting pot, because we cannot help it.”
\end{dcfootnote}

\end{fixedpage}%419
\pagebreak

\begin{fixedpage}%420
\versoheader

In The \textit{Joviall Crew}; or \textit{The Devill turn’d Ranter}, 4to., 1651,
after a catch has been sung, “the best and newest in
town,”—“Excellent (says a Ranter) did this Minerva take flight from
John Taylor’s or Martin Parker’s vein. In \textit{Naps upon Parnassus}, 1658,
Martin Parker is styled “the Ballad-maker Laureat of London,” but in
Part 2 of \textit{The Night Search}, 1646, his works are
not very respectfully treated:—

\settowidth{\versewidth}{With Martin Parker’s works, and such like things.”}
\begin{scverse}
\vleftofline{“}A box of salve, and two brass rings,\\
With Martin Parker’s works, and such like things.”
\end{scverse}

Two of his ballads are quoted by Izaak Walton in his charming
book \textit{The Angler}, 1653 (ante pp. 295 and 297); and perhaps the latest
contemporary notice of him is contained in Dryden’s comedy of \textit{Sir
Martin Mar-all}, which was acted at the Duke’s Theatre, 1668.—act v.,
sc. 1:—

\begin{quotation}
\textit{Sir Martin}— There’s five shillings for thee. What? we must
encourage good wits sometimes.”

\textit{Warn}.—“Hang your white pelf: sure, Sir, by your largess, you
mistake me for Martin Parker, the Ballad-maker.”
\end{quotation}

John Wade was another of the many ballad-writers employed on the
King’s side. He was the author of “The Royall Oak, or the wonderfull
Travells, miraculous Escapes, strange Accidents of his Sacred Majesty
King Charles the Second,” which has been reprinted, from a
cotemporary black-letter copy in Mr. Halliwell’s Collection, in \textit{Notes
and Queries} (vol. x., p. 340).

Thomas Weaver, who had been turned out of the University of
Oxford by the Presbyterians, was the author of a collection of songs,
in which he ridiculed the Puritans so effectually that the book was
denounced as a seditious libel against the government, and a capital
indictment founded upon it. He escaped with his life (according to
Anthony à Wood) in consequence of a very humane charge from the
judge. He afterwards “sank into the office of an exciseman at
Liverpool, where he was called Captain Weaver, and where he died in
inglorious obscurity.” His book of songs is not contained in the
King’s Pamphlets, nor have I been able to see a copy.

The \textit{first} who came forth as champion of the royal cause, in
English verse (according to Wood), was John Cleveland, or Cleiveland,
then a fellow of St. John’s College, Cambridge. His lines on “The
Rebel Scot,” “The Scot’s Apostacy;” “On the Death of His Royal
Majesty, Charles, late King of England,” \&c.; and his song, “The
Puritan” (to the tune of \textit{The Queen’s Old Courtier}), and others, prove
him to have been a powerful, and often dignified, yet most sarcastic
writer. He adhered to the royal cause till its ruin. At last, in
1655, after having led for some years a fugitive life, he was
arrested in Norwich, and taken before the Commissioners, who
imprisoned him at Yarmouth. Having been confined there for three
months, he petitioned Cromwell, who ordered his release. The
transaction was honourable to both parties. Cleveland’s spirit is
shown in his petition. He thus addresses the Protector: “I am induced
to believe that, next to my adherence to the royal party, the cause
of my confinement is the narrowness of my estate; for none stand
committed whose estates can bail them. I only am the prisoner, who
have no acres to be my hostage. Now, if my poverty be criminal (with
reverence be it spoken), I implead your Highness
\end{fixedpage}%420
\pagebreak

\begin{fixedpage}%421
\rectoheader

whose victorious arms have reduced me to it, as accessory to my
guilt. Let it suffice, my Lord, that the calamity of the war hath
made us poor: do not punish us for it.\dots I beseech your Highness,
put some bounds to the overthrow, and do not pursue the chase to the
other world. Can your thunder be levell’d so low as our grovelling
condition? Can your towering spirit, which hath quarried upon
kingdoms, make a stoop at us, who are the rubbish of these ruins?
Methinks I hear your former achievements interceding with you not to
sully your glories with trampling upon the prostrate; nor clog the
wheel of your chariot with so degenerous a triumph. The most renowned
heroes have ever with such tenderness cherished their captives, that
their swords did but cut out work for their courtesies.\dots For the
service of his Majesty, if it be objected, I am so far from excusing
it, that I am ready to alledge it in my vindication, I cannot conceit
that my fidelity to my prince should taint me in your opinion; I
should rather expect it should recommend me to your favour.\dots You
see, my Lord, how much I presume upon the greatness of your spirit,
that dare present my indictment with so frank a confession,
especially in this, which I may so safely deny that it is almost
arrogancy in me to own it; for the truth is, I was not qualified
enough to serve him: all I could do was to bear a part in his
sufferings, and to give myself to be crushed with his fall.\dots My
Lord, you see my crimes; as to my defence, you bear it about you. I
shall plead nothing in my justification but your Highness’s clemency,
which as it is the constant inmate of a valiant breast, if you
graciously be pleased to extend it to your suppliant, in taking me
out of this withering durance, your Highness will find that mercy
will establish you more than power, though all the days of your life
were as pregnant with victories as your twice auspicious third of
September.—Your Highness’s humble and submissive Petitioner.” After
his release, Cleveland came to London, “where he found a generous
Mæcenas,” and being much admired among all persons of his own party,
became a member of a club of wits and loyalists, which Butler, the
author of \textit{Hudibras}, frequented. He died a little before the
Protector, from an epidemic intermitting fever.

To show how much Cromwell forgave in Cleveland, two extracts from
his works are subjoined. The first from \textit{The Character of a London
Diurnal}. “This Cromwell is never so valorous as when he is making
speeches for the Association; which, nevertheless, he doth somewhat
ominously, with his neck awry, holding up his ear as if he expected
Mahomet’s pigeon to come and prompt him. He should be a bird of prey,
too, by his bloody beak,” \&c. The second is Cleveland’s \textit{Definition
of a Protector}:—

\settowidth{\versewidth}{A bladder blown, with other breaths puff’d full;}
\begin{dcverse}
“What’s a Protector? He’s a stately thing, \\
That apes it in the nonage of a king; \\
A tragic actor—Caesar in a clown:\\
He’s a brass farthing stamped with a crown;\\
A bladder blown, with other breaths puff’d full;\\
Not the Perillus, but Perillus’ bull:\\
Æsop’s proud ass veil’d in the lion’s skin;\\
In fine, lie’s one we must Protector call,—\\
An outward saint lin’d with a devil within:

An echo whence the royal sound doth come,  \\
But just as barrel-head sounds like a drum:\\
Fantastic image of the royal head, \\
The brewer’s with the King’s arms quartered:\\
He is a counterfeited piece, that shows\\
Charles his \textit{effigies} with a copper nose:\\
From whom the King of kings protect us  all.
\end{dcverse}
\attribution{Cleaveland’s \textit{Revived Poems}, p. 343, 8vo., 1687.}

\end{fixedpage}%421
\pagebreak

\begin{fixedpage}%422
\versoheader

George Wither is said to have got “The Statute Office” from
Cromwell, “by rhyming,” but I have not found any \textit{song} written by him
in his favour. Wither was a loser, rather than a gainer, by his
advocacy of the cause of the Parliament, for having been the first
person of any note in the county of Surrey who took up arms for the
parliament, his house was destroyed, and his property injured to the
uttermost, when the Cavaliers were there, and he could never obtain
adequate redress. His muse had been employed for some time before
upon sacred subjects, and he appears then to have given up
song-writing altogether. The Rev. Robert Aris Willmott, in his \textit{Lives
of Sacred Poets}, dates Wither’s accession to the Statute Office
between 1655 and 1656, and concludes that the appointment was, in
other words, to the Record Office, which was bestowed upon Prynne
after the Restoration. The passage from which I derived the
information of his having held the office is in “The last Speech and
dying words of Thomas (Lord, alias Colonel) Pride, being touched in
conscience for his inhuman murder of the Bears in the Bear garden,
when he was High-Sheriff of Surrey,” 4to., 1680. (Reprinted in \textit{Harl.
Miscellany}, 4to., iii. 135.) “I do not mean Mr. George
Withers, for he got the Statue-office by rhyming, but when will
he sell his verses? A \textit{statue} lies upon them, so as no body will buy
them.”

I have said that the “remonstrance” of the “play-poets” that they
should be compelled to enter Martin Parker’s society seemed to convey
a covert threat to those who closed the theatres, that they would
become the subjects of ballads. A few quotations from plays will
perhaps best show how general was the fear of being “balladed” in the
seventeenth century.

\settowidth{\versewidth}{This hasty work was ne’er well done: give us so much time }
\begin{scverse}
\vleftofline{“}Good Master Sheriff, your leave too;\\
This hasty work was ne’er well done: give us so much time \\
As but to sing our own ballads, for we’ll trust no man,\\
Nor no tune but our own; ’twas done in ale too,\\
And, therefore, cannot be refus’d in justice;\\
Your penny-pot poets are such pelting thieves,\\
They ever hang men twice.”
\end{scverse}

This is from an unfinished play of Fletcher’s, \textit{The bloody
Brother; or, Rolla, Duke of Normandy}, which was one of those secretly
performed “in the winter before the King’s murder.” Again, in \textit{The
Lover’s Progress}, act v., sc. 3, he makes Malfort say:— 
\settowidth{\versewidth}{I have penn’d mine own ballad}
\begin{scverse}
\vleftofline{“}I have penn’d mine own ballad\\
Before my condemnation, in fear \\
Some Rhymer should prevent me.”
\end{scverse}

In the \textit{Humourous Lieutenant}, act ii., sc. 2—
\settowidth{\versewidth}{Now shall we have damnable ballads out against us,}
\begin{scverse}
\vleftofline{“}Now shall we have damnable ballads out against us,\\
Most wicked Madrigals; and ten to one, Colonel,\\
Sung to such lamentable tunes.”
\end{scverse}

In \textit{The Pilgrim}, act iii., sc. 4— 
\begin{scverse}
\vleftofline{“}I shall be taken \\
For their commander now, their General,\\
And have a commanding gallows set up for me \\
As high as a May pole, and nasty songs made on me,\\
Be printed with a pint pot and a dagger.”
\end{scverse}
\end{fixedpage}%422
\pagebreak

\begin{fixedpage}%423
\rectoheader

In Rowley’s \textit{A Woman never vext} (1632), act i., sc. 1—
\settowidth{\versewidth}{Now shall we have damnable ballads out against us,}
 
\begin{scverse}
\vleftofline{“}And I’ll proclaim thy baseness to the world,\\
Ballads I’ll make, and make ’em tavern music \\
To sing thy churlish cruelty.”
\end{scverse}
 

In Ford’s \textit{The Lady’s Trial}, act ii., sc. 2—
\settowidth{\versewidth}{“You are grown a tavern talk}
 
\begin{scverse}
\vleftofline{“}You are grown a tavern talk \\
Matter for fiddlers’ songs.’’
\end{scverse}
 

In Ford’s \textit{Love’s Sacrifice}, act iii., sc. 1—
\settowidth{\versewidth}{Shall jig out thy wretchedness and abominations}
 
\begin{scverse}
---“Ballad singers and rhymers\\
Shall jig out thy wretchedness and abominations\\
To new tunes."
\end{scverse}
 

In Shirley’s \textit{The Court Secret}, act v., sc. 1—
\settowidth{\versewidth}{The people, when I take my leave of the world,}
 
\begin{scverse}
\vleftofline{“}I have prepar’d a ballad, Sir,\\
Before I die, to let the people know\\
How I behav’d myself upon the scaffold.\\
With other passages that will delight\\
The people, when I take my leave of the world,\\
Made to a Pavan tune.”
\end{scverse}
 

In Davenport’s \textit{The City Night-cap}, act i., sc. 1—
\settowidth{\versewidth}{“Let ballad-mongers crown him with their scorns.”}
 
\begin{scverse}
“Let ballad-mongers crown him with their scorns.”
\end{scverse}
 

In Killegrew’s Parson’s Wedding, act i., sc. 1—
\backskip{0.5}
\begin{quotation}
“I’ll put the cause in print too; I’m but a scurvy poet, yet I’ll
make a ballad shall tell how, \&c.”
\end{quotation}

The political importance of songs and ballads in aiding great
changes, whether reformatory, revolutionary, or otherwise, has been
proved not only in our own country, but in almost every other. A
well-known passage in Andrew Fletcher of Saltoun’s \textit{Political Works}
(often quoted, but not always correctly given), is so peculiarly to
the purport, that I hope to be excused for again citing it.— “I knew
a very wise man so much of Sir Christopher [Musgrave]’s sentiment,”
[as to the effect of songs and ballads, both in a political and moral
sense], “that he believed if a man were permitted to make all the
ballads, he need not care who should make the laws of a nation.” (p.
266, 12mo., Glasgow, 1749.)

It was during the Commonwealth that “honest John Playford”
commenced publishing music, and The \textit{English Dancing Master, or plaine
and easie rules for the dancing of Country Dances, with the tune to
each dance}, appears to have been his first musical publication.\textsuperscript{a}
Thomason has marked the date on the copy among the King’s pamphlets,
as 10th of March, 1650, which, according to the new style, would be
1651. In the preface, Playford speaks of “the sweet and airy activity
of the Gentlemen of the Inns of Court, which has crowned their

\begin{dcfootnote}
\textsuperscript{a} I find entries of \textit{books} printed by Playford, as early
as 1648, in the Registers of the Stationers' Company, but
no \textit{music} before 1650, old style. In 1651, be published
“A Musical Banquet, in three books, consisting of Lessons 
for the Lyra Viol, Allmains, and Sarabands, Choice
Catches and Rounds,” \&c. A copy of this rare work is
in the Douce Collection, Bodleian Library. Playford was
not only a printer, but also Clerk of the Temple~Church.

In 1652, besides a second edition of \textit{The Dancing Master}, 
he published \textit{Musick’s Recreation on the Lyra Viol},
Hilton’s \textit{Catch that Catch can} (of which a second edition
was printed in 1658), and \textit{Choice Ayres}, \&c. His musical
publications after this date are (with the exception of the
\textit{Court Ayres}, referred to in the text) more generally~known.
\end{dcfootnote}
\end{fixedpage}%423
\pagebreak

\begin{fixedpage}%424
\versoheader

grand solemnities with admiration to all spectators.” Some
allusion has already been made to their masques and dances, (ante p.
328, and note), to which I may add, that the author of “Round about
our Coal Fire, or Christmas Entertainments,” says, “the dancing and
singing of the Benchers, in the great Inns of Court, in Christmas, is
in some sort founded upon interest; for they hold, as I am informed,
some privilege, by dancing about the fire, in the middle of their
Hall, and singing the song of \textit{Round about our coal fire}, \&c. Leaving
to the gentlemen of the bar to determine what this privilege was, I
will only add, that the eulogy of their sweet and airy activity, is
contained in every edition of \textit{The Dancing Master} to 1701 inclusive,
but omitted in and after that of 1703.

A large proportion of the tunes in the first edition of \textit{The
Dancing Master}, are contained in the present collection, because they
are ballad tunes. Sir Thomas Elyot, in his \textit{Governour}, 1531, after
describing many ancient modes of dancing, says: “And as for the
special names [of those dances], they were taken, \textit{as they be now},
either of the names of the first inventors, or of the measure and
number they do contain; or, of \textit{the first words of the ditty which the
song comprehendeth, whereof the dance was made}.” If this custom of
naming them after the ditty had not been retained in Playford’s time,
it would have been almost impossible \textit{now} to identify the tunes of our
old ballads, for the words and music are very rarely to be found
together.

In 1655, Playford published “Court Ayres; or, Pavins, Almaines,
Corants, and Sarabands, Treble and Basse, for Viols or Violins*,” and
reprinted them in 1662, with additions, under the title of “Courtly
Masquing Ayres, containing Almanes, Ayres, Corants, Sarabands,
Moriscos, Jiggs,” \&c. In the preface to the latter, he says, “About
seven years since, I published a collection of ayresof this nature,
entitled Court Ayres, containing 245 lessons; \textit{it being the first of
that kind extant}, I printed, therefore, but a very small impression,
yet when it was once abroad, it found so good acceptance both in
this kingdom and beyond seas, that there it was reprinted to my great
damage, and was the chief reason that I publish’d it no more till
now.” The composers of this collection are William Lawes, Dr, Charles
Colman, John Jenkins, Benjamin Rogers, Davis Mell, John Banister,
William Gregory, Matthew Lock, and Thomas Gibbes. The republication
abroad of the music of the English Court Masques, confirms, in some
degree, Lord Orford’s view, that the Court of Charles I. was looked
upon as “\textsc{the most polite Court in Europe}.”

\centerrule

In searching for the songs and tunes of this particular period,
the reader will find it necessary to refer to the first volume, as
many of the oldest tunes were still in use, such as \textit{John Dory, Old
Sir Simon the King, Tom a Bedlam}, \&c. A very small proportion of the
songs now possess sufficient interest for republication; and some are
necessarily excluded, by their coarseness.

\centerrule

\end{fixedpage}%424
\pagebreak

\begin{fixedpage}%425
\rectoheader

\musictitle{HEY, THEN, UP GO WE.}

This song, which describes with some humour the taste of the
Puritans, might pass for a Puritan song if it were not contained in
\textit{The Shepherd’s Oracles}, by Francis Quarles, 1646. Quarles was
cup-bearer to Elizabeth, Queen of Bohemia, the daughter of James I.;
was afterwards Secretary to Archbishop Usher (Primate of Ireland),
and Chronologer to the city of London. He died in 1644, and The
\textit{Shepherds Oracles} were a posthumous publication.

Other copies of the words will be found in MSS. Ashmole, 36 and
37, fol. 96; in \textit{Loyal Songs written against the Rump Parliament}, i.
14; in Ellis’s \textit{Specimens}; and in Stafford Smith’s \textit{Musica Antiqua}. The
music in the last named is not a popular tune, but the work of some
composer unknown. It is there printed from a manuscript once in the
possession of Dr. William Boyce.

Some differences will be found in the various copies; for
instance, in \textit{The Shepherds Oracles}, the line, “Then \textit{Barrow} shall be
sainted,” is, in \textit{Musica Antiqua}, “Then \textit{Burton} shall be sainted,” and
in \textit{Loyal Songs}, “Then \textit{Burges},” \&c. In the last, there are two
additional stanzas, and the tune is changed to one already printed
(ante p. 341). In Ashmole’s manuscript, the song is entitled “The
Triumph of the Roundheads; or the Rejoicing of the Saints.”

D’Urfey calls this “an old ballad tune of forty-one”—\ie, 1641.
He wrote a song to the air, for his play of The \textit{Royalist}, which was
acted at the Duke’s Theatre in 1682. D’Urfey borrowed about five of
the seven verses of Quarles’ song, making only a few verbal
alterations. The last line of each stanza is, “Hey, \textit{then}, up go we,”
both in his play and in Quarles’ song; but in Pills to purge
Melancholy, and some other copies, “Hey, \textit{boys}, up go we.” \textit{Hey, then,
up go we} is quoted in \textit{A Satyr against Hypocrites}, 4to., 1661.

Two other names for the tune are \textit{The clean contrary way}, and \textit{The
good old cause}. “The good old cause” meant the maintenance of the
rights of the subject against the encroachments of the king.

In \textit{A Choice Collection of 120 Loyal Songs}, \&c., 12mo., 1684, is
“An excellent new Hymn, exalting the Mobile to Loyalty,” \&c., “To
the tune of \textit{Forty-one};” commencing—

\settowidth{\versewidth}{Whose threat’nings are as senseless as }
\begin{dcverse}
\begin{altverse}
\vleftofline{“}Let us advance the \textit{good old cause},\\
Fear not Tantivitiers,\\
Whose threat’nings are as senseless as \\
Our jealousies and fears.
\end{altverse}

\begin{altverse}
’Tis we must perfect this great work,\\
And all the Tories slay,\\
And make the King a glorious Saint—\\
\textit{The clean contrary way}.”
\end{altverse}
\end{dcverse}

This is a mere alteration of a song by Alexander Brome, entitled
“The Saint’s Encouragement; written in 1643,” and printed in his
\textit{Songs and other Poems},
12mo., 1644, (p. 164). It commences thus:—

\begin{dcverse}
\begin{altverse}
\vleftofline{“}Fight on, brave soldiers, for the cause,\\
Fear not the Cavaliers;\\
Their threat’nings are as senseless as\\
Our jealousies and fears.

’Tis you must perfect this brave work,\\
And all malignante slay,\\
You must bring hack the King again—\\
The clean contrary way”
\end{altverse}
\end{dcverse}

In the collection of \textit{Loyal Songs written against the Rump
Parliament}, instead of “The Saint’s Encouragement,” \&c., Brome’s
song is headed “On Colonel Venne’s Encouragement to his Soldiers: A
Song” (i. 104, edit. 1731.)
\end{fixedpage}%425
\pagebreak

\renewcommand\versoheadertext{english song and ballad music.}
\begin{fixedpage}%426
\versoheader

\textit{The clean contrary way} is a very old,\textsuperscript{a} and was a very popular
burden to songs. Some of the songs, however, like that on the Duke of
Buckingham, reprinted by Mr. Fairholt for the Percy Society (No. 90,
p. 10) are in another metre, and were therefore written to other
tunes.

It appears, from some lines in \textit{Choyce Poems, \&c., by the Wits of
both Universities} (printed for Henry Brome, 1661), that some
ballad-singers had been committed to prison, and threatened to be
whipped through the town, for singing one of these songs.

\settowidth{\versewidth}{Then, if they went the people’s tongues to stay,}
\begin{scverse}
\vleftofline{“}The fiddlers must be whipt, the people say,\\
Because they sung \textit{The clean contrary way};\\
Which, if they be, a crown I dare to lay,\\
They then \textit{will} sing, the clean contrary way.\\
And he that did those merry knaves betray,\\
Wise men will praise (the clean contrary way);\\
For whipping them no envy can allay.\\
Unless it be the clean contrary way;\\
Then, if they went the people’s tongues to stay,\\
Doubtless they went the clean contrary way.’’
\end{scverse}

One of the songs was remembered in Walpole’s time, for in a
letter to Sir Horace Mann, dated October 1, 1742, he says, “As to
German news, it is all so simple that I am peevish: the raising of
the siege of Prague, and Prince Charles and Marechal Maillebois
playing at Hunt the Squirrel, have disgusted me from enquiry about
the war. The Earl laughs in his great chair, and sings a bit of an
old ballad: 
\settowidth{\versewidth}{They both did fight, they both did beat.}
\begin{scverse}
\begin{altverse}
\vleftofline{‘}They both did fight, they both did beat.\\
They both did run away;\\
They both did strive again to meet—\\
\textit{The clean contrary way}.’”

\hspace{108pt}\textit{Walpole’s Letters}, 1840, i. 231.
\end{altverse}
\end{scverse}


Among the numerous songs and ballads to this air the following
may be named:—

1. “A Health to the Royal Family; or, The Tories’ Delight: To the
tune of \textit{Hey, boys, up go we.}” (Pepys Coll., ii. 217.) Commencing—
\settowidth{\versewidth}{Come, give’s a brimmer, fill it up,}
\begin{dcverse}
\begin{altverse}
\vleftofline{“}Come, give’s a brimmer, fill it up, \\
’Tis to great Charles our King,\\
And merrily let it go round,\\
Whilst we rejoice and sing.

Let rebels plot, ’tis all in vain,\\
They plot themselves but woe,\\
Come, loyal lads, unto the Queen,\\
And briskly let it go.”
\end{altverse}
\end{dcverse}

\begin{dcfootnote}
\textsuperscript{a} \textit{The clean contrary way}, as a burden, may he traced, in Latin,
to the fifteenth century, if not earlier, as, for instance, in a
highly popular song—

\settowidth{\versewidth}{“Of all creatures women be best.}
\begin{fnverse}
\vleftofline{“}Of all creatures women be best.\\
\textit{Cujus contrarium verum est}.”
\end{fnverse}

Copies of that are contained in tha Minstrels’ Book, reprinted by
Mr. Wright for the Percy Society (\textit{Songs and Carols}, p. 88), and in
a-Collection of Romances, Songs, Carols, \&c., in the handwriting of
Richard Hill, merchant, London, from 1483 to 1535, now in the Library
of Balliol College, Oxford (No. 105, p. 250). Among the complimentary
verses prefixed to \textit{The Wife}, by Sir Thomas Overbury, 1616, one set is
“To the clean contrary \textit{wife};” and the \textit{clean contrary way} occurs
among lines, signed W. S., upon the death of Overbury, prefixed to
his \textit{Characters}, 1616.

There are many ballads to the tune, as “Half a dozen of good
Wives, all for a Penny,” \&c. Roxburge, i. 152; another, ii, 571;
\&c.
\end{dcfootnote}
\end{fixedpage}%426
\pagebreak

\setlength{\fixedpagewidth}{375pt}
\begin{fixedpage}%427
\rectoheader

2. A satirical song by Lord Rochester (Harl. MSS., 6913, p. 267)—
\settowidth{\versewidth}{Send forth, dear Julian, all thy books}
\begin{dcverse}
\begin{altverse}
\vleftofline{“}Send forth, dear Julian, all thy books\\
Of scandal, large and wide,\\
That ev’ry knave that in ’em looks\\
May see himself describ’d.

Let all the ladies read their own,\\
The men their failings see,\\
From \textit{Nell} to him that treads the throne,\\
Then \textit{Hey, boys, up go we}.”
\end{altverse}
\end{dcverse}

3. “The Popish Tory’s Confession,” or, An Answer to the Whig's Exaltation”\&c. “A pleasant new
song to the tune of \textit{Hey, boys, up go we}.” Douce Coll., 182); beginning—
\settowidth{\versewidth}{‘Down with the’ Whigs, we’ll now grow wise.}
\begin{dcverse}
\vleftofline{“}‘Down with the’ Whigs, we’ll now grow wise.\\
Let’s cry out “Pull them down,”\\
By that we’ll rout the \textit{Good old cause},\\
And mount one of our own.

We’ll make the Roundheads stoop to us, \\
For we their betters be,\\
We’ll pull down all their pride with speed, \\
Such Tories now are we.”
\end{dcverse}

This is on Papists calling themselves Tories (printed by J.
Wright, J. Clarke, W. Thackeray, T. Passinger, and M. Coles, \textsc{b.l.},
temp. Charles. II.); and is preceded by eleven long lines, of which
the following six contain the usual derivation of “Tory”:—

\settowidth{\versewidth}{Which \textit{from the worst of Irish thieves} at first}
\begin{scverse}
\vleftofline{“}No honest man, who king and state does love,\\
Will of a name so odious approve,\\
Which \textit{from the worst of Irish thieves} at first \\
Had its beginning, and with blood was nurst.\\
Which shews it is of a right Popish breed,\\
As in their own confession you may read.”
\end{scverse}

4 and 5. The last line perhaps alludes to “The Tories’
Confession; or, A merry song in Answer to the Whig’s Exaltation: To
the tune of \textit{Forty-one}.” A copy of this (London, T. H., 1682) is in
Mr. Halliwell’s Collection, Cheetham Library (No. 3010), as well as
“A new ballad from Whig-land,” to the same air (No. 1045).

6. “The City’s thankes to Southwarke for giving the army
entrance” (Sep. 1, 1647)—
\settowidth{\versewidth}{We thank you more than we can say,}
\begin{scverse}
\vleftofline{“}We thank you more than we can say,\\
But ’tis the \textit{cleane contrary way}.”
\end{scverse}

This is among the King’s Pamphlets, and reprinted in Wright’s
\textit{Political Ballads}, Percy Soc., No. 90, p. 70.

7. “The Thames uncas’d; or, The Waterman’s Song upon the thaw. To
the tune of \textit{Hey, boys, up go we}.” Commencing—
\settowidth{\versewidth}{“Come, ye merry men all, of Waterman’s Hall,”}
\begin{scverse}
“Come, ye merry men all, of Waterman’s Hall,”
\end{scverse}
See \textit{Old Ballads illustrating the Great Frost of} 1683-4,
Percy.Soc., No. 42, p. 30.

8. “Advice to Batchelors; or, The Married Man’s Lamentation.”
Commencing— 
\settowidth{\versewidth}{You batchelors that single are,}
\begin{scverse}
\vleftofline{“}You batchelors that single are,\\
May lead a happy life.”
\end{scverse}

9. “The good Fellow’s Consideration; or, The bad Husband’s
Amendment,” \&c.— 
\settowidth{\versewidth}{Lately written by Thomas Lanfiere,}
\begin{scverse}
\vleftofline{“}Lately written by Thomas Lanfiere,\\
Of Watchat town in Somersetshire.”
\end{scverse}
(Roxburghe Coll., ii. 195. “Printed for P. Brooksby.”)

10. “The good Fellow’s Frolick; or, Kent Street Club. To the tune
of \textit{Hey, boys, up go we}; \textit{Seaman’s mournful bride}; or \textit{The fair one let
me in}. Beginning—
\begin{scverse}
“Here’s a crew of jovial blades\\
That lov’d the nut-brown ale.”—(Rox. Coll., ii. 198.)
\end{scverse}
\end{fixedpage}%427
\pagebreak

\setlength{\fixedpagewidth}{360pt}
\begin{fixedpage}%428
\versoheader

11. “All is ours and our Husband’s; or, The Country Hostess’s
Vindication: To the tune of \textit{The Carman's Whistle}, or \textit{Heigh, boys, up
go we}. (Roxburghe Coll., ii. 8.)

12 and 13. “A Farewell to Gravesend;” and “The merry Boys of
Christmas, or, The Milkmaid’s New Year’s Gift.” (Roxburghe, vol. 4.)

It would be no difficult task to add fifty more to the above
list, but it is already sufficiently lengthy.

The tune is contained in \textit{The Dancing Master} of 1686, and in every
subsequent edition; in 180 \textit{Loyal Songs}, 1685 and 1694; in \textit{Pills to
purge Melancholy}, ii. 286 (1719); and in the following ballad
operas:—\textit{Beggars' Opera}, 1728; \textit{The Patron}, 1729; \textit{The Lover’s Opera},
1629; \textit{Quaker’s Opera}, 1728; \textit{Silvia}, 1731; \textit{The Devil to pay}, 1731; and
\textit{Love and Revenge}, \textsc{n.d.} In some copies it is in common time, in
others in \timesig{6}{4} or \timesig{6}{8}.

%    insert tune
\end{fixedpage}%428
\pagebreak
\settowidth{\versewidth}{We’ll break the windows which the whore}
\begin{fixedpage}%429
\rectoheader

\begin{dcverse}
\indentpattern{01010102}
\begin{patverse}
We’ll break the windows which the whore \\
Of Babylon hath painted,\\
And when the Popish Saints are down,\\
Then Barrow shall be sainted;\\
There’s neither cross nor crucifix\\
Shall stand for men to see,\\
Rome’s trash and trumpery shall go down,\\
And hey, then up go we.
\end{patverse}

\begin{patverse}
Whate’er the Popish hands have built,\\
Our hammers shall undo,\\
We’ll break their pipes, and burn their copes,\\
And pull down churches too;\\
We’ll exercise within the groves,\\
And teach beneath a tree,\\
We’ll make a pulpit of a cask,\\
And hey, then up go we.
\end{patverse}

\begin{patverse}
We’ll put down Universities,\\
Where learning is profest,\\
Because they practise and maintain\\
The language of the beast;\\
We’ll drive the doctors out of doors,\\
And all that learned be;\\
We’ll cry all arts and learning down,\\
And hey, then up go we.
\end{patverse}

\begin{patverse}
We’ll down with deans, and prebends, too,\\
And I rejoice to tell ye\\
We then shall get our fill of pig,\\
And capons for the belly;\\
We’ll burn the Fathers’ weighty tomes,\\
And make the school-men flee;\\
We’ll down with all that smells of wit,\\
And hey, then up go we.
\end{patverse}

\begin{patverse}
If once the antichristian crew\\
Be crush’d and overthrown,\\
We’ll teach the nobles how to stoop,\\
And keep the gentry down:\\
Good manners have an ill report,\\
And turn to pride, we see,\\
We’ll therefore put good manners down,\\
And hey, then up go we.
\end{patverse}

\begin{patverse}
The name of lords shall be abhorr’d,\\
For every man’s a brother,\\
No reason why in church and state\\
One man should rule another;\\
But when the change of government\\
Shall set our fingers free,\\
We’ll make these wanton sisters stoop,\\
And hey, then up go we.
\end{patverse}

\begin{patverse}
What though the King and Parliament\\
Do not accord together,\\
We have more cause to be content,\\
This is our sunshine weather;\\
For if that reason should take place,\\
And they should once agree,\\
Who would he in a Roundhead’s case,\\
For hey, then up go we.
\end{patverse}

\begin{patverse}
What should we do, then, in this case,\\
Let’s put it to a venture,\\
If that we hold out seven years’ space,\\
We’ll sue out our indenture.\\
A time may come to make us rue,\\
And time may set us free,\\
Except the gallows claim his due,\\
And hey, then up go we.
\end{patverse}
\end{dcverse}

The two last stanzas are not contained in Quarles’ copy.

\musictitle{VIVE LE ROY.}

A copy of this song, which may be termed the “God save the King”
of Charles I., of Charles II., and James II., is to be found, both
words and music, in Additional MSS., No. 11,608, p. 54, British
Museum. The tune is in \textit{Musick’s Recreation on the Viol, Lyra-way},
1661; and in \textit{Musick’s Delight on the Cithren}, 1666. The words in \textit{Loyal
Songs}, i. 102, 1731.

The copy among the Additional Manuscripts is in three parts
(treble, tenor, and bass), but without a composer’s name. The title,
\textit{Vive le Roy}, is derived from the burden of each stanza.

It is frequently alluded to, as in the song entitled “A la Mode:
The Cities profound policie in delivering themselves, their cittie,
their works, and ammunition, unto the protection of the Armie”
(August 27,1647), King’s Pamphlets, vol. v., folio; and Wright’s
\textit{Political Ballads}, p. 64—
\settowidth{\versewidth}{The Commons will embrace their King}
\begin{dcverse}
\begin{altverse}
‘And now the Royalists will sing \\
Aloud Vive le Roy;\\
The Commons will embrace their King\\
With an unwonted joy.”
\end{altverse}
\end{dcverse}
\end{fixedpage}%429
\pagebreak

\begin{fixedpage}%430
\versoheader

And in “He that is a clear Cavalier,” the first stanza ends—
\settowidth{\versewidth}{“Freeborn in liberty we’ll ever be,}
\begin{scverse}
\begin{altverse}
“Freeborn in liberty we’ll ever be,\\
Sing \textit{Vive le Roy}.”
\end{altverse}
\end{scverse}

Again, in \textit{A. Joco-serious Discourse}, by George Stuart, 1686, a
welcome to JamesII.,—“the harmonious spheres sound \textit{Vive le Roy}” (p.
3).


Among Mr. Halliwell’s Collection of Ballads is “England’s Honour
and London’s Glory, with the manner of proclaiming Charles the Second
King of England, this eighth of May, 1660, by the Honourable the two
Houses of Parliament, Lord Generall Monk, the Lord Mayor, Aldermen,
and Common Counsell of the City. The tune is \textit{Vive la Roy}.” London,
printed for William Gilbertson. It begins—
\settowidth{\versewidth}{“Then let us sing, boyes, God save the King, boyes}
\begin{scverse}
“Come hither, friends, and listen unto me,\\
And hear what shall now related be;”
\end{scverse}
and the burden is—
\begin{scverse}
“Then let us sing, boyes, God save the King, boyes,\\
Drink a good health, and sing \textit{Vive le Roy}.”
\end{scverse}

\end{fixedpage}%430
\pagebreak
\begin{fixedpage}%431
\rectoheader

\settowidth{\versewidth}{And when you come, boys, with fife and drum, boys, }
\begin{dcverse}
\begin{altverse}
What though the wise make Alderman Isaac\\
Put us in prison and steal our estates,\\
Though we be forced to be unhorsed,\\
And walk on foot as it pleaseth the fates;\\
In the King’s army no man shall harm ye,\\
Then come along, boys, valiant and strong, boys,\\
Fight for your goods, which the Roundheads enjoy;\\
And when you venture London to enter, \\
And when you come, boys, with fife and drum, boys, \\
Isaac himself shall cry, \textit{Vive le Roy}. 
\end{altverse}

\begin{altverse}
If you will choose them, do not refuse them, \\
Since honest Parliament never made thieves,\\
Charles will not further have rogues dipt in murder,\\
Neither by leases, long lives, nor reprieves.\\
'Tis the conditions and propositions\\
Will not be granted, then be not daunted,\\
We will our honest old customs enjoy; \\
Paul’s not rejected, will be respected,\\
And in the Quier voices rise higher,\\
Thanks to the heavens and [cry] \textit{Vive le Roy}.
\end{altverse}
\end{dcverse}

\musictitle{LOVE LIES BLEEDING.}

This tune is referred to under the various names of \textit{Love lies
bleeeding}, \textit{Law lies bleeding}, \textit{The Cyclops}, \textit{The Sword}, and \textit{The power},
or \textit{The dominion, of the Sword}.

In \textit{The Loyal Garland}, fifth edition, 1686, is “The Dominion of
the Sword: A Song made in the Rebellion.” Commencing—

\settowidth{\versewidth}{Burn all your studies, and throw away your reading,” \&c.}
\begin{scverse}
\vleftofline{“}Lay by your pleading, Law lies a bleeding,\\
Burn all your studies, and throw away your reading,” \&c.
\end{scverse}

It is also in \textit{Loyal Songs}, i. 223, 1731 (there entitled “The
\textit{power} of the Sword”); in \textit{Merry Drollery complete}, 1661 and 1670; in
\textit{Pills to purge Melancholy}, vi. 190; \&c.

In the Bagford Collection, a song, “printed at the Hague, for S.
Browne, 1659,” is named “Chips of the old Block; or Hercules
cleansing the Augean Stable. To the tune of \textit{The Sword}.” It commences—

\settowidth{\versewidth}{And what chips from this treacherous block will come, you may conceive, sirs.” }
\begin{scverse}
“Now you, by your good leave, sirs, shall see the Rump can cleave, sirs.\\
And what chips from this treacherous block will come, you may conceive, sirs.” 
\end{scverse}
Other copies of this will he found in King’s
Pamphlets, vol. xvi.; in \textit{Rats rhymed to death}, 1660; and in \textit{Loyal
Songs}, ii. 53.

“\textit{Love lies a bleeding}; in imitation of \textit{Law lies a bleeding}” is
contained in \textit{Merry Drollery complete}, 1661 and 1670. There are also
copies in ballad form in which the tune is entitled \textit{The Cyclops}.

“A new Ignoramus: Being the second new song to the same old tune,
\textit{Law lies a bleeding},” was printed by Charles Leigh in 1681, and
included in \textit{Rome rhym’d to death}, 8vo., 1683. It commences—

\settowidth{\versewidth}{Sham plots are made as fast as pots are form’d by potters.”}
\begin{scverse}
\vleftofline{“}Since Popish plotters joined with hog-trotters,\\
Sham plots are made as fast as pots are form’d by potters.”
\end{scverse}

This is included in 180 \textit{Loyal Songs}, 1685 and 1694, with several
other political songs to the same tune. Among them, another
“Ignoramus,” beginning—

\settowidth{\versewidth}{“Since Reformation with Whigs is in fashion.”}
\begin{scverse}
“Since Reformation with Whigs is in fashion.”
\end{scverse}

The tune of \textit{Love lies bleeding} is contained in every edition of
\textit{The Dancing Master}, from and after 1686; in 180 \textit{Loyal Songs}, 1685 and
1694; in Walsh’s \textit{Dancing Master}; in \textit{Pills to purge Melancholy}; \&c.
\end{fixedpage}%431
\pagebreak


\begin{fixedpage}%432
\versoheader

In Shadwell’s \textit{Epsom Wells}, 1673, Clodpate sings “the old song,
\textit{Lay by your pleading, Law lies a bleeding};” and perhaps Whitlock had
the other song in his mind when he said, “Both truth and love lie a
bleeding.” (\textit{Zootomia, or Present Manners of the English}, 1654.)

The title of the ballad is “Love lies a bleeding:

\settowidth{\versewidth}{By whose mortal wounds you may soon understand,}
\begin{scverse}
By whose mortal wounds you may soon understand,\\
What sorrow we suffer since love left the land.
\end{scverse}

To the tune of \textit{The Cyclops}.”

\settowidth{\versewidth}{Though they seem to thrive at first, will make a sad conclusion,\&c.}
\begin{scverse}
When we love did nourish, England did flourish,\\
Till holy hate came in and made us all so currish;\\
Now every widgeon talks of religion,\\
But doth as little good as Mahomet and his pigeon.

Each coxcomb is suiting his words for confuting,\\
But heaven’s sooner gain’d by suff’ring than disputing;\\
True friendship we smother, and strike at our brother,\\
Apostles never went to God by killing one another.

He that doth knew me, and love will shew me,\\
Finds the nearest and the noblest way to overcome me;\\
He that hath bound me, or that doth wound me,\\
Winneth not my heart, he doth but conquer, not confound me.

In such condition, love is physician,\\
True love and reason make the purest politician;\\
Eut strife and confusion, deceit and delusion,\\
Though they seem to thrive at first, will make a sad conclusion,\&c.
\end{scverse}


\end{fixedpage}%432
\pagebreak

\begin{fixedpage}%433
\rectoheader

\musictitle{PRINCE RUPERT’S MARCH.}

This is contained in the first and subsequent editions of \textit{The
Dancing Master}; in Elizabeth Rogers’ MS. Virginal Book; in \textit{Gesangh
der Zeeden}, 12mo., Amsterdam, 1648; \&c.

Prince Rupert commanded the Royalists at the battle of Edgehill,
in 1642. He died and was interred with great magnificence in Henry
the Seventh’s Chapel, Westminster Abbey, in 1682. He was a nephew of
Charles I., and the discoverer of mezzotinto, the hint of which he is
said to have taken from seeing a soldier scraping his rusty musket.
The first mezzotinto print ever published was the work of his hands,
and may be seen in the first edition of Evelyn’s \textit{Sculptura}.

The commencement of this march resembles \textit{The British Grenadiers},
but is in a minor instead of a major key. In \textit{Gesangh der Zeeden},
there are words adapted to it; but I have not found any English
ballad name. As “The Lawyers’ Lamentation for the loss of Charing
Cross” (\textit{Loyal Songs}, i. 247) suits the measure, I have adapted the
words to the tune.
\end{fixedpage}%433
\pagebreak

\begin{fixedpage}%434
\versoheader

\settowidth{\versewidth}{Some letters about this Cross were found,}
\begin{dcverse}
\begin{altverse}
When at the bottom of the Strand, \\
They all are at a loss;\\
“That’s not the way to Westminster, \\
We must go by Charing Cross.”\\
Then fare thee well, \&c.
\end{altverse}

\begin{altverse}
The Parliament did vote it down,\\
A thing they thought most fitting, \\
For fear its fall should kill them all, \\
In the House as they were sitting. \\
Then fare thee well, \&c.
\end{altverse}

\begin{altverse}
Some letters about this Cross were found,\\
Or else it had been freed;\\
But I’ll declare, and even swear,\\
It could not write nor read.\\
Then fare thee well, \&c.
\end{altverse}

\begin{altverse}
The Whigs they do affirm and say\\
To Popery it was bent;\\
For aught I know, it may he so,\\
For to church it never went.\\
Then fare you well, \&c.
\end{altverse}

\begin{altverse}
The lawless Rump—rebellious crew—\\
They were so damn’d hard-hearted,\\
They pass’d a vote that Charing Cross\\
Should be taken down and carted.\\
Then fare thee well, \&c.
\end{altverse}

\begin{altverse}
Now, Whigs, I will advise you all\\
What I would have you do:\\
For fear the King should come again,\\
Pray pull down Tyburn, too!\\
Then fare thee well, \&c.
\end{altverse}
\end{dcverse}

A different version of the above song will be found in \textit{Pills to
purge Melancholy}, entitled “A Song made on the Downfall or pulling
down of Charing Cross, An. Dom. 1642” (a wrong date,—it should be
1647); and in Percy’s \textit{Reliques of Ancient Poetry}. The music in the
\textit{Pills} is not a popular tuue, but a composition by Mr. Farmeloe.

\musictitle{WHEN THE KING ENJOYS HIS OWN AGAIN.}

This tune is in Elizabeth Rogers’ Virginal Book (Add. MSS.,
10,337, Brit. Mus.); in \textit{Musick’s Recreation on the Lyra Viol}, 1652;
in \textit{Musick’s Delight on the Cithren}, 1666; in \textit{A Choice Collection of}
180 \textit{Loyal Songs}, 1685 and 1694; and in the third volume of \textit{The
Dancing Master}, n.d.

The words are ascertained to be Martin Parker’s, by the following
extract from \textit{The Gossips’ Feast}; or \textit{Morall Tales}, 1647:—“The gossips
were well pleased with the contents of this ancient ballad, and
Gammer Gowty-legs replied, ‘By my faith, Martin Parker never got a
fairer brat: no, not when he penn’d that sweet ballad, \textit{When the King
injoyes his own again}.’” In \textit{The Poet’s Blind Man’s Bough}, 1641,
Martin Parker says—

\settowidth{\versewidth}{Was known by ‘Martin Parker,’ or ‘M. P.;}
\begin{scverse}
\vleftofline{“}Whatever yet was published by me,\\
Was known by ‘Martin Parker,’ or ‘M. P.;’” 
\end{scverse}
but this song was
printed at a time when it would have been dangerous to give either
his own name or that of the publisher. Ritson calls this “the most
famous and popular air ever heard of in this country.” Invented to
support the declining interest of Charles I., “it served afterwards,”
he says, “with more success, to keep up the spirits of the Cavaliers,
and promote the restoration of his son,—an event it was employed to
celebrate all over the kingdom. At the Revolution” [of 1688] “it of
course became an adherent of the exiled family, whose cause it never
deserted. And as a tune is said to have been a principal mean of
depriving King James of the crown,” [see \textit{Lilliburlero}] “this very
air, upon two memorable occasions, was very near being equally
instrumental in replacing it on the head of his son. It is believed
to be a fact, that nothing fed the enthusiasm of the Jacobites, down
almost to the present reign, in every corner of Great Britain, more
than \textit{The King shall enjoy his own}
\end{fixedpage}%434
\pagebreak

\begin{fixedpage}%435
\rectoheader

\textit{again}; and even the great orator of the party, in that celebrated
harangue (which furnished the present laureat with the subject of one
of his happiest and finest poems), was always thought to have alluded
to it in his remarkable quotation from Virgil—‘Carmina tum melius
cum venerit ipse canemus!’”

Martin Parker probably wrote his song to the tune of \textit{Marry me,
marry me, quoth the bonny lass}, for the air is to be found under that
name in the Skene Manuscript (time of Charles I.); and the song was
evidently one familiar at the time. The following lines are quoted in
Brome’s play, \textit{The Northern Lass}, act iv., sc. 4 (4to., 1632):—

\settowidth{\versewidth}{As for thy wedding, lass, we’ll do well enough}
\begin{scverse}
\begin{altverse}
\vleftofline{“\textit{Constance}.} Marry me, marry me, quoth the bonny lass,\\
And when will you begin?\\
\vleftofline{\textit{Widow.}\quad\;\;} As for thy wedding, lass, we’ll do well enough,\\
In spight o’ the best of thy kin.”
\end{altverse}
\end{scverse}

In the third volume of \textit{The Dancing Master}, the tune is entitled
\textit{The Restoration of King Charles}.

The words of \textit{When the King enjoys his own again}, are in the
Roxburghe Collection of Ballads, iii. 256; in Mr. Payne Collier’s
Collection; in \textit{The Loyal Garland, containing Choice Songs and
Sonnets of our late Revolution}, London, 1671, and fifth edit., 1686
(Reprinted by the Percy Society); in \textit{A Collection of Loyal Songs},
1750; in Ritson’s \textit{Ancient Songs}; \&c.

Among the almost numberless songs and ballads that were sung to
the tune, I will only cite the following:—

1. “The World turn’d upside down,” 1646. King’s Pamphlets, No. 4,
fol.

2. “A new ballad called A Review of the Rebellion, in three
parts. To the tune of \textit{When the King enjoyes his rights againe},” dated
June 15, 1647. See King’s Pamphlets, vol. v., fol., and Wright’s
\textit{Political Ballads}, p. 13.

3. “The last news from France; being a true relation of the
escape of the King of Scots from Worcester to London, and from London
to France; who was conveyed away by a young gentleman in woman’s
apparel; the King of Scots attending on this supposed gentlewoman in
manner of a serving-man. The tune is \textit{When the King injoyes, \&c}.”
Printed by W. Thackeray, T.~Passenger, and W. Whitwood. Rox.
Collection, iii. 54. It commences thus:—

\indentpattern{0101220220}
\settowidth{\versewidth}{And the King himself did wait on me.}
\begin{dcverse}
\begin{patverse}
“All you that do desire to know \\
What is become of the King of Scots,\\
I unto you will truly show,\\
After the flight of Northern rats.\\
’Twas I did convey \\
His Highness away.\\
And from all dangers set him free,\\
In woman’s attire,\\
As reason did require,\\
And the King himself did wait on me.”
\end{patverse}
\end{dcverse}

4. “The Glory of these Nations; Or King and People’s Happiness:
Being a brief relation of King Charles’s royall progresse from Dover
to London, how the Lord Generall and the Lord Mayor, with all the
nobility and gentry of the land, brought him thorow the famous city
of London to his Pallace at Westminster, the 29 of May last, being
his Majesties birth-day, to the great comfort of his loyall subjects.
The tune is \textit{When the .King enjoys his own again}.” This is one of six
ballads of the time of Charles II., found in the lining of an old

\end{fixedpage}%435
\pagebreak

\begin{fixedpage}%436
\versoheader

trunk, and now in the British Museum. Also reprinted in Wright’s
\textit{Political Ballads}, p. 223.

5. “A Countrey Song, intituled The Restoration, May, 1661. King’s
Pamph., vol. xx., fol.; and Wright’s \textit{Political Ballads}, p. 265.
Commencing—

\settowidth{\versewidth}{That the King enjoyes his own again.”}
\indentpattern{020220220220}
\begin{dcverse}
\begin{patverse}
\vin\vin\vleftofline{“}Come, come away,\\
To the temple and pray,\\
And sing with a pleasant strain;\\
The schismatick’s dead,\\
The Liturgy’s read,\\
And the King enjoyes his own again.\\
The vicar is glad,\\
The clerk is not sad,\\
And the parish cannot refrain\\
To leap and rejoice,\\
And lift up their voice,\\
That the King enjoyes his own again.”
\end{patverse}
\end{dcverse}

6. “The Jubilee; or The Coronation Day,” from Thomas Jordan’s
\textit{Boyd Arbor of Loyal Poesie}, 12mo., 1664. As this consists of only two
stanzas, and the copy of the book, which is now in the possession of
Mr. Payne Collier; is probably unique, they are here subjoined:—

\settowidth{\versewidth}{Let every man with tongue and pen}
\indentpattern{0000220110}
\begin{dcverse}
\begin{patverse}
\vleftofline{“}Let every man with tongue and pen\\
Rejoice that Charles is come agen,\\
To gain his sceptre and his throne,\\
And give to every man his own:\\
Let all men that be.\\
Together agree,\\
And freely now express their joy:\\
Let your sweetest voices bring\\
Pleasant songs unto the King,\\
To crown his Coronation day.
\end{patverse}

\begin{patverse}
All that do tread on English earth\\
Shall live in freedom, peace, and mirth;\\
The golden times are come that we\\
Did one day think we ne’er should see:\\
Protector and Rump\\
Did put us in a dump,\\
When they their colours did display;\\
But the time is come about,\\
We are in, and they are out.\\
By King Charles his Coronation day.”
\end{patverse}
\end{dcverse}

7. “The Loyal Subject’s Exultation for the Coronation of King
Charles the Second.” Printed for F. Grove, Snow Hill.

8. “Monarchy triumphant; or, The fatal fall of Rebels,” from 120
\textit{Loyal Songs}, 1684; or 180 \textit{Loyal Songs}, 1685 and 1694. Commencing—

\settowidth{\versewidth}{And the health goes briskly all day round.”}
\begin{dcverse}
\begin{altverse}
“Whigs are now such precious things, \\
We see there’s not one to he found; \\
All roar, ‘God bless and save the King,’\\
And the health goes briskly all day round.”
\end{altverse}
\end{dcverse}

In Dr. Dibdin’s \textit{Decameron}, vol. iii., a song called “The King
enjoys his right,” is stated to be in the folio MS., which belonged
to Dr. Percy.

Ritson mentions another, of which he could only recollect that
the concluding lines of each stanza, as sung by “an old blind
North-country crowder,” were—

\settowidth{\versewidth}{When the King did enjoy his own again.”}
\indentpattern{0220}
\begin{dcverse}
\begin{patverse}
\vleftofline{“}Away with this cursed Rebellion!\\
Oh! the 29th of May,\\
It was a happy day,\\
When the King did enjoy his own again.”
\end{patverse}
\end{dcverse}

In the novel of \textit{Woodstock}, Sir Walter Scott puts the last three
lines into the mouth of Wildrake, who is represented as perpetually
singing, “The King shall enjoy his own again.”

It was not used exclusively as a Jacobite air, for many songs are
extant which were written to it in support of the House of Hanover;
such as—

1. “An excellent new ballad, call’d Illustrious George shall
come,” in \textit{A Pill to purge State Melancholy}, voi i., 3rd. edit., 1716.

2. “Since Hanover is come: a new song.” And—

3. “A song for the 28th of May, the birth-day of our glorious
Sovereign,
\end{fixedpage}%436
\pagebreak

\begin{fixedpage}%437
\rectoheader

King George,” in \textit{A Collection of State Songs, Poems, \&c., that
have been published since the Rebellion, and sung at the several
Mug-houses in the cities of London and Westminster}, 1716.

The copy of the ballad in Mr. Payne Collier’s Collection is
entitled “The King enjoys his own again. To be joyfully sung with its
own proper sweet tune.” The burthen of that, and of the Roxburghe
copy, is “When the King \textit{comes home in peace} again,” instead of
“enjoys his own again,” as in \textit{The Loyal Garland}. Neither of the
ballads has any date or publisher’s name; and therefore both were, in
all probability, privately printed during the civil war. The
Roxburghe copy has “God save the King, Amen,” in large letters at the
end.

\settowidth{\versewidth}{There’s neither Swallow, Dove, nor Dade,\textsuperscript{a}}
\begin{dcverse}
There’s neither Swallow, Dove, nor Dade,\textsuperscript{a}\\
Can soar more high, or deeper wade;\\
Nor shew a reason from the stars,\\
What causeth peace or civil wars:\\
The man in the moon may wear out his shoon,\\
By running after Charles his wain:\\
But all’s to no end, for the times will not mend\\
Till the King, \&c.
\end{dcverse}

\begin{dcfootnote}
\textsuperscript{a} Booker, Pood, Rivers, Swallow, Dove, Dade, and Hammond, whose
names are mentioned in the ballad, were all astrologers and
almanack-makers. Ritson copies his notes about Booker and others from
a small pamphlet
printed in 1711, entitled “The ballad of \textit{The King shall enjoy his
own again}; with a learned comment thereupon.” The account there given
of Booker does not agree with that of William Lilly, quoted in a note
to Dodsley's \textit{Old}
\end{dcfootnote}

\end{fixedpage}%437
\pagebreak
\setlength{\fixedpagewidth}{392pt}
\begin{fixedpage}%438
\versoheader

\settowidth{\versewidth}{Then let’s hope for a peace, for the wars will not cease}
\begin{dcverse}\footnotesizerr
Though for a time we see Whitehall\\
With cobwebs hanging on the wall,\\
Instead of silk and silver brave.\\
Which formerly it us’d to have,\\
With rich perfume in every room,\\
Delightful to that princely train, \\
Which again you shall see, when the time it shall be\\
That the King, \&c.

Full forty years the royal crown\\
Hath been his father’s and his own;\textsuperscript{b} \\
And is there any one but he\\
That in the same should sharer be?\\
For who better may the sceptre sway\\
Than he that hath such right to reign?\\
Then let’s hope for a peace, for the wars will not cease\\
Till the King, \&c.

[Did Walker\textsuperscript{c} no predictions lack\\
In Hammond’s bloody almanack?\\
Foretelling things that would ensue.\\
That all proves right, if lies be true;\\
But why should not he the pillory foresee,\\
Wherein poor Toby once was ta’en?\\
And also foreknow to the gallows he must go.\\
When the King, \&c.\textsuperscript{d}]

Till then upon Ararat’s hill\\
My Hope shall cast her anchor still,\\
Until I see some peaceful dove\\
Bring home the branch I dearly love;\\
Then will I wait till the waters abate,\\
Which now disturb my troubled brain,\\
Else never rejoice till I hear the voice,\\
That the King enjoys his own again.
\end{dcverse}

The following stanzas are not contained in \textit{The Loyal Garland},
from which

Ritson reprinted the song:—

\settowidth{\versewidth}{The which you shall see, when the time it shall be}
\begin{dcverse}\footnotesizerr
Oxford and Cambridge shall agree\\
With honour crown’d, and dignity;\\
For learned men shall then take place,\\
And bad be silenc’d with disgrace:\\
They’ll know it to be but a casualty\\
That hath so long disturb’d their brain;\\
For I can surely tell that all things will go well\\
When the King comes home in peace again.

Church Government shall settled be,\\
And then I hope we shall agree\\
Without their help, whose high-brain’d zeal\\
Hath long disturb’d the common weal;\\
Greed out of date, and cobblers that do prate\\
Of wars that still disturb their brain;\\
The which you shall see, when the time it shall be\\
That the King comes home in peace again.

Tho’ many now are much in debt,\\
And many shops are to be let,\\
A golden time is drawing near,\\
Men shops shall take to hold their ware;\\
And then all our trade shall flourishing be made,\\
To which ere long we shall attain;\\
For still I can tell all things will be well,\\
When the King comes home in peace again.

Maidens shall enjoy their mates,\\
And honest men their lost estates;\\
Women shall have what they do lack,\\
Their husbands, who are coming back.\\
When the wars have an end, then I and my friend\\
All subjects’ freedom shall obtain;\\
By which I can tell all things will be well,\\
When we enjoy sweet peace again.

Though people now walk in great fear \\
Along the country everywhere.\\
Thieves shall then tremble at the law.\\
And justice shall keep them in awe:\\
The Frenchies shall flee with their treacherie,\\
And the foes of the King asham’d remain:\\
The which you shall see, when the time it shall be\\
That the King comes home in peace again.
\end{dcverse}

\begin{dcfootnote}\scriptsizerrr
\noindent\textit{Plays}, vol. xi., p. 469. Booker is mentioned by Killegrew, in \textit{The
Parson's Wedding}, act i., sc. 2; by Pepys. in his Diary, Feb. 3,
1666-7; by Cleveland, in his \textit{Dialogue between two Zealots}; and by
Butler, in \textit{Hudibras}. One of his almanacks for 1661 was sold in
Skegg’s sale. Pond’s almanack is mentioned in Middleton's play, \textit{No
wit, no help like a woman’s}; and the Rev. A. Dyce, in a note upon the
passage, quotes the title of one by Pond, for the year 1607. An
almanack for the year 1636,“by William Dade, gent., London, printed
by M. Dawson, for the Company of Stationers,” was once in my
possession. According to the pamphlet which Ritson quotes, Dade was
“a good innocent fiddle-string maker, who, being told by a
neighbouring teacher that their music was in the stars, set himself
at work to And out their habitations, that be might be
instrument-maker to them; and having, with much ado, got
knowledge of their place of abode, was judged by the Roundheads fit
for their purpose, and had a pension assigned him to make the \textit{stars}
speak their meaning, and justify the villanies they were putting in
practice.” Hammond's almanack was called “bloody,” because he always
put down in a chronological table when such and such a Royalist was
executed, by way of reproach to them.

\textsuperscript{b} This fixes the date of the song to the year 1643. The number
was changed from time to time, as it suited the circumstances of the
party.

\textsuperscript{c} Walker was a colonel in the army of tha Parliament, and
afterwards a member of the Committee of Safety.

\textsuperscript{d} This stanza is not in the ballad copies.

\end{dcfootnote}
\end{fixedpage}%438
\pagebreak

\setlength{\fixedpagewidth}{360pt}
\begin{fixedpage}%439
\rectoheader

\settowidth{\versewidth}{For some will gladly spend their lives to defend}
\begin{dcverse}
The parliament must willing be \\
That all the world may plainly see \\
How they will labour still for peace, \\
That all these bloody wars may cease. \\
For some will gladly spend their lives to defend \\
The King in all his right to reign; \\
So then I can tell all things will go well, \\
When we enjoy sweet peace again. 

When all these things to pass shall come.\\
Then farewell musket, pike, and drum:\\
The lamb shall with the lion feed,\\
Which were a happy time indeed.\\
O let us all pray we may see the day \\
That peace may govern in his name:\\
For then I can tell all things will be well\\
When the King comes home in peace again.
\end{dcverse}

\musictitle{BY THE BORDER'S SIDE AS I DID PASS.}

A border-song, entitled “Ballad on a Scottish Courtship,” from
Ashmolean” MSS., Nos. 36 and 37, Article 128. The tune is, in
character, like \textit{Cavalilly man}.

Ashmole held a captain’s commission under Charles I., in the
civil war, and probably noted it down from hearing it sung.

The song consists of forty lines, but I did not transcribe
further.

\musictitle{FAIN I WOULD IF I COULD.}

In \textit{The Dancing Master}, from 1650 to 1665, this is entitled \textit{Fain I
would if I could}; and in the editions from 1670 to 1690 (with a
trifling difference), \textit{Parthenia}, or \textit{Fain I would}. In Elizabeth
Rogers’ MS. Virginal Book, the same air is called \textit{The King's
Complaint}.

One of the ballads-among the King’s Pamphlets, which bears the
date of the 23rd April, 1649, is “A Coffin for King Charles: A Crown
for Cromwell: A Pit for the People;” and the direction is that “you
may sing this to the tune of \textit{Fain I would}”(vol. viii., fol., and
reprinted in Wright’s \textit{Political Ballads}, 8vo., p. 117). It is a
dialogue between Cromwell on the throne, King Charles in his coffin,
and the people in the pit. The date proves it to have been printed
within three months after the King’s execution. It consists of
fifteen stanzas, of which three are subjoined. The first is—

\end{fixedpage}%439
\pagebreak


\begin{fixedpage}%440
\versoheader

\settowidth{\versewidth}{Think’st thou base slave, though in my grave,}
\begin{dcverse}
\begin{altverse}
\quad \quad\textsc{King Charles in his Coffin. }

Think’st thou base slave, though in my grave, \\
Like other men I lie?\\
My sparkling fame and royal name \\
Can, as thou wishest, die?\\
Know, caitiff, in my son I live\\
(The Black Prince call’d by some),\\
And he shall ample vengeance give\\
On those that did me doom.
\end{altverse}

\begin{altverse}
\quad\textsc{The People in the Pit.}

Suppress’d, depress’d, involv’d in woes,\\
Great Charles, thy people be,\\
Basely deceiv’d with specious shows \\
By those that murther’d thee.\\
We are enslav’d to tyrants’ hests,\\
Who have our freedom won:\\
Our fainting hope now only rests \\
On thy succeeding son, \&c.
\end{altverse}
\end{dcverse}

\musictitle{CAVALILLY MAN.}

This tune is contained in \textit{The Dancing Master} of 1670, and in
every subsequent edition; in 180 \textit{Loyal Songs}, 1685 and 1694; in \textit{Pills
to purge Melancholy} (ii. 18, and iii. 65, 1707); in \textit{The Village Opera},
and other ballad-operas.

A copy of the ballad from which the tune derives its name is in
Mr. Halliwell’s Collection, and the first stanza is here printed to
the tune. “Cavalilly” means “Cavalier.”
\end{fixedpage}%440
\pagebreak


\begin{fixedpage}%441
\rectoheader

In Harl. MSS., No. 6,913, is a satirical song by Lord Rochester,
to this tune; commencing— 

\settowidth{\versewidth}{As a mask of his valour, to Tangier he went,” \&c}
\begin{scverse}
\begin{altverse}
“Have you heard of a Lord of noble descent,\\
Hark! how the bells of Paradise ring;\\
As a mask of his valour, to Tangier he went,” \&c.
\end{altverse}
\end{scverse}

In 120 \textit{Loyal Songs}, 1684, are the following:—

P. 196. “A new Litany to be sung in all Conventicles, for
instruction of the Whigs. Tune, \textit{Cavalilly man}.” Commencing—

\settowidth{\versewidth}{“Prom councils of six, when treason prevails.”}
\begin{scverse}
“From councils of six, when treason prevails.”
\end{scverse}

P. 213. “A song of The Light of the nation turn’d into darkness.
Tune called \textit{Cavalilly man}” Commencing—

\settowidth{\versewidth}{“Come, all you caballers and parliament votes.”}
\begin{scverse}
“Come, all you caballers and parliament votes.”
\end{scverse}

In the editions of 1685 and 1694 are several other songs, and the
tune is, in one instance, entitled \textit{Which nobody can deny}. The song is
on Titus Oates. “Oates well thrashed; being a dialogue between a
country farmer and his man, Jack.” The first stanza, and one other,
end with the line, “Which nobody can deny, sir;” from which, I
assume, the name is (improperly) given to the tune.

The original ballad is entitled “The North-country Maid’s
Resolution, and Love to her Sweetheart: To \textit{a pleasant new Northern
tune}.” “Printed for F. Grove on Snow hill.” It consists of eleven
stanzas of eight lines, besides the following burthen of four, to
each verse:—

\settowidth{\versewidth}{I prithee le’ me gang with thee, man.”}
\begin{dcverse}
\vleftofline{“}O my dainty Cavalilly man, \\
My finnikin Cavalilly man, \\
For God’s cause and the Protestants’,\\
I prithee le’ me gang with thee, man.”
\end{dcverse}

I imagine that there must have been longer versions of the tune
than any I have found, because, if only consisting of eight bars, it
would be necessary to sing these three times over for every stanza,
including the burthen.

\end{fixedpage}%441
\pagebreak


\begin{fixedpage}%442
\versoheader

\musictitle{THE GLORY OF THE NORTH.}

This tune is contained in Elizabeth Rogers’ MS. Virginal Book;
in Hawkins’ Transcripts of Virginal Music; in \textit{Musick's Recreation on
the Lyra-viol}; in \textit{Musick’s Delight on the Cithren}; and is among the
violin tunes at the end of \textit{The Dancing Master} of 1665,

\musictitle{THE DEVIL’S PROGRESS.}

In the Collection of \textit{Loyal Songs written against the Rump
Parliament}, i. 50, is “The Sense of the House; or the reason why
those Members who are the remnant of the two families of Parliament
cannot consent to Peace, or an Accommodation. To the tune of \textit{The
New-England Psalm, Huggle-duggle, ho, ho, ho-, the Revil he laugh’d
aloud}.” It begins—

\settowidth{\versewidth}{There’s thirty swear they’ll have no peace, and bid me tell you why.”}
\begin{scverse}
\vleftofline{“}Come, come, beloved Londoners, fie, fie, you shame us all!\\
Your rising up for peace will make the close Committee fall:\\
I wonder you dare ask for that, whieh they must needs deny,—\\
There’s thirty swear they’ll have no peace, and bid me tell you why.”
\end{scverse}

The ballad of \textit{The Devil’s Progress on Earth, or Huggle-duggle}
(which is thus proved to be as old as the time of Charles I.), is
contained in \textit{Pills to purge Melancholy}, vol. i., 1699 and 1707; or
vol. iii, 1719, with the tune. The words of the first stanza are very
imperfectly printed in all editions. Three or four words have here
been added or altered from conjecture. “Airing” stands “Airidg,” in
the \textit{Pills}; the word after “Pluto” is deficient; “And many a
\end{fixedpage}%442
\pagebreak


\begin{fixedpage}%443
\rectoheader

\noindent goblin more” is here changed to “\textit{O’er} many a gobling \textit{crew},”
because a rhyme is required to “too.”

It was no doubt this ballad which suggested to Southey his
Devil’s Walk.

\settowidth{\versewidth}{Their scales were false, their weights were light,}
\begin{dcverse}
\begin{altverse}
Why think you that he laugh’d?\\
Forsooth he came from court;\\
And there, amongst the gallants,\\
Had spied such pretty sport:\\
There was such cunning juggling,\\
And ladies grown so proud—\\
Huggle, duggle, \&c.

With that into the City \\
Away the devil went,\\
To view the merchants’ dealings \\
It was his full intent;\\
And there, along the brave Exchange,\\
He crept into the crowd—\\
Huggle, duggle, \&c.

He went into the City,\\
To see all there was well;\\
Their scales were false, their weights were light,\\
Their conscience fit for hell;\\
And ‘bad men’ chosen Magistrates, \\
And Puritans allow’d—\\
Huggle, duggle, \&c.

With that into the country \\
Away the devil goeth,\\
For there is all plain dealing.\\
And that the devil knoweth:\\
But the rich man reaps the gains,\\
For which the poor man plough'd--\\
Huggle, duggle, \&c.

With that the devil in haste,\\
Took post away to hell,\\
And told his fellow furies \\
That all on earth was well;\\
That falsehood there did flourish, \\
Plain-dealing was in a cloud—\\
Huggle, duggle, ha, ha, ha,\\
The devils laugh’d aloud.
\end{altverse}
\end{dcverse}

\end{fixedpage}%443
\pagebreak


\begin{fixedpage}%444
\versoheader

\musictitle{THE GLORY OF THE WEST.}

This is contained in \textit{The Dancing Master} from 1650 to 1686; in
\textit{Musick’s Delight on the Cithren}, 1666; and in \textit{Musick’s Handmaid},
1678.

In a copy of \textit{The Dancing Master} for 1665 (now in my possession),
“Shall I, mother, shall I,” is written under \textit{The Glory of the West}, as
another name for the tune. I have not succeeded in finding the words
of either.

In the Bagford Collection, and in the \textit{Collection of Loyal Songs},
is “The Glory of the West; or the tenth renowned worthy and most
heroic Champion of the British Islands: Being an unparalleled
Commemoration of General Monk’s coming towards the city of London;”
but this cannot have been written in 1650, and the words do not suit
the measure of the tune. Nor does a later ballad, “The Glory of the
West; or the Virgins of Taunton-Dean, who ript open their silk
petticoats to make colours for the late D[uke] of M[onmouth]’s army,
when he came before the town.” The tune of that was the The
Winchester Wedding.


\musictitle{NONESUCH.}

Two copies of this tune are contained in \textit{The Dancing Master} of
1650; the first as \textit{Nonesuch}, the second as \textit{A la mode de Trance}. The
second name is derived from the ditty of a song which is here printed
to the air.

\textit{A la mode de France} is to be found in every edition of \textit{The
Dancing Master} (sometimes in a major key and sometimes in a minor);
in \textit{Musick’s Recreation on the Lyra Viol}, 1661; \textit{Musick’s Delight on the
Cithren}, 1666; and \textit{Youth’s Delight on the Flagelet}, 1697.

In \textit{A short History of the English Rebellion}, by Marchamont
Needham, printed

\end{fixedpage}%444
\pagebreak

\begin{fixedpage}%445
\rectoheader

in 1661, but written while Charles I. was in prison, the author
twice quotes the burden, and perhaps wrote the whole poem to the
tune. The metre is quite suitable, as will be shown by the following
stanzas, 93 and 97:—

\settowidth{\versewidth}{So \textit{we} may feast, let prince and queen}
\begin{dcverse}
\begin{altverse}
\vleftofline{“}Never such rebels have been seen \\
As since we led this dance; \\
So \textit{we} may feast, let prince and queen\\
Beg, \textit{à la mode de France}, \&c. 
\end{altverse}

\begin{altverse}
Then let us what our labours gain\\
Enjoy, and bless our chance:\\
Like kings let’s domineer and reign,\\
Thus, \textit{à la mode de France}’'
\end{altverse}
\end{dcverse}


In \textit{The Second Tale of a Tub}, 8vo., 1715, one of the tunes called
for by the company is \textit{A la mode de France}. In the \textit{Collection of Loyal
Songs}, i. 25, 1731, the song is entitled “The French Report.”

\begin{dcverse}
\begin{altverse}
A vise man dere is like a ship \\
Dat strike upon de shelves,\\
Dey prison all, behead, and vip \\
All viser dan demselves;\\
Dey send out men to fetch deyr king, \\
Who may come home, perchance: \\
O fy, fy, fy, it is, be gar,\\
Not \textit{à la mode de France}.
\end{altverse}

\begin{altverse}
Dey raise deyr valiant prentices \\
To guard deyr cause vith clubs;\\
Dey turn deyr Bishops out of doors, \\
And preash demselves in tubs: \\
De cobler and de tinker, too,\\
Dey vili in time advance;\\
Gar take dem all, it is (mort Dieu) \\
Not \textit{à la mode de France}.
\end{altverse}

\begin{altverse}
Instead of bowing to deyr king,\\
Dey vex him vith epistles;\\
Dey furnish all deyr souldiers out \\
Vith bodkins, spoons, and vhistles;\\
Dey bring deyr gold and silver in,\\
De Brownists to advance,\\
And if dey be cheat of it all,\\
’Tis \textit{à la mode de France}.
\end{altverse}

\begin{altverse}
But if ven all deyr vealth be gone, \\
Dey turn unto deyr king,\\
Dey vill all make amends again, \\
Den merrily ve vill sing,\\
\textit{Vive le Roy, vive le Roy},\\
Ve’ll sing, carouse, and dance,\\
De English men have done \textit{fort bon}, \\
And \textit{à la mode de France}.
\end{altverse}
\end{dcverse}

\end{fixedpage}%445
\pagebreak

\begin{fixedpage}%446
\versoheader

\musictitle{MY FATHER WAS BORN BEFORE ME.}


In the fourth, and all subsequent editions of \textit{The Dancing Master},
this tune is entitled \textit{Jamaica}. The island of Jamaica was taken from
the Spaniards in 1655, and the tune probably took the name from some
song on that event.

The following were sung to it:—

1. “The Prodigal’s Resolution; or, My Father was born before me”
(\textit{Pills to purge Melancholy}, vol. i., 1699 and 1707). This is taken
from Thomas Jordan’s \textit{London Triumphant}, 4to., 1672. Jordan was the
“professed pageant-writer and poet laureat for the City, and if
author of this song,” says Ritson, who includes it in his Ancient
Songs, “he seems to have possessed a greater share of poetical merit
than usually fell to the lot of his profession.” It begins with the
line, “I am a lusty, lively lad,” which was probably suggested by,
and the tune taken from, an earlier song, beginning—

\settowidth{\versewidth}{Heigh for a lad that’s seldom sad,}
\begin{dcverse}
\begin{altverse}
“Heigh for a lusty, lively lad; \\
Heigh for a lad lacks kissing; \\
Heigh for a lad that’s seldom sad,\\
But when”—
\end{altverse}
\end{dcverse}

These lines are from a medley of songs at p. 30 of \textit{Sportive Wit:
The Muses’ Merriment}, 8vo., 1656. I have not seen it complete, and it
breaks off at the words, “But when,” into another song.

2. “Two Toms and Nat in council sat. To the tune of \textit{Jamaica}.”
(\textit{State Poems}, continued, p. 140, 1697.)

4. “Slow men of London; or The Widow Brown” \textit{Pills}, vi. 93). ,
This is a song of three Londoners being outwited by a Welshman, in a
competition for the Widow Brown. It consists of twelve stanzas, and
commences thus:—

\settowidth{\versewidth}{There were three young men of this town,}
\begin{dcverse}
\begin{altverse}
\vleftofline{“}There dwelt a widow in this town \\
That was both fair and lovely;\\
Her face was comely, neat and brown:\\
To pleasure she would move thee.\\
Her lovely tresses shone like gold,\\
Most neat was her behaviour;
\end{altverse}

\begin{altverse}
In truth it has of late been told\\
That many strove to have her.\\
There were three young men of this town,\\
Slow men of London,\\
And they’d go woo the Widow Brown,\\
Because they would be undone.”
\end{altverse}
\end{dcverse}

The last four lines form the subject of another song, which is
printed in Watts’ \textit{Musical Miscellany}, ii. 74, 1729. It consists of
only sixteen lines, and is said to have been sung in the play of \textit{Wit
without Money}; I suppose on the revival of Beaumont and Fletcher’s
play, about the year 1708, with alteration's and, as the title-page
modestly asserts, “with \textit{amendments}, by some persons of quality.” It
suggests the possibility of the longer song having been introduced in
1639 or 1661. There is a situation for one near the end of the play,
but (according to the Rev. A. Dyce) it is not printed either in the
quartos or in the folio.

Three other songs are printed to the tune in \textit{Pills to purge
Melancholy}, viz., “The Angler’s Song,” beginning, “Of all the
recreations,” iii. 126; “Of the Downfall of one part of the Mitre
Tavern in Cambridge, or the sinking thereof into the cellar,” iii.
136; and “The Jolly Tradesmen,” beginning, “Some time I am a tapster
new,” vi. 91. Others will be found in the ballad-operas of \textit{Polly},
1729; \textit{Love and Revenge}, n.d.; \&c.

“The Prodigal’s Resolution” consists of eleven stanzas, of which
three are subjoined.
\end{fixedpage}%446
\pagebreak

\begin{fixedpage}%447
\rectoheader

\settowidth{\versewidth}{He scrap’d and scratch’d, she pinch’d and patch’d}
\begin{dcverse}
\begin{altverse}
My father was a thrifty sir,\\
Till soul and body sundred:\\
Some say he was a usurer.\\
For thirty in the hundred:\\
He scrap’d and scratch’d, she pinch’d and patch’d\\
That in her body bore me;\\
But I’ll let fly,—good reason why,—\\
My father was born before me.\\
My daddy has his duty done,\\
In getting so much treasure;\\
I’ll be as dutiful a son,\\
For spending it at pleasure:\\
Five pound a quart shall cheer my heart, \\
Such nectar will restore me;\\
When ladies call, make love to all,—\\
My father was born before me, \&c.
\end{altverse}
\end{dcverse}

\musictitle{THE CLEAR CAVALIER.}

This is the “effusion of loyal enthusiasm” which Sir Walter Scott
puts into the mouth of the worthy cavalier, Sir Geoffrey Peveril, in
his novel, P\textit{everil of the Peak}. The same lines are quoted by Shadwell
in his \textit{Epsom, Wells}, where Fribble says to the fiddlers, “Can’t you
sing—

\settowidth{\versewidth}{Dub-a-dub-dub, have at old Beelzebub,}
\begin{scverse}
\vleftofline{‘}Hey for Cavaliers, ho for Cavaliers,\\
Dub-a-dub-dub, have at old Beelzebub,\\
Oliver quakes for fear.’”—Act v., sc. 1.
\end{scverse}

The song is attributed to Samuel Butler, author of \textit{Hudibras}, and
is printed in
\end{fixedpage}%447
\pagebreak


\begin{fixedpage}%448
\versoheader

his \textit{Posthumous Works}; also in \textit{Westminster Drollery}, part ii., p.
48, 1672; in \textit{Loyal Songs written against the Pump Parliament}, i. 249;
in \textit{Pills to purge Melancholy}; \&c.

The music is in a manuscript, once the property of Charles
Morgan, of Magdalen College, and bearing the date of 1682; in John
Banister’s \textit{Division Violin}, MS.; in \textit{Apollo's Banquet for the Treble
Violin}; and in the ballad-opera of \textit{Love in a Riddle}, 1729; \&c. It
was introduced, as “The Card Dance,” in Mrs. Behn’s farce, \textit{The
Emperor of the Moon}, 1687.

\end{fixedpage}%448
\pagebreak


\begin{fixedpage}%449
\rectoheader

\settowidth{\versewidth}{Fly, like light about, face to the right about, charge 'em home again, seize our own again:}
\begin{scverse}
Freely let’s be then honest men, and kick at Fate,\\
For we shall live to see our loyalty be valued at high rate;\\
He that bears a sword, or says a word against the throne—\\
That doth profanely prate against the state, no loyalty can own.\\
What though plumbers, painters, players, now be prosperous men,\\
Let us but mind our own affairs, and they’ll come round again.\\
Treach’ry may in face look bright, and lech’ry clothe in fur;\\
A traitor may be made a knight, ’tis \textit{fortune de la guerre}.\\
But what is that to us, boys, that are right honest men?\\
We’ll conquer and come again, beat up the drum again,—\\
Hey for Cavaliers, ho for Cavaliers, drink for Cavaliers, fight for Cavaliers, \\
Dub-a-dub, dub-a-dub, have at old Beelzebub, Oliver quakes for fear.\\
Fifth Monarchy must down, boys, and every sect in town;\\
We’ll rally and to’t again, give ’em the rout again;\\
Fly, like light about, face to the right about, charge 'em home again, seize our own again:\\
Tantara, rara, and this is the life of an honest, bold Cavalier.
\end{scverse}

\musictitle{OLD NOLL’S JIG.}

This does not appear in \textit{The Dancing Master} before the eleventh
edition (1701); but it is included in all later editions. The song,
“When once Master Love gets into your head,” was sung to it.

It is scarcely necessary to say that “Old Noll” was the nickname
given to Oliver Cromwell by the Cavaliers; just as “Tumble-down Dick”
was that of his son Richard, and “Rowley,” or “Old Rowley,” that of
Charles II.

Some wag named this jig after the Protector, for, although
Cromwell delighted in music, both vocal and instrumental, and skill
in the art was a sure passport to

\end{fixedpage}%449
\pagebreak

\begin{fixedpage}%450
\versoheader

his favour, he certainly was not addicted to dancing. His manner
of entertaining the Ambassadors of Holland, on the occasion of the
peace between the two Commonwealths, would now be thought somewhat
peculiar. After the repast, during which there was music as usual,
the Lord Protector took them “into another room, where the Lady
Protectrice and others came to us,” says the writer, “and there also
we had music and voices, and \textit{a psalm sung which his Highness gave
them}.” (Thurloe’s \textit{State Papers}. The letter dated April
12th, 1654.)

\musictitle{MY NAME IS OLD HEWSON THE COBBLER.}

This is one of the tunes introduced in the ballad-operas, The
\textit{Jovial Crew}, and \textit{The Grub Street Opera}, both printed in 1731.

\textit{The Jovial Crew} of 1731 was an alteration of Richard Brome’s
comedy of the same name.

The words of the song have not been recovered; but there appears
little doubt of their having been a political squib upon Colonel
Hewson, who was one of Charles the First’s judges, and of those who
signed his death-warrant.

John Hewson was originally a cobbler, and had but one eye. He
took up arms
\end{fixedpage}%450
\pagebreak

\begin{fixedpage}%451
\rectoheader

on the side of the Parliament, and being a man of courage and
resolution, soon rose to be a colonel in their army. He was knighted
by Cromwell, and afterwards made one of his Lords. He quitted England
immediately before the Restoration, and died at Amsterdam in 1662.

There are numerous allusions to his former trade, and to his one
eye, in the Cavalier songs. For instance, in “A Quarrel betwixt Tower
Hill and Tyburne” (to be found in \textit{Merry Drollery Complete}; \textit{Loyal
Songs}; \&c.)—

\settowidth{\versewidth}{There is single-eyed Hewson, the Cobbler of Fate,}
\begin{scverse}
\vleftofline{“}There is single-eyed Hewson, the Cobbler of Fate,\\
Translated into buff and feather;\\
But \textit{bootless} are all his \textit{seams} of state,\\
When the \textit{soul} is ript from the upper leather.”
\end{scverse}

Two complete songs about him are in the Bagford Collection (643,
m. 9, Brit. Mus.); and in \textit{Loyal Songs}, vol. ii.

The first, “A Hymn to the Gentle Craft; or, Hewson’s Lamentation.
To the tune of \textit{The Blind Beggar};” but the name of the tune is
intended as a joke upon his one eye; the words are not in a metre
that could be sung to it.

\settowidth{\versewidth}{Of a blind cobbler that’s gone astray,}
\indentpattern{0001}
\begin{scverse}
\begin{patverse}
\vleftofline{“}Listen awhile to what I shall say \\
Of a blind cobbler that’s gone astray,\\
Out of the Parliament’s highway;\\
Good people, pity the blind,”\&c.
\end{patverse}
\end{scverse}

The second is “The Cobbler’s Last Will and Testament; Or, the
Lord Hewson’s Translation:— 

\settowidth{\versewidth}{That they may learn their \textit{souls} to mend,}
\begin{scverse}
\vleftofline{“}To Christians all I greeting send,\\
That they may learn their \textit{souls} to mend,\\
By viewing of my cobbler’s \textit{end}” \&c.
\end{scverse}

The Rev. Mark Noble, in his \textit{Memoirs of the Protectorate House of
Cromwell}, i. 411, 8vo., 1784, quotes three stanzas of the above to
prove that Elizabeth, the lady of the Protector, had “a defect in one
eye;” but the allusion is most clearly to Hewson.

\musictitle{I LIVE NOT WHERE I LOVE.}

In the Roxburghe Collection, i. 68, is a ballad entitled “The
constant lover: 

\settowidth{\versewidth}{Though he live not where he love.}
\begin{scverse}
Who his affection will not move,\\
Though he live not where he love.
\end{scverse}

To a Northern tune, called \textit{Shall the absence of my Mistresse}.” It
is subscribed P. L., London, printed for Henry Gosson, and consists
of twelve stanzas, the first of which is as follows:—

\settowidth{\versewidth}{From your sweet-hearts many a mile,}
\begin{dcverse}
\begin{altverse}
\vleftofline{“}You loyal lovers that are distant\\
From your sweet-hearts many a mile, \\
Pray come help me at this instant\\
In mirth to spend away the while.
\end{altverse}

\begin{altverse}
In singing sweetly and completely\\
In commendation of my love;\\
Resolving ever to part never,\\
\textit{Though I live not where I love}.”
\end{altverse}
\end{dcverse}

\end{fixedpage}%451
\pagebreak

\begin{fixedpage}%452
\versoheader

In the same Collection, i. 320, is “A Paire of Turtle Doves, Or a
dainty new Scotch Dialogue between a yong man and his mistresse, both
correspondent in affection,” \&c. “To a pretty pleasant tune called
\textit{The absence of my Mistresse},
or, \textit{I live not where I love}.” It is subscribed “Martin Parker,”
Printed at London for Thomas Lambert at the Horse-shoe in West
Smithfield, and commences thus:—

\settowidth{\versewidth}{Must the ahsence of my mistresse.}
\begin{scverse}
\begin{altverse}
\vleftofline{\textsc{Yong Man}. “}Must the absence of my mistresse.\\
Gar me be thus discontent,\\
As thus to leave me in distresse,\\
And with languor to lament,” \&c.
\end{altverse}
\end{scverse}

In the Pepys Collection, iv. 40, is another ballad by P. L.,
called “The valiant Trooper and Pretty Peggy,” \&c. “To the tune of
\textit{Though I live not where
I love}” beginning:—

\settowidth{\versewidth}{That had his pockets well lin’d with gold,}
\begin{dcverse}
\begin{altverse}
“Heard you not of a valiant trooper\\
That had his pockets well lin’d with gold,\\
He was in love with a gallant lady,\\
As I to you shall here unfold.
\end{altverse}

\begin{altverse}
With a kind salute, and fierce dispute, \\
He thought to make her his only one;\\
But unconstant woman, true to no man,\\
Is gone and left her bird alone.”
\end{altverse}
\end{dcverse}

A ballad very much akin to the last is contained in \textit{Pills to
purge Melancholy},
iii. 156, 1707, entitled “The unconstant woman. To \textit{a new tune}.”
It begins:

\settowidth{\versewidth}{Unconstant woman proves true to no man,—}
\begin{dcverse}
\begin{altverse}
“Did you not hear of a gallant sailor, \\
Whose pockets they were lin’d with gold;\\
He fell in love with a pretty creature,\\
As I to you the truth unfold:
\end{altverse}

\begin{altverse}
With a kind salute, and without dispute,\\
He thought to gain her for his own:\\
Unconstant woman proves true to no man,—\\
She has gone and left me all alone.”
\end{altverse}
\end{dcverse}

It consists of eight stanzas, and ends thus:—

\settowidth{\versewidth}{Unconstant woman proves true to no man.}
\begin{dcverse}
\begin{altverse}
“Since Peggy has my kindness slighted,\\
I’ll never trust a woman more;\\
In her alone I e’er delighted,\\
But since she’s false I’ll leave the shore:
\end{altverse}

\begin{altverse}
In ship I’ll enter, on seas I’ll venture,\\
And sail the world where I’m not known:\\
Unconstant woman proves true to no man.\\
She’s gone and left me here alone.”
\end{altverse}
\end{dcverse}

This last song is still sung about the country, sometimes to a
tune resembling that printed in the \textit{Pills}, but more commonly to this
air. No tune seems to be more generally known by tradition. I have
been favored with copies from various and widely distant parts of the
country. Captain Darnell had learnt it from “old Harry Smith, the
fiddler, of Nunnington, near Kirby Moorside;” Mr. Edward Loder had
repeatedly heard it in the West of England. The late George Macfarren
recollected it, and the words he heard had the burden, “I live not
where I love.” This was before the Roxburghe Collection of ballads
had been purchased for the British Museum, and (having overlooked the
one ballad in the Pepys Collection) I did not know the burden to be
so old. Although it is impossible to guarantee any considerable
antiquity to an air preserved solely by tradition, I think it a
favorable circumstance that the measure should agree with that of the
old ballads,

\end{fixedpage}%452
\pagebreak


\begin{fixedpage}%453
\rectoheader

which, I am sure, no one of my informants had seen. The versions
from different parts of the country differ in some points, especially
in the terminations of the phrases, but that might be expected, as it
was gathered from untutored singers. The following West Country
version has a thoroughly Somersetshire ending. It was given to me by
Mr. Edward Loder, and the words written by the late George Macfarren.

\settowidth{\versewidth}{Here the vine its clusters wreatheth,}
\begin{dcverse}
\begin{altverse}
Graceful are the verdant bowers \\
Where the elm and linden grow,\\
And its bloom the chesnut showers \\
On the mossy bank below.\\
Yet give me that valley dreary\\
Where the mist-clad mountains rise, \\
And the eagle builds her eyrie,\\
Monarch bird of stormy skies.
\end{altverse}

\begin{altverse}
Here the vine its clusters wreatheth, \\
There the pine its dark form shews;\\
Here the zephyr mildly breatheth, \\
There the north wind keenly blows. \\
Dearest still, my boyhood’s places;\\
Oh! for wings of woodland dove,\\
To greet the old familiar faces;\\
For I live not where I love.
\end{altverse}
\end{dcverse}
\end{fixedpage}%453
\pagebreak

\begin{fixedpage}%454
\versoheader

\musictitle{OH! FOR A HUSBAND.}

This song is taken from John Gamble’s MS. common-place book,
which has already supplied several airs to this collection.
\end{fixedpage}%454
\pagebreak

\begin{fixedpage}%455
\rectoheader

\settowidth{\versewidth}{Though he was old, and she was young,}
\begin{dcverse}
\begin{altverse}
An ancient suitor hither came, \\
His head was almost grey;\\
Though he was old, and she was young,\\
She would no longer stay;
\end{altverse}

\indentpattern{01012}
\begin{patverse}
But to her mother went this maid,\\
And told her presently,\\
That she a husband needs must have,\\
And thus began to cry:\\
“\textit{Oh! oh! oh! for a husband},” \&c.
\end{patverse}
\end{dcverse}

The maiden fulfils the old adage of “marrying in haste and
repenting at leisure,” and, in the third and fourth stanzas, the
burden of her song changes to—

\settowidth{\versewidth}{Out upon a husband, such a husband,}
\begin{dcverse}
\begin{altverse}
\textit{“Oh! oh! oh! with a husband\\
What a life lead I!\\
Out upon a husband, such a husband,\\
A husband, fie, fie, fie!”}
\end{altverse}
\end{dcverse}

\musictitle{AN OLD WOMAN CLOTHED IN GREY.}

This tune is found in two forms, the first as \textit{An old woman
clothed in grey}, the second, as \textit{Let Oliver now he forgotten}. The
difference in the music has, no doubt, arisen from the different
metres of the words adapted to it.

In The Beggars' Opera, 1728, the song \textit{Through all the employments
of life}, is written to the tune of \textit{An old woman clothed in grey}. In
\textit{Old Ballads}, ii. 230, 1726, the song of “An old woman clothed in
grey,” is to the tune of \textit{Kind husband and imperious wife}. The song of
“The kind husband but imperious wife,” is contained in \textit{Westminster
Drollery}, 1671, and in \textit{Wit and Drollery}, 1682, but the tune is not
named in either. Here, therefore, the pedigree halts. It should be
traceable higher, for I am convinced that such words as “Kind
Husband” never had music \textit{composed} for them. They are a dialogue
between a man and his wife, and commence—

\settowidth{\versewidth}{Wife, prithee come give me thy baud now,}
\begin{scverse}
\begin{altverse}
\vleftofline{“}Wife, prithee come give me thy hand now,\\
And sit thee down by me;\\
There’s never a man in the land now \\
Shall be more loving to thee.”
\end{altverse}
\end{scverse}

A copy of \textit{An old woman clothed in grey}, in Dr. Burney’s
Collection of songs, with music (Brit. Mus.), has a manuscript date
of 1662. Besides \textit{The Beggars' Opera}, it was introduced in Henry
Carey’s \textit{Musical Century}, vol. ii., and in the ballad opera, \textit{The
Humours of the Court}, or \textit{Modern Gallantry}, 1732.

The song, \textit{Let Oliver now be forgotten}, is said to be to the tune
of \textit{How unhappy is Phillis in love}. Both words and music are contained
in 180 \textit{Loyal Songs}, 1685 and 1694; and in \textit{Pills to purge Melancholy},
ii. 283, 1719. The tune, without words, is in Salter’s \textit{Genteel
Companion for the Recorder}, 1683, and in Lady Catherine Boyd’s MS.
Lyra Viol book, lately in the possession of Mr. A. Blaikie. Many
political ballads were written to it under one or other of these
names, especially about the year 1680. For instance, in Mr.
Halliwell’s Collection, Cheetham Library, are, at fol. 171, “An
excellent new ballad of the plotting head. To the tune of \textit{How unhappy
is Phillis in love}; or, \textit{Let Oliver now be forgotten}.” Printed for R.
Moor, 1681. At fol. 243, “Tony’s Lamentation; or, Potapski’s City
Case, being his last farewell to the consecrated Whigs. The tune is,
\textit{Let Oliver now be forgotten},” 1682. In 180 \textit{Loyal Songs}, “The
Conspiracy: or, The discovery of the \textit{fanatick plot}, 1684; and in Mat.
Taubman’s
\end{fixedpage}%455
\pagebreak

\begin{fixedpage}%456
\versoheader

\textit{Heroic Poem and choice Songs and Medleyes on the times},
“Philander,” fol. 1682. The following is to the version called \textit{An
old woman clothed in grey}—

\musictitle{I WOULD I WERE IN MY OWN COUNTRY.}

This tune is in Sir John Hawkins’ Transcripts of Music for the
Virginals; also in \textit{The Dancing Master}, from 1650 to 1701, under the
name of \textit{Godesses}.

A black-letter copy of the ballad, \textit{I would I were in my own
country}, is in the Roxburghe Collection, ii. 367, entitled “The
Northern Lasses Lamentation; or The Unhappy Maid’s Misfortune;” and
prefaced by the following lines:—

\settowidth{\versewidth}{Since she did from her friends depart,}
\begin{dcverse}
\vleftofline{“}Since she did from her friends depart,\\
No earthly thing can cheer her heart;\\
But still she doth her case lament,\\
Being always fill’d with discontent;\\
Resolving to do nought but mourn,\\
Till to the North she doth return.
\end{dcverse}

To the tune, \textit{I would I were in my own country}.” Printed for P.
Brooksby at the Golden Ball, in West Smithfield; and reprinted in
Evans’ \textit{Old Ballads}, i. 115 (1810).

The following were sung to the same tune:—

Pepys Coll., i. 266. “Newes from Tower Hill; or—

\settowidth{\versewidth}{A gentle warning to Peg and Kate}
\begin{scverse}
\vleftofline{“}A gentle warning to Peg and Kate \\
To walk no more abroad so late.”
\end{scverse}

\end{fixedpage}%456
\pagebreak

\begin{fixedpage}%457
\rectoheader

“To the tune of \textit{The North-country Lasse}; subscribed M[artin]
P[arker]. London, printed for E. B. Begins, “A pretty jest I’ll
tell.”

Roxburghe Collection, ii. 112. “The Dumb Maid; or, The young
Gallant trepann’d,” \&c. “To a new tune called \textit{Dum, dum, dum}, or \textit{I
would I were in my own country}.” This is an earlier version of a song
already printed (ante p. 120), which begins, “There was a bonny
blade.” It seems to have been intended for the tune of \textit{The Duke of
Norfolk}, or \textit{Dum, dum, dum}, rather than for this. It commences—

\settowidth{\versewidth}{All you that pass along, give ear unto my Song,}
\begin{scverse}
\begin{altverse}
\vleftofline{“}All you that pass along, give ear unto my Song,\\
Concerning a youth that was young, young, young,\\
And of a maiden fair,—few with her might compare;\\
But alack I and alas! she was dumb, dumb, dumb.”
\end{altverse}
\end{scverse}

Douce Collection, p. 135. “The Lancashire Lovers; or, The merry
wooing of Thomas and Betty,” \&c. To the tune of \textit{Love's Tide}, or \textit{At home would I be in my own country}.” This, which is black-letter,
printed by Wright, Clarke, Thackeray, and Passinger (early, Charles
II.), has also the burden—

\settowidth{\versewidth}{The oak, and the ash, and the ivy tree,}
\begin{scverse}
\vleftofline{“}The oak, and the ash, and the ivy tree,\\
Flourish bravely at home in my own country;”
\end{scverse}

\end{fixedpage}%457
\pagebreak

\setlength{\fixedpagewidth}{400pt}
\begin{fixedpage}%458
\versoheader

\settowidth{\versewidth}{The ewes and the lambs, with the kids and their dams,}
\indentpattern{010122}
\begin{dcverse}
\begin{patverse}
Fain would I be in the North-country,\\
Where the lads and the lasses are making of hay;\\
There should I see what is pleasant to me;—\\
A mischief light on them entic’d me away!\\
O the oak, the ash, and the bonny ivy tree, \\
Do flourish most bravely in our country.
\end{patverse}

\begin{patverse}
Since that I came forth of the pleasant North, \\
There's nothing delightful I see doth abound,\\
They never can be half so merry as we,\\
When we are a dancing of \textit{Sellinger’s Round}. \\
O the oak, the ash, and the bonny ivy tree,\\
Do flourish at home in our own coutry.
\end{patverse}

\begin{patverse}
I like pot the Court, nor to City resort,\\
Since there is no fancy for such maids as me;\\
Their pomp and their pride I can never abide,\\
Because with my humour it doth not agree.\\
O the oak, the ash, and the bonny ivy tree,\\
Do flourish at home in my own country.
\end{patverse}

\begin{patverse}
How oft have I been on the Westmoreland green,\\
Where the young men and maidens resort for to play,\\
Where we with delight, from morning till night,\\
Could feast it, and frolic, on each holiday.\\
O the oak, the ash, and the bonny ivy tree,\\
Do flourish most bravely in our country.
\end{patverse}

\begin{patverse}
A milking to go, all the maids in a row,\\
It was a fine sight, and pleasant to see;\\
But here, in the city, they’re ’void of all pity,\\
There is no enjoyment of liberty.\\
O the oak, the ash, and the bonny ivy tree. \\
They flourish most bravely in our country.
\end{patverse}

\begin{patverse}
When I had the heart from my friends to depart,\\
I thought I should be a lady at last;\\
But now do I find that it troubles my mind,\\
Because that my joys and pleasures are past.\\
O the oak, the ash, and the bonny ivy tree, \\
They flourish at home in my own country.
\end{patverse}

\begin{patverse}
The ewes and the lambs, with the kids and their dams,\\
To see in the country how finely they play; \\
The bells they do ring, and the birds they do sing,\\
And the fields and the gardens, so pleasant and gay.\\
O the oak, the ash, and the bonny ivy tree, \\
They flourish most bravely in our country.
\end{patverse}

\begin{patverse}
At wakes and at fairs, being ’void of all cares, \\
We there with our lovers did use for to dance;\\
Then hard hap had I, my ill fortune to try,\\
And so up to London my steps to advance.\\
O the oak, the ash, and the bonny ivy tree,\\
They flourish most bravely in our country.
\end{patverse}

\begin{patverse}
But still, I perceive, I a husband might have. \\
If I to the city my mind could but frame;\\
But I’ll have a lad that is North-country bred,\\
Or else I’ll not marry, in the mind that I am.\\
O the oak, the ash, and the bonny ivy tree,\\
They flourish most bravely in our country.
\end{patverse}

\begin{patverse}
A maiden I am, and a maid I’ll remain,\\
Until my own country again I do see,\\
For here in this place I shall ne’er see the face \\
Of him that’s allotted my love for to be.\\
O the oak, the ash, and the bonny ivy tree,\\
They flourish at home in my own country.
\end{patverse}

\begin{patverse}
Then farewell, my daddy, and farewell, my mammy,\\
Until I do see you, I nothing but mourn;\\
Rememb’ring my brothers, my sisters, and others,\\
In less than a year I hope to return.\\
Then the oak, the ash, and the bonny ivy tree,\\
I shall see them at home in my own country.
\end{patverse}
\end{dcverse}

\musictitle{THE BROOM, THE BONNY BROOM.}

In the Pepys Collection, i. 40, is a black-letter ballad,
entitled \textit{The new Broome} [\textit{on hill}]. London, printed for F. Coles. It
consists of seven stanzas, and commences thus:—

\settowidth{\versewidth}{He bids the broome that blooms him by}
\indentpattern{010101012323}
\begin{dcverse}
\begin{patverse}
\vleftofline{“}Poore Coridon did sometime sit \\
Hard by the broome alone,\\
And secretly complain’d to it \\
Against his only one.\\
He bids the broome that blooms him by \\
Beare witness to his wrong,\\
And, thinking that none else was nie,\\
He thus began his song:\\
\textit{The bonny broome, the wellfavour’d broome,\\
The broome blooms faire on hill;\\
What ail’d my love to lightly mee,\\
And I working her will?”}
\end{patverse}
\end{dcverse}

The second line of the burden recalls the “bunch of ballads and songs, all

\end{fixedpage}%458
\pagebreak

\setlength{\fixedpagewidth}{360pt}
\begin{fixedpage}%459
\rectoheader

ancient; as \textit{Broom, broom, on hill},” \&c., which are mentioned in
Laneham’s \textit{Letter from Kenilworth}, 1575; also the lines sung by Moros,
in Wager’s \textit{The longer thou livest the more fool thou art},—an
interlude which appears to have been written soon after Elizabeth
came to the throne. In that, Moros enters, “counterfaiting a vaine
gesture and a foolish countenance, synging the \textit{foote} of many songes,
as fooles were wont;” the first of which is—

\settowidth{\versewidth}{The brome stands on Hive hill-a.”}
\begin{dcverse}
Brome, brome on hill,\\
The gentle brome on hill, hill:\\
Brome, brome on Hive hill,\\
The brome stands on Hive hill-a.”
\end{dcverse}

This repetition does not give the metre or the correct words of
the song. The tune, or upper part, was to be sung by one person,
while others sang a foot, or burden, to make harmony. So, in the same
play, Idlenesse says—

\settowidth{\versewidth}{Thou hast songes good stoare, sing one,}
\begin{scverse}
\vleftofline{“}Thou hast songes good stoare, sing one,\\
And we three the \textit{foote} will beare.”
\end{scverse}

In \textit{The Dancing Master}, from 1650 to 1698, and in \textit{Musick’s Delight
on the Cithren}, 1666, is a tune entitled \textit{Broom, the bonny, bonny
broom}. I believe this to be the tune of \textit{The new broome on hill}, as
well as of another ballad in the same metre, and issued by the same
printer, entitled “The lovely Northern Lasse—

\settowidth{\versewidth}{What harme she got milking her daddies ewes.”}
\begin{scverse}
Who in the ditty here complaining shewes\\
What harme she got milking her daddies ewes.”
\end{scverse}

To a pleasant Scotch tune, called \textit{The broom of Cowdon Knowes}.”
London, printed for Fr. Coles, in the Old Bayly (Mr. Halliwell’s
Collection). This is the English ballad of \textit{The broom of Cowdenowes},
and the tune is here said to be Scotch. I believe it not to be
Scotch, for the following reasons:—Firstly, the tune is not in the
Scottish scale, and is to be found as a three-part song in Addit.
MSS., No. 11,608 British Museum (the same that contains \textit{Vive le Roy},
before quoted, and written at the end of Charles the First’s reign).
Secondly, because English tunes or songs were frequently entitled
“Scotch,” if they related to Scottish subjects, or the words were
written in imitation of the Scottish dialect; (so with \textit{Lilliburlero},
Purcell’s tune is called “a new Irish tune” in \textit{Musick's Handmaid}, not
because it is an imitation of Irish music, nor even a new tune, but
because a new song on Irish affairs); and I rely the more upon this
evidence from having found many other ballads to the tune of \textit{The
broom, the bonny, bonny broom}, but it is nowhere else entitled
Scotch, even in ballads issued by the same printer. Thirdly, Burton,
in his \textit{Anatomy of Melancholy}, quotes it as a common English country
tune. Under the head of “Love Melancholy—Symptoms of Love” (edit, of
1652), he says, “The very rusticks and hog-rubbers\dots have their
Wakes, Whitson-ales, Shepheard’s feasts, meetings on holidays,
Country Dances, Roundelays, writing their names on trees, true
lovers’ knots, pretty gifts.\dots
Instead of Odes, Epigrams and Elegies, \&c., they have
their Ballads, Country tunes, \textit{O the broom, the bonny, bonny broom};
Ditties and Songs, \textit{Bess a Bell she doth excel}: they must write
likewise, and indite all in rhime.” Fourthly, because 1650 is too
early a date for Scotch tunes to have been popular among the lower
classes in England:—I do not think one can be traced before the reign
of Charles II. It is a common modern error to suppose that England
was inundated with Scotch tunes at the union of the two crowns. The
first effect was
\end{fixedpage}%459
\pagebreak

\begin{fixedpage}%460
\versoheader

directly the reverse, and the popularity of Scotch tunes in
England should rather be dated from the reign of James II. I shall
hereafter have occasion to revert to this subject, and therefore will
not further enlarge at present.

I know of no other copy of \textit{The new broome on hill} than the one in
the Pepys Collection, but am persuaded it is a reprint of a much
earlier ballad. Such lines as “To ease my \textit{grieved grone},” seem to
point to the “doleful dump” period of poetry; and the tune not being
named, is an indication of its having been copied from one of the
earlier part of Queen Elizabeth’s reign, or perhaps even before. The
ballad of \textit{Brome on hill} in Mr. Gutch’s \textit{Robin Hood}, ii. 363, is a
modern fabrication.

\textit{The broom of Cowdon Knowes} is a long story in two parts. Besides
the copy in Mr. Halliwell’s Collection, it will be found among the
Roxburghe Ballads,
i. 190; and is reprinted in Evans’ \textit{Old Ballads}, i. 88, 1810. The
following are the two first stanzas:—

\settowidth{\versewidth}{Through Liddersdale as lately I went,}
\begin{dcverse}
\begin{altverse}
\vleftofline{“}Through Liddersdale as lately I went,\\
I musing on did passe;\\
I heard a maid was discontent,\\
She sigh’d and said, alas!\\
All maids that ever deceived were,\\
Bear a part of these my woes,\\
For once I was a bonny lasse,\\
When I milkt my daddies ewes.\\
\textit{With O the broom, the bonny broom,\\
The broom of Cowdon Knowes;\\
Fain would I be in the North Country, \\
To milk my daddies ewes.}\\
My love into the fields did come,\\
When my daddy was at home;\\
Sugar’d words he gave me there,\\
Prais’d me for such a one;\\
His honey breath and lips so soft,\\
And his alluring eye,\\
His tempting tongue hath woo’d me oft,\\
Now forces me to cry.\\
All maids,” \&c. 
\end{altverse}
\end{dcverse}

The balled which follows \textit{The new broom} in the Pepys Collection is “The Complaint of a Sinner. To the
tune of \textit{The bonny broome} (i. 41). It commences, “Christ is my love,
he loved me,” and has but a slightly different burden.

In the Roxburghe, i. 522, is “John Hadland’s advice; Or, a
warning for all young men that have meanes, advising them to forsake
bad company, cards, dice, and queanes. To the tune of \textit{The bonny,
bonny broome}.” Subscribed R[ichard] C[limsall], and “Printed at London
for Francis Coules.” It commences—

\settowidth{\versewidth}{For I have wrought my overthrow,}
\begin{dcverse}
\begin{altverse}
\vleftofline{“}To all men now I’ll plainly shew \\
How I have spent my time;\\
For I have wrought my overthrow,\\
With drinking beer and wine,” \&c.
\end{altverse}
\end{dcverse}

In the same Collection, iii. 174, is a ballad by M[artin]
P[arker], called “The  bonny Bryer; or—

\begin{scverse}
\vleftofline{“}A Lancashire lass her sore lamentation\\
For the death of her love and her own reputation.
\end{scverse}

To the tune of \textit{The bonny broome}.” It commences—

\settowidth{\versewidth}{I made a stay, and look’d about me then,}
\indentpattern{010101012323}
\begin{dcverse}
\begin{patverse}
\vleftofline{“}One morning early by the break of day,\\
Walking to Totnam Court,\\
Upon the left hand of the high-way,\\
I heard a sad report:\\
I made a stay, and look’d about me then,\\
Wond’ring from whence it was,—\\
At last I spyed, within my ken,\\
A blyth and buxome lasse.\\
\textit{Sing O the bryer, the bonny, bonny bryer.\\
The bryer that is so sweet; \\
Would I had stay’d in Lancashire,\\
To milk my mothers neate.”}
\end{patverse}
\end{dcverse}

\end{fixedpage}%460
\pagebreak

\begin{fixedpage}%461
\rectoheader

This was “Printed at London for F[rancis] G[rove] on Snow Hill.”
Again, in the same Collection, i. 380, is “Slippery Will; or, The old
Bachelor’s Complaint, with his advice to all yong men not to doe as
he had done—

\settowidth{\versewidth}{Which makes him now this proverb say,}
\begin{dcverse}
His youthful time he spent away,\\
Which makes him now this proverb say,\\
That \textit{he that mill not when he may,\\
When he would, he should have nay}.” 
\end{dcverse}

To the tune of \textit{The bonny, bonny broome}.” it commences—

\settowidth{\versewidth}{Long have I liv’d a bachelor’s life,}
\indentpattern{010101012323}
\begin{dcverse}
\begin{patverse}
\vleftofline{“}Long have I liv’d a bachelor’s life,\\
And had no mind to marry,\\
But now I would fain have a wife,\\
Either Doll, Kate, Sis, or Mary.\\
These four did love me very well,\\
I had my choice of many;\\
But one did all the rest excell,\\
And that was pretty Nanny.\\
\textit{O young men all, to you levy and call,\\
Make not too long delay,\\
For if you mill not when you may,\\
When ye would, ye shall have nay}” 
\end{patverse}
\end{dcverse}

Six stanzas, with a second part of eight. London, printed for E. B.

Rox. Coll., ii. 575. “The forlorn Lover’s Lament. To the tune of
\textit{The bony broom}” Black-letter (printer’s name cut off); beginning—\settowidth{\versewidth}{Sir, do not think these lines have flow’d}
\begin{scverse}
\vleftofline{“}Sir, do not think these lines have flow’d \\
From youthful hearts or hands,” \&c.
\end{scverse}

There are a great number of English songs and ballads on the
subject of “broom” and “bonny broom,” but the enumeration would
exceed the space I could devote to it. I will therefore cite but one
more, and from a very early and scarce book. In Bale’s \textit{Comedy
concernynge the Lawes of Nature}, 1538, in the second act, is a song,
with a staff of five lines ruled for writing in the music, which is
as follows:—

\settowidth{\versewidth}{Brom, brom, brom, brom, brom,}
\indentpattern{01015}
\begin{dcverse}
\begin{patverse}
\vleftofline{“}Brom, brom, brom, brom, brom, \\
Bye brom, bye, bye,\\
\columnbreak
Bromes for shoes and powcherynges,\\
Botes and buskins for new bromes.\\
\textit{Brom, brom}, \&c.
\end{patverse}
\end{dcverse}

The first \textit{Scotch} song of \textit{The broom of Cowdenknows} was printed in
Allan Ramsay’s \textit{Tea Table Miscellany}, 1724. It is there classed among
the “new words by different hands;” and commences, “How blyth ilk
morn was I to see.” The subject of the older English burden is there
retained.

The above version of the tune is not so good as that in \textit{The
Beggars’ Opera}, or in Thomson’s \textit{Orpheus Caledonius}; but those copies
are of more than seventy years later date.

\end{fixedpage}%461
\pagebreak


\begin{fixedpage}%462
\versoheader

\musictitle{I AM A POOR SHEPHERD UNDONE.}

This tune has two names, \textit{I am a poor shepherd undone}, and \textit{Hey,
ho, my honey}. It is found in \textit{The Dancing Master} of 1665; next, in the
edition of 1686, and in all later: in \textit{Pills to purge Melancholy} , yi.
284: in \textit{Apollo’s Banquet for thy Treble Violin}, \&c.

In the King’s Pamphlets, vol. 15, fol.; in the Bagford
Collection, p. 67; in \textit{Loyal Songs}, ii. 67; and in Wright’s \textit{Political
Ballads}, p. 146, are copies of “A proper new Ballad on the Old
Parliament, or the second part of Knave out of doores. To the tune
of—

\settowidth{\versewidth}{Four-and-twenty now for your money, and yet a hard pennyworth too.}
\begin{scverse}
Hei, ho, my honey, my heart shall never rue;\\
Four-and-twenty now for your money, and yet a hard pennyworth too.”
\end{scverse}

The copy in the King’s Pamphlets is dated Dec. 11,1659. The
ballad begins “Good morrow, my neighbours all, what news is this I
heard tell,” \&c.

In the Roxburghe Collection, ii. 54, and Collier’s Roxburghe
Ballads, p. 298, are “A Caveat for young men, or The bad husband
turn’d thrifty,” \&c. “To the tune of \textit{Hey, ho, my honey},” beginning
“All you young ranting blades that spend your time in vain,” by John
Wade. Printed by W. Thackeray, T. Passinger, and W. Whitwood.

Ritson quotes Wade as the author of a ballad entitled “The
Maiden’s sad Complaint for want of a Husband, \&c. To the new West
Country tune, or \textit{Hogh when shall I be married}.” It commences thus:—

\settowidth{\versewidth}{My father hath forty good shillings,}
\indentpattern{031031}
\begin{dcverse}
\begin{patverse}
\vleftofline{“}Oh! when shall I he married, \\
Oh! be married?\\
My beauty begins to decay; \\
’Tis time to find out somebody, \\
Oh! somebody,\\
Before it is quite gone away.
\end{patverse}

\begin{patverse}
My father hath forty good shillings, \\
Oh! good shillings,\\
And never a daughter but me; \\
My mother is also willing,\\
Oh! so willing,\\
That I shall have all if she die.”
\end{patverse}
\end{dcverse}

The black-letter copy of this ballad in the Douce Collection (p.
67) was printed for Richard Burton, at the Horse Shoe in West
Smithfield (time of the Commonwealth). It consists of 14 stanzas,
three of which (beginning with “My father has forty good shillings”)
have been appropriated in Collections of Scotch Songs.

\textit{Hey, ho, my honey}, is also one of the tunes to which “The valiant
Seamen’s Congratulation” to Charles II. on his accession, was to be
sung (ante p. 292).

In \textit{Pills to purge Melancholy} , the following is entitled “The
distress’d Shepherd:”—
\end{fixedpage}%462
\pagebreak

\begin{fixedpage}%463
\rectoheader

\begin{dcverse}

\settowidth{\versewidth}{But to love you is not in my power.}
\indentpattern{010101013}
\begin{patverse}
If to love me she would not incline,\\
I said I should die in an hour;\\
“To die,” said she, “is in thine,\\
But to love you is not in my power. \\
I ask’d her the reason why\\
She could not of me approve;\\
She said ’twas a task too hard,\\
To give any reason for love.\\
\textit{And alas! poor Shepherd}, \&c.
\end{patverse}

\begin{patverse}
She asked me of my estate,\\
I told her a flock of sheep;\\
The grass whereon they graze,\\
And where she and I might sleep: \\
Besides a good ten pound,\\
In old King Harry’s groats;\\
While hooks and crooks abound,\\
And birds of sundry notes.\\
\textit{And alas! poor Shepherd}, \&c.
\end{patverse}
\end{dcverse}

\musictitle{CHRISTMAS’S LAMENTATION.}

The words and tune of this ballad are contained in Gamble’s MS.
commonplace book. The ballad is also in the Roxburghe Collection, i.
48, entitled “Christmas’ Lamentation for the losse of his
acquaintance; showing how he is forst to leave the Country, and come,
to London. To the tune of \textit{Now the Spring is come}.”

The ballad, “Now the Spring is come,” is in the same Collection,
i. 200, entitled “A Lover’s desire for his best beloved; or, Come
away, come away, and do not stay. \textit{To an excellent new Court tune}.” It
commences thus;—
\end{fixedpage}%463
\pagebreak

\begin{fixedpage}%464
\versoheader

\settowidth{\versewidth}{Their sweet tunes, their sweet tunes, their and do not stay:}
\indentpattern{00000023}
\begin{dcverse}
\begin{patverse}
\vleftofline{“}Now the Spring is come, turn to thy love, \\
To thy love, to thy love, to thy love; make no delay.\\
While the flowers spring and the birds do sing\\
Their sweet tunes, their sweet tunes, their and do not stay:\\
Where I will'fill thy lap full of flowres, sweet tunes,\\
And cover thee with shady bowres,\\
Come away, come, away, come away,\\
And do not stay.”
\end{patverse}
\end{dcverse}

This copy of the ballad, having been printed by the assigns of
Thomas Symcocke, is of the reign of James I. \textit{Christmas’s Lamentation}
must also be a ballad of the reign of Elizabeth or James I., although
the Roxburghe copy is not of so early a date. Yellow starch is
mentioned in the sixth stanza, and it came into fashion in the latter
part of the reign of Elizabeth, and continued until November, 1615,
the date of the execution of the celebrated beauty, Mrs. Turner, for
participation in the poisoning of Sir Thomas Overbury. When the Lord
Chief Justice Coke sentenced her to death, he ordered that, “as she
was the person who had brought yellow starched ruffs into vogue, she
should be hanged in that dress, that the same might end in shame and
detestation.” “Even the hangman who executed this unfortunate woman
was decorated with yellow ruffs on the occasion.” (Rimbault’s \textit{Life of
Overbury}.)

The rhythm of the first part of the following tune is peculiar,
from its alternate phrases of two and three bars, but, still, not
unsatisfactory to the ear.

I have not thought it necessary to print, at length, \textit{all} the
repetitions of words that occur in the ballad, as they are
sufficiently indicated by the first stanza which is here adapted to
the music.

\end{fixedpage}%464
\pagebreak
\begin{fixedpage}%465
\rectoheader

\settowidth{\versewidth}{Christmas beef and bread is turn’d into stones,}
\indentpattern{02020030022}
\begin{dcverse}
\begin{patverse}
Christmas beef and bread is turn’d into stones, \\
Into stones and silken rags;\\
And Lady Money sleeps and makes moans, \\
And makes moans in misers’ bags: \\
Houses where pleasures once did abound, \\
Nought but a dog and a shepherd is found, \\
Welladay!\\
Places where Christmas revels did keep,\\
Now are become habitations for sheep. \\
\textit{Welladay, welladay,\\
Welladay, where should I stay?}
\end{patverse}

\begin{patverse}
Pan, the shepherd’s god, doth deface,\\
Doth deface Lady Ceres’ crown,\\
And the tillage doth go to decay,\\
To decay in every town;\\
Landlords their rents so highly enhance,\\
That Pierce, the ploughman, barefoot may dance;\\
Welladay!\\
Farmers, that Christmas would still entertain, \\
Scarce have wherewith themselves to maintain. \\
\textit{Welladay}, \&c.
\end{patverse}

\begin{patverse}
Come to the countryman, he will protest,\\
Will protest, and of bull-beef boast;\\
And for the citizen he is so hot,\\
Is so hot, he will burn the roast.\\
The courtier, sure good deeds will not scorn, \\
Nor will he see poor Christmas forlorn?— \\
Welladay!\\
Since none of these good deeds will do, \\
Christinas had best turn courtier too.\\
\textit{Welladay}, \&c.
\end{patverse}

\begin{patverse}
Pride and luxury they do devour,\\
Do devour housekeeping quite;\\
And soon beggary they do beget,\\
Do beget in many a knight.\\
Madam, forsooth, in her coach must wheel,\\
Although she wear her hose out at heel, \\
Welladay!\\
And on her back wear that for a weed,\\
Which me and all my fellows would feed. \\
\textit{Welladay}, \&c.
\end{patverse}

\begin{patverse}
Since pride came up with the yellow starch, \\
Yellow starch, poor folks do want,\\
And nothing the rich men will to them give, \\
To them give, but do them taunt;\\
For Charity from the country is fled,\\
And in her place hath nought left but need;\\
Welladay!\\
And corn is grown to so high a price,\\
It makes poor men cry with weeping eyes. \\
\textit{Welladay}, \&c.
\end{patverse}

\begin{patverse}
Briefly for to end, here I do find,\\
I do find so great vacation,\\
That most great houses seem to attain\\
To attain a strong purgation:\\
Where purging pills such effects they have shew’d,\\
That forth of doors they their owners have spued;\\
Welladay!\\
And where’er Christmas comes by, and calls, \\
Nought now but solitary and naked walls. \\
\textit{Welladay}, \&c.
\end{patverse}

\begin{patverse}
Philemon’s cottage was turn’d into gold,\\
Into gold, for harbouring Jove:\\
Rich men their houses up for to keep,\\
For to keep, might their greatness move; \\
But in the city, they say, they do live,\\
Where gold by handfulls away they do give:— \\
I’ll away,\\
And thither, therefore, I purpose to pass, \\
Hoping at London to find the golden ass.\\
\textit{I'll away, I’ll away,\\
I’ll away, for here’s no stay.}
\end{patverse}
\end{dcverse}

\end{fixedpage}%465
\pagebreak

\begin{fixedpage}%466
\versoheader

\musictitle{WHEN LOVE WAS YOUNG.}

The words and music of this song are contained in Gamble’s MS.
commonplace book.

\settowidth{\versewidth}{The flow’rs of broom he deck’d with may,}
\begin{dcverse}
\begin{altverse}
He gave her gloves as white and soft \\
As were the hands that wore ’em,\\
And many a leafy garland, sweet \\
As were the brows that bore ’em.\\
He woo’d,—she sigh’d,\\
The shepherd then was merry;\\
He stole a kiss, the loving maid \\
Blush’d red as any cherry.
\end{altverse}

\begin{altverse}
He danc’d her many a roundelay,\\
And footed it full fine;\\
The flow’rs of broom he deck’d with may,\\
All for his Rosaline.\\
He said,—he swore,\\
He lov’d her best of any;\\
She pinch’d his cheeks, and, sighing, said, \\
“O, shepherd, thon lov’st many.”
\end{altverse}

\begin{altverse}
He said that he was ever true\\
And constant, from his mother;\\
“When I am gone, thou’lt have a new, \\
And after her another.”\\
“O no!” “O yes!”\\
“Believe it, pretty maid.”\\
“I do believe:” and then they kiss’d,\\
And thus they wantons play’d.
\end{altverse}
\end{dcverse}

\end{fixedpage}%466
\pagebreak
\thispagestyle{empty}
\begin{fixedpage}%467
\headingthree{REIGNS OF CHARLES II.,}
\headingthree{JAMES II., AND WILLIAM AND MARY.}
\centerrule

\textsc{From} the restoration of Charles II. may be dated an entire change
in the style of music till then cultivated in England. The learned
counterpoint and contrivance of madrigals and motets in vocal music,
and of fancies in instrumental music, fell gradually out of esteem,
and were replaced by a lighter and more melodious style of air; such
as could be better appreciated by uncultivated ears. The viol,
hitherto the chief instrument for chamber concerted music, was
gradually replaced by the violin, and the supremacy of the lute in
vocal music was then first contested by the guitar.

Charles II. had passed the greater part of his life in exile;
where sauntering, dancing, and dallying with his mistresses, had been
his principal occupation. One of his letters of that time is so
characteristic, that it is here subjoined entire. It was written from
Cologne, and addressed to his “deerest aunte,” the Queen of Bohemia.
The orthography is preserved, as by no means the least curious part;
it would have disgraced a school-boy.



\begin{quotation}

\hfill “Collen, Augt. 6 [1665].\hspace*{4em}

\indent\indent “Madam,—I am just now begining this Letter in my Sisters Chamber,
wher ther is such a noise that I never to hope to end it, and much
less write sence. For what concernes my sisters journey and the
accidents that happened in the way, I leave her to give-your Ma\textsuperscript{ty} an
account of. I shall only tell your Ma\textsuperscript{ty} that \textit{we are now thinking of
how to passe our time; and in the first place of danceing}, in which
we find to difficultyes, the one for want of the fidelers, the other
for somebody both to teach and assist at the danceing the new Dances:
and I have gott my sister to send for Silvius as one that is able to
performe both: for the fideldedies, my L\textsuperscript{d} Taaffe does promise to be
there convoy, and in the meane time we must contente our selves with
those that makes no difference between a himme and a coranto. I have
now receaved my sisters pickture that my deare cousin the Princess
Louise was pleased to draw, and do desire your Ma\textsuperscript{ty} thank her for me,
for ’tis a most excellent pickture, which is all I can say at
present, but that I am, Madame,

\hspace{12em}Your Ma\textsuperscript{ties} most humble

\hspace{15em}and most affectionate nephew and servant 

\hspace{24em}Charles R.
\end{quotation}

The original letter is in MS. Lans. 1236 (fol. 106), British
Museum; and a copy is printed in the second series of \textit{Original
Letters illustrative of English History}, edited by Sir H. Ellis, iii.
376. On the 18th of the same month, Charles wrote, from Bruges, to
Henry Bennet (whom he afterwards created Earl of Arlington), “Pray
get me pricked down as many new Corants and Sarabands, and other
little dances, as you can, and bring them with you, for I have a
small fidler that does not play ill on the fiddle.” And on the 1st of
September of the following year, in another letter to the same
person,—“You will find, by my last, that though I am furnished with
one small fidler, yet I would have another to
\end{fixedpage}%467
\pagebreak

\renewcommand\versoheadertext{viols and violins.}
\renewcommand\rectoheadertext{reign of charles ii.}

\begin{fixedpage}%468
\versoheader

keep him company; and if you can get either he you mentioned, or
another that plays well, I would have you do it.”

The King knew enough of music to take his part in an easy
composition; and, after his restoration, would sometimes sing duets
with “that stupendous base,” Mr. Gostling, of the Chapel Royal, the
Duke of York (afterwards James II.) accompanying them on the guitar.
The Hon. Roger North says that Charles “was a professed lover of
music, but of this” [dancing] “kind only; and had an utter
detestation of fancies,” or other compositions in the fugal style;
and, not the less so, from an unsuccessful entertainment of that kind
given him by Secretary Williamson; “after which, the Secretary had no
peace, for the King, as his way was, could not forbear whetting his
wit upon the subject of the \textit{fancy-music}, and its patron the
Secretary. He would not allow the matter to be disputed upon the
point of meliority, but ran all down by saying, \textit{Have I not ears}? He
could not bear any music to which he could not keep time, and that he
constantly did to all that was presented to him; and, for the most
part, heard it standing.” Pepys describes him as beating time with
his hand “all along the anthem,” in the Chapel Royal; and Dr. Tudway
accuses the young composers of his Chapel of having so far given way
to the King’s French taste, as to introduce dancing movements and
theatrical corantos into their anthems.

Speaking of the “grand metamorphosis of music” that took place in
this reign, the Hon. Roger North says, “Upon the Restoration, the old
way of concerts were laid aside at court, and the King made an
establishment after the French model, of twenty-four violins [tenors
and bases being counted among them], and the style of the music was
accordingly.” Wood says, “he would have the twenty-four violins
playing before him while he was at meals;” but Evelyn, speaking of a
visit to the Chapel Royal, on Dec. 21, 1662, says, that, after one of
His Majesty’s Chaplains had preached, “instead of the ancient, grave,
and solemn wind music [cornets and sackbuts] accompanying the organ,
was introduced a concert of twenty-four violins, between every pause,
after the French, fantastical, light way; better suiting a tavern or
playhouse than a church.”

Violins had long been the favorite instruments for dancing,
whether with common fiddlers or at court. They were probably first
included in the \textit{Royal} band, under the name of \textit{violins}, in the fourth
year of the reign of Queen Elizabeth (1561); and the sum then paid to
performers on that instrument was 230\textit{l}. 6\textit{s}. 8\textit{d}. (MS. Lansd. No. 5).
Ten years after, there were seven “vyolons,” at an annual cost of
325\textit{l}. 15\textit{s}. (MS. Cotton, Vesp. c. xiv.). Charles the First’s band, in
1625, consisted of eight hautboys and sackhuts, six flutes, six
recorders, \textit{eleven violins}, six lutes, four viols, and a harp
(exclusive of drummers, trumpeters, and fifers); and in 1641 it
numbered fifty-eight musicians, of whom fourteen were violins. So far
as the antiquity of the instrument is concerned, it may he traced
back to the Anglo-Saxons, for the modern violin is but an improvement
upon the ancient fiddle in shape. The curious may see in a manuscript
of the tenth century, in the British Museum (Cotton, Tiberius c.
vi.), an illumination of an
\end{fixedpage}%468
\pagebreak

\begin{fixedpage}%469
\rectoheader

Anglo-Saxon gleeman, or minstrel, playing on his “fiðele,” which
has four strings, the two sound-holes, and is played on by a bow; but
in shape is more like the half of a long pear, very taper towards the
stalk. This shape would have been very inconvenient for reaching high
notes, but the use of the upper part of the finger-board was then
unknown.\textsuperscript{a}

The reason why viols had been preferred to violins, tenors, and
violoncellos, for chamber-music, was simply this: until the reign of
Charles II., the music played was in close counterpoint, of limited
compass for each instrument, and in from three to six parts, every
visitor being expected to take a part, and generally at sight. The
frets of the viols secured the stopping in tune, which one
indifferent ear in the party might otherwise have marred.

The violin had “a lift into credit” in Cromwell’s time, by the
arrival in England of Thomas Baltzar, a celebrated performer on that
instrument, born at Lubeck. The Hon. Roger North says “he did wonders
upon it by swiftness and \textit{doubling of notes}, but his hand was
accounted hard and rough.” Evelyn, in his Diary (March 4, 1656),
says, “His variety on a few notes and plain ground, with that
wonderful dexterity, was admirable. Though a young man, yet so
perfect and skilful, that there was nothing, however cross and
perplext, brought to him by our artists, which he did not play off at
sight, with ravishing sweetness and improvements, to the astonishment
of our best masters.” Wood speaks of him with equal enthusiasm, and
adds that, after his arrival, Mr. Davis Mell, who had been accounted
the best violin player in England, was not so admired; “yet he played
sweeter, was a well-bred gentleman, and not given to excessive
drinking as Baltzar was.”

At the Restoration, the King appointed Baltzar leader of his
private band of twenty-four; and, about the same time, according to
Wood, “he commenced Bachelar of Musick at Cambridge.” Baltzar died in
1663, and Charles then appointed John Banister in his place.
Banister, however, was afterwards dismissed for saying on his return
from France (whither the King had sent him), that the English
violins\textsuperscript{b} were better than the French. At that time, and for many
years before, the favorite entertainments of the French court were
ballets,

\begin{dcfootnote}
\textsuperscript{a} This fiddle has been engraved in Strutt’s \textit{Sports and Pastimes
of the People of England}, and in Wackerbarth’s \textit{Music and the Anglo
Saxons}, 8vo., 1837.

\textsuperscript{b} The names of Charles the Second’s private band of 24, and the
sums they received In the year 1674, which amounted to £1433 17\textit{s}.
8\textit{d}., are given in a note at p. 98 of \textit{Memoirs of Musick}, by the Hon.
Roger North, edited by Edward F. Rimbault, LL.D., from a document in
his possession. It is as follows “The Gentlemen of his Majesties
Private Musick paid out of the Excheker,—

\begin{tabular}{lrrr}
&£&\textit{s}.&\textit{d}.\\
Tho.Purcell\dotfill &200&0&0\\
Pelham Humphreys\dotfill &200&0&0\\
John Hardlnge \dotfill & 40&0&0\\
William Hawes\dotfill & 46&10&10\\
Tho. Blagrave, sen. \dotfill & 40&9&2\\
Alf. Marsh \dotfill & 40&0&0\\
John Goodgroome \dotfill & 40&0&0\\
Nat. Watkins \dotfill & 40&0&0
\end{tabular}

\begin{tabular}{lrrr}
&£&\textit{s}.&\textit{d}.\\
Mat. Lock \dotfill &40&0&0\\
John Clayton \dotfill & 152&13&4\\
Izaack Stagins, sen. \dotfill & 46&10&10\\
Nich. Stagins, jun.\dotfill & 46&10&10\\
Tho. Battes \dotfill & 90&0&0\\
John Lilly \dotfill & 40&0&0\\
Hen. Gregory \dotfill & 60&0&0\\
Theoph. Hills \dotfill & 46&10&10\\
Hen. Madge \dotfill &86&12&8\\
John Gambell \dotfill & 46&10&10\\
Rich. Dorney \dotfill & 20&0&0\\
John Banister, sen. \dotfill & 100&0&0\\
Phil. Beckett \dotfill & 60&2&6\\
Rob. Blagrave, jun. \dotfill & 58&4&2\\
John Singleton \dotfill & 46&10&10\\
Rob. Strange \dotfill & 46&10&10
\end{tabular}

\hspace{4em} 15 May, 1674. (Signed) \textsc{T.Purcell}.

\end{dcfootnote}
\end{fixedpage}%469
\pagebreak
\renewcommand\versoheadertext{changes in instrumental music.}

\begin{fixedpage}%470
\versoheader

and the music of the most inartificial description. The treble
part contained the whole of the melody, the base and interior part
being mere accompaniment, without variety, and inferior in
counterpoint. France had then produced fewer good musicians than any
country in Europe; and when, about 1660, Lully (a Florentine by
birth, but brought up in France from ten years of age) was placed at
the head of a band of violins, created for him by Louis XIV., and
called \textit{Les petits Violons}, to distinguish them from the twenty-four,
“not half the musicians in France were able to play at sight.” Even
the famous band of twenty-four were incompetent, says La Borde, to
play anything they had not specially studied and gotten by heart.
They were, therefore, in this respect, inferior to English gentlemen
in their own art. Nor did Lully effect any great reform in this
respect, for when the Regent, Duke of Orleans, wished to hear
Corelli’s Sonatas, which were newly brought from Rome, no three
persons in Paris could be found to play them. He was obliged to have
them \textit{sung by three voices}. This is related by Michael Corette (a
strong partizan of French music), in the Preface to his \textit{Méthode
d’Accompagnement}, and quoted from him by M. Choron. Corette was
organist of the Jesuits’ College in Paris in 1738. Louis XIV. died in
1715.

I conjecture the reason of Charles the Second’s preference for
French music to have been, in a great measure, because, as
dance-music, it was not so generally composed upon old scales as were
the “Fancies,” which were then the principal chamber-music of
England. Some of those scales sound very harshly to uninitiated ears.
There was also a rhythm in dance-music, which would bear the King’s
test of beating time, and it was the only style admired at the French
Court, the gaieties and laxities of which, during exile, had formed
so agreeable a contrast to the austere presbyterianism of his
Scottish subjects, as to have inspired him with a predilection for
everything French.

To those who are curious to know what fancies, or fantasies,
were, I recommend the perusal of the \textit{Fantasies of three parts} [for
viols] \textit{composed by Orlando Gibbons}, printed in the early part of the
reign of James I. Having been reprinted by the \textit{Musical Antiquarian
Society}, they are more accessible than any other. To those who are
satisfied with the judgment of another, I submit the following
analysis by one who is thoroughly versed in the music of the
sixteenth and seventeenth centuries, Mr. G. A. Macfarren:—

\begin{quotation}
“The fantasies of Orlando Gibbons are most admirable specimens of
pure part-writing in the strict contrapuntal style; the announcement
of the several points, and the successive answers and close
elaboration of these, the freedom of the melody of each part, and the
independence of each other, are the manifest result of great
scholastic acquirement, and consequent technical facility. Their
form, like that of the madrigals and other vocal compositions of the
period, consists of the successive introduction of several points or
subjects, each of which is fully developed before the entry of that
which succeeds it. The earlier fantasies in the set are more closely
and extensively elaborated, and written in stricter accordance with
the Gregorian modes, than those towards the close of the collection,
which, from their comparatively rhythmical character, and greater
freedom of modulation, may even be supposed to have been aimed at
popular effect. They would, it is true, be little congenial to modern
ears,
\end{quotation}
\end{fixedpage}%470
\pagebreak

\begin{fixedpage}%471
\rectoheader

\begin{quotation}
but this is because of the strangeness to us of the crude tonal
system that prevailed at the time, and upon which they are
constructed. The peculiarities that result from it are the
peculiarities of the age, and were common to all the best writers of
the school in this and every other country. Judged by the only true
standard of criticism,—judged merely as what they were designed to
be,—they must be pronounced excellent proofs of the musical
erudition, the ingenious contrivance, and the fluent invention of the
composer.”
\end{quotation}

Before the introduction of fantasies, says the Hon. Roger North,
“whole consorts for instruments of four, five, and six parts were
solemnly composed, and with wonderful art and invention, whilst one
of the parts (commonly in the middle) bore onely the plain song
throughout. And I guess that, in some time, little of other consort
musick was coveted or in use. But that which was styled \textit{In Nomine},
was yet more remarkable, for it was onely descanting upon seven
notes, with which the syllables \textit{In Nomine Domini} agreed. And of this
kind I have seen whole volumes of many parts, with the several
authors’ names inscribed. And if the study, contrivance, and
ingenuity of these compositions to fill the harmony, carry on fugues,
and intersperse discords, may pass in the account of skill, no other
sort may plead so more; and it is some confirmation that in two or
three ages last bygone the best private musick, as was esteemed,
consisted of these.” A volume of \textit{In Nomines}, formerly in the
possession of the North and L’Estrange families, is now in that of
Dr. Rimbault. They are in five, six, seven, and eight parts; and
among the composers are Shepherd, Taverner, Tye, Munday, Tallis,
Byrd, \&c. Among the earlier writers of \textit{fantasies} whose works are
still extant, are Robert White (the well-known church composer, who
died before 1581), Byrd, Morley, Dr. Bull, Michael Este or East,
Ferabosco, Cooper, and others.

Queen Elizabeth’s Virginal Book contains numerous fantasies for
that instrument, including one by John Munday, “Faire Wether,
Lightning, Thunder, Calme Wether, \&c. and in Lady Nevill’s, we have
a composition by Byrd, entitled “The Battell,” with the following
movements:—“.The March of Foote-men; The March of Horsemen; The
Trumpetts; The Irish Marche; The Bagpipe and Drone; The Flute and
Drone; The March to fight; Tantara; The Battells be joyned; The
Retreat; and The Galliarde for the Victorie.”

Speaking of “Musick designed for Instruments,” Christopher
Simpson says, “Of this kind, the chief and most excellent for art and
contrivance are Fancies of six, five, four, and three parts, intended
commonly for viols. In this sort of Musick the composer (being not
limited to words) doth imploy all his art and invention solely about
the bringing in and carrying on these Fuges according to the order
and method formerly shewed. When he has tried all the several ways
which he thinks fit to be used therein, he takes some other point and
does the like with it; or else, for variety, introduces some
chromatick notes, with bindings and intermixtures of discords; or
falls into some lighter humour like a Madrigal, or what else his own
fancy shall lead him to; but still concluding with something which
hath art and excellency in it.”

Among the lighter kinds of instrumental music, were Pavans,
Galliards
\end{fixedpage}%471
\pagebreak

\renewcommand\versoheadertext{music of london.}

\begin{fixedpage}%472
\versoheader

Corantos, \&c.; and, in the reign of James I., such collections
as that of “Courtly Masquing Ayres, composed to five and six parts,
for Violins, Consorts, and Cornets, by John Adson,\textsuperscript{a} 1621;” and others
already mentioned.

Roger North says, “The French manner of instrumental music did
not gather so fast as to make a revolution all at once; but, during
the greater part of Charles the Second’s reign, the old music was
used in the country and in many meetings and societies in London. But
the treble viol was disregarded, and the violin took its place.”
English musicians were willing to give the palm to the Italians in
vocal music, after the invention of recitative; but they claimed to
rank above every nation in instrumental music; and, so far as I can
trace, that claim was commonly admitted and well founded.

“None give so harsh a report of Englishmen as the English
themselves,” says Henry Lawes,—a remark which is too frequently true;
but it is a national peculiarity, the boundary of which is strongly
marked by the river Tweed, and which, happily for our neighbours, has
never extended to the northern bank of that stream. Charles, although
of Scottish descent, was born far south of it; and to his opinion I
would oppose that of Count Lorenzo Magalotti, a Florentine, and one
of the most eminent characters of the brilliant court of Ferdinand II,
Grand Duke of Tuscany. Magalotti (to whom Sir Isaac Newton gave the
name of “il magazzino del buon gusto”) wrote his journal while making
a tour in England in 1669, and acting as Secretary to the hereditary
Prince of Tuscany, afterwards Cosmo III. In describing the plays that
were represented at the London theatres, he says, “Before the comedy
begins, that the audience may not be tired with waiting, \textit{the most
delightful symphonies} are played; on which account, many persons come
early to enjoy \textit{this agreeable amusement}.” (Travels
of Cosmo III., Grand Duke of Tuscany, p. 191, 4to., Lond., 1821).
This is unfortunately the only notice of secular music throughout the
diary, for his object was to describe the country and to collect
statistics, rather than to draw comparisons of manners and customs,
or of the state of the arts.

We have also favorable testimony from the Sieur de la Serre,
Historiographer of France, who accompanied Mary de’ Medici to London,
in 1638. He says, in his “description of the city of London,” that
“in all public places, violins, hautboys, and other sorts of
instruments are so common, for the amusement of particular persons,
that, at all hours of the day, one may have one’s ears charmed with
their sweet melody.” Again, “The \textit{excellent} musicians of the Queen of
Great Britain sang,” \&c. I have read many accounts of foreigners
travelling in England in and before the seventeenth century, but
never yet found one to speak with the slightest disparagement of the
music. The criticism, which is usually to be found in their travels,
is invariably favorable.

Roger North, giving credit to the Italians for having first
printed \textit{Fantasias}, says that “the English, working more elaborately,
improved upon their pattern,

\begin{dcfootnote}
\textsuperscript{a} The above work was “printed by T. S., for John 
Browne, and to be sold in St. Dunstan’s Churchyard, in 
Fleet Street.” It was dedicated to “George, Marquises 
of Buckingham.” A copy of five, out of the six parts,
is in Marsh’s Library, Dublin. They are the “Cantus, 
Tenor, Bassus, Medius, aud Sextus.” Adson composed one
popular tune to which ballads were sung, called  
“Adsorbs Saraband.”
\end{dcfootnote}
\end{fixedpage}%472
\pagebreak

\begin{fixedpage}%473
\rectoheader

which gave occasion to an observation, that in vocal the
Italians, and in instrumental music the English excelled.” (p. 74).

Tuscany and Rome both claim the honor of the invention of
recitative music; Rome for Emilio del Cavaliere, and Tuscany for
Jacopo Peri. The sacred drama, or oratorio, \textit{Dell' Anima e del Corpo},
by Emilio del Cavaliere, and Peri’s opera of \textit{Eurydice}, were both
first printed in 1600. The latter was produced in the theatre at
Florence in that year, on the occasion of the marriage of Mary de’
Medici with Henry IV. of France. Although performed, on so great an
event, “in a most magnificent manner,”, and in the presence of the
Queen, the Grand Duke, the Cardinal Legate, and innumerable princes
and noblemen of Italy and France, it appears from the Author’s
preface, that only four musical instruments were employed,—a
harpsichord, a large guitar, a large lute, and a large lyre. The lyre
was probably an instrument of the harp description for the music of
Orpheus, intended to imitate the ancient lyre. (Dr. Burney translates
“lira grande,” \textit{viol da gamba!}) These four instruments were, without
doubt, to be used separately for accompanying particular voices (as
was the custom in somewhat later Italian operas), and not to be
played in concert. The only instrumental music in the opera is a
short symphony of eight bars for a \textit{triflauto}, or triple flute.\textsuperscript{a} The
employment of instruments of \textit{various sorts} in combination seems to
have been little practised in Italy, although at this time each ward
of the city of London, and the suburbs of Finsbury, Southwark, \&c.,
had its band that played habitually, with various instruments, in six
parts. Two years before Peri’s opera was produced, Hentzner wrote of
the “suavissima adhibita musica” (the charming music performed) in
the London theatres; and, to prove the variety of instruments
occasionally employed in English plays, we may quote (for an early
date) Gascoyne’s \textit{Jocasta}, 1566, in which each act is preceded by dumb
show, accompanied by the music of “viols, cythren, bandores [or large
lutes], flutes, cornets, trumpets, drums, pipes, and stillpipes.”

I have already alluded to the number of English instrumental
performers in the employ of foreign courts in the reigns of Elizabeth
and James I., and may add that, in the Court-Masques of the latter
reign, as many as from 60 to 80 instrumentalists were sometimes
engaged. As an instance, which I select because it has not before
been printed, take the following from the list of “Rewards to the
persons employed in the Maske,” by Ben Jonson, which was presented at
court at Christmas, 1610--11, the original document being among the
Pell Records:

\begin{quotation}
To 12 Musicions that were Preestes, that songe and played \dotfill £24

To 12 \textit{other} Lutes that suplied and with Flutes \dotfill 12

To 10 Violencas [Violoncellos] that continually practized to the Queen,  \dotfill 20 

To 4 more that were added at the Maske \dotfill 4

To 15 Musitions that played the pages and fooles \dotfill  20

To 13 Hoboyes and Sackbutts  \dotfill 10
\end{quotation}


\begin{dcfootnote}
\textsuperscript{a} Probably an ancient triple flute was to be held by 
\textit{Tirsi}, whilst the symphony was played behind the scenes 
by three flutes, as the music is in three parts. Dr. Burney
solves the difficulty by translating \textit{un triflauto} “three
flutes,” but it is in the singular number. He divides the symphony
of eight bars, of six in a bar, into \textit{fifteen}.
(\textit{History}, vol. iv., p. 31.)
\end{dcfootnote}

\end{fixedpage}%473
\pagebreak

\renewcommand\versoheadertext{fashion for foreign music.}

\begin{fixedpage}%474
\versoheader

There are also rewards “to Mr. Alfonso [Ferabosco], for making
the Songes, £20; to Mr. [Robert] Johnson, for setting the Songs to
the Lutes, £5; and to Mr. Thomas Lupo, for setting the Dances to the
Violins, £5.” The viol, the violin players, and other members of the
royal band, are not included in the above list, and therefore
probably received only their usual payments in the form of salary.

The splendid Court-Masques of the reigns of James I. and Charles
I. afforded ample opportunities for the development of the power of
Recitative, which gave variety and novelty to the entertainments.
Recitative seems to have been first composed in England by Nicholas
Laniere, an eminent musician, painter, and engraver, in the service
of James I. He was an Italian by birth, but lived and died in
England. There were four of the name in James’s band—“John, Nicholas,
Jerom, and Clement,”—of whom one other, at least, was painter as well
as musician. Evelyn says, in his Diary, under the date of Aug.
1,1652, “Came old Jerome Lennier, of Greenwich, a man skilled in
painting and music, and another rare musician named Mell” (the violin
player mentioned by Anthony à Wood). “Lennier had been a domestic of
Queen Elizabeth, and shewed me her head, an intaglio in a rare
sardonyx, cut by a famous Italian,, which he assured me was exceeding
like her.” Nicholas Laniere’s “Hero’s Complaint to Leander, in
Recitative Music,” gives a very favorable impression of his ability
in that style of composition. It is printed in the fourth book of
\textit{Choice Ayres and Songs, to sing to the Theorbo Lute or Bass Viol}
(fol., Playford, 1683).

From the introduction of Recitative began a fashion for Italian
vocal music, which in the latter part of the reign of Charles I. was
so predominant, that scarcely any other was esteemed by the upper
classes. They seemed to think that whatever was Italian must be
necessarily good; and that, if not Italian, it must be otherwise.
This indiscriminate preference is noticed by Henry Lawes in the
preface to his first book of \textit{Ayres and Dialogues}, published in 1653:
“Wise men have observed our generation so giddy,” says he, “that
whatsoever is native, be it never so excellent, must lose its taste,
because themselves have lost theirs. For my part I profess (and such
as know me can bear witness) I desire to render every man his due,
whether strangers or natives\dots and, without depressing the honor
of other countries, I may say our own nation hath had, and yet hath,
as able musicians as any in Europe. I confess the Italian language
may have some advantage by being better smooth’d and vowell’d for
music, which I found by many songs which I set to Italian words, and
our English seems a little overclogged with consonants, but that’s
much the composer’s fault, who, by judicious setting, and right
tuning the words, may make it smooth enough. This present generation
is so sated with what’s native, that nothing takes their ear but
what’s sung in a language which, commonly, they understand as little
as they do the music. And to make them a little sensible of this
ridiculous humour, I took a table or index of old Italian songs, and
this index, which read together made a strange medley of nonsense, I
set to a varied air, and gave out that it came from Italy, whereby it
passed for a rare Italian song. This very song have I here
\end{fixedpage}%474
\pagebreak

\begin{fixedpage}%475
\rectoheader

printed.” Again he says, “There are knowing persons who have been
long bred in those worthily admired parts of Europe, who ascribe more
to us than we to ourselves; and able musicians, returning from
travel, do wonder to see us so thirsty after foreigners. Their manner
of composing is sufficiently known to us, their best compositions
being brought over hither by those who are able enough to choose.”
Lawes was an excellent musician, and composed the music to Milton’s
\textit{Comus}. He was highly esteemed both by Milton and Waller. As some of
his songs have been recently revived, and sung in public, he is
better known to the present generation than almost any other composer
of his day; and his fame has been sufficiently vindicated from the
very unjust criticism of Dr. Burney.

The fashion for foreign music continued to spread; and in 1656,
Matthew Lock, in his preface to his \textit{Little Consort of three parts,
containing Pavans, Ayres, Corants, and Sarabands,for Viols} or
Violins, says: “For those mountebanks of wit who think it \textit{necessary}
to disparage all they meet with of their own countrymen, because
there have been, and are, some excellent things done by strangers, I
shall make bold to tell them (and I hope my known experience in this
science will inforce them to confess me a competent judge) that I
never yet saw any foreign \textit{instrumental} composition (a few French
Corants excepted) worthy an Englishman’s transcribing.” He adds, “I
only desire, in the performance of this Consort, you would do
yourselves and me the right to play plain, not tearing them in pieces
with division,—an old custom of our country fiddlers, and now, under
the title of \textit{à la mode}, endeavoured to be introduced.” In the same
strain, Christopher Simpson, in his \textit{Compendium of Practical Music},
says, “You need not seek outlandish authors, especially for
instrumental music; no nation (in my opinion) being equal to the
English in that way, as well for their \textit{excellent} as for their \textit{various
and numerous} consorts of three, four, five, and six parts, made
properly for instruments—of all which, as I said, Fancies are the
chief.” (3rd edit. 8vo., 1678.) So also Playford, in his \textit{Introduction
to the Skill of Music}: “But musick in this age, like other arts and
sciences, is in low esteem with the generality of people. Our late
and solemn musick, both vocal and instrumental, is now jostled out of
esteem by the new Corants and Jigs of foreigners, to the grief of all
sober and judicious understanders of that formerly solid and good
musick.” (6th edit. 8vo., 1672.) And in his preface to \textit{Musick’s
Delight on the Cithren} (1666), “It is observed that of late years all
solemn and grave musick is much laid aside, being esteemed too heavy
and dull for the light heels and brains of this nimble and wanton
age; nor is any musick rendered acceptable, or esteemed by many, but
what is presented by foreigners: not a City Dame, though a tap-wife,
but is ambitious to have her daughters taught by \textit{Monsieur La Novo
Kickshawibus} on the Gittar, which instrument is but a new old one,
used in London in the time of Queen Mary, as appears by a book
printed in English of Instructions and Lessons for the same, about
the beginning of Queen Elizabeth’s reign; being not much different
from the Cithren, only that [the Gittern?] was strung with gut
strings, this with wire, which was accounted the more sprightly and
cheerful musick, and was in more esteem, till of late years, than the
Gittar.”
\end{fixedpage}%475
\pagebreak

\renewcommand\versoheadertext{guitar, harp, etc.}

\begin{fixedpage}%476
\versoheader

Roger North says, “It imparts not much to the state of the world,
or the condition of human life, to know the names and styles of those
authors of musical composition whose performances gained to the
nation the credit of excelling the Italians in all but the vocal.
Nothing is more a fashion than music,—no, not clothes or language,
either of which is made a derision in after times. The grand custom
of all is to affect novelty, and to goe from one thing to another,
and to despise the former. Cannot we put ourselves \textit{in loco} of former
states, and judge \textit{pro tunc}? It is a shallow monster that shall hold
forth in favour of our fashions and relishes, and maintain that no
age shall come wherein they will not be despised and derided; and if,
on the other side, I may take upon me to be a fidling prophet, I may
with as much reason declare that the time may come when some of the
present celebrated musick will be as much in contempt as \textit{John come
kiss me now, now, now}, and perhaps with as much reason as any is
found for the contrary at present.”

The versatility of the English in the fashion of music, in the
reign of Charles II., was quite as great as their variableness in
dress; to ridicule which, Andrew Borde, a physician in the reign of
Henry VIII., in his \textit{Boke of the Introduction of Knowledge},
describing, and giving engravings of, the costume of other countries,
paints the Englishman naked, with a pair of shears in his hand', and
with the following lines:—

\settowidth{\versewidth}{For now I wyll were this, and now I wyll were that,}
\begin{scverse}
\vleftofline{“}I am an Englyshman, and naked I stand here,\\
Musyng in my mynd what rayment I shall were;\\
For now I wyll were this, and now I wyll were that,\\
Now I wyll were I cannot tel what.\\
All new fashyons be plesaunt to me,\\
I wyll have them, whether I thryve or thee:\\
Now I am a frysker, all men doth on me looke,\\
What should I do but set Cocke on the Hoope?\\
What do I care yf all the worlde me fayle?\\
I will get a garment shal reche to my tayle.\\
Then I am a minion, for I were the new gyse,\\
The yere after this I trust to be wyse,” \&c.
\end{scverse}

So in Charles the Second’s reign it was first French music, then
Italian music; first one instrument, and then another; just as some
new performer appeared, who pleased the King.

The Guitar was brought into fashion in 1662, by Francisco
Corbeta, who “had a genius for music,” says Count Grammont, “and was
the only man who
could makè anything of it\dots The king’s relish for his
compositions had
brought the instrument so much into vogue, that every person
played upon it, well or ill; and you were as sure to see a Guitar on
a lady’s toilette, as rouge or patches.” (\textit{Memoirs}, p. 174, 8vo.,
1846.) Evelyn also mentions him as playing “with extraordinary
skill.”

M. Jorevin de Rocheford, who printed his travels in England at
Paris in 1672, says, “the Harp was then the most esteemed of musical
instruments by the English.” He made this observation- at Worcester,
where an English gentleman,
\end{fixedpage}%476
\pagebreak

\begin{fixedpage}%477
\rectoheader

who had kindly acted as interpreter for him, supped with him at
the inn, and “sent for a band of music, consisting of all sorts of
instruments.” M. Jorevin also mentions going to one of the college
chapels in Cambridge, where the whole of divine service was sung
every day to music, and thinks he “there counted more than fifty
musicians, as many clerks, and the like number of ministers.” If so,
\textit{tempora vere mutantur}.

Charles II. advanced the salaries of the thirty-two Gentlemen of
the Chapel Royal to 70\textit{l}. a year each; but he sometimes left them,
like his private musicians and the public servants, from two to five
years without their money. Pepys tells us in December, 1656, that
“many of the musique are ready to starve, they being five years
behind hand with their wages;” and adds that “Evens, the man upon the
harp, having not his equal in the world, did the other day die from
mere want, and was fain to be buried from the alms of the parish, and
carried to his grave in the dark at night without one link, but that
Mr. Hingston (the organist) met the funeral by chance, and did give
12\textit{d}. to-buy two or three links.” 

Evelyn speaks in strong terms of
admiration of the harp, when well-played. In his Diary (January
20, 1653-4) he says, “Came to see me my old acquaintance and the most
incomparable player on the Irish harp, Mr. Clarke, after his travels.
He was an excellent musician, a discreet gentleman (born in
Devonshire, as I remember). Such music before or since did I never
hear, that instrument being neglected for its extraordinary
difficulty; but in my judgment, far superior to the lute itself, or
whatever speaks with strings.” Again, on November 17, 1668, “I heard
Sir Edward Sutton play excellently on the Irish harp; he performs
genteely, but not approaching my worthy friend, Mr. Clark, a
gentleman of Northumberland, who makes it execute lute, viol, and all
the harmony an instrument is capable of; pity it is that it is not
more in use; but, indeed, to play well, takes up the whole man, as
Mr. Clark has assured me, who though a gentleman of quality and
parts, was yet brought up to that instrument from five years old, as
I remember he told me.”

I suppose the harp above-mentioned to be that with a double row
of strings, which is described by Galilei, in his \textit{Dialogo della
Musica}, 1581, as the Irish harp. It could not otherwise be so
difficult an instrument. In Galilei’s time it had from fifty-four to
sixty strings, generally of metal, and was played upon by the nails,
as the Spaniards now do on the guitar. There were, at the same time,
double harps strung with gut; for the use of the intestines of
animals as strings for musical instruments, was known and practised
in very early times—even by the ancient Greeks.

In Wales, according to Edward Jones, harps with triple rows of
strings were in use in the fifteenth century. ( Welsh Bards, i. 104.)
Michael Drayton speaks of a peculiar mode of stringing the ancient
British harp, in the following passage from his \textit{Polyolbion}:—

\settowidth{\versewidth}{Th’ old British bards, upon their harps,}
\indentpattern{001001}
\begin{dcverse}
\begin{patverse}
\vleftofline{“}Th’ old British bards, upon their harps, \\
For falling flats and rising sharps \\
That curiously were strung; \\
To stir their youth to warlike rage,\\
Or their wild fury to assuage,\\
In their loose numbers sung.”
\end{patverse}
\end{dcverse}
\end{fixedpage}%477
\pagebreak
\renewcommand\versoheadertext{origin of english opera.}
\begin{fixedpage}%478
\versoheader

Upon either the double or triple harp, music in a variety of keys
might be performed; but that with a single row of strings could not
have more than one or two accidentals in the octave. The Hon. Roger
North says, “The common harp, by the use of gut strings, hath
received incomparable improvement, but cannot be a consort
instrument, because it cannot follow organs and violls in the
frequent change of keys; and the wind music, which by all stress of
invention hath been brought into ordinary consort measures, yet more
or less labours under the same infirmity, especially the chief
of them, which is the trumpet.” Ambrose Philips, in his fifth
\textit{Pastoral}, has beautifully described the effects which the harp is
peculiarly capable of producing, where he says—

\settowidth{\versewidth}{Now lightly skimming o’er the strings they pass,}
\begin{dcverse}
\vleftofline{“}His fingers, restless, traverse to and fro, \\
As in pursuit of harmony they go; \\
Now lightly skimming o’er the strings they pass,\\
Like wings that gently brush the plying grass, \\
While melting airs arise at their command; \\
And now, laborious, with a weighty hand, \\
He sinks into the chords with solemn pace, And\\
gives the swelling tones a manly grace.”
\end{dcverse}

It may now be desirable to give a few particulars of the
establishment of operas with recitative in this country, and of the
origin of public concerts, but to do so, it will be necessary to
revert to the time of the Commonwealth.

The first step towards the revival of dramatic music during the
usurpation, was the performance of Shirley’s masque, entitled \textit{Cupid
and Death}. It was presented (according to the title-page of the
printed copy) “before his Excellence the Ambassadour of Portugal,
upon the 26th of March, 1653;” and the music, of which there are two
manuscript copies in the library of the British Museum, was composed
by Christopher Gibbons and Matthew Lock. One of those copies is in
the handwriting of Matthew Lock. (See Addit. MSS., No. 17,799.)

In 1656, Sir William Davenant obtained permission to open a
theatre for the performance of operas, in a large room “at the back
of Rutland House, in the upper end of Aldersgate St.” He commenced
with “An Entertainment in Declamation and Music, after the manner of
the Ancients;” the vocal and instrumental music to which was composed
by Dr. Charles Coleman, Captain Henry Cook, Mr. Henry Lawes and Mr.
George Hudson. In the same year he produced the first opera, “The
Siege of Rhodes, made a representation by the Art of Prospective in
Scenes, and the Story sung in Recitative Musick.” From his address to
the reader, we learn that there were five changes of scenes,
“according to the ancient dramatic distinctions made for time;” but
the size of the room did not permit them to be more than eleven feet
in height and about fifteen in depth, including the places of passage
reserved for the music. There were seven performers; the part of
Solyman being taken by Captain Henry Cook, and that of Ianthe by Mrs.
Coleman, wife to Mr. Henry Coleman, who was, therefore, the first
female actress on the English stage. The remaining five parts were
doubled,—sometimes represented by one person, and sometimes by the
other. They were, Villerius, by Mr. Gregory Thorndell and Mr.
Dubartus Hunt; Alphonso, by Mr. Edward Coleman and Mr. Roger Hill;
the Admiral, by Mr. Matthew Lock and Mr. Peter Rymon; Pirrhus, by Mr.
John Harding and
\end{fixedpage}%478
\pagebreak

\begin{fixedpage}%479
\rectoheader

Mr. Alphonso March; and Mustapha, by Mr. Thomas Blagrave and Mr.
Henry Purcell. The vocal music of the first and fifth “entries,” or
acts, was composed by Henry Lawes; of the second and third by Captain
Henry Cook (who was afterwards Master of the Children of the Chapels
Royal); and the fourth by the celebrated Matthew Lock. The
instrumental music was composed by Dr. Charles Coleman and George
Hudson, and performed by Messrs. William Webb, Christopher Gibbons,
Humphrey Madge, Thomas Baltzar, “a German,” Thomas Baites, and John
Banister. The scenery was designed and “ordered” by Mr. John Web.
Davenant assigns as a reason for his “numbers being so often
diversified,” that “frequent alterations of measure are necessary to
recitative mnsic.” I have given rather minute details of the manner
in which this opera was performed, because it is not mentioned by Sir
John Hawkins, and Dr. Burney had not examined the edition of 1656,
and his account and all his deductions are consequently erroneous.\textsuperscript{a}
(\textit{History}, iv. 182.)

In 1669, Louis XIV. granted, by letters patent, to the Sieur
Perrin “une permission d’établir en nôtre bonne Ville de Paris, et
autres de nôtre Royaume, des Academies de Musique pour chanter en
public de pieces de Theatre, comme il se pratique en Italie, en
Allemagne, \textit{et en Angleterre}.” According to Menestrier, in his \textit{Des
Représentations en Musique Anciennes et Modernes} (Paris, 1681), after
Perrin had enjoyed this patent for a few years, it was revoked and
given to Lully. From this it is evident that opera was established in
England about thirteen years before France, and that Matthew Lock
was, by about twenty years, an earlier composer of dramatic music
than Lully.

We learn from Ogilby’s “Relation of His Majesty’s entertainment
passing through the city of London to his Coronation,” April 22nd,
1661, that Lock composed the whole of the music for the public entry
of Charles II., and had received the appointment of “Composer in
Ordinary” to the King. His “Psyche,” \textsuperscript{b} seems to have been the first
opera \textit{printed} in England (4to., 1675), and it is mixed with
“interlocutions (or dialogue), as more proper to our Genius” than the
Italian plan of being entirely in recitative. To that system we have
since adhered almost without exception.

Our public concerts originated from the music performed at
taverns. When the civil war commenced, and “the whole of the masters
of music in London were turned adrift, some went into the army,
others dispersed in the country and made music for the consolation of
the Cavalier gentlemen,” while many of the musicians of the theatres
were driven to earn a subsistence by frequenting taverns and inviting
the guests to hear them perform. They who went into the country “gave
great occasion,” says Roger North, “to divers [county] families to
entertain the

\begin{dcfootnote}
\textsuperscript{a} Burney comes to a conclusion directly opposed to the fact,
viz.: that “it seems as if this drama was no more like an Italian
opera than the Masques which long preceded it.” \textit{History}, vol. 4, p.
182. A copy in the British Museum wants the last leaf, and that leaf
contains many of the above particulars. I am indebted to Dr. Rimbault
for the loan of a perfect copy.

\textsuperscript{b} Lock’s instrumental music to Shakespeare's \textit{Tempest}
was printed in score with \textit{Psyche}. His music to \textit{Macbeth} was not
printed during his lifetime, and we have no copy extant of so early a
date. A tune called “Macbeth, a Jigg,” is in \textit{Musick's Delight on the
Cithhren}, 1666, and the same is in \textit{The Pleasant Companion to the
Ftlagelet} with the initials of M[atthew] L[ock] against it. Lock is
said to have composed the music to Macbeth in 1670. This jig is of
four years earlier date.
\end{dcfootnote}
\end{fixedpage}%479
\pagebreak

\renewcommand\versoheadertext{first public concerts.}
\begin{fixedpage}%480
\versoheader

skill and practice of music, and to encourage the masters, to the
great increase of composition.” As an instance, he says, that Mr.
John Jenkins, one of the court musicians of Charles I., and an
esteemed composer of instrumental music in his day, had written so
much concerted music at the houses of different gentlemen, to suit
the capabilities of the various performers, that there were
“horseloads” of his works dispersed about. “A Spanish Don having sent
some papers to Sir Peter Lely, containing one part of a concert of
four parts, of a sprightly moving kind, such as were called Fancies,
desiring that he would procure and send him the other parts, \textit{costa
che costa}” North shewed the papers to Jenkins, “who knew the concert
to be his, but when or where made he knew not. His compositions of
that kind were so numerous, that he himself outlived the knowledge of
them.”

The number of superior musicians thus added to those who
habitually performed at taverns, rendered them places of great
resort, and brought a rich harvest to the tavern-keepers. After the
theatres were closed, taverns were the only public places in which
music was to be heard. However, in 1656-7, Cromwell’s third
Parliament passed “an Act against vagrants and wandering idle
dissolute persons, in which it was ordained that, “if any person or
persons, commonly called fiddlers or minstrels, shall at any time
after the 1st of July be taken playing, fiddling, and making music,
in any inn, alehouse, or tavern, or shall be taken proffering
themselves, or desiring, or intreating any person or persons to hear
them play or make music in any of the places aforesaid,” they shall
be adjudged rogues, vagabonds, and sturdy beggars, and be proceeded
against and punished accordingly. This checked instrumental music at
the time, and the visitors being driven to amuse themselves, indulged
the more in vocal, by joining together in singing part-songs,
catches, and canons. As gentlemen had been taught to sing at sight,
as a part of their education, there was rarely a difficulty in
finding the requisite number of voices. Pepys mentions going to a
coffee-house with Matthew Lock and Mr. Purcell (Henry Purcell’s
father), where, with other visitors, in a room next the water, they
had a variety of “brave Italian and Spanish songs, and a canon for
\textit{eight} voices which Mr. Lock had lately made on these words,—“Domine,
salvum fac Regem.” This was while General Monk was in London, and
before he had declared for the King. It was therefore a bold measure
to sing such a canon at the time, and they must have been well
assured that there were none but Cavaliers in the room.

After the Restoration, according to Roger North, the first place
of entertainment where music was regularly performed was “in a lane
behind Paul’s, where there was a chamber organ that one Phillips
played upon, and some shopkeepers and foremen came weekly to sing in
concert, and to hear and enjoy ale and tobacco [as they do now in
Germany]. And after some time the audience grew strong, and one Ben
Wallington got the reputation of a notable base voice, who also set
up for a composer, and hath some songs in print, but of a very low
excellence.” He adds, that “their music was chiefly out of Playford’s
catch-book.” We know that in 1664 there was a “Musick-house at the
Mitre near
\end{fixedpage}%480
\pagebreak

\begin{fixedpage}%481
\rectoheader

the west end of St. Paul’s Church,” (where “Robert Hubert, alias
Forges, Gent.,” exhibited his “natural rarities,”) and this was
probably the original spot; but in Playford’s \textit{Catch that catch can,
or The Musical Companion}, 1667, Benjamin Wallington, citizen, is also
mentioned as one of the “endeared friends of the late Musick Society
and Meeting in the Old-Jury, London.” North says, “these meetings
shewed an inclination in the citizens to follow music; and the same
was confirmed by many little entertainments the masters voluntarily
made for their scholars, for, being known, they were always crowded.”

“The next essay was of the elder Banister, who had a good
theatrical vein, and in composition had a lively style peculiar to
himself. He procured a large room in Whitefriars, near the Temple
back gate, and made a large raised box for the musicians, whose
modesty required curtains. The room was surrounded with seats and
small tables, alehouse fashion. One shilling was the price, and call
for what you pleased. There was very good music, for Banister found
means to procure the best hands in town, and some voices to come and
perform there, and there wanted no variety of humour, for Banister
himself, among other things, did wonders upon a flageolet, to a
thorough-base, and the several masters had their solos.”

There was also “a society of gentlemen of good esteem,” who used
to meet weekly for the practice of instrumental music in concert, at
a tavern in Fleet Street, “but the taverner pretending to make formal
seats and to take money,” the society was disbanded. However, the
masters of music finding that money was to be got in this way,
determined to take the business into their own hands, and about the
year 1680, a concert room was built and furnished for public concerts
in Yilliers Street, York Buildings. This was the first public concert
room\textsuperscript{a} independent of ale and tobacco. It was called “The Musick
Meeting,” and “all the quality and \textit{beau monde} repaired to it; there
was nothing of music valued in town, but was to be heard there.”

Banister’s concerts continued till his death in 1678, and in that
year the club or private concerts established by John Britton, “the
musical small-coalman,” in Clerkenwell, had its beginning, and
continued till l7l4. The concert room in York Buildings was in use
till the middle of the last century, and was pulled down about the
year 1768.

Our musical festivals originated in the celebrations of St.
Cecilia’s Day; and the first celebration of which we have any record,
occurred in the year 1683. The reader will find full information on
this subject in the \textit{Account of the musical celebrations of St.
Cecilia's day}, recently published by Mr. W. H. Husk, librarian to the
Sacred Harmonic Society.

As Roger North says that “the tradesmen and foremen” sang chiefly
out of Playford’s Catch-book (which consists of rounds, canons,
catches, and other part-music), a few words on the subject of catches
may not be out of place.

Many quotations have already been adduced about smiths, tinkers,
pedlars,

\begin{dcfootnote}
\textsuperscript{a} I say “public concert room,” because old English mansions of the sixteenth century had generally each a concert or
music room,
\end{dcfootnote}

\end{fixedpage}%481
\pagebreak

\renewcommand\versoheadertext{catches and other part-music.}
\begin{fixedpage}%482
\versoheader

watchmen, and others of the same class, singing catches, rounds,
and roundelays, in the sixteenth century (pp. 108 to 110), but the
oustom may he traced to a more remote period.

In 1453, Sir John Norman, being the first Lord Mayor of London
who “brake that auncient and olde continued custome of riding with
greate pompe unto Westminster, to take his charge,” choosing rather
to be rowed thither by water, “the watermen made of him a roundell or
song, to his great praise, the which began,— 

\settowidth{\versewidth}{Rowe the bote, Norman,}
\begin{scverse}
Rowe the bote, Norman,\\
Rowe to thy Lemman.”
\end{scverse}

For this we have the authority of a contemporary, Robert Fabyan,
who was Sheriff of London in 1493-4. But the very “roundel” seems to
have been in Playford’s possession in 1658, when he printed an
enlarged edition of Hilton’s \textit{Catch that catch can}, because “Row the
boat, Norman,” is one of the rounds in the index to that collection.
It was omitted in the body of the work, and another substituted (if I
may judge by the only two copies I have seen); but, in 1672, Playford
printed “Row the boat, \textit{Whittington},” in a collection of a similar
nature, entitled \textit{The Musical Companion}. Sir Richard Whittington was
Lord Mayor of London long before Sir John Norman, and I have little
doubt of the name having been altered because Whittington was then
the popular Lord Mayor of history, and the story of his cat
universally known. There were many ballads about him, like “Sir
Richard Whittington’s Advancement,” \&c.; and one of the tradesmen’s
tokens in the Beaufoy Collection, of the year 1657, is of “J. M. M.,
at Whittington’s Cat” in Long Lane. Peter Short, a printseller, who
died of the plague in 1665, having obtained an old engraved plate of
Sir Richard, with his right hand resting on a skull, transformed the
skull into a cat, to make the print accord with the popular
tradition.

This round seems to have been intended to imitate the merry
ringing of the church bells on Lord Mayor’s day; it is of the
simplest construction, and of but six bars. As a musical curiosity,
it is subjoined,—

Row the boat, Whittington, thou wor-thy ci - ti-zen, Lord Mayor of London.
[Row the boat, Norman, row, row to thy le-man, thon LordMayor of London.]

Let the first voice begin, and sing it through several times, not
stopping at the end but recommencing immediately. The second to do
exactly the same; but to commence after the first has sung two bars;
and the third in like-manner after the second. If sung merrily with
three equal voices (or more to each part), it will have an agreeable
effect, like the following two bars constantly recurring, but with an
interchange of voices:—

These popular rounds, catches, and canons, seem to have been
first collected for

\end{fixedpage}%482
\pagebreak

\begin{fixedpage}%483
\rectoheader

publication by Ravenscroft, in the reign of James I. In 1609 the
first was issued under the title of “Pammelia: Musick’s Miscellanie;
or mixed varietie of pleasant Roundelayes and delightfull Catches of
3, 4, 5, 6, 7, 8, 9, 10 parts in one: none so ordinarie as musicall;
none so musicall as not to all very pleasing and acceptable.” That
many of these were “ordinary” catches and rounds is clearly proved by
the words. We find among them, “New oysters, new Waylfleet oysters;”
“A miller, a miller, a miller would I be;” “Jolly Shepherd;” “Joan,
come kiss me now;” “Dame, lend me a loaf;” “The white hen she
cackles;” “Banbury Ale;” “There lies a pudding in the fire;” “Trole,
trole, the bowl;” and others of the same description. There are a
hundred in the collection, and among them many of great excellence
and of very early date. As a specimen of the words, I give “Hey,
jolly Jenkin,” the catch which Samuel Harsnet mentioned in 1604, as
one which tinkers sang “as they sat by the fire with a pot of good
ale between their legs,”—a not unusual accompaniment to the singing.
It is the seventh in the collection.

\settowidth{\versewidth}{Now God be with old Simeon,}
\indentpattern{00100100000}
\begin{dcverse}
\begin{patverse}
\vleftofline{“}Now God be with old Simeon, \\
For he made cans for many a one, \\
And a good old man was he; \\
And Jenkin was his journeyman, \\
And he could tipple of every can, \\
And this he said to me:\\
‘To whom drink ye?’\\
‘Sir knave to you;\\
Then, hey, jolly Jenkin,\\
I spy a knave drinking,—\\
Come, pass this can to me.’”
\end{patverse}
\end{dcverse}


Another copy of the above will be found in a manuscript in the
library of Trin. Coll., Dublin (F. 5. 13,. fol 40).

In the same year (1609), Ravenscroft printed “Deuteromelia; or
the second part of Musick’s Melodie, or melodious musicke of pleasant
Roundelaies, K[ing] H[enry’s] Mirth or Freemen’s Songs, and such
delightfull Catches.” To this he affixes the motto, “Qui canere
potest canat—Catch that catch can.” It contains fourteen Freemen’s
Songs and seventeen Rounds or Catches. His third Collection was
“Melismata: Musical Phansies fitting the Court, Citie, and Countrey
humours,” consisting of “Court Varieties,” “Citie Rounds, “Citie
Conceits,” “Country Rounds,” and “Country Pastimes.” 4to., 1611.

After an interval of forty years, appeared John Playford’s first
publication containing rounds and catches, under the title of “Musick
and Mirth, presented in a choice collection of Rounds and Catches for
three voices” (1651). This is now a scarce book, and perhaps the only
copy remaining is in the Douce Collection at Oxford. In 1652, he
printed “Catch that catch can, or a choice collection of Catches,
Rounds, and Canons for 3 or 4 voices: collected and published by John
Hilton, Batch, in Musick;” and in the same year appeared “A Banquet
of Musick, set forth in three several varieties of musick: first,
Lessons for the Lyra Violi; the second, Ayres and Jiggs for the
Violin; the third, Rounds and Catches: all which are fitted to the
capacity of young practitioners in Music.” The last is also a scarce
work, the only known copy being in the Douce Collection.

Both Ravenscroft and Hilton give punning prefaces to their
books. The latter speaks of his as the times “when catches and
\textit{catchers} were never so much in
\end{fixedpage}%483
\pagebreak

\renewcommand\versoheadertext{catches, songs, and ballads.}
\begin{fixedpage}%484
\versoheader

request.” His collection became very popular, and a second
edition was printed in 1658. In 1667, Playford first published his
\textit{Musical Companion}, containing 143 catches and rounds, besides glees,
ayres, part-songs, \&c., in all 218 compositions. To this additions
were constantly made, and in 1685 he printed a second part, the
popularity of which carried through ten editions between that year
and 1730. The fourth edition, printed in 1701, contains fifty-three
catches composed by Henry Purcell, and eleven by Dr. Blow.\textsuperscript{a}

In and after the reign of Charles II., the best composers did not
disdain to write catches; but if the \textit{great} masters of Elizabeth’s
reign wrote any, they did not care to print them, for although there
are numberless canons of that period extant, and in every form of
that species of composition, I do not recollect to have seen a single
catch of their production.\textsuperscript{b}

Publication was then attended with little pecuniary advantage to
authors. Not a fiftieth part of the music we know to have been
composed by celebrated musicians was printed; and when an author was
induced to publish his works, he commonly assigned such reasons as
“the solicitation of numerous friends,” or “the many incorrect
copies” that were in circulation. The sum to be received from a
publisher was evidently small in proportion to what might be derived,
in the form of presents for copies, so long as the work remained in
manuscript; and the transcription of music required much less time
than an ordinary book. When Playford and Carr published \textit{The Theater
of Music}, in 1685, in a preface to that collection they solicited the
composers whose works they were printing to leave copies of all their
new songs, “under their own hands,” either at the shop of the one or
of the other; promising, in return,—not to pay the authors, but
“faithfully to print from such copies; whereby they may be assured to
have them perfect and exact.” Composers were expected, “in justice to
themselves, easily to grant” this favour, and so to “prevent such as
daily abuse them by publishing their songs lame and imperfect, and
singing them about the streets like ordinary ballads.” It was a great
indignity to an author to rank his works with ballad-tunes, and
Playford reproves a pupil of Mr. Birkenshaw, as “an ignorant
pretender to musick,” for having asserted that there were only three
good songs in his third book of \textit{Choice Ayres}, and that “the rest were
worse than common ballads sung about the streets by footboys and
linkboys.”

Old Thomas Mace was perhaps the only musician of the time who had
a word

\begin{dcfootnote}
\textsuperscript{a} See Notes on the Hon. Roger North’s \textit{Memoires of Musick}, by
Edward F. Rimbault.

\textsuperscript{b} The method of making rounds or catches is so simple, that I
shall here transcribe Christopher Simpson’s directions for composing
them. These will be found at the end of his \textit{Compendium or
Introduction to Practical Music}. After teaching all the “Contrivances
of Canon,” he says, “I must not omit another sort of Canon, in more
request and common use, though of less dignity, than all those which
we have mentioned; and that is a Catch or Round: some call it a Canon
in Unison; or a Canon consisting of Periods. The contrivance thereof
is not intricate; for if you compose any abort strain in three or
four parts, setting them all within the ordinary compass of the
voice; and then place one part at the end of another, in what order
you please, so that they may aptly make one continued tune, you
have finished a Catch.” He prints an example in score, and then the
same written out with the mark of the period where another voice is
to follow. That is equally exemplified in “Row the boat, Whittington”
(ante p. 482). The two bars are the score compressed; the six bars
are the three parts written out in the order they are to he sung. I
have already said that the only difference between a catch and a
round is that the former has some catch, or cross-reading in the
words,—some “latent meaning or humour, produced by the manner in
which the composer has arranged the words for singing, which would
not appear on perusing them.” See Note at p. 108.
\end{dcfootnote}
\end{fixedpage}%484
\pagebreak

\begin{fixedpage}%485
\rectoheader

to say in favour of ballad-tunes; for learned counterpoint and
skilful harmony were far more highly valued by professed musicians
than simple melodies. In his quaint and charming book, called
Musick's Monument (1676), after describing preludes, fancies, pavans
(“very grave and sober; full of art and profundity; but seldom used
in these light days”), galliards, corantos, sarabands, jigs, \&c.,
Mace speaks of “common tunes,” which “are to be known by the boys and
common people singing them in the streets and says, that “among them
are many very excellent and well-contrived;” that they have “neat and
spruce ayre,” and “in either sort of time,” common or triple.

He tells us that the theorbo, a large lute, of which an engraving
is given, is “no other than that which we called the old English
lute,” and that “in despite of fickleness and novelty, it was still
made use of in the best performances of music, viz., in vocal.” For
instrumental music, a lute of smaller size was used, because the neck
of the theorbo was so long that the strings could not be drawn up to
a sufficiently high pitch, and it could only be managed by tuning one
string to the octave. “Know,” says he, “that an old lute is better
than a new one and “you shall do well, ever when you lay it by in the
day-time, to put it into a bed that is constantly used, between the
rug and the blanket, but never between the sheets, because they may
be moist. This is the most absolute and best plan to keep it always.
There are many great commodities in so doing; it will save your
strings from breaking; it will keep your lute in good order, so that
you shall have but small trouble in tuning it; it will sound more
brisk and lively, and give you pleasure in very handling of it; if
you have any occasion extraordinary to set your lute at a higher
pitch, you may do it safely, which otherwise you cannot so well do,
without danger to your instrument and strings; it will be a great
safety to your instrument, in keeping it from decay; it will prevent
much trouble in keeping the bars from flying loose, and the belly
from sinking: and these six conveniences, considered all together,
must needs create a seventh, which is, that lute-playing must
certainly be very much facilitated, and made more delightful thereby.
Only no person must be so inconsiderate as to tumble down upon the
bed whilst the lute is there, for I have known several good lutes
spoilt with such a trick.. . . Take notice that you strike not your
strings with your nails, as some do, who maintain it the best way of
playing, but I do not, and for this reason: because the nail cannot
draw so sweet a sound as the nibble end of the flesh can do. I
confess, in a concert, it might do well enough, where the mellowness
(which is the most excellent satisfaction from a lute) is lost in the
crowd; but alone, I could never receive so good content from the nail
as from the flesh.”

Mace had seen two old lutes, “pitiful, battered, cracked things,”
which were Valued at 100\textit{l}. a piece. Charles II. had paid that sum for
the one, and the other was the property of Mr. Edward Jones, who
being minded to dispose of it, made a bargain with a merchant that
desired to have it with him in his travels, that, on his return, he
should either pay Mr. Jones 100\textit{l}. as the price, or 20\textit{l}. “for his
experience and use of it” during the voyage. Yet lutes of three or
four pounds a-piece were “more illustrious and taking to a common
eye.”
\end{fixedpage}%485
\pagebreak

\renewcommand\versoheadertext{lutes, viols, virginals, etc.}

\begin{fixedpage}%486
\rectoheader

“Of viols,” says Mace,“there are no better in the world than
those of Aldred, Jay, and Smith, yet the highest in esteem are Bolles
and Ross; one bass of Bolles I have known valued at 100\textit{l}. These were
old; but we have now very excellent good workmen, who no doubt can
work as well as those, if they be so well paid for their work as they
were; yet we chiefly value old instruments before new; for, by
experience, they are found to be far the best.”

A hundred pounds for a lute, and the same for a viol, were quite
as large sums, in relation to the comparative value of money, as are
now occasionally paid for Cremona violins of the best makers of the
sixteenth century; but the expenditure upon music generally was
certainly greater, in proportion to our wealth, in the seventeenth
than in the present century. Evelyn tells us that when Sir Samuel
Morland was blind, he “buried 200\textit{l}. worth of music-books six feet
under ground; being, as he said, love-songs and vanity.” This was a
considerable sum for an amateur to spend in books of vocal music
only; and as he continued to play “psalms and religious hymns on the
theorbo,” it may be presumed that what, was interred formed but a
portion of his vocal library.

During the great fire of London in 1666, Pepys, who was an
eye-witness, tells us that, the river Thames being full of lighters
and boats taking in goods, he “observed that hardly one lighter or
boat in three, that had the goods of a house, but there was a pair of
virginals in it.” As these were principally for the use of the fair
sex, the cultivation of music could not have declined among them to
any great extent, in spite of the long reign and depressing influence
of puritanism; or else the revival must have been singularly rapid.
The virginals, spinet, and harpsichord (or harpsicho\textit{n}, as it was
about this time more generally called), were the precursors of the
pianoforte; and, although differing from one another in shape, and
somewhat in interior mechanism, were essentially the same instrument.
Two “pairs of virginals,” manufactured in London, are now in the
possession of Mr. T. Mackinlay; the one pair by John Loosemore (the
builder of the organ of Exeter Cathedral), bearing the date of 1655,
and the second by Adam Leversidge, made in the year of the fire. In
shape they resemble “square” pianofortes; but the lids, instead of
being flat, are elevated in the centre, and are in three long pieces.
The compass of the second is from A to F, —rather less than five
octaves. The interiors of the lids are decorated by paintings.

To connect the history of the cultivation of music among ladies,
from the reign' of Elizabeth to that of Charles, it may suffice to
quote from two authors; and, as Dr. Nott truly says, “From old plays
are chiefly to be collected the manners of private life in the
sixteenth and seventeenth centuries,” the first shall be a dramatist.
Middleton’s play, \textit{A Chaste Maid}, 1630, opens with this question,
addressed by the goldsmith’s wife to her daughter: “Moll, have you
played over all your old lessons o’the virginals?” In his \textit{Michaelmas
Term}, 1607, Quomodo, the hosier, desires his daughter to leave the
shop, and to “get up to her virginals.” In his \textit{Roaring Girl}, 1611,
Sir Alexander asks Moll, “You can play
\end{fixedpage}%486
\pagebreak

\begin{fixedpage}%487
\rectoheader

any lesson?” and is answered, “At first sight, Sir.” In his
\textit{Women, beware Women}— 

\settowidth{\versewidth}{I’ve brought her up to music, dancing, and what not}
\begin{scverse}
\vleftofline{“}I’ve brought her up to music, dancing, and what not,\\
That may commend her sex, and stir her a husband.”
\end{scverse}

To the same purport writes Burton, in his \textit{Anatomy of Melancholy}.
Under the head of “Love Melancholy—Artificial Allurements,” he says,
“A thing nevertheless frequently used, and part of a gentlewoman’s
bringing up, is to sing, dance, play on the lute or some such
instrument, before she can say her \textit{Pater noster} or ten commandments:
’tis the next way their parents think to get them husbands, they are
compelled to learn.” But when they were married, “We see this daily
verified in our young women and wives, they that being maids took so
much pains to sing, play, and dance, with such cost and charge to
their parents to get these graceful qualities, now, being married,
will scarce touch an instrument, they care not for it.”

Of the opposite sex Burton says, “Amongst other good qualities an
amorous fellow is endowed with, he must learn to sing and dance, play
upon some instrument or other, as without doubt he will, if he be
truly touched with the loadstone of love: for, as Erasmus hath it,
\textit{Musicam docet amor, et poesin}, love will make them musicians, and to
compose ditties, madrigals, elegies, love-sonnets, and sing them to
several pretty tunes, to get all good qualities may be had.” He also
tells us that many silly gentlewomen are won by “gulls and swaggering
companions that have nothing in them but a few players’ ends and
compliments; that dance, \textit{sing old ballet tunes}, and wear their
clothes in fashion with a good grace;—a sweet fine gentleman! a
proper man! who could not love him?”

To return to Charles the Second’s reign, I may again quote Pepys,
who not only sang at sight, but also played upon the lute, the viol,
the violin, and the flageolet; learnt to compose music; had an organ
and pair of virginals or harpsichord in his house, and had a
thoroughly musical household. And yet, when a young man, was so
vehement a roundhead as to say on the day Charles I. was executed,
that, were he to preach upon him, his text should be “The memory of
the wicked shall rot.” His Diary abounds with amusing passages about
music, as a few brief extracts will prove. And, firstly, as to
himself.

Nov. 21, 1660. “At night to my viallin, in my dining roome, and
afterwards to my lute there, and I took much pleasure to have the
neighbours come forth into the yard to hear me.” Dec. 3. “Rose by
candle and spent my morning in fiddling till time to go to the
office.” 28th. “Staid within all the afternoon and evening at my lute
with great pleasure.” In the cellars at Audley End, “played on my
flageolette, there being an excellent echo;” and again, “I took my
flageolette and played upon the leads in my garden, when Sir W. Pen
came, and there we stayed talking and singing, and drinking great
draughts of claret.” On Sundays we find him joining with others to
sing Ravenscroft’s or Lawes’ Psalms, or else taking part in cathedral
service or an anthem. After morning prayers, “To Gray’s Inn Walk, all
alone, and with great pleasure seeing all the fine ladies walk there.
Myself humming to myself the \textit{trillo}, which now-a-days is my constant
practice since I begun to learn to sing, and find by use that it do
\end{fixedpage}%487
\pagebreak

\renewcommand\versoheadertext{music in private life.}

\begin{fixedpage}%488
\rectoheader

come upon me.” Once he says, “Lord’s day\dots composing some
ayres. God forgive me!”

According to the custom of those days, Pepys frequently dined at
taverns or ordinaries, generally choosing those where the best
musicians were to he heard. He mentions the Dolphin Tavern as having
“an excellent company of fiddlers,” and his being there, on more than
one occasion, “exceeding merry till late.” But, a year or two after,
being invited to dine there by Mr. Foly, and an excellent dinner
provided, he tells us, “but I expected musique, the missing of which
spoiled my dinner.”

Licenses were not then required for the performance of music at
taverns, as now; and Killigrew says that no ordinary fiddlers of any
country were so well paid as our own. According to Heylin, in his
\textit{Voyage of France}, 1679, the custom, at Tours, was for each man at
table to pay the fiddlers a sou; “they expect no more, and will take
no less.” In English country towns a groat for “a fit of mirth” had
long been the remuneration of the minstrel; and (according to a
ballad of this time) each villager, male or female, gave two-pence
for a dance on the green; but Pepys speaks of paying four shillings
on one occasion at the Dolphin, and 3\textit{l}. for four musicians,—“the Duke
of Buckingham’s music, the best in town,”—for a dance at his own
house. Their instruments were two violins, a base, and a theorbo.

Under the date of November 16, 1667, Pepys says, “To White Hall,
and there got into the theatre-room, and there heard both the vocall
and instrumentall musick; where the little fellow (Pelham Humphrey,
the composer) stood keeping time.” Conductors to bands are
therefore “of no modern introduction; and he even mentions a case in
which that office was held by a woman. On the 6th of June, 1661,
“Lieutenant Lambert and I went down by water to Greenwich, and eat
and drank and heard musique at the Globe, and saw the simple motion
that is there of a woman with a rod in her hand keeping time to the
musique while it plays, \textit{which is simple, methinks}.”

In one instance he dines at a club, where they have three voices
to sing catches. This is probably one of the earliest notices of
clubs in England.

His position as a clerk at the Admiralty threw him much into the
society of naval officers, and his own taste into that of
antiquaries. Meeting Ashmole in the morning at the house of Lilly,
the astrologer, they stay and sing duets and trios in Lilly’s study.
We are told that Evelyn and all his family were lovers of music, and
well skilled in the art. Evelyn also mentions his daughter Mary as
having “substantial and practical knowledge in ornamental arts of
education, especially music, both vocal and instrumental.”

We find the tedium of naval life to have been often relieved by
music;—that one captain kept a harper; another was “a perfect good
musician;” a third “a merry man that sang a pleasant song
pleasantly;” that one lieutenant played the cittern, and another, who
was “in a mighty vein of singing,” had “a very good ear and strong
voice, but no manner of skill.” Sets of viols or violins were
sometimes kept on board, because Pepys tells us, while the Nazeby was
lying off

\end{fixedpage}%488
\pagebreak

\begin{fixedpage}%489
\rectoheader

Deal, Mr. North, son of Sir Dudley North, came on board, and “did
play his part exceeding well at first sight.”

Pepys’ household included a maid to wait upon his wife, and a boy
to attend upon him. In the course of the Diary, which extends over
about nine years and a half, four maids are mentioned, and all
possessed of some skill in music. Of the first he says (Nov. 17,
1662), “After dinner, talking with my wife, and making Mrs. Gosnell
sing\dots I am mightily pleased with her humour and singing.” And
again, Dec. 5, “she sings exceeding well.” Within a few months,
Gosnell was succeeded by Mary Ashwell, who had been brought up at
Chelsea school, and he tells us in March, “I heard Ashwell play first
upon the harpsichon, and I find she do play pretty well. Then home by
coach, buying at the Temple the printed virginall book for her.” Of
the third, Mary Mercer, “a pretty, modest, quiet maid,” he says, on
Sep. 9,1664, “After dinner, my wife and Mercer, Tom (the boy) and I,
sat till eleven at night, singing and fiddling, and a great joy it is
to see me master of so much pleasure in my house. The girle (Mercer)
plays pretty well upon the harpsichon, but only ordinary tunes, but
hath a good hand: sings a little, but hath a good voyce and eare. My
boy, a brave boy, sings finely, and is the most pleasant boy at
present, while his ignorant boy’s tricks last, that ever I saw.”
Again, May 5, 1666, “It being a very fine moonshine, my wife and
Mercer came into the garden, and my business being done, we sang till
about twelve at night, with mighty pleasure to ourselves and
neighbours by their casements opening.”

After some time, Mercer went out to see her mother, and Mrs.
Pepys, finding her absent without having asked permission, followed
her to the house and beat her in her mother’s presence. It was the
custom of ladies to beat their servants in those days. The mother
having urged that her daughter was “not a common prentice girl,” Mrs.
Pepys construed it into a question of her right to inflict corporeal
chastisement, and therefore, when Mary Mercer returned home, she was
dismissed.

In October, 1666, says Pepys, “My wife came home, and hath
brought her new girle I have helped her to, of Mr. Falconbridge’s.
She is wretched poor and but ordinary favoured, and we fain to lay
out seven or eight pounds’ worth of clothes upon her back, which,
methinks, do go against my heart: and do not think I can ever esteem
her as I could have done another, that had come fine and handsome;
and which is more, her voice, for want of use, is so furred that it
do not at present please me; but her manner of singing is such that I
shall, I think, take great pleasure in it.”

Within a short time, Mercer was taken back, and we hear
constantly of trips by water to Greenwich, \&c., and then of singing
on the water, especially when returning by moonlight. The boy, Tom
Edwards, was usually of the party. Of him, Pepys says (Oct. 25,1664),
“My boy could not sleep, but wakes about four o’clock in the morning,
and in bed laying playing on his lute till daylight, and it seems did
the like last night till twelve o’clock.” And again, Dec. 26, 1668,

\end{fixedpage}%489
\pagebreak

\renewcommand\versoheadertext{italian and scotch music.}

\begin{fixedpage}%490
\versoheader

“After supper I made the boy play upon his lute, and so, my mind
in mighty content, to bed.”

Pepys evidently selected servants that could both sing and play,
but it is certain that there was no great difficulty in procuring
them. If further proof were required, I might quote the dramatists of
the time, who, as in Shirley’s \textit{Court Secret}, commonly attribute to
the servants in their plays the ability to sing “at first sight.”
Pepys’ own taste was not fashionable, for on hearing a celebrated
piece of music by Carissimi, he says, “Fine it was indeed, and too
fine for me to judge of.” And again, on hearing Mrs. Manuels sing
with an Italian, he says, “Indeed she sings mightily well, and just
after the Italian manner, but yet do not please me like one of Mrs.
Knipp’s songs, to a good English tune, the manner of their ayre not
pleasing me so well as the fashion of our own, nor so natural.”

He first speaks of Scotch music in the year 1666, and it would
seem to have been then a novelty. In January he hears Mrs. Knipp, the
actress, sing, “her little Scotch song of \textit{Barbary Allen},” at Lord
Brouncker’s, and he was “in perfect pleasure to hear her sing” it. In
the following July, he says, “To my Lord Lauderdale’s house to speak
with him, and find him and his lady, and some Scotch people, at
supper. But at supper there played one of their servants upon the
viallin some Scotch tunes only; several, and the best of their
country, as they seem to esteem them, by their praising and admiring
them: but, Lord! the strangest ayre that ever I heard in my life, and
all of one cast.” His third and last notice of Scottish music is in
June, 1667. “Here in the streets I did hear the Scotch march beat by
the drums before the soldiers, which is very odd.”

The first Scotch tunes that I have found printed in England are
among the “Select new Tunes and Jiggs for the Treble Violin,” which
were added to \textit{The Dancing Master} of 1665. These are “The Highlanders’
March,” “A Scotch Firke,” and “A Scots Rant.” They are not included
among the country-dances in that publication; neither do they appear
in any other edition. The “Select new tunes” were afterwards
transferred to \textit{Apollo’s Banquet for the Treble Violin}.”

In \textit{The Dancing Master} of 1686 we find the first Scotch tune
arranged as a country-dance.” This is “Johnny, cock thy beaver,”
which had been rendered popular by Tom D’Urfey’s song, “To horse,
brave boys, to Newmarket, to horse,” being written to it. On the
other hand, the first collection of secular music printed in
Scotland, Forbes’ \textit{Cantus}, consists entirely of English compositions,
and songs to English ballad-tunes. The first edition was published in
1662, the second in 1666, and the third in 1682. “Severall of the
choisest Italian
songs and new English Ayres in three parts” were added to the
last, and, with that exception, all are for one voice. Forbes was a
printer at Aberdeen, and this was the only secular music published in
Scotland during the seventeenth century.

\centerrule

\end{fixedpage}%490
\pagebreak

\begin{fixedpage}%491
\rectoheader

\musictitle{THE TWENY-NINTH OF MAY.}

The following song “On the King’s Birthday, May 29,” (on which
day Charles the Second entered London after his restoration), is from
a copy printed in 1667.

The spirited tune is to be found in \textit{The Dancing Master} of 1686,
and in every subsequent edition, under the title of \textit{The twenty-ninth
of May}. In several of the editions it is printed twice; the second
copy being under another name. For instance, in the “Additional
Sheet” to \textit{The Dancing Master} of 1686, it appears as \textit{May Hill}, or \textit{The
Jovial Crew}; in “The Second Part” of that of 1698, as \textit{The Jovial
Beggars}; in the third volume of The Dancing Master, \textsc{n.d.}, as the
\textit{Restoration of King Charles}.

It also bears the name of \textit{The Jovial Crew} in \textit{Apollo’s Banquet for
the Treble Violin}.
\end{fixedpage}%491
\pagebreak

\renewcommand\versoheadertext{english song and ballad music.}

\begin{fixedpage}%492
\versoheader

\musictitle{HERE’S A HEALTH UNTO HIS MAJESTY.}
This was a very popular loyal song in the reign of Charles II. It
is twice mentioned by Shadwell in his plays. Firstly, in \textit{The Miser}
(1672), where Timothy says, “We can be merry as the best of you—we
can, i’faith—and sing \textit{A boat, a boat} [\textit{haste to the ferry}], or \textit{Here’s
a health to his Majesty, with a fa, la, la, lero};” and secondly, in
his \textit{Epsom Wells} (1673), where Bisket says, “Come’ let’s all be
musitioners, and all roar and sing \textit{Here’s a health unto his Majesty,
with a fa, la, la, la, la, lero}.”

The words are in \textit{Merry Drollery Complete}, 1670; and words and
music together in Playford’s \textit{Musical Companion}, 1667, 1672, \&c.

Dr. Kitchener, in his \textit{Loyal and National Songs of England},
commits the singular mistake of printing the tenor part as the tune,
instead of the treble; and it is the more remarkable, because the
three parts, treble, tenor, and base, are printed on the same page of
the \textit{Musical Companion}. Another blunder is his ascribing it to
Jeremiah Savage, instead of Jeremiah Savile.
\end{fixedpage}%492
\pagebreak

\begin{fixedpage}%493
\rectoheader

\musictitle{GRIM KING OF THE GHOSTS.}

Black-letter copies of this ballad are to be found in the
Bagford, the Pepys, the Douce, and the Roxburghe Collections. It is
usually entitled “The Lunatick Lover: Or the Young Man’s call to Grim
King of the Ghosts for cure. To an excellent new tune.” Percy
reprinted it in his \textit{Reliques of Ancient Poetry}, and Ritson, in his
\textit{Select Collection of English Songs}; the first stanza will therefore
suffice.

\settowidth{\versewidth}{Grim King of the Ghosts! make haste,}
\begin{dcverse}
\begin{altverse}
\vleftofline{“}Grim King of the Ghosts! make haste, \\
And bring hither all your train: \\
See how the pale moon does waste. \\
And just now is in the wane. \\
Come, you night hags, with all your charms.\\
And revelling witches, away,\\
And hug me close in your arms;\\
To you my respects I'll pay.”
\end{altverse}
\end{dcverse}


Among the ballads sung to the tune, are the following:—

1. “The Father’s wholesome Admonition: To the tune of \textit{Grim King
of the Ghosts}.” See Roxburghe Collection, ii. 165.

2. “The Subjects’ Satisfaction; being a new song of the
proclaiming King William and Queen Mary, the 13th of this instant
February, to the great joy and comfort of the whole kingdom. To the
tune of \textit{Grim King of the Ghosts}, or \textit{Hail to the myrtle shades}.” See
Roxburghe Collection, ii. 437.

3. “The Protestant’s Joy; or an excellent new song on the
glorious Coronation of King William and Queen Mary, which in much
triumph was celebrated at Westminster on the 11th of this instant
April. Tune of \textit{Grim King of the Ghosts}, or \textit{Hail to the myrtle
shades}.” This has a woodcut intended to represent the King and Queen
seated on the throne. See Bagford Collection (643, m. 10, p. 172,
Brit. Mus.) “Printed for J. Deacon, in Guiltspur Street.” It
commences thus:—-

\settowidth{\versewidth}{Has prov’d the best day in the year.}
\begin{dcverse}
\begin{altverse}
\vleftofline{“}Let Protestants freely allow \\
Their spirits a happy good cheer, \\
Th’ eleventh of April now, \\
Has prov’d the best day in the year. \\
Brave boys, let us merrily sing,\\
While smiling full bumpers go round;\\
Hear joyful good tidings I bring.\\
King William and Mary are crown’d.”
\end{altverse}
\end{dcverse}



The tune was introduced into \textit{The Beggars’ Opera}, \textit{The Devil to
pay}. \textit{The Oxford Act}, and other ballad-operas; also printed in Watts’
\textit{Musical Miscellany},
i. 126 (1729) to a song entitled “Rosalind’s Complaint,”
commencing, “On the bank of a river so deep.”

It was to this air that Rowe wrote his celebrated song “Colin’s
Complaint;” in which, according to Dr. Johnson, he alluded to his own
situation with the Countess Dowager of Warwick, and his successful
rival, Addison. Goldsmith, in his preface to \textit{The Beauties of English
Poetry}, says, “This, by Mr. Rowe, is better than anything of the kind
in our language.” It commences—

\settowidth{\versewidth}{And while a false nymph was his theme,}
\begin{dcverse}
\begin{altverse}
\vleftofline{“}Despairing beside a clear stream, \\
A shepherd forsaken was laid; \\
And while a false nymph was his theme, \\
A willow supported his head: \\
The wind that blew over the plain,\\
To his sighs with a sigh did reply;\\
And the brook, in return to his pain,\\
Ran mournfully murmuring by.”
\end{altverse}
\end{dcverse}


It has been reprinted in Ritson’s English Songs, and in many
other collections. There are several parodies; one of which is
contained in “A complete Collection
\end{fixedpage}%493
\pagebreak

\begin{fixedpage}%494
\versoheader

of old and new English and Scotch Songs, with their respective
tunes prefixed.” 8vo., 1785, p. 82. It commences—


\settowidth{\versewidth}{By the side of a great kitchen fire,}
\begin{dcverse}
\begin{altverse}
\vleftofline{“}By the side of a great kitchen fire, \\
A scullion so hungry was laid; \\
A pudding was all his desire,\\
A kettle supported his head.”
\end{altverse}
\end{dcverse}


Many of my readers will recollect another, attributed to Canning,
commencing— 

\settowidth{\versewidth}{By the side of a neighbouring stream, }
\begin{dcverse}
\begin{altverse}
\vleftofline{“}By the side of a neighbouring stream, \\
As an elderly gentleman sat; \\
On the top of his head was his wig, \\
On the top of his wig was his hat.”
\end{altverse}
\end{dcverse}


The tune is also well known, from a song called \textit{The Lover’s
Mistake}, adapted to it by Balfe, and sung on the stage by the late
Madame Yestris.

When Gay introduced the air into \textit{The Beggars’ Opera}, he took the
first line (“Can love be controll’d by advice”) from the following
song by Mr. Berkeley. It is said to have been addressed to the once
well-known Viscountess Vane, whose history is related by Smollett in
the “Memoirs of a Lady of Quality,” introduced into his \textit{Peregrine
Pickle}. See the \textit{early} editions.

Dull wisdom but adds to our cares; 			Then, Molly, for what should we stay,
Brisk love will improve every joy; 				Till our best blood begins to run cold?
Too soon we may meet with gray hairs, 	Our youth we can have but to-day,
Too late may repent being coy. 					We may always find time to grow old.
\end{fixedpage}%494
\pagebreak

\begin{fixedpage}%495
\rectoheader

\musictitle{THE KING’S JIG.}

The dancing of jigs is now in a great measure confined to
Ireland; but they were formerly equally common in England and
Scotland. The word “jig” is said to be derived from the Anglo-Saxon,
and in old English literature its application extended, beyond the
tune itself, to any jigging rhymes that might be sung to such tunes.
The songs sung by clowns after plays (which, like those of Tarleton,
were often extempore,) and any other merry ditties, were called jigs.
“Nay, sit down-by my side, and I will \textit{sing} thee one of my countrey
jigges to make thee merry,” says Deloney, in his \textit{Thomas of Reading}.

Pepys speaks of his wife’s maid, Mary Mercer, as dancing a jig,
“the best he ever saw, she having the most natural way of it, and
keeping time most perfectly.” Heywood includes jigs among the dances
of the country people, in the following passage from \textit{A Woman killed
with kindness}:—

\settowidth{\versewidth}{Made with their high shoes: though their skill he small,}
\begin{scverse}
\vleftofline{“}Now, gallants, while the town musicians \\
Finger their frets\textsuperscript{a} within, and the mad lads\\
And country lasses, every mother’s child,\\
With nosegays and bride-laces in their hats,\\
Dance all their country measures, rounds, and jigs,\\
What shall we do?—Hark I they’re all on the hoigh; \textsuperscript{b}\\
They toil like mill-horses, and turn as round—\\
Marry, not on the toe. Aye, and they caper—\\
But not without cutting; you shall see to-morrow\\
The hall floor peck’d and dinted like a millstone,\\
Made with their high shoes: though their skill be small,\\
Yet they tread heavy where their hobnails fall.”
\end{scverse}

Jigs, however, were danced by persons of all ranks during the
latter half of the seventeenth century; and this having been
published as the \textit{The King’s Jig}, during the life of Charles II., we
may suppose it to be one of the tunes to which his majesty danced.
The jigs of the Inner Temple, the Middle Temple, Lincoln’s Inn,
Gray’s Inn, and many others, are to be found in the editions of
\textit{Apollo’s Banquet for the Treble Violin}, printed in this, and the
following reign.

D’Urfey wrote a descriptive song called “The Winchester Wedding;
set to \textit{The King’s Jigg}, a Country. Dance;” and it was published, with
the tune, among “Several new Songs by Tho. D’Urfey, Gent., set to as
many new tunes by the best masters in music,” fol., 1684. It became
very popular, was printed as a penny ballad, and the tune became
better known as \textit{The Winchester Wedding} than as \textit{The King’s Jig}. It is
to be found, under the one name or the other, in \textit{The Dancing Master}
of 1686, and every subsequent edition; in \textit{Pills to purge Melancholy} ;
and in many of the ballad-operas. The copies in the \textit{Pills}, and some
others, are very incorrectly printed.

Among the ballads that were sung to the tune, I have already
quoted one, printed in July, 1685, “On the Virgins of Taunton Dean,
who ript open their

\begin{dcfootnote}
\textsuperscript{a} \ie Play instruments that have frets, like viols and lutes,
or such as guitars still have.

\textsuperscript{b} Quære “dancing the Hey.”
\end{dcfootnote}

\end{fixedpage}%495
\pagebreak


\begin{fixedpage}%496
\versoheader

silk petticoats to make colours for the late D[uke] of
M[onmouth]’s army” (ante p. 444). It commences—

\settowidth{\versewidth}{Came there to make M[onmonth] king.”}
\begin{dcverse}
\vleftofline{“}In Lime began a rebellion. \\
For there the rebels came in; \\
Rebels, almost a million,\\
Came there to make M[onmonth] king.”
\end{dcverse}

and there are many others, such as “A Fairing for young men
and maids” (Roxburghe Collection, ii. 162), \&c.

Ritson reprinted \textit{The Winchester Wedding} in his \textit{Ancient Songs},
from a black-letter ballad in the British Museum, but apparently
without knowing it to have been written by D’Urfey. It is scarcely
reprintable now, and therefore the following first stanza must
suffice.
\end{fixedpage}%496
\pagebreak

\begin{fixedpage}%497
\rectoheader

\musictitle{SARABAND.}

When in exile, Charles II. wrote to Henry Bennet to. bring him as
many new corantos and sarabands, and other little dances, as he could
get written down. The following specimen of a saraband is from \textit{The
Dancing Master} of 1665:—

From the following passage in Sir W. Davenant’s \textit{Law against
Lovers} (which is a mixture of the two plots of Shakespeare’s \textit{Measure
for Measure} and \textit{Much Ado about Nothing}) it would appear that the
dancer of the saraband accompanied it with castanets.
\end{fixedpage}%497
\pagebreak

\begin{fixedpage}%498
\versoheader

\settowidth{\versewidth}{Page, call Viola with her castanietos,}
\begin{scverse}
\vleftofline{\textit{Beatrice}. “}Page, call Viola with her castanietos,\\
And bid Bernardo bring his guittar.”\\
\vin\vin\vin\vin\vin\vin\vin (\textit{Viola strikes the castaniets within}.)\\
\vleftofline{\textit{Benedict}. “}Those castanietos sound\\
Like a consort of squirrels cracking of nuts.”\\
\vin(\textit{Enter Viola dancing a saraband awhile with castanietos}.)
\end{scverse}

There are no directions for the use of castanets in \textit{The Dancing
Master}, because the tunes are there intended for country dances.

\musictitle{THE DELIGHTS OF THE BOTTLE.}

In its original form, this was a song, sung by Bacchus, in the
last act of Shadwell’s opera, \textit{Psyche}, and the music, by Matthew Lock.
Shadwell wrote but two stanzas, and as that would have been too short
for a ballad, some ballad-monger lengthened it into twelve. A copy
will be found in the Roxburghe Collection (ii. 106), containing five
stanzas in the first part, and seven in the second. The tune is there
described as “a most admirable new tune, everywhere much in request.”

Playford printed the song in his \textit{Choice Ayres} (omitting the
chorus); and it was arranged as a duet for his \textit{Pleasant Musical
Companion} (book ii., 2nd edit., 1687). The words are also contained
in the \textit{Antidote to Melancholy}, 1682.

In the Roxburghe Collection, iii. 188, is “The Prodigal Son
converted; Or the young man returned from his ramble,” \&c.; “To a
pleasant new playhouse tune, called The Delights of the
Bottle.” “London, printed for R. Burton, at the Horse-shoe in West
Smithfield.” It commences—

\settowidth{\versewidth}{The delights and the pleasures}
\begin{scverse}
\begin{altverse}
“The delights and the pleasures\\
Of a man without care.”
\end{altverse}
\end{scverse}

In the same Collection, iii. 244, is a ballad on the Customs duty
imposed upon French wines, dated 1681, and entitled “The Wine
Cooper’s Delight;” to the tune of \textit{The Delights of the Bottle}.
“Printed for the Protestant Ballad Singers.” This is also in the
\textit{Collection of} 180 \textit{Loyal Songs}, 1685 and 1694, p. 183. It consists of
sixteen stanzas, commencing, “The delights of the bottle are turn’d
out of doors.”

There are several other ballads extant, which were to be sung to
the tune, and among them, the following, which is in the Pepys
Collection (i. 474). It was printed for P. Brooksby, and licensed by
Roger L’Estrange; therefore the copy cannot be of later date than the
reign of James II., and is more probably of that of Charles II.

“\textsc{Old Christmas returned}, or Hospitality revived; Being a
Looking-glass for rich misers, wherein they may see (if they be not
blind) how much they are to blame for their penurious house-keeping;
and likewise an encouragement to those noble-minded gentry who lay
out a great part of their estate in hospitality, relieving such
persons as have need thereto:

\settowidth{\versewidth}{Who feasts the poor, a true reward shall find,}
\begin{scverse}
Who feasts the poor, a true reward shall find,\\
Or helps the old, the feeble, lame, and blind.”
\end{scverse}

To the tune of \textit{The Delights of the Bottle}.
\end{fixedpage}%498
\pagebreak

\setlength{\fixedpagewidth}{385pt}
\begin{fixedpage}%499
\rectoheader

\settowidth{\versewidth}{Plum-pudding, goose, capon, minc’d pies, and roast-beef.}
\begin{dcverse}
All you that to feasting and mirth are inclin’d,\\
Come, here is good news for to pleasure your mind,\\
Old Christmas is come for to keep open house,\\
He scorns to be guilty of starving a mouse: \\
Then come, boys, and welcome of diet the chief,\\
Plum-pudding, goose, capon, minc’d pies, and roast-beef.

A long time together he hath been forgot, \\
They scarce could afford for to hang on the pot;\\
Such miserly sneaking in England hath been,\\
As by our forefathers ne’er us’d to be seen;\\
But now he’s returned you shall have in brief,\\
Plum-pudding, goose, capon, minc’d pies, and roast-beef.

The times were ne'er good since Old Christmas was fled,\\
And all hospitality hath been so dead,\\
No mirth at our festivals late did appear, \\
They scarcely would part with a cup of March beer;\\
But now you shall have for the ease of your grief,\\
Plum-pudding, goose, capon, minc’d pies, and roast-beef.

The butler and baker, they now may be glad,\\
The times they are mended, though they  have been bad;\\
The brewer, he likewise may be of good cheer,\\
He shall have good trading for ale and strong beer,\\
All trades shall be jolly, and have for relief,\\
Plum-pudding, goose, capon, minc’d pies, and roast-beef.

The holly and ivy about the walls wind, \\
And show that we ought to our neighbours be kind,\\
Inviting each other for pastime and sport, \\
And where we best fare, there we most do resort,\\
We fail not of victuals, and that of the chief,\\
Plum-pudding, goose, capon, minc’d pies, and roast-beef.

The cooks shall be busied by day and by night,\\
In roasting and boiling, for taste and delight;\\
Their senses in liquor that’s nappy they’ll steep,\\
Though they be afforded to have little sleep;\\
They still are employed for to dress us, in brief,\\
Plum-pudding, goose, capon, minc’d pies, and roast-beef.

Although the cold weather doth hunger provoke,\\
’Tis a comfort to see how the chimneys do smoke;\\
Provision is making for beer, ale, and wine.\\
For all that are willing or ready to dine; \\
Then haste to the kitchen for diet the chief,\\
Plum-pudding, goose, capon, minc’d pies, and roast-beef.

All travellers as they do pass on their way, \\
At gentlemen’s halls are invited to stay. \\
Themselves to refresh, and their horses to rest,\\
Since that he must be Old Christmas’s guest,\\
Nay, the poor shall not want, but have for relief\\
Plum-pudding, goose, capon, minc’d pies, and roast-beef.

Now Mock-beggar-hall it no more shall stand empty\\
But all shall be furnisht with freedom and plenty,\\
The hoarding old misers who us’d to preserve\\
The gold in their coffers, and see the poor starve,\\
Must now spread their tables, and give them in brief\\
Plum-pudding, goose, capon, minc’d pies, and roast-beef.

The court, and the city, and country, are glad\\
Old Christmas is come to cheer up the sad;\\
Broad pieces and guineas about now shall fly.\\
And hundreds be losers by cogging a die, \\
Whilst others are feasting with diet the chief,\\
Plum-pudding, goose, capon, minc'd pies, and roast-beef.

Those that have no coin at the cards for to play,\\
May sit by the fire, and pass time away, \\
And drink off their moisture contented and free,\\
“My honest good fellow, come, here is to thee,”\\
And when they are hungry, fall to their relief\\
Plum-pudding, goose, capon, minc’d pies, and roast-beef.

Young gallants and ladies shall foot it along,\\
Each room in the house to the music shall throng,\\
Whilst jolly carouses about they shall pass,\\
And each country'swain trip about with his lass;\\
Meantimes goes the caterer to fetch in the chief,\\
Plum-pudding, goose, capon, minc’d pies, and roast-beef.
\end{dcverse}
\end{fixedpage}%499
\pagebreak

\setlength{\fixedpagewidth}{360pt}
\begin{fixedpage}%500
\versoheader

\settowidth{\versewidth}{Plum-pudding, goose, capon, minc’d pies, and roast-beef.}
\begin{dcverse}
The cooks and the scullion, who toil in their frocks,\\
Their hopes do depend upon their Christmas box;\\
And few there are now that do live on the earth,\\
But enjoy at this time either profit or  mirth;\\
Yea, many are charged to give for relief,\\
Plum-pudding, goose, capon, minc’d pies, and roast-beef.

Then well may we welcome Old Christmas to town,\\
Who brings us good cheer, and good liquor so brown,\\
To pass the cold winter away with delight,\\
We feast it all day, and we frolic all night, \\
Both hunger and cold we keep out with relief,\\
Plum-pudding, goose, capon, minc’d pies and roast-beef.

Then let all curmudgeons who dote on their wealth,\\
And value their treasure much more than their health.\\
Go hang themselves up, if they will be so kind,\\
Old Christmas with them but small welcome shall find;\\
They will not afford to themselves without grief.\\
Plum-pudding, goose, capon, minc’d pies, and roast beef.
\end{dcverse}

The following is the original song from \textit{Psyche}, 4to., 1675. “In musick,”
says Roger North, “Matthew Lock had a robust vein,” of which the following is
rather characteristic.
\end{fixedpage}%500
\pagebreak

\begin{fixedpage}%501
\rectoheader

\settowidth{\versewidth}{Nor would kings rule the world, but for love and good drinking.}
\begin{dcverse}
Love and wine are the bonds that fasten us all, \\
The world, but for these, to confusion would fall:\\
Were it not for the pleasures of love and good wine.\\
Mankind for each trifle their lives would resign;\\
They’d not value dull life, nor could live without thinking,\\
Nor would kings rule the world, but for love and good drinking.
\end{dcverse}

\musictitle{BONNY NELL.}

From one of the earliest editions of Playford’s \textit{Apollo's Banquet},
without a title page, probably of 1670.

In \textit{Westminster Drollery}, 3rd edit., 1674, is a song beginning “A
blithe and bonny Country Lass and in the second stanza are these
lines:—

\settowidth{\versewidth}{She simpered much like \textit{bonny Nell}.”}
\begin{scverse}
\vleftofline{“}When as the wanton girl espied \\
The means to make herself a bride,\\
She simpered much like \textit{bonny Nell}.”
\end{scverse}

I suppose Nell Gwyn to be intended, and that this tune is also
named from her.

Dr. Richard Corbett, afterwards Bishop of Norwich, wrote some
verses to a tune of \textit{Bonny Nell}, which could not be sung to this air;
and, as Dr. Corbett was a singer, and not likely to mistake the
rhythm, I have no doubt of there having been another tune, under the
same name, and of earlier date. “After he was D.D.,” says Aubrey, “he
sang ballads at the Cross of Abingdon. On a market day, he and some
of his comrades were at the tavern by the Cross (which, by the way,
was then the finest in England), and a ballad singer complained that
he had no custom; he could not put off his ballads. The jolly Doctor
put off his gown, and put on the ballad singer’s leathern jacket;
and, being a handsome man, and having a rare full voice, he presently
had a great audience, and vended a large number of ballads.”

Dr. Corbett’s verses commence—

\settowidth{\versewidth}{It is not yet a fortnight since}
\begin{scverse}
\vleftofline{“}It is not yet a fortnight since \\
Lutetia entertain’d a prince;”
\end{scverse}

and are entitled “A grave Poem, as it was presented by certain divines by
way
\end{fixedpage}%501
\pagebreak

\begin{fixedpage}%502
\versoheader

of Interlude, before his Majesty in Cambridge, stil’d \textit{Liber novus
de adventu regis ad Cantabrigiam}, faithfully done into English, with
some liberal advantages, made rather to be sung than read, to the
tune of \textit{Bonny Nell}.” A copy in MSS. Ashmole, 36, 37, art. 271, and in
Nicholls’ \textit{Progresses of King James}, iii. 66, as well as “A Cambridge
Madrigal, confuting the Oxford Ballad that was sung to the tune of
Bonny Nell.”

Massinger alludes to some “Bonny Nell,” in his \textit{Old Law}, act iv.,
sc. 1, where the Cook says, “That Nell was Helen of Greece too;” and
Gnotho answers, “As long as she tarried with her husband, she was
Ellen; but after she came to Troy, she was Nell of Troy, or Bonny
Nell.” There is much punning on musicians in this scene;—as
“\textit{wire}-drawers” they are compared to \textit{wine}-drawers, both being governed
by pegs, both having pipes and sack-buts, only the heads differ; the
one hogsheads, the other cittern or gittern heads, but still each
wooden heads, \&c.

In the Pepys Collection, i. 70, is “A Battell of Birds most
strangly fought in Ireland upon the 8th day of September, 1621, where
neere unto the Citty of Corke, by the river Lee, were gathered
together such a multytude of Stares, or Starlings, as the like for
number was never seene in any age. To the tune of \textit{Shore's Wife}, or to
the tune of \textit{Bonny Nell}.” And in the same, iii. 124 (of Roxburghe, i.
84), another “to an excellent new tune, or to be sung to \textit{Bonny Nell},”
which commences —

\settowidth{\versewidth}{To view the meadows fresh and gay,}
\begin{dcverse}
\vleftofline{“}As I went forth one summer’s day, \\
To view the meadows fresh and gay, \\
A pleasant bower I espied, \\
Standing hard by the river’s side;\\
And in’t I heard a maiden cry,\\
Alas! there’s none e’er lov’d like I.”
\end{dcverse}

\end{fixedpage}%502
\pagebreak

\begin{fixedpage}%503
\rectoheader

\musictitle{THE LASS OF CUMBERLAND.}

The copies of this ballad and tune are still numerous. The tune
is in a manuscript in the Music School, Oxford, dated 1670,—in
180 \textit{Loyal Songs}, 1685 and 1694,—in \textit{Youth's Delight on the
Flagelet},—in several of the editions of \textit{Apollo's Banquet},—and in
every edition of \textit{Pills to purge Melancholy}.

In 180 \textit{Loyal Songs}, p. 219, is “The Creditors’ Complaint against
the Bankers; or, The Iron Chest the best Security:—

\settowidth{\versewidth}{Let ’em all break their necks—my money’s my own.}
\begin{scverse}
\begin{altverse}
Since bankers are grown so brittle of late,\\
That money and bankers together are flown,\\
I'll chest up my money; and then, 'spite of fate,\\
Let ’em all break their necks—my money’s my own.
\end{altverse}
\end{scverse}

To the tune of \textit{There was a Lass of Cumberland}.” It consists of ten stanzas; and commences:—

\settowidth{\versewidth}{Lest your coin to foreign lauds do pass.}
\begin{dcverse}
\begin{altverse}
\vleftofline{“}Bankers are now such brittle ware. \\
They break just like a Venice glass; \\
If you trust them, then have a care, \\
Lest your coin to foreign lauds do pass. \\
\textit{An iron chest is still the best,\\
’Twill heep your coin more safe than they.\\
For, when they’ve feather’d well their nest. \\
Then the rooks will fly away.’’}
\end{altverse}
\end{dcverse}


In the same collection are two on James II., then Duke of York.
The first, p. 176, “The honour of great York and Albany. To a new
tune.” The second, p. 177, “Loyalty respected, and Faction confounded.
To an excellent new tune.” The music of \textit{There was a Lass of Cumberland}
is printed as the tune in question. The last commences with the
line,—

\settowidth{\versewidth}{“Let the cannons roar from sea to shore.”}
\begin{scverse}
“Let the cannons roar from sea to shore.”
\end{scverse}

In the Roxburghe Collection, ii. 368, is “The Northern Lad; or,
The Fair Maid’s Choice, who refused all for a Plowman, counting
herself therein most happy, \&c. To the tune of \textit{There was a Lass in
Cumberland}.” The printer’s name is cut off this copy, which is a
version of the ballad differing from that in the \textit{Pills} and in the
Douce Collection. It commences:—

\settowidth{\versewidth}{And swore that they to woo me come.}
\begin{dcverse}
\begin{altverse}
\vleftofline{“}I am a lass o’ th’ North Countrey, \\
And I was born and bred a-whome; \\
Many a lad has courted me.\\
And swore that they to woo me come.\\
\textit{But to bed to me, to bed to me,\\
The lad that gangs to bed with me,\\
A jovial plowman must he be.\\
The lad that comes to bed to me.”}
\end{altverse}
\end{dcverse}

The Douce copy, p. 43, is entitled “Cumberland Nelly; or, The North
Country Lovers, \&c. Tune of
\textit{The Lass that comes to bed to me.}” It commences: 

\settowidth{\versewidth}{There was a lass, her name was Nell,}
\indentpattern{01014}
\begin{scverse}
\begin{patverse}
\vleftofline{“}There was a lass of Cumberland,\\
A bonny lass of high degree:\\
There was a lass, her name was Nell,\\
The blithest lass that e’er you see.\\
Oh! to bed to me, to bed to me,” \&c.
\end{patverse}
\end{scverse}

In the same collection, p. 44, is “Cumberland Laddy; or, Willy
and Nelly of the North” to the same tune. The first printed by J.
Conyers, at the Black Raven in Duck Lane,—the second by Coles, Vere,
Wright, and Clarke.

In \textit{Youth's Delight on the Flagelet}, the tune is entitled. \textit{To bed
to me; or, The Northern Lass}:—in \textit{Apollo's Banquet}, \textit{To bed we'll go}.
\end{fixedpage}%503
\pagebreak

\begin{fixedpage}%504
\versoheader

Another song entitled “The Cumberland Lass,” commencing—

\settowidth{\versewidth}{In Cumberland there dwells a maid,}
\begin{scverse}
\begin{altverse}
\vleftofline{“}In Cumberland there dwells a maid,\\
Her charms are past compare,” 
\end{altverse}
\end{scverse}
will be found in “A Complete
Collection of Old and New English and Scotch Songs,” i. 179, 8vo.,
1735. It is in the wrong metre for this tune.

\musictitle{COME, OPEN THE DOOR, SWEET BETTY.}

In the Pepys Collection, iii. 62, and in the Roxburghe, ii. 238,
are copies of the ballad, entitled “John’s earnest request; or,
Betty’s compassionate love extended to him in a time of distress. To
\textit{a pleasant new tune} much in request.” Printed for P. Brooksby, at the
Golden Ball in Pye Corner. It consists of nine stanzas, the first of
which is here printed to the tune.

This air will be found in the ballad operas of \textit{Flora}, 1729, \textit{The
Colliers' Opera}, 1729, and \textit{Achilles}, 1733. The following words,
adapted to it in \textit{Flora}, became popular, and were reprinted in \textit{The
Syren} (12mo., 1735), and other song-books. In \textit{The Livery Rake}, the
air is named from them.

\settowidth{\versewidth}{And before the next blush of Aurora}
\begin{dcverse}
\begin{altverse}
\vleftofline{“}O fly from this place, dear Flora, \\
Thy gaoler has set thee free, \\
And before the next blush of Aurora \\
You’ll find a kind guardian in me. \\
Dearest creature, exchange for the better,\\
Confinement can have no charms,\\
Think which of your prisons is sweeter, \\
This, or a young lover’s arms.”
\end{altverse}
\end{dcverse}

In Burns’ remarks on the songs in Johnson’s \textit{Scot's Musical
Museum}, he speaks of “old words” to “Blink o’er the burn, sweet
Betty,” and says, “All that I remember are—

\settowidth{\versewidth}{It’s a’ for the sake o’ sweet Betty,}
\begin{dcverse}
\begin{altverse}
Blink over the burn, sweet Betty, \\
It is a cauld winter night;\\
It rains, it hails, it thunders,\\
The moon she gives nae light. \\
It’s a’ for the sake o’ sweet Betty, \\
That ever I tint my way; \\
Sweet, let me lie beyond thee.\\
Until it be break o’ day.\\
O Betty will bake my bread,\\
And Betty will brew my ale. \\
And Betty will be my love \\
When I come over the dale. \\
Blink over the burn, sweet Betty,\\
Blink over the burn to me,\\
And while I hae life, dear lassie,\\
My ain sweet Betty thou’t be.”
\end{altverse}
\end{dcverse}

\end{fixedpage}%504
\pagebreak

\begin{fixedpage}%505
\rectoheader

The Scotch tune, “Blink over the burn, sweet Betty,”\textsuperscript{a} bears no
resemblance to “Come, open the door, sweet Betty,”—nor do the Scotch
words, in any early collection, resemble the English; but the song
quoted by Burns, and since adopted in Wood’s Songs of Scotland, is
evidently taken from the following ballad.

“Come, open the door, sweet Betty,” re-appears in the first part
of a tune called \textit{Tom Nokes’ Jigg}. The time is changed; it is quick,
and in \timesig{9}{8} measure,—but evidently from the same root. It is to be found
in the first edition of \textit{Apollo’s Banquet}, 1669. Tom Nokes (from whom
it derives its name) was a favourite actor in the reign of Charles
the Second. The following notice of Nokes and Nell Gwyn is from the
appendix to Downes’ \textit{Roscius Anglicanus}, edition of 1789

\begin{dcfootnote}
\textsuperscript{a} It has been stated that the first line of “Blink o’er the burn”
is quoted by Shakespeare in \textit{King Lear}, act iii., sc. 6:—

\settowidth{\versewidth}{Wantest thou eyes at trial, Madam?}
\begin{fnverse}
\begin{altverse}
\vleftofline{“}Wantest thou eyes at trial, Madam?\\
\textit{Come o'er the bourn, Bessy, to me};” 
\end{altverse}
\end{fnverse}
but the allusion is to an
English ballad by William Birch, entitled “A Songe betwene the Quenes
Majestie and Englande,” a copy of which is in the library of the
Society of Antiquaries. England commences the dialogue, inviting
Queen Elizabeth in the following words:--

\settowidth{\versewidth}{Come over the born, Bessy, come over the born, Bessy,}
\begin{fnverse}
\vleftofline{“}Come over the born, Bessy, come over the born, Bessy, \\
Swete Bessy, come over to me.”
\end{fnverse}
Another, “Come o’er the burne, Bessie,” will be found in Addit.
MSS. Brit. Mus. No. 5665, with music. I may here remark, that the
tune to \textit{Take thy old cloth about thee} (one of the ballads quoted by
Shakespeare) is evidently formed out of \textit{Green Sleeves}. The earliest
known copy of the words is in English idiom, in Bishop Percy’s folio
manuscript, and I have little doubt that both words and music are of
English origin.
\end{dcfootnote}
\end{fixedpage}%505
\pagebreak

\begin{fixedpage}%506
\versoheader

“At the Duke’s theatre, Nokes appeared in a hat larger than
Pistol’s, which took the town wonderful, and supported a bad play by
its pure effect. Dryden, piqued at this, caused a hat to be made the
circumference of a hinder coach-wheel; and as Nelly (Nell Gwyn) was
low of stature, and what the French call \textit{mignonne} and \textit{piquante}, he
made her speak under the umbrella of that hat, the brims thereof
being spread out horizontally to their full extension. The whole
theatre was in a convulsion of applause; nay, the very actors
giggled, a circumstance none had observed before. Judge, therefore,
what a condition the \textit{merriest Prince alive} was in, at such a
conjuncture! 'Twas beyond \textit{odso} and \textit{odsfish}, for he wanted little of
being suffocated with laughter.”

\musictitle{THERE WERE THREE TRAVELLERS.}

In a Collection of Satirical Songs by the Earl of Rochester
(Harl. MSS., No. 6913), is “A new ditty to an old tune of \textit{Three
Travellers},” beginning—

\settowidth{\versewidth}{I’ll shew you the Captains of Aubrey Vere,}
\begin{scverse}
\begin{altverse}
\vleftofline{“}I’ll shew you the Captains of Aubrey Vere,\\
With a hey ho, langled down dilly;\\
Fit Captains to serve with so noble a peer,\\
Who has \textit{never a penny of money}.”
\end{altverse}
\end{scverse}

A copy of the ballad in the Bagford Collection (643, m. 9, p. 88)
is entitled “The Jovial Companions; or, The Merry Travellers, who
paid their shot where ever they came, without ever a stiver of money:
To \textit{an excellent North-country tune}.” Printed by C. Bates, at the Sun
and Bible, in Pye Corner.- It is also contained in \textit{Pills to purge
Melancholy}, vi. 177.

The story is, that the three travellers make themselves so
agreeable to the hostess, wherever they go, that they are suffered to
depart scot-free,—a very pleasant theory.
\end{fixedpage}%506
\pagebreak

\begin{fixedpage}%507
\rectoheader

\musictitle{WILLY WAS SO BLITHE A LAD.}

This “Northern Song” is contained in the first edition of
Playford’s \textit{Choice Ayres}, Book I. It bears a strong family likeness to
the “rare Northern tune,” \textit{Never love thee more} (ante Vol. I, p. 380).

\musictitle{TO ALL YOU LADIES NOW AT LAND.}

This ballad was written by Lord Buckhurst, afterwards Earl of
Dorset, when at sea during the first Dutch war, 1664-5. It has been
said to have been written “the night before the engagement;” but, in
all probability, was penned during the Duke of York’s first cruise,
in November, 1664, when an action was avoided by the Dutch retiring
to port.

The proof is, that it is mentioned by Pepys in his Diary, under
the date of Jan. 2, 1664-5. He says, “To my Lord Brouncker’s by
appointment, in the Piazza, Covent Garden; where I occasioned much
mirth with a ballet I brought with me, made from the seamen at sea to
the ladies in town.”
\end{fixedpage}%507
\pagebreak

\begin{fixedpage}%508
\versoheader

The statement that it was “made the night before the engagement,”
which action took place in June, 1665, is irreconcileable with Pepys’
possession of a copy in the preceding January, and has been carefully
analysed by Lord Braybrooke, in his notes upon Pepys’s Diary, v. 241,
edit. 1849. It rests upon the authority of Matthew Prior, who was
born in 1664, and who had probably heard the story with a little
embellishment.

In \textit{Merry Drollery Complete}, 1670, is the song “My mistress is a
shuttlecock,” to this tune. In \textit{A Pill to purge State Melancholy},
12mo., 1715, is “The Soldiers’ Lamentation for the loss of their
General,” \&c., to the tune of “\textit{To you fair ladies};” and the same was
printed in broadside with the date of 1712. Also, “News from Court, a
ballad to the tune of \textit{To all you ladies now at land}; by Mr. Pope,”
1719. In the \textit{Gentleman’s Magazine} for July, 1731, “To all you Ladies
now at Bath.”

The tune is in Watts’ \textit{Musical Miscellany}, vol. iii., 1730, and in
the \textit{Convivial Songster}, 1782; in the ballad-operas of \textit{The Jovial
Crew}, \textit{The Cobbler’s Opera}, \textit{The Lover’s Opera}, The Court Legacy,
\textit{Polly}, \textit{A Cure for a Scold}, \&c.; and (barbarously printed) in \textit{Pills
to purge Melancholy}, vi. 272.

\settowidth{\versewidth}{For though the Muses should prove kind,}
\indentpattern{0101009}
\begin{dcverse}
\begin{patverse}
For though the Muses should prove kind, \\
And fill our empty brain;\\
Yet if rough Neptune rouse the wind \\
To wave the azure main,\\
Our paper, pen, and ink, and we,\\
Roll up and down our ships at sea,\\
With a fa la, \&c.
\end{patverse}

\begin{patverse}
Then if we write not by each post,\\
Think not we are unkind;\\
Nor yet conclude our ships are lost \\
By Dutchmen or by wind:\\
Our tears we’ll send a speedier way,\\
The tide shall bring them twice a-day.\\
With a fa la, \&c.
\end{patverse}
\end{dcverse}

\end{fixedpage}%508
\pagebreak

\begin{fixedpage}%509
\rectoheader

\settowidth{\versewidth}{For though the Muses should prove kind,}
\indentpattern{0101009}
\begin{dcverse}
\begin{patverse}
The king, with wonder and surprise;\\
Will swear the seas grow bold;\\
Because the tides will higher rise,\\
Than e’er they did of old:\\
But let him know it is our tears \\
Brings floods of grief to Whitehall stairs.\\
With a fa la, \&c.
\end{patverse}

\begin{patverse}
Should foggy Opdam chance to know \\
Our sad and dismal story;\\
The Dutch would scorn so weak a foe,\\
And quit their fort at Goree:\\
For what resistance can they find \\
From men who’ve left their hearts behind!\\
With a fa la, \&c. 
\end{patverse}

\begin{patverse}
Let wind and weather do its worst,\\
Be you to us but kind;\\
Let Dutchmen vapour, Spaniards curse,\\
No sorrow shall we find:\\
'Tis then no matter how things go,\\
Or who’s our friend, or who’s our foe,\\
With a fa la, \&c. 
\end{patverse}

\begin{patverse}
To pass our tedious hours away,\\
We throw a merry main;\\
Or else at serious ombre play;\\
But why should we in vain \\
Each other’s ruin thus pursue?\\
We were undone when we left you.\\
With a fa la, \&c.
\end{patverse}

\begin{patverse}
But now our fears tempestuous grow,\\
And cast our hopes away:\\
Whilst you regardless of our woe,\\
Sit careless at a play:\\
Perhaps permit some happier man \\
To kiss your hand, or flirt your fan.\\
With a fa la, \&c.
\end{patverse}

\begin{patverse}
When any mournful tune you hear,\\
That dies in every note;\\
As if it sigh’d with each man’a care,\\
For being so remote:\\
Think then how often love we’ve made \\
To you, when all those tunes were play’d.\\
With a fa la, \&c. 
\end{patverse}

\begin{patverse}
In justice you cannot refuse,\\
To think of our distress;\\
When we for hopes of honour lose \\
Our certain happiness;\\
All those designs are but to prove \\
Ourselves more worthy of your love.\\
With a fa la, \&c. 
\end{patverse}

\begin{patverse}
And now we’ve told you all our loves,\\
And likewise all our fears;\\
In hopes this declaration moves \\
Some pity for our tears;\\
Let’s hear of no inconstancy,\\
We have too much of that at sea.\\
With a fa la, \&c.
\end{patverse}
\end{dcverse}

\musictitle{THE FAIR ONE LET ME IN.}

A black-letter copy of this ballad in the Roxburghe Collection,
ii. 240, is entitled, “Kind Lady; or, The Loves of Stella and Adonis:
A new court song, much in request. To \textit{a new tune}, or \textit{Hey, boys, up go
we}, \textit{The Charming Nymph}, or \textit{Jenny, gin}.” It commences—

\settowidth{\versewidth}{“The night her blackest sables wore,”}
\begin{scverse}
“The night her blackest sables wore,” \&c.
\end{scverse}

The “new tune” soon became popular, and many other ballads were
sung to it. In the same volume of the Roxburghe Collection are “The
Good Fellow’s Frolic; or, Kent Street Club: To the tune of \textit{The fair
one let me in}, p. 198;— “\textit{The love-sick Maid of Wapping},” p. 295;—and
a third ballad at p. 270.

In the Douce Collection, p. 55, is “The despairing Maiden reviv’d
by the return of her dearest love,” \&c. “To the tune of \textit{The fair
one let me in}, or
\textit{Busy Fame}, or \textit{Jenny, gin};” commencing—

\settowidth{\versewidth}{As I walkt forth to take the air,}
\begin{dcverse}
\begin{altverse}
\vleftofline{“}As I walkt forth to take the air, \\
One morning in the Spring, \\
And for to view the lilies fair,\\
To hear the small birds sing,”\&c.
\end{altverse}
\end{dcverse}

The words of the original song, “The night her blackest sables
wore,” or “The fair one let me in,” were written by D’Urfey, and the
tune composed by Thomas Farmer. They were published together in “A
new Collection of Songs and Poems, by Thomas D’Urfey, Gent. Printed
for Joseph Hindmarsh, at the Black
\end{fixedpage}%509
\pagebreak

\begin{fixedpage}%510
\versoheader

Bull, in Cornhill,” 1683 (8vo.); and there entitled, “The
generous Lover, a new song, set by Mr. Tho. Farmer.” In the same
year, they were included in the fourth book of “Choice Ayres and
Songs to sing to the Theorbo-lute or Bass-viol: being most of the
newest Ayres and Songs sung at Court, and at the Public Theatres;
Composed by several Gentlemen of His Majesty’s Musick, and others;”
and the tune alone, printed in “The Genteel Companion for the
Recorder, by Humphrey Salter, Gent.” It then passed into \textit{Pills to
purge Melancholy}, and was included in the first volume of every
edition; the tune was also introduced into many ballad-operas.

I may here remark that the \textit{Pills} of 1719, having been \textit{made up} by
D’Urfey, the two first volumes consist exclusively of his songs.
Older songs which were contained in the first and second volumes of
prior editions were then transferred from the first to the third,
from the second to the fourth, and some to the fifth. He removed only
two or three of his own songs.

Although there can be no doubt of the authorship of the words and
music of this song, it has been claimed as Scotch. About fifty years
after its first publication, the tune appears in a corrupt form, in
Thomson’s \textit{Orpheus Caledonius}, ii. 14 (1733). The alterations may have
arisen from having been traditionally sung, or may have been made by
Thomson. There are also a few changes in the words, such as the name
of “Stella” altered to “Nelly,” and “she rose and let me in” to “she
raise and loot me in.” These were copied from vol. ii. of Allan
Ramsay’s Tea-table Miscellany, in which the song is marked “z,” as
being old.

Allan Ramsay was not particular as to the nationality of his
songs,—it sufficed that they were popular in Scotland. His collection
includes many of English origin; and several of the tunes to which
the songs were to be sung are English and Anglo-Scottish. Ritson
claimed this, in his Essay on Scottish Song, as “an English song of
great merit, which has been scotified by the Scots themselves.” Upon
which, Mr. Stenhouse, in his notes to Johnson’s \textit{Scot’s Musical
Museum}, asks, “Could any person in his sound senses affirm that such
lines as the following, in Playford’s edition of the song, printed in
his fourth volume of \textit{Choice Ayres and Songs}, with the music, in 1683,
were \textit{not only English}, but English of great merit, too?” Mr.
Stenhouse’s opinion of the merits or demerits of the song are of
little importance: it suffices to say that Burns differed from
him;—but to assert that the copy in Playford’s \textit{Choice Ayres} is not
English, betrays an excess of nationality that made him utterly
regardless of his own future credit for veracity. In the forty lines,
of which the song consists, there is \textit{not a single Scotch word},—not
even one that could be mistaken for Scotch, unless it were “bern” for
“child!” If Mr. Stenhouse had only a pocket dictionary, which did not
contain old words, he certainly used a copy of Percy’s \textit{Reliques of
Ancient Poetry}, in the glossary to which he would have found “barne,
berne—man, person.” If “bairn” had been the word, the mistake would
have been more excusable, because it is the more common form in
Scotland; but whether written “barn,” “bern,” “bearn,” or “bairn,”
all are English, and words in use at that time. D’Urfey spells it
“bearn,” in his \textit{Songs and Poems}, as in \textit{Bailey’s}
\end{fixedpage}%510
\pagebreak

\begin{fixedpage}%511
\rectoheader

\textit{Dictionary}. “Awd men are twice \textit{bairnes}” is one of the Yorkshire
proverbs, at the end of \textit{The Praise of Yorkshire Ale}, by G. M., Gent.,
8vo., York, 1697. It would have been unnecessary to refer at such
length to Mr. Stenhouse’s “notes,” if they had not been transferred
to more recent works; but, in the first place, the editor of Messrs.
Blackie’s \textit{Book of Scottish Song} repeats his statement, that “the
original \textit{Scotch} words are to be found in Playford’s \textit{Choice Ayres}.” In
the second, Mr. Stenhouse telling us that this song was “originally
written by Francis Semple, Esq., of Beltrees, about the year 1650,”
it has been recently printed among poems by Francis Sempill. Even the
learned editor of Wood’s \textit{Songs of Scotland} does not question
statements so audaciously put forth, although he has frequently had
occasion to convict Mr. Stenhouse of misquoting the contents of
music-boooks that he pretended to have read, but was unable to
decipher.

In D’Urfey’s \textit{Songs and Poems}, the last line is “\textit{This angel} let
me in,” which in my copy is altered by a contemporary hand to “The
fair one,” as it stands in all other copies.
\end{fixedpage}%511
\pagebreak

\begin{fixedpage}%512
\versoheader

\musictitle{MAD ROBIN.}

This tune is in \textit{The Dancing Master} of 1686 (additional sheet),
and in all later editions. Also in \textit{Polly}, 1728; \textit{The Lovers’ Opera},
1729; \textit{The Stage Mutineers}, 1733; and many other ballad-operas.

“'Tis but a day or two ago since our mistress turn’d away her old
servant, because he would not play \textit{Mad Robin}, which the organist has
promised to do. I will say that for him, the old organist was an
excellent musician, but somewhat of a humourist; he would have his
own way, and play his own tunes.”—\textit{History of Robert Powel, the
Puppet-showman}, 8vo., 1715.

I have not succeeded in finding the song of \textit{Mad Robin}, and have
therefore taken the first and last stanzas of a ballad contained in a
manuscript of the time of James I., now in the possession of Mr.
Payne Collier. I have no authority for coupling them with the tune,
but prefer those old words to any written expressly to the air in the
ballad-operas.

\settowidth{\versewidth}{Winter’s cold, or summer’s heat,}
\indentpattern{0002}
\begin{dcverse}
\begin{patverse}
Winter’s cold, or summer’s heat,\\
Autumn’s tempests on it beat\\
It can never know defeat, \\
Never can rebel: 
\end{patverse}

\begin{patverse}
Such the love that I would gain,\\
Such love, I tell thee plain,\\
Thou must give, or woo in vain,\\
So, to thee, farewell.
\end{patverse}
\end{dcverse}

\end{fixedpage}%512
\pagebreak

\setlength{\fixedpagewidth}{400pt}
\begin{fixedpage}%513
\rectoheader

\musictitle{THE LEATHER BOTTÈL.}

Although I have not found any copy of this ballad printed before
the reign of Charles II., there appears reason for believing it to be
of much earlier date. The irregularity in the number of lines in each
stanza,—eight, ten, and sometimes twelve in the earlier copies,—gives
it the character of a minstrel production, such as Richard Sheale’s
\textit{Chevy Chace}, rather than of the Eldertons, Delonys, or Martin Parkers
of the reigns of Elizabeth and James, who all observed a just number
of lines in their ballads- The word “bottle” was not pronounced
“bottèl” in the reign of Charles II., or even in the time of
Shakespeare; such pronunciation belongs rather to the era of Chaucer
and Piers Ploughman, than to the later period. The Rev. Arthur
Bedford, in his \textit{Great Abuse of Music}, 8vo, 1711, speaks of the
commencement of the ballad,—

\settowidth{\versewidth}{“’Twas God above that made all things,”}
\begin{scverse}
“’Twas God above that made all things,” \&c., 
\end{scverse}
ending the stanza
with—

\settowidth{\versewidth}{So I wish his soul in heav’n may dwell}
\begin{scverse}
\vleftofline{“}So I wish his soul in heav’n may dwell\\
That first devised the leather bottèll,” 
\end{scverse}
as irreverent; but I
believe it by no means to have been \textit{intentionally so}, but rather that
the rambling beginning is another proof of its antiquity. A very
early ballad, written by a priest in the reign of Queen Mary (a copy
of which is in the library of the Society of Antiquaries), commences
in a very similar manner, and the metre is so like that it might be
sung to the same tune. It is entitled “A new Ballade of the
Marigolde,” and opens thus:—

\settowidth{\versewidth}{Sonne, Moone and Sterres shinying so bright,}
\begin{dcverse}
\begin{altverse}
\vleftofline{“}The God above, for man’s delight, \\
Hath heere ordaynde every thing, \\
Sonne, Moone and Sterres shinying so bright,\\
With all kind fruites, that here doth  spring, \\
And flowres that are so flourishyng:\\
Amonges all which that I beholde\\
(As to my minde best contentyng),\\
I doo commende the Marigolde.”
\end{altverse}
\end{dcverse}
In the seventh stanza—

\settowidth{\versewidth}{To Marie our Quecne, that flowre so sweete,}
\begin{dcverse}
\vleftofline{“}To Marie our Queene, that flowre so sweete,\\
This Marigolde I doo apply, \\
For that the name doth serve so meete \\
And properlee in each partie,\\
For her enduryng paciently \\
The stormes of such as list to scolde\\
At her dooynges, without cause why, \\
Loth to see spring this Marigolde.
\end{dcverse}
At the end, “God save the Queene.
Quod William Forrest, \textit{Preest}.”\textsuperscript{a} Printed by Richard Lant, in
Aldersgate Street.

But, to return to The Leather Bottèl. Copies are to be found in
the Bagford, Roxburghe, and other Collections; in the list of those
printed by Thackeray; in \textit{Wit and Drollery}, 1682; in \textit{The New Academy
of Compliments}, 1694 and 1713; in \textit{Pills to purge Melancholy} ; in
Dryden’s Miscellany Poems; and in a succession of others to the
present day. Mr. Sandys contributed a Somersetshire version to Mr.
Dixon’s \textit{Ballads and Songs of the Peasantry of England}.

We find it alluded to in “Hey for our Town, or a fig for
Zommersetshire”
(Douce Coll., p. 96):—

\settowidth{\versewidth}{In praise o’ th’ Leather Bottèl, quo’ Bob,}
\begin{scverse}
\begin{altverse}
\vleftofline{“}Come, sing us a merry catch, quo’ Bob,\\
Quo’ scraper, what’s the words?\\
In praise o’ th’ Leather Bottèl, quo’ Bob,\\
For we’ll he merry as lords.”
\end{altverse}
\end{scverse}

\begin{dcfootnote}
\textsuperscript{a} In the same volume in the library of the Society of
Antiquaries, is a ballad on the marriage of Queen Mary and
Philip, by John Heywood.
\end{dcfootnote}
\end{fixedpage}%513
\pagebreak
\setlength{\fixedpagewidth}{360pt}
\begin{fixedpage}%514
\versoheader

In \textit{Westminster Drollery}, Part II., 1672, and in \textit{Pills to purge
Melancholy},
i. 267 (1707), are two versions of a similar ballad in praise of
the Black Jack. The first has the burden—

\settowidth{\versewidth}{And I wish his heirs may never want sack,}
\begin{scverse}
\vleftofline{“}And I wish his heirs may never want sack,\\
That first devis’d the bonny black jack.”
\end{scverse}

There is a version of the tune in \textit{Pills to purge Melancholy} , but
the traditional copy is so well known, that I give it in preference.

\settowidth{\versewidth}{Now, what do you say to these cans of wood?}
\indentpattern{0000003}
\begin{dcverse}
\begin{patverse}
Now, what do you say to these cans of wood? \\
Oh no, in faith they cannot be good;\\
For if the bearer fall by the way,\\
Why, on the ground your liquor doth lay: \\
But had it been in a leather bottèl,\\
Although he had fallen, all had been well.\\
So I wish in heav’n, \&c.
\end{patverse}

\begin{patverse}
Then what do you say to these glasses fine? \\
Oh, they shall have no praise of mine,\\
For if you chance to touch the brim,\\
Down falls the liquor and all therein;\\
But had it been in a leather bottèl,\\
And the stopple in, all had been well.\\
So I wish, \&c.
\end{patverse}
\end{dcverse}


\end{fixedpage}%514
\pagebreak

\begin{fixedpage}%515
\rectoheader

\settowidth{\versewidth}{Why they’ll tug and pull till their liquor doth spill:}
\indentpattern{0000003}
\begin{dcverse}
\begin{patverse}
Then what do you say to these black pots three? \\
If a man and his wife should not agree, \\
Why they’ll tug and pull till their liquor doth spill:\\
In a leather bottèl they may tug their fill, \\
And pull away till their hearts do ake,\\
And yet their liquor no harm can take.\\
So I wish, \&c.
\end{patverse}

\begin{patverse}
Then what do you say to these flagons fine? \\
Oh, they shall have no praise of mine.\\
For when a Lord is about to dine,\\
And sends them to he filled with wine,\\
The man with the flagon doth run away, \\
Because it is silver most gallant and gay.\\
So I wish, \&c.
\end{patverse}

\begin{patverse}
A leather bottèl we know is good,\\
Far better than glasses or cans of wood,\\
For when a man’s at work in the field,\\
Your glasses and pots no comfort will yield; \\
But a good leather bottle standing by,\\
Will raise his spirits, whenever he’s dry.\\
So I wish, \&c.
\end{patverse}

\begin{patverse}
At noon, the haymakers sit them down,\\
To drink from their bottles of ale nut-brown;\\
In summer too, when the weather is warm,\\
A good bottle full will do them no harm. \\
Then the lads and the lasses begin to tattle, \\
But what would they do without this bottle?\\
So I wish, \&c.
\end{patverse}

\begin{patverse}
There’s never a Lord, an Earl, or Knight,\\
But in this bottle doth take delight;\\
For when he’s hunting of the deer,\\
He oft doth wish for a bottle of beer.\\
Likewise the man that works in the wood,\\
A bottle of beer will oft do him good.\\
So I wish, \&c.
\end{patverse}

\begin{patverse}
And when the bottle at last grows old,\\
And will good liquor no longer hold,\\
Out of the side you may make a clout,\\
To mend your shoes when they’re worn out; \\
Or take and hang it up on a pin,\\
’Twill serve to put hinges and odd things in. \\
So I wish, \&c.
\end{patverse}
\end{dcverse}

As to leather bottles, Heywood thus enumerates the various
descriptions, in his \textit{Philocothonista}, 4to., 1635, p. 45:—“Other
bottles we have of leather, but they most used amongst the shepheards
and harvest people of the countrey; small jacks we have in many
ale-houses of the citie and suburbs, tipt with silver; besides the
great black-jack and bombards at the court, which, when the Frenchmen
first saw, they reported at their returne into their countrey, that
the Englishmen used to drink out of their boots.” These bombards,
according to Taylor, the water-poet, each held a gallon and a half,
in the reign of James I.; and the merchants of London, who had to pay
a tax of two bombards of wine to the Lieutenant of the Tower, out of
every ship that brought wine into the river Thames, contended, but
unsuccessfully, that they had been unduly increased in size. 

\begin{scverse}
\vleftofline{“}When the bottle and jack stand together, O fie on’t,\\
The bottle looks just like a dwarf to a giant;\\
Then have we not reason the jacks to choose,\\
For they will make boots, when the bottle mends shoes.”
\end{scverse}

\musictitle{TURN AGAIN, WHITTINGTON.}

“The tradition of Whittington’s cat,” says Mr. J. H. Burn, “has
served to amuse and delight the childhood of many, many thousands;
nor is it possible in more adult years to shake off the delusion
cherished and imbibed in our youthful dreams. Still it has no
reality; it is a pleasing fiction, so agreeable to our better
feelings, so happy in its believed results, that regret is excited
when it happens not to be true.”

“Sir Richard Whittington, thrice Lord Mayor of London, in the
years 1397, 1406, and 1419, was born in 1360, the son of Sir William
Whittington, Knight, and dame Joan his wife. He was therefore not a
poor boy; and the story of his
\end{fixedpage}%515
\pagebreak

\begin{fixedpage}%516
\versoheader

halting, a tired, justifiable runaway, and resting on a stone at
Holloway, while Bow-bells merrily sounded to his hearing—

\settowidth{\versewidth}{“Turn again Whittington, thrice Lord Mayor of London,” }
\begin{scverse}
“Turn again Whittington, thrice Lord Mayor of London,” 
\end{scverse}
has no
other origin than a flourish of fancy created by some poetical
brain.” (\textit{Catalogue of the Beaufoy Tokens}, p. 161.)

The earliest notice I have observed of \textit{Turn again, Whittington},
as a tune (if a mere change upon hells may come under that
denomination), is in Shirley’s \textit{Constant Maid}, act ii., sc. 2, 4to.,
1640, where the niece says—

\settowidth{\versewidth}{Faith, how many churches do you mean to build Before you die?}
\begin{scverse}
\vleftofline{“}Faith, how many churches do you mean to build\\
Before you die? six bells in every steeple,\\
And let them all go to the \textit{city tune},\\
\textit{Turn again, Whittington},—who, they say,\\
Grew rich, and let his land out for nine lives,\\
’Cause all came in by a cat.”
\end{scverse}

Mr. Burn points out various earlier notices of Whittington and his
cat, as in Eastward Hoe (printed in 1605), where Touchstone assures
Golding he hopes to see him reckoned one of the worthies of the city
of London, “when the famous fable of Whittington and his puss shall
be forgotten.”

The story of the cat is, perhaps, immediately derived from
Arlotto’s “Novella delle Gatte,” contained in his \textit{Facetiæ}, which
were printed soon after his death in 1483. The story is there told of
a merchant of Genoa, but it is probably of Eastern origin. The late
Sir William Gore Ouseley, in his travels, speaking of an island in
the Persian Gulf, relates, on the authority of a Persian MS., that,
“in the tenth century, one Keis, the son of a poor widow in Siráf,
embarked for India with a cat, his only property. There he
fortunately arrived at a time when the palace was so infested by mice
or rats, that they invaded the king’s food, and persons were employed
to drive them from the royal banquet. Keis produced his cat; the
noxious animals soon disappeared, and magnificent rewards were
bestowed on the adventurer of Siráf, who returned to that city, and
afterwards, with his mother and brothers, settled on the island,
which from him has been denominated Keis, or according to the
Persians, Keish.”

The numerous charities, and the public works, with which his name
was associated, would justly transmit the name of Sir Richard
Whittington to posterity. “Amongst others, he founded a house of
prayer, with an allowance for a master, fellows, choristers, clerks,
\&c., and an alms-house for thirteen poor men, called Whittington
College. He entirely rebuilt the loathsome prison, which was then
standing at the west gate of the city, and called it Newgate. He
built the better half of St. Bartholomew’s Hospital, in West
Smithfield; and the fine library in Grey Friars, now called Christ’s
Hospital; as also a great part of the east end of Guildhall, with a
chapel and a library, in which the records of the city might be
kept.” Grafton, in his Chronicle, relates an anecdote of him, which
is not elsewhere recorded. In a codicil to his will, he commanded his
executors, as they should one day answer before God, to look
diligently over the list of the persons indebted to him, and if they
found any who was not clearly possessed of three times as much as
would fully satisfy all the claim, they were freely to forgive
\end{fixedpage}%516
\pagebreak

\begin{fixedpage}%517
\rectoheader

it. He also added, that no man whatever should be imprisoned for
any debt due to his estate. “Look upon this, ye aldermen,” says the
historian emphatically, “for it is a glorious glass!”\textsuperscript{a}

The ballad was entered at Stationers’ Hall a few months later
than a drama on the same subject. The following extracts are from the
registers of the Company. On Feb. 8, 1604-5, entered to Tho. Pavier,
“The History of Richard Whittington, of his lowe birthe, his great
fortune, as yt was plaied by the Prynce’s Servants;” and on July 6
(1605), to Jo. Wright, “a ballad called The wondrous Lyfe and
memorable Death of Sir Ri: Whittington now sometyme Lo: Maior of the
honorable Citie of London.”

Wright was the printer. The ballad (or another on the same
subject) was written by Richard Johnson, author of \textit{The Seven
Champions of Christendom}, \&c., and is contained in his \textit{Crowne
Garland of Goulden Roses}, 1612. Copies are also in the Douce
Collection, fol. 103; in \textit{Old Ballads}, i. 132, 1723; in Evans’s
Collection, ii. 325, 1810; and in Mackay’s \textit{Songs of the London
Prentices and Trades}; \&c.

In \textit{Pills to purge Melancholy} , iii. 40, 1707, the tune is called
\textit{Turn again, Whittington}; in Hawkins’s transcripts of virginal music,
\textit{The Bells of Osney};\textsuperscript{b} and as the ballad of “Sir Richard Whittington”
was to be sung to the tune of \textit{Dainty, come thou to me}, this \textit{may} be
another name for the same. A fourth seems to be \textit{Whittington's Bells};
for Ward, in \textit{The London Spy}, says “he’d rather hear an old barber
ring \textit{Whittington's Bells} upon the cittern,” than all the music-houses
then afforded.

\begin{dcfootnote}
\textsuperscript{a} For more about Sir Richard Whittington, see \textit{Antiquarian
Repertory}, ii. 343; Rimbault’s \textit{Fly Leaves}, ii. 75; Burn’s \textit{Descriptive
Catalogue of London Traders’ Tavern and Coffee-House Tokens}; \&c.

\textsuperscript{b} The bells of Osney Abbey,” says Haw\-kins,“were very famous:
their several names were Douce, Clement, Austin, Hautecter, Gabriel,
and John. Near old Windsor,” be adds,“is a public-house, vulgarly
called The Bells of Bosely; this house was originally built for the
accommodation of bargemen, and others, navigating the river Thames
between London and Oxford. It has a sign of six bells, \ie , the
bells of Osney.” (\textit{History}, 8vo., 615.) I am told that the sign is now
altered to The \textit{Five Bells of Ouseley}, and that the house is famous
for its excellent ale. “The great Bells of Oesney,” is one of the
rounds for three voices in \textit{Deuteromelia}, 1609.
\end{dcfootnote}
\end{fixedpage}%517
\pagebreak

\begin{fixedpage}%518
\versoheader

\musictitle{JOAN’S PLACKET IS TORN.}

The earliest notice I have found of this air, is in Pepys’s
Diary; where, under date of 22nd June, 1667, he speaks of a
trumpeter, on board the Royal Charles, sounding the tune of \textit{Joan’s
Placket is torn}.

It is contained in \textit{The Dancing Master} of 1686 (additional sheet),
and in all subsequent editions; also in the ballad-operas of
\textit{Achilles}, \textit{The Bays’ Opera}, and \textit{Love in a Riddle}.
Colley Cibber’s song, “When I followed a lass that was froward
and shy,” which was written to the tune, for \textit{Love in a Riddle}, in
1729, was transferred by Bickerstaff to \textit{Love in a Village}, about
thirty years later, without acknowledgment of the source from which
he derived it.

In the \textit{Collection of Loyal Songs}, 1685 and 1694, is one entitled,
“The plot cram’d into Jone’s Placket: To the tune of \textit{Jone’s Placket
is torn}.” It is also one of the tunes called for by “the hobnailed
fellows” in \textit{The History of Robert Powel, the Puppet-shewman}, 8vo.,
1715.

As to the word “placket,” in “An exact Chronologie of memorable
things” in \textit{Wit’s Interpreter}, 3rd edit., 1671, it is said to be
“sixty-six years since maids began to wear plackets.” According to
Middleton, the placket is “the open part” of a petticoat; and the
word is not altogether obsolete, since the opening in the petticoats
of the present day is still called “the placket hole,” in
contradistinction to the pocket hole.

A very good song has been written to this tune by Charles Mackay
(entited “The Return Home”); but I have not discovered the original
words.
\end{fixedpage}%518
\pagebreak

\begin{fixedpage}%519
\rectoheader

The Rev. G. R. Gleig, in his \textit{Family History of England}, ii. 111,
prints a piece of music, which, according to tradition, was “the air
played by the band at Fotheringay Castle, while Mary, Queen of Scots,
was proceeding to execution.” It is the tune of \textit{Joan’s Placket},
turned into a slow march; but as Queen Mary was executed within the
castle, and there was no procession with drums and trumpets, or music
of any kind (according to all accounts), the story is not very
probable. Some of my readers may nevertheless desire to see it in
that form; and as \textit{Joan’s Placket} is certainly a trumpet tune, it is
\textit{possible} that it may have been played outside the castle on that day.


\musictitle{THE BAFFLED KNIGHT.}

This tune is contained in \textit{Youth’s Delight on the Flagelet}, ninth
and eleventh editions. It may be in earlier editions, but I have
never seen any other than the two in my possession. The date of the
ninth has been cut off in binding; the eleventh is of 1697. It is
also in the ballad-opera of \textit{Silvia, or The Country Burial}, 8vo, 1731,
but is an indifferent version.

The story of the rakish young knight outwitted by the maiden, has
been repeatedly versified. The earliest I have seen, is, “Yonder
comes a courteous
\end{fixedpage}%519
\pagebreak


\begin{fixedpage}%520
\versoheader

Knight,” already printed (i. 62). The second, entitled “The
baffled Knight, or Lady’s Policy,” is reprinted by Percy, and
commences—

\settowidth{\versewidth}{“There was Knight was drunk with wine.”}
\begin{scverse}
“There was Knight was drunk with wine.”
\end{scverse}

The third is contained in \textit{Pills to purge Melancholy} , iii. 1707,
or v. 1719; and in \textit{A Complete Collection of old and new English and
Scotch Songs}, 8vo, 1735. It has a separate tune (see \textit{Pills}), and is
in stanzas of eight lines, commencing—

\settowidth{\versewidth}{“There was a Knight and he was young.”}
\begin{scverse}
“There was a Knight and he was young.”
\end{scverse}

A copy of the fourth is in the Roxhurghe Collection (i. 306),
entitled “The politick Maid: Or—
\settowidth{\versewidth}{A dainty new ditty, both pleasant and witty,}
\begin{scverse}
\vleftofline{“}A dainty new ditty, both pleasant and witty,\\
Wherein you may see the Maid’s policie.”
\end{scverse}

“To a pleasant new tune.” Subscribed R[ichard] C[limsall], and
“printed for Thomas Lambert, at the signe of the Horse-shoe, in
Smithfield.”

\settowidth{\versewidth}{There was a Knight was wine-drunk,}
\begin{dcverse}
\begin{altverse}
\vleftofline{“}There was a Knight was wine-drunk, \\
As he rode on the way, \\
And there he spied a bonny lasse \\
Among the cocks of hay. \\
\textit{Sing, loud whistle in the wind,\\
Blow merry, merry;\\
Up and down in yonder dale,\\
With hey tro, nonney, nonney.}
\end{altverse}
\end{dcverse}

The tune here printed, belongs to the second of the above. In
\textit{Silvia}, the first line, “There was a Knight was drunk with wine,” is
given at full length. It is also referred to, under the title of \textit{The
baffled Knight}, in a black-letter ballad of “The West Country Lawyer:
Or The witty Maid’s good fortune,” \&c., “to the tune of \textit{The baffled
Knight}” (Rox. ii. 578); commencing—

\settowidth{\versewidth}{A youthful lawyer, fine and gay,}
\begin{dcverse}
\begin{altverse}
\vleftofline{“}A youthful lawyer, fine and gay, \\
Was riding unto the city, \\
Who met a damsel on the way, \\
Right beautiful, fair, and witty. \&c.\\
‘Good morrow, then, the lawyer cried,\\
‘I prithee, where art thou going?’\\
Quoth she, ‘To yonder meadow’s side,\\
My father is there a mowing,’”
\end{altverse}
\end{dcverse}

For continuation of the words, see Percy’s \textit{Reliques of Ancient
Poetry}.

\musictitle{THE WILLOW TREE.}

When I printed this tune among the \textit{National English Airs}, in
1839,1 was but imperfectly acquainted with its history. Mr. Macfarren
had noted down the air from hearing an old ballad-singer in
Lancashire, and could recollect but one stanza:— 

\settowidth{\versewidth}{O this willow tree will twist,}
\begin{scverse}
\begin{altverse}
\vleftofline{“}O this willow tree will twist,\\
And this willow tree will twine,” \&c.
\end{altverse}
\end{scverse}
\end{fixedpage}%520
\pagebreak

\begin{fixedpage}%521
\rectoheader

These lines I have since found to form part of a ballad
commencing, “I sowed the seeds of love,” which is still in print
among the ballad-venders in Seven Dials, and was published from one
of their copies in 1846, in \textit{Songs and Ballads of the Peasantry of
England}, by Mr. J. H. Dixon.

I spoke of the air as one of the common ballad-tunes sung about
the counties of Derbyshire, Warwickshire, and Lancashire; and that in
a burlesque at the Manchester Theatre, some years before, one of the
fraternity of blind ballad-singers had been imitated, chanting rhymes
to the tune, with pauses at the end of each phrase, as peculiarly
characteristic of their manner. I have since learned that the late
Mrs. Honey, having caught the air from another ballad-singer, had
introduced the ballad on the London stage, in \textit{The Loan of a Lover};
and that the history of the words is given in Whittaker’s \textit{History of
the Parish of Whalley} (p. 318, 4to., 1801.)

Dr. Whittaker tells us that Mrs. Fleetwood Habergham, of
Habergham Hall, Lancashire, “undone by the extravagance, and
disgraced by the vices of her husband” (who squandered his large
patrimony, till, in 1689, even the mansion-house and demesne were
swallowed up by the foreclosure of a mortgage), “soothed her sorrows
by some stanzas, yet remembered among the old people of the
neighbourhood, of which the following allusions to the triumphs of
her early days, and the successive offers she had rejected, under the
emblem of flowers, are simple and not inelegant:”—

\settowidth{\versewidth}{The pink, the primrose, and the rose,}
\begin{dcverse}
\begin{altverse}
\vleftofline{“}The gardener standing by,\\
Proffered to chuse for me \\
The pink, the primrose, and the rose,\\
But I refus’d the three.\\
The primrose I forsook \\
Because it came too soon,\\
The violet I overlookt\\
And vow’d to wait till June.\\
In June the red rose sprung, \\
But was no flower for me;\\
I pluck’d it up, lo! by the stalk, \\
And planted the willow tree.\\
The willow I now must wear,\\
With sorrows twin’d among, \\
That all the world may know \\
I falsehood lov’d too long.”
\end{altverse}
\end{dcverse}

Dr. Whittaker says, “A sentimental fine lady of the present day
would have thrown her story into the shape of a novel: the good old
gentlewoman’s ballad is at least the more tolerable of the two.”

From the circumstances under which they were written, the words
may be dated as not long after 1689, and in all probability were
written to the tune of \textit{Come, open the door, sweet Betty} (ante p. 505),
which was then in the height of its popularity. Although the
traditional version consists of but, one strain, and is in common
time, such metamorphose is by no means unusual in airs preserved
solely by tradition. The resemblance is still clearly traceable.
Another traditional version will be found in \textit{Albyn’s Anthology}, i.
40, fol., 1816, or Wood’s \textit{Songs of Scotland}, iii. 85, 8vo., 1850.

Mr. Alexander Campbell, the editor of \textit{Albyn's Anthology}, gives
the following account:—“This sweetly rural and plaintive air, like
many of the ancient Border Melodies” (he did not know how far south
of the border it might be traced) “has but one part, or rather one
measure. It was taken down by the editor from the singing of Mr.
Hogg” (the Ettrick Shepherd) “and his friend, Mr. Pringle, author of
the pathetic verses to which it is united;” commencing, “I’ll bid my
heart be still.”
\end{fixedpage}%521
\pagebreak
\begin{fixedpage}%522
\versoheader

Mr. Campbell also gives three stanzas “of the original Border
ditty, which was chanted to the melody.” These were supplied by Miss
M. Pringle, of Jedburgh. They are evidently a paraphrase of Mrs.
Habergham’s ballad, as the two following will shew:—

\settowidth{\versewidth}{But the pride o’ my garden is wither’d away,}
\begin{dcverse}
\begin{altverse}
\vleftofline{“}O once my thyme was young, \\
It flourish’d night and day; \\
But by there came a false young man, \\
And he stole my thyme away. \\
Within my garden gay.\\
The rose and lily grew; \\
But the pride o’ my garden is wither’d away,\\
And it’s a’ grown o’er wi’ rue.”
\end{altverse}
\end{dcverse}

The tune was not improved in transmission to the Border, as may
be seen by comparing the copy in \textit{Albyn’s Anthology}, or Wood’s \textit{Songs
of Scotland} (in both of which Mr. Thomas Pringle’s song is united to
it), with the Lancashire version here printed.

The following lines were written to the air by Mr. H. F. Chorley,
for the \textit{National English Airs}. They are entitled “The Widow’s Song:”--

\settowidth{\versewidth}{Ere I know there are flowers, or a bright blue sky}
\begin{dcverse}
\begin{altverse}
\textsc{Oh}! leave me to dream and weep,\\
Or lift ye the churchyard stone,\\
And send me my dead, through the twilight deep,\\
For I sit by my hearth alone!
\end{altverse}

\begin{altverse}
They were three of the blythest fays!\\
But their mirth—it all is done!\\
Oh I never could think in those glad, glad days!\\
I must sit by my hearth alone! 
\end{altverse}

\begin{altverse}
The spring ’mid her bloom goes by. \\
And the summer’s glorious sun,\\
Ere I know there are flowers, or a bright blue sky,\\
While I sit by iny hearth alone!
\end{altverse}

\begin{altverse}
Then leave me to dream and weep!\\
Or lift ye the church-yard stone:\\
I am weary, weary; better sleep,\\
Than sit by my hearth alone!
\end{altverse}
\end{dcverse}

From a variety of traditional versions, I have selected the
following. The Seven Dials copies are very corrupt, and I am informed
that they are frequently reprinted from the. dictation of
ballad-singers, who require a supply for sale, instead of from
earlier copies.

\end{fixedpage}%522
\pagebreak

\begin{fixedpage}%523
\rectoheader

\settowidth{\versewidth}{He chose the primrose, the lily, and pink,}
\begin{dcverse}\footnotesize
\begin{altverse}
My garden was planted full \\
Of flowers every where,\\
But for myself I could not choose,\\
The flower I held so dear.
\end{altverse}

\begin{altverse}
My gardener was standing by,\\
And he would choose for me;\\
He chose the primrose, the lily, and pink, \\
But those I refus’d all three.
\end{altverse}

\begin{altverse}
The primrose I did reject,\\
Because it came too soon;\\
The lily and pink I overlook’d,\\
And vow’d I would wait till June.
\end{altverse}

\begin{altverse}
In June came the rose so red.\\
And that’s the flower for me;\\
But when I gather’d the rose so dear,\\
I gain’d but the willow tree.
\end{altverse}

\begin{altverse}
Oh! the willow tree will twist,\\
And the willow tree will twine;\\
And would I were in the young man’s arms, \\
That ever has this heart of mine.
\end{altverse}

\begin{altverse}
My gardener, as he stood by,\\
He bade me take great care,\\
For if I gather’d the rose so red,\\
There groweth up a sharp thorn there.
\end{altverse}

\begin{altverse}
I told him I’d take no care,\\
Till I did feel the smart,\\
And still did press the rose so dear \\
Till the thorn did pierce my heart.
\end{altverse}

\begin{altverse}
A posy of hyssop I’ll make,\\
No other flow’r I’ll touch,\\
That all the world may plainly see \\
I love one flow’r too much.
\end{altverse}

\begin{altverse}
My garden is now run wild,\\
When shall I plant anew?\\
My bed that once was fill’d with thyme\\
Is all o’errun with rue.
\end{altverse}
\end{dcverse}

\backskip{1}
\musictitle{YOUNG JEMMY.}

There are two ballads on Charles the Second’s natural son, the
Duke of Monmouth, that were sung to this tune, and both printed
during his father’s reign, when the Duke was out of favour at court.

Of the first ballad there are two copies; one in the King’s
Library, Brit. Mus., entitled “Young Jemmy: An excellent new Ballad:
\textit{To an excellent new tune}” dated 1681; and the second in the Roxburghe
Collection, ii. 140, called “England’s Darling; or, Great Britain’s
Joy and Hope in that noble Prince, James, Duke of Monmouth:

\settowidth{\versewidth}{May’st thou in thy noble father’s love remain,}
\begin{dcverse}
\begin{altverse}
Brave Monmouth, England’s glory, \\
Hated of none but Papist and Tory, \\
May’st thou in thy noble father’s love remain, \\
Who happily over this land doth reign.
\end{altverse}
\end{dcverse}

Tune of \textit{Young Jemmy}, or \textit{Philander}.”\textsuperscript{a}
Printed by J. Wright, J. Clark, W. Thackeray, and T. Passinger. It
commences—

\settowidth{\versewidth}{And bravely will maintain their part,}
\begin{dcverse}
\begin{altverse}
\vleftofline{“}Young Jemmy is a lad \\
That’s royally descended, \\
With every virtue clad, \\
By every tongue commended; \\
A true and faithful English heart,\\
Great Britain’s joy and hope,\\
And bravely will maintain their part,\\
In spite of Turk and Pope,” \&c.
\end{altverse}
\end{dcverse}

The second ballad is entitled “Young Jemmy; or, The Princely
Shepherd: Being a most pleasant and delightful new song:

\settowidth{\versewidth}{His song of love, while the fair nymphs trip round,}
\begin{scverse}
In blest Arcadia, where each shepherd feeds \\
His numerous flocks, and tunes, on slender reeds,\\
His song of love, while the fair nymphs trip round,\\
The chief amongst 'em was Young Jemmy found:\\
For he with glances could enslave each heart,\\
But fond ambition made him to depart\\
The fields, to Court; led on by Buch as sought\\
To blast his virtues,—which much sorrow brought.
\end{scverse}

\begin{scfootnote}
\textsuperscript{a} For Philander, see i. 280.
\end{scfootnote}
\end{fixedpage}%523
\pagebreak

\begin{fixedpage}%524
\versoheader

To a \textit{new Play-house tune}, or \textit{In January last},\textsuperscript{a} or 
\textit{The Gowlin}.”\textsuperscript{b} Printed by P. Brooksby, at the Golden Ball in West Smithfield
(Rox. ii. 556). Commencing— 

\settowidth{\versewidth}{A face and shape so wondrous fine,}
\begin{dcverse}
\begin{altverse}
\vleftofline{“}Young Jemmy was a lad \\
Of royal birth and breeding, \\
With every beauty clad, \\
And every swain exceeding: \\
A face and shape so wondrous fine,\\
So charming every part,\\
That every lass upon the green\\
For Jemmy had a heart,” \&c.
\end{altverse}
\end{dcverse}

Both these ballads have been reprinted in Evans’s Collection,
iii. 206 and 211 (1810). The tune is in \textit{The Genteel Companion for the
Recorder}, 1683; in 180 \textit{Loyal Songs}, 1685 and 1694; in \textit{The Village
Opera}, 1729; in \textit{Love and Revenge, or The Vintner Outwitted}, \textsc{n.d.}; in
\textit{The Bay’s Opera}, 1730; \&c.

There are two others, to the tune, in 180 \textit{Loyal Songs}; the first,
“Old Jemmy, tune of \textit{Young Jemmy}.” It is a counter-panegyric upon
James II., when Duke of York, by Mat. Taubman; commencing—

\settowidth{\versewidth}{Right lawfully descended.”}
\begin{scverse}
\begin{altverse}
\vleftofline{“}Old Jemmy is a lad\\
Right lawfully descended.”
\end{altverse}
\end{scverse}

The second, “A new song on the arrival of Prince George [of
Denmark], and his intermarriage with the Lady Anne,” afterwards Queen
Anne; commencing—

\settowidth{\versewidth}{Prince George at last is come;}
\begin{scverse}
\begin{altverse}
\vleftofline{“}Prince George at last is come;\\
Fill every man his bumper,” \&c.
\end{altverse}
\end{scverse}

In the Roxburghe Collection, ii. 504, is “The West-country Nymph;
or, The loyal Maid of Bristol,” \&c., to the tune of \textit{Young Jemmy};
beginning—

\settowidth{\versewidth}{Come, all you maidens fair,}
\begin{dcverse}
\begin{altverse}
\vleftofline{“}Come, all you maidens fair, \\
And listen to my ditty; \\
In Bristol city fair\\
There liv’d a damsel pretty.”
\end{altverse}
\end{dcverse}
In the early part of the last century, the Pretender was called
“Young Jemmy,” and the tune became a favorite with the Jacobites. “I
never can pass through Cranbourn Alley, but I am astonished at the
remissness or lenity of the magistrates in suffering the Pretender’s
interest to be carried on and promoted in so public and shameful a
manner as it there is. Here a fellow stands eternally bawling out his
Pye-Corner pastorals in behalf of \textit{Dear Jemmy, Lovely Jemmy}, \&c. I
have been credibly informed, this man has actually in his pocket a
commission, under the Pretender’s great seal, constituting him his
Ballad-singer in Ordinary in Great Britain; and that his ditties are
so well worded, that they often poison the minds of many well-meaning
people: that this person is not more industrious with his tongue in
behalf of his master, than others are, at the same time, busy with
their fingers among the audience; and the monies collected in this
manner are most of those mighty remittances the \textit{Post-boy} so
frequently boasts of being made to the Chevalier.”—From “A View of
London and Westminster: or, The Town Spy. Containing an account of
the different customs, tempers, manners, policies, \&c., of the
\textsc{People} in the several most noted Parishes within the Bills of
Mortality, respectively,” \&c. By a German Gentleman. 2nd. edit.,
8vo., 1725.

\begin{dcfootnote}
\textsuperscript{a} For \textit{In January last}, see Index.

\textsuperscript{b} \textit{The Gowlin} is called “a new Playhouse tune” in the ballad, the
last stanza of which explains that—

\settowidth{\versewidth}{By nymphs and shepherd swains,” \&c.}
\begin{fnverse}
\begin{altverse}
\vleftofline{“}The Gowlin is a yellow flower \\
That grows upon the plains,\\
Which oftentimes is gathered \\
By nymphs and shepherd swains,” \&c.
\end{altverse}
\end{fnverse}
\end{dcfootnote}
\end{fixedpage}%524
\pagebreak


\begin{fixedpage}%525
\rectoheader

\musictitle{MY LODGING IT IS ON THE COLD GROUND.}

This is a song in the play of \textit{The Rivals} (an alteration of
Fletcher’s \textit{Two Noble Kinsmen}), the performance of which Pepys
witnessed twice, “at the Duke’s house,” in the year 1664; but which
acquired its principal celebrity in or about 1667, when Moll Davis
and Betterton performed the principal characters. Downes, who was
prompter at the theatre, from 1662 to 1706, thus speaks of it: “\textit{The
Rivals}, a play, wrote by Sir William Davenant: having a very fine
interlude in it, of vocal and instrumental music, mixt with very
diverting dances\dots All the women’s parts admirably acted;
chiefly Cel[an]ia, a shepherdess, being mad for love; especially in
singing several wild and mad songs, \textit{My lodging it is on the cold
ground}, \&c. She performed that so charmingly, that, not long after,
it raised her from her bed on the cold ground to a Bed Royal.”
\textit{Roscius Anglicanus}, edit. 1781, p. 32. Downes does not here mention
the representative of Celania, but the name of Mrs. Davis is found
in the printed
\end{fixedpage}%525
\pagebreak

\begin{fixedpage}%526
\versoheader

list of characters in the play, 4to., 1668. Charles II. took her
off the stage, and had a daughter by her, named Mary Tudor, who was
married to Francis, second Earl of Derwentwater.

The original air of \textit{My lodging is on cold ground} was composed by
Matthew Lock, and is included among the violin tunes at the end of
\textit{The Dancing Master} of 1665; also in \textit{Musick’s Delight on the Cithren},
1666, and in \textit{Apollo’s Banquet}, 1669. In the two former it is entitled
“On the cold ground;” in the' latter, “I prithee, love, turn to me.”

The following is Matthew Lock’s air:—

\settowidth{\versewidth}{I’ll crown thee with a garland of straw, then,}
\begin{dcverse}
\begin{altverse}
I’ll crown thee with a garland of straw, then,\\
And I’ll marry thee with a rush ring;\\
My frozen hopes shall thaw, then,\\
And merrily we will sing:\\
O turn to me, my dear love,\\
And prithee, love, turn to me,\\
For thou art the man that alone canst \\
Procure my liberty.\\
But if thou wilt harden thy heart still\\
And be deaf to my pitiful moan, \\
Then I must endure the smart still,\\
And tumble in straw alone;\\
Yet still I cry, O turn love,\\
And prithee, love, turn to me,\\
For thou art the man that alone art \\
The cause of my misery.
\end{altverse}
\end{dcverse}
\end{fixedpage}%526
\pagebreak

\begin{fixedpage}%527
\rectoheader

The popularity of the song was very great, and may be traced in
an uninterrupted stream from that time to the present. The words were
reprinted in \textit{Merry Drollery Complete}, Part II., 1670, under the title
of “Phillis, her Lamentation;” and in the same, a parody on it,
called “Women’s Delight.” Another parody, “My lodging is on the cold
boards,” is in Howard’s play, \textit{All Mistaken}, 1672. Then the original
in \textit{The New Academy of Compliments}, 1694, 1713, \&c.; in \textit{Vocal Music,
or the Songster’s Companion}, 8vo., 1775; in Johnson’s \textit{Lottery Song
Book}, \textsc{n.d}.; and fifty others. It was lengthened into a ballad, and
became equally popular in that form. A copy is in the Roxburghe
Collection,
ii. 423, “printed by and for W. O[nley] for A. M[ilbourne], and
sold by C. Bates, at the Sun and Bible in Pye Corner.” Onley and
Milbourne were ballad-printers in the reign of Charles II. Bates I
believe to be somewhat later. It is as follows:—

“The slighted Maid; or The pining Lover.

\settowidth{\versewidth}{With sighs and moans she doth intreat her dear,}
\begin{scverse}
With sighs and moans she doth intreat her dear,\\
Whilst he seems to be deaf and will not hear;\\
At length his frozen heart begins to melt.\\
Being moved with the passion she had felt.
\end{scverse}

To the tune of \textit{I prithee, Love, turn [to] me},” \&c. “Licens’d and
enter’d
according to order.”

\settowidth{\versewidth}{Long time have I sighed and mourned,}
\begin{dcverse}
\begin{altverse}
Was ever maiden so scorned \\
By one that she loved so dear?\\
Long time have I sighed and mourned, \\
And still my love will not hear:\\
O turn to me; my own dear heart.\\
And I prithee, love, turn to me,\\
For thou art the lad I long for,\\
And, alas! what remedy?
\end{altverse}

\begin{altverse}
My lodging is on the cold ground,\\
And very hard is my fare;\\
But that which troubles me most, is \\
The unkindness of my dear:\\
O turn to me, my own dear heart,\\
And I prithee, love, turn to me,\\
For thou art the lad I long for,\\
And, alas! what remedy?
\end{altverse}

\begin{altverse}
O stop not thine ear to the wailings \\
Of me, a poor harmless maid;\\
You know we are subject to failings,— \\
Blind Cupid hath me betrayed:\\
And now I must cry, O turn, love,\\
And I prithee, love, turn to me,\\
For thou art the man that alone art \\
The cause of my misery.
\end{altverse}

\begin{altverse}
How canst thou be so hard-hearted, \\
And cruel to me alone;\\
If ever we should be parted,\\
Then all my delight is gone:\\
But ever I cry, O turn, love.\\
And I prithee, love, turn to me,\\
For thou art the man that alone art \\
The cause of my misery.
\end{altverse}

\begin{altverse}
I’ll make thee pretty sweet posies, \\
And constant I ever will prove;\\
I’ll strew thy chamber with roses,\\
And all to delight my love:\\
Then turn to me, my own dear heart, \\
And I prithee, love, turn to me, \\
For thou art the man that alone canst \\
Procure my liberty.
\end{altverse}

\begin{altverse}
I’ll do my endeavour to please thee \\
By making thy bed full soft;\\
Of all thy sorrows I’ll ease thee \\
By kissing thy lips full oft:\\
Then turn to me, my own dear heart, \\
And I prithee, love, turn to me, \\
For thou art the man that alone canst \\
Procure my liberty.
\end{altverse}
\end{dcverse}
\end{fixedpage}%527
\pagebreak
\begin{fixedpage}%528
\versoheader

\settowidth{\versewidth}{But thou wilt harden thy heart, still,}
\begin{dcverse}
\begin{altverse}
But thou wilt harden thy heart, still, \\
And be deaf to my pitiful moan, \\
So, I must endure the smart, still, \\
And tumble ip straw, all alone: \\
Whilst still I cry, O turn, love,\\
And I prithee, love, turn to me, \\
For thou art the man that alone art \\
The cause of my misery.
\end{altverse}

\begin{altverse}
If that thou still dost disdain me,\\
I never will love thee more;\\
Thy cruelty never shall pain me,\\
For I’ll have another in store:\\
But still I cry, O turn, love,\\
And I prithee, love, turn to me, \\
For thou art the man that alone art \\
The cause of my misery.
\end{altverse}

\begin{altverse}
By hearing her pitiful clamour,\\
The passion of love he felt;\\
He could no longer disdain her,\\
His frozen heart it did melt:\\
For ever she cried, O turn, love,\\
And I prithee, love, turn to me, \\
For thou art the man that alone canst\\
Procure my liberty.
\end{altverse}

\begin{altverse}
He said, My love, I will please thee,\\
Thy heaviness grieves me sore,\\
But let not sorrow now seize thee,\\
I never will grieve thee more;\\
I’ll turn to thee, my own kind heart,\\
Dear love, I’ll turn to thee,\\
For I am the man that now am come \\
To procure thy liberty.
\end{altverse}

\begin{altverse}
I’ll crown thee with a garland of straw, then, \\
And marry thee with a rush ring;\\
My frozen heart it will thaw then,\\
And merrily we will sing:\\
But ever she cried, O turn, love,\\
And I prithee, love, turn to me,\\
For thou art the man that alone canst \\
Release my misery.
\end{altverse}

\begin{altverse}
Most lovingly he embrac’d her,\\
And call’d her his heart’s delight\\
And close by his side he plac’d her,—\\
All sorrows were vanquished quite.\\
And now she, for joy, cried, Turn, love, \\
And I prithee, love, turn to me,\\
For thou art the man that alone hast \\
Released me of misery.
\end{altverse}
\end{dcverse}

The following ballads, sung to the tune, are in the Roxburghe
Collection:—

Vol. ii. 88. “The Courteous Health; or, The Merry Boys of the
Times.

\settowidth{\versewidth}{He that loves sack, doth nothing lack,}
\begin{dcverse}
\begin{altverse}
He that loves sack, doth nothing lack, \\
If he but loyal be; \\
He that denies Bacchus’ supplies\\
Shews mere hypocrisy.”
\end{altverse}
\end{dcverse}

“To a new tune, \textit{Come, boys, fill us a bumper}, or \textit{My lodging is on
the cold ground};” with the burden, “A brimmer to the King,” and
beginning—

\settowidth{\versewidth}{Come, boys, fill us a bumper,}
\begin{dcverse}
\begin{altverse}
\vleftofline{“}Come, boys, fill us a bumper, \\
We’ll make the nation roar; \\
She’s grown sick of a Rumper,\\
That sticks on the old score,” \&c.
\end{altverse}
\end{dcverse}

Vol. iii. 196. “The Old Man’s Complaint; or, The unequal matcht
Couple,” \&c. “Tune of \textit{I prithee, love, turn to me}.”

Vol. ii. 520. “Wit bought at a dear rate,” \&c. “To the tune of
\textit{Turn, love, I prethee, love, turn to me}.” Printed by F. Coles; and
begins—

\settowidth{\versewidth}{If all the world my mind did know}
\begin{scverse}
\begin{altverse}
\vleftofline{“}If all the world my mind did know,\\
I would not care a pin,” \&c.
\end{altverse}
\end{scverse}

Vol. iii. 144. “The faithful Lover’s Farewell; or, Private News
from Chatham,” \&c., “To the tune of \textit{My lodging is on the cold
ground}.” Printed for Sarah Tyus, at the Three Bibles, on London
Bridge.” Begins—

\settowidth{\versewidth}{As I in a meadow was walking,}
\begin{dcverse}
\begin{altverse}
\vleftofline{“}As I in a meadow was walking, \\
Some two or three weeks ago, \\
I heard two lovers a-talking,\\
And trampling to and fro,” \&c.
\end{altverse}
\end{dcverse}
\end{fixedpage}%528
\pagebreak

\begin{fixedpage}%529
\rectoheader

There are many more in other collections of ballads; as, for
instance, in that formed by Mr. Halliwell (see Nos. 106, 118, 161,
and 385, in the printed catalogue); but enough have already been
quoted to prove the extreme and long-continued popularity of \textit{My
lodging is on the cold ground}.

The only difficulty is in ascertaining the precise time when
Matthew Lock’s tune was discarded, and that now universally known
took its place. I have not found the former in print after 1670, but
it may have been included in some of the editions of \textit{Apollo’s
Banquet}, between 1670 and 1690, which I have never seen. The air now
known is printed on all the broadsides, with music, of the last
century; and it is possible that the ballad-singers may have
altogether discarded Matthew Lock’s tune, and adopted another,—a
liberty subsequently taken with Carey’s air to his ballad of \textit{Sally in
our Alley}, although quite as melodious as that which they substitued.
There is a song to the tune of \textit{My lodging it is on the cold ground} in
\textit{The Rape of Helen}, 1737, but that ballad-opera is printed without
music. The words and music are printed in \textit{Vocal Music, or The
Songster’s Companion}, 8vo., 1775, and it has been a stock-song in
print from that time. At the commencement of the present century it
acquired a new impetus of popularity from the singing of Mrs.
Harrison, at Harrison and Knyvett’s concerts; and subsequently from
that of Mrs. Salmon. About this time it was claimed as an Irish tune
by the late T. Moore including it among his \textit{Irish Melodies}. I believe
there is no ground whatever for calling it Irish. The late Edward
Bunting, who was engaged to note down all the airs played by the
harpers of the different provinces of Ireland, when they were
collected together at Belfast, in 1792, and who devoted a long life
to the collection of Irish music, distinctly assured me that he did
not believe it to be Irish,—that no one of the harpers played the
tune,—and that it had no Irish character. I do not think a higher
authority as to Irish music could be quoted, or one more tenacious of
any infringement upon airs which he considered to be of truly Irish
origin. I might add the testimony of Dr. Crotch, Messrs. Ayrton, T.
Cooke, J. Augustine Wade, and others, both Irish and English, who
have expressed similar opinions to that of Bunting; but, in fact,
there is a total want of evidence, external and internal, of its
being an Irish tune. About the same time that Moore claimed it, it
was printed in Dublin, in Clifton’s “\textit{British} Melodies.”

The curious will find a copy of the song for the voice, with
accompaniment for the virginals or harpsichord, reprinted from one of
the broadsides, in \textit{Nat. Eng. Airs}.

In Ritson’s \textit{Scottish Songs}, i. 187, 1794, there is a song written
by J. D., commencing, “I lo’e na a laddie but ane, to the tune of
\textit{Happy Dick Dawson}.” The tune there printed is a version of \textit{My lodging
is on the cold ground}, curtailed in each alternate phrase to suit
words in a shorter metre. I have not looked for the song of \textit{Happy
Dick Dawson}, but believe that “I lo’e na a laddie but ane” was first
printed to that tune in 1790, in the third volume of Johnson’s \textit{Scots’
Musical Museum}.
\end{fixedpage}%529
\pagebreak

\begin{fixedpage}%530
\versoheader

The following is the popular air, with the words usually sung.

\settowidth{\versewidth}{With a garland of straw I’ll crown thee, love,}
\indentpattern{01014}
\begin{dcverse}
\begin{patverse}
With a garland of straw I’ll crown thee, love, \\
I’ll marry thee with a rush ring; \\
Thy frozen heart shall melt with love, \\
So, merrily I shall sing. \\
Yet still he cried, \&c.
\end{patverse}

\begin{patverse}
But, if thou wilt harden thy heart, love,\\
And be deaf to my pitiful moan,\\
Then I must endure the smart, love,\\
And tumble in straw, all alone.\\
Yet still he cried, \&c.
\end{patverse}
\end{dcverse}


\musictitle{LAY THE BENT TO THE BONNY BROOM.}

This ballad and tune are contained in the second volume of the
early editions of \textit{Pills to purge Melancholy} , and in the fourth volume
of the later.

Copies of the ballad are also in the Pepys (iii. 19), Douce
(169), and Halliwell Collections (No. 253). It is entitled “A noble
riddle wisely expounded; or, The Maid’s answer to the Knight’s three
questions:—

\settowidth{\versewidth}{Won a young knight to joyn with her in marriage.}
\begin{scverse}
She, with her excellent wit and civil carriage,\\
Won a young knight to joyn with her in marriage.\\
This gallant couple now are man and wife,\\
And she with him doth lead a pleasant life.”
\end{scverse}

“The tune is \textit{Lay the lent to the bonny broom}.”
\end{fixedpage}%530
\pagebreak

\begin{fixedpage}%531
\rectoheader

The copy in the Halliwell Collection was printed for F. Coles, T.
Vere, and W. Gilbertson, who all commenced publishing before the
Restoration. It is in W. Thackeray’s list of ballads, and the copy in
the Douce Collection was printed by Thomas Norris, at the
Looking-glass on London Bridge.

I imagine it to be of earlier date than any copy I have found,
and probably derived from a minstrel ballad. The late T. Dibdin
informed me that the tune was introduced as a duet in O’Keefe’s
comedy, \textit{The Highland Reel}, in 1788. I have not seen that copy.

\settowidth{\versewidth}{There was a knight of noble worth,}
\begin{dcverse}
There was a knight of noble worth,\\
\vin\vin\vin\vin\vin\vin\vin\vin\textit{Lay the bent, \&c}. \\
Who also lived in the North, \textit{Fa, la, \&c.} \\
This knight, of courage stout and brave, \\
A wife he did desire to have;\\
He knocked at the lady’s gate,\\
One evening when it was late.\\
The eldest sister let him in.\\
And pinn’d the door with a silver pin.
\end{dcverse}

The knight offers to marry the youngest, if her wit should equal
her good looks; and, to test it, proposes to ask three questions.

\settowidth{\versewidth}{“Tell me what your three questions be I”}
\begin{dcverse}
“Kind sir, in love, O then,” quoth she,\\
“Tell me what your three questions be?”\\
“O what is longer than the way,\\
Or what is deeper than the sea,\\
Or what is louder than a horn,\\
Or what is sharper than a thorn,\\
Or what is greener than the grass,\\
Or what is worse than woman was?”\\
“O love is longer than the way,\\
And hell is deeper than the sea,\\
And thunder’s louder than the horn,\\
And hunger’s sharper than a thorn;\\
Aud poison’s greener than the grass,\\
And the Devil’s worse than woman was.” \\
When she these questions answered had, \\
The knight became exceeding glad\dots\\
And after, as ’tis verified,\\
He made of her his lovely bride.\\
So now, fair maidens all, adieu,\\
This song I dedicate to you,\\
And wish that you may constant prove.\\
Unto the man that you do love.
\end{dcverse}

\musictitle{COME, LASSES AND LADS.}

The earliest copy I have found of this still popular ballad is in
\textit{Westminster Drollery}, Part II., 1672, entitled “The rural dance about
the May-pole: The tune, the first Figure-Dance at Mr. Young’s Ball,
in May ’71.” It is also printed with the tune (but the words much
altered and abbreviated) in \textit{Pills to purge Melancholy} , vol. i. of the
early editions, and vol. iii. of 1719. The copy in \textit{Tixall Poetry},
4to., 1813, taken from an old manuscript, contains a final stanza not
to be found in \textit{Westminster Drollery}.

\end{fixedpage}%531
\pagebreak

\begin{fixedpage}%532
\versoheader

The tune has passed through all the processes of alteration that
tradition so frequently engenders, till at last it has become
difficult to trace any resemblance between the present version and
the primitive one. The following is the tune as printed in the
\textit{Pills}:—

The following is the traditional tune. The words are in several
other collections besides those above-mentioned, and are still in
print in Seven Dials.
\end{fixedpage}%532
\pagebreak


\begin{fixedpage}%533
\rectoheader

\settowidth{\versewidth}{Strike up, says Wat,—agreed, says Matt,}
\begin{dcverse}

\begin{altverse}
Strike up, says Wat,—agreed, says Matt,\\
And I prithee, fiddler, play;\\
Content, says Hodge, and so says Madge,\\
For this is a holiday.\\
Then every lad did doff \\
His hat unto his lass,\\
And every girl did curtsey, curtsey,\\
Curtsey on the grass.
\end{altverse}

\begin{altverse}
Begin, says Hal,—aye, aye, says Mall,\\
We’ll lead up \textit{Packington's Pound};\\
No no, says Noll, and so says Doll,\\
We’ll first have \textit{Sellinger’s Round}.\\
Then every man began \\
To foot it round about,\\
And every girl did jet it, jet it,\\
Jet it in and out.
\end{altverse}

\begin{altverse}
You're out, says Dick,—not I, says Nick, \\
’Twas the fiddler play’d'it wrong;\\
’Tis true, says Hugh, and so says Sue,\\
And so says every one.\\
The fiddler then began \\
To play the tune again,\\
And every girl did trip it, trip it,\\
Trip it to the men.
\end{altverse}

\begin{altverse}
Let’s kiss, says Jane,—content, says Nan,\\
And so says every she;\\
How many? says Batt,—why three, says Matt, \\
For that’s a maiden’s fee.\\
The men, instead of three,\\
Did give them half a score;\\
The maids in kindness, kindness, kindness,\\
Gave ’em as many more.
\end{altverse}

\begin{altverse}
Then, after an hour, they went to a bow’r, \\
And play’d for ale and cakes;\\
And kisses too,—until they were due \\
The lasses held the stakes.\\
The girls did then begin \\
To quarrel with the men,\\
And bade them take their kisses back,\\
And give them their own again.
\end{altverse}

\begin{altverse}
Now there they did stay the whole of the day, \\
And tired the fiddler quite\\
With dancing and play, without any pay, \\
From morning until night.\\
They told the fiddler then \\
They’d pay him for his play,\\
And each a twopence, twopence, two\-pence, \\
Gave him, and went away.
\end{altverse}

\begin{altverse}
[Good night, says Harry,—good night, says Mary;\\
Good night, says Dolly to John;\\
Good night, says Sue, to her sweetheart, Hugh;\\
Good night, says every one. \\
Some walk’d, and some did run;\\
Some loiter’d on the way,\\
And bound themselves by kisses twelve\\
To meet the next holiday.]
\end{altverse}

\end{dcverse}
\end{fixedpage}%533
\pagebreak


\begin{fixedpage}%534
\versoheader

\musictitle{ROGER DE COVERLEY.}

This still popular dance-tune, from which Addison borrowed the
name of Sir Roger de Coverley in \textit{The Spectator}, is contained in
Playford’s \textit{Division Violin}, 1685; in \textit{The Dancing Master} of 1696, and
all subsequent editions; also in many ballad-operas, and more recent
publications.

In a manuscript now in my possession, which was written about the
commencement of the last century, but contains tunes of a much
earlier date, it is entitled “\textit{Old Roger of Coverlay for evermore}, a
Lancashire Hornpipe;” in \textit{The Dancing Master}, “Roger \textit{of} Coverly;” in
the ballad-opera of \textit{Polly}, “Roger \textit{a} Coverly;” in \textit{Robin Hood}, “Roger
\textit{de} Coverly;” and in \textit{Tom Jones}, 1769, “\textit{Sir} Roger de Coverley.”

There is a song with the burden, “O brave Roger a Cauverly,” in
\textit{Pills to purge Melancholy}, vi. 31; and which I suppose should be to
the tune, although four bars of \textit{Old Sir Simon the King} are printed
above it. Both are in \timesig{9}{4} time. It commences very abruptly, as if it
were a fragment, instead of an entire song— 

\settowidth{\versewidth}{Bulls and bears, and lions and dragons,}
\indentpattern{010101011212}
\begin{dcverse}
\begin{patverse}
\vleftofline{“}She met with a countryman \\
In the middle of all the Green; \\
And Peggy was his delight, \\
And good sport was to be seen. \\
But ever she cried, Brave Roger,\\
I’ll drink a whole glass to thee;\\
But as for John of the Green,\\
I care not a pin for he.\\
\textit{Bulls and bears, and lions and dragons,\\
And O brave Roger a Cauverly; \\
Piggins and niggins, pints and flagons,\\
O brave Roger a Cauverly.”}
\end{patverse}
\end{dcverse}

These \timesig{9}{4} tunes are not found in the earliest editions of \textit{The
Dancing Master}, perhaps, because they were originally jig and
hornpipe tunes, rather than country-dances. I cannot, in any other
way, account for not having met with \textit{early} copies of tunes to such
well-known ballads as \textit{Arthur a Bradley} (so frequently mentioned by
Elizabethan dramatists), which, from the metre of the words, must
have been sung to an air in \timesig{9}{4} time, and in all probability to this.

According to Ralph Thoresby’s MS. account of the family of
Calverley, of Calverley, in Yorkshire, the dance of \textit{Roger de
Coverley} was named after a knight who lived in the reign of Richard
I. Thoresby was born in 1658. The following extract from his
manuscript was communicated to \textit{Notes and Queries}, i. 369, by Sir
Walter Calverley Trevelyan, Bart.:—“Roger, so named from the
Archbishop [of York], was a person of renowned hospitality, since, at
this day, \textit{the obsolete known tune of Roger a Calverley} is referred to
him, who, according to the custom of those times, kept his Minstrels,
from that, their office, named Harpers, which became a family, and
possessed lands till late years in and about Calverley, called to
this day \textit{Harpersroids} and \textit{Harper’s Spring}.”

Another correspondent of \textit{Notes and Queries}, vi. 37, says that in
Virginia, U.S., the dance is named \textit{My Aunt Margery}, but I find no
English authority for the change.

It is mentioned as one which “the hob-nailed fellows” call for,
in \textit{The History of Robert Powel, the Puppet-showman}, 8vo., 1715.
“Upon the prelude’s being ended, each party fell to bawling and
calling for particular tunes. The hobnail’d fellows, whose breeches
and lungs seem’d to be of the same leather, cried out for \textit{Cheshire
Rounds}, \textit{Roger of Coverly}, \textit{Joan’s Placket}, and \textit{Northern Nancy}.”

Finally, it is known in Scotland under the name of “The Mautman
comes on
\end{fixedpage}%534
\pagebreak

\begin{fixedpage}%535
\rectoheader

Monday,” from a song, which, on the authority of \textit{The Tea Table
Miscellany}, was written by Allan Ramsay.

As this old favorite has again come into fashion (not only here,
but also at foreign Courts), a description of the figure, as now
danced, may interest some of my readers.

Figure of Roger de Coverley.—The couples stand as in other
English country-dances, the gentlemen facing the ladies. First—The
gentleman at the top and the lady at the bottom of the dance advance
to the centre, and turning round each other (giving the right hand)
return to places (four bars of music). Second—The same figure
repeated, but giving the \textit{left} hand (four bars). Third—The same couple
advance a third time, the gentleman bowing and the lady courtesying,
retire (four bars). The fourth is a chain figure, the first gentleman
gives his right hand to his partner and left to the second lady,
right to partner and left to third lady, and so on, the lady, in like
manner, at the same time, giving her right hand to her partner and
left to every gentleman, till they reach, the bottom of the dance.
They then hold up their hands joined, and every couple pass under
them (beginning with the second gentleman and his partner) and
turning outwards, i.e., gentlemen to the right and ladies to the
left, return to places. Then the figure recommences with the second
gentleman (now at the top) and the first lady, now at the bottom of
the dance.
\end{fixedpage}%535
\pagebreak
\begin{fixedpage}%536
\versoheader

\musictitle{THE NORTHUMBERLAND BAGPIPES.}

In the Roxburghe Collection, ii. 363, and Bagford, 643, m. 10, p.
159, is the ballad of “The merry Bagpipes: the pleasant pastime
befwixt a jolly shepherd and a country damsel on a Midsummer’s day in
the morning. To the tune of \textit{March, boys}, \&c.” Licensed according to
order, and printed by C. Bates, next door to the Crown Tavern in West
Smithfield. I have not found the song of “March, hoys but this ballad
is printed, with the tune, in \textit{Pills to purge Melancholy} , ii. 136,
1700, under the title of \textit{The Northumberland Bagpipes}. It is here
arranged with the bagpipe drone.

\settowidth{\versewidth}{And whilst this harmony he did make,}
\begin{dcverse}
\begin{altverse}
And whilst this harmony he did make,\\
A country damsel from the town,\\
A basket on her arm she had,\\
A gathering rushes on the down:\\
Her bongrace was of wended straw,\\
From the sun’s beams her face to free,\\
And thus she began, when she him saw,\\
\textit{If thou wilt pipe, lad, I’ll dance to thee}, \&c.
\end{altverse}
\end{dcverse}

\musictitle{WHEN BUSY FAME.}

\textit{Busy Fame} was a popular tune at the end of the reign of Charles
II., and continued in favour for at least half a century. Several
ballads that were to he sung to it, have already been mentioned; the
following are in the Halliwell Collection:—


\end{fixedpage}%536
\pagebreak

\begin{fixedpage}%537
\rectoheader

No. 47. “Coridon and Parthenia, the languishing shepherd made
happy, or faithful love-rewarded, being a most pleasant and
delectable new Play Song: 

\settowidth{\versewidth}{Here mournful love is turn’d into delight,}
\begin{scverse}
Here mournful love is turn’d into delight,\\
To this we a chaste amorist iuvite.”
\end{scverse}

To the tune of \textit{When busy Fame}. Printed for F. Coles, T. Vere, J.
Wright, J. Clarke, W. Thackery, and T. Passinger. Also, another copy,
printed by P. Brooksby.

No. 180. “The Life of Love,” \&c. “To the tune of \textit{The fair one
let me in}, or \textit{Busy Fame}.” Printed for P. Brooksby, \&c.

No. 349. “The trepanned Virgin; or, Good Advice to Maidens,” \&c.
“Tune, \textit{When busie Fame}.” Printed for Coles, Vere, Wright, \&c.

The song, “When busy Fame,” is in Playford’s \textit{Choice Ayres}, v. 19,
1684; in \textit{Pills to purge Melancholy} , iii. 249, 1707, and v. 164, 1719.
It was composed by T. Farmer.

\settowidth{\versewidth}{Young Coridon, whose stubborn heart}
\begin{dcverse}
\begin{altverse}
Young Coridon, whose stubborn heart \\
No beauty e’er could move,\\
That’smil’d at Cupid’s bow and dart, \\
And brav’d the God of Love,\\
Would view this nymph, and pleas’d, at first,\\
Such silent charms to see,\\
With wonder gaz’d, then sigh’d, and curs’d \\
His curiosity.
\end{altverse}
\end{dcverse}
\end{fixedpage}%537
\pagebreak


\begin{fixedpage}%538
\versoheader

\musictitle{BARBARA ALLEN.}

Under this name, the English and Scotch have each a ballad, with
their respective tunes. Both ballads are printed in Percy’s \textit{Reliques
of Ancient Poetry}, and a comparison will shew that there is no
similarity in the music. It has been suggested that for “Scarlet”
town, the scene of the ballad, we should read “Carlisle” town. Some
of the later printed copies have “Reading” town.

In the Douce Collection there is a different ballad under this
title,—a Newcastle edition, without date.

Goldsmith, in his third Essay, says, “The music of the finest
singer is dissonance to what I felt when our old dairy-maid sung me
into tears with \textit{Johnny Armstrong’s Last Good Night}, or \textit{The Cruelty of
Barbara Allen}.”

A black-letter copy of this ballad, in the Roxburghe Collection,
ii. 25, is entitled “Barbara Allen’s Cruelty; or, The Young Man’s
Tragedy: With Barbara Allen’s Lamentation for her unkindness to her
Lover and herself. To the tune of \textit{Barbara Allen}.” Printed for P.
Brooksby, J. Deacon, J. Blare, and J. Back.

The following is the version printed by Percy: the tune from
tradition, and scarcely one is better known:—

\settowidth{\versewidth}{Young Jemmy Grove on his death-bed lay}
\begin{dcverse}
\begin{altverse}
All in the merry month of May,\\
When green buds they were swellin’. \\
Young Jemmy Grove on his death-bed lay,\\
For love of Barbara Allen.
\end{altverse}

\begin{altverse}
He sent his man unto her then,\\
To the town where she was dwellin’; \\
You must come to my master dear,\\
Giff your name he Barbara Allen.
\end{altverse}

\begin{altverse}
For death is printed on his face,\\
And o’er his heart is stealin’;\\
Then haste away to comfort him,\\
O lovely Barbara Allen.
\end{altverse}

\begin{altverse}
Though death he printed on his face, \\
And o’er his heart is stealin’,\\
Yet little better shall he be \\
For bonny Barbara Allen.
\end{altverse}

\begin{altverse}
So slowly, slowly, she came up,\\
And slowly she came nigh him;\\
And all she said, when there she came, \\
Young man, I think you’re dying.
\end{altverse}

\begin{altverse}
He turn’d his face unto her straight,\\
With deadly sorrow sighing;\\
O lovely maid, come pity me,\\
I'm on my death-bed lying.
\end{altverse}
\end{dcverse}
\end{fixedpage}%538
\pagebreak


\begin{fixedpage}%539
\rectoheader

\settowidth{\versewidth}{Lay down, lay down the corpse, she said,}
\begin{dcverse}
\begin{altverse}
If on your death-bed you do lie,\\
What needs the tale you’re tellin’;\\
I cannot keep you from your death;\\
Farewell, said Barbara Allen.
\end{altverse}

\begin{altverse}
He turn’d his face unto the wall,\\
As deadly pangs he fell in:\\
Adieu! adieu! adieu to you all,\\
Adieu to Barbara Allen.
\end{altverse}

\begin{altverse}
As she was walking o’er the fields,\\
She heard the bell a knellin’;\\
And every stroke did seem to say,\\
Unworthy Barbara Allen.
\end{altverse}

\begin{altverse}
She turn'd her body round about,\\
And spied the corpse a coming;\\
Lay down, lay down the corpse, she said,\\
That I may look upon him.
\end{altverse}

\begin{altverse}
With scornful eye she looked down,\\
Her cheek with laughter swellin’;\\
Whilst all her friends cried out amain,\\
Unworthy Barbara Allen.
\end{altverse}

\begin{altverse}
When he was dead, and laid in grave,\\
Her heart was struck with sorrow,\\
O mother, mother, make my bed,\\
For I shall die to-morrow.
\end{altverse}

\begin{altverse}
Hard-hearted creature him to slight,\\
Who loved me so dearly:\\
O that I had been more kind to him\\
When he was alive and near me!
\end{altverse}

\begin{altverse}
She, on her death-bed as she lay,\\
Begg’d to be buried by him;\\
And sore repented of the day,\\
That she did e’er deny him.
\end{altverse}

\begin{altverse}
Farewell, she said, ye virgins all,\\
And shun the fault I fell in:\\
Henceforth take warning by the fall\\
Of cruel Barbara Allen.
\end{altverse}
\end{dcverse}

\musictitle{O BRAVE ARTHUR OF BRADLEY.}

\settowidth{\versewidth}{“Sing him \textit{Arthur of Bradley}, or, \textit{I am the Duke of Norfolk}.”}
\begin{scverse}
“Sing him \textit{Arthur of Bradley}, or, \textit{I am the Duke of Norfolk}.”\\
\attribution{Wycherley's \textit{Gentleman's Dancing Haster}, 1673.}
\end{scverse}


When I first read the ballad of “Arthur of Bradley,” it struck
me immediately that it must have been sung to the tune of \textit{Roger de
Coverley}. The words ran so glibly to the tune, that I could scarcely
forbear to hum it over to them. I still retain the impression, and
the probabilities are strengthened by having traced \textit{Roger de Coverley}
to an earlier date, and as a Lancashire hornpipe. In the ballad,
Arthur calls upon the piper to play “a hornpipe, that went fine on
the bagpipe,” and no other dance is mentioned at the wedding. There
are many places called Bradley, in England, and, among them, one in
Yorkshire, another in Lancashire, and a third in Derbyshire.

All the black-letter copies of the ballad of “Arthur of Bradley”
that I have noticed, direct it to be sung “to a pleasant new tune;” so
that, unless a copy of \textit{Roger de Coverley} can be found under the name
of “Arthur of Bradley,” or “Saw ye not Pierce the Piper?” the
identification will remain doubtful. One thing, however, is
certain,—that “Arthur of Bradley” \textit{must} have been sung to a tune in \timesig{9}{4}
time, and to one that consisted of twelve bars, \timesig{9}{4} time is common to
English jig and hornpipe tunes.

“Arthur-a-Bradley” is referred to by Ben Jonson, Dekker, and
other Elizabethan dramatists; in Braithwait’s \textit{Strappado for the
Divell}; and in the ballad of “Robin Hood’s birth, breeding, valour,
and marriage.” See also Gifford’s notes to his edition of Ben Jonson,
iv., 401, 410, and 533.

The ballad is printed in “An Antidote against Melancholy: made up
in pills, compounded of witty ballads, jovial songs, and merry
catches,” 1661, and in Ritson’s “Robin Hood,” ii. 210. Ritson retains
the title of the black-letter copies, “A Merry Wedding; or, O brave
Arthur of Bradley.”
\end{fixedpage}%539
\pagebreak

\begin{fixedpage}%540
\versoheader

The first stanza is here adapted to a second version of \textit{Roger de
Coverley}.

There are two other ballads of “Arthur-a-Bradley,”—one
commencing, “All in the merry month of May” (included in the third
volume of the Roxburghe Ballads), and the second, “Come, neighbours,
and listen awhile,” reprinted in “Ancient Poems, Ballads, and Songs
of the Peasantry of England,” by J. H. Dixon. Both are evidently of
later date than the above.

There may have been a fourth ballad, for Gayton, in his
\textit{Festivous Notes} on \textit{Don Quixote}, 4to., 1654, p. 141, says, “’Tis not
alwaies sure that ’\textit{tis merry in hall when beards wag all}, for these
men’s beards wagg’d as fast as they could tug ’em, but mov’d no mirth
at all: They were verifying that song of—

\settowidth{\versewidth}{A beard without haire looks madly.’”}
\begin{scverse}
\vleftofline{‘}Heigh, brave Arthur o’ Bradley,\\
A beard without haire looks madly.’”
\end{scverse}

The last line is not to be found in any of the above-mentioned.
\end{fixedpage}%540
\pagebreak

\begin{fixedpage}%541
\rectoheader

\musictitle{UNDER THE GREENWOOD TREE.}

This ballad, and the tune (noted down in common time, and without
bars), are found among Ashmole’s Manuscripts, at Oxford (36 and 37,
fol. 194, b).

There are two versions in \textit{The Dancing Master} of 1686,—the first
in common, and the second in \timesig{6}{4} time: the first entitled \textit{Under the
Greenwood Tree},—the second (in the additional sheet), \textit{Oh! how they
frisk it}, or \textit{Leather Apron}.

I have only observed one other copy in common time, and that is
in \textit{The Dancing Master} of 1690. In all later editions, and in \textit{Pills to
purge Melancholy}, it is in \timesig{6}{4} time, which the words seem to require.

The popularity of the tune may be inferred from the great number
of ballad-operas in which it was introduced. Among these may be
reckoned \textit{The Devil to pay}, \textit{The Jovial Crew}, \textit{The Village Opera}, \textit{The
Cobblers, Opera}, \textit{The Mad Captain}, \textit{The Court Legacy}, \textit{The Devil of a
Duke}, and \textit{The Woman of Taste}.

Ashmole’s copy of the words differs somewhat from the
black-letter ballads; and, if written at the time when he is stated
to have been intent upon music,— soon after his father’s death, in
1634,—it may be from forty to fifty years older than any printed copy
that I have observed, the earliest of which was published by
Brooksby.\textsuperscript{a}

Ashmole noted down the tune without bars, and bars were in
general use in the reign of Charles II., but not so in that of
Charles I.\textsuperscript{b} The words in his copy begin thus:—

\settowidth{\versewidth}{In summer time, when leaves grow green,}
\begin{dcverse}
\begin{altverse}
\vleftofline{“}In summer time, when leaves grow green, \\
And birds sit on the tree, \\
Let all the lords say what they can, \\
There’s none so merry as we. 
\end{altverse}

\begin{altverse}
There’s Jeffry and Tom, there’s Ursula and John,\\
With Roger and bonny Bettee;\\
Oh! how they do firk it, caper and jerk it, \\
\textit{Under the Greenwood Tree.}
\end{altverse}
\end{dcverse}

The ballads of “King Edward the Fourth and the Tanner of
Tamworth,” and “Robin Hood and the Curtal Friar,” commence precisely
as in Ashmole’s copy, and, the metre of all being the same, it
appears very probable that they were sung to one tune, and therefore,
that this air may yet be traced back to the reign of Elizabeth.
Another ancient ballad, “Robin Hood and the Monk,” begins in a
similar manner, and the eighth line corresponds with the burden of
this ballad.

The tune is sometimes entitled \textit{Caper and firk it} (\ie caper and
\textit{frisk} it) as in “The fair Maid of Islington; or, The London Vintner
over-reach’d: To the tune of \textit{Sellinger's Round}, or \textit{Caper and firk
it}.” (Bagford 643 m. 10, p. 113.) Commencing—

\settowidth{\versewidth}{There was a fair Maid of Islington,}
\begin{dcverse}
\begin{altverse}
\vleftofline{“}There was a fair Maid of Islington, \\
As I heard many tell,\\
And she would to fair London go, \\
Fine apples and pears to sell,” \&c.
\end{altverse}
\end{dcverse}

It is included among the tunes of Christmas Carols in “A Cabinet
of Choice Jewels; or, The Christian’s Joy and Gladness, set forth in
sundry pleasant new

\begin{dcfootnote}
\textsuperscript{a} The earliest date that I have noted to any ballad printed by
Brooksby, is April 12, 1677, when Sir Roger L’Eatrange licensed to
him, “A Kind Husband; or, Advice for Married Men. To the tune of \textit{The
Ladies’ Delight}, or Never let a man take heavily.” A copy in the
Rawlinson Collection of “Olde Balades,” Bodleian Library.

\textsuperscript{b} Bars were used to music \textit{in score} in the fifteenth
century; but, in England, each part was usually printed
separately, and then bars were thought unnecessary. \textit{The Dancing
Masters} of 1651 and 1652, being for one instrument, have no bars; but
the \textit{score} in the moral play, \textit{The four Elements}, printed by Rastell
(to which Dr Dibdin assigns the date of 1510), is barred. So far as
I have observed, all music In the ordinary notation, even for one
voice or one instrument, was barred after 1660.
\end{dcfootnote}
\end{fixedpage}%541
\pagebreak

\begin{fixedpage}%542
\versoheader

Christmas Carols,” 1688. These are “A carol for Christmas Day, to
the tune of \textit{Over hills and high mountains}; for Christmas Day at
night, to the tune of \textit{My life and my death}; for St. Stephen’s Day, to
the tune of \textit{O cruel bloody fate}; for New Year’s Day, to the tune of
\textit{Caper and jerk it}; and for Twelfth Night, to the tune of \textit{O Mother,
Roger}.\textsuperscript{a} A copy of this curious collection is in Wood’s Library, at
Oxford.

“A delightful song in honour of Whitsontide, to the tune of \textit{Caper
and jerk it},” is contained in \textit{Canterbury Tales}, \&c., printed in Bow
Churchyard. It commences—

\settowidth{\versewidth}{Now Whitson holidays they are come,}
\begin{scverse}
\begin{altverse}
\vleftofline{“}Now Whitson holidays they are come,\\
Each lass shall find her mate.”
\end{altverse}
\end{scverse}

There are many more ballads to the tune. The following eight
stanzas are selected from the original, which is very long, and in
two parts. In the black-letter copies (two of which are in the Douce
Collection), it is entitled, “The West-country Delight; or, Hey for
Zommersetshire,” \&c.; in the \textit{Pills}, “The Countryman’s Delight.”

\begin{dcfootnote}
\textsuperscript{a} Of these tunes, “My life and my death are both in 
your power” is the composition of Mr. William Turner 
(see \textit{Theater of Music}, Book i., 1685); “O Mother, Roger 
with his kisses,” is to be found in \textit{Pills to purge Melancholy}, 
and in \textit{The Dancing Master}; and the remaining are
contained in this collection.

\end{dcfootnote}
\end{fixedpage}%542
\pagebreak


\begin{fixedpage}%543
\rectoheader

\indentpattern{010101017}

\settowidth{\versewidth}{And then do we skip it, caper and trip it,}
\begin{dcverse}
\begin{patverse}
Our music is a little pipe,\\
That can so sweetly play;\\
We hire \textit{Old Hal} from Whitsuntide,\\
Till latter Lammas-day;\\
On sabbath days and holy-days,\\
After ev’ning prayer comes he;\textsuperscript{a}\\
And then do we skip it, caper and trip it,\\
Under the green-wood tree.\\
In summer time, \&c.
\end{patverse}

\begin{patverse}
Come, play us \textit{Adam and Eve}, says Dick;\\
What’s that? says little Pipe;\\
\textit{The Beginning of the World}\textsuperscript{b} quoth Dick;\\
For we are dancing-ripe;\\
Is’t that you call? then \textit{have at all}—\\
He play’d with merry glee;\\
O then did we skip it, caper and trip it.\\
Under the green-wood tree.\\
In summer time, \&c.
\end{patverse}

\begin{patverse}
O’er hills and dales, to Whitsun-ales,\\
We dance a merry fytte;\\
When Susan sweet with John doth meet,\\
She gives him Hit for Hit—\\
From head to foot she holds him to ’t,\\
And jumps as high as he;\\
Oh how they spring it, flounce and fling it,\\
Under the green-wood tree.\\
In summer time, \&c.
\end{patverse}

\begin{patverse}
My lord’s son must not be forgot,\\
So full of merry jest;\\
He laughs to see the girls so hot,\\
And jumps it with the rest.\\
No time is spent with more content,\\
In camp, in court, or city,\\
So long as we skip it, frisk it and trip it,\\
Under the green-wood tree.\\
In summer time, \&c.
\end{patverse}

\begin{patverse}
We oft go to Sir William’s ground,\\
And a rich old cub is he;\\
And there we dance around, around,\\
But never a penny we see.\\
From thence we get to Somerset,\\
Where men are frolic and free,\\
And there do we skip it, frisk it and trip it,\\
Under the green-wood tree.\\
In summer time, \&c.
\end{patverse}

\begin{patverse}
We fear no plots of Jews or Scots,\\
For we are jolly swains;\\
With plow and cow, and barley-mow,\\
We busy all our brains;\\
No city cares, nor merchant’s fears\\
Of wreck or piracy;\\
Therefore we skip it, frisk it and trip it,\\
Under the green-wood tree.\\
In summer time, \&c.
\end{patverse}

\begin{patverse}
On meads and lawns we trip like fauns,\\
Like fillies, kids and lambs;\\
We have no twinge to make us cringe,\\
Or crinkle in the hams;\\
When the day is spent, with one consent,\\
Again we all agree,\\
To caper it and skip it, trample and trip it,\\
Under the green-wood tree.\\
In summer time, \&c.
\end{patverse}
\end{dcverse}

\begin{dcfootnote}
\textsuperscript{a} Bishop Earle, in his \textit{Microcosmographie}, describing a “Plain
countryfellow, or downright clown,” says, “Sunday he esteems a day to
make merry in, and thinks a bag-pipe as essential to it as evening
prayer. He walks very solemnly after service, with his hands coupled
behind him, and censures” (\ie criticises) “the dancing of his
parish.” Burton, in his
\textit{Anatomy of Melancholy}, says: “Young lasses are never better
pleased, than when, as upon a holiday, after evensong, they may meet
their sweet-hearts, and dance about a May-pole, or in a Town-green,
under a shady elm.” 

\textsuperscript{b} For \textit{The Beginning of the World}, or \textit{Sellinger's
Round}, see ante vol. i., p. 69.
\end{dcfootnote}
\end{fixedpage}%543
\pagebreak

\begin{fixedpage}%544
\versoheader

An answer to the preceding Somersetshire ballad will be found in
the Douce Collection, and to be sung to the same tune. It is “Hey for
our town, but a fig for a Zommersetshire;” and commences—

\settowidth{\versewidth}{In winter time, when flow’rs do fade,}
\begin{dcverse}
\begin{altverse}
\vleftofline{“}In winter time, when flow’rs do fade,\\
And birds forsake the tree,\\
Let lords and ladies play at cards,\\
There’s none so merry as we,” \&c.
\end{altverse}
\end{dcverse}
The burden is “Under the holly-bush tree.”

\musictitle{THE OLD LANCASHIRE HORNPIPE.}

\settowidth{\versewidth}{“There is a Lancashire Hornpipe in my throat; hark! how it tickles, with doodle, doodle, doodle.”}
\begin{scverse}\small
“There is a Lancashire Hornpipe in my throat; hark! how it tickles, with doodle, doodle, doodle.”\\
\attribution{Ford’s \textit{The Witch of Edmonton}.}
\end{scverse}

At page 12 of an edition of \textit{The Dancing Master}, the exact
reference to which I have mislaid (perhaps in one of the volumes of
Walsh’s \textit{Compleat Country Dancing Master}), this tune is entitled \textit{The
Old Lancashire Hornpipe}. In \textit{Apollo's Banquet}, 1669, 1690, and 1693,
it is called \textit{A Jigg}, and has twelve divisions or variations. There
were hornpipes of various descriptions; some being called
jig-hornpipes, or hornpipe-jigs, others bagpipe-hornpipes. One of the
former will be found in the first edition of \textit{Apollo's Banquet}; and
several of the latter in “An extraordinary. Collection of pleasant
and merry humours; containing Hornpipes, Jiggs, North-Country Frisks,
Morrises, Bagpipe-Hornpipes, and Rounds,”\textsuperscript{a} \&c. The hornpipe-jig in
\textit{Apollo's Banquet} (although not so barred) is in \timesig{9}{4} time. About 1697,
Thomas Marsden published a “Collection of original Lancashire
Hornpipes;” but I have not been able to find a copy in any library,
public or private.

In Vanbrugh’s comedy of \textit{Æsop}, act v., the trumpets were to sound
a melancholy air till Æsop appeared, and then the violins and
hautboys to “strike up a Lancashire Hornpipe.”

The instrument called the hornpipe, from which the dance derived
its name, was in use in England as late as the reign of Charles II.,
and perhaps later. It is, in all probability, the same as the
pib-corn (which means horn-pipe) said to be still in use in Wales.
The pipe of the latter is of hollow wood, with holes for the fingers
at regulated distances, and with horn at each end; a small piece for
the mouth, and a larger for the escape of the sound.

Chaucer mentions the hornpipe as a Cornish instrument,—

\settowidth{\versewidth}{Controve he would, and foule faile,}
\begin{dcverse}
\vleftofline{“}Controve he would, and foule faile, \\
With Hornpipes of Cornwaile. \\
In Floitea made he discordaunce,\\
And in his musike with mischaunce,’' \&c.\\
\end{dcverse}
\attribution{\textit{Romaunt of the Rose}.}

In the sixteenth and seventeenth centuries, the counties most
famous for the dance of the hornpipe were Derbyshire,
Nottinghamshire, and Lancashire. Ben Jonson, in his \textit{Love’s Welcome at
Welbeck}, says,—

\settowidth{\versewidth}{Shall fetch the fiddles out of France,}
\begin{dcverse}
\vleftofline{“}Your firk-hum jerk-hum to a dance,\\
Shall fetch the fiddles out of France,\\
To wonder at the hornpipes here\\
Of Nottingham and Derbyshire;”
\end{dcverse}

\begin{dcfootnote}
\textsuperscript{a} There were three publications under this title, which I have
not had the opportunity to compare. The first, mentioned by Bagford
as having been printed by Daniel Wright in 1710 (small oblong of 35
pages); the second in the British Museum, a. 10. 5, dated 1720; and
the third
printed by John Young, at the Dolphin and Crown, at the West end
of St. Paul’s Church, without date. A copy of the last is in the
possession of Mr. George Daniel, of Canonbury.
\end{dcfootnote}
\end{fixedpage}%544
\pagebreak

\begin{fixedpage}%545
\rectoheader

and he gives a song to the hornpipe tune, which was to be
accompanied on the bagpipe.

Under an engraving of Hale, the Derbyshire piper, by Sutton
Nicholls, are the music of his hornpipe and the following lines:—

\settowidth{\versewidth}{Before three monarchs I my skill did prove}
\begin{scverse}
\vleftofline{“}Before three monarchs I my skill did prove\\
Of many lords and knights had I the love;\\
There’s no musician e’er did know the peer\\
Of \textsc{Hale the Piper} in fair Darby-shire.\\
The consequence in part you here may know,\\
Pray look upon his Hornpipe here below.”\\
\attribution{Quoted from Daniel’s \textit{Merry England}.}
\end{scverse}

In “Old Meg of Herefordshire for a Mayd Marian, and Hereford town
for a Morris Dance,” 1609, the especial credit for hornpipes is given
to Lancashire. “The Court of Kings for stately Measures; the city for
light heels and nimble footing; the country for shuffling dances”
[jigs?]; “Western men for gambols; Middlesex men for tricks above
ground; Essex men for the Hay; \textit{Lancashire for Hornpipes};
Worcestershire for“bagpipes; but Herefordshire for a Morris-dance,
puts down not only all Kent, but very near (if one bad line enough to
measure it) three quarters of Christendom.”

Michael Drayton, in his \textit{Polyolbion}, also says—

\settowidth{\versewidth}{The neat Lancastrian nymphs, for beauty that excell,}
\begin{scverse}
\vleftofline{“}The neat Lancastrian nymphs, for beauty that excell,\\
That, for the Hornpipe round, do bear away the bell;’’ 
\end{scverse}
and again—

\begin{scverse}
\vleftofline{“}Ye lustie lasses, then, in Lancashire that dwell,\\
For beautie that are sayd to beare away the bell,\\
Your countries Hornpipe yee so minsingly that tread,” \&c.
\end{scverse}

Hornpipes were not then danced only by one or two persons, as
now; Drayton, above, speaks of the “hornpipe round.” So George Peele,
in his \textit{Arraignment of Paris}, 1584—

\begin{scverse}
\vleftofline{“}The round in a circle our sportance must be,\\
Hold hands in a hornpipe, all gallant in glee;”
\end{scverse}
and again Drayton—

\begin{scverse}
\vleftofline{“}So blyth and bonny now the lads and lasses are.\\
And ever as anon the bagpipe up doth blow,\\
Cast in a gallant Round, about the hearth they goe,\\
And at each pause they kisse: was never seene such rule,\\
In any place but heere, at Boon-fire, or at Yeule;\\
And every village smokes at Wakes with lusty cheere,\\
Then, Hey, they cry, for Lun, and Hey for Lancashire.”
\end{scverse}
Spenser, also, in his Pastorals, mentions the hornpipe as a dance
for many persons—

\begin{scverse}
\vleftofline{“}Before them yode a lustie tabrere,\\
That to the many a horne pype playd,\\
Whereto they dauncen, eche one with his mayd;\\
To see these folks make so jovisaunce,\\
Made my heart after the pype to daunce.”
\end{scverse}

I suppose the manner of dancing the hornpipe in Lancashire
differed, in some way, from that of other counties; because in one of
the bills of public enter-
\end{fixedpage}%545
\pagebreak


\begin{fixedpage}%546
\versoheader

tainments quoted by Mr. Daniel, in his \textit{Merry England}, of about
the year 1691, John Sleepe advertises “a young man that dances a
hornpipe, \textit{the Lancaster way}, extraordinary finely.”

Lancashire was equally famous for pipers and fiddlers; for a note
upon whom I refer the reader to Gifford’s Ben Jonson, v. 436; but
Lincolnshire disputed with Worcestershire the honor of the bagpipes.
In Drayton’s \textit{Blazons of the Shires}, he says— 

\settowidth{\versewidth}{And bells and bagpipes next, belong to Lincolneshire;”}
\begin{scverse}
\vleftofline{“}Beane-belly Lestershire, her attribute doth beare,\\
And bells and bagpipes next, belong to Lincolneshire;”
\end{scverse}
and again,
in his twenty-fifth Song,—

\begin{scverse}
\vleftofline{“}Thou, Wytham, mine own town, first water’d with my source,\\
As to the Eastern sea I hasten on my course,\\
Who sees so pleasant plains, or knows of fairer scene?\\
Whose swains in shepherd’s gray, and girls in Lincoln green.\\
Whilst some the rings of bells, and some the bagpipes ply,\\
Dance many a merry Round, and many a Hydegy.”
\end{scverse}

A variety of notices about Lincolnshire bagpipes have been
collected by the commentators on Shakespeare. The bagpipe was quite a
rustic instrument, and generally held in contempt. “It seems you
never heard good music, that commend a bagpipe,” is a figurative
speech in Middleton’s \textit{Any thing for a quiet life}; and again, in \textit{The
Witch}, “’Twill be a worthy work to put down all these pipers; ’tis a
pity there should not be a statute against them, as against
fiddlers.” Ben Jonson says, “A rhyme to him is worse than cheese or a
bagpipe,” \&c., \&c. The contemptuous similes to the bagpipe by
dramatists, such as, “that snuffles in the nose like a decayed
bagpipe,” are extremely numerous.”
\end{fixedpage}%546
\pagebreak

\begin{fixedpage}%547
\rectoheader

\musictitle{THE WAITS.}

Waits, or Waights, seem originally to have been a kind of musical
watchmen, who, in order to prove their watchfulness, were required to
pipe at stated hours of the night. The hautboy was also called a
waight,—perhaps from being the pipe upon which they commonly
played,—but there are early instances of the use of other pipes by
Waits, as in a passage quoted by Mr. Sandys, from the old lay of
Richard Cœur-de-Lion:—

\settowidth{\versewidth}{And pypyd a moot in a flagel.”}
\begin{scverse}
\vleftofline{“}A wayte ther com in a kernel (\textit{battlement}),\\
And pypyd a moot in a flagel.”
\end{scverse}

This “flagel” was probably a pipe of which the “flagelet” (or, as
now spelled, “flageolet”), is the diminutive.

Mr. Sandys remarks that “in the time of Henry the Third, Simon le
Wayte held a virgate of land at Rockingham, in Northamptonshire, on
the tenure of being castle-wayte, or watch; and the same custom was
observed in other places.” (\textit{Christmas-Tide}, p. 83.) Mr. E. Smirke,
who quotes many such cases, in his \textit{Observations on Wait Service
mentioned in the Liber Winton, or Winchester Domesday}, adds that, in
the earldom of Cornwall, they who held their lands by the tenure of
keeping watch at the castle-gate of Launceston, “owed suit to a
special court, in the nature of a court baron, called the ‘Curia
vigiliæ,’ ‘Curia de gayté,’ or ‘Wayternesse Court,' of which many
records are still extant in the offices of the Exchequer, and among
the records of the Duchy.” (Archaeological Journal, No. 12, Dec.
1846.)

The duties of a wayte are thus defined in the \textit{Liber niger Domus
Regis} (published, with additions, by Stephen Batman), which contains
an account of the musicians, and others, retained in the household
establishment of King Edward IV.:

“\textsc{A Wayte}, that nightely from Mychelmas to Shreve Thorsdaye
\textit{pipethe watche} within this courte fowere tymes; in the Somere nightes
three tymes, and makethe \textit{bon gayte} at every chambere doare and
offyce, as well for feare of pyckeres and pillers. He eatethe in the
halle with Mynstrelles, and takethe lyverey at nighte a loafe, a
galone of ale, and for Somere nightes two eandles [of] pich, and a
bushel of coles; and for Wintere nightes halfe a loafe of bread, a
galone of ale, four candles pich, a bushel coles: Daylye whilst he is
presente in Court for his wages, in Cheque-roale, allowed iiii\textit{d}. ob.
or else iii\textit{d}. by the discresshon of the Steuarde and Tressorore, and
that after his cominge and deservinge: Also cloathinge with the
Houshold Yeomen or Mynstrelles lyke to the wages that he takethe: An
he be sycke, he taketh two loaves, two messe of great meate, one
galone ale. Also he parteth with the houshold of general gyfts, and
hathe his beddinge carried by the Comptrolleres assignment; and,
under this yeoman, to be a Groome-Waitere. Yf he can excuse the
yeoman in his absence, then he takethe rewarde, clotheinge, meat, and
all other things lyke to other Grooms of Houshold. Also this
Yeoman-Waighte, at the making of Knightes of the Bathe, for his
attendance upon them by nighte-time, in watchinge in the Chappelle,
hathe to his fee all the watchinge clothing that the Knight shall
wear upon him.”

Three waytes were included among the minstrels in the service of
Edward III.

The musicians of towns and corporations were also termed waits.
The city of London had its waits, who attended the Lord Mayor on
public occasions, such as
\end{fixedpage}%547
\pagebreak

\begin{fixedpage}%548
\versoheader

Lord Mayor’s Day, and on public feasts and great dinners. They
are described as having blue gowns, red sleeves, and caps, every one
having his silver collar about his neck.

In 1599, Morley thus speaks of them in his dedication of his
\textit{Consort Lessons}, for six instruments, to the Lord Mayor and
Aldermen:—“But, as the ancient custom of this most honorable and
renowned city hath been ever to retain and maintain excellent and
expert musicians, to adorn your Honour’s favours, feasts, and solemn
meetings,—to those, your Lordship’s Wayts, I recommend the same, —to
your servants’ careful and skilful handling.”

When Charles II., on his restoration, passed through the city of
London to Whitehall, he was, according to Ogilby, entertained with
music from a band of eight waits at Crutched Friars, of six at
Aldgate, and six in Leadenhall Street. Roger North, who lived in his
reign, says: “As for corporation and mercenary musick, it was chiefly
flabile” (\ie, for wind instruments), “and the professors, from
going about the streets in a morning, to \textit{wake} folks, were and are yet
called Waits, quasi Wakes.” I doubt this derivation, for the meaning
of the word seems rather to be “to watch” than “to awaken” (in the
glossary to Tyrwhitt’s Chaucer, we find “Wake, v. Sax., \textit{To watch},” and
“Waite, v. Fr., \textit{To watch}”); but the passage proves that waits then
went about the streets at unseasonable hours, as they now do, within
a few days of Christmas, in order to earn a Christmas-box.

In Davenant’s \textit{Unfortunate Lovers}, Rampiro says:—
\settowidth{\versewidth}{So often waken me with their grating gridirons}
\begin{scverse}
\vin\vin\vin\vin\vin\vin “the fidlers do \\
So often waken me with their grating gridirons\\
And \textit{good morrows}, I cannot sleep for them.”
\end{scverse}

John Cleland, in his “Essay on the Origin of the Musical Waits at
Christmas,” appended to his “Way to things by words and to words by
things,” 8vo., 1766, says: “But at the ancient Yule, or Christmas
time especially, the dreariness of the weather, the length of the
night, would naturally require something extraordinary to wake and
rouse men from their natural inclination to rest, and from, a warm
bed at that hour. The summons, then, to the Wakes of that season were
given by music, going the rounds of invitation to the mirth or
festivals which were awaiting them. In this there was some
propriety—some object; but where is there any in such a solemn piece
of banter as that of music going the rounds and disturbing people in
vain? For surely any meditation to be thereby excited on the
holiness of the ensuing day could hardly be of great avail, in a bed
between sleeping and waking. But such is the power of custom to
perpetuate absurdities.” 

In nearly all the books of household
expenditure in early times, we find donations to waits of the towns
through which the traveller passed. In those of Sir John Howard, of
Henry VII., and of Henry VIII., there are payments to the waits of
London, Colchester, Dover, Canterbury, Dartford, Coventry,
Northampton, and others. Will. Kemp, in his celebrated Morris-dance
from London to Norwich, says that few cities have waits like those of
Norwich, and none better; and that, besides their excellency in wind
instruments, their rare cunning on the viol and violin, they had
admirable voices, every one of them being able to serve as a
\end{fixedpage}%548
\pagebreak


\begin{fixedpage}%549
\rectoheader

chorister in any cathedral church. One Richard Reede, a wait of
Cambridge, is mentioned by Mr. Sandys, as having received 20s. for
his attendance at a gentleman’s mansion during the Christmas of 1574.

Some of the tunes which the waits of different towns played, are
contained in \textit{The Dancing Master} of 1665 (among the violin tunes at
the end), and others in \textit{Apollo’s Banquet}, 1669.

The York Waits seem to have chosen a hornpipe tune, which was
printed in broadsides, with words by Mr. Durden. From these the
following are selected, as descriptive of the custom in that city,
about the end of the 17 th century:—

\settowidth{\versewidth}{And much more to hear a roopy fiddler call}
\begin{dcverse}
\vleftofline{“}In a winter’s morning,\\
Long before the dawning,\\
Ere the cock did crow,\\
Or stars their light withdraw,\\
Wak'd by a hornpipe pretty,\\
Play’d along York city,\\
By th’ help of o’ernight’s bottle,\\
Damon made this ditty, \dots\\
In a winter’s night,\\
By moon or lanthorn light,\\
Through hail, rain, frost, or snow.\\
Their rounds the music go;\\
Clad each in frieze or blanket\\
(For either heav’n be thanked),\\
Lin’d with wine a quart.\\
Or ale a double tankard.\\
Burglars scud away,\\
And bar guests dare not stay,\\
Of claret, snoring sots\\
Dream o’er their pipes and pots,\\
Till their brisk helpmates wake ’em,\\
Hoping music will make ’em\\
To find the pleasant Cliff,\\
That plays the Rigadoou.\\
\vin * \vin\vin\vin * \vin\vin\vin * \vin\vin\vin *\\
Candles, four in the pound,\\
Lead up the jolly Round,\\
Whilst cornet shrill i’ th’ middle\\
Marches, and merry fiddle,\\
Curtal with deep hum, hum,\\
Cries, we come, we come, come,\\
And theorbo loudly answers.\\
Thrum, thrum, thrum, thrum, thrum.\\
But, their fingers frost-nipt,\\
So many notes are o’erslipt,\\
That you'd take sometimes\\
The Waits for the Minster chimes:\\
Then, Sirs, to hear their music\\
Would make both me and you sick,\\
And much more to hear a roopy fiddler call\\
(With voice, as Moll would cry,\\
“Come, shrimps or cockles buy”),\\
“Past three, fair frosty morn,\\
Good morrow, my masters all.”
\end{dcverse}

The following was composed by Jeremiah Savile, and is on the last
page of Playford’s \textit{Musical Companion}, 1673, entitled \textsc{The Waits}:—
\end{fixedpage}%549
\pagebreak


\begin{fixedpage}%550
\versoheader

The following is called The Waits in \textit{The Dancing Master} of 1665,
and London Waits in Apolb's Banquet, 1663:—

\musictitle{LONDON WAITS.}

\musictitle{COLCHESTER WAITS,}
\begin{center}from \textit{Apollo's Banquet}, 1669.\end{center}
\end{fixedpage}%550
\pagebreak


\begin{fixedpage}%551
\rectoheader

\musictitle{Chester Waits,}
\begin{center}from Walsh’s \textit{Compleat Country Dancing Master}, iii. 36.\end{center}

Other tunes of the Waits might be added, as \textit{Worksop Waites}, from
Musical MSS., No. 610, Brit. Mus.; \textit{York Waits}, from the broadsides;
\textit{Bristol Waits}, from Apollo’s Banquet, \&c.; but the preceding four
specimens will probably be thought sufficient.

\musictitle{AN OLD WOMAN POOR AND BLIND.}

This is one of the ballads that were printed by W. Thackeray, in
the reign of Charles II., and subsequently by Playford and his
successors, in all the editions of \textit{Pills to purge Melancholy} , with
the tune.

There are several other ballads to the air in the \textit{Pills}, and
among them, one on \textit{The Cries of London}, beginning, “Come, buy my
greens and flowers fine;” and a second, \textit{The crafty Cracks of East
Smithfield}. The latter has the burden of \textit{I’m plunder’d of all my gold}.

The tune was introduced into several of the ballad-operas, such
as \textit{The Village Opera}, 1729, and \textit{The Fashionable Lady}, or \textit{Harlequin’s
Opera}, 1730; sometimes in minor, sometimes in a major key.

In the Bagford Collection of Ballads, are the following:—

“The toothless Bride,” \&c., “to the tune of \textit{An old woman poor
and blind}.”

“The Deptford Plumb Cake; or, The Four Merry Wives. Tune, \textit{An old
woman poor and blind}.”

In \textit{A Pill to purge State Melancholy}, v. ii., 1718, “Here’s a
health to great Eugene;” a song on Prince Eugene’s routing the Turks,
to the same air.

The following is “A Dialogue between Jack and his Mother. Tune of
\textit{Old woman poor and blind};” copies of which are in the Roxburghe and
other collections.
\end{fixedpage}%551
\pagebreak

\begin{fixedpage}%552
\versoheader

\settowidth{\versewidth}{My geese, my ducks, my cocks, my hens,}
\begin{dcverse}
\begin{altverse}
“Go to her, Jack, with all my heart,\\
And, when she is made thy spouse,\\
With half my goods I’ll freely part,—\\
My wethers and good milch cows;\\
My geese, my ducks, my cocks, my hens,\\
My waggons, my ploughs, my teams,\\
’Cause you declare in love you are,\\
And must have a wife, it seems.”
\end{altverse}

\begin{altverse}
So soon as this discourse was done.\\
Without any more dispute,\\
Jack to his chamber straight did run,\\
And put on his leathern suit;\\
His broad-brimm’d hat and ribbon red:\\
Now, when he was thus array’d,\\
Himself he view’d, and did conclude\\
That he was a brisk young blade.
\end{altverse}

\begin{altverse}
Then he away to Joan did ride,\\
And, when he came there, did cry,\\
“Sweet jewel, wilt thou be my bride,\\
My honey, my sweet piggesnie?”\\
But buxom Joan began to frown,\\
And said he was much too free;\\
She would not such a home-bred clown,\\
Her husband should ever he.
\end{altverse}

\begin{altverse}
“Why, what’s the matter?” Jack replied,\\
Without any more ado;\\
“I’d have you know, if hence I go,\\
I can have as good as you.\\
There’s Doll, the shepherd’s daughter dear,\\
And Katy of high degree,\\
Who has at least three mark a year,\\
They’re ready to die for me.”
\end{altverse}
\end{dcverse}

\end{fixedpage}%552
\pagebreak


\begin{fixedpage}%553
\rectoheader

\settowidth{\versewidth}{Joan stept and caught him by the sleeve,}
\begin{dcverse}
\begin{altverse}
With that he went to take his leave,\\
But, just as he turn’d aside,\\
Joan stept and caught him by the sleeve,\\
“I was but in jest,” she cried.\\
“What makes you be in so much haste,\\
If me thou art come to woo?\\
We must not part, thou hast my heart,\\
I’ll marry with none but you.”
\end{altverse}

\begin{altverse}
Then Joan, in merry humour, smil’d.\\
And taking him round the waist,\\
Said, “Prithee, John, be reconcil’d,\\
It was but a word in haste:\\
A kind and virtuous wife I’ll prove,\\
I’ll honour and love thee, too.”\\
“Why then,” quoth he, “I here agree\\
To marry with none but you.”
\end{altverse}
\end{dcverse}

\musictitle{GIVE EAR TO A FROLICSOME DITTY; or, THE RANT.}
\backskip{1}
A black-letter copy of this ballad, in the possession of Mr.
Payne Collier, is entitled, “The jolly Gentleman’s frolick; or, The
City Ramble: being an account of a young Gallant, who wager’d to pass
any of the Watches without giving them an answer; but, being stopp’d
by the Constable of Cripplegate, was sent to the Counter; afterwards
had before my Lord Mayor, and was clear’d by the intercession of my
Lord Mayor’s daughter: To a pleasant new tune.”

A second ballad, in the Bagford Collection, is named “The Ranting
Rambler; or, a young Gentleman’s frolick thro’ the City by night,”
\&c. “To a pleasant
new tune, called \textit{The Rant, Dal derra, rara}.”

These are different ballads on the same subject, and to the same
tune,—the first “printed for C. Bates, at the Sun and Bible in
Guiltspur St. the second by Brooksby, Deacon, Blare, and Back.

There are twenty stanzas in the former, of which a few are here
printed with the music. The second has been republished in “Songs of
the London Prentices and Trades,” by C. Mackay. 8vo, 1841. It
commences thus:—

\indentpattern{01012}
\settowidth{\versewidth}{I pray now attend to this ditty,}
\begin{dcverse}
\begin{patverse}
\vleftofline{“}I pray now attend to this ditty, \\
A merry and frolicsome song,\columnbreak \\
’Tis of a young spark in the City,\\
By night he went ranting along;\\
\textit{The Rant, dal derra, ra rara},” \&c. 
\end{patverse}
\end{dcverse}

A third ballad is in the
Roxburghe Collection, ii. 359, entitled “Mark Noble’s frolick,” \&c.
“To the tune of \textit{The New Rant}.”

The tune is in one of the editions of \textit{Apollo's Banquet}, entitled
\textit{The City Ramble}, and in many ballad operas. Among the last may be
cited \textit{The Beggars’ Opera}, \textit{Don Quixote in England}, \textit{The Sturdy Beggars},
\textit{The Wanton Jesuit}, and \textit{The Court Legacy}.

In \textit{The Beggars’ Opera}, it is called “Have you heard of a
frolicsome ditty?” and the words adapted are:—

\indentpattern{01017}
\settowidth{\versewidth}{But, whilst you thus teaze me together,}
\begin{dcverse}
\begin{patverse}
\vleftofline{“}How happy could I be with either, \\
Were t’other dear charmer away;\columnbreak\\
But, whilst you thus teaze me together,\\
To neither a word will I say,\\
But, tol de rol,” \&c.
\end{patverse}
\end{dcverse}

About fifty years later, we find it quoted in Ritson’s \textit{Bishoprick
Garland, or Durham Minstrel}, as the tune of a song of “The Hare-skin
commencing:—

\indentpattern{01019}
\settowidth{\versewidth}{Come hither, attend to my ditty,}
\begin{dcverse}
\begin{patverse}
\vleftofline{“}Come hither, attend to my ditty,\\
All you that delight in a gun,\columnbreak\\
And, if you’ll be silent a minute,\\
I’ll tell you a rare piece of fun.\\
Fal, lal,” \&c.
\end{patverse}
\end{dcverse}

And Mr. J. H. Dixon prints a ballad entitled “Saddle to rags,”
which is still sung in the North of England, to the same air. The
last will be found in \textit{Ballads and Songs of the Peasantry of England},
8vo, 1846. It is the old story of the
\end{fixedpage}%553
\pagebreak

\setlength{\fixedpagewidth}{361pt}
\begin{fixedpage}%554
\versoheader

farmer who, being overtaken by a highwayman while on his road to
pay his rent, pretends that his money is concealed in his saddle; the
highwayman demanding it, the farmer throws the saddle over the hedge,
and the thief scrambles after it, leaving his horse behind. The
opportunity of exchange is not lost upon the farmer, who rides away
with the highwayman’s horse, and all his recently-acquired booty.

The Rant is a dance of which I can give no account. It seems to
have been a rustic dance of the jig kind. In Mrs Centlivre’s Comedy,
\textit{The Platonick Lady}, 1707, where the dancing-master proposes to dance
a Courant with Mrs. Dowdy, she, supposing him to mean a \textit{Rant},
answers, “Ay, a Rant with all my heart;” but when he “leads her
about,” she says: “Hy, hy, do you call this dancing? ads heartlikins,
in my thoughts ’tis plain walking: I’ll shew you one of our country
dances; play me a Jig.” [Dances an awkward Jig.]

\settowidth{\versewidth}{Don’t tell me that—I was counted one of the best dancers
in the}
\begin{scverse}
\textit{Caper}. “O dear, madam, you’ll quite spoil your steps.”\\
\textit{Mrs. D}. “Don’t tell me that—I was counted one of the best dancers
in the parish, zo I was.”\\
\textit{Mrs. Peeper}. “Ay, round a Maypole.”
\end{scverse}

\settowidth{\versewidth}{“Stand! stand!” says the bellman,}
\begin{dcverse}
\begin{altverse}
“Stand! stand!” says the bellman,\\
“The constable now come before,\\
And if a just story you tell, man,\\
I’ll light you home to your own door. \\
This is a very late season,\\
Which surely no honest men keep,\\
And therefore it is but just reason\\
That you in the Compter should sleep.”\dots\\
The constable, on the next day, sir,\\
This comical matter to clear,\\
The gentleman hurried away, sir,\\
Before my Lord Mayor to appear.\\
“My Lord, give ear to my story,\\
While I the truth do relate,\\
The gentleman standing before you \\
Was seiz’d by me at Cripplegate.\columnbreak\\
I nothing could hear but his singing,\\
Wherefore in the Compter he lay,\\
And therefore this morning I bring him\\
To hear what your Lordship will say.’\dots\\
O then bespoke my Lord’s daughter,\\
And for him did thus intercede,\\
“Dear father, you’ll hear that, hereafter,\\
This is but a wager indeed.”\dots\\
“Well, daughter, I grant your petition,\\
The gentleman home may repair;\\
But yet, ’tis on this condition,\\
Of paying my officers there.”\dots\\
Thus seeing he might be released,\\
If he his fees did but pay,\\
He then was very well pleased,\\
And so he went singing away.\\
\vin\vin\vin\vin\vin\vin\vin\vin Dal derra, rara, \&c.
\end{altverse}
\end{dcverse}

\end{fixedpage}%554
\pagebreak

\begin{fixedpage}%555
\rectoheader

The following tune, which has much the same character as \textit{The
Rant}, is contained in the second and subsequent editions of \textit{The
Dancing-Master}, either as \textit{Winifred’s Knot}, or \textit{Open the door to three}.

\musictitle{CUPID’S TREPAN.}

This was a very popular ballad tune, and it acquired a variety of
names from the different ballads that were sung to it at different
periods. I have not, however, observed any of these to have been
issued by printers earlier than those of the reign of Charles II.
(Thackeray, Coles, \&c.), but there are many extant of later date.

Among the various names of the tune, may be cited, \textit{Cupid’s
Trappan}; \textit{Up the green Forest}; \textit{Bonny, bonny bird}; \textit{Brave Boys}; \textit{The
Twitcher}; \textit{A Damsel I’m told}; and \textit{I have left the world as the world
found me}.

The following ballads were sung to it:—

“Cupid’s Trappan, or, The Scorner scorn’d, or, The Willow turn’d
into carnation: described in The Ranting Resolution of a forsaken
maid. To \textit{a pleasant new tune now all in fashion}.” It commences:—

\indentpattern{01011}
\settowidth{\versewidth}{Up the green forest, and down the green forest,}
\begin{dcverse}
\begin{patverse}
\vleftofline{“}Once did I love a bonny brave bird,\\
And thought he had been all my own,\\
But he lov’d another far better than me,\\
And has taken his flight and is flown, \textit{Brave Boys.\\
And has taken his flight and is flown}.
\end{patverse}

\begin{patverse}
Up the green forest, and down the green forest,\\
Like one much distressed in mind,\\
I hoopt and I hoopt, and I flung up my  hood,\\
But my bonny bird I could not find, \textit{Brave Boys. \\
But my bonny bird I could not find}.”
\end{patverse}
\end{dcverse}

There are many copies of this ballad, and, among them, two will
be found in the Douce Collection, one of which is entitled, “Cupid’s
Trappan, or, \textit{Up the green Forest},” \&c.

There was quite a ballad-contest between the sexes, sung to this
air, for in answer to the above we have, firstly, “A young man put to
his shifts, or, The Ranting Young Man’s Resolution,” \&c., to the
tune of 
\textit{Cupid’s Trappan} (Rox., ii. 548, and Douce, 262,) commencing—

\settowidth{\versewidth}{And hold battle with Cupid's Trappan, Brave Boys,}
\begin{scverse}
\begin{patverse}
\vleftofline{“}Of late did I hear a young damsel complain,\\
And rail much against a young man,\\
His cause and his state I’ll now vindicate,\\
And hold battle with Cupid's Trappan, \textit{Brave Boys,\\
And hold battle with Cupidis Trappan}.
\end{patverse}
\end{scverse}

Then came “The Plowman’s art of wooing” (Rox., ii. 260):—

\settowidth{\versewidth}{The brisk young Ploughman doth believe,}
\begin{scverse}
\begin{altverse}
\vleftofline{“}The brisk young Ploughman doth believe,\\
If he were put to trial,\\
There’s not a maid in all the Shire \\
Could give him the denial.”
\end{altverse}
\end{scverse}

\end{fixedpage}%555
\pagebreak

\begin{fixedpage}%556
\versoheader

Tune of \textit{Cupid’s Trappan}. He commences thus:—

\settowidth{\versewidth}{And can, when I please, with abundance of ease,}
\begin{scverse}
\begin{altverse}
“I am a young man that do follow the plough,\\
But of late I have found out an art.\\
And can, when I please, with abundance of ease,\\
Deprive any maid of her heart, \textit{Brave Boys}, \&c.
\end{altverse}
\end{scverse}

In rejoinder to this, came “The Milkmaid’s Resolution” (Rox., ii.
347):—

\begin{scverse}
\begin{altverse}
\vleftofline{“}Let young men prate of what they please,\\
’Cause women have been kind,\\
They'll find no more such fools as these,\\
To please each apish mind.”
\end{altverse}
\end{scverse}

Tune, \textit{Cupid’s Trappan}; commencing:—

\begin{scverse}
\begin{altverse}
\vleftofline{“}Of late I did hear a young man domineer,\\
And vapour of what he could do,\\
But I think he knew how to manage the plough,\\
Far better than maidens to woo, \textit{Brave Boys}, \&c.
\end{altverse}
\end{scverse}

The tune is found in \textit{The Devil to pay}; in \textit{The Female Parson, or
The Beau in the Suds}; \textit{The Fashionable Lady, or Harlequin’s Opera};
\textit{Love and Revenge, or The Vintner outwitted}; in \textit{Flora}, and other
ballad operas. It was also printed on broadsides to a song called \textit{The
Twitcher}, sung by Mr. Pack at the Lincoln’s Inn Theatre, commencing,
“A Damsel I’m told.” In some copies, the tune consists of but eight
bars, as I printed it in National English Airs (p. 68 of the music),
in others of eleven; when of eight bars, the burden “Brave Boys,” and
the repetition of the last line, are omitted, but all the \textit{ballads}
require them.

After the ballad operas, came a variety of other songs to be sung
to it, of which I will only quote three stanzas of one which was in
great favour in the last century, and is still occasionally to be
heard. It is “Rural Sport,” printed in \textit{The Musical Companion, or
Lady’s Magazine}, 8vo., 1741, and in \textit{St. Cecilia, or The British
Songster}, 1782, commencing—

\settowidth{\versewidth}{The hounds are all out, and the morning does peep,}
\begin{scverse}
\indentpattern{01012}
\begin{patverse}
\vleftofline{“}The hounds are all out, and the morning does peep,\\
Why, how now? you sluggardly sot,\\
How can you, how can you lie snoring a-sleep,\\
While we all on horsebaek have got, \textit{Brave Boys,\\
While we all on horseback have got}?
\end{patverse}

\begin{altverse}
I cannot get up, for the overnight’s cup\\
So terribly lies in my head;\\
Besides, my wife cries. My dear, do not rise,\\
But cuddle me longer in bed, \textit{Dear Boy}, \&c.
\end{altverse}

\begin{altverse}
Come, on with your boots, and saddle your mare,\\
Nor tire us with longer delay,\\
The cry of the hounds and the sight of the hare\\
Will chase all dull vapours away, \textit{Brave Boy}s, \&c.
\end{altverse}
\end{scverse}

The following is one of the ballads that were printed by
Thackeray (Rox. iii. 100): “The patient Husband and the scolding
Wife: shewing how he doth complain of hard fortune he had to marry
such a cross-grain’d quean as she was, and he wishes all young men to
be advised to look before they leap. To the tune of \textit{Bonny, bonny
bird}.” The tune from \textit{Flora}, 8vo., 1729, air 13; the ballad
abbreviated.
\end{fixedpage}%556
\pagebreak

\begin{fixedpage}%557
\rectoheader

\indentpattern{01016}
\settowidth{\versewidth}{She’ll pout and she’ll lour, she’ll frown and look sour,}
\begin{dcverse}
\begin{patverse}
When as I was single, as some of you are,\\
I was loved, like other young men,\\
I liv’d at my ease, and did what I pleas’d,\\
And the world it went well with me then.\\
\textit{Brave Boys}, \&c. 
\end{patverse}

\begin{patverse}
Thus bravely I liv’d without any control,\\
And had silver, good store, laying by,\\
I could sing and be merry, drink claret and sherry,\\
Then who but “Sweet William was I?\\
\textit{Brave Boys}, \&c.
\end{patverse}

\begin{altverse}
When I went to church I was led by two maids,\\
And the music did play gallantly;\\
My wife she did dance, and her spirits advance,\\
Till she skipt up and down like a fly.
\end{altverse}

\begin{altverse}
I married in haste, but at leisure repent,\\
That I could be so fool’d by a wife:\\
She’ll pout and she’ll lour, she’ll frown and look sour,\\
Then dare I not stir for my life.
\end{altverse}

\begin{altverse}
Thrice happy is he that hath a good wife,\\
But far better off the young man\\
That settles himself to live single through life:\\
Would I were unmarried again,
\end{altverse}

\begin{altverse}
Now, honest young men, you have need to beware,\\
(For my part, my own ruin I’ve brought,)\\
Then of flattering damsels have a great care,\\
For wit’s never dear till ’tis bought.
\end{altverse}
\end{dcverse}

\indentpattern{01012}
\settowidth{\versewidth}{O the world would go well with me then, Brave Boys,}
\begin{scverse}\small
\begin{patverse}
So, bachelors all, now my leave I will take,\\
Take counsel, all honest young men,\\
Were I thut of this quean, (you know what I mean,)\\
O the world would go well with me then, \textit{Brave Boys, \\
O the world would go well with me then}.
\end{patverse}
\end{scverse}

\musictitle{WOMAN’S WORK IS NEVER DONE.}

This tune has a yariety of names, derived from different ballads
that were sung to it. Among these are, \textit{The Doubting Virgin}; or \textit{Shall
I, shall I}; \textit{O that I had never married}; \textit{Woman’s work is never done}:
\textit{The Soldier’s Departure}; and perhaps, \textit{The Bed-making}.
\end{fixedpage}%557
\pagebreak

\begin{fixedpage}%558
\versoheader

In the Douce Collection, p. 190, is “Shall I, shall I, no, no, no,” \&c.--Tune
of \textit{The Doubting Virgin};” commencing--
\settowidth{\versewidth}{Pretty Betty, now come to me,}
\begin{scverse}
\vleftofline{“}Pretty Betty, now come to me,\\
Thou hast set my heart on fire,”
\end{scverse}
and having the burden:
\begin{scverse}
\vleftofline{“}Never dally, shall I? shall I?\\
Still she answered, No, no, no.”
\end{scverse}

Whenever the tune of \textit{The Doubting Virgin} is referred to in the Douce Collection, either Mr. Douce, or some prior possessor, has pencilled against it, “O that I had never married,” as the other name.

“O that I had never married” is the first line of “Woman’s work is never done, or The Crown Garland of Princely Pastime and Mirth; the Woman has the worst of it, or her work is never done. To the tune of \textit{The Doubting Virgin}.” A copy of this is in Mr. Payne Collier’s Collection: it consists of seven stanzas, the first of which is here printed with the tune:—

In the Roxburghe Collection, i., 534, is a second ballad on the same subject:—

\begin{center}“\textsc{A woman’s work is never done.}\end{center}

\settowidth{\versewidth}{Which will much pleasure to them bring.}
\begin{dcverse}
Here is a song for maids to sing,\\
Both in the winter and the spring:\\
It is such a pretty-conceited thing,\\
Which will much pleasure to them bring.\\
Maids may sit still, go, or run,\\
But a woman’s work is never done.
\end{dcverse}

To a delicate Northern tune, \textit{A Woman's work is never done}, or \textit{The Bed’s-making}.” It commences:—

\settowidth{\versewidth}{Ever since the time she was made a wife.}
\begin{dcverse}
\vleftofline{“}As I was wand’ring on the way,\\
I heard a married woman say\\
That she had lived a sorry life\\
Ever since the time she was made a wife.\\
For why, quoth she, my labour’s hard,\\
And all my pleasures are debarr’d;\\
Both morning, evening, night and noon,\\
I’m sure a woman’s work is never done.”
\end{dcverse}
\end{fixedpage}%558
\pagebreak

\begin{fixedpage}%559
\rectoheader

After detailing all the troubles of married life, such as, rising
very early to sweep and cleanse the house, to light the fire, make
her husband’s breakfast, send the elder children to school, and tend
upon the younger, “till the eleven o’clock bell doth chime,

\settowidth{\versewidth}{Then I know ’tis near upon dinner time,”}
\begin{scverse}
Then I know ’tis near upon dinner time,”
\end{scverse}
and after, to find full
employment till night, and suffer disturbed rest from her youngest
child during the night, she gives the following advice to the
unmarried:—
\settowidth{\versewidth}{You know a woman’s work is never done.}
\begin{dcverse}
\vleftofline{“}All you merry girls that hear this ditty,\\
Both in the country and the city, \\
Take good notice of my lines, I pray, \\
And make the use of the time you may. \\
You see that maids live merrier lives\\
Than do the best of married wives;\\
And so, to end my song as I begun,\\
You know a woman’s work is never done.”
\end{dcverse}

The last consists of eleven stanzas, black letter,
“printed for John Andrews, at the White Lion, in Pye Corner,” and
“entred according to order.”

The tune is printed, under the name of \textit{Woman’s work is never
done}, in some of the ballad-operas, such as \textit{Momus turned Fabulist,
or Vulcan’s Wedding}, 1729.

In the Bagford Collection, 643, m. 10, p. 99, is “The Soldier’s
Departure;” to a pleasant new tune, or \textit{The Doubting Virgin}; and at page
98, one to the tune of \textit{The Soldier’s Departure}.

\musictitle{THE NORTHERN LASS.}

Oldys, in his MS. additions to Langbaine, says, “In a collection
of Poems, called \textit{Folly in Print, or a Book of 
Rhimes}, 8vo., 1667, p.
107, there is a ballad called \textit{The Northern Lass}. She was the Fair
Maid of Doncaster, named Betty Maddox; who, when an hundred horsemen
woo’d her, she conditioned, that he who could dance her down, she
would marry; but she wearied them all, and they left her a maid for
her pains.”

There are two songs on the Fair Maid of Doncaster, in \textit{Folly in
Print}; the first, entitled \textit{The Day Starre of the North}, is preceded
by the following lines:—
\settowidth{\versewidth}{From Maddocks, Princes of North Wales,}
\begin{dcverse}
\vleftofline{“}A maid so fair, so chaste and good, \\
And anciently of British blood, \\
From Maddocks, Princes of North Wales,\\
Doth now in Doncaster reside.\\
So fam’d of all. both far and wide.”
\end{dcverse}

It consists of sixteen stanzas of four lines, and commences
thus:—

\indentpattern{0001}
\settowidth{\versewidth}{The French, the Dutch, the Danish fleet,}
\begin{dcverse}
\begin{patverse}
\vleftofline{“}This wonder of the Northern starre, \\
Which shines so bright at Doncaster, \\
Doth threaten all mankind a warre, \\
Which nobody can deny. 
\end{patverse}

\begin{patverse}
The French, the Dutch, the Danish fleet,\\
If ever they should chance to meet,\\
Must all lye captives at her feet.\\
Which nobody can deny.”
\end{patverse}
\end{dcverse}


The above was evidently written to the tune of \textit{Green Sleeves}.

The second song is entitled, “\textit{The Northern Lass}; to the same
person: to \textit{a new tune}.” It begins thus: —

\settowidth{\versewidth}{Like snow doth melt, so soon as felt,}
\begin{dcverse}
\begin{altverse}
\vleftofline{“}There dwells a maid in Doncaster,\\
Is named Betty Maddocks,\\
No fallow deer, so plump and fair,\\
E’er fed in park or paddocks:\\
Her skin as sleek as Taffy’s leek,\\
And white as t’other end on’t,\\
Like snow doth melt, so soon as felt,\\
Could you but once descend on’t.”
\end{altverse}
\end{dcverse}

The “new tune” is found in Apollo’s Banquet, 1669 (within two
years of the
\end{fixedpage}%559
\pagebreak

\begin{fixedpage}%560
\versoheader

date of the book), under the name of \textit{The Northern Lass}.\textsuperscript{a} It is
there arranged for the violin, and seems to have been copied from
some pipe-version of the air. By the repetition of one phrase, the
second part of the tune is extended to sixteen bars (instead of
eight, which the words require), but if bars twelve to nineteen,
inclusive, were omitted, it would be of the proper ballad-length. All
later versions contain only eight bars in each part.

The following is the air, as it stands in \textit{Apollo's Banquet}.

The above is still popular, but in a very different form. Instead
of being a slow and plaintive air, it has been transformed into a
cheerful one.

\begin{dcfootnote}
\textsuperscript{a} One of Richard Brome’s Comedies, printed in 1632, 
was entitled \textit{The Northern Lasse}, but it does not contain
any song that could have been sung to this tune. The 
music to Brome's play was composed by Dr. John
Wilson, and three, or more, of the songs are extant in
Gamble’s MS., now in the possession of Dr. Rimbault.
\end{dcfootnote}
\end{fixedpage}%560
\pagebreak

\begin{fixedpage}%561
\rectoheader

In 1830, it was published under the title of “An old English air,
arranged as a Rondo by Samuel Wesley;” but between 1669 and 1830 it
appeared in \textit{Pills to purge Melancholy} , in \textit{The merry Musician}, and in
several ballad operas.

It is printed twice in \textit{The Merry Musician}; firstly to a song by
D’Urfey, and secondly to one from the ballad-opera of \textit{Momus turn'd
Fabulist}, commencing—

\settowidth{\versewidth}{At Athens in the market-place,}
\begin{scverse}
\begin{altverse}
\vleftofline{“}At Athens in the market-place,\\
A learned sage mounted a stage.”
\end{altverse}
\end{scverse}

In the ballad-operas it generally takes its name from D’Urfey’s
song, commencing—

\settowidth{\versewidth}{Great Lord Frog to Lady Mouse, \textit{Croakeldom he, croaheldmn ho};}
\begin{scverse}
\vleftofline{“}Great. Lord Frog to Lady Mouse, \textit{Croakeldom he, croaheldmn ho};\\
Dwelling near St. James's House, \textit{Cocky mi chari she};\\
Rode to make his court one day,\\
In the merry month of May.\\
When the sun shone bright and gay, \textit{Twiddle come, tweedle twe}.”
\end{scverse}

The versions in the ballad-operas—even the two in \textit{The Merry Musician}—differ 
considerably, but it may suffice here to give the tune as it
is now known, and in the form in which it was published by Samuel
Wesley.
\end{fixedpage}%561
\pagebreak

\begin{fixedpage}%562
\versoheader

\musictitle{NEWMARKET.}

This tune is contained in \textit{The Dancing Master} of 1675, and in
every subsequent edition.

A tune called \textit{Newmarket} is sometimes referred to in ballads, as
“The Country Farmer, or The buxom Virgin: to a new tune called
\textit{Newmarket}, or \textit{King James’ Jigg}” (Rox. ii. 77), but “To horse, brave
boys, to horse” seems intended, rather than this.

In the Travels of Cosmo, 3rd Grand Duke of Tuscany, throughout
England, in 1669, he says, “Newmarket has, in the present day, been
brought into repute by the King [Charles II], who frequents it on
account of the horse-races; having been before celebrated only for
the market for victuals, which was held there, and was a very
abundant one.” When Charles visited Newmarket, Tom D’Urfey used often
to sing to him: one of his songs, which is named after the town, and
begins “The golden age is come,” was printed in one of D’Urfey’s
collections, and in the \textit{Pills}, as having been “sung to the King
there.”
\end{fixedpage}%562
\pagebreak
\renewcommand\rectoheadertext{reign of charles ii. to william iii.}
\begin{fixedpage}%563
\rectoheader

\musictitle{TOBACCO IS AN INDIAN WEED.}

\settowidth{\versewidth}{Musicke, tobacco, sacke, and sleepe,}
\begin{scverse}
\vleftofline{“}Musicke, tobacco, sacke, and sleepe,\\
The tide of sorrow backward keepe.”\\
\vin\vin\vin\vin\vin\vin\vin\vin Marston's \textit{What you will}.
\end{scverse}

The verse that has been written in the praise and dispraise of
tobacco would, of itself, fill a volume; but, among the quantity, no
piece has been more enduringly popular than the song of \textit{Tobacco is an
Indian weed}. It has undergone a variety of changes (deteriorating
rather than improving it), and through these it may be traced, from
the reign of James I., down to the present day.

The earliest copy I have seen is in a manuscript volume of poetry
transcribed during James’s reign, and which was most kindly lent to
me by Mr. Payne Collier. It there bears the initials of G[eorge]
W[ither], now better known by his celebrated song of— 

\settowidth{\versewidth}{Shall I, wasting in despair,}
\begin{scverse}
\vleftofline{“}Shall I, wasting in despair,\\
Die because a woman’s fair?” 
\end{scverse}
than by any other of his numerous
productions. Wither is a very likely person to have written such a
song. A courtier poet would not have sung the praises of smoking—so
obnoxious to the King as to induce him to write a \textit{Counterblaste to
Tobacco}—but Wither despised the servility which might have tended to
his advancement at court. “He could not refrain,” says Wood, “from
shewing himself a Presbyterian satirist.” It was the publication of
his \textit{Abuses stript and whipt} which caused his committal to the
Marshalsea prison.

The following is Wither’s song:—

\indentpattern{0002}
\settowidth{\versewidth}{Even such—and gone with a small touch:}
\begin{dcverse}
\begin{patverse}
“Why should we so much despise \\
So good and wholesome an exercise\\
As, early and late, to meditate?\\
Thus think, and drink tobacco.
\end{patverse}

\begin{patverse}
The earthen pipe, so lily white, \\
Shews that thou art a mortal wight;\\
Even such—and gone with a small touch: \\
Thus think, and drink tobacco. 
\end{patverse}

\begin{patverse}
And when the smoke ascends on high, \\
Think on the worldly vanity \\
Of worldly stuff—’tis gone with a puff: \\
Thus think, and drink tobacco.
\end{patverse}

\begin{patverse}
And when the pipe is foul within, \\
Think how the soul’s defil’d with sin--\\
To purge with fire it doth require: \\
Thus think, and drink tobacco.
\end{patverse}

\begin{patverse}
Lastly, the ashes left behind\\
May daily shew, to move the mind,\\
That to ashes and dust return we must:\\
Thus think, and drink tobacco.”
\end{patverse}
\end{dcverse}

In the times of Elizabeth and James I., it was customary in
England to inhale and swallow the smoke, as Spaniards and Russians do
at the present time,— hence the expression, “to \textit{drink} tobacco.” It
was afterwards puffed out “through the nostrils, like funnels.” Ben
Jonson describes a young gallant endeavouring to acquire this
accomplishment, as “sitting in a chair, holding up his snout like a
sow under an apple-tree, while th’other open’d his nostrils with a
poking-stick, to give the smoke a more free delivery.”

About 1670, we find several copies of Wither’s song, but the
first stanza changed in all, besides other minor variations. In \textit{Merry
Drollery Complete}, 1670, it commences, “Tobacco, that is withered
quite.” On broadsides, bearing date the same year, and having the
tune at the top, the first line is, “The Indian weed withered quite.”
The last agrees, so far, with a copy quoted by Mr. Bertrand Payne,
from \textit{Two Broadsides against Tobacco}, 1672.
\end{fixedpage}%563
\pagebreak

\begin{fixedpage}%564
\versoheader

One stanza of these intermediate versions will suffice,\textsuperscript{a}—

\settowidth{\versewidth}{The Indian weed, withered quite,}
\begin{dcverse}
\vleftofline{“}The Indian weed, withered quite, \\
Green at morn, cut down at night, \\
Shews thy decay—all flesh is hay:\\
Thus think, then drink tobacco.”
\end{dcverse}

In 1699 it appeared in its present form, in the first volume of
\textit{Pills to purge Melancholy}, and so remained until 1719, when D’Urfey
became editor of that collection, and transferred it, with others, to
the third.

After the \textit{Pills}, it was printed with alterations, and the
addition of a very inferior second part, by the Rev. Ralph Erskine, a
minister of the Scotch Church, in his \textit{Gospel Sonnets}. This is the
“Smoking Spiritualized,” which is still in print among the
ballad-vendors of Seven Dials, and a copy of which is contained in
\textit{Songs and Ballads of the Peasantry of England}, by J. H. Dixon, or the
new edition by Robert Bell.

In the Rev. James Plumptre’s \textit{Collection of Songs} (8vo., 1805),
\textit{Tobacco is an Indian weed} was adapted to a more modern tune by Dr.
Hague; and about 1830, the late Samuel Wesley again re-set the words,
to music of his own composition. The following is the tune printed on
the broadsides, and in the \textit{Pills}:—

\indentpattern{0001}
\settowidth{\versewidth}{Is broke with a touch—man’s life is such:}
\begin{dcverse}
\begin{patverse}
The pipe, that is so lily white,\\
Wherein so many take delight,\\
Is broke with a touch—man’s life is such: \\
Think of this when you smoke tobacco.
\end{patverse}

\begin{patverse}
The pipe, that is so foul within.\\
Shews how man’s soul is stain’d with sin,\\
And then the fire it doth require:\\
Think of this when you smoke tobacco.
\end{patverse}

\begin{patverse}
The ashes that are left behind \\
Do serve to put us all in mind\\
That unto dust return we must:\\
Think of this when you smoke tobacco.
\end{patverse}

\begin{patverse}
The smoke, that does so high ascend, \\
Shews us man’s life must have an end, \\
The vapour’s gone—man’s life is done: \\
Think of this when you smoke tobacco.
\end{patverse}
\end{dcverse}

\musictitle{POPULAR ROUNDS AND CATCHES.}

Some account of the public meetings of music-clubs in the reign
of Charles the Second has already been given, but there were also
private meetings for the practice of part music, both vocal and
instrumental, which “were extremely fashionable with people of
opulence.” Hence, in \textit{The Citizen turn'd Gentleman},

\begin{dcfootnote}
\textsuperscript{a} The curious will find more on this subject in the 
articles of Dr. RimbauIt, Mr. Husk, Mr. Payne, and 
others, in \textit{Notes and Queries}, 2nd series, March 1st to
May 10th, 1856. Also in Dr. Rimbault’s \textit{Little Book of Songs and Ballads}, 8vo., 1851.
\end{dcfootnote}


\end{fixedpage}%564
\pagebreak

\begin{fixedpage}%565
\rectoheader

a comedy by Edward Ravenscroft, published in 1672, the citizen is
told that, in order to appear like a person of consequence, it is
necessary for him “to have a music-club once a week at his house.”
Glees, Rounds, and Catches were the. favorite vocal music, but the
words of \textit{some} of the Catches were more fitted for the tavern than for
good society. The readers of Macaulay’s \textit{History} will, recollect the
passage in which he speaks of Judge Jeffreys singing Catches in his
nightly revels with his boon companions; and it can scarcely he
considered a digression that one specimen should be offered, as
Rounds and Catches certainly come under the definition of \textit{Popular
Music of the olden time}.

Among those most in favour in the reign of Charles II. (as well
as long after), were Dr. Aldrich’s \textit{Hark! the bonny Christchurch
Bells}, and Mr. Fishburn’s \textit{Fie, nay, prithee, John}.\textsuperscript{a}

Dr. Aldrich’s was composed in the quiet retirement of Oxford,
about sixteen years before he became Dean of Christchurch, and was
first printed in Playford’s \textit{Musical Companion}, 1673.

Although particularly unsuitable for a ballad tune, from its
requiring a voice of great compass, and from its length, it even
became popular in that form. There is scarcely one of the great
collections which does not contain one or more ballads to be sung to
it. Of these I will cite but two: the first, a pæan of triumph on the
execution of Lord William Russell, which, though in the vilest taste,
may be thought to possess historical interest; and the second, on the
cries of London about the commencement of the last century, which may
deserve the notice of the local historian.

The first is “Russell’s Farewell,” and commences—

\settowidth{\versewidth}{Oh! the mighty innocence }
\begin{scverse}
\vleftofline{“}Oh! the mighty innocence \\
Of Russell, Bedford’s son.”
\end{scverse}
It is printed in the 120 \textit{Loyal Songs}, by N. T[hompson], 1684, and
again in the enlarged edition of 1686. It was even retained in the
edition of 1694, five years after the attainder had been reversed.\textsuperscript{b}

Copies,of the second ballad are contained in the Roxburghe (iii.
466) and Douce (p. 7) Collections. They commence—

\settowidth{\versewidth}{Hark! how the cries in every street}
\begin{scverse}
\vleftofline{“}Hark! how the cries in every street \\
Make lanes and alleys ring.”
\end{scverse}

The music of Dean Aldrich’s “Catch” is still in print, and
therefore the republication becomes unnecessary. It is also contained
in \textit{The Dancing Master}, and many ballad-operas.

Mr. Fishhurn’s “Fie, nay, prithee, John,” is to depict two
persons quarrelling in a tavern, at the top of their voices, and a
third endeavouring to soothe them, each voice taking the three parts
alternately, as in all Catches. It is found in \textit{The Delightful
Companion for the Recorder}, 1686; in \textit{Apollo’s Banquet}, 1690 and 1693;
and in \textit{The Dancing Master}. I have not seen any printed ballads to be
sung to it, but it was frequently introduced in the ballad operas,
with other words. The author seems to have been a student of the
Middle Temple.

\begin{dcfootnote}
\textsuperscript{a} Although these were commonly termed “Catches,” 
they are, strictly speaking, Rounds, as there is no catch 
in the words of either. The latter, however, requires to 
be acted, like a true Catch. 

\textsuperscript{b} Another of similar character, but not so offensively
triumphant, has been reprinted by Evans, iii., 203, 1810. 
It is entitled “Lord Russell's Farewell,” \&c. to the tune 
of \textit{Tender hearts of London city}. This is more like the 
genuine production of the ballad-monger.

\end{dcfootnote}
\end{fixedpage}%565
\pagebreak

\begin{fixedpage}%566
\versoheader

The following is printed in score, in compliance with modern
custom, but surely the old plan of placing the whole in consecutive
order, as it is to be sung, is to be preferred. The musician has the
advantage of seeing the harmonies by the score, but here the eye of
the singer must wander over two or three lines backwards or forwards,
at every two bars, to find the place.

Catches should be learnt by memory, and half acted when they are
sung. The manner of singing them has been explained at p. 482, but
the second singer commences here after the \textit{fourth} bar, and the third
singer after the \textit{eighth}.

\musictitle{THE MAN OF KENT.}

In “The Essex Champion; or, The famous History of Sit Billy of
Billericay and his squire Ricardo,” 1690, the following songs are
mentioned: “Three merry wives of Green-Goose Fair,” “Tom a Lincoln,”
and “The Man of Kent.”

The song of \textit{The Man of Kent} is by D’Urfey, and the tune by
Leveridge, composer of \textit{The Roast Beef of Old England}, \textit{Black-ey’d
Susan}, \&c.

D’Urfey wrote a second song to the same air for his play of
\textit{Masaniello}, and Leveridge, who was a base singer, sang it on the
stage.

The latter is in praise of fishing, commencing, “Of all the
world’s enjoyments,” and has the following burden:—

\settowidth{\versewidth}{Then who a jolly fisherman, a fisherman will be,}
\begin{scverse}
\vleftofline{“}Then who a jolly fisherman, a fisherman will be,\\
His throat must wet, just like his net,\\
To keep out cold at sea.”
\end{scverse}
\end{fixedpage}%566
\pagebreak

\begin{fixedpage}%567
\rectoheader

The tune is in \textit{The Quakers}, and other ballad operas; also in
\textit{Pills to purge Melancholy}, 2. 5. 1719, with the words. It is there
entitled, “A new song, inscribed to the brave Men of Kent, made in
honour of the nobility and gentry of that renowned and ancient
county.”

Some of the stanzas are still sung at social public meetings in
the county of Kent, and others have been added from time to time.

\settowidth{\versewidth}{He turn’d his arms, allow’d their terms,}
\begin{dcverse}
\begin{altverse}
The hardy stout freeholders. \\
That knew the tyrant near,\\
In girdles, and on shoulders, \\
A grove of oaks did bear:\\
Whom when he saw in battle draw,\\
And thought how he might need ’em;\\
He turn’d his arms, allow’d their terms, \\
Complete with noble freedom.
\end{altverse}
\end{dcverse}
\end{fixedpage}%567
\pagebreak

\begin{fixedpage}%568
\versoheader

\settowidth{\versewidth}{But joined with York, soon did the work,}
\begin{dcverse}
\begin{altverse}
And when by barons wrangling,\\
Hot faction did increase,\\
And vile intestine jangling\\
Had banished England’s peace,\\
The men of Kent to battle went,\\
They fear’d no wild confusion;\\
But joined with York, soon did the work, \\
And made a blest conclusion.
\end{altverse}

\begin{altverse}
At hunting, or the race too,\\
They sprightly vigour shew;\\
And at a female chase too,\\
None like a Kentish beau;\\
All blest with health; and as for wealth, \\
By fortune’s kind embraces,\\
A yeoman gray shall oft outweigh\\
A knight in other places.
\end{altverse}

\begin{altverse}
The generous, brave, and hearty,\\
All o’er the shire we find;\\
And for the low-church party,\\
They’re of the brightest kind:\\
For king and laws, they prop the cause,\\
Which high church has confounded;\\
They love with height the moderate right,\\
But hate the crop-ear’d roundhead.
\end{altverse}

\begin{altverse}
The promised land of blessing,\\
For our forefathers meant,\\
Is now in right possessing—\\
For Canaan sure was Kent:\\
The dome at Knole, by fame enroll’d.\\
The church at Canterbury,\\
The hops, the beer, the cherries here,\\
May fill a famous story.
\end{altverse}
\end{dcverse}

\musictitle{LILLIBURLERO.}

“The following rhymes,” says Dr. Percy, “slight and insignificant
as they may now seem, had once a more powerful effect than either the
Philippics of Demosthenes or Cicero; and contributed not a little
towards the great revolution in 1688. Let us hear a contemporary
writer:”

“A foolish ballad was made at that time, treating the Papists,
and chiefly the Irish, in a very ridiculous manner, which had a
burden, said to be Irish words, ‘Lero, lero, lilliburlero,’ that
made an impression on the [King’s] army, that cannot be imagined by
those that saw it not. The whole army, and at last the people, both
in city and country, were singing it perpetually. And, perhaps, never
had so slight a thing so great an effect.”—\textit{Burnet's History of his
own Times}.

“It was written, or at least re-published, on the Earl of
Tyrconnels going a second time to Ireland, in 1688\dots \textit{Lilliburlero}
and \textit{Bullen-a-lah} are said to have been the words of distinction used
among the Irish Papists in their massacre of Protestants, in 1641.”

In “A True Relation of the several Facts and Circumstances of the
intended Riot and Tumult on Queen Elizabeth’s Birth-day” (3rd edit.,
1712\textsuperscript{a}), the authorship of the words is ascribed to Lord Wharton—who
is said to have penned it in revenge for James II. having given the
appointment of Lord Deputy of Ireland to Tyrconnel. “A late Viceroy
[of Ireland], who has so often boasted himself upon his talent for
mischief, invention, lying, and for making a certain \textit{Lilliburlero}
song; with which, if you will believe himself, he sung a deluded
prince out of three kingdoms.”

Mr. Markland, in a note to Boswell’s \textit{Life of Johnson}, says that,
“according to Lord Dartmouth, there was a particular expression in
it, which the King remembered he had made use of to the Earl of
Dorset, from whence it was con-

\begin{dcfootnote}
\textsuperscript{a} “Queen Elizabeth’s Birthday was then kept as an Anti-Jacobite
Festival. A ballad for those occasions will be found in the Roxburghe
Coll., iii. 557, dated, in manuscript, 1711. It is entitled, “Queen
Elizabeth's Day; or, The Downfall of the Devil, the Pope, and the
Pretender. To the tune of \textit{Bonny Dundee}:” commencing,— 

\settowidth{\versewidth}{Let’s sing to the memory of glorious Queen Bess,}
\begin{scverse}
\vleftofline{“}Let’s sing to the memory of glorious Queen Bess, \\
Who long did the hearts of her subjects possess,\\
And whose mighty actions did to us secure \\
Those many great blessings that now do endure:\\
For she then did lay that solid foundation\\
On which our religion is fix’d, in this nation;\\
For Popery was put into utter disgrace,\\
And Protestantism set up in its place.”
\end{scverse}
Five stanzas of eight lines. It is also printed in \textit{A Pill to
purge State Melancholy}, 1716.
\end{dcfootnote}
\end{fixedpage}%568
\pagebreak

\setlength{\fixedpagewidth}{390pt}
\begin{fixedpage}%569
\rectoheader

cluded that \textit{he} was the author.” I think there are very sufficient
reasons for doubting this conclusion. In the first place, the Earl of
Dorset laid no claim to it, and it is scarcely to be believed that
the author of \textit{To all you ladies now at land} could have penned such
thorough doggrel. Although \textit{poetry} was not required for the purpose,
he would certainly have paid more attention to rhythm than is there
exhibited. Secondly, the ballad contains no expression that the King
would have used, which might not equally have been employed by any
other person. And thirdly, Lord Wharton being alive when the attacks
in \textit{The True Relation} and other pamphlets were made upon him, we may
infer that his opponents, who freely charge him with lying, would not
have omitted the falsehood of this claim, if there had been any
ground for disputing it.

“In \textit{The Examiner}, and in several pamphlets of 1712,” says Lord
Macaulay, “Wharton is mentioned as the author.”\textsuperscript{a}

The tune of \textit{Lilliburlero} was printed before the time at which the
words are supposed to have been written. “In February, 1687,
Tyrconnel began to rule his native country with the powers and
appointments of Lord Lieutenant, but with the humbler title of Lord
Deputy.” It was against such appointment that the ballad was
levelled. The tune will be found in the second edition of \textit{The
Delightful Companion, or Choice new Lessons for the Recorder or Flute}
(by Robert Carr), 1686, and in all probability in the first edition
of the same book.\textsuperscript{b} It

\begin{dcfootnote}
\textsuperscript{a} The writing of lampoons was a favorite amusement during the
reigos of the Stuarts, when every courtier was expected to handle a
pen in rhyme. Passing by minor personages, how many there are still
extant which were written by the Earl of Rochester and others upon
Charles II.! I quote a few odd stanzas, principally from memory:—

\indentpattern{0000121211232301012323232322232322232322}
\settowidth{\versewidth}{For my luxury and ease they brought me o’er the seas.}
\begin{scverse}
\begin{patverse}
\vleftofline{“}I am a senseless thing, \textit{with a hey},\\
Men call me a king, \textit{with a ho},\\
For my luxury and ease they brought me o’er the seas.\\
\textit{With a hey, tronney, nonney, nonney no}.”\dots\\
“Chaste, pious, prudent Charles the Second,\\
The miracle of thy restoration\\
May like to that of quails be reckon'd,\\
Rain’d on the Israelitish nation:\\
The wish’d-for blessing, from heaven sent,\\
Became their curse and punishment.”\dots\\
“Rowley too late will understand\\
What now he shuns to find,\\
That nothing's quiet in this land,\\
Except his careless mind.”\dots\\
\vleftofline{“}Beyond sea he began, where such riot he ran,\\
That ev'ry one there did leave him,\\
And now he's come o’er, ten times worse than hefore,\\
When none but we fools would receive him.”\dots\\
\vleftofline{“}His dogs would sit at council board,,\\
Like judges in their furs;\\
We question much which has more sense,\\
The master or the curs.”\dots\\
\vleftofline{“}His father’s foes he doth reward,\\
Preserving those that cut off’s head;\\
Old Cavaliers, the Crown’s best guard,\\
He lets them starve for want of bread:\\
Never was any king endow’d\\
With so much grace and gratitude.”\dots\\
\vleftofline{“}New upstarts, bastards, pimps, \&c.,\\
That, locust-like, devour the land,\\
By shutting up the Exchequer doors,\\
Whither our money is trepann’d,\\
Have render’d Charles’s restoration\\
But a small blessing to the nation.”\dots\\
\vleftofline{“}Then, Charles, beware thy brother York,\\
Who to thy government gives law;\\
If once you fall to the old sport,\\
Both must away again to Brĕdā,—\\
When, ’spite of all that would restore you,\\
Grown wise by wrongs, we shall abhor you.”
\end{patverse}
\end{scverse}

Even to Charles's face, things of this kind were occasionally
said, with a good motive, but such as the sterner nature of his
brother would not have suffered to he uttered with impunity. Pepys
records Tom Killegrew’s having told Charles, in the presence of
Cowley the poet, that matters were in a very ill state, but yet there
was one way to help all. “There is,” said he, “a good, honest, able
man, that I could name, that if your Majesty would employ, and
command to see things well executed, all things would soon be mended:
and \textit{this is one Charles Stuart, who now spends his time employing his
lips about the court, and hath no other employment}: he were the
fittest man in the world to perform it.” To this Pepys adds: “This is
most true, but the King do not profit by any of this, but lays all
aside, and remembers nothing, but to his pleasures again; which is a
sorrowful consideration.”—\textit{Diary}, Dec. 8, 1666.

\textsuperscript{b} I have never seen the first edition of \textit{The Delightful
Companion}, neither can I trace any other copy of the second than the
one in my own possession, which came from Gostling’s library. The
second edition is professedly “corrected,” but not “enlarged;” and,
as the work is engraved on plates (not set up in type, like \textit{The
Dancing Master}), the contents of the two editions are probably the
same. \textit{Lilliburlero} is found about the middle of the book, Sig. F.
\end{dcfootnote}

\end{fixedpage}%569
\pagebreak
\setlength{\fixedpagewidth}{360pt}
\begin{fixedpage}%570
\versoheader

appears without any name, and merely as a lesson. There are
“theatre tunes,” song tunes, airs, catches, and other \textit{compositions},
in the collection, but no air that I can trace to have been used for
ballads except this. It is the only copy I have met with that was
printed before the revolution.

In 1689, Lilliburlero was included in the second part of \textit{Music’s
Handmaid}, as “A new Irish Tune,” by “Mr. Purcell;” in 1690, in \textit{The
Dancing Master} and \textit{Apollo’s Banquet}; in 1691, Purcell used it as a
ground to the fifth air in his opera, The \textit{Gordian Knot unty’d}; and
afterwards it appeared in \textit{Pills to purge Melancholy} , and in many
ballad-operas, \&c.

James II. fled from England on the 23rd of December, 1688, and
the ballad printers took immediate advantage of the change of
affairs. A copy of \textit{Lilliburlero}, published in that very month, is
extant in Wood’s Collection of Broadsides. The printer professes to
give the “excellent new tune;” but, instead of it, used a block, or
type, with the air of \textit{Stingo, or Oil of Barley}. Nor is this a
solitary instance; for “The Irish Lasses Letter; or, Her earnest
Request to Teague, her dear Joy,” which was also to he sung to the
excellent new tune, and was printed in the same month, has the same
music. Sufficient time had not elapsed to prepare the type, or to cut
a new wood-block with the proper air.

In Nicholson and Burn’s \textit{Westmoreland and Cumberland} (4to., 1777,
i. 550), Henry Wharton (brother of the reputed author) is said to
have “assumed the habit of a player, and sung before the King [James
II.], in the playhouse, the famous party song of \textit{Lilliburlero}.” This
is quoted, from Nicholson and Burn, by Banks in his \textit{Extinct Baronage},
and from Banks, in Ellis’s \textit{Dover Correspondence}. It is a story that
should he received with caution; for it may be asked, what would have
become of the players who permitted, of the musicians who played, and
of Henry Wharton, who sang such a song in the presence of so
unforgiving a monarch as James?

As to the authorship of the tune, it is distinctly ascribed to
Henry Purcell in \textit{Music’s Handmaid}. The only question is whether he
took the first four bars' from a Somersetshire song, “In Taunton Dean
che were bore and bred,” the \textit{words} of which are evidently as old as
the civil wars, because, among the sights of London, one is St.
Paul’s Cathedral turned into a stable. On the other hand, \textit{that} air
may not be as old as \textit{Lilliburlero}, for I know of no copy earlier than
1729, and there is another under the same name (but said to be ‘‘a
new tune”), printed with the words of \textit{In Taunton Dean} in \textit{The Merry
Musician}, i. 305, 1716. Again, although \textit{There was an old fellow at
Waltham Cross} was sung to the tune of \textit{In Taunton Dean} (the one that
resembles \textit{Lilliburlero}) in the \textit{ballad-operas} of \textit{Flora}, and \textit{The Jovial
Crew}, there is no proof that the same music was sung in Brome’s
original play. On the contrary, there is other music to “There was an
old fellow,” in Hilton’s \textit{Catch that catch can}, and Playford’s \textit{Musical
Companion}.

The first \textit{collection} in which the words of \textit{Lilliburlero} appeared
was \textit{The Muses’ Farewell to Popery and Slavery}, 1689. It was
afterwards published in \textit{Poems on Affairs of State}, and some others.
Percy prints but the first part.

Shadwell seems to refer to the copy of the tune in \textit{Music’s
Handmaid} (where
\end{fixedpage}%570
\pagebreak

\begin{fixedpage}%571
\rectoheader

it is arranged for the virginals or harpsichord), when, in his
play of \textit{The Scowerers} (1691\textsuperscript{a}), Eugenia says: “And another music master
from the next town, to teach one to twinkle out \textit{Lilliburlero} upon an
old pair of virginals, that sound worse than a tinker’s kettle, that
he cries his work upon.” It is also alluded to by Vanbrugh, in his
comedy of \textit{Æsop}, and by Sterne, in \textit{Tristram Shandy}, where Uncle Toby
is said to be constantly whistling it.

The ballads that were sung to the tune are so numerous, that
space will only permit the mention of a very small proportion.

“Dublin’s Deliverance; or, The Surrender of Drogheda:”
commencing, “Protestant Boys, good tidings I bring.” This, singularly
enough, is omitted in Mr. Crofton Croker’s \textit{Historical Songs of
Ireland}. A copy is in the Pepys Collection, ii. 303.

“Undaunted London-derry; or, The Victorious Protestants’ constant
success against the proud French and Irish forces:” commencing,
“Protestant Boys, both valiant and stout.” Bagford Collection, 643,
m., 10, p. 116; and in the same volume, “The Courageous Soldiers of
the West,” and “The Reading Skirmish.”

The Roxburghe Collection contains “The Protestant Courage,”
“Courageous Betty of Chick Lane,” \&c., \&c.

In the later editions of The Garland of Goodwill, is “Teague and
Sawney; or, The unfortunate success of dear Joy’s devotion.” It is
about a windmill, which Sawney mistakes for St. Andrew’s Cross, and
Teague for St. Patrick’s. The latter kneels before it, and is caught
up by the wind setting the mill in motion.

The following are still commonly sung to the air:—“The Sussex
Whistling Song:” beginning, “There was an old farmer in Sussex did
dwell.” To this the company whistle in chorus, wherever the words
\textit{Lilliburlero} and \textit{Bullen a lah} would occur. It is printed in Dixon’s
\textit{Songs of the Peasantry of England}, p. 210. A second is—

\settowidth{\versewidth}{A very good song, and very well sung,}
\begin{scverse}
\vleftofline{“}A very good song, and very well sung,\\
Jolly companions every one.”
\end{scverse}

This is a common chorus after any song that has been approved by
the hearers. Lastly, the well-known nursery rhyme:—

\settowidth{\versewidth}{There was an old woman went up in a basket,}
\begin{scverse}
\begin{altverse}
\vleftofline{“}There was an old woman went up in a basket,\\
Seventeen times as high as the moon,\\
And where she was going I could not but ask it,\\
Because in her hand she carried a broom,—\\
‘Old woman, old woman, old woman,’ said I,\\
‘Where are you going? whither so high?’\\
‘To sweep the cobwebs off the sky,\\
And I shall be back again bye-and-bye.’”
\end{altverse}
\end{scverse}

The tune was, and still is, so popular, that two versions are
submitted to the reader,—the old way and the present. The following
is the old way, with the first part of the words of \textit{Lilliburlero}. The
second part of the words was added after the landing of King William.

\begin{dcfootnote}
\textsuperscript{a} Mr. m,            Dauney misdates this play “about 1670:” thereby 
making the song of Lilliburlero to have been written 
eighteen years before the revolution.—\textit{Ancient Melodies
of Scotland}, p. 19, 4to., 1838.
\end{dcfootnote}
\end{fixedpage}%571
\pagebreak

\begin{fixedpage}%572
\versoheader

\begin{center}
\textsc{“A new Irish song of Lil-li bur lero, to an excellent new tune.”}
\end{center}

\settowidth{\versewidth}{We’ll hang Magna Charta and demselves in a rope.}
\begin{dcverse}
Ho! by my shoul it is de Talbot,\\
And he will cut all de English throat;\\
Tho’, by my shoul, de English do praat,\\
De law’s on dare side, and Creish knows what.\\
But, if dispence do come from de Pope,\\
We’ll hang Magna Charta and demselves in a rope.\\
And de good Talbot is made a lord,\\
And he with brave lads is coming aboard,\\
Who all in France have tauken a sware,\\
Dat dey will have no Protestant heir.\\
O, but why does he stay behind?\\
Ho! by my shoul, ’tis a Protestant wind.\\
Now Tyrconnel is come ashore,\\
And we shall have commissions gillore;\\
And he dat will not go to mass\\
Shall turn out, and look like an ass.\\
Now, now de hereticks all go down,\\
By Creish and St. Patrick, de nation’s our own.
\end{dcverse}

The following four lines are added to the song in \textit{The Muses’
Farewell to Popery and Slavery}, but are printed separately in \textit{State
Poems}, and entitled “An Irish Prophecy:”—

\settowidth{\versewidth}{That Ireland should be rul’d by an ass and a dog.}
\begin{scverse}
\vleftofline{“}There was an old prophecy found in a bog,\\
That Ireland should be rul’d by an ass and a dog.\\
The prophecy’s true, and now come to pass,\\
For Talbot’s a dog, and Tyrconnel’s an ass.”
\end{scverse}

In some later copies, the credit of being the ass is transferred
to King James. 

The following version of the tune is more generally
adopted in the present day. Three stanzas of a song in praise of the
ale' of Nottingham, or Newcastle (for it is printed both ways), are
adapted to it. A copy in praise of Newcastle ale is in the Roxburghe
Collection, iii. 421; and one giving the credit to Nottingham is on a
broadside with music, now before me, The tune is copied from the
latter.
\end{fixedpage}%572
\pagebreak

\begin{fixedpage}%573
\rectoheader

\end{fixedpage}%573
\pagebreak

\begin{fixedpage}%574
\versoheader

\indentpattern{010101012}
\settowidth{\versewidth}{Ye Bishops and Curates, Priests, Deacons, and Vicars,}
\begin{dcverse}
\begin{patverse}
Ye Bishops and Curates, Priests, Deacons, and Vicars,\\
When once you have tasted, you’ll own it is true,\\
That Nottingham ale is the best of all liquors,\\
And none understand what is good like to you.\\
It dispels ev’ry vapour, saves pen, ink, and paper,\\
For, when you’ve a mind in the pulpit to rail,\\
’Twill open your throats, you may preach without notes,\\
When inspir’d with a bumper of Nottingham ale.\\
\textit{Chorus}. Nottingham ale, boys, \&c.
\end{patverse}

\begin{patverse}
Ye Doctors, who more execution have done,\\
With powder and potion, and bolus and pill,\\
Than hangman with halter, or soldier with gun,\\
Or miser with famine, or lawyer with quill;\\
To despatch us the quicker, you forbid us malt liquor,\\
Till our bodies consume, and our faces grow pale;\\
Let him mind you who pleases—what cures all disease is\\
A comforting glass of good Nottingham ale.
\textit{Chorus}. Nottingham ale, boys, \&c.
\end{patverse}
\end{dcverse}

\musictitle{JAMES THE SECOND’S MARCH.}

This march is contained in \textit{The Dancing Master} of 1690, and in
every subsequent edition. In the earlier of the above-named it is
entitled \textit{The Garter}, and in the later, \textit{King James’s March}, or \textit{The
Garter}.
\end{fixedpage}%574
\pagebreak

\begin{fixedpage}%575
\rectoheader

\musictitle{IN JANUARY LAST.}

This is a song in D’Urfey’s play, \textit{The Fond Husband, or The
Plotting Sisters}, which was acted in 1676.

The words and music are to he found in Playford’s \textit{Choice Ayres},
ii. 46, 1679, and in vol. i. of all editions of \textit{Pills to purge
Melancholy}. The tune is in \textit{Apollo's Banquet}, 1690, and probably in
some of the earlier editions which I have not seen.

The words are in the Roxhurghe Collection, ii. 414, entitled “The
Scotch Wedding, or A short and pretty way of wooing: To a new
Northern tune, much us’d at the theatres.” Printed for P. Brooksby.
In the same Collection, iii. 116, is “The new-married Scotch Couple,
or The Second Part of the Scotch Wedding,” \&c., “To a new Northern
tune, or \textit{In January last}.” Printed by Thackeray, Passinger, and
Whitwood.

Many other ballads were sung to it, of which one or two have
already been quoted. I will only add to the list, “Northern Nanny, or
The Loving Lasses Lamentation,” \&c., a copy of which is in the Douce
Collection (164). It commences—

\settowidth{\versewidth}{‘Alas!’ quoth she, ‘why was I born}
\begin{dcverse}
\vleftofline{“}On Easter Monday last, \\
When lads and lasses play. \\
As o’er the green I past \\
Near noon-time of the day, \\
I heard a pensive maiden mourn.\\
Tears trickling down amain;\\
‘Alas!’ quoth she, ‘why was I born\\
To live in mickle pain? ’”
\end{dcverse}



This identifies \textit{In January last} as \textit{one} of the tunes called
\textit{Northern Nanny}.

Allan Ramsay included “In January last” in vol. ii. of \textit{The
Tea-Table Miscellany}, as “a song to be sung to its own tune.” He
altered some of the lines, and improved the spelling of the
Anglo-Scottish words, but made no addition.\textsuperscript{a} Ramsay’s version was
followed by Thomson, in his \textit{Orpheus Caledonius} (ii. 42, 1733), but he
changed the name to \textit{The glancing of her Apron}; taking that title from
the seventh line of the song. In one of the Leyden MSS. (about 1700),
the tune bears the name of \textit{The bonny brow} from the eighth line of the
same.

Both the words and music became extremely popular in Scotland.
Even so late as 1797, they were reprinted in Johnson’s \textit{Scot's Musical
Museum}; but on that occasion Burns “brushed up the three first stanzas
of Ramsay’s version, and omitted the remainder for an obvious
reason.”

The increasing refinement of manners was causing a gradual change
in the style of popular poetry, and the rejection of many of the
older pieces, so that when, in 1815, Mr. Alexander Campbell was on a
tour on the borders of Scotland for the purpose of collecting Scotch
airs, he received a traditional version of the air from Mr. Thomas
Pringle, with a verse of other words, which Mr. Pringle had heard his
mother sing to it. This was the first stanza of the now-celebrated
song of \textit{Jock o' Hazledean}, which Sir Walter Scott so admirably
completed. It was first printed in \textit{Albyn's Anthology} (vol. i., 1816,
fol.), with the air arranged by Campbell. Campbell mistook it for an
old Border melody.

\begin{scfootnote}
\textsuperscript{a} Mr. Stenhouse says that Allan Ramsay reprinted it as “an old
song \textit{with additions};” which is a mistake.
\end{scfootnote}
\end{fixedpage}%575
\pagebreak

\begin{fixedpage}%576
\versoheader

The following is the old tune, with the first stanza of the old
words:—

\musictitle{DULCE DOMUM.}

“The tradition connected with this song is, that a Wykehamist,
being for some misdemeanor confined to his rooms at Winchester,
during a vacation, and thus disappointed in his expectation of
returning home, he composed and set this song to music, while
languishing for domestic endearments, and that incessantly playing it
to relieve his heart-ache, he pined away and died.”

\indentpattern{01011010001}
\settowidth{\versewidth}{And suits to every sigh the sweetly-warbling strings.}
\begin{scverse}
\begin{patverse}
“And see in durance the fast-fading boy,\\
Midst Wykeham’s walls his dulcet sorrows heave;\\
Fled are his fairy dreams of homely joy.\\
Ah! frowns too chilling, that his soul bereave \\
Of all that frolic fancy long’d to weave\\
In his paternal woods! His hands he wrings\\
In anguish! Yet some balm his sorrows leave\\
To soothe his fainting spirits, as he sings,\\
And suits to every sigh the sweetly-warbling strings.\\
O he had notch’d, unweeting of distress,\\
The hours of school-boy toil! Nor irksome flew
\end{patverse}
\end{scverse}
\end{fixedpage}%576
\pagebreak

\setlength{\fixedpagewidth}{380pt}
\begin{fixedpage}%577
\rectoheader

\indentpattern{01101006}
\settowidth{\versewidth}{Droops his sick heart. And “Ah, dear fields, (he cries)}
\begin{scverse}\small
\begin{patverse}
The moments; for, each morn, his score was less!—\\
Visions of vacant home yet brighter grew;\\
When lo! stern fate obscur'd the blissful view:\\
Droops his sick heart. And “Ah, dear fields, (he cries)\\
“Ye bloom no more! dear native fields, adieu!”\\
“Home, charming home!” still plaintive Echo sighs,\\
And to his parting breath the dulcet murmur dies.\\
\textit{Polwheels Influence of Local Attachment}, p. 57.
\end{patverse}
\end{scverse}

Dr. Milner, in his \textit{History of Winchester}, says, “We shall now
conclude this account of the college, with inserting the famous song
of Dulce Domum, which is publicly sung by the scholars and
choristers, aided by a band of music, previously to the summer
vacation. The existence of this song can only be traced up the
distance of about a century [from the time at which he wrote], yet
the real author of it, and the occasion of its composition, are
already clouded with fables.”

It has been justly remarked by J. P. Malcolm, in the \textit{Gentleman's
Magazine} for 1796, that the sentiments of the words are rather those
of a scholar looking forward with an early expectation of enjoying
the delights of the home joys he describes, than of a boy who died of
sorrow, chained to a post.

Dr. Hayes, and other authorities, attribute the composition of
the music to John Reading, who was organist of Winchester College and
of Winchester Cathedral,—of the College from 1681 to 1689,\textsuperscript{a} and
probably till 1695, in which year he is said to have died. Reading
composed the music to the three Latin Graces, which are sung at the
annual college elections,—the Ante cibum, Post cibum, and the Oratio,
“Agimus tibi gratias, omnipotens Deus, pro fundatore nostro, Gulielmo
de Wykeham,” \&c.

The printed copies of \textit{Dulce Domum} also ascribe the music to
“Johannes Reading,” and “the poetry” to “Turner.”\textsuperscript{b} Such of these as
I have seen are of comparatively late date—perhaps no one older than
the latter part of the last century—but they were most probably
reprints from earlier editions.

\textit{Dulce Domum} is still sung at Winchester on the eve of the
break-up-day. The collegians sing it first in the school-room, and
have a band to play it. Afterwards they repeat it at intervals
throughout the evening, before the assembled visitors, in the College
mead or play-ground; and continue to sing

\begin{dcfootnote}
\textsuperscript{a} The rolls of Winchester College give the date of Reading’s
appointment. These rolls are lists of the officers, prepared yearly.
Those between 1689 and 1697 are missing, but in the latter year
Bishop was organist. This John Reading has sometimes been confounded
with a later writer of the same name, who was organist of St. John’s,
Hackney. Both composed anthems, but the \textit{Biographical Dictionary of
Musicians} is incorrect, as to date, when it states that the latter
“published a collection of anthems of his own composition towards the
end of the seventeenth century.” The second John Reading’s “\textit{first}
essays” were \textit{A Book of New Songs}, which must have been printed after
1708, because he describes himself, on the title page, as having been
“educated in the Chapel Royal, under \textit{the late} Dr. John Blow,” and Dr.
Blow died in that year. Reading composed the well-known “Adeste,
fideles,” commonly called “The Portuguese Hymn.” The accident by
which it acquired the latter name is thus related in Novello’s \textit{Home
Music}: The Duke of Leeds, who was a director of the Ancient Concerts
about 1785, heard the hymn performed for the
first time at the Portuguese Ambassador’s chapel, in South
Street, Park Lane, and he, supposing it to be peculiar to the service
in Portugal, introduced it at the Ancient Concerts, giving it that
title.

\textsuperscript{b} No scholar of the name of Turner is to be found on the
Registers of the College in Reading’s time, and but one who had been
a scholar was his cotemporary. This was Francis Turner, admitted in
1650, superannuated in 1655; who then proceeded to St. John's
College, Cambridge, became Prebendary of St. Paul’s and Bishop of
Ely,—one of the seven bishops who were brought to trial before the
Court of King’s Bench by James II. The Registers contain the names of
all the scholars from the very first. Before Francis Turner there
were Edward Turner, in 1477, John Turner, in 1530, Edward Turner, in
1551, and Edward Turner, in 1620; also, two Turn\textit{a}rs, in 1522 and
1529. The remoteness of these dates (the nearest being sixty years
before Reading’s appointment) leads to the inference that Francis
Turner, afterwards Bishop of Ely, was the author.
\end{dcfootnote}
\end{fixedpage}%577
\pagebreak

\setlength{\fixedpagewidth}{360pt}
\begin{fixedpage}%578
\rectoheader

even after darkness has dispersed their guests, and without the
introduction of any other vocal music.

It was formerly sung round “an old tree that stood in the ground
recently used as a wharf, but now converted into a garden.” \textit{Notes and
Queries}, Sep. 9, 1854.

The following translation is by Dr. Charles Wordsworth, present
Bishop of St. Andrew’s, and formerly second master of the College.
The chorus from ' another copy:—
\end{fixedpage}%578
\pagebreak

\begin{fixedpage}%579
\rectoheader

\indentpattern{01221}
\settowidth{\versewidth}{Hearts for home and freedom yearning.}
\begin{dcverse}
\begin{patverse}
See, the wish’d-for day approaches, \\
Day with joys attended:\\
School’s heavy course is run, \\
Safely the goal is won,\\
Happy goal, where toils are ended.
\end{patverse}

\begin{patverse}
Quit, my weary muse, your labours,\\
Quit your books and learning;\\
Banish all cares away,\\
Welcome the holiday,\\
Hearts for home and freedom yearning.
\end{patverse}

\begin{patverse}
Smiles the season, smile the meadows, \\
Let us, too, be smiling;\\
Now the sweet guest is come, \\
Philomel, to her home, \\
Homeward, too, our steps beguiling.
\end{patverse}

\begin{patverse}
Roger, ho! 'tis time for starting, \\
Haste with horse and traces;\\
Seek we the scene of bliss,\\
Where a fond mother’s kiss \\
Longing waits her boy’s embraces.
\end{patverse}

\begin{patverse}
Sing once more, the gate surrounding, \\
Loud the joyous measure.\\
Lo! the bright morning star,\\
Slow rising from afar,\\
Still retards our dawn of pleasure.
\end{patverse}

\begin{patverse}
Appropinquat, ecce! felix! \\
Hora gaudiorum:\\
Post grave tedium, \\
Advenit omnium \\
Meta petita laborum.
\end{patverse}

\begin{patverse}
Musa, libros mitte, fessa, \\
Mitte pensa dura: \\
Mitte negotium,\\
Jam datur otium; \\
Me mea mittito cura!
\end{patverse}

\begin{patverse}
Ridet annus, prata rident; \\
Nosque rideamus.\\
Jam repetit domum, \\
Daulias advena:\\
Nosque domum repetamus.
\end{patverse}

\begin{patverse}
Heus: Rogere! fer caballos; \\
Eja, nunc eamus;\\
Limen amabile.\\
Matris et oscula,\\
Suaviter et repetamus!
\end{patverse}

\begin{patverse}
Concinamus ad Penates;\\
Vox et audiatur:\\
Phosphore! quid jubar \\
Segnius emicans,\\
Gaudia nostra moratur?
\end{patverse}
\end{dcverse}

\musictitle{HARVEST-HOME.}

There are still many harvest-home and harvest-supper songs
extant; but formerly the \textit{labours} of the field were accompanied with
song, as well as the after rejoicings. “How heartily,” says Dr. John
Case, “doth the poorest swain both please himself, and flatter his
beast, with whistling and singings. Alas! what pleasure could they
take at the whip and plough tail, in so often and incessant labours;
such bitter weather-beatings; sometimes benumbed with cold; otherwise
melted with heat; unless they quieted, and even brought asleep their
painfulness, with this their homely, yet comfortable and
self-pleasing exercise?\dots Those with a light heart make their
plough go lighter, and while they use the solace of their natural
instruments, both quicken themselves and encourage forward their
over-laboured horse.” (“\textit{The Praise of Music}, Printed at Oxenford by
Joseph Barnes, Printer to the University,” 1586.) Mr. Surtees, in his
\textit{History of Durham}, mentions having read a report of a trial “in which
a Mr. Spearman made a forcible entry into a field of Mrs. Wright’s at
Birtley, and mowed and carried away the crop whilst his piper played
from the top of the loaded wains,” for the purpose of making the men
work the faster, so as to get away before they could be interrupted.
If harvest men were introduced on the stage in the early drama, it
was almost invariably for the purpose of making them sing or dance.
In Peele’s \textit{Old Wives' Tale}, 1571, the harvest men appear at this
speech,—
\end{fixedpage}%579
\pagebreak

\begin{fixedpage}%580
\versoheader

“O, these are the harvest men; ten to one they sing a song of
mowing.” However, they sing one of sowing,—“Lo, here we come a sowing,
a sowing;” and, in another part of the play,—

\settowidth{\versewidth}{Lo, here we come a reaping, a reaping,}
\begin{dcverse}
\begin{altverse}
\vleftofline{“}Lo, here we come a reaping, a reaping, \\
To reap our harvest fruit! \\
And thus we pass the year so long,\\
\textit{And never be we mute}.
\end{altverse}
\end{dcverse}

In Nashe’s \textit{Summer's last Will and Testament} (printed in 1600),
Harvest enters “with a scythe on his neck, and all his reapers with
sickles, and a great black bowl with a posset in it, borne before
him.” They come in singing, and this is their song:—

\settowidth{\versewidth}{Hey, derry, derry'; with a poup and a leary}
\begin{dcverse}
\begin{altverse}
“Merry, merry, merry; cheary, cheary, cheary;\\
Trowl the black bowl to me;\\
Hey, derry, derry'; with a poup and a leary;\\
I’ll trowl it again to thee: \\
\textit{Hooky hooky, we have shorn, \\
And we have bound,\\
And we have brought Harvest \\
Home to town}.”
\end{altverse}
\end{dcverse}

The editor of Dodsley’s \textit{Old Plays} (ix. 41, 1825), remarks that
the above was probably “a harvest-home song, usually sung by reapers
in the country and that “the chorus or burden, ‘Hooky, hooky,’ \&c.,
is still heard in some parts of the kingdom, with this variation:—

\settowidth{\versewidth}{And we have brought the harvest home.}
\begin{dcverse}
\begin{altverse}
‘Hooky, hooky, we have shorn, \\
And bound what we did reap; \\
And we have brought the harvest home.\\
To make bread good and cheap.’ ”
\end{altverse}
\end{dcverse}

The ceremony of an English harvest-home is thus described by
Hentzner, who travelled through England (as well as through Germany,
France, and Italy) towards the close of the sixteenth century, and
published his \textit{Itinerarium} in 1598: —“As we were returning to our inn”
(at Windsor), “we happened to meet some country people celebrating
their harvest-home: Their last load of corn they crown with flowers,
having besides an image richly dressed, by which they would perhaps
signify Ceres; this they keep moving about, while men and women, men
and maid servants, riding through the streets in the cart, shout as
loud as they can till they arrive at the barn.” Dr. Moresin, another
foreigner, who published, in the reign of James I., an elaborate work
on the “Origin and Increase of Depravity in Religion,” relates that
he saw “in England the country people bringing home, in a cart from
the harvest field, a figure made of corn, round which men and women
were promiscuously singing, preceded by a piper and a drum.”
Sometimes, instead of a figure made of corn, a young girl was dressed
as the Harvest Queen, being crowned with flowers, a sheaf of corn
placed under her arm, and a sickle in her hand, and so drawn along..
Another crown of flowers was placed upon the head of the most expert
reaper.

The harvest festivities are described by Dr. Drake, in his
\textit{Shakspeare and his Times}, as “a scene not only remarkable for
merriment and hospitality, but for, a temporary suspension of all
inequality between master and man.” The whole family sat down at the
same table, and conversed, danced, and sang together during the
entire night, without difference or distinction of any kind; and in
many places, indeed, this freedom of manner subsisted during the
whole period of getting in the harvest. Thus Tusser,\textsuperscript{a} recommending the
social equality of the harvest-tide, exclaims,—
\end{fixedpage}%580
\pagebreak

\begin{fixedpage}%581
\rectoheader

\settowidth{\versewidth}{And fill out the black bowl, so blithe to their song,}
\begin{scverse}
\vleftofline{“}In harvest time, harvest folte, servants and all, \\
Should make, all together, good cheer in the hall, \\
And fill out the black bowl, so blithe to their song, \\
And let them be merry, all harvest-time long.”
\end{scverse}

In the reign of Charles I., we have the following admirable
description by
Herrick, of “The Hock-Cart,\textsuperscript{b} or Harvest-Home” (\textit{Hesperides}, i. 139 Pickering’s edit.):—

\settowidth{\versewidth}{Those with a shout, and these with laughter.}
\begin{dcverse}
“Come, sons of summer, by whose toil, \\
We are the lords of wine and oil,—\\
By whose tough labours and rough hands, \\
We rip up first, then reap our lands. \\
Crown’d with the ears of corn, now come, \\
And, to the pipe, sing Harvest-home. \\
Come forth, my lord, and see the cart \\
Drest up with all the country art.\\
See, here a Maukin,\textsuperscript{c} there a sheet,\\
As spotless pure as it is sweet;\\
The horses, mares, and frisking fillies, \\
Clad all in linen white as lilies.\\
The harvest swains and wenches bound \\
For joy, to see the hock-cart crown’d. \\
About the cart hear how the rout \\
Of rural younglings raise the shout. \\
Pressing before, some coming after,\\
Those with a shout, and these with laughter. \\
Some bless the cart, some kiss the sheaves, \\
Some prank them up with oaken leaves, \\
Some cross the fill-horse, some with great\\
Devotion stroke the home-borne wheat; \\
While other rustics, less attent \\
To prayers than to merriment,\\
Run after with their breeches rent.\\
Well, on, brave boys, to your lord’s hearth,\\
Glitt’ring with fire, where, for your mirth,\\
Ye shall see first the large and chief \\
Foundation of your feast, fat beef;\\
With upper stories, mutton, veal,\\
And bacon, which makes full the meal; \\
With sev’ral dishes standing by,\\
As, here a custard, there a pie,\\
And here all tempting frumenty.\\
And for to make the merry cheer,\\
If smirking wine be wanting here, \\
There’s that which drowns all care, stout beer;\\
Which freely drink to your lord's health, \\
Then to the plough, the commonwealth, \\
Next to your flails, your vanes, your vats, \\
Then to the maids with wheaten hats;\\
To the rough sickle, and crookt scythe, \\
Drink, frolic, boys, till all be blythe.\\
Feed and grow fat, and as ye eat,\\
Be mindful that the lab’ring neat,\\
As you, may have their full of meat.\\
And know, besides—ye must revoke \\
The patient ox unto the yoke,\\
And all go back unto the plough \\
And harrow, though they’re bang’d up now.\\
And that this pleasure is like rain,\\
Not sent ye for to drown your pain,\\
But for to make it spring again.”
\end{dcverse}


So Stevenson, in his \textit{Twelve Moneths}, 1661, says, “In August the
furmety pot welcomes home the Harvest Cart, and the garland of
flowers crowns the captain of the Reapers: the battle of the field is
now stoutly fought. The pipe and the tabor are now busily set a-work;
and the lad and the lass will have no lead on their heels. O ’tis the
merry time wherein honest neighbours make good cheer, and God is
glorified in his blessings on the earth.”

\begin{dcfootnote}
\textsuperscript{a} \textit{Tusser redivivus}, p 104. In the first edition of Tusser, 1557
(reprinted by Sir Egerton Brydges), this stanza is as follows:—

\settowidth{\versewidth}{Then welcome thy harvest folke, serveauntes and all;}
\begin{fnverse}
\vleftofline{“}Then welcome thy harvest folke, serveauntes and all; \\
With mirth and good chere, let them furnish the hall. \\
The Harvest-Lorde nightly, must give thee a song:\\
Fill him then the blacke boll, or els he hath wrong.” 
\end{fnverse}

\textsuperscript{b} “\textit{Hock-cart},—By this word is meant the \textit{high} or
\textit{rejoicing-cart}, and it was applied to the last load of corn, as
typical of the close of harvest., Thus \textit{Hock-tide} is derived from the
Saxon \textit{Hoah-tid} or high tide, and is expressive of the height of
festivity.” (Dr. Drake.) Horkey, Hockey, and Hooky, seem all to he
derived from this root.

\textsuperscript{c} \textit{Maukin},—a country maid.
\end{dcfootnote}
\end{fixedpage}%581
\pagebreak

\begin{fixedpage}%582
\versoheader

The original of the following harvest-song is to be found in the
fifth act of Dryden’s opera of \textit{King Arthur}. It there forms part of
the incantations of Merlin, and is sung by Comus and three peasants
to Arthur and Emmeline.

\settowidth{\versewidth}{We’ve cheated the parson, we’ll cheat him again,}
\begin{scverse}
\vleftofline{\textit{Comus.} “}Your hay it is mow’d, and your corn is reap’d;\\
Your barns will be full, and your hovels heap’d \\
Come, boys, come; come, boys, come;\\
And merrily roar out Harvest Home.\\
\vleftofline{\textit{Chorus. }}Come, boys, come; come, boys, come;\\
And merrily roar out Harvest Home.\\
\vleftofline{\textit{1st Man. }}We’ve cheated the parson, we’ll cheat him again,\\
For why should a blockhead have one in ten?\\
One in ten, one in ten,\\
For why should a blockhead have one in ten?\\
\vleftofline{\textit{Chorus. }}One in ten, one in ten,\\
For why should a blockhead have one in ten?\\
\vleftofline{\textit{2nd Man. }}For prating so long like a book-learn’d sot,\\
’Till pudding and dumpling do burn to th’ pot?\\
Burn to pot; burn to pot;\\
’Till pudding and dumpling burn to pot?\\
\vleftofline{\textit{Chorus. }}Burn to pot; burn to pot;\\
’Till pudding and dumpling burn to pot.\\
\vleftofline{\textit{3rd Man. }}We’ll toss off our ale till we cannot stand,\\
And hoigh for the honour of Old England,\\
Old England, Old England,\\
And hoigh for the honour of Old England.\\
\vleftofline{\textit{Chorus. }}Old England, Old England,\\
And hoigh for the honour of Old England.”
\end{scverse}

It appears that the actors were to dance and sing at the same
time, for at the end is a stage direction: “The dance varied into a
round country dance.”

Dryden tells us, in his dedication of \textit{King Arthur}, that the
writing of the “poem” was “the last piece of service he had the
honour to do for King Charles II,” who died before its performance on
the stage. The music was composed by Purcell, but this song is not
included in any extant manuscript, even in those which contain the
other music of the incantation scene. The Hon. Roger North tells us
that Purcell’s score was “unhappily lost” within a few years after
the opera was produced, and all the manuscripts now remaining are
more or less imperfect. However, the tune and words are found in
\textit{Pills to Purge Melancholy}, and the former in at least a dozen
ballad-operas, under the name of \textit{We’ve cheated the Parson}. Purcell,
no doubt, composed a part-song, and this was probably extracted from
it.

When transformed into a ballad, the words underwent some
modification, and a second part was added as an antidote to the
first. Dryden’s introduction of Comus to sing with three peasants
about cheating the parson of his tithes, among the incantations of
Merlin, is rather anomalous.

The ballad-printers entitled it “The Country Farmer’s Vain-Glory,
in a new.

\end{fixedpage}%582
\pagebreak

\begin{fixedpage}%583
\rectoheader

song of Harvest Home; sung to a new tune much in request:” and
the second part, “An Answer to Harvest Home: a true character of such
countrymen who glory in cheating the vicar, and prefer bag-pudding
and dumpling before religion and learning.”

\settowidth{\versewidth}{They’d rather hear Robin, the Piper play:\dots}
\begin{scverse}
\vleftofline{“}Tell them of going to Church to pray,\\
They’d rather hear Robin, the Piper play:\dots\\
Their hungry appetite to suffice,\\
Bag-pudding and dumpling they idolize.\dots\\
Likewise, by the laws of this potent land,\\
They in the pillory ought to stand.”
\end{scverse}

These were issued by printers who were contemporaries of Dryden
and Purcell, for we find copies in the Roxburghe and other
collections, printed by Brooksby, Deacon, Blare, and Back. (Rox. ii.
82, Halliwell No. 54, \&c.)

The first part has been reprinted in \textit{Festive Songs}, by W. Sandys,
F.S.A., and in both editions of \textit{Ballads and Songs of the Peasantry of
England}.

\indentpattern{001001001}
\settowidth{\versewidth}{We’ve cheated the parson, we’ll cheat him again;}
\begin{scverse}
\begin{patverse}
\vleftofline{“}We’ve cheated the parson, we’ll cheat him again;\\
For why should the Vicar have one in ten?\\
One in ten; one in ten; \&c.\\
For staying while dinner is cold and hot,\\
And pudding and dumpling are burnt to th’ pot?\\
Burnt to pot; burnt to pot; \&c.\\
We’ll drink off our liquor while we can stand,\\
And hey for the honour of Old England!\\
Old England; Old England;” \&c.
\end{patverse}
\end{scverse}

After a time, the first part of the above tune was discarded, and
the second joined on to the nursery tune of \textit{Boys and Girls, come out
to play}. It is found in that form in the ballad-opera of \textit{Polly}, 1729,
under the name of \textit{We’ve cheated the Parson}, and in the third volume
of \textit{The Dancing Master}, under that of “\textit{Girls and Boys, come out to
play}: the new way.” The “old way” was to repeat the first strain with
a little variation, or to take it a fifth higher.
\end{fixedpage}%583
\pagebreak

\begin{fixedpage}%584
\versoheader

\musictitle{AYE, MARRY, AND THANK YE TOO.}

This is the burden of the ballad, “I live in the town of Lynn,” a
continuation of which (with a somewhat similar tune) will be found in
\textit{Pills to purge Melancholy}, iii. 131, 1707, commencing:

\settowidth{\versewidth}{Who often said, ‘Thank yon too.’”}
\begin{scverse}
“I am the young lass of Lynn,\\
Who often said, ‘Thank you too.’”
\end{scverse}

This air is found under the title of \textit{I marry, and thank ye too},
in \textit{Youth’s Delight on the Flagelet}, 1697; of \textit{I live in the town of
Lynn}, in \textit{Silvia}, or \textit{The Country Buria}l, 1831; and of \textit{The Bark in
Tempest tost}, in \textit{Robin Hood}, 1730. The last name is from the song
adapted to it in Silvia.

There is a variety of ballads extant that were sung to the tune;
for instance, in the second volume of the Roxburghe Collection there
are three, printed by Brooksby, Deacon, Blare, and Back.

A few stanzas of one of these, “The London Lass’s Lamentation: or
Her fear she should never be married,” are here printed with the
music.
\end{fixedpage}%584
\pagebreak

\begin{fixedpage}%585
\rectoheader

\settowidth{\versewidth}{Because I’m so loth to die an old maid.}
\begin{dcverse}
\begin{altverse}
Mine eyes do like fountains flow,\\
As I on my pillow lie,\\
There’s none know what I undergo, \\
Yet cannot be married, not I.
\end{altverse}

\begin{altverse}
My father is grey and old,\\
And, surely, ere long will die,\\
And though he’ll leave me all his gold, \\
Yet cannot be married, not I.
\end{altverse}

\begin{altverse}
In silks I am still array’d.\\
And ev’ry new fashion buy,\\
Because I’m so loth to die an old maid. \\
Yet cannot be married, not I.
\end{altverse}

\begin{altverse}
The gold which I have in store,\\
I value no more than clay,\\
I’d give it all, and ten times more.\\
So I might be married to-day.
\end{altverse}
\end{dcverse}

\musictitle{BARTHOLOMEW FAIR.}

“The Countryman’s Ramble through Bartholomew Fair” is contained
in Vol. I. of \textit{Pills to purge Melancholy} , 1699 to 1714, and in Vol.
III. of the 1719 edition. The tune in \textit{The Dancing Master} of 1695, and
in subsequent editions, also in \textit{The Quaker’s Opera}, and other
ballad-operas. In \textit{The Dancing Master} it is entitled, \textit{The Whim}, or
\textit{Bartholomew Fair}.

It will be observed that the rhythm is somewhat peculiar, each
phrase consisting of three bars. This is one of the forms common to
English jigs and hornpipes. Many examples of similar metre might be
adduced, but it may be sufficient to cite “A Northern Jigg,” on the
first page of \textit{Apollo’s Banquet}, 1690 or 1693.

The termination of each phrase with the same note three-times,
has been considered as a characteristic of Irish music, but there are
many old English tunes which share this peculiarity.\textsuperscript{a} I imagine it
to have arisen, in both countries, from the music keeping time with
the steps of the dance,—

\settowidth{\versewidth}{Why should or you or we so much forget}
\begin{scverse}
\vleftofline{“}Why should or you or we so much forget \\
The season in ourselves, as not to make \\
Use of our youth and spirits, to awake \\
The nimble hornpipe, and the tambourine.\\
And \textit{mix our songs and dances} in the wood.”\\
\vin\vin\vin\vin\vin\vin\vin\vin Ben Jonson’s \textit{Sad Shepherd}.
\end{scverse}

\begin{dcfootnote}
\textsuperscript{a} I might quote some of the earliest copies of \textit{John, 
come kiss me now}, of \textit{I am the Duke of Norfolk}, and others. 
The copy of \textit{John, come kiss me now} that I have chosen,
is a violin version, and the second note of the three is
there taken, usually, the octave lower.
\end{dcfootnote}

\end{fixedpage}%585
\pagebreak

\begin{fixedpage}%586
\versoheader

The hornpipe concludes with three beats of the feet, and I am
informed by those who have seen dancing among the factory people in
Lancashire, that in their dances they tap the heels together three
times, so as to make them ring, at each close. In Ireland, according
to Dr. Petrie, “in general the floor is struck, or rather, tipped
lightly, three times during every bar of the tune” of the “hop-jig.”

\settowidth{\versewidth}{Where trumpets and bagpipes, kettledrums, fiddlers, were all at work. }
\begin{scverse}
In gold and zilver, zilk and velvet, each was drest,\\
A Lord in his zattin, was busy a prating among the rest,\\
But one in blue jacket did come, whome some do Andrew call,\\
Adsheart, talk’d woundy wittily to them all.

At last, cutzooks, he made such sport, I laugh’d aloud,\\
The rogue being fluster’d, he flung me a custard, amidst the croud.\\
The volk vell a laughing at me; and then the vezen said,\\
“Be zure, Ralph, give it to Doll, the dairy maid.”

I zwallow’d the affront, but I would stay no longer there,\\
I thrust and I scrambled, till further I rambled into the Fair,\\
Where trumpets and bagpipes, kettledrums, fiddlers, were all at work. \\
And the cooks sung, “Here’s your delicate pig and pork.”
\end{scverse}

\end{fixedpage}%586
\pagebreak

\begin{fixedpage}%587
\rectoheader

He then went to see the tumbling and the dancing, and ends thus:—
\settowidth{\versewidth}{But never a penny was left of my money, che’ll vow and zwear,}
\begin{scverse}
I thrust and shov’d along as well as ever I could,\\
At last I did grovel into a dark hovel where drink was sold,\\
They brought it in cans which cost a penny a-piece, adsheart!\\
I’m zure twelve ne’er could vill a country quart.

Che went to draw her purse, to pay for their beer,\\
But never a penny was left of my money, che’ll vow and zwear,\\
They took my hat for a groat, and turn’d me out o’ th’ door,\\
Adswo'unds, Ralph, I never will go with such rogues any more.
\end{scverse}

\musictitle{MAY FAIR.}

This is sometimes entitled May Fair, and sometimes, \textit{O Jenny,
Jenny, where hast thou been}? The latter is from a song by D’Urfey,
entitled, \textit{The Willoughby Whim}.

It is contained in \textit{Pills to purge Melancholy}  (i. 169, 1719); in
\textit{The Beggars’ Opera}, 1728; \textit{The Grub Street Opera}, 1731; \textit{The
Fashionable Lady}, or \textit{Harlequin's Opera}, 1730; and in some editions of
\textit{The Dancing Master}.

May Fair was established as a Fair, “in the fields behind
Piccadilly,” in the time of Charles~II. About the commencement of the
last century, the ground was partially built over, and among other
erections was a chapel, that became as celebrated for clandestine
marriages as the precincts of the Fleet. The registers of those
marriages are now in the parish church of St. George, Hanover Square.

This tune was probably a favorite at the fairs. The words that
were written to it for ballad-operas possess but little interest
apart from the dramas. I have therefore adapted an old lullaby.

\settowidth{\versewidth}{Care you know not, therefore sleep}
\begin{dcverse}
Care you know not, therefore sleep, \\
While I o’er you watch do keep;\\
Sleep, pretty darlings, do not cry, \\
And I will sing a lullaby.
\end{dcverse}
\end{fixedpage}%587
\pagebreak


\begin{fixedpage}%588
\versoheader

\musictitle{THREE MERRY MEN OF KENT.}

In the ballad-opera of \textit{The Jovial Crew}, the old name of this air
is given as \textit{Three merry men of Kent}.

In \textit{Folly in print, or a Book of Rymes}, 1667, is a song entitled
“Three merry \textit{boys} of Kent,” to the tune of \textit{I rode from England into
France}; but I have not found “Three merry \textit{men} of Kent.”

The words sung to the tune in \textit{The Jovial Crew}, form the fourth
stanza of a song commencing, “He that will not merry be.”

It was printed on broadsides, and in several of the collections
of old songs which were published in the early part of the last
century. The three first stanzas are here copied from \textit{A complete
Collection of old and new English and Scotch Songs, with their
respective tunes prefixed}, 8\textit{vo}, 1735, i. 137:—

\indentpattern{01013}
\settowidth{\versewidth}{May he be obliged to drink small beer,}
\begin{dcverse}
\begin{patverse}
He that will not merry, merry be,\\
And take his glass in course,\\
May he be obliged to drink small beer, \\
Ne’er a penny in his purse.\\
\textit{Let him be merry, merry there, \&c}.
\end{patverse}

\begin{patverse}
He that will not merry, merry be,\\
With a company of jolly boys.\\
May he be plagued with a scolding wife, \\
To confound him with her noise.\\
\textit{Let him be merry, \&c}.
\end{patverse}
\end{dcverse}

\end{fixedpage}%588
\pagebreak

\begin{fixedpage}%589
\rectoheader

\musictitle{ROUND AND ROUND, THE MILL GOES ROUND.}

In the second part of The Dancing Master, 1696 and 1698, this
tune is entitled, \textit{The happy Miller.} It is printed three or four times
over in \textit{Pills to purge Melancholy} , under different names, and is
contained in several of the ballad-operas.

One of the songs in the Pills is—

\indentpattern{00003}
\settowidth{\versewidth}{By the sleight of his hand, and the strength of his back,}
\begin{scverse}
\begin{patverse}
\vleftofline{“}How happy’s the mortal that lives by his mill,\\
That depends on his own, not on Fortune’s wheel;\\
By the sleight of his hand, and the strength of his back,\\
How merrily his mill goes, clack, clack, clack.\\
How merrily, \&c.
\end{patverse}

\begin{patverse}
If his wife proves a scold, as too often 'tis seen,\\
For she may be a scold, sing God bless the Queen;\\
With his hand to the mill, and his shoulder to the sack,\\
He drowns all discord with the merry clack, clack, clack.\\
He drowns all discord,” \&c.
\end{patverse}
\end{scverse}

There must be another Miller’s Song, which I have not found, as
the words, “Round and round, the mill goes round,” do not occur in
the above.

Another of the songs is “The Jovial Cobbler of Saint Helen’s.
Tune of \textit{Mill goes clack} (\textit{Pills}, iii. 151, 1707.)

\settowidth{\versewidth}{But there’s Dick the carman, and Hodge, who drives the dray}
\begin{scverse}
\vleftofline{“}I am a jovial cobbler, bold and brave,\\
And, as for employment, enough I have \\
For to keep jogging my hammer and my awl,\\
Whilst I sit singing and whistling in my stall.

But there’s Dick the carman, and Hodge, who drives the dray \\
For sixteen or eighteen pence a day,\\
They slave in the dirt, whilst I, with my awl,\\
Do get more money sitting, singing, whistling in my stall.

And there’s Tom the porter, companion of the pot,\\
Who stands in the street, with his rope and knot,\\
Waiting in a corner to hear who will him call,\\
Whilst I am getting money, money, money in my stall.

And there’s the jolly broom-man, his bread for to get,\\
Cries “Brooms” up and down in the open street,\\
And one cries “Broken glasses, though never so small,”\\
Whilst I am getting money, money, money in my stall.

And there is a gang of poor smutty souls,\\
Who trudge up and down, to cry “Small coals,"\\
With a sack on their back, at the door stand and call,\\
Whilst I am getting money, money, money in my stall.

And others there are with another note,\\
Who ery up and down “An old suit or coat,”\\
And perhaps, on some days, they get nothing at all,\\
Whilst I sit singing, getting money, money in my stall.
\end{scverse}

\end{fixedpage}%589
\pagebreak

\begin{fixedpage}%590
\versoheader

\indentpattern{0001}
\settowidth{\versewidth}{Their casks to be made tight, with hoops great and small}

\begin{scverse}
\begin{patverse}
And there’s the jolly cooper, with hoops at his back,\\
Who trudgeth up and down to see who lack\\
Their casks to be made tight, with hoops great and small.\\
Whilst I sit singing, getting money, \&c.
\end{patverse}

\begin{patverse}
And there’s a jolly tinker, who loves a bonny lass,\\
Who trudges up and down to mend old brass,\\
With his long smutty pouch, to force holes withal,\\
Whilst I sit, \&c.
\end{patverse}

\begin{patverse}
And there is another, call’d old Tommy Terrah,\\
Who, up and down the city, does drive with a barrow,\\
To try to sell his fruit to great and to small,\\
Whilst I sit, \&c.
\end{patverse}

\begin{patverse}
And there are the blind, and the lame with wooden leg,\\
Who, up and down the city, are forc’d to beg:\\
They get crumbs of comfort, the which are but small,\\
Whilst I sit, \&c.
\end{patverse}

\begin{patverse}
And there’s a gang of wenches, who oysters do sell,\\
And then Powder Moll, with her scent-sweet smell;\\
She trudges up and down with powder and with ball,\\
Whilst I sit, \&c.
\end{patverse}

\begin{patverse}
And there are jovial girls with their milking pails,\\
Who trudge up and down, with their draggle-tails\\
Flip-flapping at their heels; for customers they call,\\
Whilst I sit, \&c.
\end{patverse}
\end{scverse}

\begin{scverse}
These are the gang who do take great pain,\\
And it is these who me maintain,\\
But when it blows and rains, I do pity them all,\\
To see them trudge about, while I am in my stall.

And there are many more who slave and toil,\\
Their living to get, but it’s not worth while \\
To mention them all; so I’ll sing in my stall,\\
I am the happiest mortal, mortal of them all.”
\end{scverse}

The third, in the \textit{Pills}, is “The jolly Sailor’s Resolution.” (vi.
41.) It is a long ballad of fourteen stanzas, relating how the sailor
had been well received by his hostess, at Limehouse, when he had
“abundance of gold,” and was to have married her daughter; and how
the daughter was coy, and the mother handed him over to a press-gang,
as soon as it was exhausted. Now, having replenished his store, his
resolution is to forsake the “canting crew,” who were again beginning
to flatter him, and to marry another. He begins thus:—

\settowidth{\versewidth}{To show myself a jolly, jolly, brisk young man.”}
\begin{scverse}
\vleftofline{“}That I am a sailor, ’tis very well known,\\
And never, as yet, had a wife of my own;\\
But now I’m resolv’d to marry if I can,\\
To show myself a jolly, jolly, brisk young man.”
\end{scverse}

There are several copies of the above, and it has been reprinted
in Halliwells \textit{Early Naval Ballads of England}.
\end{fixedpage}%590

\pagebreak

\begin{fixedpage}%591
\rectoheader

The following is from the ballad-opera of \textit{The Jovial Crew}:—

\settowidth{\versewidth}{To our master’s good health shall the cup he crown’d:}
\begin{scverse}
Nor sorrow, nor pain, amongst us shall he found,\\
To our master’s good health shall the cup be crown’d:\\
That long he may live, and in bliss may abound,\\
Shall he ev’ry man’s wish, while the howl goes round.\\
\vin\vin \textit{Chorus}. Shall he ev’ry man’s wish, \&c.
\end{scverse}

\musictitle{I OFTEN FOR MY JENNY STROVE.}

This is contained in book iii. of \textit{The Banquet of Music},
consisting of “songs sung at the Court and theatres,” 1689; in
\textit{Apollo’s Banquet}, 1690; in \textit{The Dancing Master}, from 1695; in all
editions of \textit{Pills to purge Melancholy} ; and in \textit{The Jovial Crew}, and
other ballad-operas.

One of the ballads sung to the air, entitled \textit{Cupid’s Revenge}, is
almost a paraphrase of \textit{King Cophetua and The Beggar Maid},—alluded to
by Shakespeare, and reprinted by Percy in the \textit{Reliques}. “Cupid’s
Revenge” is contained in \textit{Old Ballads}, i. 138, 8vo., 1723, and in
Evans’ \textit{Old Ballads}, ii. 361, 1810. Evans, as usual, omits the name of
the tune. It commences thus:—

\settowidth{\versewidth}{Whom no fair face could ever please:}
\begin{scverse}
\begin{altverse}
“A king once reign’d beyond the seas,\\
As we in ancient stories find,\\
Whom no fair face could ever please:\\
He cared not for womankind.\\
He despis’d the sweetest beauty,\\
And the greatest fortune too;\\
At length he married to a beggar;\\
See what Cupid’s dart can do,’’ \&c.
\end{altverse} 
\end{scverse}


\end{fixedpage}%591
\pagebreak

\begin{fixedpage}%592
\versoheader

There are several black-letter ballads to the tune in the
Roxburghe and Douce Collections, such as “The love-sick Serving Man;
showing how he was wounded with the charms of a young lady, and did
not dare to reveal his mind” (Rox., ii. 299); “The old Miser slighted”
(Rox., ii. 387); \&c.

The original words, which are in \textit{The Banquet of Music}, and in
the \textit{Pills}, are here reprinted with the music.

\settowidth{\versewidth}{And for ever would be, should he, could be}
\begin{dcverse}
When first I saw thy lovely charms, \\
I kiss’d thee, wish’d thee in my arms;\\
I often vow’d and still protest \\
Tis Joan alone that I love best.\\
I have gotten twenty pounds,\\
My father’s house, and all his grounds,\\
And for ever would be, should be, could be\\
Join’d with none, but only thee.
\end{dcverse}


\musictitle{LADIES OF LONDON, BOTH WEALTHY AND FAIR.}

The tune is in \textit{The Dancing Master} of 1690, and in subsequent
editions; in \textit{Apollo’s Banquet}, 1690; in all editions of \textit{Pills to
purge Melancholy}; and in many ballad-operas. It is sometimes entitled
\textit{London Ladies}, instead of \textit{Ladies of London}.

A black-letter copy of the ballad is in the Roxburghe Collection,
ii. 5, printed
\end{fixedpage}%592
\pagebreak

\begin{fixedpage}%593
\rectoheader

for J. Back, on London Bridge, and entitled “Advice to the Ladies
of London in the choice of their husbands: to an excellent new Court
tune.”

The following were also sung to it:—

“Advice to the Ladies of London to forsake their fantastical
top-knots, since they are become so common with Billingsgate women,
and the wenches that cry kitchen stuff,” \&c. To the tune of \textit{Ye
Ladies of London} beginning—

\settowidth{\versewidth}{“Now you young females that follow the mode.”}
\begin{scverse}
\vleftofline{“}Now you young females that follow the mode.”
\end{scverse}

“The Country Maiden’s Lamentation:” beginning—
\begin{scverse}
\begin{altverse}
\vleftofline{“}There came up a lass from a country town,\\
Intending to live in the city,\\
In steeple-crown hat, and a paragon gown,\\
Who thought herself wondrous pretty.\\
Her petticoat serge; her stockings were green,”\&c.
\end{altverse}
\end{scverse}

The two last are in the Douce Collection. In the Roxburghe, ii.
101, is—

\begin{scverse}
\begin{altverse}
\vleftofline{“}A country gentleman came up to town,\\
To taste the delights of the city,\\
Who had to his servant a jocular clown,\\
Accounted to be very witty,”\&c.
\end{altverse}
\end{scverse}

There are several more in the same volume. See pages 97, 444, 519,
and 530.
\end{fixedpage}%593
\pagebreak

\begin{fixedpage}%594
\versoheader

\musictitle{THERE WAS AN OLD WOMAN LIV’D UNDER A HILL.}

This is contained in all the editions of \textit{Pills to purge
Melancholy}, and the tune introduced in \textit{The Jovial Crew}, and other
ballad-operas.

The following words are from \textit{The Jovial Crew}:—

\settowidth{\versewidth}{The maid was well pleas’d, and the miller content,}
\begin{scverse}
\begin{altverse}
The miller he kiss’d her; away she went,\\
Sing trolly, lolly, lolly, lolly lo;\\
The maid was well pleas’d, and the miller content,\\
Oh ho! Oh ho! Oh ho! was It so?
\end{altverse}

\begin{altverse}
He danc’d and he sung, while the mill went clack;\\
Sing trolly, lolly, lolly, lolly lo;\\
And he cherish’d his heart with a cup of old sack,\\
Oh ho! Oh ho! Oh ho! did he so?
\end{altverse}
\end{scverse}

\musictitle{THE COBBLERS’ HORNPIPE.}

From \textit{The Dancing Master} of 1701, and contained in subsequent
editions; also in vol. i. of Walsh’s \textit{Compleat Country Dancing Master}.
\end{fixedpage}%594
\pagebreak

\begin{fixedpage}%595
\rectoheader

NOBE’S MAGGOT [WHIM OR FANCY],

In \textit{The Dancing Master} of 1703, this is entitled Nobe’s Maggot. In
\textit{The Devil to Pay} another version is named \textit{There was a maid in the
West}.

There are many tunes of this class, closely resembling each other
in character, and sometimes in actual notes. I think them all to be
hornpipes or jigs.

Not having found the words of “There was a maid in the West,” I
have adapted a song in Round about our \textit{Coal-fire, or Christmas
Entertainments}, 4th edit., 1734.

\settowidth{\versewidth}{Kate, Dick, Ralph, and Molly.}
\begin{dcverse}
\begin{altverse}
Then, for your Christmas-box,\\
Sweet plum-cakes, and money,\\
Delicate Holland smocks,\\
Kisses sweet as honey.\\
Hey for the Christmas ball, \\
Where we shall be jolly, \\
Coupling short and tall,\\
Kate, Dick, Ralph, and Molly.\\
Then to the hop we’ll go, \\
Where we’ll jig, and caper \\
\textit{Cuckolds all a row};\\
Will shall pay the scraper: \\
Hodge shall dance with Prue, \\
Keeping time with kisses;\\
We’ll have a jovial crew\\
Of sweet and smiling misses.
\end{altverse}
\end{dcverse}

\musictitle{THERE WAS A PRETTY LASS, AND A TENANT OF MY OWN.}

This ballad is printed on broadsides with music, under the title
of \textit{The condescending Lass}. The air was extremely popular, and
introduced into the following ballad operas: \textit{The Beggars' Wedding};
\textit{The Jovial Crew}; \textit{The Generous Freemason}; \textit{Robin Hood}; \textit{The Livery Rake};
\textit{The Lover his own Rival}; \textit{The Court Legacy}; and \textit{The Grub Street Opera}.

It is sometimes entitled, \textit{A Tenant of my own}; and sometimes, \textit{I
had a pretty girl and a tenant}, \&c.
\end{fixedpage}%595
\pagebreak

\begin{fixedpage}%596
\versoheader

Among the songs which were written to it, and attained
popularity, are, “Sure marriage is a fine thing,” from \textit{The Beggars’
Wedding} (reprinted in vol. v. of Watts’s \textit{Musical Miscellany}, 1731),
and—

\settowidth{\versewidth}{I’m a bold' recruiting sargeant,}
\begin{scverse}
\textit{“}I’m a bold' recruiting sargeant,\\
From London I am come.”
\end{scverse}

The following song on the Italian Opera is from \textit{The Livery Rake},
1733. It shews that the exclusive patronage of foreign singers by the
English aristocracy is by no means a \textit{new} national peculiarity. The
fashion has become so old that we may almost hope for a change.

\settowidth{\versewidth}{But I hope the time will come, when their favourers will find,}
\begin{scverse}
But I hope the time will come, when their favourers will find,\\
With their fal, lal, la; fa, la la, la la, la,\\
They have paid too great a sum to Italian pipes for wind. \\
With their fal, lal, la; fa, la la, la la, la.\\
When English wit again, and merit too shall thrive,\\
And men of fortune to support that wit and merit strive, \\
Without ha, ha, ha, \&c.
\end{scverse}
\end{fixedpage}%596
\pagebreak

\begin{fixedpage}%597
\rectoheader

\musictitle{COME, AND LISTEN TO MY DITTY.}

The old sea song, \textit{Come, and listen to my ditty}, or \textit{The Sailor’s
Complaint}, is to be found in \textit{The Universal Musician}, and in vol. iv.
of \textit{The British Musical Miscellany}, published by Walsh. The air is now
commonly known as “Cease, rude Boreas,” from a song which, according
to Ritson and others, was written by George Alexander Stevens. It is
an amplification of a “Marine Medley” in Stevens’s \textit{Songs, Comic and
Satyrical}, Oxford, 1772.

In the ballad-opera of \textit{Silvia, or the Country Burial}, printed in
1731, the song, “On some rock, by seas surrounded,” is adapted to the
tune, and the old name is there given as \textit{How happy are young lovers};
so, also, in \textit{Robin Hood}, 1730.

The title, \textit{How happy are young lovers}, is derived from the ballad
of \textit{The Distracted Sailor}; a copy of which is in the Douce Collection,
and a second in that of Mr. J. M. Gutch. In the latter copy it is
said to be to the tune of \textit{What is greater joy or pleasure}, which
carries the air a stage further back.

\textit{The Distracted Sailor} is a long ballad of ten stanzas. The
following are the first two:—

\settowidth{\versewidth}{He would wed, if she’d not leave him,}
\begin{dcverse}
\begin{altverse}
\vleftofline{“}O how happy are young lovers, \\
When they courtship first begin;\\
How their faces do discover \\
The great pleasure they are in! \\
When one seems to like the other, \\
Hand in hand these lovers move, \\
And with kisses they do smother, \\
While they prattle tales of love.\\
Just so Billy, the sailor, courted \\
Molly, and she was mostly kind;\\
For they oft had kiss’d and sported, \\
Each persuaded was in mind.\\
She consented for to have him,\\
He made vows to her again;\\
He would wed, if she’d not leave him, \\
When he did return from Spain,” \&c.
\end{altverse}
\end{dcverse}

Many other sea-songs were sung to this air. Among them, Glover’s
ballad of \textit{Hosier’s Ghost} (commencing, “As near Portobello lying”),
and \textit{Admiral Vernon’s Answer to Admiral Hosier’s Ghost},—“Hosier! with
indignant sorrow.” These are reprinted in Halliwell’s \textit{Early Naval
Ballads of England}.

The following is \textit{The Sailor’s Complaint }:--

\settowidth{\versewidth}{From foreign parts I was just come over,}
\begin{dcverse}
\begin{altverse}
\vleftofline{“}Come, and listen to my ditty,\\
All ye jolly hearts of gold;\\
Lend a brother Tar your pity,\\
Who was once so stout and bold.\\
But the arrows of Cupid,\\
Alas! have made me rue;\\
Sure, true love was ne’er so treated,\\
As am I by scornful Sue.
\end{altverse}

\begin{altverse}
When I landed first at Dover,\\
She appear’d a goddess bright;\\
From foreign parts I was just come over, \\
And was struck with so fair a sight.\\
On shore pretty Sukey walked,\\
Near to where our frigate lay,\\
And altho’ so near the landing,\\
I, alas! was cast away.
\end{altverse}

\begin{altverse}
When first I hail’d my pretty creature, \\
The delight of land and sea,\\
No man ever saw a sweeter,\\
I’d have kept her company;\\
I’d have fain made her my true love, \\
For better, or for worse;\\
But alas! I cou’d not compass her.\\
For to steer the marriage course.
\end{altverse}

\begin{altverse}
Once, no greater joy and pleasure \\
Could have come into my mind,\\
Than to see the bold Defiance \\
Sailing right before the wind,\\
O’er the white waves as she danced, \\
And her colours gaily flew:\\
But that was not half so charming \\
As the trim of lovely Sue.
\end{altverse}
\end{dcverse}

\end{fixedpage}%597
\pagebreak

\begin{fixedpage}%598
\versoheader

\settowidth{\versewidth}{Where the rolling mountain billows}
\begin{dcverse}
\begin{altverse}
On a rocky coast I’ve driven,\\
Where the stormy winds do rise, \\
Where the rolling mountain billows \\
Lift a vessel to the skies:\\
But from land, or from the ocean, \\
Little dread I ever knew,\\
When compared to the dangers \\
In the frowns of scornful Sue.\\
Long I wonder’d why my jewel \\
Had the heart to use me so.\\
Till I found, by often sounding. \\
She’d another love in tow:\\
So farewell, hard-hearted Sukey,\\
I’ll my fortune seek at sea,\\
And try in a more friendly latitude. \\
Since in yours I cannot be.”
\end{altverse}
\end{dcverse}

The descriptive song of “The Storm,” or “Cease, rude Boreas,” is
printed in so many collections (in Ritson’s \textit{English Sungs}, in the
Rev. James Plumtre’s Collection, in \textit{The Universal Songster}, \&c.)
that it may suffice here to republish the first stanza with the tune.

\musictitle{CHESHIRE ROUNDS.}

This is contained in the eleventh and subsequent editions of \textit{The
Dancing Master}, in the first volume of Walsh’s \textit{Compleat Country
Dancing Master}, in \textit{Polly}, and other ballad operas;

Mr. George Daniel, in his \textit{Merry England}, remarks that the only
known portrait of Dogget, the actor (of coat and badge notoriety), is
a small engraving representing him dancing the \textit{Cheshire Round}. Mr.
Daniel prints one of Dogget’s play-bills, issued in 1691, and the
following, from other bills of the time of
\end{fixedpage}%598
\pagebreak

\begin{fixedpage}%599
\rectoheader

William III., shewing how popular the dance then was:—“In
Bartholomew Fair, at the Coach-house on the pav’d stones at
Hosier-Lane end, you will see a Black that dances the \textit{Cheshire Rounds}
to the admiration of all spectators.”—“John Sleepe now keeps the
Whelp and Bacon in Smithfield Rounds, where are to be seen, a young
lad that dances a Cheshire Round to the admiration of all people.” A
third and similar advertisement was issued by Michael Root.

Cheshire Rounds is one of the tunes called for by “the hobnailed
fellows” in “A Second Tale of a Tub,” 8vo., 1715.

\musictitle{SHROPSHIRE ROUNDS}
From the second volume of \textit{The Dancing Master}, and
the second volume of Walsh’s \textit{Compleat Country Dancing Master}.

\end{fixedpage}%599
\pagebreak

\begin{fixedpage}%600
\versoheader

\musictitle{GREENWICH PARK; \textsc{or}, COME, SWEET LASS.}

The tune of \textit{Greenwich Park} is contained in Part II. of \textit{The
Dancing Master} of 1698, and in all subsequent editions.

In the first edition of \textit{The Beggars’ Opera} the air is named
“Come, sweet Lass,” from the first line of a song which, when printed
in ballad form, is sometimes entitled “Slighted Jockey: or Coy
Moggy’s unspeakable Cruelty.” The words of that song are contained in
\textit{The Compleat Academy of Complements}, 1685, and in several other
collections. The first two stanzas are printed with the air in all
the editions of \textit{Pills to purge Melancholy} . It is here presented
entire.

\indentpattern{100110000}
\settowidth{\versewidth}{’Cause Moggy proves unkind.}
\begin{dcverse}
\begin{patverse}
\vin On our green \\
The loons are sporting, \\
Piping, courting,\\
On our green\\
The blithest lads are seen; \\
There, all day,\\
Our lasses dance and play, \\
And every one is gay,\\
But I, when you’re away.
\end{patverse}

\begin{patverse}
\vin How can I \\
Have any pleasure \\
While my treasure\\
Is not by?\\
The rural harmony\\
I’ll not mind,\\
But, captive like, confin’d,\\
I lie in shades behind,\\
’Cause Moggy proves unkind.
\end{patverse}

\begin{patverse}
\vin There is none \\
That can delight me,\\
If you slight me;\\
All alone,\\
I ever make my moan. \\
Life’s a pain\\
Since by your coy disdain, \\
Like an unhappy swain,\\
I sigh and weep in vain.
\end{patverse}
\end{dcverse}

\end{fixedpage}%600
\pagebreak

\begin{fixedpage}%601
\rectoheader

\indentpattern{100110000}
\settowidth{\versewidth}{But have my love in sight,}
\begin{dcverse}
\begin{patverse}
\vin I could be \\
Right blythe and jolly; \\
Melancholy\\
Ne’er should be \\
My fatal destiny,\\
If I might\\
But have my love in sight, \\
Whose angel-beauty bright \\
Was ever my delight.
\end{patverse}

\begin{patverse}
\vin Have I not,\\
In Moggy’s dances \\
Seen those glances,\\
Which have shot,\\
And, like a fowler, caught \\
My poor heart?\\
Yes, and I feel the smart \\
Of Cupid’s fatal dart,\\
Since we have been apart.
\end{patverse}

\begin{patverse}
\vin Jemmy can,\\
With pretty Nancy \\
Please his fancy;\\
Jemmy can,\\
Tho’ not so blythe a man, \\
Have his will,\\
Kiss and delight her still, \\
While I on each green hill, \\
Weep and lament my fill.
\end{patverse}

\begin{patverse}
\vin I’ll not wear \\
The wreath of willow; \\
Floramella,\\
Charming fair,\\
Shall ease me of my care: \\
Who can tell,\\
But she may please as well? \\
No longer will I dwell \\
In love’s tormenting cell.
\end{patverse}
\end{dcverse}

\musictitle{HOBBY-HORSE DANCE.}

\settowidth{\versewidth}{“For, O, for, O, the hobby-horse is forgot.”—\textit{Hamlet}, act iii., sc. 2.}
\begin{scverse}
“For, O, for, O, the hobby-horse is forgot.”—\textit{Hamlet}, act iii., sc. 2.
\end{scverse}

“At Abbot’s, or now Paget’s, Bromley,” says Dr. Plott, “they had,
within memory, a sort of sport, which they celebrated at Christmas
(on New-Year and Twelfth Day), called \textit{The Hobby-Horse Dance}, from a
person that carried the image of a horse between his legs, made of
thin boards, and in his hand a bow and arrow, which, passing through
a hole in the bow, and stopping upon a shoulder it had in it, he made
a snapping noise as he drew it to and fro, keeping time with the
musick. With this man danced six others. . . . They danced the Hays,\textsuperscript{a}
and other country dances. To this Hobby-Horse Dance there also
belonged a pot, which was kept by turns by four or five of the chief
of the town, whom they called Reeves, who pounded cakes and ale to
put in this pot; all people who had any kindness for the good intent
of the institution of the sport, giving pence a-piece for themselves
and families, and so foreigners too that came to see it; with which
money (the charge of the cakes and ale being defrayed) they not only
repaired their church, but kept their poor too; which charges are not
now perhaps so cheerfully borne.”—\textit{Natural History of Staffordshire},
fol, 1686, p. 434.

There are several hobby-horse dances extant: one in \textit{Musicks
Delight on the Cithren}, 1666, in \textit{Apollo's Banquet}, 1669 to 1693, and
in some later collections; a second in \textit{Pills to purge Melancholy} , i.
19, 1719; a third in the \textit{Antidote to Melancholy}, 1719.

In the Bagford Collection, there is a ballad to the first,
entitled “A new ballad of a famous German Prince [Rupert] and a
renowned English Duke [of Albemarle], who, on St. James’s Day, 1666,
fought with a beast with seven heads called Provinces, not by land,
but by water. Not to be said, but sung.” It begins:— 
\settowidth{\versewidth}{“There happened of late a terrible fray,}
\begin{scverse}
“There happened of late a terrible fray,\\
Begun upon our St. James’s Day.”
\end{scverse}

\begin{scfootnote}
\textsuperscript{a} The Hay is described by Strutt as a rustic dance, where they
lay hold of hands, and dance round in a ring.
\end{scfootnote}
\end{fixedpage}%601
\pagebreak

\begin{fixedpage}%602
\versoheader

To the second, D’Urfey wrote the song commencing “Jolly Roger
Twangdillo, of Plowden Hall and to the third, “The Yeoman of Kent,”
commencing—

\settowidth{\versewidth}{In Kent, I hear, there lately did dwell}
\begin{scverse}
\vleftofline{“}In Kent, I hear, there lately did dwell \\
Long George, a yeoman by trade.”
\end{scverse}

The last (slightly altered, and with the addition of \textit{tol de rol}
at the end) is the tune of the satirical ballad of “The Vicar and
Moses,” beginning—

\settowidth{\versewidth}{At the sign of the Horse, old Spintext, of course,}
\begin{scverse}
\vleftofline{“}At the sign of the Horse, old Spintext, of course,\\
At night took his pipe and his pot;”
\end{scverse}
and, before that, seems to
have served for a similar attack upon the Reliques exhibited by the
Jesuits at the Savoy Chapel in the Strand, entitled “Religious
Reliques; or, The Sale at the Savoy, upon the Jesuits breaking up
their School and Chapel” (1689). The following is the first stanza:—

\musictitle{OF ALL THE SIMPLE THINGS WE DO.}

The words of this are by D’Urfey, and “made to a comical tune in
\textit{The Country Wake}.” The play of \textit{The Country Wake} was written by
Hogget, the actor, who bequeathed the annual coat and badge to the
Thames watermen. It was printed in 1696.

The tune is in the second volume of \textit{The Dancing Master}, and was
introduced into \textit{The Beggars’ Opera}, \textit{The Generous Freemason}, \textit{The
Patron}, and \textit{An Old Man taught Wisdom}.

D’Urfey’s song is printed in \textit{Pills}, i. 250, 1719; and in Watts’s
\textit{Musical Miscellany}, v. 108, 1781. In the latter, entitled “Marriage;”
in the former, “The Mouse-trap.” In \textit{The Dancing Master}, “Old Hob, or
The Mouse-trap.”
\end{fixedpage}%602
\pagebreak

\begin{fixedpage}%603
\rectoheader

\settowidth{\versewidth}{I gam’d and drank, and play’d the fool,}
\begin{dcverse}
\begin{altverse}
I gam’d and drank, and play’d the fool, \\
And a thousand mad frolicks more;\\
I rov’d and rang’d, despis’d all rule,\\
But I never was married before.\\
This was the worst plague could ensue,\\
I’m mew’d in a smoky house;\\
I used to tope a bottle or two,\\
But now ’tis small beer with my spouse!\\
My darling freedom crown’d my joys, \\
And I never was vex’d in my way;\\
If now I cross her will, her voice \\
Makes my lodging too hot for my stay. \\
Like a Fox that is hamper’d, in vain \\
I fret out my heart and soul,\\
Walk to and fro the length of my chain, \\
Then am forc’d to creep into my hole.
\end{altverse}
\end{dcverse}

\musictitle{MAD MOLL.}

There are two versions of this tune in \textit{The Dancing Master}. The
first appeared, under the name of \textit{Mad Moll}, in Part II. of the
edition of 1698; the second, under that of \textit{The Virgin Queen}, in the
edition of 1703. Both were retained in all editions issued after
these dates.

Dean Swift’s song, “Oh! my Kitten, my Kitten!” was written to the
second version, which Allan Ramsay (in printing the song in the
fourth volume of the \textit{Tea Table Miscellany}, 1740), calls \textit{Yellow
Stockings}.
\end{fixedpage}%603
\pagebreak

\begin{fixedpage}%604
\versoheader

“Oh! my Kitten!” was also printed in \textit{The Trader’s Garland}, with
an “Answer from the Bishop to the Dean,” beginning—

\settowidth{\versewidth}{O my sweet Jonathan, Jonathan,}
\begin{scverse}
\vleftofline{“}O my sweet Jonathan, Jonathan,\\
\vin O my sweet Jonathan Swifty 
\end{scverse}
and “The Dean’s Answer to the Bishop,”
to the same tune.

\textit{Mad Moll} was introduced into Gay’s ballad-opera of \textit{Polly}, and is
mentioned in the popular ballad of “Arthur o’Bradley’s Wedding,”
written by a Mr. Taylor, early in the present century. In \textit{Momus
turn’d Fabulist}, 1729, instead of \textit{Mad Moll}, the old name is' given as
“Shall I be sick for love?”

Having printed \textit{The Virgin Queen}, or \textit{Yellow Stockings}, in my first
collection, the earlier version is now, for variety, subjoined.

Jigs and bagpipe hornpipes of this class became so much alike
towards the end of the seventeenth and early part of the eighteenth
centuries, that it is unnecessary to multiply specimens.

The first stanza of “Arthur o’Bradley’s Wedding” is printed to
the tune. It will be found entire in \textit{Songs and Ballads of the
Peasantry of England}, by J. H. Dixon.
\end{fixedpage}%604
\pagebreak

\begin{fixedpage}%605
\rectoheader

\musictitle{PORTSMOUTH.}

This tune is contained in the eleventh and subsequent editions of
\textit{The Dancing Master}.

I have not succeeded in finding the words, although there appears
but little doubt of its having been a ballad-tune,—perhaps to some
sailor’s parting with his love at Portsmouth.

The following words were written to the air bj Mr. John
Oxenford:—

\settowidth{\versewidth}{Vain thought! the moments fly, love,—}
\begin{dcverse}
\begin{altverse}
Vain thought! the moments fly, love,— \\
All are nearly gone;\\
Alas! too soon shall I, love,\\
Find myself alone.\\
But still my eyes to seek thee \\
Will wildly gaze around:\\
Hard heart, will nothing break thee? \\
Art with iron bound?\\
Nay, do not bid me hope, love,— \\
Hope I cannot bear;\\
Nay, rather let me cope, love,\\
Boldly with despair.\\
Should thoughts that may deceive me \\
Within my heart be nurs’d?\\
No,—leave me, dearest—leave me, \\
Now I know the worst.
\end{altverse}
\end{dcverse}

\end{fixedpage}%605
\pagebreak

\begin{fixedpage}%606
\versoheader

\musictitle{COURTIERS, COURTIERS, THINK NOT IN SCORN.}

This song is found in broadsides, with music, about the date of
1695. The first stanza only of the words is printed in the New
Academy of Complements, 1694 and 1713.

\settowidth{\versewidth}{Content’s the thing that mortals doth bless,}
\begin{dcverse}
\begin{altverse}
State and pomp no happiness brings,\\
A lower place more joys doth prove; \\
For Lords and Ladies, Princes and Kings, \\
With all on a level are in love.\\
And pretty brown Mary, making hay, \\
Hath charms as killing, killing, killing, \\
Always as killing charms as they.
\end{altverse}

\begin{altverse}
Content’s the thing that mortals doth bless, \\
And better far than a golden mine;\\
In Mary I the world possess,\\
And at no other’s lot repine.\\
Sweet Mary to me in careless hair\\
Has treasures far more taking, taking, \\
Than they that tow’rs and di’monds wear.
\end{altverse}
\end{dcverse}

\musictitle{NEW WELLS.}

This tune is contained in Walsh’s \textit{New Country Dancing Master}. It
seems to have been made out of \textit{Come, sweet Lass}, ante p. 600.

There were formerly several places of public amusement called New
Wells in the vicinity of London: New Wells at Richmond, 1698 to 1760;
New Wells at Islington, 1712 to 1740; New Wells “near the London
Spaw,” Clerkenwell, 1739-40; and New Wells at the bottom of Leman
Street, Goodman’s Fields.
\end{fixedpage}%606
\pagebreak

\begin{fixedpage}%607
\rectoheader

Of these the Wells at Islington (sometimes called the New
Tunbridge Well9) seem to have attained the highest repute.

“In 1733,” says Mr. George Daniel, “their Royal Highnesses the
Princesses Amelia and Caroline frequented them in the summer time,
for the pleasure of drinking the waters. They have furnished a
subject for pamphlets, plays, songs, and medical treatises, by N.
Ward, George Colman the elder, Bickham, Dr. Hugh Smith, \&c. Nothing
now remains of them but the original chalybeate spring, which is
still preserved in an obscure nook.” (\textit{Merrie England}, ii. 31.)

Although the neighbourhood is now “poverty-stricken and squalid,”
even within the memory of Mr. Daniel “beautiful \textit{tea}gardens”
encompassed the site.

The tune of \textit{New Wells} is essentially vocal, and is probably that
of some favorite song which was sung at the gardens. The name,
however, gives no clue to the words, and I have not met with it under
any other.

The following lines were written to the air by the late George
Macfarren:—

\settowidth{\versewidth}{Hark! yon joyous bird, morning’s light awakes him;}
\begin{scverse}
\begin{altverse}
Hark! yon joyous bird, morning’s light awakes him;\\
Warbling, free and pure, up he mounts secure: \\
Hark! yon joyous bird—lo! a shot o’ertakes him— \\
Such is life—be ours more calm and sure.
\end{altverse}

\begin{altverse}
Taste this crystal stream, oft by pilgrims chosen,\\
Born of summer show’rs, kiss’d by sweetest flow’rs: \\
Taste this crystal stream, purer still when frozen— \\
Such is truth, my fair, and such be ours.
\end{altverse}
\end{scverse}
\end{fixedpage}%607
\pagebreak

\begin{fixedpage}%608
\versoheader

\musictitle{THE DUSTY MILLER.}

This is contained in the first volume of Walsh’s \textit{Compleat Country
Dancing Master} and in \textit{The Lady’s Banquet}, published by Walsh; also in
a manuscript which was recently in the possession of the late Andrew
Blaikie, of Paisley, and there entitled \textit{Binny’s Jigg}.

It has been said that the tune of \textit{The Dusty Miller} is contained
in Queen Elizabeth’s Virginal Book, but I believe it to be only one
of the many random assertions that have been made about the contents
of that manuscript. The Virginal Book contains tunes that have
similar accent (such as \textit{The Carman’s Whistle}), but after turning over
every page, I found no \textit{Dusty Miller}.
\end{fixedpage}%608
\pagebreak

\renewcommand\versoheadertext{anglo-scottish songs.}
\renewcommand\rectoheadertext{anglo-scottish songs.}

\begin{fixedpage}%609
\rectoheader

\headingfour{ANGLO-SCOTTISH SONGS.}
\centerrule

\textsc{Before} closing this division of the book, it may be desirable to
devote a short space to the subject of the English and Anglo-Scottish
songs and tunes which are incorporated in collections of Scottish
music. They who have not enquired into the subject may not be aware
that many of the songs of Allan Ramsay, Burns, and other Scotch
poets, were written to English tunes, and that those tunes, being now
known by the names of their songs, pass with the world for Scotch.

Ritson tells us, in his \textit{Historical Essay on Scotch Song}, that
“the vulgar language of the lowland Scots was always called English
by their own writers till a late period,” and that “the vulgar toung
in Scottis” meant Gaelic or Erse. The quotations he adduces carry the
proof down to the first half of the sixteenth century; but, in the
early part of the eighteenth, this use of the word “English” was
altogether dropped, and “Scots Sangs” included not only songs written
by Scotchmen, whether in the lowland dialect or in English, but also
the meaning was extended to any purely English songs that were
popular in Scotland. As the works of Scotch poets are now sometimes
included under the head of English literature, where the
preponderance is English, so Allan Ramsay entitled his \textit{Tea Table
Miscellany} “a collection of Scots Sangs,” the preponderance in the
two first volumes (of which the work originally consisted) being
Scotch. Although it was soon extended to three volumes, and the third
was entirely English, still the exclusive title of “Scots Sangs” was
retained. In 1740 a fourth was added, partly consisting of Scotch and
partly of English. In this are twenty-one songs by Gay, from \textit{The
Beggars’ Opera}, ranged consecutively.

It would have been a great assistance to after-enquiry if Ramsay
had confined his selection to songs by Scotch authors, instead of
thus mixing up those of the two countries; and it would have been
more easy to separate the respective tunes if he had in all cases
given the names by which they were previously known. How far this was
required to divide the English from the Scotch will be best
exemplified by supplying the names of the tunes to half a dozen of
Ramsay’s own songs.

“My mither’s ay glowran o’er me,” to the country dance of \textit{A
Health to Betty}; “The maltman comes on Monday,” to the tune of \textit{Roger
de Coverley}; “Peggy, I must love thee,” to the tune of The Deel
assist the plotting Whigs,\textsuperscript{a}

\begin{dcfootnote}
\textsuperscript{a} “The Deel assist the plotting Whigs” is the first line 
of “The Whigs lamentable condition; or, The Royalists’ 
resolution: To \textit{a pleasant new tune}.” The words and 
music are contained in 180 \textit{Loyal Songs}, 1685 and 1694, 
and the music alone in \textit{Musick’s Handmaid}, Part II., 1689, 
as “a Scotch tune,” composed by Purcell. In \textit{Pills to 
purge Melancholy}, Vol. I., 1699 to 1714, the song of “Tom 
and Will were Shepherd Swains” is adapted to the air.

\end{dcfootnote}
\end{fixedpage}%609
\pagebreak

\begin{fixedpage}%610
\versoheader
\versoheader

composed by Purcell; “The bonny grey-ey’d morn begins to peep,”
to the tune of “an excellent new Play-house song, call’d \textit{The bonny
grey ey'd morn}, or \textit{Jockey rous'd with love},” composed by Jeremiah
Clark; “Corn riggs are bonny,” to the tune of \textit{Sawney was tall and of
noble race}, a song in D’Urfey’s play, \textit{The Virtuous Wife}; “Nanny O,”
to the tune of the English ballad of \textit{Nanny O}.\textsuperscript{a}

If this kind of scrutiny were carried through the songs in the
\textit{Tea Table Miscellany}, in Thomson’s \textit{Orpheus Caledonius}, or any other
collection, the bulk of Scottish music would be sensibly diminished;
but, on the whole, it would gain in symmetry. Many good and popular
tunes would be given up, but a mass of indifferent would be rejected
at the same time.

The mixture of English and Anglo-Scottish with the genuine
Scottish music has been gradually increasing since Thomson’s time.
Successive collectors have added songs that were.popular in their
day, without care as to the source whence they were derived; each
seeking only to render his own publication more attractive than those
of his predecessors. The songs of English musicians—often of living
authors—have been thus included, and their names systematically
suppressed. Although the authorship of these songs may have been
known to many at the time of publication, it soon passed out of
memory, and the Scotch have themselves been deceived into a belief in
their genuineness. Thus Burns, writing to Mr. Candlish, in June,
1787, about Johnson’s \textit{Scots Musical Museum}, says, “I am engaged in
assisting an honest Scotch enthusiast, a friend of mine, who is an
engraver, and has taken it into his head to publish a collection of
all our songs set to music, of which the words and music are done by
Scotsmen.” And again, in October, to another correspondent,—“An
engraver, James Johnson, in Edinburgh, has, not from mercenary views,
but from an honest Scotch enthusiasm, set about collecting \textit{all our
native songs}'’ \&c. And yet, within the first twenty-four songs of
the only volume then published, are compositions by Purcell, Michael
Arne, Hook, Berg, and Battishill.

Thomson’s \textit{Orpheus Caledonius} was printed in London; but the \textit{Scots
Musical Museum} was published in Edinburgh.

Although the popularity of Scottish music in England cannot be
dated further back than the reign of Charles II.,\textsuperscript{b} it may be proved,
from various sources, that English music was in favour in Scotland
from the fifteenth century, and that many English airs became so
popular as at length to be thoroughly domiciled

\begin{dcfootnote}
\textsuperscript{a} This ballad and the answer to it are in the Roxburghe
Collection. The first (ii. 415) is “The Scotch wooing of“Willy and
Nanny: \textit{To a pleasant new tune, or Nanny, O}.” Printed by P. Brooksby.
Although entitled “The Scotch wooing,” it relates to the most
southern part of Northumberland. It commences, “As I went forth one
morning fair,” and has for burden—

\settowidth{\versewidth}{It is Nanny, Nanny, Nanny O,}
\begin{fnverse}
\begin{altverse}
\vleftofline{“}It is Nanny, Nanny, Nanny O,\\
The love I bear to Nanny O,\\
All the world shall never know \\
The love I hear to Nanny O.”
\end{altverse}
\end{fnverse}

Tynemouth Castle is spelled “Tinmouth” in the ballad, just as it
is now pronounced in the North of England; it is, therefore,
probably, of Northumbrian origin. The answer is in Rox. ii. 17; also
printed by Brooksby.

\textsuperscript{b} It is difficult to account wholly for this, but it may be
attributed partially to the prejudice against the Scotch, who were
long viewed as interlopers, and somewhat to their broad dialect; for,
although they would naturally sing the airs of their country, I
cannot find that any attained popularity in England before the
Restoration, either by notices of dramatists and other writers, by
being used as ballad tunes, or by being found in print or manuscript.
I should say that one or two airs are the most that could be adduced.
The upper classes of both countries seem to have sung only scholastic
music, and the lower order of English had abundant ballad tunes of
their own, and were apparently loth to change them.
\end{dcfootnote}
\end{fixedpage}%610
\pagebreak

\begin{fixedpage}%611
\rectoheader

there. The “Extracts from the accounts of the Lords High
Treasurers of Scotland,” from the year 1474 to 1642, printed by Mr.
Dauney, shew that there were English harpers, lutenists, pipers, and
pipers with the drone, or bagpipers, among the musicians at the
Scottish Court, besides others under the general name of English
minstrels. Among the sweet songs said to be sung by the shepherds in
Wedderburn’s \textit{Complainte of Scotlande}, 1549, are several English still
extant (one composed by Henry VIII, taking precedence on the list);
and the religious parodies, such as in \textit{Ane Compendious Booke of Godly
and Spirituall Songs}, are commonly upon English songs and ballads.
English tunes have hitherto been found in every Scottish manuscript
that contains any Scotch airs, if written before 1730. There is, I
believe, no exception to this rule,—at least I may cite all those I
have seen, and the well-authenticated transcripts of others. They
include Wood’s manuscripts; the Straloch, the Rowallan, and the Skene
MSS.; Dr. Leyden’s Lyra-viol Book; the MSS. that were in the
possession of the late Andrew Blaikie; Mrs. Agnes Hume’s book, and
others in the Advocates’ Library; those in the possession of Mr.
David Laing, and many of minor note. Some of the Scotch manuscripts
contain English music exclusively. I have recently analyzed the
contents of Hogg’s \textit{Jacobite Relics of Scotland}, and find half the
songs in the first volume to have been derived from English printed
collections, but if the modern were taken away and only the old
suffered to remain, the proportion would be much larger. As Hogg took
these songs from Scotch manuscripts, his book shews the extent to
which the words of old English songs are still stored in Scotland.
The appendix of Jacobite songs, and those of the Whigs at the end of
the volume, are almost exclusively from these collections.

Before the publication of Ramsay’s \textit{Tea Table Miscellany}, the
“Scotch tunes” that were popular in England were mostly spurious, and
the words adapted to them seem to have been invariably so. Of this I
could give many instances, but it may suffice to quote one from \textit{A
second Tale of a Tub}, which being printed in 1715, is within nine
years of Ramsay’s publication. “Each party call for particular tunes\dots
the blue bonnets” (\ie, the Scotch) “had very good voices, but
being at the furthest end of the room, were not distinctly heard. Yet
they split their throats in hollowing out \textit{Bonny Dundee}, \textit{Valiant
Jockey}, \textit{Sawney was a dawdy lad}, [bonny lad?] and ’\textit{Twas within a
furlong of Edinborough town}.'”

\textit{Bonnie Dundee} commences thus:—

\settowidth{\versewidth}{Where gott’st thou the haver-meal bannock? [oatmeal cake]}
\begin{scverse}
\vleftofline{“}Where gott’st thou the haver-meal bannock? [oatmeal cake]\\
\vin Blind booby, canst thou not see?\\
Ise got it out of the Scotchman’s wallet,” \&c.
\end{scverse}
The subject of the ballad is “Jockey’s Escape from Dundee,” and
it ends, “Adieu to bonny Dundee,” from which the tune takes the title
of \textit{Adew Dundie} in the Skene manuscript, and of \textit{Bonny Dundee} in \textit{The
Dancing Master}. It first appeared in the latter publication in a
second appendix to the edition of 1686, printed in 1688. “Valiant
Jockey’s march’d away,” and “’Twas within a
\end{fixedpage}%611
\pagebreak

\begin{fixedpage}%612
\versoheader

furlong of Edinborough town,” are by D’Urfey; and “Sawney was a
\textit{bonny} lad”\textsuperscript{a} by P. A. Motteux, the tune by Purcell,

Songs in imitation of the Scottish dialect seem to have been
confined to the stage till about the years 1679 and 1680,\textsuperscript{b} when the
Duke of York, afterwards James II., was sent to govern Scotland,
pending the discussion on the Exclusion Bill in the Houses of
Parliament, The Whigs were endeavouring to debar him from succession
to the throne, as being a Roman Catholic, while the most influential
Scotch and the English loyalists, then newly named Tories, were as
warmly espousing his cause.

Among the ballad-writers, the royalists greatly preponderated,
and the Scotch were in especial favour with them. Mat. Taubman, the
city of London pageant-writer, was one of these loyal poets. He
published many songs in the Duke’s favour, which he afterwards
collected into a volume, with “An Heroic Poem,” on his return from
Scotland. Nat. Thompson, the printer, collected and published \textit{120}
Loyal Songs, which he subsequently enlarged to 180. Besides these,
there are songs extant on broadsides, with music, which are not
included in any collection. Occasional attempts at the Scottish
dialect are to be found in all these sources. Purcell, and other
musicians in the service of the court, readily set such songs to
music; indeed, from the time of the Exclusion Bill until he became
king, James seems to have had all the song-writers in his favour.

Perhaps the earliest extant specimen of a ballad printed in
Scotland, may also be referred to this period;—I mean by “ballad”
that which was intended to be sung, and not poetry printed on
broadsides, without the name of the tune, even though such may
sometimes have been called “ballets.” Of the latter we have specimens
by Robert Sempill, or Semple, printed in Edinburgh as early as 1570;
but, as a real ballad, intended to be sung about the country, as
English ballads were, I know none earlier than “The Banishment of
Poverty, by his R. H., J. D. A. [James, Duke of Albany], to the tune
of the \textit{Last Good Night}.” It is to be observed that this is to an
English tune, and so are many of the ballads that were printed in
Scotland, some being reprints of those published in London. Among
others in the possession of Mr. David Laing, are “A proper new Ballad
intituled The Gallant Grahames: To its own proper tune, \textit{I will away
and will not stay}.” This is a white-letter reprint of “An excellent
new Ballad entituled The Gallant Grahams \textit{of Scotland},” a Copy of
which is in the Roxburghe Collection,
iii. 380, to the same tune. “Bothwell Banks is bonny: Or a
Description of the new Mylne of Bothwell,” is to the English tune of
\textit{Who can blame my woe}. “The Life and bloody Death of Mrs. Laurie’s
Dog” is “to the tune \textit{The Ladies Daughter}” [of Paris properly]. See
Evans’s \textit{Old Ballads}. The above are on Scottish subjects, but there
are also reprints of the Anglo-Scottish, such as “Blythe Jockie,
young and gay,” (the tune of which is by Leveridge,) and “Valiant
Jockey’s march’d away,” before mentioned; as well as of purely

\begin{dcfootnote}
\textsuperscript{a} If not this, It must be “\textit{Jockey} was a dowdy lad,” a 
Scotch song by D’Urfey in \textit{The Campaigners}. There is 
a Sawney in that song, but he is the favoured lover. The 
music was composed by Mr. Wilkins. 

\textsuperscript{b} I do not include songs like ‘'Sing, home again, 
Jockey,’’(upon the defeat of the Scottish army,) or others 
written \textit{against} the Scotch, which may contain a few 
words in imitation of the dialect.

\end{dcfootnote}
\end{fixedpage}%612
\pagebreak

\begin{fixedpage}%613
\rectoheader

English ballads, like “Room, room for a Rover; or An innocent
Country Life prefer’d before the noisy clamours of a restless town.
To \textit{a new tune};”—

\settowidth{\versewidth}{Room, room for a rover,}
\begin{scverse}
\begin{altverse}
\vleftofline{“}Room, room for a rover,\\
London is so hot,” \&c.
\end{altverse}
\end{scverse}

The mixture of English music in Scotch collections is not without
inconvenience to the Scots themselves, for an essayist who intends to
write about Scottish music, must either be content to deal in
generalities, or he will be liable to the mistake of praising English
music where he intends to praise Scotch. Dr. Beattie, in one of his
published letters, says of the celebrated Mrs. Siddons, “She loves
music, and is fond of Scotch tunes, many of which I played to her on
the violoncello. One of these, \textit{She rose and let me in}, which you know
is a favorite of mine, made the tears start from her eyes, ‘Go on,’
said she, ‘and you will soon have your revenge;’ meaning that I
should draw as many tears from her as she had drawn from me” by her
acting. (Life of James Beattie,
LL.D., by Sir W. Forbes, ii. 139.) Dr. Beattie was evidently not
aware, that both the music and words of \textit{She rose and let me in}, are
English. Again, in one of his Essays,—“I do not find that any
foreigner has ever caught the true spirit of Scottish music;” and he
illustrates his remark by the story of Geminiani’s having blotted
quires of paper in the attempt to write a second part to the tune of
\textit{The Broom of Cowdenknow}s. This air is, to say the least, of very
questionable origin. The evidence of its being Scotch rests upon the
English ballad of \textit{The Broom of Cowdenknows}, for in other ballads to
the same air it is not so described; and Burton, in his \textit{Anatomy of
Melancholy}, quotes “O the broom, the bonny, bonny broom,” as a
“country tune.” The frequent misapplication of the term “Scotch,” in
English songs and ballads, has been remarked by nearly every writer
on Scottish music, and this air is not upon the incomplete scale,
which is commonly called Scotch. I am strongly persuaded that it is
one of those ballads which, like \textit{The gallant Grahams}, and many
others, became popular in Scotland because the subject was Scotch.
\textit{The Broom of Cowdenknows} is in the metre of, and evidently suggested
by, the older ballad of \textit{New Broom on Hill} (see p. 458). A copy of the
original Broom on Hill\textsuperscript{a} may even yet be discovered, or at least an
earlier copy of the tune, and thus set the question at rest.

It is not only by essayists that mistakes are made, for even in
historical works like “Ancient Scottish Melodies from a Manuscript of
the reign of James VI., with an introductory enquiry illustrative of
the History of the Music of Scotland, by William Dauney, F.S.A.,
Scot.,” airs which bear no kind of resemblance to Scottish music, are
claimed as Scotch. Mr. Dauney seems to have been a firm believer in
the authenticity of the collections of Scottish music, and to have
thought the evidence of an air being found in a Scotch manuscript
sufficient to prove its Scottish origin. In such cases dates were to
him of minor importance. Thus, \textit{Franklin is fled away}; \textit{When the King
enjoys his own again}; \textit{I pray you, love},

\begin{dcfootnote}
\textsuperscript{a} \textit{Broom on hill}, according to Laneham, was “ancient” 
in 1575. The three-part song of Charles the First’s reign, 
to which I have referred at p. 459, is subscribed “Bassus 
per T. C.” T. C. was perhaps the writer of that manuscript, 
or one of his intimate friends,—otherwise we might
expect the full name instead of initials.

\end{dcfootnote}
\end{fixedpage}%613
\pagebreak

\begin{fixedpage}%614
\versoheader

\textit{turn to me}; \textit{Macbeth}; \textit{The Nightingale}; \textit{The Milking-pail};
\textit{Philporter’s Lament}, and many others, are set down as airs of “which
Scotland may claim the parentage.” As to the Anglo-Scottish and
English Northern songs, at the very opening of his book Mr. Dauney
claims five in \textit{Pills to purge Melancholy} , without noticing Ritson’s
counter-statement as to two (yet appropriating them under those
names), or that a third was stated to be a country-dance tune in the
book he quotes. This is indeed driving over obstacles.

The manuscripts from which the “Ancient Scottish Melodies” are
derived, are known as the Skene Manuscripts, from having been in the
possession of the family of that name. They consist of seven small
books of lute music of uniform size, and are now bound in one. Mr.
Dauney admits that a portion of the airs are English, but follows the
Ramsay precedent in the title of his book. I have recently examined
these manuscripts with some care,\textsuperscript{a} and am decidedly of opinion, both
from the writing and from the airs they contain, that they are not,
and cannot be, of the reign of James VI. James VI. of Scotland and I.
of England died in 1625.

As to the sixth manuscript, which Mr. Dauney considers to be
“evidently the oldest of all,” the first fourteen airs in the fifth,
and the whole of the sixth, are, in my opinion, in the same
handwriting. The music is there written in the lozenge-shaped note,
which is nowhere else employed. Among the airs in the fifth, we find
\textit{Adieu, Dundee}, which was not included in \textit{The Dancing Master} before
the appendix of 1688; and Three Sheep-skins, an English country-dance
(\textit{not} a ballad tune), which first appeared in \textit{The Dancing Master} of
1698. In the sixth, “Peggy is over the sea with the Soldier,” which
derives its name from a common Aldermary churchyard ballad, to which,
I believe, no earlier date than 1710 can reasonably be assigned. It
is “The Gosport Tragedy: Peggy’s gone over the sea with the Soldier;”
commencing—

\settowidth{\versewidth}{“In Gosport of late there a damsel did dwell.”}
\begin{scverse}
“In Gosport of late there a damsel did dwell.”
\end{scverse}

When Mr. Dauney expressed his opinion that the sixth was the
oldest part, he was evidently deceived by the shape of the note; but
as round notes were used in manuscripts in the reign of Henry VIII.,
it must have been quite a matter of fancy whether the round or
lozenge should be employed one or two centuries later.

\begin{dcfootnote}
\textsuperscript{a} My attention has recently been drawn to these manuscripts,
which I had not seen for twenty years, from finding, in the course of
my attempts at chronological arrangement, that their supposed date
could not be reconciled with other evidence. I have hitherto quoted
the Skene MSS. as about 1630 or 1640, and many of the airs they
contain are undoubtedly of that date,—some, like those of Dowland and
the masque tunes of James I., unquestionably earlier. In Mr. Dauney’s
book, the airs are not published in the order in which they are found
in the manuscripts, and some airs (besides duplicates) are omitted.
The printed index is not very correct,—for instance, “Let never
crueltie dishonour beauty” is not included in it. The earliest
writing appears to be “Lady, wilt thou love me? “at the commencement
of Part II.; but all the remainder of that part seems to be a century
later. Pages 62 to 80 are blank. At the end of the first manuscript
are the words “Finis quod Skine,” which Mr. Dauney considers to be
the writing of John Skene, who
died in 1644. Independently of other evidence, the large number
of duplicates would shew the improbability of the collection having
been made for one person. For instance, “Horreis Galziard” is
contained in Parts I. and III.—“I left my love behind me,” in Parts
II. and III.—“My Lady Lauckian’s Lilt,” “Scerdustis,” “Scul-lione,”
and “Pitt on your shirt on Monday,” in Parts III. and V. “My Lady
Rothemais Lilt,” in Parts III and VI. “Blew Breiks,” in Parts III.
and VII. “I love my love for love again,” in Parts V. and VI.

This is not the only manuscript, English or Scotch, the age of
which I now find reason to doubt. Among the Scotch, that of Mr.
Andrew Blaikie, said to bear a date of 1692, (which I by no means
deny, although I did not observe it in the book when lent to me,)
cannot have been written before 1745. It contains “God save the
King,” and other airs not to be reconciled with the usually
attributed date.
\end{dcfootnote}
\end{fixedpage}%614
\pagebreak

\begin{fixedpage}%615
\rectoheader

The Scotch adhered to old notation\textsuperscript{a} longer than the English,
especially in writing music on six lines.\textsuperscript{b}

I leave it to Scottish antiquaries to determine, whether
corroborative evidence of the date of the manuscripts may not be
found among the titles of their own airs. Mr. Dauney even passed over
Leslei’s Lilt without a suspicion that it derived its name from the
Scotch general in the civil wars. A march\textsuperscript{c} and another air were
certainly named after him before the Restoration.

It is curious to mark the difference between English and Scotch
writers on the music of their respective countries; Dr. Burney, like
the fashionable Englishman, minutely chronicling the Italian operas
of his day, and hesitating not to misquote Hall, Hollinshcd, and
Hentzner, to get rid of the trouble of writing about the music of
England; and the Scotch sturdily maintaining the credit of
Scotland—some being intent rather upon putting forth fresh claims
than too nicely scrutinizing those already advanced, if they tell in
favour of their country.

It is time, however, that we should have one collection to
consist exclusively of Scottish music. Burns and George Thomson
confess in their published correspondence, to having taken any Irish
airs that suited them, and even in Wood’s \textit{Songs of Scotland}, the
publisher’s plan has been to include all the best and most popular
airs, and not to limit the selection to such as are strictly of
Scottish origin.

The separation of the English and Irish tunes from the Scotch in
these collections, was nominally attempted by Mr. Stenhouse in his
notes upon airs in Johnson’s \textit{Scots Musical Museum}. I say “nominally,”
for those notes are like historical novels,—wherever facts do not
chime in with the plan of the tale, imagination supplies the
deficiencies. Mr. Stenhouse’s plan was threefold,— firstly, to claim
every good tune as Scotch, that had become popular in Scotland;
secondly, to prove that every song of doubtful or disputed parentage
came to England from Scotland “at the union of the two crowns;” and,
thirdly, to supply antiquity to such Scotch airs as required it. All
this he accomplished in a way quite peculiar to himself. Invention
supplied authors and dates, and fancy inscribed the tunes in sundry
old manuscripts, where the chances were greatly against any one’s
searching to find them. If the search should be made, would it not be
made by Scotchmen? Englishmen care only for foreign music, and do
not trouble themselves about the matter; and will Scotchmen expose
what has been done from such patriotic motives? Upon no other ground
than this imaginary impunity, can I account for the boldness of Mr.
Stenhouse’s inventions.

Unfortunately for his fame, two of his own countrymen did not
think all this ingenuity necessary for the reputation of Scottish
music. Mr. David Laing, therefore, made a tolerably clear sweep of
his dates, and Mr. George Farquhar

\begin{dcfootnote}
\textsuperscript{a} I believe it was the retention of the old form of the 
letter “d” in the musical notation that deceived an acute 
Scotch antiquary and excellent judge of the age of literary 
manuscripts. In a portion of the tablature it has a stroke 
through the top (like the Anglo-Saxon letter which corresponds 
with our th), and this is also found in the title 
of “Lady, wilt thou love me?” which appears to be the 
oldest writing, and differing from any other, in the manuscripts. 

\textsuperscript{b} Witness Mrs. Agnes Hume’s hook, dated 1704.

\textsuperscript{c} I do not mean the tune which Oswald prints in the
second volume of his \textit{Caledonian Pocket Companion} under 
the name of \textit{Lasly’s March}, but the \textit{Lesleyes March} in 
Playford’s \textit{Musick’s Recreation on the Lyra Viol}, 1656,

\end{dcfootnote}
\end{fixedpage}%615
\pagebreak

\setlength{\fixedpagewidth}{380pt}
\begin{fixedpage}%616
\versoheader

Graham of his quotations from old musical manuscripts. The former
supposed Mr. Stenhouse “mistaken”—“deceived;” the variety of his
accomplishments was not to be discovered at once. The second
occasionally administered rebuke in more explicit language; but, to
the present day, the depths of Stenhouse’s invention have not been
half fathomed.

Some of the effects of his ingenuity will never be wholly
obviated. One class of inventions is very difficult to disprove,
where he fixes upon an author for a song, or makes a tale of the
circumstances under which it was written. Such evidence, as in the
case of \textit{She rose and let me in}, will not always be at hand to refute
him (ante p. 509 to 511), and much of this class of fiction still
remains for those who are content to quote from so imaginative a
source.

It is to be hoped that any who may henceforth quote from him will
give their authority, for he has sometimes been copied without
acknowledgement, and thus his fictions have been endorsed by
respectable names.\textsuperscript{a}

It is a pleasure to turn from such an annotator, to the editor of
Wood’s \textit{Songs of Scotland}, for, besides exposing a great number of
Stenhouse’s misstatings, he has given judgment with strict
impartiality wherever he felt called upon to exercise it in cases of
disputed nationality. It is only to be regretted that Mr. Graham’s
opinion upon the internal evidence of airs was not more frequently
expressed, and that any portion of Stenhouse’s imaginative notes
should have been incorporated in the work. Sometimes it is difficult
to distinguish between what is on the authority of Mr. Graham and
what of Stenhouse, without having a copy of his notes by our side;
but all I have had occasion to controvert \textit{originated} with the latter.

The following two specimens of Anglo-Scottish songs will suffice
as examples of that class of popular music of the olden time.

\begin{dcfootnote}
\textsuperscript{a}“Although Dauney’s \textit{Ancient Scottish Melodies} were printed in
1838, and Stenhouse’s \textit{Notes} issued in 1839 (after having been kept
for many years in Messrs. Blackwood’s cellars), it is evident that
Dauney had access to, and was one of those led into error by them. As
an instance, at p. 17 be says, “It was in the year 1680 when the
Scottish air, \textit{Katherine Ogie}, was sung by Mr. Abell, a gentleman of
the Chapel Royal, at his concert in Stationers’ Hall.” The date of
1680 is purely Stenhousian, and can only have been copied from the
following characteristic specimen of the \textit{Notes}:—“This fine old
\textit{Scottish} song, beginning, ‘As I went forth to view the plain,’ was
introduced and sung by Mr. John Abell, a gentleman of the Chapel
Royal, at his concert in Stationers’ Hall, London, \textit{in the year} 1680,
\textit{with great applause}. It was also printed with the music and words, by
an engraver of the name of Cross, as a single-sheet song, \textit{in the
course of that year, a copy of which is now lying before me}.” In the
first place, Cross did not engrave in 1680, and the single-sheet
song, “Bonny Kathern Oggy, as it was sung by Mr. Abell at his consort
in Stationers’ Hall,” bears no date. Abell was a gentleman of the
Chapel Royal during the latter part of the reign of Charles II. and
the whole of that of James II. Having turned Papist when James became
King, he quitted England at the Revolution of 1688, but was permitted
to return by William III., towards the close of the year 1700. From
\textit{that} time, being without any fixed employment, and having acquired
great repute as a singer, he occasionally gave public concerts, the
first,
of which I find any announcement, having taken place at Covent
Garden, on the 29th Dec., 1702. Stenhouse, to make his story
complete, tells us that Abell died “about the year 1702,” although
Hawkins (from whom lie was copying so mnch of the story as suited his
purpose) says that “about the latter end of \textit{Queen Anne’s reign}, Abell
was at Cambridge with his lute.”

Now, why all this invention? It was to get rid of the fact that
the earliest known copy of the tune is in the Appendix to \textit{The Dancing
Master} of 1686, under the title of “\textit{Lady Catherine Ogle}, a new
Dance.’’ D’Urfey wrote the first song to it, “Bonny Kathern Loggy,”
commencing, “As I came down the Highland town.” This is contained in
the \textit{Pills} and in \textit{The Merry Musician or a Cure for the Spleen}, i. 224
(1716). The latter publication includes also, the “New song to the
tune of \textit{Katherine Loggy},” commencing, “As I walk’d forth to view the
plain” (i. 295), which Ramsay, after making some alterations, printed
in the \textit{Tea Table Miscellany}. The following is the first stanza of
what Stenhouse terms the “fine old \textit{Scottish} song,” sung by Abell:—

\settowidth{\versewidth}{With May’s sweet scent to chear my brain,}
\begin{fnverse}
\begin{altverse}
“As I went forth to view the spring.\\
Upon a morning early,\\
With May’s sweet scent to chear my brain,\\
When flowers grow fresh and fairly;\\
A vary pratty maid I spy’d,\\
Sha shin’d tho’ it was foggy,\\
I ask’d her name, Sweet sir, sha said,\\
My name is Kathern Oggy.
\end{altverse}
\end{fnverse}
\end{dcfootnote}
\end{fixedpage}%616
\pagebreak

\setlength{\fixedpagewidth}{360pt}
\begin{fixedpage}%617
\rectoheader

\musictitle{FAIREST JENNY.}

This is included in Scotch collections, under the name of \textit{Fife
and a’ the lands about it}. The words were written by P. A. Motteux,
and the music by Samuel Akeroyd. It was first printed in \textit{The
Gentleman’s Journal}, of Jan., 1691-2, under the title of “Jockey and
Jenny, a Scotch song set by Mr. Akeroyde.” In the letter which
precedes it, Motteux says, “I do not doubt but you will like the
tune, and that is generally the more valuable part of our \textit{English
Scotch} songs, it being improper to expect a refin’d thought and
expression in a plain light humour.”

It was also printed in \textit{The Banquet of Music}, Book 6, 1692; and in
\textit{Apollo’s Banquet}, 7th edit., 1693; in \textit{Pills to purge Melancholy} , vol.
i. of early editions, and vol. iii. of the last.

The name of \textit{Fife and a’ the lands about it}, is taken from the
first line of the second stanza.

\settowidth{\versewidth}{hundred thousand pound.}
\begin{dcverse}
\vleftofline{\textit{Jockey}. }For aw Fife, and lands about it \\
Ize not yield thus to be bound. \\
\vleftofline{\textit{Jenny}. }Nor I lig by thee without it \\
For twa hundred thousand pound. \\
\vleftofline{\textit{Jockey}. }Thou wilt die if I forsake thee.\\
\vleftofline{\textit{Jenny}. }Better die than be undone.\\
\vleftofline{\textit{Jockey}. }Gin ’tis so, come on, Ize taak thee,\\
’Tis too cauld to lig alone.
\end{dcverse}

\end{fixedpage}%617
\pagebreak

\setlength{\fixedpagewidth}{370pt}
\begin{fixedpage}%618
\versoheader

\musictitle{SAWNEY WAS TALL AND OF NOBLE RACE.}

This is one of Tom D’Urfey’s songs, in his comedy of \textit{The Virtuous
Wife}, 4to., 1680. I have not seen any copy bearing the name of a
composer; but, as other music in this play (such as “Let traitors
plot on,” and the chorus, “Let Cæsar live long”) was composed by
Farmer, this may also be reasonably attributed to him.

Playford printed it, in 1681, in the third book of his \textit{Choice
Ayres}, as “a Northern song;” but he also printed \textit{She rose and let me
in}, in the fourth book of the same collection, as a Northern song,
although the music was undoubtedly composed by Farmer, and D’Urfey
was, as in this case, the author of the words. The fact of their
appearing in that collection is sufficient to prove that they were
compositions of the time, for not only are the \textit{Choice Ayres}
professedly “the newest ayres and songs, sung at Court and at the
publick theatres, \textit{composed} by several gentlemen of his Majesty’s
musick, and others,” but, also, Playford, in reference to this very
third book, expresses great indignation that any of the songs should
be thought to be ballad-tunes. That they became so subsequently, was
beyond his control.

Two of Farmer’s airs have already been printed in this volume;
and others which became popular on the stage, may yet be traced to
him. Farmer was an excellent musician and particularly successful as
a melodist. He was originally one of the waits of London, which may
account for his having paid more attention to rhythmical tune than
others who were educated in the Chapels Royal, or the Cathedral
schools. In 1684, after having attained some repute as a composer for
the theatres, he was admitted to the degree of Bachelor in Music at
the University of Cambridge. He died at an early age, and the
estimation in which he was held by his contemporaries may be judged
by the elegy which was written upon his death by Tate, to which
Purcell composed the music.\textsuperscript{a}

\textit{Sawney was tall} soon became popular as a penny ballad, and some
other ballads and political squibs were written to the tune.

In the Roxburghe Collection, ii. 223, is a sequel to \textit{Sawney was
tall}, entitled “Jenny’s Answer to Sawny, Wherein Love’s cruelty is
requited; or, The inconstant Lover justly despised. Being a relation
how Sawney being disabled and turn’d out of doors by the Miss of
London, is likewise scorned and rejected by his Country Lass, and
forced to wander where he may,” \&c. “To the tune of
\textit{Sawney will ne’er be my love again}.” Printed for P. Brooksby, at
the Golden Ball, \&c. It begins—.

\settowidth{\versewidth}{When Sawney left me he.had store of gilt,}
\begin{dcverse}
\begin{altverse}
\vleftofline{“}When Sawney left me he.had store of gilt, \\
But he hath spent it in London town, \\
And now is return’d to his sun-burn’d face, \\
His own dear joy in a russet gown: \\
He’s come for another sark and band,\\
And coakses me for more of my coin, \\
But Ise, guid faith, shall hold my hand,\\
For Sawney shall never more be mine.”
\end{altverse}
\end{dcverse}

\begin{dcfootnote}
\textsuperscript{a} “The “Elegy upon the death of Mr. Thomas Farmer, B.M., is printed in
the second volume of the \textit{Orpheus Britannicus}. As Dr, Burney most
strangely omits all mention of Farmer, it is here subjoined:—

\settowidth{\versewidth}{What makes the spring retire, and groves their songs decline?}
\begin{fnverse}
\textit{“}Young Thirsis’ fate, ye hills and groves, deplore! \\
Thirsis, the pride of all the plains,\\
The joy of nymphs and envy of the swains,\\
The gentle Thirsis is no more!\\
What makes the spring retire, and groves their songs decline?\\
Nature for her lov’d Thirsis seems to pine;\\
Whose artful strains and tuneful lyre\\
Made the spring bloom, and did the groves inspire.\\
What can the drooping sons of art\\
From this sad hour impart\\
To charm the cares of life and ease the lover’s smart? \\
While thus in dismal notes we mourn \\
The skilful shepherd’s urn,\\
To the glad skies his harmony he bears,\\
And as he charmed earth, transports the spheres.”
\end{fnverse}
\end{dcfootnote}
\end{fixedpage}%618
\pagebreak

\setlength{\fixedpagewidth}{370pt}
\begin{fixedpage}%619
\rectoheader

In the same Collection, ii. 254, is another, printed by Brooksby,
“The Poet’s Dream, Or the great outcry and lamentable complaint of
the land against Bayliffs and their Dogs: wherein is expressed their
villanous out-rages to poor men, with a true description of their
knavery and their debauch’d actions, prescribed and presented to the
view of all people. To the tune of \textit{Sawney}, \&c.” The first line is
“As I lay slumbering in a dream.”

Among the political ballads are (Rox. ii. 109) “The Disloyal
Favourite; or The Unfortunate States-man:

\settowidth{\versewidth}{To him that wears a proud ambitious mind. }
\begin{dcverse}
Who seeks by fond desire for to climb, \\
May chance to catch a fall before his time, \\
For Fortune is as fickle as the wind \\
To him that wears a proud ambitious mind. 
\end{dcverse}

Tune of \textit{Sawney will n’er be my love again}.” It begins—

\indentpattern{01010001}
\settowidth{\versewidth}{For Tommy will ne’er be belov’d again.”}
\begin{dcverse}
\begin{patverse}
\vleftofline{“}Tommy was a Lord of high renown, \\
And he was rais’d from a low degree; \\
He had command ore every town; \\
There was never one so great as he: \\
But he, like an ungrateful wretch,\\
Did set his conscience on the stretch,\\
And now is afraid of Squire Ketch,\\
For Tommy will ne’er be belov’d again.”
\end{patverse}
\end{dcverse}

This is on some nobleman who was charged with being “concerned
with France,” and “some say concern’d in the plot.” Printed for W.
Thackeray, T. Passinger, and W. Whitwood.

In Mr. Gutch’s collection is a broadside entitled “The Loyal
Feast designed to be kept in Haberdashers’ Hall on Friday, 21 April,
1682, by His Majesty’s most loyal true blue Protestant subjects, and
how it was defeated. To the tune of \textit{Sawney will never be my love
again}.” London, printed for Allan Banks, 1682. This commences—

\settowidth{\versewidth}{He broached his taps, and it ran apace}
\begin{dcverse}
\begin{altverse}
\vleftofline{“}Tony was small but of noble race, \\
And was beloved of every one; \\
He broached his taps, and it ran apace\\
To make a solemn treat for all:”
\end{altverse}
\end{dcverse}
and it was reprinted, with another to the same air, in N.
Thompson’s collection of 180 \textit{Loyal Songs}, 1685 and 1694.

The tune is found in the \textit{Choice Ayres} and \textit{Loyal Songs}, above
quoted; in \textit{The Dancing Master}, from 1686 to 1725; in \textit{Apollo’s
Banquet}, 1690 and 1693; in the ballad-operas of \textit{Polly}, \textit{The Village
Opera}, \textit{The Devil to pay}, \textit{The Chambermaid}, \&c. The words are
contained in D’Urfey’s \textit{New Collection of Songs and Poems}, 8vo., 1683;
and both words and music are in every edition of \textit{Pills to purge
Melancholy}.

In all the above-named works, the tune takes its name from
D’Urfey’s song, except in \textit{The Dancing Master}, where it is named
\textit{Sawney and Jockey},\textsuperscript{a} but evidently by mistake. ’ It is nowhere called
\textit{Corn riggs are bonny}, until after the publication of Allan Ramsay’s
song, commencing, “My Patie is a lover gay,” in the \textit{Tea Table
Miscellany}. Ramsay does not say that his song is “to the tune of”
\textit{Corn riggs are bonny}, but gives that title to his song.

Stenhouse would have us believe that there was “a much older
Scottish song” of “Corn riggs” to this tune than Ramsay’s, but the
four lines he gives are evidently a parody of the four last of
Ramsay’s song.\textsuperscript{b} He does not condescend

\begin{dcfootnote}
\textsuperscript{a} \textit{Sawney and Jockey} is another song of D’Urfey’s in his play of
\textit{The Royalist}. It commences with the line, “Twa bonny lads were Sawney
and Jockey,” and it was printed as a penny ballad by Brooksby, in
1682, with the tune. A copy is in the Roxburghe Collection, iii, 913,
“The Scotch Lasses Constancy; or, Jenny's Lamentation for the death
of Jockey, who for her sake was unfortunately killed by Sawney in a
duel. Being a most pleasant new
song to \textit{a new tune}” There are two songs to the tune in 180 \textit{Loyal
Songs}, pages 282 and 365; and Mat. Taubman’s “Jockey, away man”
(printed with his “Heroic Poem”) is to the same.

\textsuperscript{b} This is one of Stenhouse’s favorite remedies for deficient
evidence of antiquity. He produces some “original words,” stating
them to be of the age required to meet the necessities of the case,
but they rarely tally with
\end{dcfootnote}

\end{fixedpage}%619
\pagebreak

\setlength{\fixedpagewidth}{360pt}
\begin{fixedpage}%620
\versoheader

to give any proof of this antiquity, but tells us that “the tune
appears in Craig’s Collection, in 1730,” and that “Craig was a very
old man when he published his collection.”

Whether Craig was old or young I will not dispute, but he
certainly took the titles of the tunes in his collection from the \textit{Tea
Table Miscellany}. Out of thirty-five tunes that he published,
twenty-nine agree with the names in that work, and this is the total
number that could agree, for there are no songs in it to the
remaining six.


\begin{dcfootnote}
information derived from other sources. Francis Semple, of
Beltrees, is one of his favorite scapegoats in these cases. He gives
him the credit, among other songs, of \textit{Maggie Lauder}. Now, in the
ballad-opera of \textit{The Beggar’s Wedding}, 2nd edit., 8vo., 1729, it is
called “Moggy Lawther on a day,” which does not at all agree with
the song
of which Francis Semple is the \textit{supposed} author. Again, as to
\textit{Logan Water}, in \textit{Flora}, 8vo., 1729, it is named “The Logan water is
so deep,” which is not at all like the words Stenhouse gives. It
would be easy to multiply instances of this kind.
\end{dcfootnote}
\end{fixedpage}%620
\pagebreak

\renewcommand\versoheadertext{english song and ballad music.}
\renewcommand\rectoheadertext{reign of queen anne to george ii.}

\begin{fixedpage}%621
\rectoheader

\vfill

\headingfour{REIGN OF QUEEN ANNE TO GEORGE II.}
\headingfive{(1702 to 1745.)}
\centerrule

Tom D’Urfey, to whose songs I have so frequently had occasion to
refer in the preceding pages, was a poet and dramatist who flourished
about 1675 to 1720. His father was a Protestant, who fled from
Rochelle before the memorable siege in 1628, and settled at Exeter,
where, in 1649, Tom was born.\textsuperscript{a} He was intended for the law, a
profession very uncongenial to his own taste, and for which he was
disqualified by an impediment in his speech; but this did not affect
him in singing. In his 27th year he produced his comedy, \textit{The Fond
Husband, or The Plotting Sisters},\textsuperscript{b} which “was honoured with the
presence of King Charles II. three of the first five nights,” and had
a long-continued success. It was frequently revived, and three
editions were printed during the author’s life, viz: in 1678, 1685,
and 1711. Tom tried his hand at tragedy, but the bombast of his
heroic verse met with little encouragement. The success of \textsuperscript{The
Plotting Sisters} was the turning point of his fortune, by leading to
his introduction at Court. It was well known that Charles II. liked
no music to which he could not beat time; and, as the rhythm most
easily marked was that of dance and ballad tunes, D’Urfey
accommodated his songs to the royal taste by writing them to airs of
that class, or in such metres as might enable composers to adopt a
similar style of composition. Before his time, it had been a rule
with English poets, especially the greater, to select metres that
should effectually prevent their songs being sung to ballad tunes;
and for that reason, those songs are rarely, if ever, heard in the
present day. The exceptions are almost invariably those to which
music has been composed at comparatively recent dates. Since
D’Urfey’s time, English poets have generally pursued the old course,
but the Scotch have acted otherwise. They sang D’Urfey’s
songs,—adopted many of the tunes,— their poets wrote other words to
them, and continue, to the present day, to write to airs of a similar
class. “Roy’s Wife of Aldivalloch,” “Caller Herrin,” “Auld lang
syne,” and numberless others, are taken from books of Scottish dance
music, printed during the latter half of the last century; and many
of the most pathetic airs were originally quick tunes of the same
kind. If English poets wish their songs to endure, the safest course
will be to follow the example of Tom D’Urfey, and of the Scotch.
Dibdin’s sea songs are already fading from memory, because he
composed music to them, instead of writing to airs which had stood
the test of time.

\begin{dcfootnote}
a His mother was probably an Englishwoman, for Tom 
Brown addresses D’Urfey in one of his satires, as “Thou 
cur! half French, half English breed.” 

b It was licensed June 15th, 1676, and, according to the 
\textit{Biographia Dramatica}, published in that year. After
D’Urfey’s death it was revived in 1726, 1732, and 1740.

\end{dcfootnote}
\end{fixedpage}%621
\pagebreak


\begin{fixedpage}%622
\versoheader

D’Urfey printed five collections of his own songs, and many of
the same were afterwards included in \textit{Pills to purge Melancholy} . In
the first two collections are various songs which were “sung to the
King indeed, wherever Charles went, D’Urfey seems to have been
engaged to entertain him. “Quoth John to Joan,” or “I cannot come
every day to woo “The Spinning Wheel,” (“Upon a sunshine summer’s
day”); “Pretty Kate of Edinburgh and “Advice to the City,” are among
those which were sung to the King at Windsor. The last commences— “

\settowidth{\versewidth}{Remember, ye Whigs, what was formerly done,}
\begin{scverse}
\vleftofline{“}Remember, ye Whigs, what was formerly done,\\
Remember your mischiefs in \textit{Forty} and \textit{One};”
\end{scverse}
and D’Urfey tells us, in the \textit{Pills}, that the King held one part
of the paper, and sang it with him. Others were heard at Newmarket,
at Winchester, “at his entertainment at my Lord Conway’s,” and one
“sung to the King and Queen, upon Sir John Moor’s being chosen Lord
Mayor.”

D’Urfey was one of those who wrote panegyrics upon James, when
Duke of York, and congratulatory verses upon his return from
Scotland. In the preface to the \textit{Pills}, he boasts of having “performed
some of his own things before their Majesties, King Charles II., King
James, King William, Queen Mary, Queen Anne, and Prince George of
Denmark and that, on such occasions, he never quitted them “without
happy and commendable approbation.” He also wrote a “Vive le Roy” for
George the First, and “A new song on his happy accession to the crown;”
but Tom was then grown old, and we have no proof of his having been
in favour with that Monarch. Moreover, the King could not have
approved, if he knew, of a song which D’Urfey is said to have written
upon his mother, the Princess Sophia, Electress and Duchess Dowager
of Hanover. This was to please Queen Anne, by complimenting her upon
her youth at the expense of the Princess, who was next heir to the
crown, and no great favorite with Queen Anne, as excluding her
brother from the succession. The Queen is said to have given D’Urfey
fifty guineas for singing it.

\indentpattern{0202}
\settowidth{\versewidth}{She could not sustain such a trophy,}
\begin{dcverse}
\begin{patverse}
\vin\vin \vleftofline{“}The crown’s far too weighty \\
For shoulders of eighty, \\
She could not sustain such a trophy, \\
Her hand, too, already 
\end{patverse}

\indentpattern{22206}
\begin{patverse}
\vin\vin Has grown so unsteady\\
She can’t hold a sceptre;\\
So Providence kept her\\
Away,—poor old Dowager Sophy.”\\
Hone’s \textit{Table Book}, p. 560.
\end{patverse}
\end{dcverse}

A very amusing sketch of the life of D’Urfey will be found in
\textit{Household Words}, in which the writer quotes a note to \textit{The Dunciad}, to
prove that Tom was the last English poet who appeared in the streets
attended by a page. The popularity of his songs in the country is
alluded to by Pope, in a letter dated April 10, 1710. He says:—

\begin{quotation}
“I have not quoted one Latin author since I came down, but have
learned without book a song of Mr. Thomas D’Urfey’s, who is your only
poet of tolerable reputation in this country. He makes all the
merriment in our entertainments, and, but for him, there would be so
miserable a dearth of catches, that I fear they would put either the
parson or me upon making some for ’em. Any man, of any quality, is
heartily welcome to the best topeing-table of our gentry, who can
roar out some rhapsodies of his works.\dots Alas, sir, neither you
with your Ovid, nor I with my
\end{quotation}
\end{fixedpage}%622
\pagebreak

\begin{fixedpage}%623
\rectoheader

\begin{quotation}
Statius, can amuse a whole hoard of Justices and extraordinary
’Squires, or gain one hum of approbation, or laugh of admiration! ‘These things,’ 
they would say, ‘are
too studious, they may do well enough with such as love reading,
but give us your ancient poet, Mr. D’Urfey.” (\textit{Pope's Literary
Correspondence}, Curll, i. 267.)
\end{quotation}


The secret of D’Urfey’s popularity as a song-writer, lay in his
selection of the tunes. He trenched upon the occupation of the
professed ballad-writers, by adopting the airs which had been their
exclusive property, and by taking the subjects of their ballads;
altering them to give them as his own. If the reader will compare
Martin Parker’s “Milkmaid’s Life” with D’Urfey’s “Bonny Milkmaid”
(ante pp. 295, 297), he will see how these transformations were
effected; and there are many similar examples in the \textit{Pills}.

Perhaps no man was ever so general a favorite with his
contemporaries as Tom D’Urfey. His brother poets pleaded for him in
his old age, and, by their good offices and those of the actors, he
was rescued from the effects of the improvidence which has been
proverbial with men of his class. Steele and Addison were his great
friends, and equally urged his claims upon the public. Addison, on
the occasion of a play to be acted for D’Urfey’s benefit, wrote in
these words:—“He has made the world merry, and I hope they will make
him easy as long as he stays among us. This I will take upon me to
say, they cannot do a kindness to a more diverting companion, or a
more cheerful, honest, good-natured man.”

The Londoner who enters St. James’s Church from Jermyn Street
will see a stone with this inscription:—“Tom D’Urfey: Dyed Feb\textsuperscript{ry} y\textsuperscript{e}
26th, 1723.” The stone has been removed to the back of the church,
for within my recollection, it stood by the principal entrance. The
following “Epitaph upon Tom D’Urfey” is from \textit{Miscellaneous Poems by
several hands}, i. 6. 1726:—

\settowidth{\versewidth}{Grieve, Reader, grieve, that he too soon grew old,—}
\begin{scverse}
\vleftofline{“}Here lyes the Lyrick, who, with tale and song.\\
Did life to three score years and ten prolong;\\
His tale was pleasant and his song was sweet.\\
His heart was cheerful—but his thirst was great.\\
Grieve, Reader, grieve, that he too soon grew old,—\\
His song has ended, and his tale is told.”
\end{scverse}

The only great use which had been made of old tunes by the upper
classes before D’Urfey’s time (except for dancing) was for political
songs or lampoons, and they were continuously employed for those
purposes to the middle of the last century, and occasionally at later
dates. Lady Luxborough says in a letter to Shenstone, “It is the
fashion for every body to write a couplet to the same tune (viz., an
old country dance) upon whatever subject occurs to them,—I should say
upon whatever \textit{person}, with their names to it. Lords, gentlemen,
ladies, flirts, scholars, soldiers, divines, masters, and misses, are
all authors upon this occasion and also the objects of each other’s
satire.” (\textit{Monthly Review}, liv., 62.)

In the petition of Thomas Brown, by Sir Fleetwood Shepherd, he
thus alludes to their frequent use in his day:—
\end{fixedpage}%623
\pagebreak

\begin{fixedpage}%624
\versoheader

\settowidth{\versewidth}{May hang up themselves and their harps on the willows;}
\begin{scverse}
\vleftofline{“}E’en D’Urfey himself, and such merry fellows,\\
That put their whole trust in tunes and twangdillos,\\
May hang up themselves and their harps on the willows;\\
For, if poets are punished for libelling-trash,\\
Jo. Dryden, at sixty, may yet fear the lash.”
\end{scverse}

Political songs were mainly kept alive by the mug-houses in
London and Westminster, and many of the songs sung at those clubs
were afterwards collected and published. The author of \textit{A Journey
through England} in 1724,” says,—

“In the city of London, almost every parish hath its separate
club, where the citizens, after the fatigue of the day is over in
their shops and on the Exchange, unbend their thoughts before they go
to bed. But the most diverting or amusing of all were the mug-house
clubs in Long Acre, Cheapside, \&c., where gentlemen, lawyers, and
tradesmen, used to meet in a great room, seldom under a hundred.

“They had a president, who sate in an armed chair some steps
higher than the rest of the company, to keep the whole room in order.
A harp played all the time at the lower end of the room; and every
now and then one or other of the company rose and entertained the
rest with a song, and (by the by) some were good masters. Here was
nothing drank but ale, and every gentleman had his separate mug,
which he chalked on the table where he sate, as it was brought in;
and every one retired when he pleased, as from a coffee-house.

“The rooms were always so diverted with songs, and drinking from
one table to another to one another’s healths, that there was no room
for anything that could sour conversation.

“One was obliged to be there by seven to get room, and after ten
the company were for the most part gone.

“This was a winter's amusement, agreeable enough to a stranger
for once or twice, and he was well diverted with the different
humours when the mugs overflow.

“On King George’s accession to the throne, the Tories had so much
the better of the friends to the Protestant succession, that they
gained the mobs on all public days to their side. This induced this
set of gentlemen to establish mug-houses in all the corners of this
great city, for well-affected tradesmen to meet and keep up the
spirit of loyalty to the Protestant succession, and to be ready upon
all tumults to join their forces for the suppression of the Tory
mobs. Many an encounter they had, and many were the riots, till at
last the Parliament was obliged by a law to put an end to this city
strife; which had this good effect, that upon the pulling down of the
mug-house in Salisbury Court, for which some boys were hanged on this
Act, the city has not been troubled with them since.” (Malcolm’s
\textit{Manners and Customs}, p. 532.)

Political songs seem to have been the only kind of poetry in
general favour, after the reign of Queen Anne. Horace Walpole writes
to Richard West, in 1742,—

“’Tis an age most unpoetical; 'tis even a test of wit, to dislike
poetry: and though Pope has half a dozen old friends that he has
preserved, from the taste of last century, yet, I assure you, the
generality of readers are more diverted with any paltry prose answer
to old Marlborough’s Secret History of Queen Mary’s Robes. I do not
think an author would be universally commended for any production in
verse unless it were an Ode to the Secret Committee, with rhymes of ‘liberty and property,’
‘nation and administration.’” (\textit{Correspondence}, i. 100.)

\end{fixedpage}%624
\pagebreak

\begin{fixedpage}%625
\rectoheader

The cultivation of music among gentlemen began to decline in the
reign of Charles II., slowly but progressively. The style of music in
favour in his day required less cultivation than the contrapuntal
part-writing of earlier times. Playford remarks that “of late years
all solemn and grave music has been laid aside, being esteemed too
heavy and dull for the light heels and brains of this nimble and
wanton age.” Salmon, writing in 1672, attributes its decline to the
intricate and troublesome nature of the clefs, and says, that “for
ease sake, many gentlemen gave themselves over to whistling upon the
flageolet, and fiddling upon the violin, till they were so rivalled
by their lacquies and barbers’ boys that they were forced to quit
them, as ladies do their fashions when the chambermaids have
inherited their old clothes.” (\textit{Essay to the Advancement of Music}, p.
36.)

Among ladies, the cultivation seems to have remained in nearly
the same state as before. In “The Levellers: A Dialogue between two
young ladies concerning Matrimony,” (4to., 1703), Politica, who is a
tradesman’s daughter, describing her education at a boarding school,
says, she “learned to sing, to play on the base-viol, virginals,
spinnet, and guitair.” Here we find the base-viol still in use by
ladies; and again, in Vanbrugh’s play, \textit{The Relapse}, “To prevent all
misfortunes, she has her breeding within doors; the parson of the
parish teaches her to play on the base-viol, the clerk to sing, her
nurse to dress, and her father to dance.” (Act i., sc. 1.)
Nevertheless, some opposition to its use had existed long before, for
Middleton, in his \textit{Roaring Girl}, says, “There be a thousand close
dames that will call the viol an unmannerly instrument for a woman.”

The dancing schools of London are described by Count Lorenzo
Magalotti, on his visit to England with Cosmo, III. Grand Duke of
Tuscany, in 1669. He says, “they are frequented both by unmarried and
married ladies, who are instructed by the master, and practise, with
much gracefulness and agility, various dances after the English
fashion. Dancing is a very common and favorite amusement of the
ladies in this country; every evening there are entertainments at
different places in the city, at which many ladies and citizens’
wives are present, they going to them alone, as they do to the rooms
of the dancing masters, at which there are frequently upwards of
forty or fifty ladies. His Highness had an opportunity of seeing
several dances in the English style, exceeding well regulated, and
executed in the smartest and genteelest manner by very young ladies,
whose beauty and gracefulness were shewn off to perfection in this
exercise.” (p. 319.) And again, “He went out to Highgate to see a
children’s hall, which, being conducted according to the English
custom, afforded great pleasure to his Highness, both from the
numbers, the manner, and the gracefulness of the dancers.”

The English had long been celebrated for their dancing. “In
saltatione et arte musicâ excellunt,” says Hentzner, describing us in
1598; and while a man might hope to become Lord Chancellor by good
dancing, without being bred to the law (like Sir Christopher Hatton),
it was certainly worth while to endeavour to excel. Fletcher, in the
opening scene of his \textit{Island Princess}, to depict forcibly the pleasure
that a certain prince took in the management of a sailing boat,
\end{fixedpage}%625
\pagebreak

\begin{fixedpage}%626
\versoheader

likens it to the delight which the Portuguese or Spaniards have
in riding great horses, the French in courteous behaviour, or “the
\textit{dancing English} in carrying a fair presence.” In 1581, according to
Barnaby Rich, the dances in vogue were measures,\textsuperscript{a} galliards, jigs,
brauls, rounds, and hornpipes. In 1602, the Earl of Worcester writes
to the Earl of Shrewsbury, “We are frolic here in Court; much dancing
in the Privy Chamber of \textit{country dances} before the Queen’s Majesty,
who is exceedingly pleased therewith.” (Lodge’s \textit{Illustrations of
British History},
ii. 578.) In the reign of James I., Weldon, sneering at
Buckingham’s kindred, observes, that it was easier to put on fine
clothes than to learn the French dances, and therefore that “none but
country dances” must be used at Court. This was not quite correct,
for although country dances were most in fashion, others were not
excluded. At Christmas, 1622-3, after the masque, “the Prince”
(afterwards Charles I.) “did lead the measures with the French
Embassador’s wife. The measures, braules, corrantos, and galliards,
being ended, the masquers with the ladies did daunce two country
dances, where the French Embassador’s wife and Mademoysal St. Luke
did daunce.” (Malone, from a MS. in Dulwich College.) In the reigns
of Charles II. and James II., country dances continued in much the
same use. They were the merriment after the first formalities of the
evening had worn off. In \textit{The Mysteries of Love and Eloquence, or The
Arte of Wooing and Complimenting}, by Edward Philips (Milton’s
nephew), there is a chapter on “The Mode of Balls,” which opens with
the following speech from the dancing master:—“Come, stir yourselves,
maidens, ’twill bring a fresh colour into your cheeks; rub hard, and
let the ladies see their faces in the boards,” \&c.; to which Bess,
who has not yet been properly tutored, replies, “And, by the mass,
that will I do, and make such fine drops and curtsies in my best
wastecoat, that they shall not chuse but take notice of me; and Sarah
shall dance a North-country jigg before ’em, too: I warrant it will
please the ladies better than all your French whisks and frisks. I
had rather see one freak of jolly milkmaids than all that will be
here to-night.” After some directions as to what should be done, the
dancing master says, “Ladies, will you be pleased to dance a country
dance or two, for ’tis that which makes you truly sociable, and us
truly happy; being like the chorus of a song, where all the parts
sing together.”

I have mentioned more particularly the subject of country dances,
because the fashion of dancing our national dance, has extended, at
various times, to every court in Europe. Yet we English have so great
a mania for catching at the first foreign word that resembles our
own, and immediately settling that ours must have been derived from
it, that, let but one person propose such a derivation, there will
always be plenty to follow, and to vouch for it upon their own
responsibility. From this the country dance has not escaped.

I cannot tell to whom the brilliant anachronism of deriving it
from “Contredanse” is due, for, although asserted very positively by
three contemporaneous

\begin{dcfootnote}
\textsuperscript{a} The measure was a grave and solemn dance, with 
slow and measured steps, like the minuet. To \textit{tread} a 
measure was the usual term, like to \textit{walk} a minuet. Sir 
John Davies says,—
\settowidth{\versewidth}{Are only spondees, solemn, grave, and slow.}
\begin{fnverse}
\vleftofline{“}Yet all the feet whereon these measures go\\
Are only spondees, solemn, grave, and slow.
\end{fnverse}

\end{dcfootnote}
\end{fixedpage}%626
\pagebreak

\begin{fixedpage}%627
\rectoheader

modern writers, no one of the three condescends to give other
authority than his own.

The late John Wilson Croker, in his \textit{Memoirs of the Embassy of
Marshall de Bassompierre to the Court of England} in 1626, says, in a
note: “Our Country Dances are a corruption in name, and a
simplification in figure, of the French Contredanse.” Mr. De Quincey,
in his \textit{Life and Manners}, and the late Dr. Busby, in his \textit{Dictionary of
Music}, tell us the same. The discovery was not made when Weaver and
Essex wrote their \textit{Histories of Dancing},\textsuperscript{a} in 1710, and 1712, nor when
Hawkins published his \textit{History of Music}, in 1776. French etymologists
have been equally in the dark, for they have reversed the position.
“Ce mot [Contredanse] paroît venir de l’Anglois, \textit{country-danse}, danse
de campagne; en effet, c’est au village sur-tout que l’on aime à se
réunir et que l’on préfère les plaisirs partagés. Le grave menuet,
qui n’emploie que deux personnes, et qui ne laisse aux spectateurs
d’autre occupation que celle d’admirer, n’a pu prendre naissance que
dans les villes où l’on danse pour amour-propre. Au village l’on
danse pour le seul plaisir de danser, pour agiter les membres
accoutumés à un violent exercise; on danse pour exhaler un sentiment
de joie qui n’a pas besoin de spectateurs.” (\textit{Encyclopedie
Méthodique: Musique}, i. 316, 4to., 1791.)

I have quoted the passage from the \textit{Encyclopedie} at length,
because M. Framery’s reasons are exactly those which account for the
long-enduring popularity of the country-dance. The French contredanse
(known in England by the name of quadrille) cannot be traced to an
earlier period than the latter part of the seventeenth century, and
it seems to have originated in the first quarter of the eighteenth.
It is not described by Thoinot Arheau, or any of the early French
writers on dancing. J. J. Rousseau, Compan (author of the
\textit{Dictionnaire de Danse}), and other writers of the last century, if
they do not give the etymology, either say that it was danced after
minuets, or with gavotte steps, therefore subsequent to both. The
first French dictionary in which I have been enabled to trace the
word, is that of P. Richelet, printed at Amsterdam in 1732. It is not
contained in the Geneva edition of 1680, or in that of 1694.

“Contre” certainly means “opposite,” and men stand opposite their
partners in modern country dances, but this was by no means a rule in
early times. There were great varieties of figure, and some of the
earliest (such as \textit{Sellinger’s Round}) were danced in circles, often
round a tree or maypole.

In The English \textit{Dancing Master} of 1651, besides those danced in
the modern way (which are described as “\textit{Longways} for as many as
will”), there are the following Rounds “for as many as will:”—\textit{The
chirping of the Nightingale}; \textit{Gathering peascods}; \textit{If all the world
were paper} (a still-remembered nursery song); \textit{Millfield}; \textit{Pepper's
black}; and \textit{Rose is white, rose is red}. There are also rounds for four
and for eight.

In \textit{Dargason} (a country dance older than the Reformation) men and
women

\begin{dcfootnote}
\textsuperscript{a} Weaver says, “Country-dances are a dancing the peculiar 
growth of this nation; tho now transplanted into 
almost all the Courts of Europe.” \textit{Essay towards the 
History of Dancing}, 8vo., 1712, p. 170. Essex, in the 
preface to his \textit{Treatise on Chorography, or the art of dancing
Country Dances}, 1710, says “This which we call Country 
Dancing is originally the product of this nation.” Haw-
kins quotes Weaver.

\end{dcfootnote}
\end{fixedpage}%627
\pagebreak

\begin{fixedpage}%628
\versoheader

stand in one straight line at the commencement; the men together
and the women together.

\textit{Fain I would} is “a Square Dance for eight,” and men and women
stand as in a quadrille, except that the man is on the right of his
partner.

\textit{Dull Sir John and Hide Parke} are also square dances for eight;
and in those the couples stand exactly as in the modern quadrille.
This is the form the French copied, and with it some of the
country-dance figures. To one of these they gave the name of “Chaine
Anglaise.”

Although Playford alludes, in his preface, to the dancing of the
ancient Greeks, and to “the sweet and airy activity” of our gentlemen
of the Inns of Court (who were no doubt looking out to become Lord
Chancellors); he does not mention French dancing, neither is there
one French term in the book.

It is time to protest against Mr. De Quincey’s derivation since
it has been quoted in a work of such authority as \textit{English, Past and
Present}, by Richard Chevenix Trench, B.D. (the present Dean of
Westminster), and this book the substance of lectures delivered to
the pupils of King’s College, London. I would add that, to this day,
French dances have made no way in English villages. The amusements of
our peasantry are the hornpipe, the country-dance, the jig, and
occasionally the reel.

I have no doubt that, if time permitted me to make the search, I
should find much English dance music in early French collections, as
well as in those of other countries; for, on a recent visit of a few
hours to the Bibliothêque Imperial, in Paris, three books of lute
music were shown to me,\textsuperscript{a} and among the contents I observed, “Courante
d’Angleterre,” “Gigue d’Angleterre,” “Sarabande d’Angleterre,”
“Pavane d’Angleterre” (several), “Galliarda Joannis Doolandi”
(Dowland), “Chorea Anglicana,” \&c. One of these was a manuscript
with a printed title-page by Robert Ballard, the early French music
publisher (No. 2,660); a second, \textit{Le Tresor d’Orphée}, printed by his
widow and son in 1600. In the preface to the third, I read “Prout
sunt illi Anglicani Concentus suavissimi quidem, ac elegantes,” \&c.
This was \textit{Thesaurus Harmonicus divini Laurencini, Romani}. Cologne,
1603. fol.

Playford recommends dancing as making the body active and strong
and the deportment graceful; but I imagine that when country-dances
were danced in the country, activity and lofty springing were the
principal tests of excellence.

The genuine country way was perhaps as described in Rastell’s
interlude, \textit{The Four Elements}, where one of the characters says,—

\indentpattern{001001}
\settowidth{\versewidth}{With fryscas and with gambawdes round,}
\begin{dcverse}
\begin{patverse}
\vleftofline{“}I shall bryng hydyr another sort \\
Of lusty bluddes, to make dysport, \\
That can both daunce and spryng, \\
And torne dene above the grounde\\
With fryscas and with gambawdes round,\\
That all the hall shall ryng.”
\end{patverse}
\end{dcverse}

It may have been otherwise at Court, for, as the song says,—
\settowidth{\versewidth}{Not in the English lofty manner.”}
\begin{scverse}
\vin\vin \vleftofline{“}There they did dance\\
\vin\vin As in France,\\
Not in the English lofty manner.”
\end{scverse}

\begin{dcfootnote}
\textsuperscript{a} I am indebted to the courtesy of Mr. Anders, one of 
the librarians, for shewing me these books of lute music, 
and for assisting me in the search after the origin of the 
Contredanse. Readers are not permitted to see the
catalogues of the library, as in England; so that, unless 
a book has been quoted before, it is only by such assistance that it can be discovered.

\end{dcfootnote}
\end{fixedpage}%628
\pagebreak

\begin{fixedpage}%629
\rectoheader

Now as to jigs and reels. Jigs seem to have been danced at Court
until the crown passed to the house of Hanover. There are jigs named
after every king and Queen from Charles II. to Queen Anne, and many
from noblemen of the Court. I have not observed them enumerated among
the dances on state occasions, and imagine therefore that they were
only used for relaxation. Jigs were also danced upon the stage, for,
in the epilogue to \textit{The Chances}, a play which the Duke of Buckingham
altered from Beaumont and Fletcher, he speaks of dramatists
appropriating to themselves the applause intended for Nell Gwynne,—

\settowidth{\versewidth}{Swell, and believe \textit{themselves} the Lord knows what.}
\begin{scverse}
\vleftofline{“}Besides the author dreads the strut and mien \\
Of new-prais’d poets, having often seen \\
Some of his fellows, who have writ before,\\
When Nell has danc’d her jig, steal to the door,\\
Hear the pit clap, and with conceit of that,\\
Swell, and believe \textit{themselves} the Lord knows what.”
\end{scverse}

In speaking of the reel it is necessary to include the hay, for
dancing a reel is but one of the ways of dancing the hay.

Strutt describes the hay as “a rustic dance, where they lay hold
of hands and dance round in a ring;” but I think this a very
imperfect, if not an incorrect definition. The hay was danced in a
line as well as in a circle, and it was by no means a rule that hands
should be given in passing. To dance the hey or hay became a
proverbial expression signifying to twist about, or wind in and out
without making any advance. So in Hackluyt’s \textit{Voyages}, iii. 200, “Some
of the mariners thought we were in the Bristow Channell, and other in
Silly Channell; so that, through variety of judgements and evill
marinership, we were faine to \textit{dance the hay} foure dayes together,
sometimes running to the north-east, sometimes to the south-east, and
again to the east, and east north-east.” In Sir John Davies’s
\textit{Orchestra}, “He taught them rounds and \textit{winding heys} to tread.”\textsuperscript{a} (In
the margin he explains “rounds and winding heys” to be
country-dances.) In \textit{The Dancing Master} the hey is one of the figures
of most frequent occurrence. In one country-dance, “the women stand
still, the men going the hey between them.” This is evidently winding
in and out. In another, two men and one woman dance the hey,—like a
reel. In a third, three men dance this hey, and three women at the
same time—like a double reel. In \textit{Dargason}, where many stand in one
long line, the direction is “the single hey, all handing as you pass,
till you come to your places.” When the hand was to be given in
passing, it was always so directed; but the hey was more frequently
danced without “handing.” In “the square dances,” the two opposite
couples dance the single hey twice to their places, the woman
standing before her partner at starting. When danced by many in a
circle, if hands were given, it was like the “grande chaine” of a
quadrille.

Old dance and ballad tunes were greatly revived at the
commencement of the

\begin{dcfootnote}
\settowidth{\versewidth}{He taught them rounds and winding heys to tread.}
\begin{fnverse}
\textsuperscript{a} “Thus, when at first Love had them marshalled, \\
As erst he did the shapeless mass of things, \\
He taught them rounds and winding heys to tread. \\
And about trees to cast themselves in rings:\\
As the two Bears, whom the First Mover flings,\\
With a short turn, about Heaven's axle-tree,\\
In a \textit{round dance} for ever wheeling be.”
\end{fnverse}
\end{dcfootnote}
\end{fixedpage}%629
\pagebreak

\begin{fixedpage}%630
\versoheader

reign of George II. through the medium of the ballad-operas. The
first of these was \textit{The Beggars’ Opera}, which contained the necessary
amount of political satire to suit the taste of the day in song, and
was a caricature of Italian operas, then in the height of fashion. It
was first offered to Cibber, at Drury Lane, and rejected by him, but
accepted by Rich, the manager of the Theatre Royal in Lincoln’s Inn
Fields, and produced on the 29th of January, 1727-8. It was written
by Gay,—the success was extraordinarily great, and it was said, by
one of the wits of the day, to have made Gay \textit{rich}, and Rich \textit{gay}. The
following account of it is from the notes to \textit{The Dunciad}:—“This
piece was received with greater applause than was ever known. Besides
being acted in London sixty-three days without intermission, and
renewed the next season with equal applause, it spread into all the
great towns of England; it made a progress into Wales, Scotland, and
Ireland; the ladies carried about with them the favorite songs of it,
on fans; houses were furnished with it on screens; furthermore, it
drove out of England, for that season, the Italian opera, which had
carried all before it for ten years.” Lavinia Fenton, who acted the
part of Polly, became the toast of the town, and was soon after
married by Charles, third Duke of Bolton.

One of the \textit{Miscellaneous Poems by several Hands}, published by D.
Lewis (8vo., 1730), is on the success of \textit{The Beggars’ Opera}. It is
“Old England’s Garland; or, The Italian Opera’s Downfall. An
excellent new ballad, to the tune of \textit{King John and the Abbot of
Canterbury} and commences thus:—

\indentpattern{00006}
\settowidth{\versewidth}{How old England grew fond of old tunes of her own,}
\begin{scverse}
\begin{patverse}
\vleftofline{“}I sing of sad discords that happened of late,\\
Of strange revolutions, but not in the state;\\
How old England grew fond of old tunes of her own,\\
And her ballads went up and our operas down.\\
Derry down, down, hey derry down.”
\end{patverse}
\end{scverse}

It is again alluded to in the epilogue to \textit{Love in a Riddle},—

\settowidth{\versewidth}{When warbling dames were all in flames,}
\begin{dcverse}
\begin{altverse}
“Poor English mouths, for twenty years, \\
Have been shut up from music;\\
But, thank our stars, outlandish airs \\
At last have made all you sick.\\
When warbling dames were all in flames, \\
And for precedence wrangled,\\
One English play cut short the fray,\\
And home again they dangled.\\
Sweet sound on languid sense bestow’d\\
Is like a beauty married \\
To th’ empty fop who talks aloud,\\
And all her charms are buried.\\
But late experience plainly shews \\
That common sense, and ditty,\\
Have ravish’d all the belles and beaux, \\
And charmed the chaunting city.”
\end{altverse}
\end{dcverse}

For the six years that ensued after the production of the
Beggars’ Opera, scarcely any other kind of drama was produced on the
stage. Even for the booths in Bartholomew Fair new ballad operas were
written, and subsequently published with the tunes. In many the music
was printed in type with the book; for others it was engraved and
sold separately.

I may here remark that the engraving of music on metal plates
seems to have been practised in England before it was used in Italy,
or any other country. In England it commenced in the reign of James
I. Before that time all music had been printed from moveable types,
except perhaps an occasional short specimen from a wooden block. The
two first music engravers were William and
\end{fixedpage}%630
\pagebreak

\begin{fixedpage}%631
\rectoheader

Robert Hole. William engraved \textit{Parthenia}, a collection of pieces
for the Virginals, dedicated to the Princess Elizabeth, daughter of
James I. These were “composed by three famous masters, William Byrd,
Dr. John Bull, and Orlando Gibbons, Gentlemen of His Majesties Most
Illustrious Chappell,” and first published in 1611. Robert Hole
engraved a work of similar character for virginals and base-viol,
under the title of \textit{Parthenia inviolata}, which was published without
date.

It was, no doubt, the demand for instrumental music, that first
suggested the resort to engraving, and instrumental music was more
cultivated in England than in any other country. Proofs of this have
already been given, but it does not rest wholly upon the testimony of
English writers. Many allusions to the excellence of our
instrumentalists might be cited from foreigners, like that of
Giovanni Battista Doni, in his \textit{De Præstantiâ Musicœ veteris}, a book
written in dialogue, and printed in 1647. One of the speakers is the
advocate of the then modern music, the other of that of the ancients.
On the subject of the tibiæ or pipes of the Greeks, the latter says
“The English are allowed to excel on the flute, and there are many
good performers on the cornet in that kingdom, but I cannot believe
them equal to the ancient players on the tibia, such as Antigenides,
Pronomus, and Timotheus.” No mention is here made of other
instruments than the flute and cornet, because the discussion is
confined to tibiæ and their modern representatives.

The cornet was an extremely difficult instrument to play well.
The Lord Keeper North says of it, “Nothing comes so near, or rather
imitates so much, an excellent voice, as a cornet-pipe; but the
labour of the lips is too great, and it is seldom well sounded.” He
adds, that in the churches of York and Durham, cornets and other wind
music were used in the choirs at the Restoration, to supply the
deficiency of voices and organs, but afterwards disused.

Instrumental music was much employed at our theatres, not only in
operas, but also when tragedies and comedies were performed. Orazio
Busino, in his account of the Venetian Embassy to the Court of James
I., says, “We saw a tragedy [at the Fortune Theatre] which diverted
me very little, especially as I cannot understand a word of English,
though some little amusement may be derived from gazing at the very
costly dresses of the actors, and from the various interludes of
instrumental music, and dancing and singing; but the best treat was
to see such a crowd of nobility, so very well arrayed that they
looked like so many princes, listening as silently and soberly as
possible.” (\textit{Quarterly Review}, October, 1857.)

Down to the time of \textit{The Beggars’ Opera}, it had been the custom to
perform three movements of instrumental music, termed “first, second,
and third music,” before the commencement of each play. A story is
told of Rich, the manager, who when the customary music was called
for by the audience at the first performance of \textit{The Beggars’ Opera},
came forward and said, “Ladies and gentlemen, there is no \textit{music} to an
opera” (setting the house in a roar of laughter),—“I mean, ladies and
gentlemen, an opera is all music.”
\end{fixedpage}%631
\pagebreak

\begin{fixedpage}%632
\versoheader

Before 1690, engraving may be said to have been employed only for
instrumental music. There were a few exceptions, such as Dr. Child’s
\textit{Psalms for three voices}, printed in 1639, and reprinted by Playford
in 1650, from the same plates; but types were greatly preferred for
vocal music, on account of the greater distinctness of the words.
After 1690, the town began to teem with single songs, printed on one
side of the paper, from engraved plates. Every one who had any
knowledge of music, however slight, seemed ready to rush into print,
and many wrote songs and published them to old tunes,—a class that
old John Playford would have deemed unworthy of his press.

Among the encomiastic verses prefixed to Dr. Blow’s \textit{Amphion
Anglicus}, in 1700, are the following allusions to these
publications:—

\settowidth{\versewidth}{The mightiest of them cry, ‘Let’s please the town’}
\begin{scverse}
\vleftofline{“}The mightiest of them cry, ‘Let’s please the town’\\
(If that be done, they value not the gown);\\
And then, to let you see ’tis good and taking,\\
’Tis soon in ballad howl'd, ere mob are waking.\\
O happy men, who thus their fames can raise.\\
And lose not e’en one inch of Kent Street praise!\\
But yet the greatest scandal's still behind,—\\
A baser dunce among the crew we find;\\
A wretch bewitched to see his name in print,\\
Will own a song, and not one line his in’t!\\
I mean of the foundation—sad’s the case,\\
He treble writes, no matter who the bass;\\
Just like some over-crafty architect,\\
Would first the garret, then the house erect.\\
Such trash, we know, has pester’d long the town,\\
But thou appear, and they as soon are gone.’’
\end{scverse}

Although Dr. Blow did appear, these would-be composers did not
expire quite so soon as the writer expected. Perhaps there remain
some a little like them even at the present day.

Another of Dr. Blow’s encomiasts says,—

\settowidth{\versewidth}{They print the names of those who set and wrote ’em,}
\begin{scverse}
\vleftofline{“}Long have we been with balladry oppress'd;\\
Good sense lampoon’d, and harmony burlesqu’d:\\
Music of many parts hath now no force,\\
Whole reams of Single Songs become our curse,\\
With bases wondrous lewd, and trebles worse.\\
But still the luscious lore goes gliby down,\\
And still the \textit{double entendre} takes the town.\\
They print the names of those who set and wrote ’em,\\
With Lords at top and blockheads at the bottom:\\
While at the shops we daily dangling view \\
False concords by Tom Cross engraven true.”
\end{scverse}

The following are specimens of the popular music of this period.
\end{fixedpage}%632
\pagebreak
\begin{fixedpage}%633
\rectoheader

\musictitle{OLD KING COLE.}

The first question that may be asked here, is, “Who \textit{was} old King
Cole?” I should say that he was “old Cole” the famous cloth
manufacturer, of Reading, one of “the sixe worthie yeomen of the
West;”—that his name became proverbial through an extremely popular
story-book of the sixteenth century; and that he acquired his
kingship much in the same manner as another celebrated worthy, “Old
Sir Simon the King.”

There was some joke or conventional meaning among Elizabethan
dramatists, when they gave a man the name of Old Cole, which it is
now difficult to discover. Gifford supposes it to be a nickname given
to Ben Jonson by Dekker, because in the \textit{Satiromastix}, where Horace
says, “I’ll lay my hands under your feet, Captain Tucca,” Tucca
answers, “Say’st thou to me so, old Cole? come do it, then;” but
Dekker uses it elsewhere when there can be no allusion to Ben Jonson.
In the \textit{Second part of The Honest Whore}, Matheo gives the name to
Orlando, who had promised to assist him: “Say no more, old Cole; meet
me anon at the sign of The Shipwreck.” Marston, too, in \textit{The
Malcontent}, makes Malevole apply it to a woman,—

“\textit{Malevole to Maquarelle}. Ha, Dipsas! how dost thou, old Cole?

\textit{Maquarelle}. Old Cole!

\textit{Malevole}. Ay, old Cole; methinks thou liest like a brand under
billets of green wood.”

This play was printed in 1604, and dedicated to Ben Jonson, with
whom Marston was then on the most friendly terms. It is true that Ben
Jonson, in \textit{Bartholomew Fair}, gives the name of Old Cole to the
sculler in the puppet-show of Hero and Leander; but this was first
acted in 1614, and Dekker’s \textit{Satiromastix} printed in 1602.

Perhaps the name originated in the ridicule of some drama upon
the story of \textit{The Six worthy Yeomen of the West}. “Old Cole” is thus
mentioned by Deloney:—

“It chanced on a time, as he [King Henry I.] with one of his
sonnes, and divers of his nobility, rode from London towards Wales,
to appease the fury of the Welshmen, which then began to raise
themselves in armes against his authority, that he met with a number
of waines loaden with cloth, comming to London; and seeing them still
drive one after another so many together, demanded whose they were;
the waine-men answered in this sort: Cole’s of Reading (quoth they).
Then by and by the King asked another, saying, Whose cloth is all
this? \textit{Old Cole’s}, quoth hee: and again anon after he asked the same
question to others, and still they answered, Old Cole’s. And it is to
be remembered, that the King met them in such a place, so narrow and
so streight, that hee, with the rest of his traine, were faine to
stand as close to the hedge, whilest the carts passed by, the which
at that time being in number about two hundred, was neere hand an
houre ere the King could get roome to be gone: so that by his long
stay he began to be displeased, although the admiration of that sight
did much qualify his furie;\dots and so, soon after, the last wain
passed by, which gave present passage unto him and his nobles:
and thereupon entring into communication of the commoditie of
cloathing, the King gave order at his home returne, to have \textit{Old Cole}
brought before his Majestie, to the intent he
\end{fixedpage}%633
\pagebreak

\begin{fixedpage}%634
\versoheader

might have conference with him, noting him to be a subject of
great ability,” \&c. From \textit{The Pleasant Historie of Thomas of Reading:
or the Sixe worthie Yeomen of the West}. “Now the sixth time corrected
and enlarged.” London: Printed by Eliz. Allde for Robert Bird, 1632.

Dr. Wm. King, a humourous writer, who was born in 1663, quotes
some of the words of “Old King Cole” in No. 6 of his \textit{Useful
Transactions}; but he mixes them up with those of “Four-and-twenty
fiddlers all of a row.”

\indentpattern{11011001}
\settowidth{\versewidth}{And he called for his fiddlers three;}
\begin{dcverse}
\begin{patverse}
\vin \vleftofline{“}Good King Cole \\
And he called for his bowl, \\
And he called for his fiddlers three; \\
And there was fiddle, fiddle, \\
And twice fiddle, fiddle,\\
For ’twas my lady’s birthday;\\
Therefore we keep holyday.\\
And come to be merry.”
\end{patverse}
\end{dcverse}


I have no earlier authority for the tune than is to be found in
the ballad-operas. In Gay’s \textit{Achilles}, published in 1733 (after the
death of the author), the song, “No more be coy, give a loose to
joy,” is to the air of Old King Cole, and it differs altogether from
the way in which it is now commonly sung.

The following version of the air is from \textit{Achilles}:—

The following traditional air bears a slight family likeness to
\textit{The British Grenadiers}, although the one is major and the other
minor.
\end{fixedpage}%634
\pagebreak

\begin{fixedpage}%635
\rectoheader

\indentpattern{0101230301}
\settowidth{\versewidth}{Then twang, twang-a-twang, twang-a-twang went the harper,}
\begin{dcverse}
\begin{patverse}
Old King Cole was a merry old soul, ,\\
And a merry old soul was he; \\
He call’d for his pipe, and he call’d for his bowl,\\
And he called for his harpers three, \\
Ev’ry harper he had a fine harp, \\
And a very fine harp had he;\\
Then twang, twang-a-twang, twang-a-twang went the harper,\\
(imitating the harp)\\
Twee, tweedle dee, tweedle dee, went the fiddler,\\
And so merry we’ll all be.
\end{patverse}
\end{dcverse}

In the second and subsequent stanzas, the part of the tune which
goes to the line, “Then twee, tweedle dee,” \&c., must be repeated as
required by the multiplication of words.

In the third verse, Old King Cole calls for his pipers three, and
the words are the same as before, except the change of the word
fiddle or harp for pipe,— 

\indentpattern{0001}
\settowidth{\versewidth}{Twang, twang-a-twang, twang-a-twang, went the harper;}
\begin{scverse}
\begin{patverse}
Then tootle, tootle too, tootle too, went the piper;\\
Twang, twang-a-twang, twang-a-twang, went the harper;\\
Tweedie, tweedle dee, tweedle dee, went the fiddler;\\
And so merry we’ll all be.
\end{patverse}
\end{scverse}

In the fourth verse he calls for his drummers three,—

\begin{scverse}
\vleftofline{“}Then rub a dub, a dub, rub a dub, went the drummer,” \&c.
\end{scverse}
And thus in each verse the strain, with the line, “Twee, tweedle
dee,” \&c., is repeated as the imitation of the different instruments
may require.
\end{fixedpage}%635
\pagebreak

\begin{fixedpage}%636
\versoheader

\musictitle{THE ROAST BEEF OF OLD ENGLAND.}

This famous old song has been admirably illustrated by Hogarth\textsuperscript{a}
in his picture of “The gate of Calais”:—

\indentpattern{001001}
\settowidth{\versewidth}{See how the half-starv’d Frenchmen strut,}
\begin{dcverse}
\begin{patverse}
\vleftofline{“}With lanthorn jaws and meagre cut,\\
See how the half-starv’d Frenchmen strut, \\
And call us English dogs, \\
But soon we’ll teach these bragging foes \\
That \textit{Beef and Beer} give heavier blows \\
Than soup and roasted frogs.’’
\end{patverse}
\end{dcverse}

There are two songs on this subject: the one by Henry Fielding,
in his comedy of \textit{Don Quixote in England}; the other by Richard
Leveridge, the composer of the tune.

Fielding’s song which was sung to the air of \textit{The Queen's old
Courtier}, consists of but two verses, and the comedy in which it is
contained was published in 1733. Leveridge’s song is printed in
Walsh’s \textit{British Musical Miscellany}, and in \textit{The Universal Musician},
both undated.

\centredsc{H. Fielding’s Song.}

\indentpattern{0001100011}
\settowidth{\versewidth}{Our soldiers were brave, and our courtiers were good.}
\begin{dcverse}
\begin{patverse}
When mighty roast beef was the Englishman’s food, \\
It ennobled our hearts, and enriched our blood;\\
Our soldiers were brave, and our courtiers were good.\\
Oh, the roast beef of old England!\\
And oh, for old England’s roast beef!\\
Then, Britons, from all the nice dainties refrain,\\
Of effeminate Italy, France, or Spain;\\
And mighty roast beef shall command on the main.\\
Oh, the roast beef of old England!\\
And oh, for old England’s roast beef!
\end{patverse}
\end{dcverse}

\begin{dcfootnote}
\textsuperscript{a} Hogarth was very inveterate in his enmity to the French, having
been seized, and narrowly escaping being
shot as a spy, while sketching the gate of Calais.
\end{dcfootnote}
\end{fixedpage}%636
\pagebreak

\begin{fixedpage}%637
\rectoheader

\centredsc{R. Leveridge’s Song.}

\indentpattern{00011}
\settowidth{\versewidth}{Our soldiers were brave, and our courtiers were good.}
\begin{dcverse}
\begin{patverse}
When mighty roast beef was the Englishman’s food, \\
It ennobled our hearts, and enriched our blood;\\
Our soldiers were brave, and our courtiers were good.\\
Oh, the roast beef of old England!\\
And oh, for old England’s roast beef!
\end{patverse}

\begin{patverse}
But since we have learn’d from effeminate France\\
To eat their ragouts, as well as to dance,\\
We are fed up with nothing but vain complaisance.\\
Oh, the roast beef, \&c.
\end{patverse}

\begin{patverse}
Our fathers of old were robust, stout, and strong, \\
And kept open house, with good cheer all day long,\\
Which made their plump tenants rejoice in this song.\\
Oh, the roast beef, \&c. 
\end{patverse}

\begin{patverse}
When good Queen Elizabeth sat on the throne,\\
Ere coffee and tea, and such slip-slops were known,\\
The world was in terror if e’en she did frown.\\
Oh, the roast beef, \&c.
\end{patverse}

\begin{patverse}
In \textit{those} days, if fleets did presume on the main,\\
They seldom or never return’d back again;\\
As witness the vaunting Armada of Spain.\\
Oh, the roast beef, \&c.
\end{patverse}

\begin{patverse}
Oh, then we had stomachs to eat and to fight,\\
And when wrongs were cooking, to set ourselves right;\\
But now we’re a—hm!—I could, but good night.\\
Oh, the roast beef of old England!\\
And oh, for old England’s roast beef!
\end{patverse}
\end{dcverse}

Many other songs have been written to this tune, one in praise
of old English brown beer, and several anti-Jacobite songs; but the
new application of the fable of the Frog and the Ox, written by
Hogarth’s friend, Theophilus Forest, as an illustration for his
picture of “The Gate of Calais,” must not be omitted.

\musictitle{THE ROAST BEEF CANTATA.}

\settowidth{\versewidth}{He lick’d his chaps, and thus the knight address’d:}
\begin{dcverse}
\indentpattern{0000}
\begin{patverse}
’Twas at the gate of Calais, Hogarth tells,\\
Where sad despair and famine always dwells,\\
A meagre Frenchman, Madame Grandsire’s cook, \\
As home he steered, his carcase that way took.
\end{patverse}

\begin{patverse}
Bending beneath the weight of famed Sirloin,\\
On whom he’d often wish’d in vain to dine.\\
Good Father Dominick by chance came by,\\
With rosy gills, round paunch, and greedy eye;
\end{patverse}

\begin{patverse}
Who, when he first beheld the greasy load,\\
His benediction on it he bestowed;\\
And as the solid fat his fingers press’d,\\
He lick’d his chaps, and thus the knight address’d:
\end{patverse}

\indentpattern{010101}
\begin{patverse}
“Oh, rare roast beef, lov’d by all mankind,\\
If I was doom’d to have thee, \\
When dress’d and garnish’d to my mind,\\
And swimming in thy gravy,\\
Not all thy country’s force combin’d \\
Should from my fury save thee.
\end{patverse}

\begin{patverse}
Renown’d sirloin, ofttimes decreed \\
The theme of English ballad;\\
On thee e’en kings have deign’d to feed. \\
Unknown to Frenchmen’s palate:\\
Then how much more thy taste exceeds \\
Soup meagre, frogs, and sallad!”
\end{patverse}

\indentpattern{0000}
\begin{patverse}
A half-starv’d soldier, shirtless, pale, and lean,\\
Who such a sight before had never seen,\\
Like Garrick’s frighted Hamlet, gaping stood,\\
And gazed with wonder on the British food.
\end{patverse}

\begin{patverse}
His morning’s mess forsook the friendly bowl,\\
And in small streams along the pavement stole;\\
He heav’d a sigh which gave his heart relief,\\
And then in plaintive tone declared his grief;
\end{patverse}
\end{dcverse}

\end{fixedpage}%637
\pagebreak

\begin{fixedpage}%638
\versoheader

\settowidth{\versewidth}{The gallows, more kind would have sav’d me from starving.’}
\begin{dcverse}
\begin{altverse}
“Ah! sacre Dieu! vat do me see yonder, \\
Dat look so tempting red and vite? \\
Begar it is de roast beef from Londre;\\
Oh, grant to me von litel bite! \\
But to my pray'r if you give no heeding, \\
And cruel fate dis boon denies,\\
In kind compassion unto my pleading, \\
Return, and let me feast mine eyes!”
\end{altverse}

\indentpattern{000000}
\begin{patverse}
His fellow guard, of right Hibernian clay, \\
Whose brazen front his country did betray, \\
From Tyburn’s fatal tree had thither fled, \\
By honest means to gain his daily bread, \\
Soon as the well-known prospect he descry’d,\\
In blubb’ring accents dolefully he cried:
\end{patverse}

\indentpattern{0220}
\begin{patverse}
“Sweet beef, that now causes my stomach to rise, \\
So taking thy sight is,\\
My joy that so light is,\\
To view thee, by pailsfull, tears run from my eyes.
\end{patverse}

\begin{patverse}
While here I remain, my life’s not worth a farthing;\\
Ah, hard-hearted Lewy, \\
Why did I come to ye?\\
The gallows, more kind would have sav’d me from starving.’’
\end{patverse}

\indentpattern{000000}
\begin{patverse}
Upon the ground, hard by, poor Sawney sate,\\
Who fed his nose, and scratch’d his ruddy pate;\\
But when old England’s bulwark he espy’d, \\
His dear lov'd mull, alas! was thrown aside:\\
With lifted hands he blest his native place, \\
Then scrubb’d himself, and thus bewail’d his case:
\end{patverse}

\indentpattern{01011212}
\begin{patverse}
“How hard, O Sawney, is thy lot,\\
Who was so blithe of late,\\
To see such meat as can’t be got,\\
When hunger is so great.\\
Oh, the beef! the bonny, bonny beef!\\
When roasted nice and brown;\\
I wish I had a slice of thee,\\
How sweet it would gang down!
\end{patverse}

\indentpattern{01012}
\begin{patverse}
Ah, Charley! hadst thou not been seen, \\
This ne’er had happ’d to me:\\
I wou’d the de’il had pick’d mine ey’n, \\
Ere I had gaug’d with thee.\\
Oh, the beef,” \&c.
\end{patverse}

\indentpattern{000000}
\begin{patverse}
But, see my muse to England takes her flight!\\
Where health and plenty socially unite; \\
Where smiling freedom guards great George’s throne, \\
And whips, and chains, and tortures, are not known.\\
That Britain’s fame in loftiest strains should ring,\\
In rustic fable give me leave to sing.
\end{patverse}

\begin{patverse}
(\textit{Tune of “The Roast Beef of Old England}.” )
\end{patverse}

\indentpattern{00022}
\begin{patverse}
As once on a time, a young frog, pert and vain, \\
Beheld a large ox grazing on the wide plain,\\
He boasted his size he could quickly attain.\\
Oh, the roast beef of old England! \\
And oh, the old English roast beef!
\end{patverse}

\begin{patverse}
Then eagerly stretching his weak little frame, \\
Mamma, who stood by, like a knowing old dame,\\
Cry’d, Son, to attempt it you're surely to blame.\\
Oh, the roast beef, \&c. 
\end{patverse}

\begin{patverse}
But deaf to advice, he for glory did thirst, \\
An effort he ventur’d more strong than the first, \\
Till swelling and straining too hard, made him burst.\\
Oh, the roast beef, \&c.
\end{patverse}

\begin{patverse}
Then, Britons, be careful, the moral is clear,\\
The ox is old England, the frog is Monsieur;\\
Whose threats and bravadoes we never need fear.\\
While we have roast beef in old England. \\
Sing oh, for old England’s roast beef!
\end{patverse}

\begin{patverse}
For while by our commerce and arts we are able, \\
To see the sirloin smoking hot on our table,\\
The French must e'en burst, like the frog in the fable!\\
Oh, the roast beef, \&c.
\end{patverse}
\end{dcverse}
\end{fixedpage}%638
\pagebreak


\begin{fixedpage}%639
\rectoheader

\musictitle{POOR ROBIN’S MAGGOT.}

This tune is contained in the second volume of \textit{The Dancing
Master}; in \textit{Pills to purge Melancholy}, i. 132, 1719; in \textit{The Beggars’
Opera}; \textit{The Generous Freemason}, and other ballad-operas. It is also
known in the present day as one of “The Lancers Quadrilles.”

In \textit{The Dancing Master} it is named \textit{Poor Robin's Maggot}; in the
\textit{Pills} and ballad-operas, “\textit{Would you win a young virgin of fifteen
years}.” This is from a song by D’Urfey, in his play of \textit{Modern
Prophets}, 4to., 1709.

The words in \textit{The Beggars’ Opera}, “If the heart of a man is
deprest with cares,” are still occasionally sung to the air; but I
have here adopted the song in \textit{The Generous Freemason} (8vo., 1731),
one of the ballad-operas performed at Bartholomew Fair.

The words carry out the adage that “faint heart never won fair
lady.”
\end{fixedpage}%639
\pagebreak

\begin{fixedpage}%640
\versoheader

\musictitle{BLACK-EYED SUSAN.}

This still popular song was composed by Leveridge, author of \textit{The
roast beef of Old England}, and of several other favorite songs. He
was a bass singer at the Theatre in Lincoln’s Inn Fields; and, when
more than sixty years of age, still thought his voice so good that he
offered for a wager of a hundred guineas to sing a bass song with any
man in England. The tune is very like another which he composed to
the words, “Send back my long-strayed eyes;” and, in both, he seems
to have drawn more on memory than imagination. One of the snatches
sung by Ophelia, in Hamlet, and several other old songs begin in the
same manner.

The words of “Sweet William’s farewell to black-ey’d Susan” are
by Gay, and are printed in his Poems, as well as on numerous extant
broadsides with music; in Watts’s \textit{Musical Miscellany}, iv. 148, \&c.

The tune was introduced into \textit{The Devil to pay}; \textit{The Village Opera};
\textit{Robin Hood}, 1730; \textit{The Chambermaid}; \textit{The Grub Street Opera}; \textit{The Welsh
Opera}; \&c.

The same words were set by Henry Carey, and others; but
Leveridge’s became the popular tune.

The following version is as it is now sung:—
\end{fixedpage}%640
\pagebreak

\begin{fixedpage}%641
\rectoheader

\indentpattern{010100}
\settowidth{\versewidth}{And (quick as lightning) on the deck he stands.}
\begin{dcverse}
\begin{patverse}
William, who, high upon the yard,\\
Rock’d with the billows to and fro,\\
Soon as her well-known voice he heard,\\
He sighed, and cast his eyes below: \\
The cord slides swiftly through his glowing hands.\\
And (quick as lightning) on the deck he stands.
\end{patverse}

\begin{patverse}
So the sweet lark, high-pois’d in air,\\
Shuts close his pinions to his breast,\\
(If, chance, his mate’s shrill voice he hear,) \\
And drops at once into her nest.\\
The noblest captain in the British fleet \\
Might envy William’s lip those kisses sweet.
\end{patverse}

\begin{patverse}
“O Susan, Susan, lovely dear,\\
My vows shall ever true remain:\\
Let me kiss off that falling tear,\\
We only part to meet again.\\
Change as ye list, ye winds; my heart shall be \\
The faithful compass that still points to thee. 
\end{patverse}

\begin{patverse}
Believe not what the landmen say,\\
Who tempt with doubts thy constant mind: \\
They’ll tell thee, sailors, when away,\\
In every port a mistress find.\\
Yes, yes, believe them when they tell thee so,\\
For thou art present wheresoe’er I go.
\end{patverse}

\begin{patverse}
If to fair India’s coast we sail,\\
Thy eyes are seen in di’monds bright;\\
Thy breath is Afric’s spicy gale,\\
Thy skin is ivory so white.\\
Thus every beauteous object that I view,\\
Wakes in my soul some charm of lovely Sue.
\end{patverse}

\begin{patverse}
Though battle call me from thy arms,\\
Let not my pretty Susan mourn;\\
Though cannons roar, yet safe from harms, \\
William shall to his dear return.\\
Love turns aside the halls that round me fly,\\
Lest precious tears should drop from Susan’s eye.”
\end{patverse}

\begin{patverse}
The boatswain gave the dreadful word,\\
The sails their swelling bosom spread;\\
No longer must she stay on hoard:\\
They kiss’d, she sigh’d, he hung his head.\\
Her less’ning boat unwilling rows to land;\\
Adieu! she cries, and wav’d her lily hand.
\end{patverse}
\end{dcverse}

\musictitle{ADMIRAL BENBOW.}

The subject of this ballad is mentioned in Evelyn’s Diary, under
the date of January, 1702-3. “News of Vice-Admiral Benbow’s conflict
with the French fleet in the West Indies, in which he gallantly
behaved himself, and was wounded, and would have had extraordinary
success, had not four of his men-of-war stood spectators without
coming to his assistance; for this, two of their commanders were
tried by a council of war and executed; a third was condemned to
perpetual imprisonment, loss of pay, and incapacity to serve in
future. The fourth died.”

Admiral Benbow was a thoroughly gallant seaman. He received his
commission in the navy for his bravery in beating off a corsair,
while in command of a merchant vessel. When the Moors boarded him,
they were driven back, leaving thirteen of their number dead upon his
deck. He was twice sent to the West Indies by King William. On the
second occasion, he fell in with the French Admiral, Du Casse, in
August, 1702, near the Spanish coast. A skirmishing action continued
for four days, but on the last the Admiral was left alone to engage
the French, the other ships having fallen astern. Although thus
single-handed, and having his leg shattered by a chain-shot, he would
not suffer himself to be removed from the quarter-deck (in this
respect the ballad is incorrect), but continued fighting until the
following morning, when the French sheered off. The Admiral made
signal for his ships to follow, but his orders received no attention,
and he was obliged to return to Jamaica, where he caused the officers
who behaved so basely, to be tried. The report of the court-martial
will be found in \textit{The Harleian Miscellany}, vol. i., 4to., 1744. There
was a treasonable conspiracy among the officers of his fleet, not to
fight the French. Admiral Benbow did not long survive this
disappointment; it aggravated the effects of his wound, and he
expired.
\end{fixedpage}%641
\pagebreak

\begin{fixedpage}%642
\versoheader

This favorite old sea-song is in a collection of penny
song-books, formerly belonging to Ritson; and, with music, in Dale’s
Collection, i. 68.

The Rev. James Plumptre wrote “When in war, on the ocean we meet
the proud foe,” to the tune. It is published in his collection of
songs with music, 8vo., 1805.

Another song on the death of Admiral Benbow is contained in
Halliwell’s \textit{Early Naval Ballads of England}. It commences,—

\settowidth{\versewidth}{Come, all you sailors bold, lend an ear, lend an ear,}
\begin{scverse}
\begin{altverse}
\vleftofline{“}Come, all you sailors bold, lend an ear, lend an ear,\\
Come, all you sailors bold, lend an ear:\\
\vleftofline{’}Tis of our Admiral’s fame, Brave Benbow call’d by name,\\
How be fought on the main you shall hear, you shall hear.”
\end{altverse}
\end{scverse}

The tune of \textit{Admiral Benbow} is the vehicle of several country
songs at the present time, and used for Christmas carols. In the
month of January last, Mr. Samuel Smith noted it down from the
singing of some carollers at Marden, near Hereford, to the words
commencing,—

\settowidth{\versewidth}{“A virgin unspotted the prophets foretold.”}
\begin{scverse}
“A virgin unspotted the prophets foretold.”
\end{scverse}
\end{fixedpage}%642
\pagebreak

\begin{fixedpage}%643
\rectoheader

\settowidth{\versewidth}{And when he was wounded, to his merry men he did say,}
\begin{scverse}
The first we came up with was a brigantine sloop,\\
And we ask’d if the others were big as they look’d;\\
But turning to windward as near as we could lie,\\
We found there were ten men of war cruizing by.

Oh! we drew up our squadron in very nice line,\\
And boldly we fought them for full four hours’ time;\\
But the day being spent, boys, and the night coming on,\\
We let them alone till the very next morn.

The very next morning the engagement prov’d hot,\\
And brave Admiral Benbow receiv’d a chain shot;\\
And when he was wounded, to his merry men he did say,\\
“Take me up in your arms, boys, and carry me away.”

Oh the guns they did rattle, and the bullets did fly,\\
But Admiral Benbow for help would not cry;\\
Take me down to the cockpit, there is ease for my smarts,\\
If my merry men see me it will sure break their hearts.

The very next morning, by break of the day,\\
They hoisted their topsails, and so bore away;\\
We bore to Port Royal, where the people flock’d much \\
To see Admiral Benbow carried to Kingston Church.

Come all you brave fellows, wherever you’ve been,\\
Let us drink to the health of our King and our Queen,\\
And another good health to the girls that we know,\\
And a third in remembrance of brave Admiral Benbow.
\end{scverse}

I suspect that this was originally a much longer ballad, and that
the last stanza was substituted for the remaining verses at a later
date. The story is only half told, all notice of the treachery of the
four captains is omitted, as well as of their trial, and the death of
the Admiral. Perhaps the ballad was thus curtailed to be sung upon
the stage.

\musictitle{DOWN AMONG THE DEAD MEN.}

This tune is in the third volume of \textit{The Dancing Master} printed by
Pearson and Young, Playford’s successors, and in the third volume of
Walsh’s \textit{Dancing Master}.

There are many half-sheet copies of the song with music; and one
that I conceive to be the earliest, commences, “Here’s a health to
the \textit{Queen} and a lasting peace.”

In one of the volumes of half-sheet songs in the British Museum
(H. 1601, p. 205), is “A health to the memory of Queen Anne,” to the
tune of \textit{Down among the dead men}. It commences—

\settowidth{\versewidth}{Here’s a health to the mem’ry of Queen Anne,}
\begin{scverse}
\vleftofline{“}Here’s a health to the mem’ry of Queen Anne,\\
Come pledge me ev’ry English man,\\
For, though her body’s in the dust,\\
Her memory shall live, and must.\\
And they that Anna’s health deny,\\
Down among the dead men let them lie,’’ \&c.
\end{scverse}

In the same volume is “a song sung by Mr. Dyer, at Mr. Bullock’s
booth in Southwark Fair.” This is a George I. copy of “\textit{Down among the
dead men};”
\end{fixedpage}%643
\pagebreak

\begin{fixedpage}%644
\versoheader

therefore commencing, “Here’s a health to the \textit{King},”\&c. A third version gives “Mr. Robert Dyer’s additional stanzas, as sung by him at Lincoln’s Inn Theatre.”

The author of the words, whoever he may have been, had in mind the drinking-song in Fletcher’s \textit{Bloody Brothers}, from which he borrowed two lines,—

\settowidth{\versewidth}{Best, while you have it, use your breath,}
\begin{scverse}
\vleftofline{“}Best, while you have it, use your breath,\\
There is no drinking after death.”
\end{scverse}

The tune of \textit{Down among the dead men} was a great favorite with the late Samuel Wesley, who used constantly to fugue upon it.

\end{fixedpage}%644
\pagebreak

\setlength{\fixedpagewidth}{370pt}
\begin{fixedpage}%645
\rectoheader

\settowidth{\versewidth}{While Bacchus’ treasure crowns the board,}
\begin{dcverse}
Let charming beauty’s health go round, \\
In whom celestial joys are found,\\
And may confusion still pursue \\
The senseless woman-hating crew;\\
And they that woman's health deny, \\
Down among the dead men let them lie!

In smiling Bacchus’ joys I'll roll,\\
Deny no pleasure to my soul;\\
Let Bacchus’ health round briskly move,\\
For Bacchus is a friend to Love.\\
And he that will this health deny,\\
Down among the dead men let him lie.

May love and wine their rites maintain, \\
And their united pleasures reign,\\
While Bacchus’ treasure crowns the board, \\
We’ll sing the joys that both afford;\\
And they that won’t with us comply, \\
Down among the dead men let them lie.
\end{dcverse}

\musictitle{SALLY IN OUR ALLEY.}

This extremely popular ballad was written and composed by Henry
Carey.

Carey’s tune is to be found in his \textit{Musical Century}, ii. 32; in
Walsh’s \textit{Dancing Master}, vol. ii. 1719; in \textit{The Beggars’ Opera}; \textit{The
Devil to pay}; \textit{The Fashionable Lady}; \textit{The Merry Cobbler}; \textit{Love in a
Riddle}; \textit{The Rival Milliners}; and on numerous half-sheet songs.

The following is the author’s account of the origin of the
ballad:—

%\quotation
{\small
“A vulgar error having prevailed among many persons, who imagine
Sally Salisbury the subject of this ballad, the author begs leave to
undeceive and assure them it has not the least allusion to her, he
being a stranger to her very name at the time this song was composed:
for, as innocence and virtue were ever the boundaries of his muse,
so, in this little poem, he had no other view than to set forth the
beauty of a chaste and disinterested passion, even in the lowest
class of human life. The real occasion was this: a shoemaker’s
’prentice, making holiday with his sweetheart, treated her with a
sight of Bedlam, the puppet-shows, the flying-chairs, and all the
elegancies of Moorfields, from whence proceeding to the
farthing-pye-house, he gave her a collation of buns, cheesecakes,
gammon of bacon, stuffed beef, and bottled ale, through all which
scenes the author dodged them. Charmed with the simplicity of their
courtship, he drew from what he had witnessed this little sketch of
nature; but, being then young and obscure, he was very much ridiculed
by some of his acquaintance for this performance, which nevertheless
made its way into the polite world, and amply recompensed him by the
applause of the \textit{divine Addison}, who was pleased more than once to
mention it with approbation.”}

Among the songs printed to Carey’s tune are the following:—

1. “Sally’s Lamentation; or, The Answer to Sally;” beginning—

\settowidth{\versewidth}{That ev’ry brat would sing so soon,}
\begin{dcverse}
\begin{altverse}
\vleftofline{“}What pity ’tis so bright a thought \\
Should e’er become so common; \\
At ev’ry corner brought to naught \\
By ev’ry bawling woman.\\
I little thought, when you began \\
To write of charming Sally,\\
That ev’ry brat would sing so soon,\\
‘She lives in our alley.’”
\end{altverse}
\end{dcverse}

2. “Sally in our Alley to Billy in Piccadilly; with proper graces
to the tune.”

\settowidth{\versewidth}{Of all the lads that are so smart}
\begin{dcverse}
\begin{altverse}
\vleftofline{“}Of all the lads that are so smart \\
There’s none I love like Billy; \\
He is the darling of my heart,\\
And he lives in Piccadilly,” \&c.
\end{altverse}
\end{dcverse}

3. “Sally in her own cloaths;” beginning—

\settowidth{\versewidth}{Of all the mauxes in the land}
\begin{scverse}\small
\begin{altverse}
\vleftofline{“}Of all the mauxes in the land\\
There’s none I hate like Sally.”
\end{altverse}
\end{scverse}

4. “Sally rivall’d by Country Molly;” commencing—

\settowidth{\versewidth}{Since Sally’s charms so long have been}
\begin{dcverse}
\begin{altverse}
\vleftofline{“}Since Sally’s charms so long have been \\
The theme of court and city, \\
Pray give me leave to raise the song \\
And praise a girl more pretty.”
\end{altverse}
\end{dcverse}

\end{fixedpage}%645
\pagebreak

\setlength{\fixedpagewidth}{360pt}
\begin{fixedpage}%646
\versoheader

5. “Blowzabel. A Song;” commences—

\settowidth{\versewidth}{Of Anna’s charms let others tell, }
\begin{dcverse}
\begin{altverse}
\vleftofline{“}Of Anna’s charms let others tell, \\
Of bright Eliza’s beauty; \\
My song shall be of Blowzabel,\\
To sing of her’s my duty.”
\end{altverse}
\end{dcverse}

6. “As Damon late with Chloe sat.”

There are many more printed to Carey’s tune, but the above
suffice to shew how very popular it was; and yet, about 1760, it was
discarded. “Sally in our Alley” is now only sung to the much older
ballad-tune of \textit{The Country Lass}. It is difficult to account for this,
except from the extended compass of voice which Carey’s air required.
The two ballads were concurrently popular. “The Virtuous Country
Lass” was engraved, as a single song, by Cross, as well as printed in
\textit{The Merry Musician}. Both tunes were introduced in \textit{The Devil to pay},
\&c.

The following is the ballad with Carey’s music:—
\end{fixedpage}%646
\pagebreak

\begin{fixedpage}%647
\rectoheader

The following is the tune to which the words have been sung for
nearly a century. By comparing it with the older version of \textit{The
Country Lass}, at p. 376, the reader will see what variations time has
made.

\settowidth{\versewidth}{And through the streets does cry them;}
\begin{dcverse}
\begin{altverse}
Her father he makes cabbage-nets,\\
And through the streets does cry them; \\
Her mother she sells laces long,\\
To such as please to buy them:\\
But sure such folks could ne’er beget \\
So sweet a girl as Sally;\\
She is the darling of my heart,\\
And lives in our alley.
\end{altverse}

\begin{altverse}
When she is by, I leave my work, \\
I love her so sincerely;\\
My master comes, like any Turk, \\
And bangs me most severely: \\
But let him bang, long as he will, \\
I’ll bear it all for Sally;\\
She is the darling of my heart, \\
And lives in our alley.
\end{altverse}

\begin{altverse}
Of all the days are in the week,\\
I dearly love but one day,\\
And that’s the day that comes betwixt \\
A Saturday and Monday:\\
For then I'm dress’d in all my best,\\
To walk abroad with Sally;\\
She is the darling of my heart,\\
And lives in our alley.
\end{altverse}

\begin{altverse}
My master carries me to church, \\
And often I am blamed, \\
Because I leave him in the lurch, \\
Soon as the text is named:\\
I leave the church in sermon time, \\
And slink away to Sally;\\
She is the darling of my heart, \\
And lives in our alley.
\end{altverse}
\end{dcverse}

\end{fixedpage}%647
\pagebreak

\begin{fixedpage}%648
\versoheader

\settowidth{\versewidth}{But when my sev’n long years are out,}
\begin{dcverse}
\begin{altverse}
When Christmas comes about again, \\
O then I shall have money;\\
I’ll hoard it up, and box and all,\\
I’ll give unto my honey:\\
I would it were ten thousand pounds, \\
I’d give it all to Sally;\\
She is the darling of my heart,\\
And lives in our alley.\\
My master and the neighbours all. \\
Make game of me and Sally,\\
And but for her I’d better be \\
A slave, and row a galley:\\
But when my sev’n long years are out, \\
Oh, then I’ll marry Sally,\\
And then how happily we’ll live—\\
But not in our alley.
\end{altverse}
\end{dcverse}

Incledon sang only the first, second, fourth, and last verses.

\musictitle{AS DOWN IN THE MEADOWS.}

This pretty and graceful song is to be found in \textit{The Merry
Musician, or A Cure for the Spleen}, ii. 129; in Watts’s \textit{Musical
Miscellany}, i. 62; and on many broadsides with music.

The tune was introduced by Gay in his ballad-opera of \textit{Polly},
1729; also in \textit{The Cobblers’ Opera}, \textit{The Court Legacy}, \textit{The Lovers’
Opera}, \&c.

The same words were afterwards set by Oswald, but he was not
successful in his music. A copy will he found in the Burney
Collection.

This is sometimes entitled “Susan’s Complaint and Remedy.”
\end{fixedpage}%648
\pagebreak

\setlength{\fixedpagewidth}{390pt}
\begin{fixedpage}%649
\rectoheader

\settowidth{\versewidth}{Will leave her, and then she with me may complain:}
\begin{dcverse}
Why does my love Willy prove false and unkind, \\
O why does he change like the wavering wind, \\
From one that is loyal in every degree?\\
Ah! why does he change to another from me ? \\
In the meadows as we were a making of hay, \\
Oh there did we pass the soft minutes away; \\
And then was I kiss’d and set down on his knee, \\
No man in the world was so loving as he.

But now he has left me, and Fanny the fair \\
Employs all his wishes, his thoughts and his care ;\\
He kisses her lip as she sits on his knee,\\
And says all the sweet things he once said to me:\\
But if she believe him, the false-hearted swain \\
Will leave her, and then she with me may complain:\\
For naught is more certain, believe, silly Sue, \\
Who once has been faithless can never be true.

She finished her song, and rose up to be gone, \\
When over the meadow came jolly young John, \\
Who told her that she was the joy of his life, \\
And if she’d consent, he would make her his wife:\\
She could not refuse him, to church so they went\\
Young Willy’s forgot,and young Susan’s content. \\
Most men are like Willy, most women like Sue, \\
If men will be false, why should women be true?
\end{dcverse}

\musictitle{O MOTHER, A HOOP!}

To this tune Cibber wrote the song “What woman could do, I have
tried, to be free,” for his ballad-opera of \textit{Love in a Riddle}, 1729.
It is also printed in \textit{The Merry Musician}, ii. 7.

In \textit{The Livery Rake}, 1733, the air takes the name of Cibber’s
song; but in \textit{Ramon and Phillida}, 1734, it is entitled \textit{O Mother, a
hoop!}

There are two versions of “O Mother, a hoop!” the one as a song,
the other “A Dialogue between Miss Molly and her Mother about a
hoop.” A copy of the latter will be found in one of the collections
in the British Museum (H. 1601, p. 532). It consists of ten stanzas,
commencing thus:—

\settowidth{\versewidth}{Men’s eyes and men’s hearts will be roving still; O, <\&c.}
\begin{scverse}
\begin{altverse}
\vleftofline{\textit{Daughter}.—“}What a fine thing have I seen to-day,\\
O Mother, a hoop:\\
I pray let me have one, and do not say nay,\\
O Mother, a hoop.”
\end{altverse}

\indentpattern{0022}
\begin{patverse}
\vleftofline{\textit{Mother}.—“}You must not have one, dear Moll, to be sure,\\
For hoops do men’s eyes and men’s hearts so allure,\\
No, Molly, no hoop, no hoop,\\
No, Molly, no hoop.”
\end{patverse}

\vleftofline{\textit{Daughter}.—“}Dear Mother, let women wear what they will, O, \&c.\\
Men’s eyes and men’s hearts will be roving still; O, <\&c.\\
Whether decently clothed or sluttishly dress’d,\\
Some men prefer these and others the rest. O, \&c.

Men wear lac’d hats and ladies lac’d shoes,\\
Men with canvas and whalebone do stiffen their clothes,\\
Then why should the men the ladies abuse \\
For applying the same things, and to the same use.

Pray hear me, dear Mother, what I have been taught—\\
Nine men and nine women o’erset in a boat,\\
The men were all drown’d, but the women did float,\\
And by help of their hoops they all safely got out,” \&c.
\end{scverse}

In some of the broadsides with music, the tune is attributed to
Mr. Bradford. The following is the first stanza of the song:—
\end{fixedpage}%649
\pagebreak

\setlength{\fixedpagewidth}{360pt}
\begin{fixedpage}%650
\versoheader

\musictitle{THE DUSKY NIGHT RIDES DOWN THE SKY.}

The song of “The dusky night rides down the sky” was written hy
Henry Fielding, for his ballad-opera of \textit{Don Quixote in England}
(1734), to the tune of \textit{A begging we will go}; but, on the broadsides
with music—in \textit{Vocal Music, or The Songster’s Companion}—in \textit{The Vocal
Enchantress}—in Dale’s, and other collections, it is printed to this
tune, and still sung to it.

Several other songs have been written to the same air. Among
them, “Father Paul;” commencing, “When grave divines preach up dull
rules.” A copy of this is in Dr. Burney’s Collection, 8, 240, Brit.
Mus. It has for burden or chorus— 

\settowidth{\versewidth}{For flowing bowls inspire the souls}
\begin{scverse}
\begin{altverse}
\vleftofline{“}Here’s a health to Father Paul,\\
A health to Father Paul,\\
For flowing bowls inspire the souls \\
Of jolly friars all.”
\end{altverse}
\end{scverse}
\end{fixedpage}%650
\pagebreak

\begin{fixedpage}%651
\rectoheader

The following May-day song was written by Miss Mary Herron, of
Durham:—

\indentpattern{01012}
\settowidth{\versewidth}{When from the East, with dappled grey,}
\begin{dcverse}
\begin{patverse}
\vleftofline{“}When from the East, with dappled grey,\\
The morn begins to peep,\\
And ushers in the welcome May,\\
We shake off drowsy sleep.\\
And a maying we will go, \&c. 
\end{patverse}

\begin{patverse}
With rural dance, and jocund song,\\
We gambol on the green;\\
And, shepherd-like, among the throng \\
Select our May-day queen.\\
And a maying we will go, \&c.
\end{patverse}

\begin{patverse}
Then let us all in chorus join,\\
To celebrate the day;\\
And wish through life our fate may shine \\
A smiling month of May.\\
And a maying we will go,” \&c.
\end{patverse}
\end{dcverse}

In the Rev. James Plumptre’s Collection is “The Health,” for the
harvest supper, to this tune,—

\settowidth{\versewidth}{With grateful hearts we’ll drink thehealth}
\begin{scverse}
\begin{altverse}
\vleftofline{“}With grateful hearts we’ll drink the health\\
Of him who gives this cheer:\\
May Providence increase his wealth \\
With ev’ry coming year,” \&c.
\end{altverse}
\end{scverse}

The following is Henry Fielding’s song:—
\end{fixedpage}%651
\pagebreak

\begin{fixedpage}%652
\versoheader


\indentpattern{01014}

\settowidth{\versewidth}{The wife around her husband throws}
\begin{dcverse}

\begin{patverse}
The wife around her husband throws \\
Her arms, and begs his stay;\\
My dear, it rains, it hails and snows,\\
You will not hunt to-day.\\
But a hunting we will go.
\end{patverse}

\begin{patverse}
A brushing fox in yonder wood,\\
Secure to find we seek;\\
For why, I carried, sound and good,\\
A cartload there last week.\\
And a hunting we will go.
\end{patverse}

\begin{patverse}
Away he goes, he flies the rout,\\
Their steeds all spur and switch;\\
Some are thrown in, and some thrown out. \\
And some thrown in the ditch.\\
But a hunting we will go.
\end{patverse}

\begin{patverse}
At length his strength to faintness worn, \\
Poor reynard ceases flight;\\
Then hungry, homeward we return,\\
To feast away the night.\\
Then a drinking we do go.
\end{patverse}
\end{dcverse}

Instead of the last three stanzas of the above, the four following are usually sung:—

\begin{dcverse}
\begin{patverse}
Th’ uncavern’d fox, like lightning flies,\\
His cunning’s all awake;\\
To gain the race he eager tries;\\
His forfeit life the stake!\\
When a hunting we do go, \&c.
\end{patverse}

\begin{patverse}
Arous’d, e’en Echo huntress turns,\\
And madly shouts her joy;\\
The sportsman’s breast enraptur’d burns, \\
The chace can never cloy.\\
Then a hunting we will go, \&c.
\end{patverse}

\begin{patverse}
Despairing, mark! he seeks the tide,\\
His heart must now prevail;\\
Hark! shout the hunters, death betide,\\
His speed, his cunning fail.\\
When a hunting we do go, \&c.
\end{patverse}

\begin{patverse}
For lo! his strength to faintness worn,\\
The hounds arrest his flight;\\
Then hungry homewards we return,\\
To feast away the night.\\
Then a drinking we do go, \&c.
\end{patverse}
\end{dcverse}

\musictitle{THE VICAR OF BRAY.}

Simon Aleyn, Canon of Windsor, was Vicar of Bray, in Berkshire,
from 1540 to 1588: “He was a Papist under the reign of Henry VIII.,
and a Protestant
under Edward VI.; he was a Papist again under Mary, and once more
became a Protestant in the reign of Elizabeth. When this scandal to
the gown was reproached for his versatility of religious creeds, and
taxed for being a turncoat and an inconstant changeling, as Fuller
expresses it, he replied, ‘Not so neither; for if I changed my
religion, I am sure I kept true to my principle; which is, to live
and die the Vicar of Bray.’”

This vivacious and reverend hero gave birth to a proverb, “The
Vicar of Bray will be Vicar of Bray still.” In a sermon preached
before the Lord Mayor and Aldermen of London, by John Evans, in 1682,
after describing the common notion of a Moderate Minister in the
church, as one who would comply with the humours and fancies of all
parties, he says, “And if this be moderation, the old \textit{Vicar of Bray}
was the most moderate man that ever breathed.” (Southey’s \textit{Common
Place Book}, p. 159.)

Nichols in his Select Poems says that the song of the Vicar of
Bray “was written by a soldier in Colonel Fuller’s troop of Dragoons,
in the reign of George I.”

In the ballad operas, such as \textit{The Quakers' Opera}, 1728, and \textit{The
Grub Street Opera} and the \textit{Welsh Opera}, both 1731, the original name
of the tune is given as \textit{The Country Garden}.

In some of the copies the tune is printed in \timesig{3}{4} time, which
entirely changes its character; it then becomes a plaintive love
ditty instead of a sturdy and bold air. The curious will find the \timesig{3}{4}
version in \textit{National English Airs} (No. 26, p. 14).
\end{fixedpage}%652
\pagebreak

\begin{fixedpage}%653
\rectoheader

% music only
\end{fixedpage}%653
\pagebreak

\begin{fixedpage}%654
\versoheader

\indentpattern{010101014}
\settowidth{\versewidth}{When royal James possess’d the crown,}
\begin{dcverse}
\begin{patverse}
When royal James possess’d the crown, \\
And popery grew in fashion,\\
The penal laws I hooted down,\\
And read the Declaration:\\
The church of Rome I found would fit \\
Full well my constitution;\\
And I had been a Jesuit,\\
But for the Revolution.\\
And this is law, \&c.
\end{patverse}

\begin{patverse}
When William was our King declar’d, \\
To ease the nation’s grievance;\\
With this new wind about I steer’d, \\
And swore to him allegiance:\\
Old principles I did revoke,\\
Set conscience at a distance;\\
Passive obedience was a joke,\\
A jest was non-resistance.\\
And this is law, \&c.
\end{patverse}

\begin{patverse}
When royal Anne became our queen, \\
The church of England’s glory,\\
Another face of things was seen,\\
And I became a tory:\\
Occasional conformists base,\\
I blam’d their moderation;\\
And thought the church in danger was,\\
By such prevarication.\\
And this is law, \&c.
\end{patverse}

\begin{patverse}
When George in pudding-time came o’er, \\
And moderate men look’d big, sir,\\
My principles I chang’d once more,\\
And so became a whig, sir;\\
And thus preferment I procur’d \\
From our new faith’s-defender;\\
And almost ev’ry day abjur’d \\
The Pope and the Pretender.\\
And this is law, \&c.
\end{patverse}

\begin{patverse}
Th’ illustrious house of Hanover,\\
And Protestant succession,\\
To these I do allegiance swear—\\
While they can keep possession:\\
For in my faith and loyalty,\\
I never more will falter,\\
And George my lawful king shall be—\\
Until the times do alter.\\
And this is law, \&c.
\end{patverse}
\end{dcverse}

The above air was also rendered 
popular by the song of “The Neglected Tar,” commencing—

\settowidth{\versewidth}{I sing the British seaman’s praise;}
\begin{scverse}
\vleftofline{“}I sing the British seaman’s praise;\\
A theme renown’d in story,” \&c.
\end{scverse}
It is printed in the Rev. James Plumptre’s dull, but highly moral
collection, 8vo., 1805.

\musictitle{THE SPRING’S A COMING.}

This tune is contained in the third volume of \textit{The Dancing Master},
and in the third volume of Walsh’s \textit{Dancing Master}, under the name of
\textit{Humours of the Bath}. It was introduced in many ballad-operas, such as
\textit{The Wedding}, \textit{The Beggars’ Wedding}, \textit{The Lovers’ Opera}, \textit{The Devil to
pay}, and \textit{A Rehearsal of a new Ballad-Opera Burlesqued}, and generally
under the title of “The Spring’s a coming,” from the first line of
“The Bath Medley,” written by Tony Aston.

This Tony Aston was an actor, who, in 1735, petitioned the House
of Commons to be heard against the bill then pending for regulating
the stage, and was permitted to deliver a ludicrous speech, which was
afterwards published. His way of living was then peculiar to himself;
resorting to the principal cities or towns in England, with his
\textit{Medley}, as he termed it, which was composed of some scenes of humour
out of the most celebrated plays, and filling up the intervals
between the scenes by a song or dialogue of his own writing.

“The Bath Medley” is printed with the tune, in Watts’s \textit{Musical
Miscellany}, i, 161 (1729), and Coffey’s song, “Young Virgins love
pleasure,” to the same air, in the fifth volume of that work. Coffey
wrote it for his play, \textit{The Beggars’ Wedding}.

The words here adapted were written to the air by the late George
Macfarren.
\end{fixedpage}%654
\pagebreak

\begin{fixedpage}%655
\rectoheader

\settowidth{\versewidth}{With gay serenades from her chorister birds,}
\begin{scverse}
The Spring is coming to waken the roses \\
With gay serenades from her chorister birds, \\
Ev’ry breathing flow’ret’s lip discloses \\
A gratitude sweeter than mortal words.\\
Shall we be the last to swell the measure \\
That all nature’s children in harmony sing? \\
Ah no! we’ll tune with a holier pleasure\\
The carol of welcome to joyful Spring.
\end{scverse}
\end{fixedpage}%655
\pagebreak

\begin{fixedpage}%656
\versoheader

\musictitle{SWEET NELLY, MY HEART’S DELIGHT.}

This song is usually entitled \textit{The Farmer’s Son}; it was extremely
popular at the commencement of the last century, and remains so to
the present day. Mr. J. H. Dixon informs me that “it is still
regularly printed in Yorkshire, and that no song is more in favour
with the small farmers and the peasantry.”

It is contained in \textit{The Merry Musician, or A Cure for the Spleen},
ii. 78; in Watts’s \textit{Musical Miscellany}, i. 130 (1729); in \textit{The British
Musical Miscellany, or The Delightful Grove}, published by Walsh, and
there are numerous extant copies on broadsides.

The air was introduced in many ballad-operas, such as \textit{The Lovers’
Opera}, \textit{The Footman}, \&c.; and the words printed in many song-books.
\end{fixedpage}%656
\pagebreak

\begin{fixedpage}%657
\rectoheader

\settowidth{\versewidth}{Their loves will soon be won;}
\begin{dcverse}
\indentpattern{0011011011110}
\begin{patverse}
\vleftofline{\textit{She}. }No! I am a lady gay,\\
It is very well known I may \\
Have men of renown,\\
In country or town;\\
So, Roger, without delay,\\
Court Bridget or Sue,\\
Kate, Nancy, or Prue,\\
Their loves will soon be won;\\
But don’t you dare \\
To speak me fair,\\
As if I were \\
At my last pray’r.\\
To marry a farmer’s son.
\end{patverse}

\begin{patverse}
\vleftofline{\textit{He}. }My father has riches in store, \\
Two hundred a year, and more;\\
Besides sheep and cows,\\
Carts, harrows and ploughs: \\
His age is above three-score; \\
And when he does die,\\
Then merrily I \\
Shall have what he has won; \\
Both land and kine,\\
All shall he thine,\\
If thou’lt incline \\
And wilt be mine,\\
And marry a farmer’s son.
\end{patverse}

\begin{patverse}
\vleftofline{\textit{She}. }A fig for your cattle and corn! \\
Your proffer’d love I scorn.\\
’Tis known very well \\
My name it is Nell,\\
And you’re but a bumpkin born. \\
\vleftofline{\textit{He}. }Well, since it is so,\\
Away I will go,\\
And I hope no harm is done. \\
Farewell! adieu!\\
I hope to woo \\
As good as you,\\
And win her, too,\\
Though I’m but farmer’s son.
\end{patverse}

\begin{patverse}
\vleftofline{\textit{She}. }Be not in such haste, quoth she, \\
Perhaps we may still agree;\\
For, man, I protest \\
I was but in jest;\\
Come, prythee, sit down by me:\\
For thou art the man \\
That verily can \\
Win me, if e’er I’m won:\\
Both straight and tall,\\
Genteel with all,\\
Therefore I shall \\
Be at your call,\\
To marry a farmer’s son.
\end{patverse}

\begin{patverse}
\vleftofline{\textit{He}. }Dear Nelly, believe me, now,\\
I solemnly swear and vow,\\
No lords in their lives \\
Take pleasure in wives \\
Like we that do drive the plough: \\
Whatever we gain \\
With labour or pain \\
We don’t after harlots run,\\
As courtiers do;\\
And I never knew \\
A London beau \\
That could out-do \\
A country farmer’s son.
\end{patverse}
\end{dcverse}

\musictitle{COME, JOLLY BACCHUS.}

In the second volume of \textit{The Dancing Master}, this tune is called
“Frisky Jenny, or The tenth of June;” in the third volume it is again
printed under the title of “The Constant Lover.” In Walsh’s \textit{Lady’s
Banquet} it appears as “The Swedes Dance at the new Playhouse;” in \textit{The
Devil to pay}, and \textit{The Rival Milliners, or The Humours of Covent
Garden},. as “Charles of Sweden;” and in \textit{The Beggar’s Wedding} as
“Glorious first of August.” The song of \textit{Come, jolly Bacchus}, by the
name of which it is now best known, was written to the tune in \textit{The
Devil to pay}.

The following ballads and songs were also sung to it:—

1. On the taking of Portobello in 1739, entitled “English Courage
display’d: Or brave news from Admiral Vernon. To the tune of \textit{Charles
of Sweden}.” Contained in \textit{The Careless Batchelor’s Garland}. It is a
long ballad of eleven stanzas, commencing thus:—

\settowidth{\versewidth}{To Admiral Vernon drink a health, likewise to each brave fellow,}
\begin{scverse}\small
\vleftofline{“}Come, loyal Britons, all rejoice, with joyful acclamation,\\
And join with one united voice upon this just occasion.\\
To Admiral Vernon drink a health, likewise to each brave fellow,\\
Who with that noble Admiral was at the taking of Portobello.”
\end{scverse}

\end{fixedpage}%657
\pagebreak

\begin{fixedpage}%658
\versoheader

2. “A song to the tune of \textit{Come, Jolly Bacchus, god of wine}.” Two
stanzas. 

\settowidth{\versewidth}{Come, gallant Vernon, come, and prove}
\begin{scverse}
\indentpattern{01010001}
\begin{patverse}
\vleftofline{“}Come, gallant Vernon, come, and prove \\
How firm your friends are here, Sir;\\
Supported by the Public Love,\\
You will have nought to fear, Sir.\\
Soon shall mistaken boasters know\\
That we can still some virtue shew,\\
Resolved to ward corruption’s blow,\\
And check its swift career, Sir.”
\end{patverse}
\end{scverse}

3. “A new song made on board the Salamander, Privateer.” 

\settowidth{\versewidth}{Come, let’s drink a health to George our King,}
\begin{scverse}
\begin{altverse}
\vleftofline{“}Come, let’s drink a health to George our King, \\
And all his brave Commanders:\\
Another glass let us then toss off,\\
To the valiant Salamander,” \&c,
\end{altverse}
\end{scverse}

4. “A Jigg danc’d in the Schoole of Venus, or the Three-penny
Hops, burlesqu’d by Mr. John Vernham commencing—

\settowidth{\versewidth}{“O how I doat upon that lass.”}
\begin{scverse}
“O how I doat upon that lass.”
\end{scverse}

\settowidth{\versewidth}{Whilst brimmers flow with juice like this,}
\begin{dcverse}
\indentpattern{01010001}
\begin{patverse}
Let lovers whine, and statesmen think, \\
Always void of pleasure;\\
And let the miser hug his chink, \\
Destitute of pleasure:\\
But we like sons of mirth and bliss, \\
Obtain the height of happiness,\\
Whilst brimmers flow with juice like this, \\
In the midst of pleasure.
\end{patverse}

\begin{patverse}
Thus, mighty Bacchus, shalt thou be \\
Guardian to our pleasure;\\
That under thy protection we \\
May enjoy new pleasure;\\
And as the hours glide away,\\
We’ll in thy name invoke their stay, \\
And sing thy praises, that we may \\
Live and die in pleasure!
\end{patverse}
\end{dcverse}
\end{fixedpage}%658
\pagebreak

\begin{fixedpage}%659
\rectoheader

\musictitle{COUNTRY BUMPKIN.}

This tune is found in many of the ballad-operas in the first half
of the last century, such as \textit{The Cobbler’s Opera}; \textit{Robin Hood}; \textit{Momus
turn’d Fabulist, or Vulcan’s Wedding}; \textit{Don Quixote in England}; and \textit{The
Welsh Opera, or The Grey Mare the better Horse}.

The song from which it appears to derive its name is entitled
“\textit{The Politick Club},” and contained in \textit{Pills to purge Melancholy}, ii.
277 (1700 and 1707); but there printed to \textit{Green Sleeves}.

The tune is generally known at the present time. A few years ago
it was the vehicle of a song commencing—

\settowidth{\versewidth}{Thinks he’s going to die most surely,}
\begin{scverse}
\begin{altverse}
\vleftofline{“}When a man’s a little bit poorly,\\
He makes a fuss—wants a nurse,\\
Thinks he’s going to die most surely,\\
Sends for a doctor and soon gets worse.”
\end{altverse}
\end{scverse}

\musictitle{THE HATHERSAGE COCKING.}

The barbarous amusement which is the subject of this song, was
with the Athenians at first partly a religious and partly a political
institution, and afterwards continued for improving the seeds of
valour in the minds of their youth, but eventually perverted, both
there and in other parts of Greece, to a common pastime, without any
political or religious intention. It was afterwards adopted by the
Romans, and by them probably introduced into England. Cockfighting
has been called by some a royal diversion; and the Cockpit at
Whitehall was added to the palace by Henry the Eighth, and enlarged
by Charles II., for the purpose of giving greater patronage and
importance to the amusement.

This tune is an especial favourite in Derbyshire and
Warwickshire, and may frequently be heard in the alehouses, to these
and to other words. It was contributed in 1835, by the late Mr. Ward,
a teacher of music in Manchester, who used occasionally to entertain
his friends by singing it in the provincial dialect. From the
testimony of two persons he then traced it back one hundred and
twenty years. I do not, however, think that any such tracings are
very reliable as to the integrity of a tune, and beg the reader to
compare the two following.
\end{fixedpage}%659
\pagebreak

\begin{fixedpage}%660
\versoheader

There are several old ballads about cockfighting still extant, as
“The Wednesbury Cocking,” in the Douce Collection, commencing—

\indentpattern{00101010}
\settowidth{\versewidth}{A match between Newton and Scrogging,}
\begin{scverse}
\begin{patverse}
\vleftofline{“}At Wednesbury there was a cocking,\\
A match between Newton and Scrogging,\\
The colliers and nailers left their work,\\
And all to Spittles went jogging,\\
To see this noble sport.\\
Many noted men there resorted,\\
And though they’d but little money,\\
Yet that they freely sported,” \&c.
\end{patverse}
\end{scverse}

Hathersage is situated in the midst of a mountainous tract of
country near the eastern extremity of Hope Dale. The churchyard is
the reputed burial-place of Little John, the companion of Robin Hood.

I received but one stanza of the ballad from Mr. Ward, and have
not found it in print.

\musictitle{O GOOD ALE, THOU ART MY DARLING.}

This tune is still current in three different shapes. The first
as \textit{O good ale, thou art my darling}; the second to a song about
Turpin, the highwayman; and the third to the above song about
cock-fighting. They differ so much at the beginnings and endings that
it is necessary to treat them as separate tunes.

The following is the song of \textit{O good ale, thou art my darling},
from a broadside with music. The first part of this version resembles
\textit{John, come kiss me now} (ante p. 148).
\end{fixedpage}%660
\pagebreak

\begin{fixedpage}%661
\rectoheader

\indentpattern{00003}
\settowidth{\versewidth}{Come up, my friend,—and down he goes,}
\begin{dcverse}
\begin{patverse}
The brewer brew’d thee in his pan,\\
The tapster draws thee in bis can;\\
Now I with thee will play my part,\\
And lodge thee next unto my heart.\\
For ’tis O, good ale, \&c.
\end{patverse}

\begin{patverse}
Thou oft hast made my friends my foes,\\
And often made me pawn my clothes;\\
But since thou art so nigh my nose,\\
Come up, my friend,—and down he goes,\\
For ’tis O, good ale, \&c.
\end{patverse}
\end{dcverse}

\musictitle{O RARE TURPIN, HERO.}

This is one of several ballads about  Richard Turpin, the
highwayman, exalting him into a hero. It is contained in a pamphlet,
entitled “The Dunghill Cock; or Turpin’s valiant exploits,” \&c.,
“entered according to order” at Stationers’ Hall, but undated. It is
entitled “Turpin’s valour: \textit{to its own proper tune}?'’

Mr. W. Harrison Ainsworth makes Turpin one of the characters' in
his novel of \textit{Rookwood}, and represents him as singing snatches of this
ballad. It was evidently written in 1739, just before Turpin was
executed; yet is commonly known at the present time. Charles Sloman,
the comic singer, sang the ballad to me in 1840, for the purpose of
having the tune noted down.

In the Kilkenny Archæological Society’s publications (new
series, March, 1856, No. 2), is a ballad about Captain Freney, an
Irish highwayman, which was evidently suggested by, and partially
derived from this. The Kilkenny ballad commences— 

\indentpattern{000022}
\settowidth{\versewidth}{One morning, being free from care,}
\begin{scverse}
\begin{patverse}
\vleftofline{“}One morning, being free from care,\\
I rode abroad to take the air;\\
’Twas my fortune for to spy \\
A jolly Quaker riding by:\\
And it’s O bold Captain Freney,\\
O bold Freney O.”
\end{patverse}
\end{scverse}


\end{fixedpage}%661
\pagebreak

\begin{fixedpage}%662
\versoheader

The tune is printed in the Kilkenny Journal, but I believe it to
have been incorrectly noted down. The Irish are a nation possessed of
great musical taste and feeling, and I cannot imagine that any one,
having ears, could either sing or listen to so barbarous a thing.
Still there are traces of its being a corruption of \textit{O rare Turpin, O}.

I make no apology to my readers for printing one highwayman’s
ballad; after all, these are but continuations of the exploits of
Robin Hood. Nor need we go back to the Robin Hood era to find
instances of the greatest ladies of the court interceding to save the
lives of highwaymen, provided they were brave and handsome—witness
the case of Claude Duval.

As to the origin of the tune, see “\textit{O good ale, thou art my
darling}” (p. 660).

\indentpattern{00005}
\settowidth{\versewidth}{My gold, for its stitched in my cape behind.}
\begin{dcverse}
\begin{patverse}
Says Turpin, he’d ne’er find me out,\\
I’ve hid my money in my boot.\\
O, says the lawyer, there’s none can find\\
My gold, for its stitched in my cape behind.\\
О rare Turpin, \&c.
\end{patverse}

\begin{patverse}
As they rode down by the powder mill,\\
Turpin commands him to stand still;\\
Said he, Your cape I must cut off,\\
For my mare she wants a saddle cloth.\\
О rare Turpin, \&c.
\end{patverse}

\begin{patverse}
This caus’d the lawyer much to fret,\\
To think he was so fairly bit;\\
And Turpin robb d him of his store,\\
Because he knew he’d lie for more.\\
О rare Turpin, \&c.
\end{patverse}

\begin{patverse}
As Turpin rode in search of prey,\\
He met an exciseman on the way;\\
Then boldly he did bid him stand;\\
Your gold, said he, I do demand.\\
О rare Turpin, \&c.
\end{patverse}

\begin{patverse}
To that the exciseman did reply,\\
Your proud demands I must deny;\\
Before my money you receive,\\
One of us two shall cease to live.\\
О rare Turpin, \&c.
\end{patverse}
\end{dcverse}

\end{fixedpage}%662
\pagebreak

\begin{fixedpage}%663
\rectoheader

\indentpattern{00005}
\settowidth{\versewidth}{And robb’d the judge as he sat in his coach.}
\begin{dcverse}
\begin{patverse}
Turpin then, without remorse,\\
Soon knock’d him quite from off his horse,\\
And left him on the ground to sprawl,\\
So off he rode with his gold and all.\\
О rare Turpin, \&c.
\end{patverse}

\begin{patverse}
As he rode over Salisbury Plain,\\
He met Lord Judge with all his train ;\\
Then, hero-like, he did approach,\\
And robb’d the judge as he sat in his coach.\\
О rare Turpin, \&c.
\end{patverse}

\begin{patverse}
An usurer, as I am told,\\
Who had in charge a sum of gold,\\
With a cloak was clouted from side to side;\\
Just like a palmer he did ride.\\
О rare Turpin, \&c.
\end{patverse}

\begin{patverse}
And as he jogg’d along the way,\\
He met with Turpin that same day:\\
With hat in hand, most courteously\\
He asked him for charity.\\
О rare Turpin, \&c.
\end{patverse}

\begin{patverse}
If that be true thou tell’st to me,\\
I’ll freely give thee charity;\\
But I made a vow, and that I’ll keep,\\
To search all palmers I may meet.\\
О rare Turpin, \&c.
\end{patverse}

\begin{patverse}
He searched his bags, wherein he found\\
Upwards of eight hundred pound,\\
In ready gold and white money,\\
Which made him to laugh heartily.\\
О rare Turpin, \&c.
\end{patverse}

\begin{patverse}
This begging is a curious trade,\\
For in thy way thou hast well sped;\\
This prize I count as found money,\\
Because thou told’st me an arrant lie.\\
О rare Turpin, \&c.
\end{patverse}

\begin{patverse}
For shooting of a dunghill cock,\\
Poor Turpin now at last is took,\\
And carried straight unto a jail,\\
Where his ill luck he does bewail.\\
О \textit{poor} Turpin, \&c.
\end{patverse}

\begin{patverse}
Now some do say that he will hang,\\
Turpin the last of all the gang:\\
I wish this cock had ne'er been hatch’d,\\
For like a fish in a net he’s catch’d.\\
О poor Turpin, \&c.
\end{patverse}

\begin{patverse}
But if he had his liberty,\\
And were upon yon mountains high,\\
There’s not a man in old England,\\
Dare bid bold Turpin for to stand.\\
О poor Turpin, \&c.
\end{patverse}

\begin{patverse}
He ventur’d bold at young and old,\\
And fairly fought them for their gold;\\
Of no man he was e’er afraid,\\
But now, alas! he is betray’d.\\
О poor Turpin, \&c.
\end{patverse}

\begin{patverse}
Now Turpin is condem’d to die,\\
To hang upon yon gallows high:\\
His legacy is a strong rope,\\
For stealing a poor dunghill cock.\\
О poor Turpin, \&c.
\end{patverse}
\end{dcverse}


\musictitle{THE FREEMASONS’ TUNE.}

This tune was very popular at the time of the ballad-operas, and
I am informed that the same words are still sung to it at masonic
meetings.

The air was introduced in \textit{The Village Opera}, \textit{The Chambermaid}, \textit{The
Lottery}, \textit{The Grub-Street Opera}, and \textit{The Lover his own Rival}. It is
contained in the third volume of \textit{The Dancing Master}, and of Walsh’s
\textit{New Country Dancing Master}.

Words and music are included in Watts’s \textit{Musical Miscellany}, iii.
72, and in \textit{British Melody, or The Musical Magazine}, fol. 1739. They
were also printed on broadsides.

In \textit{The Gentleman’s Magazine} for October, 1731, the first stanza
is printed as “A Health, by Mr. Birkhead.” It seems to be there
quoted from “The Constitutions of the Freemasons, by the Rev. James
Anderson, A.M., one of the worshipful Masters.”

There are several versions of the tune. One in \textit{Pills to purge
Melancholy},
ii. 230, 1719, has a second part, but that, being almost a
repetition of the first,
\end{fixedpage}%663
\pagebreak

\begin{fixedpage}%664
\versoheader

taken an octave higher, is out of the compass of ordinary
voices, .and has therefore been generally rejected.

In \textit{A Complete Collection of Old and New English and Scotch Songs},
ii. 172 (1735), the name is given as “Ye Commons and Peers,” but
Leveridge composed another tune to those words. See \textit{Pills}.

In “The Musical Mason, or Free Mason’s Pocket Companion, being a
Collection of Songs used in all Lodges: to which are added The Free
Mason’s March and Ode,” (8vo., 1790), this is entitled “The Enter’d
Apprentice’s Song.” Many stanzas have been added from time to time,
and others have been altered. The following is the old copy:—

\indentpattern{220220}
\settowidth{\versewidth}{This our myst’ry to put a good grace on,}
\begin{dcverse}
\begin{patverse}
\vin\vin The world is in pain\\
Our secret to gain,\\
But still let them wonder and gaze on,\\
Till they’re shewn the light\\
They’ll ne’er know the right\\
Word or sign of an accepted Mason.
\end{patverse}

\begin{patverse}
\vin\vin ’Tis this, and ’tis that,\\
They cannot tell what,\\
Why so many great men of the nation\\
Should aprons put on,\\
To make themselves one\\
With a free and an accepted Mason.
\end{patverse}

\begin{patverse}
\vin\vin Great kings, dukes, and lords,\\
Have laid by their swords,\\
This our myst’ry to put a good grace on,\\
And ne’er been asham’d\\
To hear themselves nam’d\\
With a free and an accepted Mason.
\end{patverse}

\begin{patverse}
\vin\vin Antiquity’s pride\\
We have on our side,\\
It makes each man just in his station\\
There’s nought but what’s good,\\
To be understood\\
By a free and an accepted Mason.
\end{patverse}

\begin{patverse}
\vin\vin We’re true and sincere,\\
We’re just to the fair,\\
They’ll trust us on ev’ry occasion ;\\
No mortal can more\\
The ladies adore\\
Than a free and an accepted Mason,
\end{patverse}

\begin{patverse}
\vin\vin Then join hand in hand,\\
To each other firm stand,\\
Let’s be merry and put a bright face on ;\\
What mortal can boast\\
So noble a toast\\
As a free and an accepted Mason.
\end{patverse}
\end{dcverse}

\musictitle{CEASE YOUR FUNNING.}

This and \textit{You'll think, ere many days ensue} are the only two songs
in The, Beggars’ Opera of which the original, or at least earlier,
names are not given in the first edition. \textit{You'll think ere many days}
has been handed down through the
\end{fixedpage}%664
\pagebreak

\begin{fixedpage}%665
\rectoheader

traditions of the stage as one of the snatches of old songs sung
by Ophelia in \textit{Hamlet}; but we have now no sufficient evidence to prove
the origin of \textit{Cease your funning}. There are half-sheet songs to the
same tune, such as “Charming Billy,” commencing, “When the hills and
lofty mountains;” but it is not certain that any are of earlier date
than \textit{The Beggars’ Opera}. In all probability, Gay was unable to
recollect the names of the two airs, although they were familiar to
him.

The excessive popularity of Gay’s song caused the adoption of its
title when the tune was introduced in other ballad-opèras, as, for
instance, in \textit{The Fashionable Lady, or Harlequin’s Opera}, 1730.

In the year 1833, the late John Parry published “The Welsh
Melody, sung with such distinguished approbation by Miss Kelly, in
her entertainment called \textit{Dramatic Recollections}, written in Welsh
and English, and adapted to the favorite air, \textit{Llwyn on, or The Ash
Grove}, by John Parry, editor of Welsh and Scotch Melodies.” To this
he added the following note:—“The celebrated song of \textit{Cease your
funning}, in \textit{The Beggars’ Opera}, is this beautiful and simple melody
ornamented.” The air of \textit{Cease your funning} is really quite as simple
as the Welsh melody; and, if there has been any copying, it is
infinitely more probable that the Welsh air was derived from \textit{Cease
your funning}, than that a tune noted down seventy years after \textit{The
Beggars’ Opera} had been publicly performed in Wales, should prove to
be the original of one of its melodies. The Welsh air which resembles
\textit{Cease your funning} is neither to be found in the \textit{Ancient British
Music}, collected by John Parry and Evan Williams in 1742, nor in
\textit{British Harmony}, “being a collection of ancient Welsh airs,” by John
Parry of Ruabon Denbighshire, in 1781. It was first printed by Edward
Jones, in his \textit{Bardic Museum}, 1802, and the resemblance there is
confined to the first part, and is not very strong; but Parry
increased it by slightly altering the first and entirely changing the
second part of the tune. Again, who can say that this Welsh air is
old? Jones entitles it, “Llwynn-onn, the name of Mr. Jones’s mansion,
near Wrexham, in Denbighshire.” I do not know how long Mr. Jones’s
mansion has stood, or who composed the Welsh air; but it is not
improbably the production of some grateful bard whom Mr. Jones
entertained there. Let it be remembered that the succession of Welsh
bards continues to the present day, and that some of their
compositions are incorporated in collections of Welsh music, without
any marks to distinguish the new from the old. Edward Jones was Bard
to George IV.; the late John Parry, “Bardd Alaw.” Parry printed “A
Selection of Welsh Melodies,” in three volumes; and in the first, as
well as in the second, included an air named \textit{Cader Idris}, after a
venerable mountain in Merionethshire. Some years after this
publication, Mr. Charles Matthews sang the air on the stage, with
great success, and Parry then claimed it as his own composition. He
was too honorable a man to make such a claim, if not really his own,
but \textit{Cader Idris} (alias \textit{Jenny Jones}), might still be passing for an
ancient Welsh melody, if the copyright had not become thus suddenly
and unexpectedly valuable. These matters are not always revealed to
the public.
\end{fixedpage}%665
\pagebreak

\begin{fixedpage}%666
\versoheader

\musictitle{THE BUDGEON IT IS A DELICATE TRADE.}
This tune is now familiarly known as \textit{There was a jolly Miller}; it
is also the vehicle of a harvest-supper song, “Here’s a health unto
our Master;” but a still earlier name (or at least a name under which
I find it at an earlier date) is \textit{The budgeon it is a delicate trade}.

\textit{The budgeon it is a delicate trade}, is contained in \textit{The Triumph
of Wit, or Ingenuity displayed}, and in \textit{A new Canting Dictionary},
\&c., “with a complete collection of Songs in the canting dialect,”
8vo., 1725. From this it appears that a “budge” is a thief who slips
into houses in the dark, to steal cloaks and other clothes. The
dialect of the song might be intelligible to a police-officer, but
would not be so to the general reader, as the following sample will
shew:—

\settowidth{\versewidth}{For when that we have bit the blow,}
\begin{dcverse}
\begin{altverse}
\vleftofline{“}The budgeon it is a delicate trade, \\
And a delicate trade of fame,\\
For when that we have bit the blow, \\
We carry away the game.\\
But if the cully nab us, and \\
The lurries from us take,\\
O then he rubs us to the whit.\\
Though we are not worth a make.”
\end{altverse}
\end{dcverse}

The tune was introduced into several of the ballad-operas (\textit{The
Quaker’s Opera}, 1728; \textit{The Devil to pay}; \textit{The Fashionable Lady, or
Harlequin’s Opera}; \&c.), under the name of \textit{The budgeon it is a
delicate trade}.

One stanza of \textit{There was a jolly Miller} was sung in \textit{Love in a
Village}, 1762; and it is therefore supposed to have been written by
Bickerstaffe, but he appropriated so many songs from other sources,
without acknowledgement, that this may also have been introduced.
However, I have not seen the words in print before 1762.
\end{fixedpage}%666
\pagebreak

\begin{fixedpage}%667
\rectoheader

The following version is from \textit{The Convivial Songster}, 1782:—

\settowidth{\versewidth}{He danc’d and he sang from morn till night, no lark so blithe as he.}
\begin{scverse}
\vleftofline{“}There was a jolly miller once liv'd on the river Dee;\\
He danc’d and he sang from morn till night, no lark so blithe as he.\\
And this the burden of his song for ever us’d to be—\\
I care for nobody, no, not I, if nobody cares for me.

I live by my mill, God bless her! she's kindred, child, and wife;\\
I would not change my station for any other in life.\\
No lawyer, surgeon, or doctor, e’er had a groat from me—\\
I care for nobody, no, not I, if nobody cares for me.

When Spring begins its merry career, oh! how his heart grows gay;\\
No summer drought alarms his fears, nor winter’s sad decay;\\
No foresight mars the miller’s joy, who's wont to sing and say—\\
Let others toil from year to year, I live from day to day.

Thus like the miller, bold and free, let us rejoice and sing;\\
The days of youth are made for glee, and time is on the wing.\\
This song shall pass from me to thee, along this jovial ring—\\
Let heart and voice and all agree to say Long live the King.”
\end{scverse}

About two years ago, the following stanzas were sent to the
editor of \textit{The Illustrated London News} to be printed among the
“Memorabilia” in that journal. They were found written on the
fly-leaf of a volume of Dryden’s \textit{Miscellany Poems} (printed in 1716),
and the finder supposed them to be the original song of \textit{The jolly
Miller}:—

\indentpattern{0101013}
\settowidth{\versewidth}{He work’d and sang from mom till night,}
\begin{dcverse}
\begin{altverse}
There was a jolly miller once \\
Lived on the river Dee;\\
He work’d and sang from morn till night,\\
No lark more blithe than he.\\
And this the burden of his song \\
For ever used to be—\\
I care for nobody, no, not I,\\
If nobody cares for me.
\end{altverse}

\begin{patverse}
The reason why he was so blithe,\\
He once did thus unfold—\\
The bread I eat my hands have earn’d;\\
I covet no man’s gold;\\
I do not fear next quarter-day;\\
In debt to none I be.\\
I care for nobody, \&c.
\end{patverse}

\begin{patverse}
A coin or two I’ve in my purse, \\
To help a needy friend;\\
A little I can give the poor,\\
And still have some to spend.\\
Though I may fail, yet I rejoice, \\
Another’s good hap to see.\\
I care for nobody, \&c.
\end{patverse}

\begin{altverse}
So let us his example take,\\
And be from malice free;\\
Let every one his neighbour serve. \\
As served he’d like to be.\\
And merrily push the can about, \\
And drink and sing with glee;\\
If nobody cares a doit for us,\\
Why not a doit care we.
\end{altverse}
\end{dcverse}

When the harvest-supper song is sung to this tune, it is
generally in a major key. I have copies so noted down in Kent, in
Suffolk, and in Wiltshire; and it is printed in that form in “Old
English Songs as now sung by the peasantry of the Weald of Surrey and
Sussex” (collected by the Rev. John Broadwood), harmonized by G. A.
Dusart.

The following are the harvest-supper words as commonly sung:—

\settowidth{\versewidth}{I hope his soul, whenever he diea, to heav’n may go to rest;}
\begin{scverse}
\vleftofline{“}Here’a a health unto our master, the founder of the feast;\\
I hope his soul, whenever he dies, to heav’n may go to rest;\\
That all his works may prosper, whatever he takes in hand;\\
For we are all his servants, and all at his command.\\
Then, drink—boys—drink—and see you do not spill,\\
For if you do, you shall drink two, it is our master’s will.
\end{scverse}
\end{fixedpage}%667
\pagebreak

\begin{fixedpage}%668
\versoheader

\settowidth{\versewidth}{He takes his work so light in hand, can leave it when he please;}
\begin{scverse}
Now harvest it is ended, and supper it is past,\\
To our good mistress’ health, boys, a full aud flowing glass,\\
For she is a good woman, and makes us all good cheer:\\
Here’s to our mistress’ health, boys, so all drink off your beer.\\
Then drink—boys—drink—and see you do not spill,\\
For if you do, you shall drink two, it is our master’s will.”
\end{scverse}

Sometimes the following verse is added, or the song commences
with it:—

\begin{scverse}
\vleftofline{“}Here’s a health unto the woodcutter, that lives at home at ease;\\
He takes his work so light in hand, can leave it when he please;\\
He takes the withe and winds it, and lays it on the ground,\\
And round the faggot he binds it,—so let his health go round.\\
Then drink—boys—drink—and pass it round to me,\\
The longer we sit here and drink, the merrier we shall be.”
\end{scverse}

The tune of \textit{The jolly Miller} was one of those harmonized by
Beethoven for George Thomson, in 1824. Thomson included it in his
collection of Scotch songs, not because it was Scotch, but on account
of “its merited popularity, and the great additional interest which
Beethoven has conferred upon it by his truly original and
characteristic accompaniments.”

The following are the words now usually sung:—

\begin{scverse}
I love my mill, she is to me like parent, child, and wife;\\
I would not change my station for any other in life:\\
Then push, push, push the bowl, my hoys, and pass it round to me;\\
The longer we sit here and drink, the merrier we shall be.
\end{scverse}

\end{fixedpage}%668
\pagebreak

\begin{fixedpage}%669
\rectoheader

\musictitle{HOW STANDS THE GLASS AROUND.}

This is commonly called General Wolfe’s song, and is said to have
been \textit{written} by him on the night before the battle of Quebec; but
this tradition is sufficiently, disproved by a copy of the tune,
under the title of “Why, soldiers, why?” in \textit{The Patron, or The
Statesman's Opera}, performed at the little theatre in the Haymarket,
in 1729. Probably General Wolfe \textit{sang} it on that occasion.

The words and music are contained in \textit{Vocal Music, or The
Songster's Companion}, ii. 49 (1775), and were introduced by Shield in
\textit{The Siege of Gibraltar}. In \textit{Vocal Music} they are entitled “A Soldier’s
Song.”

\indentpattern{20222022022022202202}
\settowidth{\versewidth}{Send us to Him who made us, boys,}
\begin{dcverse}
\begin{patverse}
\vin\vin Why, soldiers, why\\
Should we be melancholy, boys?\\
Why, soldiers, why?\\
Whose business ’tis to die!\\
What! sighing? fie!\\
Damn fear, drink on, be jolly boys!\\
’Tis he, you, or I;\\
Cold, hot, wet, or dry,\\
We’re always bound to follow, boys,\\
And scorn to fly.\\
’Tis but in vain,\\
(I mean not to upbraid you, boys),\\
’Tis but in vain\\
For soldiers to complain:\\
Should next campaign\\
Send us to Him who made us, boys,\\
We’re free from pain;\\
But should we remain,\\
A bottle and kind landlady\\
Cures all again.
\end{patverse}
\end{dcverse}


\end{fixedpage}%669
\pagebreak

\begin{fixedpage}%670
\versoheader

\musictitle{THE JOLLY FELLOW.}

This convivial song is still popular, and there are several
extant versions of the words. They are founded on the following, from
Fletcher’s play, \textit{The Bloody Brother, or Rollo, Duke of Normandy}, act
ii., sc. 2:—

\settowidth{\versewidth}{Who drinks well loves the commonwealth;}
\begin{dcverse}
\vleftofline{“}Drink to-day and drown all sorrow, \\
You shall, perhaps, not do it to-morrow; \\
Best, while you have it, use your breath, \\
There is no drinking after death.  

Wine works the heart up, wakes the wit,\\
There is no cure ’gainst age but it; \\
It helps the head-ache, cough, and tisic, \\
And is for all diseases physic.

Then let us swill boys for our health; \\
Who drinks well loves the commonwealth;\\
And he that will to bed go sober,\\
Falls with the leaf still in October.”
\end{dcverse}

One of the current versions is as follows:—

\settowidth{\versewidth}{Come', landlord, fill a flowing bowl, until it does run over;}
\begin{scverse}
\vleftofline{“}Come', landlord, fill a flowing bowl, until it does run over;\\
To-night we all will merry be, to-morrow we’ll get sober.

He that drinks strong beer, and goes to bed mellow,\\
Lives as he ought to live, and dies a hearty fellow.

Punch cures the gout, the colic, and the tisic,\\
And is to all men the very best of physic.

He that drinks small beer, and goes to bed sober,\\
Falls, as the leaves do, that die in October.

He that courts a pretty girl, and courts her for his pleasure,\\
Is a fool to marry her without store of treasure.

Now let us dance and sing, and drive away all sorrow,\\
For perhaps we may not meet again to-morrow.”
\end{scverse}

Another version will be found in \textit{Vocal Miscellany}, vol. ii.,
1734. It is more like the following, which is from a half-sheet copy
printed with the music. Owing to the numerous repetitions of words,
only two lines are here taken up by the tune; but I believe it is
more frequently sung with four lines, and then without repetitions.

\begin{scverse}
Wine cures the gout, the cholic, and the tisic,\\
And is for all men the very best of physic.
\end{scverse}
\end{fixedpage}%670
\pagebreak

\begin{fixedpage}%671
\rectoheader

\settowidth{\versewidth}{But he that drinks all day, and goes to bed mellow,}
\begin{scverse}
He that drinks small beer, and goes to bed sober, \\
Falls, as the leaves do, that die in October.

But he that drinks all day, and goes to bed mellow, \\
Lives as he ought to do, and dies a hearty fellow.
\end{scverse}

\musictitle{A MAY-DAY DANCE.}

From the second volume of \textit{The Dancing Master}.

After the year 1717, the celebrations of May-day in London were
limited to the dances of milkmaids (described ante p. 282, and in
Hone’s \textit{Every-day Book}, i. 570), and to the Jack-in-the-green of the
sweeps.

The great May-pole in the Strand (which-stood close to the site
of the church of St. Mary-le-Strand) was given to Sir Isaac Newton in
1717, and removed to Wanstead, where it was used in raising the
largest telescope then known. (Pennant’s \textit{London}.)

\musictitle{COUNTRY COURTSHIP.}

The tune of Country Courtship is contained in the third volume of
\textit{The Dancing Master}, in the third volume of Walsh’s \textit{New Country
Dancing Master}, and in many later publications. It is in common use
at the present time.

The first part is nearly the same as \textit{There was an old fellow at
Waltham Cross}, which was sung to the tune of \textit{In Taunton Dean} (see p.
262). It is also curious that the words of \textit{In Taunton Dean}, being in
eight line stanzas, do not suit the version of the tune \textit{In Taunton
Dean} as printed in the ballad-operas of \textit{Flora} and \textit{The Jovial Crew} so
well as this, because they require the repetition of the four bars.

Some copies of \textit{Country Courtship} differ in the second part.
Having printed it one way in the \textit{National English Airs}, I now adopt
the other, which is better
\end{fixedpage}%671
\pagebreak

\begin{fixedpage}%672
\versoheader

known at the present time. The “song entitled The Country
Courtship, beginning, ‘Honest Sir, give me thy hand, ’” was entered
at Stationers’ Hall, to John Back, March 31, 1688. I have not
discovered the words, and have therefore adapted the first stanza of
\textit{In Taunton Dean}, which will be found entire in \textit{The Merry Musician, or
a Cure for the Spleen}, i. 306.

\musictitle{GOOD MORROW, GOSSIP JOAN.}

This tune was introduced into \textit{The Beggars’ Opera}, \textit{The Court
Legacy}, \textit{The Oxford Act}, and other ballad-operas. The song, \textit{Good
morrow. Gossip Joan}, is in \textit{Pills to purge Melancholy} , vi. 315; and
“Happy Dick,” to the same tune, in Watts’s \textit{Musical Miscellany}, iv.
36, and in \textit{Vocal Miscellany}, vol. i., 1734. The latter commences
thus:—
\end{fixedpage}%672
\pagebreak

\begin{fixedpage}%673
\rectoheader

\indentpattern{01016}
\settowidth{\versewidth}{Whence comes it, neighbour Dick,}
\begin{scverse}
\begin{patverse}
\vleftofline{“}Whence comes it, neighbour Dick, \\
That you, with youth uncommon,\\
Have serv’d the girls this trick,\\
And wedded an old woman?\\
Happy Dick.”
\end{patverse}
\end{scverse}

\settowidth{\versewidth}{I had drink would please a king,}
\begin{dcverse}
\begin{patverse}
My sparrow’s flown away,\\
And will no more come to me;\\
I’ve broke a glass to-day,\\
The price will quite undo me,\\
Gossip Joan.
\end{patverse}

\begin{patverse}
l’ve lost a \textit{Harry} groat\\
Was left me by my granny;\\
I cannot find it out,\\
I’ve search’d in ev’ry cranny.\\
Gossip Joan.
\end{patverse}

\begin{patverse}
I’ve lost my wedding ring,\\
That was made of silver gilded;\\
\columnbreak
I had drink would please a king,\\
But that my cat has spill’d it,\\
Gossip Joan.
\end{patverse}

\begin{patverse}
My pocket is cut off,\\
That was full of sugar-candy;\\
I cannot stop my cough \\
Without a gill of brandy,\\
Gossip Joan.
\end{patverse}

\begin{patverse}
Let’s to the ale-house go,\\
And wash down all our sorrow,\\
My griefs you there shall know,\\
And we’ll meet again to-morrow, \\
Gossip Joan.
\end{patverse}
\end{dcverse}

\musictitle{A HEALTH TO ALL HONEST MEN.}

This tune is contained in the second volume of \textit{The Dancing
Master}, 1718 and 1728; in Watts’s \textit{Musical Miscellany}, iii. 142, 1730;
in the ballad-opera of \textit{The Jovial Crew}; in \textit{The Convivial Songster},
1782; \&c.

The old song called “Love and Innocence,” beginning, “My days
have been so wondrous free,” is apparently ,the same air, slightly
altered.
\end{fixedpage}%673
\pagebreak


\begin{fixedpage}%674
\versoheader

\indentpattern{010100331}
\settowidth{\versewidth}{For our king and our church, our laws and right,}
\begin{dcverse}
\begin{patverse}
Tis not owning a whimsical name \\
That will prove a man loyal or just; \\
Let him fight for his country’s fame,\\
Be impartial at home, if in trust;\\
’Tis this that proves him an honest soul; \\
His health we’ll drink in a brimful bowl; \\
Then leave off all debate,\\
No confusion create:\\
Here’s a health to all honest men!
\end{patverse}

\begin{patverse}
When a company’s honestly met,\\
With intent to be jolly and gay,\\
Their drooping souls for to whet,\\
And drown the fatigues of the day, \\
What madness it is thus to dispute,\\
When neither side can his man confute!\\
When you’ve said what you dare, \\
You’re but just where you were:\\
Here’s a health to all honest men!
\end{patverse}
\begin{patverse}

Then agree, ye true Britons, agree;\\
Never quarrel about a nickname;\\
Let your enemies tremblingly see\\
That an Englishman’s always the same.\\
For our king and our church, our laws and right,\\
Let’s lay by all feuds, and straight unite.\\
O then, why care a fig \\
Who’s a Tory or Whig?\\
Here’s a health to all honest men!
\end{patverse}

\end{dcverse}
\end{fixedpage}%674
\pagebreak

\begin{fixedpage}%675
\rectoheader

\musictitle{ONE EVENING, HAVING LOST MY WAY.}
 
This tune is contained in the second, volume of \textit{The Dancing
Master}, 1718 and 1728; in Walsh’s \textit{Compleat Country Dancing Master}, i.
13; and in the following ballad-operas:—\textit{The Beggars’ Opera}, \textit{The
Grub-Street Opera}, and \textit{The Welsh Opera, or The Grey Mare the better
Horse}. There are also numerous extant half-sheet copies of words and
music.

Sometimes the air is entitled “The happy Clown,” and sometimes
“Walpole, or the happy Clown;” but it is now more generally known by
the words, “I’m like a skiff on ocean toss’d,” in The Beggars’ Opera.

The song of “The happy Clown,” commencing, “One evening, having
lost my way,” was written by Mr. Burkhead. In \textit{The Convivial Songster},
“As one bright sultry summer’s day” is printed to the tune, and those
words may be older than any of the above.
\end{fixedpage}%675
\pagebreak

\begin{fixedpage}%676
\versoheader

\musictitle{PRETTY POLLY OLIVER.}

This is the tune of an old ballad, entitled \textit{Polly Oliver’s
Ramble}, which is still in print in Seven Dials. It commences thus:—

\settowidth{\versewidth}{Nor father nor mother shall make me false prove,}
\begin{scverse}
\vleftofline{“}As pretty Polly Oliver lay musing in bed,\\
A comical fancy came into her head;\\
Nor father nor mother shall make me false prove,\\
I’ll ’list for a soldier, and follow my love.”
\end{scverse}
The old song on the Pretender, beginning,

\begin{scverse}
\vleftofline{“}As Perkin one morning lay musing in bed,\\
The thought of three kingdoms ran much in his head;’’
\end{scverse}
appears to be a parody on it.

The words of the following are by Lord Cantalupe.

\settowidth{\versewidth}{When like lightning she darts through each throbbing vein;}
\begin{scverse}
I repair’d to my Reason, entreating her aid,\\
She paus’d on my case, and each circumstance weigh’d; \\
Then gravely pronounc’d, in return to my pray’r,\\
That Hebe was fairest of all that was fair.

That’s a truth, replied I, I’ve no heed to be taught,\\
I came for a council to find out a fault;\\
If that’s all, quoth Reason, return as you came,\\
To find fault with Hebe would forfeit my name.

What hopes then, alas! of relief from my pain,\\
When like lightning she darts through each throbbing vein; \\
My senses surpris’d, in her favour took arms,\\
And Reason confirms me a slave to her charms.
\end{scverse}

\end{fixedpage}%676
\pagebreak

\begin{fixedpage}%677
\rectoheader

\musictitle{DERRY DOWN.}

To this air George Alexander Stevens wrote the song of \textit{Liberty
Hall}, which is printed in \textit{The Muses’ Delight}, 1757, and in his
Collection of Songs, 1772. It was introduced in \textit{Midas}, 1764, and is
now well known as the tune of George Colman’s song, “Lodgings for
Single Gentlemen,” contained in his \textit{My Night Grown and Slippers}, and
in \textit{Broad Grins}.

The song of Liberty Hall begins thus:—

\indentpattern{000022}
\settowidth{\versewidth}{Old Homer,—but with him what have we to do?}
\begin{scverse}
\begin{patverse}
\vleftofline{“}Old Homer,—but with him what have we to do?\\
What are Grecians or Trojans to me or to you?\\
Such heathenish heroes no more I’ll invoke,\\
Choice spirits, assist me! attend, hearts of oak!\\
Down a down down, down a down down,\\
Down a down down derry, down a down down.”
\end{patverse}
\end{scverse}
The first verse of Colman’s Song is here printed with the tune..

\settowidth{\versewidth}{Or like two single gentlemen roll’d into one. Down a down, \&c}
\begin{scverse}
Will Waddle, whose temper was studious and lonely,\\
Hir’d lodgings that took Single Gentlemen only;\\
But Will was so fat, he appear’d like a ton,—\\
Or like two single gentlemen roll’d into one. Down a down, \&c.
\end{scverse}

\end{fixedpage}%677
\pagebreak

\begin{fixedpage}%678
\versoheader

\musictitle{BENBOW, THE BROTHER TAR’S SONG.}

This is taken from a broadside printed with the tune in the first
half of the last century; but the words are evidently much corrupted.
For instance, the line, “With their noise,” at the end of the fourth
stanza, cannot be correct, as it ought to rhyme with “French,” and
the same words are again substituted, at the end of the last stanza,
for a line that should rhyme with “crying out.”

The tune is both quaint and characteristic.

Mr. Halliwell prints the words in \textit{Early Naval Ballads of England},
from a broadside published at Salisbury, by Fowler, a noted
ballad-printer of the last century, but the same corruptions are in
both copies.

Admiral Benhow was called “the brother tar” because he rose, from
being a common sailor, to the rank of Admiral. His father was Colonel
John Benbow, a Shropshire gentleman and loyal Cavalier, who
distinguished himself at the battle of Worcester, and was there taken
prisoner. At the Restoration he could obtain no better post than one
of subordinate rank in the Tower of London at a salary of eighty
pounds a year, and left his family penniless.

Portraits of Admiral Benhow may he seen at Hampton Court Palace
and in the town-hall at Shrewsbury.
\end{fixedpage}%678
\pagebreak

\setlength{\fixedpagewidth}{400pt}
\begin{fixedpage}%679
\rectoheader

\settowidth{\versewidth}{’Twas the Ruby and Noah's Ark fought the French:}
\begin{dcverse}
Brave Benbow he set sail \\
For to fight, for to fight,\\
Brave Benbow be set sail for to fight:\\
Brave Benbow he set sail,\\
With a fine and pleasant gale,\\
But his Captains they turn’d tail \\
In a fright, in a fright.

Says Kirby unto Wade,\\
“I will run, I will run,”\\
Says Kirby unto Wade, “I will run:\\
I value not disgrace,\\
Nor the losing of my place,\\
My enemies I’ll not face \\
With a gun, with a gun.”

'Twas the Ruby and Noah’s Ark\\
Fought the French, fought the French,\\
’Twas the Ruby and Noah's Ark fought the French:\\
And there was ten in all, \\
Poor souls they fought them all,\\
They valued them not at all,\\
Nor their noise, nor their noise.

It was our Admiral's lot\\
With a chain shot, with a chain shot.\\
It was our Admiral’s lot, with a chain shot \\
Our Admiral lost his legs,\\
And to his men he begs,\\
“Fight on, my boys,” he says,\\
“'Tis my lot, 'tis my lot”

While the surgeon dress’d his wounds.\\
Thus he said, thus he said, \\
While the surgeon dress’d his wounds, thus he said:\\
“Let my cradle now in haste \\
On the quarter-deck be plac’d,\\
That my enemies I may face \\
Till I’m dead, till I’m dead.”

And there hold Benbow lay \\
Crying out, crying out,\\
And there bold Benbow lay, crying out:\\
“Let us tack about once more,\\
We’ll drive them to their own shore,\\
I value not half a score.\\
Nor their noise, nor their noise.”
\end{dcverse}

\musictitle{THE WOMEN ALL TELL ME I’M FALSE TO MY LASS.}

This is to be found on many broadsides with music, printed
between the years 1740 and 1750. The words are included in \textit{The
Wreath}, second edition, 1753 (and perhaps in the first edition, which
I have not seen); also in \textit{The Bullfinch}, \textit{The Convivial Songster}, and
many similar collections. It is still one of the most popular of
English bacchanalian songs.

“The English,” says Camden, “who of all the Northern nations, had
been till now the moderatest drinkers, and most commended for their
sobriety, learned in these Netherland wars, first to drown themselves
with immoderate drinking, and by drinking others’ healths to impair
their own. And, ever since, the vice of drunkenness hath so diffused
itself over the whole nation, that in our days first it was fain to
be restrained by severe laws.” (\textit{Reign of Elizabeth}, p. 263.)

“Though I am not old in comparison of other ancient men,” says
Sir Richard Hawkins, “I can remember Spanish wine rarely to be found
in this kingdom. Then, hot, burning fevers were not known in England,
and men lived many more years. But since Spanish sacks have been
common in our taverns, which (for conservation\textsuperscript{a}) is mingled with lime
in its making, our nation complaineth of calenturas, of the stone,
the dropsy, and infinite other diseases not heard of before this wine
came in frequent use, or but very seldom. To confirm which my belief,
I have heard one of our learnedest physicians affirm that he thought

\begin{dcfootnote}
\textsuperscript{a} This passage explains Falstaffs exclamation, “You rogue,
here’s lime in this sack,” which has led many to suppose sack to have
been what is termed a “dry” wine. That it was not so is proved by an
act of parliament in the reign of Henry VIII., which I have not seen
quoted anywhere. It is entitled “An acte to set prices upon wines
tu be sold by retaile,” and enacts that “No maner of persons should
sel by retayle any Gascoyne, Guion, or Frenche wines above eyght pens
the gallon; that is to saye, a peny the pinte, two pence the quarte,
four pence
the pottle, and eight pence the gallon: And that Malmesies,
Romneis, Sackes, \textit{nor other swete wines}, shoulde be solde by retayls
above twelve pence the gallon, sixpence the pottle, three pence the
quarte, thre halfepence the pinte.” (Anno 34, 35, cap. vii.,
1543-4.) The progressive increase in the prices of wine may be noted
by the various proclamations, one of which, in 1632, fixes the price
of “Sacks and Malagas” at £13 per butt, or ninepence the quart; and
another in 1676, at tenpence per pint.
\end{dcfootnote}
\end{fixedpage}%679
\pagebreak
\setlength{\fixedpagewidth}{360pt}
\begin{fixedpage}%680
\versoheader

there died more persons of drinking wine, and using hot spices in
their meats and drinks, than of all other diseases.” \textit{Observations on
his Voyage to the South Sea}, p. 103, foL, 1622.

\settowidth{\versewidth}{But, though she could smile, yet in truth she could frown:}
\begin{dcverse}
Although I have left her, the truth I’ll declare;\\
I believe she was good, and I’m sure she was fair;\\
But goodness and charms in a bumper I see,\\
That make it as good and as charming as she.

My Chloe had dimples and smiles I must own;\\
But, though she could smile, yet in truth she could frown:\\
But tell me, ye lovers of liquor divine,\\
Did you e’er see a frown in a bumper of wine?

Her lilies and roses were just in their prime;\\
Yet lilies and roses are conquer’d by time:\\
But in wine, from its age, such a benefit flows,\\
That we like it the better the older it grows.

They tell me, my love would in time have been cloy’d,\\
And that beauty’s insipid when once ’tis enjoy’d:\\
But in wine I both time and enjoyment defy;\\
For the longer I drink the more thirsty am I.

Let murders, and battles, and history prove\\
The mischiefs that wait upon rivals in love;\\
But in drinking, thank heaven, no rival contends,\\
For the more we love liquor, the more we are friends.

She, too, might have poison’d the joy of my life,\\
With nurses, and babies, and squalling and strife:\\
But my wine neither nurses nor babies can bring;\\
And a big-bellied bottle’s a mighty good thing.

We shorten our days when with love we engage,\\
It brings on diseases and hastens old age;\\
But wine from grim death can its votaries save,\\
And keep out t’other leg, when there’s one in the grave.

Perhaps, like her sex, ever false to their word,\\
She had left me, to get an estate, or a lord;\\
But my bumper (regarding nor title nor pelf)\\
Will stand by me when I can’t stand by myself.

Then let my dear Chloe no longer complain;\\
She’s rid of her lover, and I of my pain:\\
For in wine, mighty wine, many comforts I spy;\\
Should you doubt what I say, take a bumper and try.
\end{dcverse}

\end{fixedpage}%680
\pagebreak

\begin{fixedpage}%681
\rectoheader

\musictitle{THE BREAST KNOT.}

This is still a favorite Morris Dance in some parts of Derbyshire
and Lancashire. It is contained in Thompson’s, and several other
Collections of Country Dances, subsequent to \textit{The Dancing Master}.

\musictitle{ON YONDER HIGH MOUNTAINS.}

This is one of the airs introduced in \textit{The Cobbler’s Opera}, 1729,
and in \textit{Silvia, or The Country Burial}, 1731.

I have not found any song or ballad commencing, “On yonder high
mountains,” but “Over hills and high mountains” was a very popular
ballad in the latter part of the preceding century, and the tune
often referred to.

This is evidently a ballad-tune, and as the metre of “Over hills
and high mountains” exactly suits it, as well as the character of the
words, it is probably the right air.

Copies of “Over hills and high mountains” are in the Bagford
Collection (643, m. 10, p. 165), and in the Pepys Collection, iii.
165. The ballad is entitled “The Wandering Maiden, or True Love at
length united,” \&c., “\textit{to an excellent new tune}.” “Printed by J.
Deacon, at the Angel in Guiltspur Street, without Newgate.” It
commences thus:

\settowidth{\versewidth}{Over hills and high mountains long time have I gone;}
\begin{scverse}
\vleftofline{“}Over hills and high mountains long time have I gone;\\
Ah! and down by the fountains, by myself all alone;\\
Through bushes and briars, being void of all care,\\
Through perils and dangers for the loss of my dear.”
\end{scverse}
These lines are quite a paraphrase of “Love will find out the
way,” and were it not that the tune is said to be “new,” and at a
date when “Love will find out the way” was extremely popular, I
should infer them to have been intended for that air. However, \textit{Over
hills and high mountains} is often referred to as a distinct tune.

In \textit{The True Loyalist, or Chevalier’s Favourite}, 12mo., 1779, is a
Jacobite parody of \textit{Over hills and high mountains}, but there are too
many feet in the lines. It commences thus:—

\settowidth{\versewidth}{And down by yon murm’ring crystal fountain,}
\begin{dcverse}
\begin{altverse}
\vleftofline{“}Over yon hills, and yon lofty mountain,\\
Where the trees are clad with snow,\\
And down by yon murm’ring crystal fountain,\\
Where the silver streams do flow;
\end{altverse}

\begin{altverse}
There fair Flora sat complaining,\\
For the absence of onr King,\\
Crying, Charlie, lovely Charlie,\\
When shall we two meet again?”
\end{altverse}
\end{dcverse}

I suppose “fair Flora” to be intended for Flora Macdonald.
\end{fixedpage}%681
\pagebreak

\begin{fixedpage}%682
\versoheader

In the Roxburghe Collection, ii. 470, is “True love without
deceit,” \&c., “to the tune of \textit{Over hills and high mountains};”
commencing—

\settowidth{\versewidth}{Unfortunate Strephon! well may’st thou complain,}
\begin{scverse}
\vleftofline{“}Unfortunate Strephon! well may’st thou complain,\\
Since thy cruel Phillis thy love doth disdain.”
\end{scverse}

Also (ii. 508), “The Wandering Virgin, or The coy lass well
fitted: Or the answer to The Wandering Maiden,” \&c., “To a pleasant
new tune, \textit{Over hills and high mountains}.”

Both the above were printed by P. Brooksby. The first stanza of
the latter ballad is here printed with the tune.

\musictitle{FAREWELL, MANCHESTER.}

This tune was composed in the early part of the last century by
the Rev. Wm. Felton, prebendary of Hereford. It formed a part of one
of his Concertos, and was afterwards published with variations as
Felton’s Gavot. It is said to have been played by the troops of
Charles Stuart on quitting Manchester in December, 1745: also when
the unfortunate Manchester youth, Jemmy Dawson,
\end{fixedpage}%682
\pagebreak

\begin{fixedpage}%683
\rectoheader

was led to the scaffold in 1746. About the same period some words
were written to it, entitled “A song made on the Peace,” a copy of
which, bearing the prefix of “\textit{Farewell, Manchester},” and printed with
the music, is in the British Museum (G. 307, p. 230). The song of
Farewell, Manchester, is, in all probability, irrecoverably lost.

The tune has continued in public favour ever since. Felton’s
variations on it were kept in print till within the last thirty or
forty years, and the Song on the Peace, “Fill, fill, fill the glass,”
was sung to the air, within the memory of several of my musical
friends, as arranged for three voices. The following are the words:—

\indentpattern{01010012}
\settowidth{\versewidth}{Lovely nymphs, jolly swains;}
\begin{dcverse}
\begin{patverse}
\vleftofline{“}Fill, fill, fill the glass,\\
Briskly put it round;\\
Joyful news at last \\
Let the trumpet sound.\\
Join, with lofty strains, \\
Lovely nymphs, jolly swains;\\
Peace and plenty shall again \\
With wealth be crown’d.
\end{patverse}

\begin{patverse}
Come, come, come, sweet peace,\\
Ever welcome found;\\
Let all discord cease,\\
Harmony abound. \hfill Join with, \&c.
\end{patverse}
\end{dcverse}

The tune is now well known by T. Haynes Bayly’s song, “Give that
wreath to me,” which Sir John Stevenson adapted to it about
twenty-five years ago, and which was arranged for three voices by the
late T. Phillips. Charles Mackay also wrote a song, “Through the
summer night,” which was published in \textit{The Illustrated London News},
arranged to the air by Sir Henry Bishop.

Two versions of the tune were printed in \textit{National English Airs}.
It is here coupled with the first stanza of Haynes Bayly’s song.
\end{fixedpage}%683
\pagebreak

\begin{fixedpage}%684
\versoheader

\musictitle{SWEET, IF YOU LOVE ME.}

This tune was introduced into several of the ballad-operas, such
as \textit{The Fashionable Lady}, 1730; \textit{The Livery Rake}, 1733; \textit{The Woman of
Taste}, 1738; \&c. It was also printed on broadsides to a “Dialogue
between Sly and Lovett, at Fielding’s Booth, at Bartholomew Fair.”

There are four different songs to it, “Sweet, if you love me,
tell me so;” “Sweet, if you love me, come away;” “Sweet, if you love
me, smiling turn;” and “Sweet, if you love me, let me go.”

\musictitle{TOUCH THE THING.}

\textit{Touch the thing} being a vulgar song with a good tune, Miss Catley
sang other words to it in \textit{The Golden Pippin}, and with great success.
From that time (1773) comic songs have been written to it without
number.

“Push about the jorum” is the burden of the song in The Golden
Pippin, and the tune is now generally known by that name.

In the Roxburghe Collection, iii. 765, and in Ritson’s \textit{Durham
Garland}, are copies of “A new song called Hark to Winchester! or the
Yorkshire Volunteers’ Farewell to the good folks of Stockton. Tune,
\textit{Push about the jorum}.” The Roxburghe copy was printed at Stockton.

Among the late songs which were sung to the tune, and attained
popularity, is one on the coronation of her present Majesty, and a
second on an order from the Admiralty which obliged sailors to cut
off their pigtails. The latter is entitled “The British Sailor’s
Lament,” and was written by Mr. William Ball.

The first stanza of “Hark to Winchester!” is here adapted to the
tune.
\end{fixedpage}%684
\pagebreak


\begin{fixedpage}%685
\rectoheader

\musictitle{THERE LIVES A LASS UPON THE GREEN.}

This is one of the airs which were introduced in the ballad-opera
of \textit{The Jovial Crew} in 1731. I have not found the original words, but
a song commencing in a very similar manner, “A lass there lives upon
the green,” was set to music by Mr. Courteville. On comparing the
two, I find Courteville’s music to be quite different, and therefore
the words were probably different also.

The song in \textit{The Jovial Crew} is thus prefaced by Rachel, who sings
it: “I remember an old song of my nurse’s, every word of which she
believ’d as much as her Psalter, that used to make me long, when I
was a girl, to be abroad in a moonlight night.”
\end{fixedpage}%685
\pagebreak

\begin{fixedpage}%686
\versoheader

\indentpattern{0101001}
\settowidth{\versewidth}{But where they danc’d their cheerful round}
\begin{scverse}
\begin{patverse}
But where they danc’d their cheerful round\\
The morning would disclose,\\
For where their nimble feet do bound\\
Each flow’r unbidden grows:\\
The daisy, fair as maids in May,\\
The cowslip in his gold array.\\
And blushing violet, ’rose.
\end{patverse}
\end{scverse}

\musictitle{RULE, BRITANNIA.}

The music of this noble “ode in honour of Great Britain,” which,
according to Southey, “will be the political hymn of this country as
long as she maintains her political power,” was composed by Dr. Arne
for his masque of \textit{Alfred}, and first performed at Cliefden House, near
Maidenhead, on August 1, 1740. Cliefden was then the residence of
Frederick, Prince of Wales, and the occasion was to commemorate the
accession of George I., and in honour of the birthday of the young
Princess Augusta. The masque gave so much satisfaction that it was
repeated on the following night.

Dr. Arne afterwards altered it into an opera, and it was so
performed at Drury Lane Theatre, on March, 20, 1745, for the benefit
of Mrs. Arne. In the advertisements of that performance, and in
another of the following month, Dr. Arne entitles \textit{Rule, Britanni}a, “\textit{a
celebrated ode};” from which it may be
\end{fixedpage}%686
\pagebreak

\begin{fixedpage}%687
\rectoheader

inferred that (although the entire masque had not been performed
in public), \textit{Rule, Britannia}, had then attained popularity. Some
detached pieces of the masque had been sung in Dublin, on the
occasion of Arne’s visit with his wife, but no record of any other
public performances has hitherto been discovered.

The words of the masque were by Thomson and Mallet, but Thomson
seems to have taken the lead in the affair, since, in the newspapers
of the day, he alone is mentioned as the author. In the book, the
names of Thomson and Mallet are both given.

The authorship of \textit{Rule, Britannia}, has been ascribed to Thomson,
by Ritson and other authorities, but a claim has recently been made
for Mallet, on the strength of an advertisement prefixed by him to an
altered edition of \textit{Alfred}, in 1751, after Thomson’s death. He writes
thus: “According to the present arrangement of the fable, I was
obliged to reject a great deal of what I had written in the other;
neither could I retain of my friend’s part more than three or four
single speeches and a part of one song.” It appears, however, that
three stanzas of Rule, Britannia, were retained, and three others
added by Lord Bolingbroke: such an argument in favour of Mallet is
therefore very inconclusive. The only point in it is, that Mallet
uses the word “song” in the advertisement, and retains the title of
“ode” in the book; but \textit{Rule, Britannia}, may with equal accuracy be
described as a song. Would Mallet have allowed Lord Bolingbroke so to
mutilate the most successful song in the piece, if it had been his
own? For internal evidence in favour of Thomson, see his poems,
“Britannia,” and Liberty.” Further information about \textit{Rule, Britannia},
will be found in Dr. Dinsdale’s excellent edition of Mallet’s works,
and in the pages of \textit{Notes and Queries}, including a refutation of M.
Schœlcher’s charge against Arne of having copied from Handel. See
2nd Series, Nos. 86, 99, 108, 109, 111, and 120.

\textit{Rule, Britannia}, soon became a favorite with the Jacobite party.
Ritson mentions a Jacobite parody, of which he was unable to procure
a copy, but the chorus
ran thus:— 

\settowidth{\versewidth}{Rise, Britannia! Britannia, rise and fight!}
\begin{scverse}
\vleftofline{“}Rise, Britannia! Britannia, rise and fight!\\
Restore your injured monarch’s right.”
\end{scverse}
Another will be found in \textit{The True Royalist; Or Chevalier’s
favorite, being a collection of Elegant Songs never before printed}.
It is entitled “A Song. Tune, \textit{When Britain first, at heav’n’s
command}. As the book is not easily procured, the song is subjoined:—

\indentpattern{010100}
\settowidth{\versewidth}{Come, Britons, sing! Britannia draw thy sword,}
\begin{dcverse}
\begin{patverse}
“Britannia, rouse at heav’n’s command!\\
And crown thy native Prince again;\\
Then Peace shall bless thy happy land,\\
And Plenty pour in from the main:\\
Then shalt thou be—Britannia, thou shalt be\\
From home and foreign tyrants free.
\end{patverse}

\begin{patverse}
Behold, great Charles ! thy godlike son,\\
With majesty and sweetness crown’d;\\
His worth th' admiring world doth own,\\
And fame’s loud trump proclaims the sound.\\
Thy captain him, Britannia, him declare!\\
Of kings and heroes he's the heir.
\end{patverse}

\begin{patverse}
The second hope young Hero claims,\\
Th’ extended empire of the main;\\
His breast with fire and courage flames,\\
With Nature’s bounds to fix thy reign.\\
He (Neptune-like), Britannia, will defy\\
All but the thunder of the sky.
\end{patverse}

\begin{patverse}
The happiest states must yield to thee,\\
When free from dire corruption’s thrall;\\
Of land and sea thou’lt Emp’ror be,\\
And ride triumphant round the ball:\\
Britannia, unite! Britannia must prevail,\\
Her powerful hand must guide the scale.
\end{patverse}

\begin{patverse}
Then, Britons, rouse! with trumpets’ sound\\
Proclaim this solemn, happy day! [June 12]\\
Let mirth, with cheerful music crown’d,\\
Drive sullen thoughts and cares away!\\
Come, Britons, sing! Britannia draw thy sword,\\
And usе it for thy rightful lord!”
\end{patverse}
\end{dcverse}

\end{fixedpage}%687
\pagebreak

\begin{fixedpage}%688
\versoheader

This is followed by another, commencing—

\indentpattern{010100}
\settowidth{\versewidth}{Heav’n smil’d with pleasure, with pleasure on the land.}
\begin{scverse}
\begin{patverse}
\vleftofline{“}When our great Prince, with his choice band,\\
Arriv'd from o’er the azure main,\\
Heav’n smil’d with pleasure, with pleasure on the land.\\
And guardian Angels sung this strain:\\
Go, brave hero; brave hero, boldly go,\\
And wrest thy sceptre from thy foe.”
\end{patverse}
\end{scverse}

The music of \textit{Rule, Britannia}, was first printed at the end of the
masque of \textit{The Judgment of Paris}, which appeared before \textit{Alfred},—Arne
having composed the music to both.
\end{fixedpage}%688
\pagebreak

\begin{fixedpage}%689
\versoheader

\indentpattern{01016}
\settowidth{\versewidth}{Blest Isle! with matchless beauty crown’d.}
\begin{dcverse}
\begin{patverse}
The nations not so blest as thee,\\
Must in their turns to tyrants fall;\\
While thou shalt flourish great and free,\\
The dread and envy of them all.\\
Rule, Britannia, \&c.
\end{patverse}

\begin{patverse}
Still more majestic shalt thou rise,\\
More dreadful from each foreign stroke;\\
As the loud blast that tears the skies,\\
Serves but to root thy native oak.\\
Rule, Britannia, \&c.
\end{patverse}

\begin{patverse}
Thee haughty tyrants ne’er shall tame,\\
All their attempts to bend thee down\\
\columnbreak
Will but arouse thy generous flame;\\
But work their woe, and thy renown.\\
Rule, Britannia, \&c.
\end{patverse}

\begin{patverse}
To thee belongs the rural reign ;\\
Thy cities shall with commerce shine;\\
All thine shall be the subject main,\\
And every shore it circles thine.\\
Rule, Britannia, \&c.
\end{patverse}

\begin{patverse}
The Muses, still with freedom found,\\
Shall to thy happy coast repair;\\
Blest Isle! with matchless beauty crown’d.\\
And manly hearts to guide the fair.\\
Rule, Britannia, \&c.
\end{patverse}
\end{dcverse}

\musictitle{BEGONE, DULL CARE.}

The late T. Dibdin informed me that the great popularity of
\textit{Begone, dull Care}, may be dated from its revival in a pantomime
ballet called \textit{William Tell}, performed at Sadler’s Wells in 1793. His
own first dramatic attempt, \textit{The Rival Loyalists}, was produced on the
same night.

The tune seems to have been derived from \textit{The Queen’s Jigg}, which
is contained in \textit{The Dancing Master}, in and after 1701, and was
reprinted in \textit{National English Airs}.

One verse of the words is in Playford’s \textit{Pleasant Musical
Companion}, Part II., 1687, set as a catch by John Jackson, and two are
to be found in \textit{The Syren}, \textit{The Merry Companion}, \textit{The Aviary}, \textit{The Buck’s
Delight}, and other collections of the last century.

The stanza in the \textit{Pleasant Musical Companion} is as follows:—

\settowidth{\versewidth}{’Tis long thou hast liv’d with me, and fain thon wouldst me kill,}
\begin{scverse}
\vleftofline{“}Begone, old Care, and I prithee be gone from me,\\
For i’faith, old Care, thee and I shall never agree;\\
’Tis long thou hast liv’d with me, and fain thou wouldst me kill,\\
But i’faith, old Care, thou never shalt have thy will.”
\end{scverse}

The next version is—

\settowidth{\versewidth}{Long time you have been vexing me, and fain you would me kill,}
\begin{scverse}
\vleftofline{“}Begone, old Care, I prithee be gone from me;\\
Begone, old Care, you and I shall never agree;\\
Long time you have been vexing me, and fain you would me kill,\\
But i’faith, old Care, thou never shalt have thy will.

Too much care will make a young man look grey,\\
And too much care will turn an old man to clay:\\
Come, you shall dance, and I will sing, so merrily we will play,\\
For I hold it one of the wisest things to drive old Care away.”
\end{scverse}

The words seem to have been suggested by a song of much earlier
date; one very popular in the reigns of Elizabeth and James I.:—

\settowidth{\versewidth}{Care, away go thou from me,}
\begin{scverse}
\vleftofline{“}Care, away go thou from me,\\
I am no fit mate for thee,” \&c.
\end{scverse}

This is to be found, with music, in a manuscript of the sixteenth
century, in Trinity College, Dublin (F. 5, 13, No. 5); and in another
(dated 1639), which passed through the hands of Cranston, Leyden, and
Heber, and is now in the Advocate’s Library, Edinburgh.

The first time I find “Begone, dull Care,” instead “Begone, \textit{old} Care,” is in

\end{fixedpage}%689
\pagebreak

\begin{fixedpage}%690
\versoheader

\textit{The Buck's Delight} for 1798. It is there stated to be as “sung
this season at Sadler’s Wells;” and has a third stanza, which it is
not necessary to reprint.

\settowidth{\versewidth}{For I hold it one of the wisest things to drive dull care away.}
\begin{scverse}
Too much care will make a young man turn grey,\\
And too much care will turn an old man to clay.\\
My wife shall dance and I will sing, so merrily pass the day,\\
For I hold it one of the wisest things to drive dull care away.
\end{scverse}

\musictitle{GEE HO, DOBBIN.}

This song of \textit{Gee ho, Dobbin}, was printed with the tune on
broadsides, one of which is in the Burney Collection, British Museum;
also in \textit{Apollo’s Cabinet, or The Muses' Delight}, ii. 232, 1757. This
last-named Collection was printed in Liverpool, by John Sadler, in
Harrington Street.

\textit{Laugh and lay down} is another name for the tune, and it derives
it from a song commencing—

\settowidth{\versewidth}{With Ombre, with Commerce, Picquette, and Quadrille.”}
\begin{scverse}
“While others attempt heavy minutes to kill,\\
With Ombre, with Commerce, Picquette, and Quadrille.”
\end{scverse}

This was also printed on broadsides with the tune.
\end{fixedpage}%690
\pagebreak

\begin{fixedpage}%691
\rectoheader

\textit{Gee ho, Dobbin}, was introduced in \textit{Love in a Village}, 1762, to the
words, “If you want a young man with a true honest heart.” It is also
to be found in Thompson’s and many other collections of
country-dances.

Oliver Goldsmith, in his description of “The Club of Choice
Spirits,” makes the pimple-nosed gentleman sing \textit{Gee ho, Dobbin}.

The new and old versions of the tune differ considerably, but the
limit of space forbids the printing more than one. The following is
the popular form: —

\musictitle{GOD SAVE THE QUEEN.}

The simplicity and grandeur of our national air is too
universally admitted, to require comment. Its adoption in Hanover,
Brunswick, Prussia, Saxony, Weimar, Sweden, and Russia (at least till
1833, when the new Russian anthem was composed), sufficiently proves
that its admiration is not confined to England. In Switzerland it is
the air of the federal cantons, “Rufst du, mein Vaterland,” and is
occasionally played as a voluntary in the churches. In Germany it is
“Hail to thee in the crown of victory” (“Heil dir im Sieges Kranz”);
or a song of united Germany, for God, Freedom, and Fatherland
(“Brause, du Freiheitsang”). The Austrians sing Haydn’s hymn, “Gott
erhalte Franz den Kaiser!” but it has been justly remarked that,
“with all its melody and sweetness, the Austrian hymn has too much of
the psalm in it; it wants the manly, majestic, full-hearted boldness
of the strains, in which we are accustomed to express not more our
respect for our monarch, than our love for our country.” 

Much
research has been bestowed on the endeavour to ascertain the origin
both of the words and the music; and to collect all that has been
said, would fill
\end{fixedpage}%691
\pagebreak

\begin{fixedpage}%692
\versoheader

volumes, and far exceed the limits that can he here devoted to
it. Dismissing, therefore, many of the vague and unsupported
assertions that have at various times been made, the enquiry will be
confined to a few of the favourite theories which have obtained more
or less credence as they have appeared to be supported by proof.

1. In the \textit{Souvenirs de la Marquise de Créqui}, “Grand Dieu, sauve
le Roi” is said to have been sung by the nuns of St. Cyr to Louis
XIV., the music to have been composed by Lully, and Lully’s music to
be the same as our “God save the King.” This story has been recently
revived in Raikes’s \textit{Diary}. In answer, it is only necessary to refer
the reader to the June number of the \textit{Quarterly Review}, for 1834,
where he may satisfy himself, that the memoirs of Madame de Créqui
are fictitious, and that the work is a modern novel. The music of
Lully is a myth; and as to Handel’s having procured a copy when in
France, and palmed it on George I. and the English nation as his own
composition, not one syllable can be found throughout his life or
writings, of his having made such a claim. On the contrary, his
musical amanuensis, John Christopher Smith, is the very person who
ascribes the authorship to Henry Carey.

2. Mr. Pinkerton, in his \textit{Recollections of Paris}, ii. 4, says that
“the supposed national air is a mere transcript of a Scottish
anthem.” Pinkerton’s “Scottish anthem” is an English Christmas Carol,
copied into a Scotch publication. See “Remember, O thou man,” \textit{ante} i.
373.

3. A writer in The Gentleman’s Magazine, for March, 1796, p.
208, says, “The original tune of God save the King, the tune at least
which evidently furnished the subject of it, is to be found in a book
of Harpsichord lessons, published by Purcell’s widow, in Dean’s Yard,
Westminster.” The work referred to is “A choice Collection of Lessons
for the Harpsichord or Spinnet, composed by the late Mr. Henry
Purcell.” Printed for Mrs. Frances Purcell, \&c., 1696. The following
is the lesson:—

It resembles “God save the King,” but is not more like it, than
“Franklin is fled away” (\textit{ante} i. 370), Dr. Bull’s “Ayre,” and several
others.

4. In 1849, the Rev. W. H. Henslowe published new words of his
own to “the royal anthem of England,” and claimed the music for
Anthony Young, organist of Allhallows, Barking, in the reign of James
II. This was on the authority of Mrs. Henslowe, then living, who
stated that she received “a legacy of £100, on the death of Mrs. Arne
(6th October, 1789), being the accumulated
\end{fixedpage}%692
\pagebreak

\begin{fixedpage}%693
\rectoheader

amount of a yearly pension of £30, awarded to Mrs. Arne (as the
eldest surviving descendant of Anthony Young, the composer of the
Royal anthem) by King George III., through the representation of
Francis Godolphin, then Marquis of Carmarthen, afterwards Duke of
Leeds.” I suppose the words in the parenthesis, “as the eldest
surviving descendant,” \&c., to be Mrs. Henslowe’s inference; but if
not, it would appear that George III. granted a pension to the widow
of Dr. Arne, not on account of her deceased husband’s great eminence
as a composer, but because she was the granddaughter of a musician
who composed a national anthem \textit{for the Stuarts}. Mrs. Henslowe does
not explain how, if Mrs. Arne’s grandfather composed the air, Dr.
Arne could have been so ignorant of the fact, as to have said, when
interrogated upon the subject, that “he had not the least knowledge,
nor could he guess at all, who was the author or the composer.” Even
if Mrs. Arne only made the discovery after her husband’s death, Dr.
Burney, who was a pupil of Dr. Arne, would surely have heard of it;
but he also expressed his inability to give any account of the
authorship. This claim is too feebly supported to receive any serious
attention.

The enquiry into the three remaining claims, will be best
prefaced by the accounts that were given at the time of the first
public performance of “God save the King” at the theatres. In the
month of September, 1745, and during the rebellion, it was sung both
at Drury Lane and Covent Garden theatres; Dr. Arne harmonizing it for
Drury Lane, and his pupil, Burney, for Covent Garden. The first of
these performances is thus noticed in \textit{The Daily Advertiser} of Monday,
Sept. 30, 1745:—“On Saturday night last, the audience at the Theatre
Royal, Drury Lane, were agreeably surprised by the gentlemen
belonging to that house performing the anthem of \textit{God save our noble
King}. The universal applause it met with,—being encored with repeated
huzzas,—sufficiently denoted in how just an abhorrence they hold the
arbitrary schemes of our insidious enemies, and detest the despotick
attempts of Papal power.” Next, in \textit{The General Advertiser} of Oct. 2,
1745:—“At the Theatre in Goodman’s Fields, by desire, \textit{God save the
King}, as it was performed at the Theatre Royal in Drury Lane, with
great applause.” Thirdly,—among the published letters of “that
dramatic enthusiast,” Benjamin Victor (i. 118, 8vo., 1776), is one
addressed to Garrick, bearing the date of Oct., 1745, in which he
says, “The stage (at both houses) is the most \textit{pious}, as well as the
most \textit{loyal} place in the three kingdoms. Twenty men appear at the end
of every play; and one, stepping forward from the rest, with uplifted
hands and eyes, begins singing, to an old anthem tune, the following
words:—

\indentpattern{0010001}
\settowidth{\versewidth}{Long to reign over us,}
\begin{dcverse}
\begin{patverse}
“O Lord, our God, arise,\\
Confound the enemies\\
Of George our King!.\\
\columnbreak
Send him victorious,\\
Happy and glorious,\\
Long to reign over us,\\
God save the King!
\end{patverse}
\end{dcverse}

which are the very words, and music, of an old anthem that,was
sung at St. James’s Chapel, for King James the Second, when the
Prince of Orange landed to deliver us from popery and slavery;
which God Almighty, in his goodness, was pleased \textit{not} to grant.”
\end{fixedpage}%693
\pagebreak

\begin{fixedpage}%694
\versoheader

The above letter is the authority for the fifth claim, and it
derives some support from the evidence of Dr. Burney, who tells us
that, when Dr. Arne was applied to for information about it, he said,
“He had not the least knowledge, nor could he guess at all, who was
either the author or the composer, but that it was a received opinion
that it was written for the Catholic Chapel of James II.” Dr. Burney
stated to the Duke of Gloucester, that “the earliest copy of the
words with which we are acquainted, begins ‘God save great James our
King’” (see \textit{Morning Post}, Nov. 2, 1814); and in Rees’s \textit{Cyclopœdia},
he says, “We belieye that it was written for King James II., while
the Prince of Orange was hovering over the coast; and when he became
king, who durst own or sing it?” It also appears that Dr. Benjamin
Cooke, organist of Westminster Abbey, from about 1780 to 1790, had
heard it sung, “God save great James our King.” (See letter of E. J.,
in the \textit{Gentleman’s Magazine}, Jan. 20, 1796.)

It is singular that neither Hawkins nor Burney should have
mentioned “God save the King” in their respective histories of music.
In the year 1745, Hawkins was twenty-six years of age, and Burney
nineteen. Burney came to London the year before, and was then a
performer in the orchestra. He therefore had peculiar facilities for
obtaining information, if he had desired it. No interest seems to
have been taken in the enquiry, until some years after those
histories were published.

The sixth claim is on behalf of Henry Carey. About the year 1795,
when a pension of £200 a year had been granted to Charles Dibdin, on
account of the favourable influence which his naval songs had over
the British seamen, George Savile Carey made a journey to Windsor in
the hope of a similar recompense. He relates in his \textit{Balnea}, that he
was advised to beg the interference of a gentleman residing in the
purlieus of Windsor Castle, that he would be kind enough to explain
this matter rightly to the Sovereign, “thinking it not improbable
that some consideration might have taken place, and some little
compliment be bestowed on the offspring of one ‘who had done the
state some service.’” He was met with this answer, “Sir, I do not
see, because your father was the author of \textsc{God save the King}, that
the king is under any obligation to his son.” G. S. Carey could not
assert anything respecting the authorship from his own knowledge,
having been born in 1742, and his father having died in 1743.

Henry Carey is the first person who is recorded to have sung “God
save the King” in public, and he was in the habit of writing both the
words and music of his songs. John Christopher Smith, who composed
the music to an opera called \textit{Teraminta}, of which Carey wrote the
drama, asserts that Carey took the words and music of “God save the
King” to him, to correct the base. His evidence is contained in a
letter from Dr. Harington, the celebrated physician and amateur
musician of Bath, addressed to G. S. Carey, and dated June 13th,
1795:

\begin{quotation}
“Dear Sir,—The anecdote you mention, respecting your father’s
being the author and composer of the words and music of ‘God save
the King,’ is certainly true. That most respectable gentleman, my
worthy friend and patient, Mr. Smith, has often told me what follows:
viz., ‘that your father came to him with the words and music,
\end{quotation}
\end{fixedpage}%694
\pagebreak


\begin{fixedpage}%695
\rectoheader

\begin{quotation}
desiring him to correct the bass, which was not proper; and at
your father’s request, Mr. Smith wrote another bass in correct
harmony.’ Mr. Smith, to whom I read your letter this day, repeated
the same account, and on his authority I pledge myself for the truth
of the statement.—\textsc{H. Harington}.”
\end{quotation}

The proof of Carey’s having sung it in 1740 (five years before it
became generally known), rests upon the evidence of Mr. Townsend, who
in 1794 stated to Mr. John Ashley, of Bath, that his father dined
with Henry Carey at a tavern in Cornhill, in the year 1740, at a
meeting convened to celebrate Admiral Vernon’s capture of Portobello,
and that “Carey sang it on that occasion.” He adds that “the applause
he received was very great, especially when he announced it to be his
own composition.” (Vide Ashley’s letter to the Rev. W. L. Bowles,
1828.) This receives some confirmation from the writer of a letter to
the \textit{Gentleman’s Magazine}, in 1796, who says, “The first time I ever
heard the anthem of ‘God save the King,’ was about the year 1740, on
some public occasion at a tavern in Cornhill.”

7. Now as to the claim of Dr. John Bull.

This was first suggested by the writer of the following letter in
the \textit{Gentleman’s Magazine}, dated from W—m Hall, Sep. 9, 1816.

\begin{quotation}
“In Ward’s Lives of Professors of Gresham College, page 200, it
is stated that Dr. John Bull was, in 1596, chosen first Professor of
Music in Gresham College, and that he was chief organist to King
James I.; and at p. 201, it states that in 1607 he resigned his
professorship, but lived in England until 1613, when he went abroad,
and did not return: then follows a list of his musical works in
manuscript, in the possession of Dr. Pepusch; among them, at p. 205,
is ‘God save the King.’ I think it is somewhere said, that these
manuscripts of Dr. Bull, as in Dr. Pepusch’s collection, were placed
in Sion College. If this be so, the reference is easy: and if the
tune there, be the same with the popular air all Englishmen hear with
pleasure, the enquiry is set at rest; and it will be no stretch of
imagination to suppose, that it was brought forward in compliment to
King James I., when, according to the anecdote, Dr. Bull played
before him at Merchant Tailors’ Hall, upon a small pair of organs. If
the tune be different, Mr. Carey will have a stronger claim from the
enquiry to be considered as the author of the favorite air: one
claimant will be struck off the list.”—“R.S.”
\end{quotation}

The late Richard Clark, one of the Gentlemen of Her Majesty’s
Chapels Royal, had published an account of “God save the King” in the
preface to “Poetry of the most favorite Glees, Madrigals, Duets,
\&c.” two years before the appearance of this letter, and he had then
given Henry Carey the credit of the authorship; but in 1822, he
produced another \textit{Account of the National Anthem}, transferring it to
Dr. Bull, without having even seen the manuscript.\textsuperscript{a} The errors in
Clark’s book have already been so frequently exposed, that it will
only be necessary to allude to one of his mis-representations in the
present enquiry. At p. 57, he

\begin{dcfootnote}
\textsuperscript{a} This is proved not only by his note on “God save the
King,” in the book, but also by the following passage in
his circular addressed to the “Masters, Wardens,” \&c., of
the City Companies, one of which is now before me, dated
November, 1841. After alluding to his publication of
1822, he says, "I continued my enquiries until eventually 
I was enabled to obtain a sight of, and finally to
purchase (in the handwriting of the composer, Dr. John
Bull) this long-lost manuscript,” Clark purchased it
in 1840.


\end{dcfootnote}
\end{fixedpage}%695
\pagebreak

\begin{fixedpage}%696
\versoheader

cites a copy of the music in a manuscript book, once the property
of Thomas Britton, the musical small-coal-man, as a proof “that the
air was known some years before James II. was crowned, the date of
the book being 1676.” This manuscript (now in the library of the
Sacred Harmonic Society) was then in the possession of John Sydney
Hawkins, F.S.A., by whom it was shown to me. It bears the following
inscription, “Deane Montcage, given him by his father, 1676,” but the
music could not have been written even in the time of Thomas Britton,
who died in 1714. It is in the same handwriting as “Sweet Annie fra’
the sea beach came,” by Dr. Greene, several pieces by Bononcini and
Handel, and among others, “The dead March” in \textit{Saul}. Handel’s oratorio
of \textit{Saul} was first published in 1740. Clark was quite aware that the
music of “God save the King” could not have been written there, at
the date of the book, for Hawkins had” drawn his attention to the
preceding pieces, which are in the same hand-writing. His
mis-statement has been copied without acknowledgment, in “An account
of the Grand Musical Festival at York, by John Crosse, Esq., F.S.A.,
F.R.S.”

Instead of making proper search for Dr. Bull’s manuscript, Clark
contented himself with printing the list of its contents from Ward’s
Lives of the Professors of Gresham College, and when he arrived at
the piece entitled “God save the King,” adding the following curious
note:—“Here then is a positive, incontrovertible, and undeniable
claim \textit{by} Dr. Bull, to the tune of “God save the King,” as composed by
him in honour of King James I. It must be the same tune which is sung
at the present time, because it has never yet appeared that there
were two of a similar description. This circumstance alone proves
that fact, at least it must be so admitted until another is produced,
supported by evidence sufficiently strong to invalidate the title
claimed by the former.”

Dr. Bull’s manuscript was not in Sion College, but in the
possession of Dr. Kitchener, who entirely disproved Clark’s theory,
by publishing Dr. Bull’s “God
save the King.” It is a piece on four notes, <insert music here>
corresponding
with the four words, “God save the King,” and was probably
intended to represent the cry when the king appeared. These four
notes are repeated over and over, with twenty-six different bases,
and occupy seven pages of the manuscript.

At the death of Dr. Kitchener, Clark purchased the book for £20,
and then announced that the air of “God save the King” was really
contained in it. It is a curious fact (of which he could not have
been aware when he published his account) that an “ayre” at page 98
of the manuscript is very like our “God save the King.” The piece
which is therein entitled “God save the King,” is at page 66, and the
same which Kitchener published. When Clark played the “ayre” to me,
with the book before him, I thought it to be the original of the
national anthem; but afterwards, taking the manuscript into my own
hands, I was convinced that it had been tampered with, and the
resemblance strengthened, the
\end{fixedpage}%696
\pagebreak

\begin{fixedpage}%697
\rectoheader

sharps being in ink of a much darker colour\textsuperscript{a} than other parts.
The additions are very perceptible, in spite of Clark’s having
covered the face of that portion with varnish. In its original state,
the “ayre” commenced with these notes:—

The \textit{g} being natural, the resemblance to “God save

the King” does not strike the ear, but by making the \textit{g} sharp, and
changing the whole from an old scale without sharps or flats, into
the modern scale of A major (three sharps), the tune becomes
\textit{essentially} like “God save the King.” When I reflected further upon
the matter, it appeared very improbable that Dr. Bull should have
composed a piece for the organ in the modern key of A major. The most
curious part of the resemblance between Dr. Bull’s ayre and “God save
the King” is, that the first phrase consists of six bars, and the
second of eight, which similarity does not exist in any other of the
airs from which it is supposed to have been taken. It is true, that
the eight bars of the second phrase are made out by holding on the
final note of the melody through two bars, therefore it differs
decidedly from all copies of our more modern tune; but the words
maybe sung to Dr. Bull’s “ayre” by dividing the time of the long
notes, —in fact, it has been so performed in public.

My readers may be curious to see the “ayre” as it was sung before
the late King of Hanover, at the Concerts of Ancient Music, and at
other public concerts; and I am enabled, through the kindness of Dr.
Rimbault, to gratify them. The late R. Clark lent the voice-parts,
which had been used on those occasions, to Dr. Rimbault, for
performance at his lectures on music in Liverpool. Dr. Rimbault
copied them in score for his own use (to conduct the performance),
and has favoured me with the following transcript.

\begin{dcfootnote}
\textsuperscript{a} With regard to the alterations that have been made in
this manuscript, I offered in the pages of \textit{Notes and Queries}
(2nd S., No. 74) that if Mrs. Clark would submit the
manuscript to any competent judges of writing, and they
should decide that it has not been tampered with, I would
forfeit £10 to a charity. This offer was communicated to
Mrs. Clark, and declined. The manuscript had been in
the possession of Dr. Pepusch until 1752, and " God save
tire King" was performed at both the great theatres in
1745. Although some may possess rare books and not
acquaint themselves with their contents, Dr. Pepusch can
not be classed among the number; indeed, he gave Ward
the catalogue of contents for his \textit{Lives of the Gresham
Professors}, and taught his wife to play from old books of
this kind. Had the resemblance of Dr, Bull's “ayre”
been then as great to “God save the King” as it now is
I can scarcely imagine it could have escaped his observation. 
Again, while in Dr. Kitchener’s possession, the

\end{dcfootnote}
\end{fixedpage}%697
\pagebreak

\setlength{\fixedpagewidth}{400pt}
\begin{fixedpage}%698
\versoheader

From what I have said above, it will be understood, that in this
copy the “ayre” has been transposed, and changed into the modern key
of G major. The first note of the tune should (in this key) be D,
and, instead of four G’s at the end, the first G in the thirteenth
bar should be held through that and the fourteenth, to the
termination of the tune. I have other doubts about the accuracy of
the copy, but cannot resolve them from memory, and the permission to
compare it with the original has been refused.

If we could suppose the sharps to have been omitted by the error
of the copyist, (for it is not the autograph of the composer, as
stated by Clark, but a Dutch transcript of his compositions,
throughout which he is styled Dr. \textit{Jan} Bull,\textit{a}) we might imagine our
“God save the King” to have been copied imperfectly from it, but
there are two other treatments of the same subject in the manuscript,
which do not bear out the supposition.

One particular point to which I would draw attention, is, that
all the research devoted to the subject, has hitherto failed in
adducing a single instance of such a hymn or anthem having been sung
on a public occasion before 1740. “We have an abundance of national
songs, anthems, hymns, \&c., including many in which these words have
been introduced, but not this. As to the \textit{cries} of “God save the
King,” and “Long live the King,” they are to be found in the
translations of the Old Testament, and most abundantly in the history
of our country. We have an anthem for Henry VII., and his Queen,
Elizabeth of York—

\settowidth{\versewidth}{God save King Henrie, wheresoever he be,}
\begin{scverse}
\vleftofline{“}God save King Henrie, wheresoever he be,\\
And for Queen Elizabeth now pray we,\\
And all her noble projeny.”
\end{scverse}

In the “State Papers published under the authority of Her
Majesty’s Commission,” we find among the Lord Admiral’s Orders on the
10th of August, 1545— “No. 11. The watch wourde in the night shal be
thus, ‘God save King Henrye,’ thother shall aunswer, ‘And long to
raign over us.’”

Mr. J. G. Nichols, in his \textit{London Pageants}, quotes a “God save the
King” for Edward VI., from Leland’s \textit{Collectanea}, iv., 310. It
commences—

\indentpattern{001}
\settowidth{\versewidth}{King Edward, King Edward,}
\begin{scverse}
\begin{patverse}
\vleftofline{“}King Edward, King Edward,\\
God save King Edward,\\
King Edward the Sixth,” \&c,
\end{patverse}
\end{scverse}

\begin{dcfootnote}
manuscript was submitted to the scrutiny of Edward Jones, the
Welsh Bard, who wrote out one of the pieces for Dr. Kitcheoer in
modern notation. Finally, in 1840, T looked through' it to find any
popular tunes, when asked by Mr. Edward Walsh to estimate its value.
This waa prior to its passing into the hands of Mr. Clark.

\textsuperscript{a} In the course of making enquiries at Antwerp, as to whether any
of Dr. Bull’s manuscripts were still in the library of that Cathedral
(which, I regret to say, was answered in the negative), I received
through M. Jules de Glimes, the following letter from a distinguished
antiquary, the Chevalier Léon de Burbure. It will serve to correct
some of the mistakes about Dr. Bull's history, and it shows how many
English were at Antwerp at the time. The letter beara date the 19th
June, 1856:—

“Impossible de rien vous dire sur le manuscrit dont vous me
parlez dans votre lettre d'hier. J’ignore si jamais la Cathédrale
d’Anvers en à possédé du Docteur John Bull, mais en tout cas il n’en
reste plus de traces depuis longtemps. Les seuls faits relatifs à
John Bull que j’ai decouverte sont: qu’il devint organiste de Notre
Dame à Anvers en 1617, en remplacement de feu Rumold
Waelrant; qu'en 1620 il habitoit la maison joignant l’Eglise du
côté de la Place Verte; actuellement habitée par le Concierge de
Notre Dame; qu’il mourût le 12 ou 13 Mars, 1628, et fût enterré le 15
du même mois; que pendant le temps qu’il fût Organiste à Anvers, de
grandes ameliorations furent apportés aux orgues, et qu'il surveilla
les travaux, en y coopérant même. Enfin qu’il dut sa nomination à la
place d'Anvers, en grande partie à la raccommendation du Magistrat
de cette ville. Sa signature est à peu prés celleci\dots . Dans les
comptes et quittances Flamandes on l’appelle Doctor \textit{Jan} Bull. Dr.
John Bull n’etoit, du reste, pas le seul Anglais qui residat à Anvers
à la même epoque: je trouve parmi les prêtres chapelains ‘Joannes
Beake (en Latin Beckius), Anglus, 1598 à 1607; Joannes Starkeus, 1613
à 1636; Anthoinus Sanderus, Anglus, 1611 à 1622; Adamus Gordonius,
Scottus, 1627 à 1640; Thomas Covert, 1598; Edmundus Lewkenor, 1598;
Gulielmus Clederoe, 1598; Robertus Bruckius, 1598; Fitzgerald,
1600.’” 

In printing a translation of this letter in \textit{The Musical
World}, the Irish editor of that periodical added “Irlandus” after
Fitzgerald’s name. Ha may have guessed rightly, but it is not so
stated in the letter.
\end{dcfootnote}
\end{fixedpage}%698
\pagebreak
\setlength{\fixedpagewidth}{360pt}

\begin{fixedpage}%699
\rectoheader

For James I. we have “A song of Praise and Thanksgiving to God
for the King’s Majesty’s Happy Reigne;” reprinted by Dr. Rimbault in
\textit{Notes and Queries} (2nd S. No. 126), with the burthen:—

\settowidth{\versewidth}{God save King James, and still pull downe,}
\begin{scverse}
\textsuperscript{“}God save King James, and still pull downe,\\
All those that would annoy his crowne:” 
\end{scverse}
as well as “A Song or
Psalme of Thanksgiving, in remembrance of our great deliverance from
the Gun-powder treason, the fift of November, 1605,” commencing— 

\settowidth{\versewidth}{O Lord! we have continuall cause}
\begin{scverse}
\begin{altverse}
\vleftofline{“}O Lord! we have continuall cause\\
Thy mercies to remember.”
\end{altverse}
\end{scverse}
This is among the proclamations, \&c., in the Library of the
Society of Antiquaries. 

In Naile’s Account of the Queen’s
Entertainment at Bristol, 1613, we find—

\settowidth{\versewidth}{And still these words, ‘God save our Queene,’ re-echoed in the skie.”}
\begin{scverse}
\vleftofline{“}The bels most joyously did ring with musick’s symphony,\\
And still these words, ‘God save our Queene,’ re-echoed in the skie.”
\end{scverse}
And at James’s entertainment at both Universities, 1614-15—
\begin{scverse}
\vleftofline{“}Oxford cried, ‘God Save the King,’ and ‘Bless him,’ too, cried some,\\
But Cambridge men, more learnedly, ‘Behold, the King doth come.’”
\end{scverse}
For Charles I., in Herbert’s \textit{Vox Secunda Populi}, 4to., 1641—
\settowidth{\versewidth}{God save our King and the Lord Chamberlaine?”}
\begin{scverse}
\vleftofline{“}Have you not seen men holloo forth this straiue,\\
God save our King and the Lord Chamberlaine?”
\end{scverse}
And in \textit{The Last Age's Looking Glass}—
\settowidth{\versewidth}{Let Charles’s glorie through England ring,}
\begin{scverse}
\vleftofline{“}Let Charles’s glorie through England ring,\\
Let subjects say, ‘God save the King.’”
\end{scverse}
At Charles the Second’s coronation (and perhaps at others
preceding it), the anthem sung by the quire, was, “Sadoc the priest
and Nathan the prophet anointed Solomon King, and all the people
rejoiced and said, God save the King.” 

The favorite national songs
for all the Stuarts from Charles I. downwards, were, “The King shall
enjoy his own again” (or, “enjoys his own,” according to
circumstances), and “Vive le Roy.” (Ante 429 and 434.) Before I had
seen a copy of the latter, it puzzled me to find such passages, as in
Pepys’s \textit{Diary}, where, on May 4, 1660, “The loud \textit{Vive le Roys} were
echoed from one ship’s company to another.” I could not understand
the sailors’ singing out in Norman French; nor why, as on March
28, 1660, before Charles II. was proclaimed, “a gentleman was brought
as a prisoner, because he called out of a vessel that he went in,
\textit{Vive le Roy}.” We have even “God save the King” \textit{sung to the tune of
Vive le Roy}, on Charles the Second’s restoration. The following is
the chorus:— 
\settowidth{\versewidth}{Come, let us sing, boys, God save the King, boys,}
\begin{scverse}
\vleftofline{“}Come, let us sing, boys, God save the King, boys,\\
Drink a good health, and sing Vive le Roy.’’
\end{scverse}

Finally, D’Urfey wrote a “Vive le Roy” for George I. See \textit{Pills to
purge Melancholy}, i. 116, 1719.

It is certainly a singular fact, that there should be an air of
the peculiar metre of “God save the King” in Dr. Bull’s manuscript;
but there is really nothing to identify it with the words. On the
contrary, the very, fact of a “God save the King” being in the same
book, and that an imitation of the popular cry, rather than a tune,
tends to disprove the connection.

A passage in Lord Macaulay’s \textit{History of England}, on the battle of
La Hogue,
\end{fixedpage}%699
\pagebreak

\begin{fixedpage}%700
\versoheader

might give the impression, that “God save the King” was a
national song in 1692. “The victorious flotilla slowly retired,
insulting the hostile camp with a
thundering \textit{chant} of ‘God save the King’” (iv. 240, 1855). I am
enabled, through his lordship’s kindness, to give the original words,
from Foucault’s report to the French Minister of Marine, in M.
Capefigue’s \textit{Louis XIV}., cap. xxxix.:—

“Ils eurent l’audace d’avancer dans une espèce de havre, ou il y
avoit vingt bâtimens marchands, deux frégates légères, un yacht, et
un grand nombre de chaloupes, tous échoués près de terre, et
brulèrent huit vaisseaux marchands: ensuite ils entrèrent dans
plusieurs bâtimens, qu’ils eurent la liberté et le loisir
d’appareiller et d’emmener avec eux, en criant \textit{God save the King}.
Sans la mer qui se retiroit, ils auroient brulé ou enlevé le reste.”

This, therefore, like all the rest, is the shout of “God save the
King,” and not the song or hymn.

Were I to sum up the case from the evidence before us, I should
say:—

That the first four theories may at once be discarded.

That there is an air very like “God save the King” in a
manuscript of Dr. Bull’s compositions, dated 1619, but no tittle of
evidence to connect the words with that period.

Now, as to their having been written for James II.:

Benjamin Victor asserts that “the very words and music” are an
old \textit{anthem} that was “sung at St. James’s Chapel, for King James II.,
when the Prince of Orange was landed.” Arne does not say -“anthem,”
but “for the Catholic Chapel of James II.” If sung at the Roman
Catholic Chapel, the words would have been in Latin. \textit{Quere}, was there
any Protestant Chapel at St. James’s in 1688? The words have never
yet been found in any collection of the words of anthems, whether in
print or manuscript, and although custom sanctions our applying the
title of “national anthem” to \textit{God save the King}, it is not, strictly
speaking, an anthem, but a song or hymn. No musician of the reign of
James II. (or even of George II.) would have entitled such a
composition an anthem; neither could Dr. Bull have intended that for
sacred music which he arranges as an air in one part of his
manuscript, and as a dance-tune in another.
The words of anthems are taken from the bible, or from some
authorized form of prayer, and are never in rhyme.

This is not the only seeming inaccuracy in Victor’s statement. He
says,
“Twenty men appear at the end of every play, and \textit{one}, stepping
forward from

the rest, begins singing;” whereas, according to Dr. Arne’s
score,\textsuperscript{a} each part was first sung as \textit{duet}, and then repeated in
chorus. The printed copies of the time are all for two voices.

The words of “God save the King” were inapplicable to the period
of the accession of James II., because he had then no enemies to
scatter; and, when he landed in Ireland (after his flight from
England), he, to please the native Irish, adopted an Irish air as his
March,\textsuperscript{b} while the native “pipers and harpers played,

\begin{dcfootnote}
\textsuperscript{a} Dr. Arne’s manu script score of “God save the King”
is in the possession of Mr. Oliphant. It is for male
voices only, accompanied by horns, violins, tenors and
basses.

\textsuperscript{b} This air was known in England, from the reign of
Charles II. down to 1730, as “Since Celia’s my foe.” It
derived that name from a “song to the Irish tune” written 
by Thomas Buffett, and printed in his \textit{New Poems,
Songs, \&c}., 8vo., 1676. It is the air to which Allan
Ramsay wrote the song of “ Farewell to Lochaber,” or
“Lochaber no more,” and which had been known in Scotland 
before Ramsay’s publication, as “King James's
“March to Ireland,” or “King James’s March to Dublin.”

\end{dcfootnote}
\end{fixedpage}%700
\pagebreak


\begin{fixedpage}%701
\rectoheader
‘The King shall enjoy his own again.’” (Macaulay’s \textit{History of
England}, iii. 175, 1855.

There are several witnesses to the fact of the words having been
sung “God save great James, our King,” but as neither Victor, Arne,
Burney, nor Benjamin Cooke (the three last being the persons who
\textit{heard} it sung “God save great James”) were born even in the lifetime
of James II., this James could have been no other than the Pretender,
his son, whom the Jacobites entitled “James III.” James II. died in
1701, “James III.” on the 80th of December, 1765. It is impossible to
suppose that any persons would sing “long life” to a dead king. These
Jacobite parodies were very common; I have already quoted three on
\textit{Rule, Britannia}, and subjoined is one on “God save the King;” but no
parody on the latter would be so easy or so natural as the mere
substitution of James for George.

The last claim to be analyzed is that of Henry Carey.

It is needless to quote any of the second-hand testimony in his
favour, since we have the direct evidence of a witness who was living
at the time the enquiry was instituted.

John Christopher Smith had been intimately acquainted with Carey,
when a young man. He composed the music to Carey’s \textit{Teraminta}, which
had great success, and passed through four editions. It was first
published in 1732. Smith asserts that Carey wrote the words and
composed the music of “God save the King,” and took the manuscript to
him to correct the base. At the first glance it seems improbable that
Carey should have required such assistance, but in the preface to his
Musical Century, vol. ii., dated Jan. 23, 1740, Carey says, “I had
some thoughts of giving the reader a detail of this work\dots what
basses I have added; \textit{what amended},” \&c. This was his last musical
publication, and the admission removes the only reasonable doubt upon
Smith’s testimony. In the postcript to the letter which Smith
dictated, Dr. Harington says, “Mr. Smith understood your father
intended this air as part of a birthday ode.” Carey seems to have had
something of the kind in his mind, when he printed “A new year’s Ode
for 1736-7, compos’d in a dream, the author imagining himself to be
Poet Laureate,”—an appointment he would, no doubt, have been
delighted to hold.

Carey gives evidence, throughout his works, of having been a
thoroughly loyal man, and a strong adherent to the Protestant
succession. His dreaming ode ends thus:— 
\settowidth{\versewidth}{King George he was born in the month of October,}
\begin{scverse}
\vleftofline{“}King George he was born in the month of October,\\
’Tis a sin for a subject that month to be sober.”
\end{scverse}
His first poem, in his first book of poems, 1713, is “An Ode
presented to her Majesty on her Birthday,” beginning—
\begin{scverse}
\vleftofline{“}Darling of Heav’n, and glory of the earth,\\
Illustrious Anna, whose auspicious birth.”
\end{scverse}
In the third edition of his poems, 1729, when George II. was on
the throne, we have an “Ode on their Majesties’ succession,” ending—
\settowidth{\versewidth}{God send No end To line Divine}
\begin{scverse}
\vleftofline{“}God send No end To line Divine \\
Of George and Caroline;”
\end{scverse}
\end{fixedpage}%701
\pagebreak


\begin{fixedpage}%702
\versoheader
In his “Wish,” he recommends Whigs and Tories to agree; and, if
they will follow his advice,—
\settowidth{\versewidth}{And so, God save the King and Queen.”}
\begin{scverse}
\vleftofline{“}Then shall we see a glorious scene,\\
And so, God save the King and Queen.”
\end{scverse}
and, among his works, there are other songs of the same class,
besides an entire Musical Entertainment “on the happy nuptials of the
Princess Royal of England with the Prince of Orange,” performed at
the theatre in Goodman’s Fields, in 1734.

Much error and mis-statement has been mixed up with the enquiry
into the history of “God save the King,” not only as regards Carey,
but nearly everything else connected with it.

Carey is said to have been a Jacobite. This was a guess which his
works utterly disprove. He is said to have been the natural son of
George Savile, Marquis of Halifax. Henry Carey certainly named one of
his sons “Savile,” but the Marquis died in 1695, and Carey speaks of
his “parents” as living in 1713. Probably his mother then kept a
school, as we find in the edition of his poems printed in that year,
“A Pastoral Eclogue on the Divine Power of God, spoken by two young
ladies, in the habits of shepherdesses, at an entertainment performed
at Mrs. Carey’s school, by several of her scholars.” Carey is said to
have “put a period to a life which had been led without reproach,”
when at \textit{the advanced age of eighty}, by suicide. I do not deny the
suicide, since it is stated by Hawkins, who is a good authority for
fact, although he wrote thirty years later. Hawkins may have had the
means of knowing; but it is not so stated in the newspapers of the
time. For instance, in The Daily Post, 5th Oct., 1743, “Yesterday
morning, Mr. H. Carey, well known to the musical world for his droll
compositions, got out of bed from his wife, in perfect health, and
was soon after found dead. He has left six children behind him.” On
the 17th Nov., of the same year, the performances at Drury Lane were
for the benefit of the widow and four \textit{small} children of the late H.
Carey. If any further proof of his age, than the four small children,
be necessary, it will be found in the prefaces to his \textit{Poems on
several Occasions}, which are rather three distinct books than three
editions of the same. In that of 1713, he hopes the reader will
conceive no prejudice against him on account of his age (\ie ,
youth), and that critics will not, “by unlimited detraction, obstruct
the hopes of his parents and the end of his education.” The songs in
this collection were either written to old tunes, or set to music by
other composers; but in the next edition, that of 1720, he
particularizes eleven songs as having music composed by himself. The
verses are there again entitled “the offsprings of his youthful
genius.” In the third edition, 1729, he addresses Geminiani and
Roseingrave as his instructors in music. Geminiani arrived in London
in 1714; Roseingrave commenced teaching in London after 1720. If
Carey was eighty years of age at the time of his death, he must have
been taking lessons at sixty. These are specimens of the difficulties
to be encountered by those who are content to take evidence at
second-hand.
\end{fixedpage}%702
\pagebreak

\begin{fixedpage}%703
\rectoheader

Objections may be taken to Carey’s claim, because “God save the
King” was published anonymously. I do not attach any importance to
that fact, because I have before me several others of his songs so
printed. The copies were, in all probability, obtained
surreptitiously. He complains of this piracy in the preface to the
first volume of \textit{The Musical Century}, 1737, and states his losses on
that account to have averaged nearly 300\textit{l}. a year. I do not
understand why he could not have repressed such piracy, under the act
of Queen Anne; but he was evidently not aware that he possessed the
power, since he prays the legislature to pass a bill, then pending,
for the protection of authors, such as was already enjoyed by
engravers.

Carey’s last musical publication bears date Jan., 1740, and that
is the year in which he is stated to have sung “God save the King” at
a tavern in Cornhill. The celebration of Admiral Vernon’s victory was
certainly an appropriate time for its production.

Carey died in October, 1743, and “God save the King” first became
extensively popular in October, 1745. It was the rebellion of that
year that called forth such repeated expressions of loyalty, and
caused so much enthusiasm when the song was sung at all the theatres.

“God save the King” consists of six bars in the first part, and
eight in the second. The rhythm is peculiar, but not defective, since
all the phrases consist of two bars. No composer of the time seems so
likely to have used this rhythm as Carey. Several of his songs
contain six bars in the first part, and some have more than six in
the second. A glance at his Musical Century, and other songs, will
shew this.

If Carey wrote both words and music of “God save the King,”
without having seen Dr. Bull’s “ayre,” he, in all probability, had in
mind the song of \textit{Vive le Roy} (ante p. 430). The music begins in the
same way, and we know by D’Urfey’s song that \textit{Vive le Roy} was in use,
or at least remembered, in the reign of George I. I have not seen any
old German copies, but in the modern, such as, “157 Alte und neue
Studenten,—Soldaten,—und Volks-Lieder,” Leipzig, 1847, the
composition is attributed to H. Carey.

I now leave the verdict as to the authorship in the hands of my
readers. There will, no doubt, be differences of opinion among them.
Some will be for Dr. Bull; others will say that the only four bars in
Bull’s “ayre” which are identical with “God save King” are a common
passage; and will instance the four bars in the Christmas Carol,
which is older than Bull’s tune, as nearly the same. (The reader may
compare the second part of the carol at p. 374, with the second part
of Bull’s “ayre.”) These will argue the coincidence to be
accidental.

Without speculating further upon opinions, I will now place
before my readers a printed copy of earlier date than any yet known.
In this copy neither George nor James is mentioned. It is applicable
to any king, but was printed in the reign of George II. It consists
of but two stanzas instead of three.
\end{fixedpage}%703
\pagebreak

\begin{fixedpage}%704
\versoheader
\indentpattern{0010001}
\settowidth{\versewidth}{Frustrate their knavish tricks,}
\begin{dcverse}
\begin{patverse}
O Lord our God, arise,\\
Scatter his enemies,\\
And make them fall:\\
\columnbreak
Confound their politicks,\\
Frustrate their knavish tricks,\\
On him our hopes are fix’d,\\
О save us all.”
\end{patverse}
\end{dcverse}

The above is taken from p. 22, of “Harmonia Anglicana; A
Collection of two, three, and four-part songs; several of them never
before printed. To which are added some Choice Dialogues, set to
music by the most eminent masters, viz., Dr. Blow, H. Purcell,
Handel, Dr. Green, Dl. Purcell, Eccles, Weldon, Leveridge, Lampe,
Carey, \&c. The whole revis’d, carefully corrected, and figur’d by a
judicious master. London, Printed for, and sold by John Simpson, at
the Bass Viol and Flute in Sweeting’s Alley, opposite the East Door
of the Boyal Exchange.”

The copy of “God save the King” in \textit{The Gentleman’s Magazine} for
October, 1745, has hitherto been referred to, as the earliest printed
authority. That version consists of the three stanzas which are still
usually sung, and commences “God save great George our King.” There
are two wrong notes in the fourth bar of the melody in that copy,
viz., B and C, which should be A and B. The base proves these to be
typographical errors, and not intentional alterations of the tune. In
the table of contents of \textit{The Gentleman’s Magazine} the older title of
“God save our Lord the King” is retained, agreeing with the copy now
produced; and when the \textit{Harmonia Anglicana} was extended to two
volumes, and the name changed to \textit{Thesaurus Musicus}, although the song
was then printed as “God save \textit{great George} our King,” the index
remained unaltered—“God save our Lord the King.” In \textit{Harmonia
Anglicana} the only heading is, “For two voices.” In \textit{Thesaurus
Musicus}, it is “A Loyal Song, sung at the Theatres Royal, for two
voices.” There is not a word about “anthem” in either copy, nor does
the original publication contain any other than secular music. In the
\textit{Gentleman’s Magazine}, it is “A Song for two voices, sung at both
play-houses.” 

The \textit{Harmonia Anglicana} is printed without date, but a
clue to the time of publication is obtained in the following way.
There are several works advertised by the publisher on the title
page, and three or four more seem to have been added subsequently to
fill up vacant space on the index plate. The last of these are “Two
collections of favourite Scotch tunes, set for a violin, German
flute, or harpsichord, by Mr. Oswald.” These two collections were
advertised in November, 1742.

I cannot understand how the above copy of “God save the King” can
have escaped Dr. Burney's notice, if he took any trouble in the
matter. Perhaps he
\end{fixedpage}%704
\pagebreak
\setlength{\fixedpagewidth}{410pt}
\begin{fixedpage}%705
\rectoheader
contented himself with a few superficial enquiries, and took his
text from Victor’s Letters, which were published before he gave his
account.

Now, as to the words. The copies “sung at the Theatres ’’contain
the three stanzas still usually sung, but during the progress of the
rebellion a fourth was added—
\indentpattern{0060003}
\settowidth{\versewidth}{Lord, grant that Marshal Wade,}
\begin{dcverse}
\begin{patverse}
\vleftofline{“}Lord, grant that Marshal Wade,\\
May, by Thy mighty aid,\\
Victory bring!\\
\columnbreak
May he sedition hush\\
And like a torrent rush,\\
Rebellious Scots to crush,\\
God, save the King!”
\end{patverse}
\end{dcverse}

In the December number of the \textit{Gentleman's Magazine}, for the same
year, is “An attempt to improve the song ‘God save the King,’ the
former words having no merit but their loyalty.” This commences,
“Fame, let the trumpet sound,” but the alteration was not adopted.
Many similar attempts were subsequently made without better success.
A copy of “God save the King” rendered into Latin, will be found in
\textit{The Gentleman's Magazine} for 1795.

On the 15th of May, 1800, George III. having been shot at by
James Hatfield, at Drury Lane Theatre, the following stanza (said to
have been written on the spot, by the Right Hon. R. B. Sheridan) was
sung by Mr. Kelly, at the end of the farce, and encored with
enthusiasm:—

\indentpattern{0030003}
\settowidth{\versewidth}{Our father, prince, and friend,}
\begin{dcverse}
\begin{patverse}
\vleftofline{‘‘}From every latent foe,\\
From the assassin’s blow,\\
God, save the King!\\
\columnbreak
O’er him thine arm extend,\\
For Britain’s sake, defend\\
Our father, prince, and friend,\\
God, save the King!”
\end{patverse}
\end{dcverse}

The following Jacobite parody is contained in \textit{The True Loyalist,
or Chevalier's Favourite}, 1779. It is also entitled “God save the
King:”

\indentpattern{0030003}
\settowidth{\versewidth}{For church, king, and law we’ll fight;}
\begin{dcverse}
\begin{patverse}
\vleftofline{“}Britons, who dare to claim\\
That great and glorious name,\\
Rouse at the call!\\
See British honour fled,\\
Corruption’s influence spread,\\
Slavery rear its head,\\
And freedom fall.
\end{patverse}

\begin{patverse}
Church, King, and liberty,\\
Honour and property,\\
All are betrayed:\\
Foreigners rule the land,\\
Our blood and wealth command,\\
Obstruct with lawless hand\\
Justice and truth.
\end{patverse}

\begin{patverse}
Shall a usurper reign,\\
And Britons hug the chain?\\
That we’ll deny :\\
Then let us all unite\\
For church, king, and law we’ll fight;\\
To retrieve James’s right,\\
Conquer or die.
\end{patverse}

\begin{patverse}
Join in the just defence\\
Of James, our lawful prince\\
And native King;\\
Then shall true greatness shine,\\
Justice and mercy join,\\
Restor’d by Stuart’s line,\\
Virtue’s great spring.
\end{patverse}

\begin{patverse}
Down with Dutch politics,\\
Whigs and their fanatics,\\
The old Rump’s cause.\\
Recall your injured prince,\\
Drive Hanoverians hence,\\
Such as rule here against\\
All British laws.
\end{patverse}

\begin{patverse}
Borne on the wings of Fame\\
James’s heroic name\\
All his foes dread.\\
He, from his father’s throne\\
Pulls usurpation down,\\
Glorious success shall crown\\
His sacred head.”
\end{patverse}
\end{dcverse}

Now as to the tune. The alterations in the first and fourth bars
of the melody were made within a short time of its having been sung
at the theatres. The A, in the first bar (instead of a third G), is
even to be found in Dr. Arne’s score.

A change of later years, is, however, greatly to be deprecated.
When the anticipation of the key-note at the termination of a tune
grew out of fashion, the
end of “God save the King” was changed to
\end{fixedpage}%705
\pagebreak


\begin{fixedpage}%706
\versoheader
alteration is tame and trivial, and quite out of character with
the rest of the air; indeed, it is so much felt to be out of
character, that, in order to avoid it, singers have been in the habit
of making a long holding-note upon the word “us,” and a run, or
triplet, upon “God.” Such changes, however, have only tended from bad
to worse.

Believing that all musical readers will agree with me that these
alterations ought to be rejected, the tune is here printed with the
old and correct termination. They who prefer it as now commonly
printed, can play the above notes to the same base.
\end{fixedpage}%706
\pagebreak
\setlength{\fixedpagewidth}{360pt}
\begin{fixedpage}%707
\rectoheader

\indentpattern{0010001}
\settowidth{\versewidth}{To sing, with heart and voice,}
\begin{dcverse}
\begin{patverse}
О Lord our God, arise,\\
Scatter her enemies,\\
And make them fall.\\
Confound their politics,\\
Frustrate their knavish tricks,\\
On Thee our hopes we fix,\\
God save us all.
\end{patverse}

\begin{patverse}
Thy choicest gifts in store,\\
On her be pleased to pour,\\
Long may she reign.\\
May she defend our laws,\\
And ever give us cause\\
To sing, with heart and voice,\\
God save the Queen.
\end{patverse}
\end{dcverse}


\musictitle{DRINK TO ME ONLY WITH THINE EYES.}
All attempts to discover the author of this simple and beautiful air have
hitherto proved unavailing, and, in all probability, will now remain so. Among
those who essayed was Dr. Burney. The poetry is by Ben Jonson.

\settowidth{\versewidth}{Since when, it grows and smells, I swear,}
\begin{dcverse}
\begin{altverse}
I sent thee, late, a rosy wreath,\\
Not so much honouring thee,\\
As giving it a hope, that there\\
It could not withered be;\\
But thou thereon didst only breathe,\\
And sent’st it back to me;\\
Since when, it grows and smells, I swear,\\
Not of itself, but thee.
\end{altverse}
\end{dcverse}

\end{fixedpage}%707
\pagebreak

\begin{fixedpage}%708
\versoheader
\musictitle{THE GIRL I LEFT BEHIND ME}

This air is contained in a manuscript in the possession of Dr.
Rimbault, of date about 1770, and in several manuscript collections
of military music of the latter half of the last century. It is a
march, and is either entitled \textit{The girl I left behind me}, or \textit{Brighton
Camp}.

One of the lines in the song “The girl I left behind me,” is,
“But now I’m bound to Brighton Camp,” and this gives a clue to the
date of the words.

Although there were encampments along the coast between 1691 and
1693, before the victory of La Hogue, I do not attribute the song to
so early a date, because I find no traces of words or music in the
numerous publications in the first half of the eighteenth century;
but in 1758 and 9 there were also encampments, whilst Admirals Hawke
and Rodney were watching the French fleet in Brest harbour. The
French had prepared “flat-bottomed boats” for the landing of troops.
In 1759 all danger of a descent upon our coast was averted by Admiral
Boscawen’s victory over one French fleet, and Admiral Hawke’s over
another. These and other successes of the year were chronicled in a
song entitled “The year fifty-nine.” In that year, also, a farce was
printed, entitled \textit{The Invasion}, to ridicule the unnecessary
apprehensions which some persons had entertained of a nocturnal
descent upon our coast by means of the flat-bottomed boats, and
Garrick produced a pantomime, entitled \textit{Harlequin's Invasion}, with the
same object.

It appears, therefore, that the song of \textit{The girl I left behind me}
may be dated, with great probability, in 1758.

In 1795 a song was written, entitled “Blyth Camps, or The Girl I
left behind
me.” It was printed in Bell’s \textit{Rhymes of the Northern Bards}, 8vo.,
Newcastle-upon-Tyne, 1812. It is a lame alteration of “Brighton Camp,”
commencing thus:— 

“I’m lonesome since I left Blyth camps,
And \textit{o’er} the moor that’s \textit{sedgy}.
With heavy thoughts my mind is filled,
Since I parted with my \textit{Betsy}.”

About 1790, when the celebrated John Philip Kemble became manager
of Drury Lane Theatre (and subsequently of Covent Garden), he
introduced this air as the Morris-dance for village festivities on
the stage, and as the march for processions. It has since been
constantly applied to the same purposes.

It has also been played for at least seventy years, as a
\textit{Loth-to-depart} when a man-of-war weighs anchor, and when a regiment
quits the town in which it has been quartered. The custom has become
so universal, that any omission to perform it would now be regarded
as a slight upon the ladies of the place.

“The girl I left behind me” is included in two collections of
Irish music—in Moore’s \textit{Irish Melodies}, and in Bunting’s last
collection, 4to., 1840. Each editor gives a different termination to
the first and second parts of the tune, and these variations are
quite necessary to establish an Irish origin. The question is of
priority.

All the evidence I have been able to collect is against the
authenticity of Moore’s version. Among Irish musical authorities I
enquired of the late Edward Bunting, J. A. Wade, J. C. Clifton, and
Tom Cooke; among English, of Dr. Crotch, W. Ayrton, and of several
band-masters. All were well
acquainted with the tune, but no one had heard it as printed by
Moore before the publication of his Melodies. I have also the best
means of knowing that
\end{fixedpage}%708
\pagebreak

\begin{fixedpage}%709
\rectoheader
Moore was in the habit of making alterations in the airs. In the
year 1825, my father was engaged as arbiter between the late John
Power, Moore’s publisher, and the late Robert Purdie, of Edinburgh.
Purdie had published a work entitled \textit{The Irish Minstrel}, the editor
of which had taken some of the melodies from Moore’s collection,
believing those versions to be genuine. Power resisted this, as an
infringement of his copyright, and proved that so many of the airs
had been altered by Moore, that Purdie chose rather to suppress his
entire work than to make such numerous alterations as would have been
required. “The girl I left behind me” was not one of the airs in
dispute, because it was so universally known, that Purdie’s editor
had no occasion to copy from Moore. The terminations in Purdie’s
\textit{Irish Minstrel} agree with my copy, and are those which I argue to be
correct.

With Bunting the case is different. I cannot suppose that he
would alter any air, and he is undoubtedly a great authority upon
Irish music. If Bunting’s version can be shewn to be older than the
English copy, I readily give up the point.

The case stands thus. Bunting prints it as “procured from A.
O’Neil, harper, \ad 1800—author and date unknown.” It is singular
that, having it from a harper in 1800, Bunting did not include so
very popular an air in his second \textit{General Collection of the Ancient
Music of Ireland}, printed in 1811. Did he think it decidedly Irish at
that time? If not, the omission is accountable.

Bunting informed me that he had not heard it played by any of the
harpers at the Congress in 1792, when they were assembled at Belfast,
from every part of Ireland, and liberal premiums were distributed
among them. If the air had then been known for thirty years in
Ireland, as it was in England, surely such would not have been the
case.

In a letter addressed to me by Bunting, and now before me (dated
24th January, 1840), he says of this air, “It is a pretty tune, and
has been played for the last fifty years, to my knowledge, by the
fifes and drums, and bands of the different regiments, on their
leaving the towns for new quarters.” Thus Bunting’s own memory
carries it back in Ireland as a military air to a period ten years
before he received his copy from a harper. Surely the harper may also
have heard one of the bands. His arrangement is a florid one, not the
simple air, and no band ever yet played it with those Irish
terminations.

Finally, this harper’s copy cannot have been generally known, not
even to Moore, or he would certainly have adopted the alteration
instead of his own very inferior one. The harper’s terminations would
also have answered equally well for the march.

Moore and Bunting both adopt the English name, “The girl I left
behind me.” The words relate to England, and to England only. It does
not commence, “I’m lonesome since I cross’d the sea.”

If it be necessary to prove that Irish harpers occasionally play
airs that are not Irish, I have abundant proof at hand; but cannot
suppose that any one conversant with the subject will dispute it.

My readers have now the facts before them, and can draw their own
Conclusions. I care not which way they decide. These enquiries should
be conducted in the calm spirit of research after truth, and not as
contentions of nationality.
\end{fixedpage}%709
\pagebreak


\begin{fixedpage}%710
\versoheader

\settowidth{\versewidth}{But now I’m bound to Brighton camp,}
\begin{dcverse}
\begin{altverse}
Oh, ne’er shall I forget the night,\\
The stars were bright above me,\\
And gently lent their silv’ry light,\\
When first she vow’d to love me.\\
But now I’m bound to Brighton camp,\\
Kind Heaven, then, pray guide me,\\
And send me safely back again\\
To the girl I’ve left behind me.
\end{altverse}

\begin{altverse}
Had I the art to sing her praise\\
With all the skill of Homer,\\
One only theme should fill my lays—\\
The charms of my true lover.\\
So, let the night be e’er so dark,\\
Or e’er so wet and windy,\\
Kind Heaven send me back again\\
To the girl I’ve left behind me.
\end{altverse}

\begin{altverse}
Her golden hair, in ringlets fair,\\
Her eyes like diamonds shining,\\
Her slender waist, with carriage chaste,\\
May leave the swan repining.\\
Ye gods above! oh, hear my prayer,\\
To my beauteous fair to bind me,\\
And send me safely back again\\
To the girl I’ve left behind me.
\end{altverse}

\begin{altverse}
The bee shall honey taste no more,\\
The dove become a ranger,\\
The falling waves shall cease to roar,\\
Ere I shall seek to change her.\\
The vows we register’d above\\
Shall ever cheer and bind me,\\
In constancy to her I love,—\\
The girl I’ve left behind me.
\end{altverse}
\end{dcverse}

\end{fixedpage}%710
\pagebreak

\begin{fixedpage}%711
\rectoheader

\settowidth{\versewidth}{My mind her form shall still retain,}
\begin{dcverse}
\begin{altverse}
My mind her form shall still retain, \\
In sleeping or in waking,\\
Until I see my love again,\\
For whom my heart is breaking.\\
If ever I return that way,\\
And she should not decline me,\\
I evermore will live and stay \\
With the girl I’ve left behind me.
\end{altverse}
\end{dcverse}

\musictitle{THE ROGUES’ MARCH.}

Why so graceful and pastoral a melody as this should have been
condemned to be the “Cantio in exitu” of deserters and reprobates who
are to be drummed out of regiments, is not easily to be accounted
for; but such is the case, and has been for more than a
century—possibly much longer.

In another form, this tune has been long in use for cheerful
songs. See p. 720.

\musictitle{WITH JOCKEY TO THE FAIR.}

There are many extant copies of this song without dates, and it
is printed in \textit{Vocal Music, or The Songsters Companion}, iii. 26, 2nd
edit., 1772.

It was originally a song for the public gardens, and has been
somewhat simplified by popular use. The tune, in this instance, has
been rather improved than deteriorated by the change.

Not having space for two versions, it is here presented to the
reader in its popular form. The following are the original words:—
\end{fixedpage}%711
\pagebreak
\begin{fixedpage}%712
\versoheader
music only
\end{fixedpage}%712
\pagebreak
\setlength{\fixedpagewidth}{380pt}
\begin{fixedpage}%713
\rectoheader
\indentpattern{0001000015}
\settowidth{\versewidth}{He tapp’d the window—“ Haste, my dear,”}
\begin{dcverse}
\begin{patverse}
The cheerful parish bells had rung;\\
With eager steps he trudg’d along;\\
Sweet flow’ry garlands round him hung,\\
Which shepherds us’d to wear:\\
He tapp’d the window—“Haste, my dear,”\\
Jenny, impatient, cried, “Who’s there? ”\\
“’Tis I, my love, and no one near ;\\
Step gently down, you’ve naught to fear\\
With Jockey, to the fair.\\
Step gently,” \&c.
\end{patverse}

\begin{patverse}
“My dad and mammy’re fast asleep,\\
My brother’s up, and with the sheep;\\
And will you still your promise keep,\\
Which I have heard you swear?\\
And will you ever constant prove? ”\\
“I will, by all the pow’rs above,\\
And ne’er deceive my charming dove :\\
Dispel these doubts, and haste, my love,\\
With Jockey to the fair.\\
Dispel these doubts,” \&c.
\end{patverse}

\begin{patverse}
“Behold the ring," the shepherd cried;\\
“Will Jenny be my charming bride ?\\
Let Cupid be our happy guide,\\
And Hymen meet us there!”\\
Then Jockey did his vows renew;\\
He would be constant, would be true;\\
His word was pledg’d—away she flew,\\
With cowslips sparkling with the dew,\\
With Jockey to the fair.\\
With cowslips, \&c.
\end{patverse}

\begin{patverse}
Soon did they meet a joyful throng,\\
Their gay companions, blithe and young,\\
Each joins the dance, each joins the song,\\
To hail the happy pair.\\
What two were e’er so fond as they?\\
All bless the kind, propitious day,\\
The smiling morn and blooming May,\\
When lovely Jenny ran away\\
With Jockey to the fair.\\
When lovely Jenny, \&c.
\end{patverse}
\end{dcverse}

The following song was recently written for me to the above air, by Charles
Mackay:—

\settowidth{\versewidth}{Come youth, come age, come childhood fair,}
\begin{dcverse}
\begin{patverse}
\vleftofline{“}When swallows dart from cottage eaves,\\
And farmers dream of barley sheaves;\\
When apples peep amid the leaves,\\
And woodbines scent the way.\\
We love to fly from daily care,\\
To breathe the buxom country air,\\
To join our hands and form a ring,\\
To laugh and sport, to dance and sing,\\
Amid the new-mown hay.\\
To laugh, \&c.
\end{patverse}

\begin{patverse}
We’ve room for all, whoe’er they he,\\
Who have a heart for harmless glee,\\
And in the shadow of our tree\\
Can fling their pride away.\\
So, join our sport, ye maidens true,\\
With eyes of beaming black or blue;\\
Come youth, come age, come childhood fair,\\
We’ve welcome kind, and room to spare,\\
Amid the new-mown hay.\\
We’ve welcome,” \&c.
\end{patverse}
\end{dcverse}

\musictitle{THE GOLDEN DAYS OF GOOD QUEEN BESS.}

The earliest form in which I have found this tune is as “No more, fair virgins,
boast your power,” introduced in \textit{Love in a riddle}, in 1729. It has three other
names, “The golden days of good Queen Bess,” “Ally Croaker,” and “Unfortunate Miss Bailey.”

“The golden days of good Queen Bess” was written by Collins, and not improbably
for one of the celebrations of Queen Elizabeth’s birthday, which were so
much in vogue, as anti-jacobite demonstrations, during the last century (see
p. 568 and note). The words consist of eleven stanzas, and commence thus:—

\indentpattern{000022}
\settowidth{\versewidth}{Whose name and whose memory posterity may bless, Sir.}
\begin{scverse}
\begin{patverse}
\vleftofline{“}To my muse give attention, and deem it not a mystery,\\
If we jumble together music, poetry, and history ;\\
The times to display in the days of Queen Bess, Sir,\\
Whose name and whose memory posterity may bless, Sir.\\
О the golden days of good Queen Bess,\\
Merry be the memory of good Queen Bess.”
\end{patverse}
\end{scverse}

In Bell’s \textit{Rhymes of Northern Bards}, Newcastle, 1812, is “Barber’s News, or
Shields in an uproar,” to the tune of \textit{О the golden days of good Queen Bess}.
“Ally Croaker” is a song by Foote, in his comedy, \textit{The Englishman in Paris},

\end{fixedpage}%713
\pagebreak

\setlength{\fixedpagewidth}{360pt}
\begin{fixedpage}%714
\versoheader
1753, and was sung by Miss Macklin, to the guitar. It was
printed, with the tune, in \textit{Apollo's Cabinet, or The Lady’s Delight},
ii. 218 (Liverpool, 1757), and the tune, under that name, in
Thompson’s \textit{Country Dances}, i. 41. The song commences thus:—

\indentpattern{000011}
\settowidth{\versewidth}{Arrah, will you marry me, dear Ally Croaker?”}
\begin{scverse}
\begin{patverse}
\vleftofline{“}There lived a man in Ballymecrazy,\\
Who wanted a wife to make him unaisy,\\
Long had he sighed for dear Ally Croaker,\\
And thus the gentle youth bespoke her:\\
Arrah, will you marry me, dear Ally Croaker?\\
Arrah, will you marry me, dear Ally Croaker?”
\end{patverse}
\end{scverse}

“Unfortunate Miss Bailey” was written to the tune by George
Colman, and has been much sung to it during the last fifty years.

\end{fixedpage}%714
\pagebreak

\setlength{\fixedpagewidth}{360pt}
\begin{fixedpage}%715
\rectoheader

\musictitle{LOVELY NANCY.}

This is one of the airs contained in the folio edition of \textit{The
Jovial Crew} (which has the basses to the airs), but not in the
octavo. It was added after the first performance. The following words
were sung by the female beggars, in the opera:

\musictitle{HEART OF OAK.}

The words of this still popular song are by David Garrick, and it
was sung by Mr. Champnes in \textit{Harlequin’s Invasion}, in 1759. The tune
is by Mr. (afterwards Dr.) Boyce.

Many songs have been written to the air, and, among them, two in
the Burney Collection. The first, “Keppel’s Triumph,” commencing—

\settowidth{\versewidth}{In spite of each charge from Sir Hugh Palliser:”}
\begin{scverse}
\vleftofline{“}Bear a hand, jolly tars, for bold Keppel appear,\\
In spite of each charge from Sir Hugh Palliser:”
\end{scverse}
\end{fixedpage}%715
\pagebreak

\setlength{\fixedpagewidth}{360pt}
\begin{fixedpage}%716
\versoheader
the second, “The hardy tars of old England; or, The true Hearts
of Oak,” beginning— 

\settowidth{\versewidth}{Come, cheer up, my lads, let us haste to the main,}
\begin{scverse}
\vleftofline{“}Come, cheer up, my lads, let us haste to the main,\\
And rub out old scores with the dollars of Spain.”
\end{scverse}

\indentpattern{00003}
\settowidth{\versewidth}{They frighten our women, our children, and beaus;}
\begin{dcverse}
\begin{patverse}
We ne’er see our foes but we wish them to stay;\\
They never see us but they wish us away :\\
If they run, why, we follow, and run them ashore,\\
For, if they won’t fight us, we cannot do more.\\
Heart of oak, \&c.
\end{patverse}

\begin{patverse}
They swear they’ll invade us, these terrible foes;\\
They frighten our women, our children, and beaus;\\
But, should their flat bottoms in darkness get o’er,\\
Still Britons they’ll find to receive them on shore.\\
Heart of oak, \&c.
\end{patverse}
\end{dcverse}

\end{fixedpage}%716
\pagebreak


\setlength{\fixedpagewidth}{360pt}
\begin{fixedpage}%717
\rectoheader
\indentpattern{00003}
\settowidth{\versewidth}{We'll still make them fear, and we’ll still make them flee,}
\begin{dcverse}
\begin{patverse}
We'll still make them fear, and we’ll still make them flee,\\
And drub ’em on shore, as we’ve drubb’d ’em at sea:\\
Then cheer up, my lads, with one heart let us sing,\\
Our soldiers, our sailors, our statesmen, our king.\\
Heart of oak, \&c.
\end{patverse}

\begin{patverse}
We’ll still make ’em run, and we'll still make ’em sweat,\\
In spite of the devil, and Brussels Gazette:\\
Then cheer up, my lads, with one heart let us sing,\\
Our soldiers, our sailors, our statesmen and king.\\
Heart of oak, \&c.
\end{patverse}
\end{dcverse}


\musictitle{FAIR ROSALIND.}

This song is contained in \textit{Mercurius Musicus}, 1735; in Watts’
\textit{Musical Miscellany}, ii. 176, 1729; in \textit{The Convivial Songster}, 1780;
\&c.

In the \textit{Musical Miscellany} it is entitled \textit{The Jilt}.

\settowidth{\versewidth}{Wretched, And Only Wretched, be,}
\begin{scverse}
\begin{altverse}
Wretched, and only wretched, be, \\
To whom that lot shall fall;\\
For, if her heart aright I see,\\
She means to please them all.
\end{altverse}
\end{scverse}

\musictitle{THE BARKING BARBER.}

This tune is to be found in two or three different forms, the
variations having been caused by the different metres that have been
adapted to it. For instance, one of the songs is \textit{Date obolum
Belisario}, which has twelve syllables in the first line:— 
\settowidth{\versewidth}{“O Fortune! how strangely thy gifts are awarded.”}
\begin{scverse}
“O Fortune! how strangely thy gifts are awarded.”
\end{scverse}

Another is the comic song of \textit{Guy Fawkes}, which, having sixteen
syllables, requires fifteen notes in the first two bars of music:—

\settowidth{\versewidth}{“I sing a doleful tragedy, Guy Fawkes that prince of sinisters.”}
\begin{scverse}
“I sing a doleful tragedy, Guy Fawkes that prince of sinisters.”
\end{scverse}

So also with the burdens—one is “Date obolum Belisario,” and
another is “Bow, wow, wow.”
\end{fixedpage}%717
\pagebreak

\setlength{\fixedpagewidth}{360pt}
\begin{fixedpage}%718
\versoheader

Two versions of the tune were printed in \textit{National English Airs}.
The following is the older.

Having lost the transcript of the words of “The Barking Barber,”
the first four lines of “Date obolum” are printed with the tune.

\musictitle{NANCY DAWSON.}

Nancy Dawson, from whom this tune is named, was a celebrated
dancer in the reign of George II. One of her portraits is at the
Garrick Club; and there are four different prints of her, one of
which, by Spooner, is in Dr. Burney’s Collection of Theatrical
Portraits in the British Museum. Another is by G. Pulley (folio),
dancing a hornpipe, with the song; and a third by Watson. Her life
was published in 1760; and Stevens’s \textit{Dramatic History of Master
Edward, Miss Ann, and others}, “the extraordinaries of these times,”
was “a satire upon Edward Shuter, the comedian, and Nancy Dawson, the
far-famed toast.” From this work it appears that she first appeared,
as a dancer, at Sadler’s Wells; and as “she was extremely agreeable
in her figure, and the novelty of her dancing added to it, with her
excellence in her execution, she soon grew to be a favorite with the
town; and at the ensuing season was engaged at Covent Garden
playhouse. She became vastly celebrated, admired, imitated, and
followed by everybody.” Her death took place at Hampstead on 27th
May, 1767, and she was buried in the Chapel of St. George the Martyr,
Queen Square, Bloomsbury, where there is a tombstone to her memory,
with the laconic inscription, “Here lies Nancy Dawson.” She had many
good qualities, and among others was very charitable.
\end{fixedpage}%718
\pagebreak

\setlength{\fixedpagewidth}{360pt}
\begin{fixedpage}%719
The tune became very popular from her dancing. It was printed in
many collections as a country-dance; was arranged with variations for
the harpsichord, as Miss Dawson’s Hornpipe; was introduced in \textit{Love in
a Village} (1762), as the housemaid’s song; and is still sung in
children’s games as “Here we go round the mulberry-bush.”

I have already spoken of English country-dances having been
fashionable in France, and have now before me one of the printed
collections of those dances, in which \textit{Nancy Dawson} is included, as
the “sixième Anglaise de la Reine.” It is the “5\textsuperscript{eme.} Recueil
d’Anglaises, arrangées avec leurs Traits, telle quel se danse ché la
Reine. Mis au jour par M. Landrin, M\textsuperscript{tre.} de Danse, et Compositeur des
traits des Contre-Danse. Prix 18s. le recueil. A Paris, chez Landrin
M\textsuperscript{d.} de Musique et M\textsuperscript{tre.} de Danse, Rüe des Boucheries St. Germains,
proche le petit Marché, et chez M\textsuperscript{lle} Castagnery, Rüe des Prouvairs,
et aux addresses ordinaires.” 8vo., n.d.

The words are printed in \textit{The Bullfinch} and other collections of
songs, as well as under one of the engraved portraits.

\end{fixedpage}%719
\pagebreak

\setlength{\fixedpagewidth}{360pt}
\begin{fixedpage}%720
\versoheader

\indentpattern{0001}
\settowidth{\versewidth}{See how she comes to give surprise,}
\begin{scverse}
\begin{patverse}
See how she comes to give surprise, \\
With joy and pleasure in her eyes; \\
To give delight she always tries,\\
So means my Nancy Dawson, \&c.
\end{patverse}
\end{scverse}

\musictitle{THE TIGHT LITTLE ISLAND.}

This tune is a vocal version of \textit{The Rogues’ March} (ante p. 711),
and several popular songs have been sung to it. Among these are \textit{The
tight little Island} and \textit{Abraham Newland}.

If it could be ascertained to be of the time of James II., I
should imagine that the old song of which Sir Wilfull sings a snatch
in Congreve’s \textit{Way of the World}, act iv., sc. 10, was also sung to it;
but I am unable to adduce evidence of so early a date. The lines are—

\indentpattern{110110}
\settowidth{\versewidth}{Of ale that is potent and mellow;}
\begin{dcverse}
\begin{patverse}
\vin \vleftofline{“}Prithee, fill me a glass, \\
Till it laugh in my face, \\
Of ale that is potent and mellow;\\
He that pines for a lass\\
Is an ignorant ass,\\
For a bumper has not its fellow;”
\end{patverse}
\end{dcverse}
and they seem to trip to the measure.

\textit{The tight little Island} was included in several collections of
songs published towards the close of the last century, with the tune.

\textit{Abraham Newland} was written by Charles Dibdin, jun., on the
cashier of the Bank of England, whose name was formerly attached to
bank-notes. It commences—

\indentpattern{22022022220}
\settowidth{\versewidth}{Through air, through ocean, and through land,}
\begin{dcverse}
\begin{patverse}
\vin\vin \vleftofline{“}Ne’er yet was a name\\
So bandied by Fame,\\
Through air, through ocean, and through land,\\
As one that is wrote\\
Upon every bank-note,—\\
You all must know Abraham Newland.\\
Oh! Abraham Newland!\\
Notified Abraham Newland!\\
I’ve heard people say,\\
Sham Abraham you may,\\
But you must not sham Abraham Newland.”
\end{patverse}
\end{dcverse}

“Shamming Abraham” means feigning madness as an excuse for
begging; but a short extract from an old black-letter pamphlet will
more fully explain the term: “These \textit{Abraham-men} be those that faine
themselves to have been mad, and have been kept either in Bethlehem,
or in some other prison, a good time; and not one amongst twenty that
ever came in prison for any such cause: yet will they say how
piteously and most extreamly they have been beaten and dealt withall.
Some of these be merry and very pleasant; they will daunce and sing.”
(\textit{The Groundwork of Conny-catching}.) Dekker, in his \textit{English
Villanies}, also says: “Of these \textit{Abraham-men} some be exceeding merry,
and doe nothing but sing songs fashioned out of their owne braines,”
\&c. I suspect they succeeded much better than the whining beggars of
the present day.

Percy says that the English have more mad songs than any of their
neighbours. True,—but at least half of these were written for, or in
burlesque of, Abraham-men.

The first stanza of \textit{The tight little Island} is here adapted to
this tune.
\end{fixedpage}%720
\pagebreak


\setlength{\fixedpagewidth}{360pt}
\begin{fixedpage}%721
\rectoheader

\musictitle{THE KEEL ROW.}

“O weel may the keel row” is perhaps the most popular of all the
Northumbrian tunes at the present time. It is contained in several
manuscripts of the latter half of the last century; in \textit{Topliffs
Selection of Melodies of the Tyne and Wear}; and in other more modern
publications.
\end{fixedpage}%721
\pagebreak

\setlength{\fixedpagewidth}{360pt}
\begin{fixedpage}%722
\versoheader

The earliest form in which I have observed it in print is as a
country-dance, entitled \textit{Smiling Polly}. In several of the collections
of the last century, such as Thompson’s 200 \textit{Country Dances}, ii. 63
[1765], it is so included. In these copies the second part of the
tune differs.

The words of \textit{The keel row} are in Ritson’s \textit{Northumberland Garland},
1793; in Bell’s \textit{Rhymes of the Northern Bards}, 1812; and in several
later collections.

\indentpattern{0001000100010001}
\settowidth{\versewidth}{Sae leish, sae blithe, sae bonny?}
\begin{dcverse}
\begin{patverse}
O whe’s like my Johnny,\\
Sae leish, sae blithe, sae bonny? \\
He's foremost among the mony\\
Keel lads o’ coaly Tyne:\\
He’ll set and row so tightly,\\
Or in the dance—so sprightly— \\
He’ll cut and shuffle sightly; \\
’Tis true—were he not mine.\\
He wears a blue bonnet,\\
Blue bonnet, blue bonnet; \\
He wears a blue bonnet,— \\
A dimple in his chin:\\
And weel may the keel row, \\
The keel row, the keel row;\\
And weel may the keel row, \\
That my laddie’s in.
\end{patverse}
\end{dcverse}

\musictitle{CARE, THOU CANKER OF OUR JOYS.}

This air is now better known as “When the rosy morn appearing,”
from the words which were sung to it, as a Round, in the opera of
\textit{Rosina}. “Care, thou canker of our joys,” was written by the Rev. Dr.
Grant, and I was informed by the late Ralph Banks, organist of
Rochester Cathedral, that the tune was composed by John Garth, of
Durham, the adapter of English words to Marcello’s Psalms. It has
never been published with any name attached.

Charles Mackay’s song, “Trusting heart, though men deceive thee,”
was written to the tune.
\end{fixedpage}%722
\pagebreak

\setlength{\fixedpagewidth}{360pt}
\begin{fixedpage}%723
\rectoheader

\settowidth{\versewidth}{Night, my boys, for you and me.}
\begin{dcverse}
\begin{altverse}
Seize the villain, plunge him in;\\
See the bated miscreant dies:— \\
Mirth and all thy train come in,\\
Banish sorrow, tears, and sighs.\\
O’er our merry midnight bowls,\\
Oh! how happy shall we be; \\
Day was made for vulgar souls,\\
Night, my boys, for you and me.
\end{altverse}
\end{dcverse}

\musictitle{I MADE LOVE TO KATE.}

There are certain tunes common to England and Scotland, about
which the existing evidence is so nearly balanced that it is very
difficult to prove to which country they owe parentage. One of these
is now commonly known in England as \textit{For that’s the time o’ day}, and
in Scotland as \textit{Woo’d and married and a’}.

\textit{Woo’d and married and a’} is a song that was “sung by Mr. Lauder
at the little theatre in the Hay Market,” and “printed for J. Oswald
in St. Martin’s Church Yard.” After a little touching up, the words
were included in Herd’s \textit{Ancient and Modern Scottish Songs}, 1769. The
tune was printed by Bremner in the seventh number of his \textit{Collection of
Scots’ Reels and Country Dances}, and the seventh and eighth numbers
are included in a list of new publications in the \textit{Scots’ Magazine} for
November, 1759.

\textit{I made love to Kate} is the English name, and it is derived from a
song “sung by Mr. Beard at Ranelagh.” When \textit{The Jovial Crew} was
revived at Covent Garden Theatre (Feb. 14th, 1760), this was so
popular that he introduced it at the end of the second act. Johnson,
who published “The Airs in The Jovial Crew, or Merry Beggars, as
performed at the theatre in Covent Garden,” desirous of saving the
expense of re-engraving the song, employed the same plate (“sung by
Mr. Beard at Ranelagh”) in the opera. In so doing, he was obliged to
print it out of its proper place, because it would have commenced in
the middle of a page. If we suppose this plate to have been only a
few months old, the dates will be tolerably balanced.

The tune has the character of the hornpipes, rounds, and jigs of
which so many collections were advertised from 1710 to 1760, but
which are very difficult to
\end{fixedpage}%723
\pagebreak

\setlength{\fixedpagewidth}{360pt}
\begin{fixedpage}%724
\versoheader

obtain. I am persuaded that it was originally a dance tune, and
that neither in England nor Scotland have we yet arrived at the
fountain head. Bremner makes a whole tone between the seventh and
octave at the terminations of each part; but the copy sung by Lauder
has the semitone, and agrees so far with the \textit{Caledonian Pocket
Companion}, and with the English version.

\musictitle{PETTICOAT LOOSE.}

A favorite old country dance. It is included in Peter Thompson’s
Collection (1753), in that of Charles and Samuel Thompson (1765), and
of Samuel, Ann, and Peter Thompson (1790). Also in Rutherford’s and
several others.
\end{fixedpage}%724
\pagebreak

\setlength{\fixedpagewidth}{360pt}
\begin{fixedpage}%725
\rectoheader

\musictitle{SINCE HODGE PROVES UNGRATEFUL.}

This is one of Dr. Arne’s tunes, introduced in \textit{Love in a Village}.
It has long been a favorite, but more especially of late years, in
consequence of the following words having been written to it by
Charles Mackay.
\end{fixedpage}%725
\pagebreak

\setlength{\fixedpagewidth}{360pt}
\begin{fixedpage}%726
\versoheader

\settowidth{\versewidth}{That cling to the scenes where our hearts were so free;}
\begin{scverse}
And joyous to us shall the memories be\\
That cling to the scenes where our hearts were so free;\\
If care should perplex us, if sorrow should frown.\\
Or weariness follow the moil of the town;\\
We'll think of the days when our faces were bright \\
With the rambles of morn and the songs of the night, \\
And nourish the hope, ’mid the winter and rain,\\
That we come back, with summer, to see you again.
\end{scverse}

\centerrule

The following are either tunes of some interest of the reign of
George III., or are traditional songs of uncertain date.

\centerrule
\end{fixedpage}%726
\pagebreak
\renewcommand\rectoheadertext{traditional tunes of uncertain date.}


\setlength{\fixedpagewidth}{360pt}
\begin{fixedpage}%727
\rectoheader

\musictitle{’TWAS DOWN IN CUPID’S GARDEN.}

This is one of the most generally known of traditional songs.
Several sets of words are sung to the tune, but all are about
“Cupid’s Garden.” Some of the untutored singers chant only the second
half of the air, and occasionally make nine bars, by turning the
dotted minim on the word “grow” (in the last bar but two) into a
crotchet, thus:—

“Cupid’s Garden” is a corruption of “Cuper’s Gardens,” which were
once a celebrated place of amusement on the Surrey side of the
Thames, exactly opposite to Somerset House.

Of these gardens, Dr. Rimbault gives the following account in \textit{Fly
Leaves} (2nd Series, p. 52):—“They derived their name from Boydell
Cuper, a gardener in the family of Thomas, Earl of Arundel, who, when
Arundel House in the Strand was taken down, had interest enough to
procure many of the mutilated marbles, which he removed to the
gardens he was then forming as a place of popular amusement.” They
were opened in 1678, and Aubrey, in his \textit{Account of Surrey}, thus
speaks of them:—“Near the Bankside, lies a very pleasant garden, in
which are fine walks, known by the name of Cupid’s (\ie , Cuper’s)
Gardens. They are the estate of Jesus College, in Oxford, and erected
by one who keeps a public house; which, with the conveniency of its
arbours, walks, and several remains of Greek and Roman antiquities,
have made this place much frequented.”

“About the year 1736, Mrs. Evans, the widow of a man who kept the
ancient tavern known as the ‘Hercules’ Pillars’, in Fleet Street,
opposite Clifford’s Inn, took \textit{Cuper's Gardens}, and erected an
orchestra and an organ, intending it as a place of entertainment for
the summer evenings, similar to Vauxhall. It subsequently became
famous for its displays of fireworks. Warburton, the well-known
antiquary, writing to his friend Hurd, July 9th, 1753, thus describes
them:—‘I dined the other day with a lady of quality, who told me she
was going that evening to see the ‘finest fireworks’ at Marybone. I
said fireworks was a very odd refreshment for this sultry weather;
that, indeed, Cuper's Gardens had been once famous for this summer
entertainment; but then his fireworks were so well understood, and
conducted with so superior an understanding, that they never made
their appearance to the company till they had been well cooled by
being drawn through a long canal of water, with the same kind of
refinement that the Eastern people smoke their tobacco through the
same medium.”

“Cuper’s Garden kept up its celebrity for many seasons, but at
length yielded to its formidable rival, \textit{Vauxhall}, and was finally
closed in 1753. Some accounts say that it was suppressed in
consequence of the dissoluteness of its visitors. Indeed, from the
following lines in Welsted’s Epistle ‘On False Fame,’ the company
was evidently not always the most select:—
\end{fixedpage}%727
\pagebreak

\setlength{\fixedpagewidth}{360pt}
\begin{fixedpage}%728
\versoheader

\settowidth{\versewidth}{A different face by turns, or dress does borrow,}
\begin{scverse}
\vleftofline{“}The light coquettish trip, the glance askew\\
To slip the vizor, and to skulk anew—\\
For \textit{Cuper’s Bowers}, she hires the willing scull;\\
A cockswain’s now, and now a sharper’s trull!\\
A different face by turns, or dress does borrow,\\
To-day a Quaker, and in weeds to-morrow!\\
At windows twitters, or from hacks invites;\\
While here a ’prentice, there a captain bites;\\
With new success, new ’ffrontery she attains;\\
And grows in riot, as she grows in gains.”
\end{scverse}

\settowidth{\versewidth}{When there I saw two pretty maids}
\begin{dcverse}
\begin{altverse}
I’d not walk’d in that garden\\
The past of half an hour,\\
When there I saw two pretty maids\\
Sitting under a shady bow’r.\\
The first was lovely Nancy,\\
So beautiful and fair,\\
The other was a virgin,\\
Who did the laurel wear, \&c.
\end{altverse}

\begin{altverse}
I boldly stepp’d up to her,\\
And unto her did say,\\
Are you engaged to any young man ?\\
Do tell to me, I pray !\\
\columnbreak
I’m not engag’d to any young man,\\
I solemnly do swear;\\
I mean to live a virgin,\\
And still the laurel wear, \&c.
\end{altverse}

\indentpattern{010101011}
\begin{patverse}
Then hand in hand together\\
This lovely couple went;\\
Resolved was the sailor boy\\
To know her full intent;\\
To know if he would slighted he,\\
When to her the truth he told :\\
Oh no ! oh no! oh no! she cried,\\
I love a sailor bold,\\
I love a sailor bold!
\end{patverse}
\end{dcverse}

\end{fixedpage}%728
\pagebreak

\setlength{\fixedpagewidth}{360pt}
\begin{fixedpage}%729
\rectoheader

The author of “The Scouring of the White Horse, or The Long
Vacation Rambles of a London Clerk,” has given another version of the
above. He compresses the first three stanzas into two, and varies the
termination very amusingly.

\musictitle{BRITONS, STRIKE HOME, MY BOYS.}

Not Purcell’s “Britons, strike home,” but an old sea-song,
contributed by Mr. Charles Sloman.

It is one which I well remember in the play-ground at Fulham,
about forty years ago. Sometimes half-a-dozen boys would chant it in
unison, using most emphatic action at the words “strike home.”
\end{fixedpage}%729
\pagebreak

\setlength{\fixedpagewidth}{360pt}
\begin{fixedpage}%730
\versoheader

\musictitle{UNDER THE ROSE.}

This is one of the common street ballad tunes of London, to which
numberless songs have been sung at various times.

The words here printed, were propagated by the press of Mr.
Catnach, in Monmouth Court, Seven Dials, about the year 1820. It was
at a time when trade was depressed, and many mechanics were thrown
out of employment. The object of the ballad was to propose a renewal
of war as a remedy.

The tune is worthy of better words.

There are several old songs of “Under the Rose.” One written in
1647, is entitled “Prattle your pleasure (under the rose),” and
commences—

\settowidth{\versewidth}{There is an old proverb, which all the world knows,}
\begin{scverse}
\vleftofline{“}There is an old proverb, which all the world knows,\\
Anything may be spoke, if’t be under the rose.”
\end{scverse}

In 1526, roses were placed over Confessionals, the rose being
considered as the emblem of silence. How it came to be so considered
has been fully discussed in the pages of \textit{Notes and Queries}.
\end{fixedpage}%730
\pagebreak

\setlength{\fixedpagewidth}{360pt}
\begin{fixedpage}%731
\rectoheader

\musictitle{SAW YOU MY FATHER?}

This song is printed on broadsides, with the tune, and in \textit{Vocal
Music, or the Songster’s Companion}, ii. 36, 2nd edition, 1772. This
collection was printed by Robert Horsfield, in Ludgate Street, and
probably the words and music will also be found in the first edition,
which I have not seen.

The words are in several “Songsters,” such as “The new Pantheon
Concert, being a choice collection of the newest songs, sung this and
the last season, at the Pantheon, Vauxhall, Ranelagh, and other
places of entertainment,” 8vo., \textsc{n.d.} The tune is in Thompson’s
Collection of 200 Country Dances, iii. 99, (1775), in Straight and
Skillern’s Collection of 204 Country Dances, \&c.

Herd included a Scottified version of the words, in his \textit{Ancient
and Modern Scottish Songs}, 2nd edition, 1776, (together with “There
was a Jolly Miller,” “Old King Cole,” and sundry other English
songs), and he has since been copied by others.

James Hook (the author of \textit{The Lass of Richmond Hill}, and many
other charming songs) oomposed variations to this air, if not the air
itself. It is much in his style of composition.

\indentpattern{110110}
\settowidth{\versewidth}{And he twirl’d, he twirl’d at the pin;}
\begin{dcverse}
\begin{patverse}
\vin I saw not your father,\\
I saw not your mother,\\
But I saw your true love John;\\
He has met with some delay,\\
Which has caused him to stay,\\
But he will he here anon.
\end{patverse}

\begin{patverse}
\vin Then John he up arose,\\
And to the door he goes,\\
And he twirl’d, he twirl’d at the pin;\\
The lassie took the hint,\\
And to the door she went,\\
And she let her true love in.
\end{patverse}

\begin{patverse}
\vin Fly up, fly up,\\
My bonny grey cock,\\
And crow when it is day ;\\
Your breast shall be\\
Of the beaming gold,\\
And your wings of the silver grey.
\end{patverse}

\begin{patverse}
\vin Tho cock he proved false,\\
And untrue he was,\\
For he crow’d an hour too soon;\\
The lassie thought it day,\\
So she sent her love away,\\
And it prov’d hut the blink of the moon.
\end{patverse}
\end{dcverse}

\end{fixedpage}%731
\pagebreak

\setlength{\fixedpagewidth}{360pt}
\begin{fixedpage}%732
\versoheader

\musictitle{O DEAR, WHAT CAN THE MATTER BE?}

\textit{O dear, what can the matter be}? came into great public favour
towards the close of the last century, by being sung as a duet at
Harrison’s concerts. This must have been not later than 1792, as it
is entitled “the favorite \textit{duet}” in \textit{The British Lyre}, or \textit{Muses’
Repository}, the preface to which is dated Jan. 5th, 1793. It is
probably not many years older.

\indentpattern{0001}
\settowidth{\versewidth}{He promis’d he’d bring me a basket of posies,}
\begin{dcverse}
\begin{patverse}
Oh! dear! what can the matter be?\\
Dear! dear! what can the matter be?\\
Oh! dear! what can the matter be?\\
Johnny’s so long at the fair.
\end{patverse}

\begin{patverse}
He promis’d he’d bring me a basket of posies,\\
A garland of lilies, a garland of roses,\\
A little straw hat, to set off the blue ribbons\\
That tie up my bonny brown hair.
\end{patverse}
\end{dcverse}


\musictitle{THE LINCOLNSHIRE POACHER.}

This song is rather \textit{too} well known among the peasantry. A friend
informed me, twenty years ago, that he had heard it sung by several
hundred voices together, at Windsor, on the occasion of one of the
harvest-homes of King George IV.

It is well known to another class, by Mr. J. R. Planché’s
charming song, “In the Spring time of the year.”
\end{fixedpage}%732
\pagebreak

\setlength{\fixedpagewidth}{360pt}
\begin{fixedpage}%733
\rectoheader

\settowidth{\versewidth}{Oh ! ’tis my delight on a shining night, in the season of the year.}
\begin{scverse}
As me and my comarade were setting of a snare,\\
’Twas then we spied the gamekeeper, for him we did not care,\\
For we can wrestle and fight, my boys, and jump o’er anywhere,\\
Oh! ’tis my delight on a shining night, in the season of the year.

As me and my comarade were setting four or five,\\
And taking on 'em up again, we caught a hare alive,\\
We took the hare alive, my boys, and thro’ the woods did steer,\\
Oh ! ’tis my delight on a shining night, in the season of the year.

I threw him on my shoulder, and then we trudged home,\\
We took him to a neighbour’s house, and sold him for a crown,\\
We sold him for a crown, my boys, but I did not tell you where,\\
Oh! ’tis my delight on a shining night, in the season of the year.

Success to every gentleman that lives in Lincolnshire,\\
Success to every Poacher that wants to sell a hare,\\
Bad luck to every gamekeeper that will not sell his deer,\\
Oh! ’tis my delight on a shining night, in the season of the year.
\end{scverse}

\end{fixedpage}%733
\pagebreak

\setlength{\fixedpagewidth}{360pt}
\begin{fixedpage}%734
\versoheader

\musictitle{THE MANCHESTER ANGEL.}

The poaching song, “When I was bound apprentice in famous
Lincolnshire,” is sung to this tune in the North of England; and the
site is then changed from Lincolnshire to Lancashire.

A comparison of the two tunes will afford a curious instance of
the mode in which airs become corrupted and altered by untutored
singers. The construction proves, more than the actual resemblance of
notes, that these were originally one air. In each tune, the third
phrase (of four bars) is a repetition of the second, and the fourth
phrase, of the first. The second and third phrases of both are also
nearly the same. The question of priority of date is not easily
determined.

Among the ballads sung to this air are “The Sandgate Lass’s
Lament,” commencing, “It was a young maiden truly, that liv’d in
Sandgate Street,” and “The Manchester Angel,” commencing—

The fair maiden falls in love with the soldier who courts her,
and the song ends thus:— 

\settowidth{\versewidth}{I’ll go down to some nunnery, and there I’ll end my life;}
\begin{scverse}
\vleftofline{“}I’ll go down to some nunnery, and there I’ll end my life;\\
I never will get married—I will not be a wife—\\
But constant and true-hearted for ever I’ll remain;\\
I never will be married—till my soldier comes again.”
\end{scverse}
\end{fixedpage}%734
\pagebreak

\setlength{\fixedpagewidth}{360pt}
\begin{fixedpage}%735
\rectoheader

\musictitle{EARLY ONE MORNING.}

If I were required to name three of the most popular songs among
the servant-maids of the present generation, I should say, from my
own experience, that they are \textit{Cupid's Garden}, \textit{I sow'd the seeds of
love}, and \textit{Early one morning}. I have heard \textit{Early one morning} sung by
servants, who came from Leeds, from Hereford, and from Devonshire,
and by others from parts nearer to London.

The tune of \textit{Early one morning} was, I believe, first printed in my
collection of \textit{National English Airs}; but the words are contained in
many old song-books, such as \textit{Sleepy Davy’s Garland}, \textit{The Songster's
Magazine}, \&c.

In the \textit{National English Airs}, a version was printed from one of
the penny song-books collected by Ritson; and it is curious that
scarcely any two copies agree beyond the second line, although the
subject is always the same,—a damsel’s complaint for the loss of her
lover.

The following was given to me by the late R. Scrafton Sharpe, who
recollected it from childhood:—

\settowidth{\versewidth}{What was your cruel notion, to plough the raging ocean,}
\begin{scverse}
\begin{altverse}
\vleftofline{“}Early one morning, just as the sun was rising,\\
I heard a damsel to sing and to sigh.\\
Crying, O Cupid! O send my lover to me,\\
Send me my sailor, or else I shall die.
\end{altverse}

\begin{altverse}
How can you slight a young girl that loves you?\\
False-hearted young man! tell me for why.\\
What was your cruel notion, to plough the raging ocean,\\
And leave me behind you, to soh and to sigh?”
\end{altverse}
\end{scverse}

In \textit{Sleepy Davy’s Garland}, it commences thus:—

\begin{scverse}
\begin{altverse}
\vleftofline{“}Early one morning, near the sun rising,\\
I heard a damsel most sweetly to sing,\\
Crying, kind Cupid! pray now defend me,\\
Send my poor yielding heart into my breast again.”
\end{altverse}
\end{scverse}

Here the particular occupation of the youth is not mentioned; but
in \textit{The Songster’s Magazine} he is a “gentle shepherd.”

\begin{scverse}
\begin{altverse}
\vleftofline{“}Early one morning, just as the sun was rising,\\
I heard a pretty damsel to sigh and complain;\\
Oh! gentle shepherd, why am I forsaken?\\
Oh! why should I in sorrow complain?”
\end{altverse}
\end{scverse}

An utter disregard of rhyme pervades many of the copies, but they
nearly all end with, or include, a complaint like the following:—

\begin{scverse}
\begin{altverse}
\vleftofline{“}How can you slight a heart that doth love you,\\
Perjured young man, now tell me for why?\\
It was your false wooing that first prov’d my ruin,\\
And now for the falsest of men I must die.”
\end{altverse}
\end{scverse}

Of the tune I can say no more than that it bears relationship to
a hornpipe that was formerly played at the theatres, and was known by
the name of “Come, all you young blades that in robbing take
delight,” from a slang song, commencing with that line.
\end{fixedpage}%735
\pagebreak

\setlength{\fixedpagewidth}{360pt}
\begin{fixedpage}%736
\versoheader

The words here printed with the tune are those most frequently
sung in the present day:—

\settowidth{\versewidth}{Thus sung the poor maiden, her sorrows bewailing,}
\begin{scverse}
Oh! gay is the garland, and fresh are the roses\\
I’ve culled from the garden to bind on thy brow;\\
Oh, don’t deceive me! Oh, do not leave me!\\
How could you use a poor maiden so?

Remember the vows that you made to your Mary,\\
Remember the bow’r where you vow’d to be true ;\\
Oh, don’t deceive me! Oh, do not leave me!\\
How could you use a poor maiden so?

Thus sung the poor maiden, her sorrows bewailing,\\
Thus sung the poor maid in the valley below;\\
“Oh, don’t deceive me! Oh, do not leave me!\\
How could you use a poor maiden so?”
\end{scverse}


\musictitle{FAREWELL TO YOU, YE FINE SPANISH LADIES.}

The tune, and one verse of the words, of this famous old sea-song
were con tributed to my former collection by Lord Vernon. The words
seem to he very generally known, since I have been favored with
copies by Mr. W. Durrant Cooper, F.S.A., Mr. W. Sandys, F.S.A., and
Mr. Oliphant.

Mr. Durrant Cooper procured them from an old seaman, at Corson
Bay, Devon; Mr. Sandys from a hale and hearty septuagenarian friend,
Mr. J. C. Schetky. They have since been printed by Captain Marryat,
in his novel of \textit{Poor Jack}, and by Mr. J. H. Dixon, in \textit{Songs of the
Peasantry}.

The copies vary, but the limits of space will only permit me to
give two versions of the first stanza:—

\settowidth{\versewidth}{Now farewell to you, ye fine Spanish ladies;}
\begin{scverse}
\begin{altverse}
\vleftofline{“}Now farewell to you, ye fine Spanish ladies;\\
Now farewell to you, ye ladies of Spain!\\
For we’ve receiv’d orders to sail for old England,\\
And perhaps we may never more see you again.”
\end{altverse}
\end{scverse}
\end{fixedpage}%736
\pagebreak


\setlength{\fixedpagewidth}{360pt}
\begin{fixedpage}%737
\rectoheader

\settowidth{\versewidth}{Next Ram Head, off Plymouth, Start, Portland, and Wight;}
\begin{scverse}
\begin{altverse}
We’ll range and we’ll rove like true British sailors;\\
We’ll range and we’ll rove all on the salt seas;\\
Until we strike soundings in the channel of England;\\
From Ushant to Scilly is thirty-five leagues.
\end{altverse}

\begin{altverse}
We hove our ship to, with the wind at sou’west, boys,\\
We hove our ship to, for to strike soundings clear,\\
Then fill’d the main topsail, and bore right away, boys,\\
And straight up the Channel our course we did steer.
\end{altverse}

\begin{altverse}
The first land we made, it is called the Deadman,\\
Next Ram Head, off Plymouth, Start, Portland, and Wight;\\
We sailed by Beachy, by Fairly, and Dungeness,\\
And then bore away for the South Foreland Light.
\end{altverse}

\begin{altverse}
Then the signal was made for the grand fleet to anchor,\\
All in the Downs that night for to sleep;\\
Now stand by your stoppers, see clear your shank painters,\\
Hawi up your clew garnets, stick out tacks and sheets.
\end{altverse}

\begin{altverse}
Now let ev’ry man toss off a full bumper,\\
Now let ev’ry man take off his full bowl,\\
For we will be jolly, and drown melancholy,\\
With a health to each jovial and true-hearted soul.
\end{altverse}
\end{scverse}


\musictitle{O DEAR TWELVE PENCE.}

This song affords a whimsical exhibition of the uncertainty of
human resolution in point of matrimonial or domestic felicity:—

\settowidth{\versewidth}{O dear twelve pence, I’ve got twelve pence,}
\begin{scverse}
\begin{altverse}
\vleftofline{“}O dear twelve pence, I’ve got twelve pence,\\
I love twelve pence as I love my life;\\
I’ll grind a penny on’t, and I’ll end another on’t,\\
And I’ll carry tenpence home to my wife.”
\end{altverse}
\end{scverse}
\end{fixedpage}%737
\pagebreak

\setlength{\fixedpagewidth}{360pt}
\begin{fixedpage}%738
\versoheader

The last sum, however, by the diminuendo of two pence at each
verse, causes the song to end with—“I’ll carry nothing home to my
wife.”

Another version beginning—

\settowidth{\versewidth}{“I love sixpence, a jolly, jolly sixpence.”}
\begin{scverse}
“I love sixpence, a jolly, jolly sixpence.” 
\end{scverse}
will be found in
Ritson’s \textit{Gammer Gurton’s Garland, or The Nursery Parnassus}, and in
Rimbault’s \textit{Nursery Rhymes}.

The tune was contributed to my first collection by the late T.
Dibdin, “the last of the three Dibdins,” and I then recollected
having heard it sung to the words, “My man Thomas did me promise,”
when a schoolboy. This is my reason for printing those words to the
tune, and the reader can form his own opinion of the probability or
improbability of the tradition. I confess to small faith in anything
of the kind, when the tunes are not to be found in print or
manuscript; but some of the words of the nursery songs are certainly
very old, and very few of the airs were published until quite a late
period.

The words of “My man Thomas” are alluded to in \textit{A Crew of kind
Gossips, all met to be merrie}, by S[amuel] R[owlands] 4to., 1613, and
in Fletcher’s play, \textit{Monsieur Thomas}.

In the former, one of the wives says of her husband—

\settowidth{\versewidth}{And \textit{Pretty Birds}, with \textit{Garden Nightingale}:}
\begin{scverse}
\vleftofline{“}He hath a song cald \textit{Mistris will you doe}?\\
And \textit{My man Thomas did me promise} too;\\
He hath \textit{The Pinnace rig'd with silken saile}.\\
And \textit{Pretty Birds}, with \textit{Garden Nightingale}: \\
\textit{Ile tie my mare in thy ground}, a new way,\\
Worse than the players sing it in the play:\\
\textit{Besse for abuses}, and a number more \\
That you and I have never heard before.”
\end{scverse}

Of the above-named songs, \textit{A pinnace rigg'd with silken sail} is
preserved in a manuscript written in the time of James I., now in the
possession of Mr. Payne Collier, and \textit{I’ll tie my mare in thy ground} is
a well-known catch. In Fletcher’s \textit{Monsieur Thomas}, act iii., sc. 3,
the maid sings the first three lines of the following, and Thomas
answers:—

\end{fixedpage}%738
\pagebreak

\setlength{\fixedpagewidth}{360pt}
\begin{fixedpage}%739
\rectoheader

\musictitle{THE MOON SHALL BE IN DARKNESS.}

A fragment of another well-known street ditty, contributed by Mr.
Charles Sloman.


\musictitle{THE BLUE BELL OF SCOTLAND.}

Ritson prints this song in his \textit{North Country Chorister}, 1802,
under the title of “The new Highland Lad.” He says, in a note, “This
song has been lately introduced upon the stage by Mrs. Jordan, who
knew neither the words, nor the tune.” As to the words, all the
verses were not fit for the stage; therefore, Mrs. Jordan selected
four, made trifling alterations in them, and sang them to a tune of
her own. The old tune (although not at all like a Scotch air), is
included in Johnson’s \textit{Scots’ Musical Museum} (vi. 566). It has been
entirely superseded in popular favour by that of Mrs. Jordan.

“The blue bell of Scotland, a favourite ballad, as composed and
sung by Mrs. Jordan at the Theatre Royal, Drury Lane,” was entered at
Stationers’ Hall on the 13th of May, 1800, and the music published by
Longman and Co.

In the Douce Collection, p. 105, is a ballad entitled “Joyful
news for maids and young women,” \&c., “to the tune of \textit{The blue bells
of Ireland};” but I have not met with any tune under that name. The
burden of the ballad is—

\settowidth{\versewidth}{And the blue bells of Ireland}
\begin{dcverse}
\begin{altverse}
\vleftofline{“}And the blue bells of Ireland \\
Rings well and rings well;\\
And the blue bells of Ireland \\
Rings ding, dong, bell.”
\end{altverse}
\end{dcverse}
\end{fixedpage}%739
\pagebreak

\setlength{\fixedpagewidth}{360pt}
\begin{fixedpage}%740
\versoheader

The song quoted by Mr. C. Kirkpatrick Sharpe, “O, fair maid,whase
aught that bonny bairn,” is in a different metre, and could not he
sung to any of these airs. Mrs. Grant’s song, “O where, tell me
where, is your Highland laddie gone?” was written after this tune
had been rendered popular by Mrs. Jordan’s singing. Stenhouse (as
usual) gives a wrong date.

\settowidth{\versewidth}{Oh! where, and oh! where does your Highland laddie dwell?}
\begin{scverse}
Oh! where, and oh! where does your Highland laddie dwell?\\
He dwells in merry Scotland, at the sign of the Blue Bell;\\
And it’s oh! in my heart, that I love my laddie well.

What clothes, in what clothes is your Highland laddie clad?\\
His bonnet’s of the Saxon green, his waistcoat of the plaid;\\
And it’s oh! in my heart, that I love my Highland lad.

Suppose, oh! suppose that your Highland lad should die!\\
The bagpipes shall play over him,—I’ll lay me down and cry;\\
And it's oh! in my heart, that I wish he may not die.
\end{scverse}


\musictitle{THE COLLEGE HORNPIPE.}

All hornpipes in common time are of comparatively late
date,—perhaps in no case earlier than the last century, and generally
of the latter half.

The genuine old English hornpipe was in triple time, simple or
compound; and although, about the commencement of the last century,
some were reprinted, and then marked \timesig{6}{4}, they are, nevertheless, in \timesig{3}{2}
time. For instance, “The famous Darbysheire Hornpipe,” in “An
extraordinary Collection of pleasant and merry
\end{fixedpage}%740
\pagebreak


\setlength{\fixedpagewidth}{360pt}
\begin{fixedpage}%741
\rectoheader

Humours, never before published: containing Hornpipes, Jiggs,
North Country Frisks, Morrises, Bagpipe-Hornpipes, and Rounds, with
severall additional Fancies added; fit for all that play [in]
publick.” Although this collection was entered at Stationers’ Hall
in 1713 (21st May), the hornpipe was composed by Hale, the Derbyshire
piper, in the reign of Charles II. If there were not the copy of the
music printed under Hale’s portrait to refer to, the division, or
variation, would clearly prove it to be in triple time. In modern
notation, instead of \timesig{6}{8} time, it should be thus:—


I make these remarks because the manner of dancing the hornpipe
has certainly been changed. The stage hornpipes of the latter half of
the last century, and the steps taught by dancing-masters within the
last forty years to tunes in common time, cannot have agreed with the
ancient country way of dancing.

\textit{The College Hornpipe}, in spite of its extended compass, is the
tune to which an old sailor’s song, called \textit{Jack's the lad}, is sung. A
copy of the words, printed in Seven Dials, was once in my possession.

\end{fixedpage}%741
\pagebreak
742

\setlength{\fixedpagewidth}{360pt}
\begin{fixedpage}%742
\versoheader

\musictitle{THE STORMY WINDS DO BLOW.}

A fragment of an old sea song, contributed by Mr. Charles Sloman
in 1840, and the tune noted down from his singing. I have since
received several other copies, and fragments of various songs, which
have the same burden, and are sung to it. One is as follows:—

\settowidth{\versewidth}{With a comb and a glass in her band, her hand, her hand,}
\begin{scverse}
\indentpattern{0101123233}
\begin{patverse}
\vleftofline{“}One Friday morn when we set sail,\\
Not very far from land,\\
We there did espy a fair pretty maid\\
With a comb and a glass in her hand, her hand, her hand,\\
With a comb and a glass in her hand.\\
While the raging seas did roar,\\
And the stormy winds did blow,\\
While we jolly sailor boys were up unto the top,\\
And the land-lubbers lying down below, below, below,\\
And the land-lubbers lying down below.
\end{patverse}

\indentpattern{01016}
\begin{patverse}
Then up starts the captain of our gallant ship,\\
And a brave young man was he;\\
I’ve a wife and a child in fair Bristol town,\\
But a widow I fear she will be, \&c.\\
For the raging seas, \&c.
\end{patverse}

\begin{patverse}
Then up starts the mate of our gallant ship,\\
And a bold young man was he;\\
Oh! I have a wife in fair Portsmouth town,\\
But a widow I fear she will be, \&c.\\
For the raging seas, \&c.
\end{patverse}

\begin{patverse}
Then up starts the cook of onr gallant ship,\\
And a gruff old soul was he;\\
Oh! I have a wife in fair Plymouth town.\\
But a widow I fear she will be, \&c.\\
For the raging seas, \&c.
\end{patverse}

\begin{patverse}
And then up spoke the little cabin-boy,\\
And a pretty little boy was he;\\
Oh! I am more griev’d for my daddy and my mammy,\\
Than you for your wives all three, \&c.\\
For the raging seas, \&c.
\end{patverse}

\begin{patverse}
Then three times round went our gallant ship,\\
And three times round went she;\\
For the want of a life-boat they all went down,\\
And she sank to the bottom of the sea, \&c.\\
For the raging seas, \&c.’’
\end{patverse}
\end{scverse}

I have also the second, third, fourth, and fifth stanzas of the
above, with but slight variation, to another tune.

In the chorus of the following song, upon the word “flash” there
is a flourish which some singers omit. They hold on the first note of
it (D) as a dotted minim. It is, however, more frequently to be heard
as here printed.
\end{fixedpage}%742
\pagebreak

\setlength{\fixedpagewidth}{360pt}
\begin{fixedpage}%743
\rectoheader
music only

\end{fixedpage}%743
\pagebreak

\setlength{\fixedpagewidth}{360pt}
\begin{fixedpage}%744
\versoheader

\musictitle{LET THE TOAST PASS.}

This is the tune of Sheridan’s song of “Here’s to the maiden of
bashful fifteen,” in his comedy, \textit{The School for Scandal} (1777). The
second part is nearly the same as the first part of the very old
country dance, \textit{Half Hannikin} (ante p. 74).

\indentpattern{01014}
\settowidth{\versewidth}{Here’s to the charmer whose dimples we prize,}
\begin{dcverse}
\begin{patverse}
Here’s to the charmer whose dimples we prize,\\
Now to the damsel with none, sir,\\
Here’s to the girl with a pair of blue eyes,\\
And now to the nymph with but one, sir.\\
Let the toast pass, \&c.
\end{patverse}

\begin{patverse}
Here’s to the maid with a bosom of snow,\\
Now to her that’s as brown as a berry,\\
Here’s to the wife with a face full of woe,\\
And now to the damsel that’s merry.\\
Let the toast pass, \&c.
\end{patverse}

\begin{patverse}
For let her be clumsy, or let her be slim,\\
Young or ancient, I care not a feather;\\
So fill up a bumper, nay, fill to the brim,\\
And let us e’en toast ’em together.\\
Let the toast pass, \&c.
\end{patverse}
\end{dcverse}

\end{fixedpage}%7444
\pagebreak

\setlength{\fixedpagewidth}{360pt}
\begin{fixedpage}%745
\rectoheader

\musictitle{THE BARLEY-MOW.}

The barley-mow is a song still well known in many of the counties
of England. In Hertfordshire, it is frequently sung by the countrymen
in ale-houses after their daily labour. Mr. J. H. Dixon prints a
Suffolk version in his \textit{Songs of the Peasantry}, and Mr. Sandys, the
Devonshire and Cornwall version, in his \textit{Specimens of Cornish
Provincial Dialect}.

It is customarily chanted at the supper after the carrying of the
barley is completed, when the stack, rick, or mow of barley is
finished.-


The size of the drinking measure is doubled at each verse. The
brown bowl is supposed to contain half-a-pint; the next is “We’ll
drink it out of the pint, my boys;” then the quart, pottle, and
gallon, on to the barrel or hogshead, if the lungs of the singer
enable him to hold out for so many verses. The words increase in
number as the song goes on, for after “nipperkin, pipperkin,” the
singer adds one of the larger measures, pint, quart, pottle, \&c., at
each successive verse, always finishing (as in verse 1), “and the
brown bowl.”

This is after the manner of one of the \textit{Freemen’s Songs in
Deuteromelia}, beginning “Give us once a drink, gentle butler,” where
the singers first ask for the black bowl, then the pint pot, quart
pot, pottle, gallon, verkin (firkin), kilderkin, barrel, hogshead,
pipe, butt, and finally the tun.
\end{fixedpage}%745
\pagebreak

\setlength{\fixedpagewidth}{360pt}
\begin{fixedpage}%746
\versoheader

\musictitle{NEAR THE TOWN OF TAUNTON DEAN.}

A well-known Somersetshire tune, first printed in my former
collection. The words are here completed from a fragment by Mr. John
Oxenford, only the first six lines being old.

\indentpattern{110101011}
\settowidth{\versewidth}{Oh, of Taunton Dean she is the queen.}
\begin{dcverse}
\begin{patverse}
I’m saving up my money fast,\\
And will be rich at last,\\
Because I mean that girl to wed\\
Before a year is past,\\
I soon shall buy her wedding-dress,\\
E’en now I’ve bought the ring;\\
Oh, of Taunton Dean she is the queen.\\
And I shall be her king, her king,\\
And I shall be her king.
\end{patverse}

\begin{patverse}
The lads around are looking out\\
To win her heart, no doubt,\\
But I can watch as sharp as they,\\
And wield a cudgel stout.\\
So, youngster, now your distance keep ;\\
Upon my wedding-day\\
You shall be a guest to share the feast,\\
And help us to be gay, be gay,\\
And help us to be gay.
\end{patverse}
\end{dcverse}

\end{fixedpage}%746
\pagebreak

\setlength{\fixedpagewidth}{360pt}
\begin{fixedpage}%747
\rectoheader

\musictitle{O RARE BOTHAM BOY.}

A common country tune, the words of which I have been unable to
obtain. 

\musictitle{WE ARE POOR FROZEN OUT GARDENERS.}

This is the tune of many songs. If the reader should meet any
half-a-dozen men perambulating the streets of London together, and
singing, the probabilities are great that they sing to this tune.
Sometimes the men are dressed like sailors; at other times they look
like workmen out of employment. I recollect hearing the tune at
Kilburn, full forty years ago, and have, with tolerable annual
regularity, ever since. I regret never having stopped to hear the
words.
\end{fixedpage}%747
\pagebreak

\setlength{\fixedpagewidth}{360pt}
\begin{fixedpage}%748
\versoheader

\musictitle{RANTERS' HYMNS.}

The Primitive Methodists, or Ranters, acting upon the principle
of “Why should the devil have all the pretty tunes?” collect the
airs which are sung at pot and public houses, and write their hymns
to them. If the original words should be coarse, or indelicate, they
are thought the more to require this transformation. I do not stop to
enquire whether the hearers can readily divest themselves of the old
associations,—the motive is good, without doubt, however ill-directed
the effort.

In this sect we have living examples of the “puritans who sing
psalms to hornpipes.” They do not mince the matter by turning them
into slow tunes, and disguising them by harmony, but sing them in
their original lively time.

The system of employing secular music for sacred purposes is not,
however, confined to Ranters. Even now, in France, Roman Catholic
children sing their
\textit{cantiques} in the churches to—

\settowidth{\versewidth}{C’est l'amour, l’amour, l’amour}
\begin{scverse}
\vleftofline{“}C’est l'amour, l’amour, l’amour \\
Qui fait la monde à la ronde;”
\end{scverse}
and to other tunes of the same class: nor are we of the Church of
England very unlike them, while a portion of our clergy will have
such an Advent Hymn, as “Lo! He comes, in clouds descending,” to the
the tune of—

\begin{scverse}
“Guardian Angels, now protect me.\\
Send to me the youth I love”—
\end{scverse}
\end{fixedpage}%748
\pagebreak

\setlength{\fixedpagewidth}{360pt}
\begin{fixedpage}%749
\rectoheader

(a song in \textit{The Golden Pippin}); or sing other hymns to such tunes
as \textit{Rousseau's Dream}, a pantomime air in J. J. Rousseau’s opera, \textit{Le
Devin du Village}, It is inexcusable with us, for no Church can boast
of finer music in the true ecclesiastical style.

The following is one of the Ranters’ most favorite hymns:—
\end{fixedpage}%749
\pagebreak

\setlength{\fixedpagewidth}{360pt}
\begin{fixedpage}%750
\musictitle{THE CORN GRINDS WELL.}

This is a well-known tune to which several songs are sung. , I am
told that \textit{The Derbyshire Miller} is one, but have not seen the words.

The following fragment was noted down, with the tune, from the
singing of Mr. Charles Sloman:—

\musictitle{CHRISTMAS CAROLS.}

Christmas Carols were of two sorts: the one serious, and commonly
sung through the streets, or from house to house, to usher in the
Christmas morning; the other of a convivial character, and adapted to
the festive entertainments of the season. We have seen how, in the
fifteenth century, a minstrel could make one tune to answer for
both,—singing

\settowidth{\versewidth}{This is the salutation of the Angel Gabriel,” }
\begin{scverse}
“Nowell, nowell, nowell, nowell,\\
This is the salutation of the Angel Gabriel,” 
\end{scverse}
in the morning, to
the same tune as

\settowidth{\versewidth}{For our blessed lady’s sake, bring us in good ale,”}
\begin{scverse}
“Bring us in good ale, and bring us in good ale,\\
For our blessed lady’s sake, bring us in good ale,”
\end{scverse}
\end{fixedpage}%750
\pagebreak

\setlength{\fixedpagewidth}{360pt}
\begin{fixedpage}%751
\rectoheader

in the evening (ante i. 42); but he adds “If so be that ye will
have another tune, it may be at your pleasure,” and I have no doubt
that the festive carols were usually sung to dance-tunes. I have
found many which are directed to be sung to such airs, and one of the
significations of the word “caroling,” and the sense in which it was
most frequently used in the fourteenth and fifteenth centuries, was
to sing or warble to dancing. (See Chaucer \textit{passim}.) Caroling was
afterwards used to express the singing or warbling of a lively tune,
with or without the dancing.

I imagine the word to be used in this sense by Trevisa, vicar of
Berkeley, who, in the year 1398, made a free translation of a book on
the nature and qualities of different things, written in Latin about
thirty years before, by an English Franciscan friar (Bartholomæus,
De Proprietatibus Rerum). He tells us that when boys had passed the
age of seven years, they were “sette to lernynge, and compellid to
take lernynge and chastysynge.” That at that age, they are “plyaunt
of body, able and lyghte to moevingc, \textit{wytty to lerne carolles}, and
wythoute besynesse, and drede noo perylles more than betynge with a
rodde; and they love an apple more than golde,” \&c. I suspect that
the boys were more ready to warble lively tunes, and perhaps to catch
up a few of the words, than to learn religious songs.

Warton, in his \textit{History of English Poetry}, attributes the
introduction of the religious carol to the Puritans; but this is
clearly a mistake, for there are many extant which were in use long
before the age of puritanism. Nevertheless, the “jolly carols,” as
Tusser calls them, were by far the more popular in early times.

\indentpattern{010101016}
\settowidth{\versewidth}{And pype and dansen, and hem rage ; ne swinke [\ie , labour]}
\begin{scverse}
\begin{patverse}
“The lewid peple than algates agre,\\
And caroles singen ever? Criste messe tyde,\\
Not with schamfastenes, bot jocondle ;\\
And, holly bowghes aboute, and al asydde\\
The brenning fyre, hem eten and hem drinke,\\
And langhen mereli, and maken route ;\\
And pype and dansen, and hem rage ; ne swinke [\ie , labour]\\
Ne noe thynge els, twalve daye’ thei woldè not.”\\
\textit{Lud. Coll}., xlv. H. 1.
\end{patverse}
\end{scverse}


The oldest printed collection of Christmas Carols is that which
was published by Wynkyn de Worde in 1521, but the songs are of a
festal character, including the famous “Boar’s-head Carol,” which is
still sung annually, on Christmas Day, at Queen’s College, Oxford.

“In the West of England,” says Mr. Sandys, “and especially in the
western parts of Cornwall, carol-singing is still kept up, the
singers going about from house to house, wherever they can obtain
encouragement.” In the West of England also, until very lately,
rejoicings of all kinds commenced on Christmas Eve. The day was
passed in the ordinary manner; “but at seven or eight o’clock in the
evening, cakes were drawn hot from the oven; cyder or beer
exhilarated the spirits in every house; and the singing of carols was
continued late into the night. On Christmas Day, these carols took
the place of psalms in all the churches, especially at afternoon
service, the whole congregation joining; and at the end, it was usual
for the parish clerk to declare, in a loud voice, his wishes for a
merry Christmas and a happy new year to all the parishioners.”
(Preface to \textit{Christmas Carols}, \&c., by Davies Gilbert, 2nd edit.,
1823.)
\end{fixedpage}%751
\pagebreak

\setlength{\fixedpagewidth}{360pt}
\begin{fixedpage}%752
\versoheader

According to Wordsworth, the singing of carols also commenced in
the North of England on Christmas Eve. In some lines addressed to his
brother, the Rev. Dr. Wordsworth, he writes thus:—

\settowidth{\versewidth}{That scraped the chords with strenuous hand.}
\begin{dcverse}
\vleftofline{“}The minstrels played their Christmas tune,\\
To-night beneath my cottage eaves :\dots \\
Keen was the air, but could not freeze,\\
Nor check the music of their strings;\\
So stout and hardy were the band\\
That scraped the chords with strenuous hand.

And who but listen’d? till was paid\\
Respect to every inmate's claim;\\
The greeting given, the music played\\
In honour of each household name,\\
Duly pronounced with lusty call,\\
And ‘merry Christmas’ wished to all!\dots 

For pleasure hath not ceased to wait\\
On these expected annual rounds,\\
Whether the rich man’s sumptuous gate\\
Call forth the unelaborate sounds,\\
Or they are offered at the door\\
That guards the lowliest of the poor.

How touching, when at midnight sweep\\
Snow-muffled winds, and all is dark,\\
To hear—and sink again to sleep!\\
Or, at an earlier call, to mark,\\
By blazing fire, the still suspense\\
Of self-complacent innocence.

The mutual nod—the grave disguise\\
Of hearts with gladness brimming o’er;\\
And some unbidden tears that rise\\
For names once heard, and heard no more;\\
Tears brightened by the serenade\\
For infant in the cradle laid!\\
Hail, ancient manners! sure defence,\\
Where they survive, of wholesome laws,” \&c.
\end{dcverse}

The singing of religious carols is also heard in some of the
midland counties,
and, even in the streets of London, boys go about on the morning
of Christmas
Day, singing and selling them. Hone gives a list of eighty-nine
carols in use
within the last few years, excluding the numerous compositions
published by
religious societies, under the name of carols.

The reader who seeks for information about carols, wassail songs,
and other
celebrations of Christmas, will find an ample fund of amusement
and instruction
in \textit{Christmas-tide, its History, Festivities, and Carols}, by W.
Sandys, F.S.A., and
some further collections towards the history of carol-singing in
the preface to
\textit{A little book of Christmas Carols}, by Edward F. Rimbault, LL.D.

To Mr. Sandys’s Collection I am chiefly indebted for the
following traditional
tunes to religious carols:—

\musictitle{GOD REST YOU, MERRY GENTLEMEN.}

The words of this carol are in the Roxburghe Collection (iii.
452), together
with three other “choice Carols for Christmas Holidays,” for St.
Stephen’s,
St. John’s, and Innocents’ days. The tune was printed by Hone, in
his \textit{Facetiæ},
to a “political Christmas Carol,” beginning—

\settowidth{\versewidth}{God rest you, merry gentlemen,}
\begin{dcverse}
\begin{altverse}
\vleftofline{“}God rest you, merry gentlemen, \\
Let nothing you dismay; \\
Remember we were left alive \\
Upon last Christmas Day,\\
With both our lips at liberty,\\
To praise Lord C[astlereag]h\\
For his ‘practical’ comfort and joy,” \&c.
\end{altverse}
\end{dcverse}

I have seen no earlier copy of the tune than one in the
handwriting of Dr. Nares, the cathedral composer, in which it is
entitled “The old Christmas Carol;” but I have received many versions
from different sources, for no carol seems to be more generally
known.

In the Halliwell Collection of Broadsides, No. 263, Chetham
Library, is “The overthrow of proud Holofernes, and the Triumph of
virtuous Queen Judith; to the tune of \textit{Tidings of comfort and joy}.” As
those words form the burden of “God rest you, merry gentlemen,” the
two are to the same air.
\end{fixedpage}%752
\pagebreak

\setlength{\fixedpagewidth}{360pt}
\begin{fixedpage}%753
\rectoheader

The May-day, or Mayers’ Song, which is printed by Hone, in his
\textit{Every-day Book} (i. 569), “as sung at Hitchin, in Hertfordshire,” is
also to this tune. It is a semi-religious medley,—a puritanical
May-song (“of great antiquity,” says Hone), and begins thus:—

\settowidth{\versewidth}{We have been rambling all-the night,}
\begin{dcverse}
\begin{altverse}
\vleftofline{“}Remember us poor Mayers all, \\
And thus we do begin, \\
To lead our lives in righteousness, \\
Or else we die in sin. \\
We have been rambling all-the night,\\
And almost all the day,\\
And now, returned back again,\\
We have brought you a branch of May.”
\end{altverse}
\end{dcverse}

The carol is sometimes sung in a major key, and sometimes in a
minor; besides which difference, scarcely any two copies agree in the
second part.

Having printed Hone’s version, and one in a major key, in
\textit{National English Airs}, I now give a third copy, noted down by Dr.
Rimbault. It has a repetition of words at the end which some others
have not.
\end{fixedpage}%753
\pagebreak

\setlength{\fixedpagewidth}{360pt}
\begin{fixedpage}%754
\versoheader

\indentpattern{0009}
\settowidth{\versewidth}{But when to Bethlehem they came, where our dear Saviour lay, They}
\begin{scverse}
\begin{patverse}
In Bethlehem, in Jewry, this blessed babe was born,\\
And laid within a manger, upon this blessed morn;\\
The which his mother Mary did nothing take in scorn.\\
O tidings, \&c.
\end{patverse}

\begin{patverse}
From God, our Heavenly Father, a blessed Angel came,\\
And unto certain Shepherds brought tidings of the same,\\
How that in Bethlehem was born the Son of God by name.\\
O tidings, \&c.
\end{patverse}

\begin{patverse}
Fear not, then said the Angel, let nothing you affright,\\
This day is horn a Saviour of a pure Virgin bright,\\
To free all those who trust in Him from Satan’s pow’r and might.\\
O tidings, \&c.
\end{patverse}

\begin{patverse}
The Shepherds at those tidings rejoiced much in mind,\\
And left their flocks a feeding, in tempest, storm, and wind,\\
And went to Bethlehem straightway, this blessed babe to find.\\
O tidings, \&c.
\end{patverse}

\begin{patverse}
But when to Bethlehem they came, where our dear Saviour lay, \\
They found Him in a manger where oxen feed on hay;\\
His mother Mary, kneeling, unto the Lord did pray.\\
O tidings, \&c.
\end{patverse}

\begin{patverse}
Now to the Lord sing praises, all you within this place,\\
And with true love and brotherhood each other now embrace; \\
This holy tide of Christmas all others doth deface.\\
O tidings, \&c.
\end{patverse}
\end{scverse}

\musictitle{GOD BLESS YOU, MERRY GENTLEMEN.}

Another carol tune, to the same words, from Sandys’s Collection.
\end{fixedpage}%754
\pagebreak


\setlength{\fixedpagewidth}{360pt}
\begin{fixedpage}%755
\rectoheader

\musictitle{A VIRGIN MOST PURE.}

A Christmas Carol still sung in the West of England, taken from
Mr. Sandys’s Collection. The tunes of this and other Carols are not
exclusively appropriated to the words with which they are here
united; various Carols are sung to each air.

\indentpattern{00006}
\settowidth{\versewidth}{That Joseph and Mary, whose substance was small,}
\begin{dcverse}
\begin{patverse}
In Bethlehem city, in Jewry it was,\\
Where Joseph and Mary together did pass,\\
And there to be taxed, with many one mo’,\\
For Caesar commanded the same should be so.\\
Rejoice and be merry, \&c.
\end{patverse}

\begin{patverse}
But, when they had entered the city so far,\\
The number of people so mighty was there,\\
That Joseph and Mary, whose substance was small,\\
Could get in the city no lodging at all. \\
Rejoice, \&c.
\end{patverse}

\begin{patverse}
Then they were constrain’d in a stable to lie,\\
Where oxen and asses they used to tie ;\\
Their lodging so simple, they held it no scorn,\\
But against the next morning our Saviour was born.\\
Rejoice, \&c.
\end{patverse}
\end{dcverse}
\end{fixedpage}%755
\pagebreak


\setlength{\fixedpagewidth}{360pt}
\begin{fixedpage}%756
\versoheader

\indentpattern{00006}
\settowidth{\versewidth}{Then the Shepherds consent, and to Bethlehem did go,}
\begin{dcverse}
\begin{patverse}
The King of all Glory to the world being brought,\\
Small store of fine linen to wrap him was wrought;\\
When Mary had swaddled her young Son so sweet,\\
Within an ox manger she laid him to sleep.\\
Rejoice, \&c.
\end{patverse}

\begin{patverse}
Then God sent an Angel from heaven so high,\\
To certain poor Shepherds in fields where they lie,\\
And bid them no longer in sorrow to stay,\\
Because that our Saviour was born on this day.\\
Rejoice, \&c.
\end{patverse}

\begin{patverse}
Then presently after, the Shepherds did spy\\
A number of Angels appear in the sky,\\
Who joyfully talked, and sweetly did sing,\\
To God he all Glory, our Heavenly King.\\
Rejoice, \&c.
\end{patverse}

\begin{patverse}
Three certain Wise Princes, they thought it most meet\\
To lay their rich off’rings at our Saviour’s feet;\\
Then the Shepherds consent, and to Bethlehem did go,\\
And when they came thither, they found it was so.\\
Rejoice, \&c
\end{patverse}
\end{dcverse}

\musictitle{THE FIRST NOWELL.}

A Carol for the morning of Christmas Day; the tune from Mr.
Sandys’ Collection.
\end{fixedpage}%756
\pagebreak

\setlength{\fixedpagewidth}{360pt}
\begin{fixedpage}%757

\indentpattern{00006}
\settowidth{\versewidth}{And which to the earth did give a great light,}
\begin{dcverse}
\begin{patverse}
They looked above, and there saw a star\\
That shone in the East, beyond them afar,\\
And which to the earth did give a great light,\\
And so it continued by day and by night.\\
Nowell, \&c.
\end{patverse}

\begin{patverse}
And by the light of that same star\\
Three wise men came from a country afar;\\
To seek for a King it was their intent,\\
And to follow the star wherever it went.\\
Nowell, \&c.
\end{patverse}

\begin{patverse}
The star went before them unto the North-West,\\
At length over Bethlehem seemed to rest,\\
And there did remain by night and by day,\\
Right over the place where Jesus Christ lay.\\
Nowell, \&c.
\end{patverse}

\begin{patverse}
The wise men did know then, assuredly,\\
The King whom they sought in that house must be;\\
So one enter’d in, the babe for to see,\\
And found him surrounded by poverty.\\
Nowell, \&c.
\end{patverse}

\begin{patverse}
Then entered in those wise men three,\\
Most reverently, with bended knee,\\
And offered there, in his presence,\\
Both gold and myrrh, with frankincense.\\
Nowell, \&c,
\end{patverse}

\begin{patverse}
Between the stalls of an ox and ass,\\
This child there truly born he was ;\\
For want of bed-clothing they did him lay\\
All in the manger, among the hay.\\
Nowell, \&c.
\end{patverse}

\begin{patverse}
Then let us all, with one accord,\\
Sing praises to our heavenly Lord,\\
That made both heaven and earth of nought,\\
And with his blood mankind hath bought.\\
Nowell, \&c.
\end{patverse}

\begin{patverse}
For if we, in our time, do well,\\
We shall be freed, in death, from hell,\\
And find, instead of Satan’s thrall,\\
A heavenly resting-place for all.\\
Nowell, \&c.
\end{patverse}
\end{dcverse}

\musictitle{THE BOAR’S HEAD CAROL.}

From Dr. Rimbault’s \textit{Little Book of Christmas Carols}. This carol possesses
historical interest, as being still sung annually, on Christmas Day, at Queen’s
College, Oxford.

\indentpattern{00036}
\settowidth{\versewidth}{When thus bedeck’d with a gay garland,}
\begin{dcverse}
\begin{patverse}
\vleftofline{\textit{Solo}. }The boar’s head, as I understand,\\
Is the bravest dish in all the land ;\\
When thus bedeck’d with a gay garland,\\
Let us \textit{servire cantico}.\\
\textit{Chorus}. Caput apri, \&c.
\end{patverse}

\begin{patverse}
\vleftofline{\textit{Solo}. }Our steward hath provided this\\
In honour of the King of Bliss ;\\
Which on this day to be served is\\
\textit{In regimensi atrio}.\\
\textit{Chorus}. Caput apri, \&c.
\end{patverse}
\end{dcverse}

\end{fixedpage}%757
\pagebreak

\setlength{\fixedpagewidth}{360pt}
\begin{fixedpage}%758
\versoheader

The above are altered from the old words printed by Wynkin de
Worde. In his day, every one was supposed to be able to sing. Mark
the difference in the
third line:—

\settowidth{\versewidth}{Be gladde, lordes, bothe more and lasse,}
\begin{dcverse}
\indentpattern{000366}
\begin{patverse}
\vleftofline{“}The bore’s heed in hande bring I,\\
With garlans gay and rosemary ;\\
\textit{I pray you all synge merely}\\
Qui estis in convivio.\\
Caput apri defero\\
Reddens laudes Domino.
\end{patverse}

\begin{patverse}
The bore’s heed I understande\\
Is the chefe servyce in this lande;\\
Loke, where ever it be fande\\
Servite cum cantico.\\
Caput apri, \&c.
\end{patverse}

\indentpattern{00006}
\begin{patverse}
Be gladde, lordes, bothe more and lasse,\\
To chere you all this Christmasse,\\
For this hath ordeyned our stewards,\\
The bore’s heed with mustarde.\\
Caput apri, \&c.”
\end{patverse}
\end{dcverse}

\centerrule

\end{fixedpage}%758
\pagebreak

\setlength{\fixedpagewidth}{360pt}
\begin{fixedpage}%759

\headingthree{APPENDIX.}

\headingfour{CONTAINING ADDITIONAL REMARKS, \&c.}

p. 4. \textsc{Aldhelm, Abbot of Malmsbury}.—The first specimen of musical
notation given by the learned Abbot Gerbert, in his \textit{De Cantu et
Musica Sacra, a prima ecclesiæ ætate} (i. 202), is to a poem by St.
Aldhelm, in Latin hexameters, in praise of virginity. This was
written for the use of Anglo-Saxon nuns. The manuscript from which it
is taken is, or was, in the monastery of St. Blaise, in the Black
Forest, and Gerbert dates it as of the ninth or tenth century. It
contains various poems of St. Aldhelm, all of which are with music,
and the \textit{Paschale Carmen} of Sedulius, one of the early Irish
Christians, which is without music. Many very early English and Irish
manuscripts were, without doubt, taken to Germany by the English and
Irish priests, who assisted in converting the Germans to
Christianity. St. Boniface, “the apostle of Germany,’’ and first
Archbishop of Mentz (Mayence), who was killed in the discharge of his
duties in the year 755, was an Anglo-Saxon whose name had been
changed from Winfred to Boniface by Pope Gregory II. “Boniface seems
always to have had a strong prejudice in favour of the purity of the
doctrines of the church of his native country, as they had been
handed down by St. Augustine; in points of controversy he sought the
opinions of the Anglo-Saxon bishops, even in opposition to those
inculcated by the pope; and he sent for multitudes of Anglo-Saxons,
of both sexes, to assist him in his labours.”(\textit{Biog. Brit. Lit}., i.
315.) He placed English nuns over his monastic foundations, and
selected his bishops and abbots from among his countrymen. His
successor in the Archbishoprick was also an Englishman.

To revert to St. Aldhelm—Faricius (a foreign monk of Malmsbury).
who wrote his life about the year 1100, tells us that he exercised
himself daily in playing upon the various musical instruments then in
use, whether with strings, pipes, or any other variety by which
melody could be produced. The words are, “Musicæ autem artis omnia
instrumenta quæ fidibus vel fistulis aut aliis varietatibus melodiæ
fieri possunt, et memoria tenuit et in cotidiano usui
habuit.”(\textit{Faricius}, Col. 140, vo.) The anecdote of Aldhelm’s
stationing himself on the bridge in the character of a glee-man or
minstrel, to arrest the attention of his countrymen who were in the
habit of hurrying home from church when the singing was over, instead
of waiting for the exhortation, or sermon; and of his singing poetry
of a popular character to them in order to induce them gradually to
listen to more serious subjects,—was derived by William of Malmsbury
from an entry made by King Alfred in his manual or note-book. Aldhelm
died
\end{fixedpage}%759
\pagebreak

\renewcommand\versoheadertext{appendix}

\setlength{\fixedpagewidth}{380pt}
\begin{fixedpage}%760
\versoheader

in 705, and King Alfred in 901,—yet William of Malmsbury, who
flourished about 1140, tells us that one of the “trivial songs” to
which Alfred alludes as written by Aldhelm for one of these
occasions, was still sung by the common people.\textsuperscript{a} The literary
education of youth, even of the upper classes, in Anglo-Saxon times,
was limited to the being taught to commit the songs and literature of
their country to memory. Every one of gentle blood was instructed in
“harp and song,” but it was only thought necessary for those who were
to be priests or minstrels to be taught to read and write.

p. 4. \textsc{St. Dunstan}.—Osbern, the monk of Canterbury, who wrote the
life of Dunstan soon after 1070, says that when a boy of fifteen, he
was a great favorite at the Court of King Athelstan, on account of
his various accomplishments, especially for his skill in music. That
when “he saw the King and his nobles weary with labouring on the
affairs of state, he cheered them all by singing and playing on the
harp and other instruments.”\textsuperscript{b} One of the stories related of him is
that he had an enchanted harp, which performed tunes without the
agency of man, when hung against a wall,—a thing by no means
impossible in houses that would not keep out the wind. He was
requested by a lady to assist in designing ornaments for a handsome
stole. Dunstan, as usual, carried his harp with him (sumpsit secum
cytharam suam, quam lingua paterna \textit{hearpan} vocamus), and when he
entered the apartment of the ladies, he hung it beside the wall; and
in the midst of their work they were astonished by strains of
excellent music which issued from the instrument. (\textit{Bridferth}, fol.
70, vo.)

p. 6, l. 4. \textsc{The Anglo-Saxon Fiddle or Fithele}.—As I observe that
M. Fétis, in his Recherches Historiques et Critiques sur l’origine
et les transformations des

\begin{dcfootnote}
\textsuperscript{a} “Nativæ quoque linguæ non negligebat carmina; adeo ut, teste
libro Elfredi de quo superius dixi, nulla unquam ætate par ei fuit
quispiam, poesim Anglicam posse facere, tantum componere, eadem
apposite vel canere vel dicere. Denique commemorat Elfridus carmen
triviale, quod adhuc vulgo cantitatur, Aldhelmus feciase; adjiciens
causam qua probat rationabiliter tantum virum his quaæ videntur
frivola institisse: populum eo tempore semi-barbarum, parum divinja
sermonibus intentum, statim cantatis missis domos cursitare solitum;
ideoque sanctum virum super pontem qui rura et urbem continuat,
abeuntibus se opposuisse obicem, quasi artem cantandi professum. Eo
plus quam hoc commento, sensim inter ludicra verbis scripturarum
insertis, cives ad sanitatem reduxisse; qui si severe et cum
excommunicatione agendum putasset, profecte profecisset nihil.”(\textit{Biog,
Brit. Lit}., i. 215.)

\textsuperscript{b} The passage is “Iterum cum videret dominum regem sæcularibus
curis fatigatum, psallebat in timphano sive in cithara, sive alio
quolibet musici generis instrumento, quo facto tam regis quam omnium
corda principum exhilarabat.” (Osbern, \textit{Vit. Dunst}., p. 94.) I have
not attempted to translate “in timphano” in the above extract, for
although commonly rendered “timbrel, tabor, or drum,” I believe a
kind of bagpipe is here intended. Taking an English manuscript of the
tenth century (Tiberius, c. vi., in the Cotton Collection), we find
“Tympanum pellis pillacis est in flata, abens calamos duos in labiis et
unum in collo,” The meaning of ‘‘pillacis” is not very clear, but
between “pilax,” a catskin, and “pilosus,” hairy, the passage may be
translated, “The
tympanum is a musical instrument made of the skin of an animal,
inflated, having two pipes in the lips and one in the neck.” If by
“in the lips” the lips of the animal are intended, and the pipe in the
neck was at the back, ready for the lips of the player, the tympanum
of the tenth century probably resembled an instrument depicted in
Gerbert’s \textit{De Cantu}, Vol. 2, Tab. xxxiv., No. 22, where a man holds
up a pigskin, blowing in at the back of the neck, and having his arms
on the sides, ready to squeeze out the wind. This pig, however, has
only one pipe in the lips.

There ia a difficulty in translating the names of several
instruments which come to us from the Latin, and to the Latin from
the Greek. We have to consider not only the time, but also the
country of the writer. In a Latin Psalter of the eighth century,
with Anglo-Saxon interlineation (Vesp., A. 1, Cotton Coll.), we find
the instruments mentioned in the 150th psalm translated thus— “in
tympano—\textit{in timpanan}” in sono tube—\textit{hornes},” “in psalterio—\textit{hearpan},”
“in cithara—\textit{citran},”“in organo—\textit{organan},” “in cymbalis—\textit{cymbalan}.” “In
coro” is there rendered as “by many people"—not, as sometimes, a
musical instrument. If, however, we go from the eighth to the tenth
century, we find in the manuscript above quoted (\textit{Tib}., C. vi.),
“Corus est pellis simplex cum duobus cicutis,” and a delineation of
the instrument,—a skin stretched like a drum-head in the curve formed
by two pipes, evidently intended for percussion, and not for
inflation. In the fifteenth century, we find, “chorus, a crowde, an
instrument of musyke."
\end{dcfootnote}
\end{fixedpage}%760
\pagebreak

\renewcommand\rectoheadertext{origin of the fiddle or violin.}

\setlength{\fixedpagewidth}{360pt}
\begin{fixedpage}%761
\rectoheader

\textit{instruments à archet} (prefixed to \textit{Antoine Stradivari}, 8vo.,
Paris, 1856), gives no examples of instruments of the violin kind of
such early dates as we have in England, and as my account will differ
in many essential points from his, I am tempted first to reproduce an
Anglo-Saxon “fithele,” from a manuscript of the tenth century in the
British Museum (Cotton MSS., Tiberius, c. vi.), although it has been
accurately copied by Strutt, and was published in his \textit{Sports and
Pastimes of the People of England}

\noindent\includegraphics*[width=\textwidth]{images/fithele.pdf}

Here we find the four strings and the bow. The bridge is not
shown in the manuscript, but as bridges had certainly been in use for
two centuries on other instruments, there can be no doubt that it had
one. M. Fétis remarks, very justly, that the bridge is frequently
omitted in early drawings and sculptures (although to be found in
others of corresponding dates), and even in books of the sixteenth
century. Some draughtsmen were less exact in detail than others.

The bent sides of the fiddle seem to have been introduced for the
purpose of holding the instrument between the knees. We find them so
formed, and held between the knees, in sculpture of the twelfth
century, yet not greatly increased in size.

The long neck is found in the thirteenth and fourteenth
centuries, and we have a fiddle of this kind, with the body shaped
like a cruth or crowd (\ie , somewhat resembling that of a long,
narrow guitar), in the brass monumental plate of Robert Braunch,
erected at Lynn, in Norfolk, in 1364.

The use of the upper part of the finger-board is shewn in one of
the sculptures on the outside of St. John’s Church, Cirencester,
where the player on a fiddle with a neck of well-proportioned length,
has his left hand close to the body of the instrument, and the bow
upon the strings, as if ready to draw out the high notes. The date is
between 1504 and 1522.

The history of instruments of the violin tribe is interesting,
because the use of the bow was wholly unknown to the ancients. The
two earliest instruments known to
\end{fixedpage}%761
\pagebreak

\setlength{\fixedpagewidth}{360pt}
\begin{fixedpage}%762
\versoheader

have been played with a bow, are the crowd and fiddle. Having
made but slight search, I have not found any drawing of the crowd or
cruth of so early a date as the above fiddle, yet the crowd was in
all probability the precursor of the fiddle. The former is mentioned
as an English instrument by Venantius, an Italian poet, who wrote in
the year 566, and about 570. In his elegiac poem to Loup, Due de
Champagne, he thus addresses him:—

\indentpattern{002}
\settowidth{\versewidth}{Romanusque lyra plaudat tibi, Barbarus harpa,}
\begin{scverse}
\begin{patverse}
\vleftofline{“}Romanusque lyra plaudat tibi, Barbarus harpa,\\
Græcus achilliaca, chrotta Britanna canat.”\\
\textit{Venantii Fortunati Poemata}, edit. 1617, 4to., p. 169.
\end{patverse}
\end{scverse}



“Let the Roman applaud thee with the lyre, the Barbarian with the
harp, the Greek with the cithara (?), \textit{let the British crowd sing}”
The last phrase is particularly expressive, as the crowd is the only
instrument of those above named, that could sustain its tone. There
are some differences of opinion as to the origin of the word. \textit{Crowd},
according to Spelman, is “crotta, fidicula Britannica.’' Although
Skinner derives it from the Anglo-Saxon word, \textit{cruth}, which signifies
a \textit{crowd} in the sense of a \textit{multitude}, Nares says, “certainly from the
Welsh \textit{crwth}.” That instrument remained in use in Wales within the
last century. An engraving of the modern \textit{crwth} will be seen in
Jones’s \textit{Welsh Bards}, i. 89; the ancient one was smaller, and had but
three instead of six strings. There were apertures in both to admit
the left hand of the player through the back, so as to enable him to
press the strings down upon the finger-board (for the distinguishing
feature of the \textit{crwth} was that it had no neck); yet the ancient
differed from the modern in shape. The former, from and after the
eleventh century, was not unlike the body (only) of a very long and
narrow-formed Spanish guitar.

The fiddle retained its Anglo-Saxon name of \textit{fithele}, in England,
for at least a hundred and fifty years after the Norman conquest
(see, for instance, Layamon’s romance of \textit{Brut}); but the Normans, not
approving the pronunciation of the “th” (which is represented by a
single letter, ð, in Anglo-Saxon), omitted it, and softened the
remaining letters, \textit{fiele}, into \textit{viele}. The \textit{viele} is included in the
following description of minstrelsy from the \textit{Roman de Brut}, a
metrical chronicle of English history, by Wace, a poet who was in
great favour with our Henry II. Wace was born in Jersey, but educated
in Normandy. The passage is here given in two different dialects, the
one being sometimes a guide to the meaning of the other:—

\settowidth{\versewidth}{Aliquant demandent dez et tables.”}
\begin{dcverse}
\vleftofline{“}Mult ot à la cort jugléors,\\
Chantéors, estrumentéors;\\
Mult poïssiés oïr chançons,\\
Rotruanges et noviax sons.\\
\textit{Vieléures}, lais et notes,\\
Lais de \textit{vieles}, lais de rotes ;\\
Lais de harpe et de fretiax ;\\
Lyre, tympres et chalemiax,\\
Symphonies, psaltérions,\\
Monacordes, cymbes, chorons.\\
Asez i ot tresgitéors,\\
Joeresses et joéors;\\
Li un dient contes et fables,\\
Auquant demandent dez et tables.”\\
\vleftofline{“}Mult i aveit à la curt jugleurs,\\
Chantéurs, estrumenturs;\\
Mult puissez oïr chançons,\\
Rotuenges e novels sons.\\
Lais de \textit{vieles}, lais de rotes,\\
\textit{Vielers} lais de notes ;\\
Lais de harpe, lais de frestelles;\\
Lyres, cympes, chalemeles,\\
Symphonies, psalterions,\\
Monacordes, cymbes, corons.\\
Assez i out tregeteurs,\\
Joeresses et jugleurs;\\
Li un dient contes e fables,” \&c.\\
\vin\vin (MSS. \textit{Cotton Vitellius}, A. x.)
\end{dcverse}


The instrument which Savoyard boys play about the streets of
London (and here called \textit{hurdy-gurdy}), is now known in France by the
name of \textit{viele}. However, M. Fétis
\end{fixedpage}%762
\pagebreak
\renewcommand\rectoheadertext{the invention attributed to an englishman.}


\setlength{\fixedpagewidth}{360pt}
\begin{fixedpage}%763
\rectoheader

appeals to the proofs collected by Roquefort to establish the
fact that in the twelfth and thirteenth centuries \textit{vielle} signified an
instrument of the violin kind, and quotes a song by Colin Muset, a
minstrel of the thirteenth century, who tells of his going to a lady
in the meadow to sing to her with “vielle \textit{et l’archet}.”(\textit{Biographie
Universelle des Musiciens}, vi. 526.) The ancient hurdy-gurdy (in
Latin, \textit{organistrum}) had a handle to turn a wheel, but no bow. M.~Fétis 
also states that “the author of an anonymous treatise on
musical instruments, to which it appears impossible to affix a later
date than the thirteenth century,” attributes the invention of the
“viole à quatre cordes” to Albinus, and gives the tuning of the
instrument, as well as a very imperfect drawing. The tuning was by
fourths, the lowest note being the A below tenor C, then rising to D,
G, and treble C. The title of this treatise is “\textit{De diversis
monochordis, tetrachordis, pentachordis, exachordis, eptachordis,
octochordis, \&c., ex quibus diversa formantur instrumenta musicæ,
cum figuris instrumentorum}.” It is included in a manuscript
collection of works on music, in the library of the University of
Ghent, No. 171.

M. Fetis asks “who was this Albinus?” There can be very little
doubt that the Albinus, to whom the invention is attributed, was
Alcuin, who died in 804.\textsuperscript{a} He assumed the name of Flaccus Albinus in
his writings, it being the fashion at the Court of Charlemagne for
scholars to take literary names and surnames. Alcuin first met
Charlemagne at Parma on his return from Rome in 781, and finally
quitted England for the Court of Charlemagne in 792, taking with him
a number of other English ecclesiastics. Among his works was a
treatise on the liberal arts; but of this only Grammar, Rhetoric, and
the opening of Logic, are extant. The portion containing Music,
Arithmetic, Geometry, and Astronomy, is supposed to be lost. Albinus
composed most of his writings at Tours, and, when he founded the
monastic school there (which produced so many remarkable scholars in
the following age), he sent a mission to England to procure books for
its library. It was probably through his treatise, \textit{De Artibus
liberalibus}, that Albinus obtained the credit of the invention.

From \textit{viele}, the transition was easy to \textit{vielle}, \textit{viola}, \textit{viole}, and
\textit{viol}; but, hitherto, the use of these words has not been traced
abroad before the twelfth and thirteenth centuries. \textit{Violin} is a
diminutive of later date, probably not earlier than the sixteenth
century. Galilei, who wrote on ancient and modern music in 1582,
speaks of the \textit{viola da braccio}, of the \textit{viola da gamba}, and of the
\textit{violone} (viol for the arm, viol for the legs, and the great bass
viol), but does not mention the \textit{violino}. It could not, therefore,
have taken its proper rank in Italy at that time. He says the \textit{viola
da braccio} was called, “not many years before,” \textit{lira}. We had violins
(by name) on the English stage in 1561 (see the play of \textit{Gorboduc, or
Ferrex and Porrex}), and they were included in Queen Elizabeth’s band
in the same year. In 1561, the violin-players in the Royal band cost
£230 6\textit{s}. 8\textit{d}. (MSS. Lansdowne, No. 5), and in 1571, they received £325
15\textit{s}. (MSS. Cotton, Vesp., c. xiv.)

In Monteverde’s opera, \textit{Eurydice} (1607), where each character is
accompanied by different instruments, Hope sings to \textit{two violini
piccoli alia Francese}. This is the first use that has been traced to
the Italian \textit{stage}; nevertheless, the Italians soon became famous for
making the best instruments, not only from their skill as workmen,
but also from being favored by their climate in not requiring so much
glue, and in the facility for obtaining the best and dryest wood.

\begin{dcfootnote}
\textsuperscript{a} There was another Albinus (Albin, Abbot of Canterbury), 
who died in 732. He was also an Englishman, and 
Bede's principal assistant in his Ecclesiastical History: 
but although Bede styles him “ vir per omnia doctissimus,” 
we have no record of his having written upon
music, nor was he, probably, much known out of England,

\end{dcfootnote}
\end{fixedpage}%763
\pagebreak

\setlength{\fixedpagewidth}{360pt}
\begin{fixedpage}%764
\versoheader

Baltazarini, an Italian musician, and native of Piedmont, is said
to have been “le meilleur violon” of his time. He was taken to France
by Marshal de Brissac in 1577, and appointed director of music to
Catherine de Medecis. It is as difficult, however, to distinguish
between viola da braccio and violin in French history as in English;
because, at least during the sixteenth and seventeenth centuries,
instruments of all sizes were included under the name of violin.
Mersenne speaks of the French royal band of “twenty-four violons,”
although the instruments were of four sizes, just as Ben Jonson, or,
as in the time of Charles II., we called our royal band
“four-and-twenty fiddlers.”

In the \textit{Promptorium Parvulorum}, the date of which is about 1440,
“fydyll” and “fyyele” (viol) are Latinized “viella, fidicina,
vitula,” while “crowde, instrument of musyke,” is translated “chorus.”
In the sixteenth and seventeenth centuries, the names of crowd,
fiddle, and violin, were often indiscriminately applied to the same
instrument.

In Spain, the viol can have been but little known in the first
half of the sixteenth century, for Juan Bermudo, who published a
folio volume on musical instruments in 1555, does not include any one
in which the bow was employed. The Spanish \textit{vihuelas} or \textit{viguelas} were
guitars, distinguished from the \textit{guitarra} by having six strings
instead of four. Bermudo says, “no es otra cosa esta guitarra sino
una vihuela quitada la sexta y la prima cuerda.” The “gitterons which
are called \textit{Spanish vialles},” among the musical instruments of our
Henry VIII., were, no doubt, Spanish \textit{vihuelas}, in all respects the
same as the modern Spanish guitar.

The German word, \textit{bratsche}, for tenor, is evidently an
abbreviation of \textit{viola da braccio}.

I will not follow M. Fétis in his newly-adopted Eastern theory
for the bow. The only evidence he adduces is its \textit{present} use in the
East, and the primitive form of Eastern instruments; coupled with a
tradition among Buhddist priests, that one of the instruments to
which it is now applied was “invented by Ravana, King of Ceylon, five
thousand years before the Christian era.” This is a tolerably lengthy
“tradition.” I would ask, however, “how comes it that the bow was
unknown to the Greeks, Romans, and other nations? Did not Alexander
the Great conquer India and Persia? and were not those countries
better known to the ancients than to the moderns until within the
last three hundred years?” The Spaniards derived their instruments
from the Moors, but the bow was not among them. Once seen, it is an
easy thing to imitate, and the power of imitation is by no means
confined to the West.

The earliest drawing of the bow, now extant, is probably that
which was copied by Gerbert into his \textit{De Cantu}, ii. 138, plate 32, for
an instrument of the fiddle kind with one string. It is taken from
the same manuscript as the \textit{Cythara Anglica} (a well-formed harp with
twelve strings), which is beside it. To this manuscript M. Fétis
assigns the date of “commencement of the ninth century.” Gerbert places
it much earlier.

p. 7. \textsc{Chanson Roland}.—It is a curious coincidence that this tune
is exactly fitted to the Anglo-Norman romance, \textit{Chanson de Roland}, by
Turold, which, according to Mr. T. Wright (author of the \textit{Biographia
Britannica Literaria}), “was undoubtedly intended to be recited with
the accompaniment of the minstrel’s harp.” Dr. Crotch first printed
the air in the \textit{Appendix to the Specimens of various kinds of Music},—
therefore many years before this romance was published. Mr. Wright
dates the manuscript in the Bodleian Library, “as old as the latter
half of the twelfth century,”
\end{fixedpage}%764
\pagebreak
\renewcommand\rectoheadertext{hymns to popular tunes, etc.}

\setlength{\fixedpagewidth}{360pt}
\begin{fixedpage}%765
\rectoheader

and the language induces him to believe that the author
flourished in England about the time of King Stephen. The following
four lines form part of the narrative of the death of Roland, and
will serve as a specimen of the poem:--

\indentpattern{00006}
\settowidth{\versewidth}{Tient l’olifan que unques perdre ne volt.”\dots}
\begin{dcverse}
\begin{patverse}
\vleftofline{“}Co sent Rollans que s’espée li tolt, \\
Uverit les oilz, si li ad dit un mot: \\
\columnbreak
‘Men escientre! tu n’ies mie des noz.’\\
Tient l’olifan que unques perdre ne volt.”\dots\\
\textit{Biog. Brit. Lit}., ii. 122. 
\end{patverse}
\end{dcverse}



Dr. Crotch may have obtained the tune
from one of the musical manuscripts in the Bodleian Library, or from
Douce, the antiquary, who was possessed of some of very early date.
(I have only seen the musical manuscripts in the Music School at
Oxford.) As Dr. Crotch says, “\textit{probably} a French tune,” I suppose he
derived it from an English source.

p. 9. \textsc{The Norman Conquest}.—We may date the custom of singing
hymns to secular tunes from this time, if, indeed, it may not be
carried back to the time of St. Aldhelm. William of Malmsbury records
of Thomas, Archbishop of York (created in 1070), that “whenever he
heard any new secular song or ballad sung by the minstrels, he
immediately composed sacred parodies on the words, to be sung to the
same tune:

“Nec cantu nec voce minor, multa ecclesiastica composuit carmina:
si quis in auditu ejus arte joculatoria aliquid vocale sonaret,
statim illud in divinis laudes effigiabat.”

In a contribution to \textit{Notes and Queries} (ii. 385), Mr. James
Graves gives a curious list of eight songs similarly parodied, in \textit{The
Red Book of Ossory}, a manuscript of the fourteenth century, which is
preserved in the archives of that see. Six of the songs are English
(there are two parodies upon one of them), and the remaining two are
Anglo-Norman. The Latin hymns seem to have been written by Richard de
Ledrede, Bishop of Ossory from 1318 to 1360. The names of the six
English songs are as follows, the spelling being here modified:—

\begin{tabular}{ll}
1.&Alas! how should I syng, yloren is my playinge.\\
& \parbox{160pt}{How should I with that olde man,\par To leven and let my leman.}\mbox{{\Large\}}Sweetest of all, singe.}\\
2.&Have mercie on me, frere, barefoot that I go.\\
3.&Do, do, nightingale, syng ful mery.\\
&Shall I never for thine love longer kary.\\
4.&Have good day, my leman, \&c.\\
5.&Gaveth me no garland of greene,\\
&But it ben of wythones [withies—wyllowes?] yrought.\\
6.&Hey, how the chevaldoures woke all nyght.
\end{tabular}


p. 17, l. 2. \textsc{Pope Vitalian sent singers into Kent}.—This was to
secure conformity with Rome in the performance of the ritual
throughout the year, and was rendered necessary by the state of
musical notation at the time. The points, accents, hooks, and
up-and-down strokes written over the words (called \textit{neumæ}), being
without lines or spaces, were a very uncertain guide for any to learn
by, although they would serve to refresh the memory of those who had
received oral instruction.

p. 18. \textsc{Airs and graces of Church singers}.—A still more curious
description of Church singers at this period will be found in the
note commencing at p.~404. This was written by Ethelred or Ailred
about twenty years before the attack upon them by John of Salisbury,
but I did not discover the passage in time for insertion in the order
of chronology.

\end{fixedpage}%765
\pagebreak

\setlength{\fixedpagewidth}{360pt}
\begin{fixedpage}%766
\versoheader

p. 25. \textsc{Singing in ancient times}.—“In 1279, Roger de Mortimer held
jousts at Kenilworth, and set out from London to Kenilworth, with a
hundred knights well armed, and as many ladies going before, singing
joyful songs.” (Smith’s \textit{Lives of the Berkeley Family}, edited by
Fosbroke, 4to., 1821, p. 103.)

p. 31. \textsc{The Beverley Minstrels}.—In the Appendix to the Report of
the Commissioners on the public Records (Inns of Court, p. 388), the
Rev. Joseph Hunter makes the following remarks on “a manuscript on
vellum of the half folio form, containing several early English
Metrical Romances, in a hand of the fourteenth century,” in the
Lincoln’s Inn Library:—“On examining the covers attentively, I
discovered that there had been used in binding, a large piece of a
document relating to the Hospital of St. John at Beverley; and,
connecting this with the fact that at Beverley there was in the times
when this manuscript was written a noted fraternity of minstrels, a
probability is raised that the contents of this book were originally
translated for their use, and that the manuscript may, without much
hazard of misleading, be called hereafter \textit{The Book of the Minstrels
of Beverley}.” Mr. Hunter also refers to Lansdowne MSS., No. 896, for
memoranda respecting the Corpus Christi plays at Beverley (at fol.
157), and for “the orders of the ancient company or fraternity of
Minstrels at Beverley” (fol. 180). The orders are only of the year
1555, but they recite that it hath been a very ancient custom, out of
the memory of divers ages of men, that all, or the more part of the
minstrels serving any man or woman of honour or worship, or city or
town corporate, or otherwise, between the rivers of Trent and Tweed,
have been accustomed yearly to resort to this town and borough of
Beverley on Rogation days, and then to choose yearly one Alderman of
the Minstrels, with stewards and deputies authorized to take names
and to receive customable duties of the brethren of the said
minstrels’ fraternity. One of the orders issued in 1555, was that no
miller, shepherd, or husbandman playing upon pipe or other
instrument, should sue to perform at any wedding or merry-making out
of his own parish, as this would interfere with the privileges of the
corporation. The minstrels’ column is in St. Mary’s Church, Beverley,
but there are equally curious figures of musicians over the columns
of the Minster. All have been copied in vol. ii. of Carter’s
\textit{Specimens of Ancient Sculpture and Painting} (fol. 1786), and are
there accompanied with descriptions of the instruments by Douce.

p. 33. \textsc{Rote}.—When I gave hurdy-gurdy as the modern word for \textit{rote},
I hastily adopted the definition of Dr. Burney, who had “\textit{not the
least doubt} but that the instrument called a \textit{rote}, so frequently
mentioned by our Chaucer, as well as the old French poets, was the
same as the modern \textit{vielle}, and had its name from \textit{rota}, the wheel with
which its tones were produced.” (History, ii. 270, in note.) I am now
convinced that this is a mistake,—that the instrument had no wheel,
and therefore could not have been derived from the Latin \textit{rota}.

The rote was in use among the Anglo-Saxons, and in their language
“rót” and “rott” signified “cheerful—rejoicing.” There is no
mistaking the character of the instrument after the description given
of it by Notker in the tenth century, and that description agrees
with all others that I have found, down to the times of Chaucer and
Spenser.

The first notice of the rote is in the correspondence of two
Englishmen in the eighth century. We can fix the date of the letter
within twenty years, as it was
\end{fixedpage}%766
\pagebreak
\renewcommand\rectoheadertext{beverley minstrels, rote.}

\setlength{\fixedpagewidth}{360pt}
\begin{fixedpage}%767
\rectoheader

written after the death of Bede, 735, and before the death of
Boniface, the apostle of Germany, 755. Lull, who was created a bishop
by Boniface, had written from Germany to England for some of the
minor works of Bede, and the letter is in answer to that request.
Abbot Gutberct sends him Bede’s life of St. Cuthbert in prose and
verse, and excuses himself for not sending more because the winter
had been so severe in cold, ice, and tempests of wind and rain, that
the hands of the young men were retarded. He asks Lull if it will be
possible to send two young men to the monastery, one who can make
glass vessels well, and the other skilful in playing the rote. He
reminds him that he had sent to him a present of a gown made of
otter-skins, and twenty knives for the fraternity abroad, six years
before, and that the receipt of them had not been acknowledged. He
hopes his present request will not be treated as trivial. The passage
of the letter in which the rote is mentioned is as follows:—“Truly it
delights me (meaning “it \textit{will} delight me”) to have a harper who can
play upon that kind of harp which we call rotta, because I have an
instrument, but no one skilful in the craft.”\textsuperscript{a} The abbot here added
the Anglo-Saxon name, because the Latin word \textit{cithara} answered for
several instruments, and would not define which of them he required.

Among the musical instruments which are copied from a manuscript
of the ninth century, formerly in the monastery of St. Blaise,
representations will be found of the “Cythara Anglica,” and of the 
“Cythara Teutonica.”(Gerbert’s \textit{De Cantu}, ii., tab. 32.) The latter
agrees completely with the descriptions given of the rote, and we
find the same instrument depicted in Anglo-Saxon psalters in and
after the eighth century. The following is a representation of King
David playing upon one of these, from an Anglo-Saxon illumination of
the eighth century (Vesp., A. i., Cotton MSS.):

\noindent\includegraphics*[width=\textwidth]{images/cythara.pdf}

\begin{dcfootnote}
\textsuperscript{a} This letter is printed among the \textit{Epistolæ S. Bonifacii Martyris,
Primi Moguntini Archiepiscopi, Germanorum Apostoli}, per
Nicolaum Serarium, 4to. Moguntiæ, 1629. Epist. 89, p. 123-4.
“Delectat me quoque cythariatam habere, qui possit cytharisare in
cithara quam noa appellamus \textit{rottæ}, quia citharam habeo,
et artificem non habeo.”
(Here he writes \textit{rottæ} instead of \textit{rotta}, and so in the following
pasaage.) “Si grave not sit, et istum quoque meæ dispositione
mitte.” The abbot says he had lived forty-three years in the
monastery, but evidently he had forgotten his Latin grammar.
\end{dcfootnote}
\end{fixedpage}%767
\pagebreak

\setlength{\fixedpagewidth}{360pt}
\begin{fixedpage}%768
\versoheader

Notker, who wrote a tract on Church Music in one of the Teutonic
dialects, towards the close of the ninth century, says that the rote,
like the lyre, had seven strings for the seven notes of the
scale,—“àndero lîrûn únde àndero rótûn iô siben síeten.”
(Gerbert’s \textit{Scriptores}, i. 96.) Another Notker, a monk of St. Gall,
who wrote in Latin about a century later, says that the ancient
\textit{psalterium} was in the form of the Greek ∆, and that it had ten
strings. He considered this triangular shape as emblematic of the
Trinity, and complains that, after the instrument had been adopted by
singers and players (ludicratores) for their uses, they added to the
number of strings, altered the form to suit their convenience, and,
giving it the barbarous name of \textit{rotta}, destroyed its mystical
signification.\textsuperscript{a}

In Chaucer’s description of his mendicant friar, he says: “Wel
couthe he synge and playe on a rote;” and, although many lines
intervene, yet, when he adds “And in his \textit{harpyng}, whan that he had
\textit{sung},” it is a continuation of the portrait, and no other instrument
has been named.

The resemblance of the rote to the ancient lyre will account for
Spenser’s having applied the term of “Phœbus’s rote” to the lyre of
Phœbus in the \textit{Fairy Queen}.

Finally, there is no old authority for giving the Latin name of
\textit{rota} to the \textit{hurdy-gurdy}. Mersenne and Kircher style it \textit{Lyra
Mendicorum}, and in the manuscript “of the ninth century” quoted by
Gerbert, it is entitled \textit{Organistrum}.

p. 35, note b. \textsc{Shawm}.—The different descriptions of the shawm may
be reconciled by the fact of their having been made of various sizes.

p. 37, l. 28. \textsc{New College, Oxford}.—The words of the Statute are
“Post tempus prandii aut cenæ liceat gratia recreationis in aula in
cantilenis et aliis solatiis honestis moram facere condecentem,” \&c.
This does not prove the singing of part-music.

p. 52, l. 10. \textsc{Ophelia’s Song}.—The burden, “you must sing down,
adown, an you call him adown-a,” will be found almost verbatim in a
ballad commencing—

\settowidth{\versewidth}{When as King Edgar did govern this land,}
\begin{dcverse}
\begin{altverse}
\vleftofline{“}When as King Edgar did govern this land, \\
\textit{Adown, adown, down, down, down,} \\
\columnbreak
And in the strength of his years he did stand, \\
\textit{Call him down-a}.”
\end{altverse}
\end{dcverse}

See Evans’s \textit{Old Ballads}, ii. 22, 1810, or \textit{Old Ballads}, ii. 28,
1727.

p. 66. \textsc{As I walked the woods so wild}.—This is parodied in Andro
Hart’s \textit{Compendium}: “I am woe for their wolves so wylde.”

p. 76. \textsc{Who’s the fool now}?—Archie Armstrong, Charles the First’s
jester, quoted this song when he tauntingly asked Archbishop Laud
“Who’s the fool now?’’ after the stool had been thrown at the dean’s
head, for reading the English liturgy in Edinburgh. It is also quoted
by Dryden, in his play of Sir Martin Mar-all.

p. 77. \textsc{Bransle, or Braule}.—The following description of this
dance is from the \textit{Dictionnaire de Danse} [par Ch. Compan], Paris,
8vo., 1787:—“Branle est une danse par ou commencent tous les Bale, ou
plusieurs personnes dansent en Rond en se

\begin{dcfootnote}
\textsuperscript{a} " Sciendum est quod antiquum psalterium, instrumentum 
decachordum, utique erat, in hac videlicet deltæ
litterae figura, multipliciter mystica. Sed post quam illud,
symphoniaci quidam et ludicratores, ut quidam ait, ad
suum opus traxerant, formam utique ejus et figuram
commoditate suæ habilem fecerunt, et plures chordas
annectentes et nomine barbarico \textit{rottam} appellantes,
mysticam illam Trinitatis formam transmutaverunt.”
\end{dcfootnote}
\end{fixedpage}%768
\pagebreak
\renewcommand\rectoheadertext{bransle, trenchmore, lusty gallant, etc.}

\setlength{\fixedpagewidth}{360pt}
\begin{fixedpage}%769
\rectoheader

tenant par la main et se donnant un \textit{branle} continuel et concerté,
avee des pas convenables, selon la difference des air qu’on joue
alors. Les Branles consistent en trois pas et un pied-joint qui se
font en quatre mesures, ou coups d’archet, qu’on disoit autrefois
\textit{battement de tambourin}. Quand ils sont repétés deux fois, ce sont des
\textit{Branles doubles}; au commencement on danse des \textit{Branles simples}, et
puis le \textit{Branle gui}, par deux mesures ternaires, et il est ainsi
appellé parce qu’on a toujours un pied en l’air.” Thoinot Arbeau
gives “\textit{Les Branles du Poictu}, qui se dansent par mesure ternaire, en
allant toujours à gauche,” also “\textit{Branles d'Ecosse et de Bretagne}: on
appelle ceuxci le Triory.” He also tells us that “Les \textit{danses aux
chansons} sont des espèces de Branles.”

Here we have it clearly laid down that the \textit{Bransle de Poictu}, or
\textit{Bransle double}, is in triple time, and so by Morley, in his
\textit{Introduction}, 1597 and 1611; therefore, the name of \textit{Bransle de Poictu}
is improperly given to “We be three poor Mariners,” in the Skene
Manuscript, unless it be in the sense of “une danse a chanson.”

p. 83.\textsc{ Trenchmore}.—This is mentioned in Holinshed’s Description
of Ireland, c. 2: “And trulie they suit a Divine as well as for an
ass to twang \textit{Quipassa} on a harpe or gitterne, or for an ape to friske
\textit{Trenchmoore} in a pair of buskins and a doublet.” In \textit{Pills to purge
Melancholy}, i. 51, 1699, the song, “Willy, prithee go to bed, for
thou wilt have a drowsy head,” is to a version of Trenchmore.

p. 87. \textsc{Quoth John to Joan}.—The version of the words printed with
the tune is by D'Urfey. See his \textit{New Collection of Songs and Poems},
8vo., 1683, p. 48. The old ballad of “John wooinge of Jone” was
entered at Stationers’ Hall in January, 1591-2.

Puttenham, in his \textit{Art of English Poesie}, quotes a song “in our
interlude called \textit{The Wooer}, where the country clown came and wooed a
young maid of the city, and being agrieved to come so oft and not
have his answer, said to the old nurse very impatiently:

\settowidth{\versewidth}{‘Iche pray you, good mother, tell our}
\begin{scverse}
\vleftofline{\textit{Wooer}. }‘Iche pray you, good mother, tell our young dame,\\
Whence I am come, and what is my name;\\
I cannot come a-wooing every day.\\
\vleftofline{(\textit{Quoth the Nurse}.) }They be \textit{lubbers}, not \textit{lovers}, that so use to say.’”
\end{scverse}

The copy of “I cannot come every day to woo” in the Pepys
Collection (iii. 134) consists of fourteen stanzas.

p. 91. \textsc{Lusty Gallant}.—There is a “proper dittie” to this tune in
the \textit{Gorgious Gallery of Gallant Inventions}, 1578, and many more
ballads were sung to it than I have space to enumerate. Holinshed, in
his \textit{Chronicles of England}, i. 290, speaks of lusty gallant as a newly
devised colour: “I might here name a sort of hues devised for the
nonce, wherewith to please fantastical heads.” Among these are
“pease-porridge-tawney, popinjay-blue, lusty-gallant, the devil in
the hedge, and such like."

p. 102. \textsc{The Lute}.—There are several other derivations proposed
for the word, “lute.” Gerbert says from \textit{la ut}, and considers the name
to have been given to signify its extended compass. M. Fétis, who
looks only to the East, derives it from \textit{eoud}, an instrument now in
use among the Arabs.

\end{fixedpage}%769
\pagebreak

\setlength{\fixedpagewidth}{360pt}
\begin{fixedpage}%770
\versoheader

p. 107. \textsc{Deloney’s Ballads}.—I have not seen a copy of “John for
the King,” but it was entered at Stationers’ Hall on 24th Oct., 1603,
as “A newe ballet, called \textit{John for the King}, to the tune of \textit{Hey downe
derrye}.” Deloney’s “Repent, England, repent,” is perhaps “England’s
new Bellman,” a copy of which is in the Roxburghe Collection, iii.
222. It begins, “Awake, awake, O England,” and the burden is “Repent,
therefore, O England, the day it draweth near,” \&c. It may also be
the ballad of “The great Earthquake,” a copy of which is contained in
a manuscript of the time of James I. in Mr. Payne Collier's
possession. The former is to the tune of \textit{O man in desperation},—the
latter is in the same metre as the ballad on the burning of St.
Paul’s steeple (quoted at p. 117), and, in all probability, to the
same tune. It commences—

\indentpattern{0101330}
\settowidth{\versewidth}{Take warning, London, and beware, }
\begin{dcverse}
\begin{patverse}
\vleftofline{“}Take warning, London, and beware, \\
By what you late have seen: \\
O let it fill your minds with care,— \\
The Earthquake I do mean.\\
It is a sign\\
Of wrath divine,—\\
A warning to all subjects of the Queen.”
\end{patverse}
\end{dcverse}

There are twelve stanzas, and the eleventh begins thus:—

\settowidth{\versewidth}{Again I say repent, repent,}
\begin{scverse}
\begin{altverse}
\vleftofline{“}Again I say repent, repent,\\
Repent, O England, now.”
\end{altverse}
\end{scverse}

p. 113. \textsc{Row well, ye Mariners}.—The tune is also in \textit{Pills to purge
Melancholy}, ii. 195, 1707, to a song called “John and Joan.”

p. 114. \textsc{Lord Willoughby}.—Perhaps the name of Rowland, given to
this tune in Queen Elizabeth’s Virginal Book, is derived from a
ballad commencing “Now welcome, neighbour Rowland,” which is in the
same metre as that of “Lord Willoughby.” A copy of this Rowland is in
the Pepys Collection, i. 210, “printed for J. Trundle.” It is
entitled “News, good and new! to the tune of £20 a yeere.”

p. 117. I am the Duke of Norfolk, or Paul’s Steeple.—I have
omitted one very popular ballad which was sung to this tune. Many
half-sheet copies of it were printed, with the music, during the last
century, and it is still remembered. It commences: 

\settowidth{\versewidth}{I have little more to say than, will you? aye or nay?}
\begin{scverse}
\begin{altverse}
\vleftofline{“}There was a little man, and he woo'd a little maid,\\
And he said “Little maid, will you wed, wed, wed?\\
I have little more to say than, will you? aye or nay?\\
For little said is soonest mended-ed.”
\end{altverse}
\end{scverse}

There cannot, I think, be a doubt that the Irish \textit{Cruiskeen Lawn},
and the Scotch \textit{John Anderson, my Jo}, are mere modifications of this
very old English tune. I have already said that our Country Dances
travelled not only over Scotland and Ireland, but over all Europe;
and this tune has remained in constant and popular use from the early
part of the reign of Queen Elizabeth down to the present time. \textit{John
Anderson, my Jo}, is first found in the Skene Manuscript, and if
anyone should wish to be assured of the identity of the two airs, he
has only to look to that copy, printed in Dauney’s \textit{Ancient Scottish
Melodies}, p. 219. Dauney greatly exaggerates the age of the Skene
Manuscript when he dates it in the time of our James I., for it
includes an English Country Dance that first appeared in 1698, and
the writing alone would sufficiently disprove his idea of its
antiquity. Stenhouse asserts that the words of \textit{John Anderson, my Jo},
are preserved in Bishop Percy’s old manuscript, written as early, if
not before the year 1560. Here the date of the manuscript, and its
containing \textit{John}


\end{fixedpage}%770
\pagebreak

\renewcommand\rectoheadertext{packington’s pound, dulcina, etc.}

\setlength{\fixedpagewidth}{360pt}
\begin{fixedpage}%771
\rectoheader

\textit{Anderson, my Jo}, are Stenhonse’s inventions. The “tradition” of
the words bearing reference to the seven sacraments of the Church of
Rome has already been sufficiently refuted. In the first edition of
Percy’s Reliques, the number of “bairns” was five, and the subsequent
alteration to seven was “a new reading communicated by a friend, who
\textit{thinks} by the seven bairns are meant the Seven Sacraments.” The words
of \textit{John Anderson} printed by Percy seem uncouth to an Englishman, on
account of the use of “z” for “y,” but that is no proof of antiquity,
for the Scotch still employ the one for the other.

When Moore appropriated the air under the name of \textit{Cruiskeen Lawn},
he was under the misconception that the terminations were peculiarly
characteristic of Irish music. For the same reason, and with equal
impropriety, he included “The pretty girl of Derby, O” among his
Irish Melodies. I will here only remark that Bunting (a far higher
authority for Irish music) rejects both these airs, and refer the
reader for further remarks to “Characteristics of English National
Airs,” where I shall endeavour to invert Moore’s position.

p. 123. \textsc{Packington’s Pound}.—The ballad which Shakespeare is said
to have written on Sir Thomas Lucy was evidently intended for the
tune of \textit{Packington’s Pound}. See Dyce’s \textit{Shakespeare}, vol. i., p. xxii.

Instances of the use of the tune at later dates than any I have
cited, will be found among the jingles of Sir Charles Hanbury
Williams, such as his election-squib upon Bubb Doddington, “A grub
upon Bubb,” beginning:—

\settowidth{\versewidth}{“When the Knights of the Bath by King George were created;” }
\begin{scverse}
“When the Knights of the Bath by King George were created;” 
\end{scverse}
also
in \textit{The Convivial Songster}, 1782. It is there printed to a song
commencing,— 

\settowidth{\versewidth}{“Ye maidens and wives, and young widows, rejoice.”}
\begin{scverse}
“Ye maidens and wives, and young widows, rejoice.”
\end{scverse}

p. 112. \textsc{Dulcina}.—The tune may perhaps be carried a stage further
back under another name. In the registers of the Stationers’ Company,
under the date of May 22, 1615, there is an entry transferring the
right of publication from one printer to another, and it is described
as “A Ballett of Dulcina, to the tune of \textit{Forgoe me none, come to me
soone}.

p. 147. \textsc{John, come kiss me now}.—This tune is also included in
\textit{Musicks Delight on the Cithren}, 1666.

Sir W. Davenant alludes to it in his play of \textit{Love and Honour}; J.
Phillips (Milton’s nephew), in his translation of Don Quixote, 1687,
p. 278; and the Hon. Roger North, in his \textit{Memoires of Musick}. North
says, prophetically, “The time may come when some of the present
celebrated musick will be as much in contempt as \textit{John, come kiss me
now, now, now}, and perhaps with as much reason as any is found to the
contrary at present.” (4to., 1846, p. 92.)

When I said that the tune of \textit{John, come kiss me}, had not
“hitherto been found in any old Scotch copy,” I should have excepted
a manuscript of music for the base viol, which was in the possession
of the late Andrew Blaikie. This has been commonly quoted as of the
year 1692, but I date it as not earlier than 1745. Blaikie was
possessed of two such manuscripts, and, according to Dauney’s \textit{Ancient
Scottish Melodies}, p. 143, they “were both written in the same hand,
and their respective contents arranged nearly in the same order.” He
says the one was dated 1683, and the other 1692. Blaikie very
obligingly lent me one manuscript (having
\end{fixedpage}%771
\pagebreak

\setlength{\fixedpagewidth}{360pt}
\begin{fixedpage}%772
then lost the other), and I translated a considerable portion of
its contents. Within the cover was written, “Lady Katharine Boyd
aught this book,” but I did not observe a date of any kind, There may
be “1692” somewhere within it, but that can have no reference to the
time of the writing. “Alas! that I came o’er the moor,” appears there
under the name of Ramsay’s song, “\textit{The last time} I came over the
moor,” and that alone would disprove the date of 1692. The manuscript
contains a hundred and twelve tunes, of which the tenth is “New
Hilland Laddie,” and the fifteenth, “God save the King.”

p. 153. \textsc{The British Grenadiers}.—The \textit{words} of this song cannot be
older than 1678, when the “Grenadier Company” was first formed, or
later than the reign of Queen Anne, when grenadiers ceased to carry
hand-grenades.

p. 157. \textsc{Pavan}.—“Instrumental players play the Pavan faster,” says
Thoinot Arbeau, “and call it the Passamezzo”—\textit{Anglice}, the Passing
Measures’ Pavan.

Puttenham says, “Songs, for secret recreation and pastime in
chambers, with company or alone, were the ordinary musickes amorous;
such as might be sung with voice, or to the lute, citheron, or harpe;
or daunced by measures—as the \textit{Italian} pavan and galliard are at these
daies in Princes’ courts, and other places of honourable or civil
assembly.” (\textit{Art of Poesie}, p. 37, reprint.) \textit{Pavana}, according to
Italian writers, was derived from \textit{Paduana},—and not from \textit{Pavo}\textit{Pavo}, a
peacock, as I have stated, taking Hawkins for my authority.

p. 164. \textsc{Death and the Lady}.—This ballad is mentioned by Oliver
Goldsmith in his fourth essay:—“Every man had his song, and he saw no
reason why he should not be heard as well as any of the rest: one
begged to be heard while he gave Death and the Lady in high taste,”
\&c.

p. 171. \textsc{The Gipsies’ Round}.—Perhaps the words of this round are
in Middleton’s play, \textit{The Spanish Gipsy}. (Dyce’s \textit{Middleton}, iv.
141.) They suit the tune:—

\settowidth{\versewidth}{Shew tricks and lofty capers,” \&c.}
\begin{scverse}
\vleftofline{“}Trip it, trip it, gipsies fine,\\
Shew tricks and lofty capers,” \&c.
\end{scverse}

p. 171. \textsc{Guy of Warwick}.—Old Puttenham says, in his \textit{Art of Poesie},
“And we ourselves, who composed this treatise, have written for
pleasure a little brief Romance or Historical Ditty, in the English
tong, of the Isle of Great Britain; in short and long metres, and by
breaches or divisions, \textit{to be more commodiously sung to the harpe} in
places of assembly where the company shall be desirous to hear of old
adventures and valiaunces of noble knights in times past—as are those
of King Arthur and his knights of the round table,—Sir Bevis, of
Southampton—\textit{Guy of Warwicke}—and others like.” (Haslewood’s reprint,
p. 33-4.)

p. 173. \textsc{Loth to depart}.—These words (by Dr. Donne) were also set
by Orlando Gibbons. In his copy, they commence differently:—“Ah! dear
heart, why do you rise?”

p. 178. \textsc{Crimson Velvet}.—The ballad of \textit{Constance of Cleveland},
here printed with the tune, was entered at Stationers’ Hall, on the
11th of June, 1603, to William White, as “Of the fayre Lady Constance
of Cleveland and of her disloyall Knight together with eight other
ballads.
\end{fixedpage}%772
\pagebreak

\renewcommand\rectoheadertext{phillida flouts me, the spanish lady, etc.}

\setlength{\fixedpagewidth}{360pt}
\begin{fixedpage}%773
\rectoheader

Martin Parker’s ballad of the siege of Rochelle is included in
the Pepys Collection,
i. 96, as “Rochell her yielding to the obedience of the French
King, on the 28 October, 1628, after a long siege by land and sea, in
great penury and want. To the tune of \textit{In the days of old}.” It is
subscribed “M. Parker,” was “printed at London for
J. Wright,” and begins “You that true Christians be.” “In the
days of old,” from which the tune here derives its name, has already
been quoted. In Forbes’s \textit{Cantus}, 1682, and in the Straloch MS., the
same air is entitled “Shepherd, saw thou not my fair, lovely Phillis
?”

p. 183. \textsc{Phillida flouts me}.—The copy of this ballad in \textit{Wit
restored}, 1658, is older, and in many respects preferable to the
version I have printed, which agrees with the one in Ritson’s \textit{Ancient
Songs}.

p. 186. \textsc{The Spanish Lady}.—This ballad is quoted in Mrs. Behn’s
comedy, \textit{The Rovers, or The banished Cavaliers}, and in Richard Brome’s
\textit{Northern Lasse}.

p. 187. \textsc{The Jovial Tinker}.—The song in \textit{Robin Goodfellow}, “to the
tune of \textit{The Jovial Tinker},” was, no doubt, intended to be sung to \textit{Tom
a Bedlam}. See further remarks upon that tune, p. 779.

p. 191. \textsc{The Oxfordshire Tragedy}.—The ballad of “The Miller’s
advice to his three Sons on the taking of toll” is still sung to this
tune in the North of England. A copy in the Roxburghe Collection,
iii. 681, commences thus:—

\settowidth{\versewidth}{There was a miller who had three sons,}
\begin{scverse}
\vleftofline{“}There was a miller who had three sons,\\
And, knowing his life was almost run,\\
He call’d them all, and ask’d their will,\\
If that to them he left his mill.”
\end{scverse}

The miller reproves the eldest son, and the second also, for not
intending to take toll enough, but the youngest wins his heart by
saying:—

\begin{scverse}
\vleftofline{“}Father, you know I am your boy,\\
And in taking toll is all my joy:\\
Rather than I’d good living lack,\\
I’d take the whole and forswear the sack.”
\end{scverse}

p. 196. \textsc{Up tails all}.—This seems to have been Herrick’s favorite
tune, for he not only wrote a song under the name, but also five more
in the peculiar metre. Of these the first is “Ceremonies for
Christmas:”—

\settowidth{\versewidth}{And drink to your hearts’ desiring,” \&c.}
\begin{scverse}
\indentpattern{001001}
\begin{patverse}
\vleftofline{“}Come, bring with a noise,\\
My merry, merry boyes,\\
The Christmas log to the firing;\\
While my good dame, she \\
Bids ye all be free,\\
And drink to your hearts’ desiring,” \&c.
\end{patverse}
\end{scverse}

The second is “The hag is astride;” the third, “The Maypole is
up;” the fourth, “The Peter-penny;” and the fifth, “Twelfth Night, or
King and Queen.” (See \textit{Hesperides}, vol. ii., 1825.)

p. 199. \textsc{Chevy Chace}.—The celebrated John Locke, when secretary to
the embassy sent by Charles II. to the Elector of Brandenburg, wrote
home a description of the Brandenburg Church singing: “He that could
not, though he had a cold, make
\end{fixedpage}%773
\pagebreak

\setlength{\fixedpagewidth}{360pt}
\begin{fixedpage}%774
\versoheader

better music with a Chevy Chace over a pot of smooth ale,
deserved well to pay the reckoning, and to go away athirst.” (\textit{Life of
Locke}, by Lord King.)

p. 204. \textsc{It was a lover and his lass}.—This music was composed by
Morley, and is included in “The first booke of Ayres or Little Short
Songs to sing and play to the Lute, with the Base Viole, newly
published by Thomas Morley, Bacheler of Musicke, and one of the
gentlemen of Her Majestie’s Royal Chappell.” This collection was
“imprinted at London” by William Barley, in 1600, and dedicated to
Ralph Bosvile, Esq. (folio). An imperfect copy of this now rare book
was a few years ago in the possession of Rodd, the bookseller. Mr.
Oliphant had then the opportunity of transcribing the music of this
song, and to him I am indebted for the information, and for a copy.

p. 206. \textsc{O willow, willow}.—The music is included in Thomas
Dallis’s MS. Lute-book, under the name of “All a greane willowe.”
Dallis was a music-teacher at Cambridge, and his book, which bears
the date of 1583, is now in the library of Trinity College, Dublin.
(D. iii., 30.)

“Shall Camillo then sing \textit{Willow, willow, willow}?” says Middleton,
in his \textit{Blurt Master Constable}. The ballad was quite proverbial, and
parodied as late as 1686, when, in Book ii. of \textit{Playford’s Pleasant
Musical Companion}, we find “A poor soul sate sighing near a
gingerbread stall,” \&c.

p. 208. \textsc{Whoop! do me no harm}.—The tune was arranged with
variations by W. Corkine, and printed in \textit{Lessons for the Lyra-Viol},
\&c., 1610. In the \textit{Famous History of Friar Bacon}, there is a ballad
to the tune of “\textit{O} do me no harme, good man.” In the Pepys Collection,
i. 152, is “The golden age, or an age of plain dealing: to a pleasant
new court tune, or \textit{Whoope, doe me no harme, good man};” and at p. 156,
“The honest age,” \&c., “to the tune of The golden age” At p. 384,
“The wiving age, to the tune of \textit{The golden age}.” At p. 400, “The
Cooper of Norfolk, to the tune of \textit{The wiving age}.” At p. 248, “A merry
ballad of a rich maid that had eighteen severall suitors of severall
countries: otherwise called The scornefull Maid. To the tune of \textit{Hoop,
doe me no harme, good man}.” These ballads were printed by J[ohn]
T[rundle] or Henry Gosson.

In the second part of \textit{Westminster Drollery}, 1672, is a ballad “Of
Johnny and Jinny,” which seems to have been intended for the tune. It
commences:—

\settowidth{\versewidth}{He tun’d his quill, and sung to her still,}
\begin{scverse}
\begin{altverse}
\vleftofline{“}The sweet pretty Jinny sate on a hill,\\
Where Johnny the swain her see,\\
He tun’d his quill, and sung to her still,\\
\textit{Whoop, Jinny, come down to me}.”
\end{altverse}
\end{scverse}

p. 213. \textsc{Song on the Spanish Armada}.—This is also contained in “A
Banquet of Jests new and old,” by Archie, the King's Jester, 8vo.,
Lond., 1657, and entitled “An old song on the Spanish Armado in ’88.”
It varies but slightly from the copy in \textit{Westminster Drollery}.

p. 219. \textsc{London is a fine town}.—Other versions of this ballad are
in Ashmole’s MSS. 36 and 37, p. 318, and in Mr. Payne Collier’s MS.,
time of James I.

In the Pepys Collection, i. 406, is “The Cuckowe's Comendation,”
\&c., “a merry Maying song, in praise of the cuckow: To the tune of
\textit{The buttoned smock},” beginning
\end{fixedpage}%774
\pagebreak

\renewcommand\rectoheadertext{jew’s trump, ditties, etc.}

\setlength{\fixedpagewidth}{360pt}
\begin{fixedpage}%775
\rectoheader

“Of all the birds that haunt the woods.” The ballad of “The bonny
Lass, or the buttoned smock,” is printed in \textit{Pills to purge
Melancholy}, vi. 144, to the tune of \textit{O London is a fine town};
therefore, \textit{The buttoned smock} seems to be another name for the tune.

p. 222. \textsc{Jew’s Trump}.—I should have added that this is the old
name for what we now call the \textit{Jew’s harp}.

p. 222, note a.—\textsc{Ditties} are not only the phrases of melody that
recur at the end of every stanza, as in the passage quoted from
Rowley’s \textit{Match at Midnight}, but often signify the whole tune. So in
P. Fletcher’s \textit{Purple Island}, c. 1:—

\settowidth{\versewidth}{Which bears the under-song unto your cheerful dittying.”}
\begin{scverse}
\vleftofline{“}But you, O Muses! by soft Chamus sitting,\\
Your dainty songs unto his murmurs fitting,\\
Which bears the under-song unto your cheerful dittying.”
\end{scverse}

p. 229. \textsc{Green Sleeves}.—Perhaps \textit{Triumph and Joy} was another early
name for this popular air. The ballad of Queen Elizabeth at Tilbury
is in stanzas of twelve lines (like \textit{York, York for my money}), and was
sung to the tune of \textit{Triumph and Joy} (Collier’s \textit{Old Ballads}, p. 110);
but others to the same air have but eight. See, for instance:—

\settowidth{\versewidth}{And hath been since the world began.”}
\begin{scverse}
“Mas Mault he is a gentleman,\\
And hath been since the world began.”
\end{scverse}
(Rox., i. 342; Pepys, i. 427; or Evans’s \textit{Old Ballads}, iv. 220.)

\textit{York for my money} is mentioned in Richard Brome’s comedy, \textit{The
Northern Lasse}, where the widow says “You said she sung and spoke it
northernly—I have a great many southern songs already, but northern
ayres nip it dead—\textit{Yorke, Yorke for my money}’’

p. 232. \textsc{Green Sleeves and Pudding Pies}.—This is one of the songs
enumerated in “Sportive Wit: The Muses Merriment,—a new spring of
lusty drollery, jovial fancies, and à la mode lamponnes,” 8vo., 1656.
All there quoted is:—

\settowidth{\versewidth}{Green sleeves and pudding pies,}
\begin{scverse}
\vleftofline{“}Green sleeves and pudding pies,\\
And wot you not where--”
\end{scverse}
But this does not agree, even so far, with the version in
Boswell’s \textit{Journal}, and the date of 1656 proves that the song could
have no reference to Jacobitism.

p. 237. \textsc{O Death! rock me}.—Another song about Anne Boleyn and
Henry VIII.
will be found in Harl. MSS., No. 2252, fol. 155. It is “A Ditty
setting forth the inconstancy of Fortune,” and begins:—

\settowidth{\versewidth}{“In a freshe morninge, among the flowrys.”}
\begin{scverse}
“In a freshe morninge, among the flowrys.”
\end{scverse}

It is reprinted in Collett’s \textit{Relics of Literature}. In the same
manuscript is a ballad on the battle of Flodden Field (fol. 43, b.),
called “The Lamentacione of the Kynge of Scotts,” beginning:—

\settowidth{\versewidth}{“As y lay musinge, myself alone;”}
\begin{scverse}
“As y lay musinge, myself alone;”
\end{scverse}
 and Article 156 is on Cardinal
Wolsey.

p. 240. \textsc{Bara Faustus’s Dream}.—In the \textit{Golden Garland of Princely
Delights}, 3rd edit., 1620, the song of “Come, sweet love, let sorrow
cease,” is entitled “The Shepheard’s Joy: to the tune of \textit{Bara
Faustus’s Dream}.”

Another name for the tune is “Phoebus is long over the sea.” It
is found under that title in \textit{Nederlandtsche Gedenck-Clanck}, 1626; in
\textit{Friesche Lust-Hof}, 1634; and
\end{fixedpage}%775
\pagebreak

\setlength{\fixedpagewidth}{360pt}
\begin{fixedpage}%776
\versoheader

in Dr. Camphuysen’s \textit{Stichtelycke Rymen}, 1647; sometimes with the
addition of \textit{Barra Faustus's Dream}.

p. 241. \textsc{Spanish Pavan}.—The “Engelsche indraeyende Dans
Londesteyn” (the turning dance of London) in \textit{Friesche Lust-Hof}, 1634,
is another version of this tune. The two first bars are identical. \textit{I
love my love for love again}, in the Skene MS., is the same, after the
first eight bars.

p. 244, note a.—\textsc{The Bandora} is proved to have been strung with
wire by a passage in Heywood’s \textit{Fair Maid of the Exchange}, where he
compares a lady's hair to “bandora \textit{wires}.”

p. 256. \textsc{The Hunter in his Career}.—Among the ballads to this tune
under the title of \textit{Basse's Career}, are “Hubert’s Ghost” (Bagford
Coll., 643, m. 10, fol. 49); “The hasty Bridegroom” (Rox., ii. 208);
and “Wit’s never good till ’tis bought” (Collier’s \textit{Roxburghe Ballads},
p. 264).

p. 260, note a. \textsc{Johnny Armstrong}.—Another English ballad about
this hero is entitled “Johnny Armstrong’s last Good-night; shewing
how John Armstrong with his eight-score men fought a bloody battle
with the Scotch King, at Edenborougb. \textit{To a pretty Northern Tune}:”
commencing:—

\settowidth{\versewidth}{From the highest estate to the lowest degree,}
\begin{scverse}
\begin{altverse}
\vleftofline{“}Is there ever a man in all Scotland,\\
From the highest estate to the lowest degree,\\
That can shew himself before our King? \\
Scotland is so full of treachery,” \&c.
\end{altverse}
\end{scverse}

A copy in the Bagford Collection (643, m. 10, p. 94), printed by
and for W. O[nley]; also in \textit{Old Ballads}, 1727, i. 170, and in Evans’s
\textit{Old Ballads}, iii. 101, 1810. The tune is referred to in the Roxburghe
Coll., ii. 499, where “The West-country Damosel’s Complaint” is to be
sung to it. Evans prints a third ballad (commencing “As it fell out
one Whitsunday”), under the impression that it relates to the same
person, but he is there transformed into \textit{Sir} John Armstrong, and
competes with Sir Michael Musgrave, a Scotch knight, for the daughter
of Lady Dacres. After winning her, Armstrong is killed by his rival.
See Rox., ii. 261, or Evans, iii. 107.

p. 262. \textsc{Old Sir Simon the King}.—In “Hans Beer-pot, his invisible
Comedie of See me and see me not” (4to., 1618), Cornelius says that
gentlemen did not formerly avow drunkenness, but “now beggars say
they are drunk like gentlemen.” He adds that he has heard “an old
fantastique rime:”—

\settowidth{\versewidth}{But serving men are drunlce, and all have one disease.”}
\begin{scverse}
\vleftofline{“}Gentlemen are \textit{sicke}, and Parsons \textit{ill at ease},\\
But serving men are \textit{drunke}, and all have one disease.”
\end{scverse}

These lines are a paraphrase of the two following in “Old Sir
Simon:”—

\settowidth{\versewidth}{The tapster was drunk in his dumps; they were all of one disease.”}
\begin{scverse}
\vleftofline{“}My hostess was sick of the mumps, the maid was ill at her ease,\\
The tapster was drunk in his dumps; they were all of one disease.”
\end{scverse}

Again in “The famous Historic of Fryer Bacon —

\settowidth{\versewidth}{But poor men they are drunke, and all is one disease.”}
\begin{scverse}
\vleftofline{“}Lawyers they are sicke, and Fryers are ill at ease,\\
But poor men they are drunke, and all is one disease.”
\end{scverse}

I am informed by Mr. Payne Collier that Friar Bacon was printed
soon after 1580, and these quotations increase the probability of
Ritson’s conjecture that the “hey, ding a ding,” mentioned in
Laneham’s Letter in 1575, was \textit{Old Sir Simon}.

\end{fixedpage}%776
\pagebreak
\renewcommand\rectoheadertext{down in the north country, the merry milkmaids, etc.}

\setlength{\fixedpagewidth}{360pt}
\begin{fixedpage}%777
\rectoheader

In \textit{Sportive Wit: The Muses' Merriment}, 1656, the burden is quoted
in a medley of songs:—

\settowidth{\versewidth}{With a thread-hare coat and malmsey nose,}
\begin{scverse}
\indentpattern{304}
\begin{patverse}
\vin\vin\vin \vleftofline{“}Old Simon the king,\\
With a thread-bare coat and malmsey nose,\\
Sing heigh----”
\end{patverse}
\end{scverse}

The last line quoted on p. 266, leads to another identification
for the tune. In the Roxburghe Collection, i. 170, is “Joy and sorrow
mixt together; or a pleasant new ditty, wherein you may find conceits
that are pretty, to pleasure your mind: To the tune of \textit{Such a Rogue
would he hang’d}.” It commences:—

\settowidth{\versewidth}{We’ll drink some good ale and strong beer,}
\begin{scverse}
\begin{altverse}
\vleftofline{“}Hang sorrow, let’s cast away care,\\
For now I do mean to be merry,\\
We’ll drink some good ale and strong beer,\\
With sugar, and claret, and sherry.\\
Now I’ll have a wife of mine own,” \&c.
\end{altverse}
\end{scverse}

The second part of this ballad makes the young man complain, and
wish in heart he were unmarried again. M[artin] P[arker] wrote “Have
among you, good women,” \&c., to the tune of \textit{O such a rogue}. See
Rox., i. 146. And I may add that “Time’s alteration,” beginning “When
this old cap was new,” was by him.

p. 279. \textsc{Down in the North Country}.—The three tunes, “Down in the
North Country,” “Ah, cruel, bloody fate,” and “The merry Milkmaids”
(pages 280, 281, and 282), belong to the reign of Charles II., and
not to that of Charles I. I was misled as to the date by supposing
“Within the North Country,” and “Down in the North Country,” to be
the same tune, for the words of the one could be sung to the tune of
the other.

“Ah, cruel, bloody fate” is by Purcell, and was sung between the
acts in Nat. Lee’s play of \textit{Theodosius}, 1680. Of the two remaining,
the one is merely an alteration, and the other an arrangement of that
air.

p. 282. \textsc{The Merry Milkmaids}.—The print in Tempest’s \textit{Cryes of
London} coincides with the description of the milkmaids given by
Misson in his \textit{Observations on his travels in England}, in the reigns
of James II. and William III. He says,
“On the first of May, and the five and six days following, all
the pretty young country girls that serve the town with milk, dress
themselves up very neatly, and borrow abundance of silver plate,
whereof they make a pyramid, which they adorn with ribbands and
flowers, and carry upon their heads instead of their common
milk-pails. In this equipage, accompanied by some of their fellow
milkmaids, and a bagpipe or fiddle, they go from door to door,
dancing before the houses of their customers, in the midst of boys
and girls that follow them in troops, and everybody gives them
something.” (Ozell’s translation, 8vo., 1719, p. 307.) We are told
that during Mary’s reign, the princess, afterwards Queen Elizabeth,
had little opportunity for meditation or amusement: that she was
closely guarded, yet sometimes suffered to walk in the gardens of the
palace. “In this situation,” says Holinshed, “no marvell if she,
hearing upon a time out of her garden at Woodstock, a certain
milkmaid singing pleasantlie, wished herself to be a milkmaid as she
was; saying that her case was better, and life merrier.” The remark
gave birth to an elegant ballad by Shenstone.

p. 283. \textit{Morris Dance}.—The first part of this tune will be. found
in the form of a tune for chimes, in Hawkins’s \textit{History}, 8vo., p. 770.
Hawkins supposes it to have
\end{fixedpage}%777
\pagebreak

\setlength{\fixedpagewidth}{360pt}
\begin{fixedpage}%778
\versoheader

been composed by Stephen Jeffries about 1680, but I have traced
it about forty years before Jeffries was born. It is probable,
however, that \textit{Engelsche Kloche Dans} should be translated English
Chime-tune, and not English Morris Dance, as I had supposed from the
mention of bells.

p. 283.—\textsc{Amarillis told her swain} is a song in Porter’s play, \textit{The
Villain}, 1663.

p. 288. \textsc{The Healths}.—In the novel of \textit{Woodstock}, Sir Walter Scott
makes Charles II. sing this song when in disguise; but Sir Walter
changed the last line of the stanza to “While the goblet goes merrily
round;” and the alteration does not suit the tune.

p. 289. \textsc{Mall Peatly}.—D’Urfey’s song of “Gillian of Croydon’’ is
perhaps an alteration of another under the same title, which is
included in “A Complete Collection of old and new English and Scotch
Songs,” i. 126, 1735. The latter commences, “Fame loudly through
Europe passes.”

p. 291. \textsc{When the Stormy Winds}.—In the Pepys Collection, i. 418
(No. 215), is another version, entitled “The praise of Sailors here
set forth, with their hard fortunes which doe befall them on the
seas, when Landmen sleep safe in their beds: \textit{To a pleasant new tune}.”
This was printed for J. Wright, and begins, “As I lay musing in my
bed.” The version from which Ritson prints, called “Neptune’s raging
fury,” \&c., bears the initials of J. P., and was printed by T. Mabb
for Ric. Burton.

The tune of \textit{The stormy winds do blow} was also used for “England’s
Valour and Holland’s Terrour, being an encouragement for seamen and
souldiers to serve his Majesty in his wars against the Dutch,” \&c.

p. 294. \textsc{Red Bull}.—In the Epistle Dedicatory to his \textit{Histriomastix},
Prynne says that “two old play-houses, the Fortune and the Red Bull,
have been lately re-edified and enlarged, and one new theatre, White
Friars play-house, erected.” He adds that the stationers informed him
that “above forty-thousand play-books had been printed within two
years, they being more vendible than the choicest sermons.”

p. 299. \textsc{The Queen’s old Courtier}.—Southey remarks very justly on
the complaints of the decay of hospitality, that “while rents were
received in kind, they must have been chiefly consumed in kind; at
least there could be no accumulation of disposable wealth.” He
supposes this mode of payment to have fallen generally into disuse
during the reign of James I. Without doubt, many of the poor would
feel the change.

p. 302. \textsc{Joan, to the Maypole}.—This tuns is also printed in \textit{Pills
to purge Melancholy} (i. 262, 1719), to a song by D’Urfey, commencing,
“The clock had struck.”

p. 329, note. \textsc{Dancing Barristers}.—See more of this custom in
Roger North’s \textit{Discourse upon the Laws}, p. 64 et seq.

p. 331. \textsc{Mad Tom}.—“At the Club of Choice Spirits, Mr. Spriggins
gave us \textit{Mad}
\end{fixedpage}%778
\pagebreak
\renewcommand\rectoheadertext{tom a bedlam, come you not from newcastle? etc.}

\setlength{\fixedpagewidth}{360pt}
\begin{fixedpage}%779
\rectoheader

\textit{Tom} in all its glory, and as he required chains to act in, the
president of the club ordered in the jack-chain.”—\textit{Goldsmith’s Essays}.

p. 332. \textsc{Tom a Bedlam}.—This tune had several other names, two of
which were \textit{Fly, Brass}, and \textit{The jovial Tinker}. In the Pepys
Collection, i. 460, is “A pleasant new songe of a joviall Tinker, to
a pleasant new tune called \textit{Fly Brasse}.” It is in ten-line stanzas, and
commences, “There was a joviall tinker.” In the same volume, and
immediately preceding it, is “The famous Rat Ketcher, with his
travels into France, and his return to London: To the tune of \textit{The
Joviall Tinker}.” It commences “The was a rare rat-catcher.” Both were
“imprinted by John Trundle,” and the latter, when he lived “at the
signe of the Nobody in Barbican.”

The following were also sung to the tune: “The Oakerman,”
beginning, “The star that shines by daylight” (\textit{Westminster Drollery},
Part ii., 1671); “I am a rogue, and a stout one” (written out to the
tune in Gamble’s MS.); “Tobacco’s a musician, and in a pipe
delighteth” (Nicholls’s \textit{Progresses}, or Rimbault’s \textit{Little Book of
Songs and Ballads}, p. 175); “All in the Land of Essex” (Sir John
Denham’s \textit{Poems}, 1671); “There was a jovial Tinker, dwelt in the town
of Turvey” (\textit{Merry Drollery Complete}, Part i., p. 27, 1670); and “The
zealous Puritan” (\textit{Loyal Songs}, i. 4, 1731).

The “Dr. G,” master of St. Paul’s School, who is celebrated for
his flogging propensities (p. 333), must have been Dr. Gale, who was
chosen high master in 1676, and held the appointment for twenty-five
years. He died April 8, 1702, and is mentioned both by Pepys and
Evelyn.

Some copies of the tune make the thirteenth and fourteenth bars
almost the same as the ninth and tenth, and it is better suited to
some of the songs in that form.

p. 339. \textsc{Come you not from Newcastle}?—The reference to this tune
in \textit{Friar Bacon}, carries back the date to the reign of Queen
Elizabeth. Robert Greene, who dramatised the pamphlet, died in 1592.

p. 344. \textsc{Buff Coat}.—The burden of a ballad entitled “The
kind-hearted creature,” seems particularly suitable to this air:—

\settowidth{\versewidth}{And her love beyond the ferry.”}
\begin{scverse}
\begin{altverse}
\vleftofline{“}Sing, boys! drink, boys!\\
Why should not we be merry?\\
I’ll tell you of a bonny lass,\\
And her love beyond the ferry.”
\end{altverse}
\end{scverse}

In Thompson’s \textit{Country Dances}, i. 59, the tune is found under the
name of \textit{Miss Peachey}, and in ii. 77, under that of \textit{The Retreat}.

p. 354. \textsc{Northern Nancy}.—D’Urfey alludes to the dance in his song,
“Jolly Roger Twangdillo of Plowden Hall” (\textit{Pills}, i. 20, 1719):—

\settowidth{\versewidth}{Ask’d \textit{parlez vous Fransay},” \&c.}
\begin{scverse}
\vleftofline{“}She danc’d \textit{Northern Nancy},\\
Ask’d \textit{parlez vous Fransay},” \&c.
\end{scverse}

The ballad of \textit{Mock-Beggars' Hall} is quoted as “to the tune of \textit{Is
it not your Northern Nanny}? or \textit{Sweet is the lass that loves me}.” The
last name is derived from a ballad by Martin Parker, entitled “Love’s
Solace, to \textit{a new court tune}” (Rox., i. 102); or, as in some later
copies, “Sweet is the lass that loves me: A young
\end{fixedpage}%779
\pagebreak


\setlength{\fixedpagewidth}{360pt}
\begin{fixedpage}%780
\versoheader

man’s resolution to prove constant to his sweetheart. To the tune
of \textit{Omnia vincit amor}” It commences:—

\settowidth{\versewidth}{Which makes me thus her praises tell,}
\begin{dcverse}
\begin{altverse}
\vleftofline{“}The damask rose, or lily fair,\\
The cowslip and the pansy,\\
With my true love cannot compare \\
For beauty or for fancy.\\
The fairest dames she doth excel \\
In all the world that may he,\\
Which makes me thus her praises tell,\\
\textit{So sweet is the lass that loves me}.”
\end{altverse}
\end{dcverse}

The tune of \textit{Omnia vincit amor} is to be found in the Skene
manuscript, and perhaps it is also the air referred to under the name
of \textit{The Damask Rose}, as the ballad commences with those words.

p. 356. \textsc{Grammercy Penny}.—This name is probably derived from a
ballad in the Pepys Collection, i. 218, “Oh gramercy Penny: Being a
Lancashire ditty, and chiefly penn’d To prove that a penny’s a man’s
best friend: To the tune of \textit{Its better late thrive than never}.” It is
subscribed L. P., and “printed for M. Trundle, widdow.” The first
line is “When I call to mind those jovial days.’’

p. 362.—\textsc{Gather your rosebuds} is included in Playford’s \textit{Select
Musical Ayres}, 1652, and a ballad to the tune will be found in the
Bagford Collection, 643, m. 11, p. 57.

p. 362. \textsc{Three merry boys are we}.—The words are in the \textit{Antidote to
Melancholy}, 1661, and are parodied in D’Urfey’s play. \textit{The Modern
Prophets}.

p. 363. \textsc{Cupid’s Courtesy}.—This ballad was licensed, with others,
to Coles, Wright, Vere, and Gilbertson, during the Protectorate,
viz., on 13th March, 1655.

p. 367.—\textsc{Shackley-Hay} was entered at Stationers’ Hall on 16th
March, 1612, to “Mystres White, late wife of Mr. Edward White, sen.,”
as “A pleasant songe of Yonge Palmus and fayr Sheldra.”

p. 370.—\textsc{Franklin is pled away} was parodied as late as 1782 in \textit{The
Convivial Songster}, and the tune there printed. The words commence,
“O let no eyes be dry, O hone, O hone.”

p. 390. \textsc{Robin Hood}.—“When Sherwood forest was surveyed in the
reign of James I., it consisted of 95,117 acres.” (\textit{Nichols’s
Progresses}, ii. 460.)

p. 408. \textsc{Old Custom of Kissing}.—Philip, second Earl of
Chesterfield, says in his \textit{Short Notes}, that, in 1662, when the
Infanta of Portugal was met by the Duke of York, afterwards James
II., “His royal Highness, out of compliment to the King, would not
salute her, to the end that his Majesty might be the first man that
ever had received that favour; she coming out of a country where it
was not the fashion.” Pepys, however, tells us that within ten days
of their arrival in England, the Portuguese ladies who came with the
Queen had “learnt to kiss and look freely up and down,” and he adds,
“I do believe they will soon forget the recluse practice of their own
country.” “To kiss and to be kissed,” says Burton, “is as a burden to
a song, and a most forcible battery, a great allurement, a fire in
itself.” (\textit{Love Melancholy}, Part iii., Sec. 2.)
\end{fixedpage}%780
\pagebreak
\renewcommand\rectoheadertext{mark how the blushful morn, martin parker, etc.}

\setlength{\fixedpagewidth}{360pt}
\begin{fixedpage}%781
\rectoheader

p. 414. \textsc{Mark how the blushful morn}.—Although attributed to
Charles I. in the manuscript, the music of this song is printed with
the name of Nicholas Lanier in \textit{Select Ayres and Dialogues}, Book ii.,
1669, and I suspect the printed authority to be better than the
manuscript.

p. 418. \textsc{Martin Parker}.—In 1640, the London petition complained of
“the swarming of lascivious, idle, and unprofitable books and
pamphlets, play-books and ballads, as namely, Ovid’s Fits of Love,
The Parliament of Women, Barns’s Poems, and \textit{Parker’s Ballads}.”
(Southey’s Common-place Book, p. 531.) In the introductory poem to
Austen's \textit{Naps upon Parnassus}, 8vo., 1658, Parker is styled “The
Ballad-Maker Laureat of London.” One of his little books, “The most
admirable Historie of that most renowned Christian worthy, Arthur,
King of the Britaines,” remained long in popular favour.

p. 419. \textsc{Thomas Herbert}.—The author of “Mercurie’s Message
defended against the vain, foolish, and absurd cavils of Thomas
Herbert, a ridiculous ballad-maker,’’ accuses Herbert of having
written \textit{Rome's A B C} against Archbishop Laud, and says, “In a blind
alehouse, I heard a crew of roaring ballad-singers trouling out a
merry ballad, called \textit{The more knaves the better company}. And one
among the rest cried out, ‘Well sung, Herbert,’ who, as it seems,
bore up the base among them, and in that deboist [debauched] manner
consumeth his time; and when his money is all spent (as for most part
it is six or seven times a week), writes a new merry book, a good
godly ballad, or some such excellent piece of stuffe, even as the
droppings of the spigot enliveneth his muddy muse to put his feeble
purse into fresh stock again.’’

p. 425, l. 26. \textsc{The good old cause}.—The Puritans’ definition of
their by-word, “The good old Cause,” was “religion and the laws.”
(See Dryden's \textit{Marriage à la Mode}, Act iv., sc. 3.) They who sided
with the king called themselves “Tantivy-boys” and “Tantivitiers,”
the name of “Cavaliers” being commonly applied to the upper classes
only.

p. 426, l. 1. \textsc{The clean contrary way}.—There are many more ballads
to the time of \textit{The clean contrary way}. One in the Roxburghe
Collection, ii. 571, was printed in 1681, and another, in the third
volume, has the burden of \textit{The clean contrary way}, and the name of the
tune given as \textit{Hey, boys, up go we}. This is entitled “Animadversions
on the Lady Marquess,” and begins—

\settowidth{\versewidth}{With coach and men on her to tend.}
\begin{dcverse}
\begin{altverse}
\vleftofline{“}The lady marquess and her gang \\
Are most in favour seen,\\
With coach and men on her to tend. \\
As if she were a queen.\\
But if she be, ’tis of the Sluts,\\
For all her fine array;\\
Her honour reaches to the skies. \\
But the \textit{clean contrary way},” \&c.
\end{altverse}
\end{dcverse}
“Printed for J. Jordan, at the Angel in Guiltspur Street.”

p. 430. \textsc{Vive le Roy}.—In Mrs. Behn’s comedy, \textit{The Round-heads, or
The good old Cause}, she twice represents the mob as shouting “Vive le
Roy.” Evelyn tells us in his diary that when James II. made his first
speech to the Houses of Parliament, they answered by cries of \textit{Vive le
Roi}. Other instances of its use have been cited under the head of \textit{God
save the King} (ante p. 699).
\end{fixedpage}%781
\pagebreak

\setlength{\fixedpagewidth}{360pt}
\begin{fixedpage}%782
\versoheader

p. 431. \textsc{Love lies bleeding}.—An old song on the times of James II.
and Wm. III., to this tune, will be found in \textit{Notes and Queries}, 2nd
series, ii. 43. It begins:—

\settowidth{\versewidth}{Truly out of season,” \&c.}
\begin{scverse}
\vleftofline{“}Lay by your reason,\\
Truly out of season,” \&c.
\end{scverse}

p. 437. \textsc{When the King enjoys his own}.—In Daniel Wright’s \textit{Country
Dances}, i. 32, the tune is entitled \textit{Trusty Dick}. “An excellent new
song of the unfortunate Whigs: to the tune of \textit{The King enjoys},”
\&c., is in the Roxburghe Collection, iii. 914, “printed for S.
Maurel,” in 1682. It begins—

\settowidth{\versewidth}{“The Whigs are but small, and of no good race.”}
\begin{scverse}
“The Whigs are but small, and of no good race.”
\end{scverse}

p. 451. \textsc{I live not where I love}.—The late Douglas Jerrold and his
circle of friends would often call upon Hazlitt to entertain them by
singing a West-country version of this ballad, which he gives with
all the richness of the West-country dialect. Jerrold was
particularly amused at the relation between cause and effect in the
second stanza, and used to call it “sublime.” I am indebted to Mr.
Hazlitt for the copy.

\settowidth{\versewidth}{Come, all you young maids as live at a distance,}
\begin{dcverse}
\begin{altverse}
\vleftofline{“}Come, all you young maids as live at a distance, \\
Many a mile from off your swain, \\
Come and assist me at this very instant \\
For to pass away some time;\\
Singing sweetly and completely,\\
Songs of pleasure from above.\\
My heart is with him altogether,\\
Though I live not where I love.
\end{altverse}

\begin{altverse}
If all the world was of one religion,\\
Many a living thing should die \\
Before that I would forget my true love,\\
Or in any way his love deny.\\
My heart should change and be more strange\\
If ever I’d inconstant prove;\\
My heart is with him altogether,\\
Though I live not where I love.
\end{altverse}

\begin{altverse}
Farewell, lads, and farewell, lasses,\\
Now I thinks I’ve got my choice,\\
I will away to yonder mountain,\\
For ’tis there I hears his voice.\\
If he hollow, I will follow\\
Through the world as is so wide,\\
For young Thomas did me promise\\
I should he his lawful bride.
\end{altverse}
\end{dcverse}

A comparison will prove that the above is a corruption of the
ballad which was printed more than two hundred years ago by Gosson;
but in all probability, it was kept in print very long after that,
time, and may be even now. Another current West-country version
begins, “Over hills and over mountains.”

p. 454. \textsc{Oh! for a husband}.—When Shakespeare makes Beatrice say,
in \textit{Much ado about Nothing}, “I am sunburned, I may sit in a corner,
and cry \textit{Heigh ho, for a husband},” it is by no means improbable that
he alludes to the burden of this song. The manuscript from which it
is derived is a collection of songs and ballads that were popular in
the reigns of James I. and Charles I. The writer flourished about
forty years after Shakespeare. \textit{Oh! for a husband} is included in “A
Complete Collection of Old and New English and Scotch Songs, with
their respective tunes prefixed,” i. 91, 1735, and in all the
editions of \textit{Pills to purge Melancholy}, but there reset by Akeroyde.

p. 455. \textsc{An old woman clothed in grey}.—In Walsh’s \textit{Country Dancing
Master},
iii. 86, this air is entitled \textit{Unconstant Roger}. The song, “Let
Oliver now be forgotten,” is by Tom D’Urfey, and included in his \textit{New
Collection of Songs and Poems}, 8vo., 1683.

p. 456. \textsc{I would I were in my own country}.—This tune is included
in Queen Elizabeth’s Virginal Book, under the name of the \textit{Quodling’s
Delight}.
\end{fixedpage}%782
\pagebreak
\renewcommand\rectoheadertext{the broom, the bonny broom, etc.}

\setlength{\fixedpagewidth}{360pt}
\begin{fixedpage}%783
\rectoheader

p. 458. \textsc{The Broom, the bonny Broom}.—In \textit{The Carnival}, a comedy by
Thomas Porter, 4to., 1664, a song “to the tune of \textit{The broom, the
bonny broom},” begins thus: 

\settowidth{\versewidth}{The beard, the beard, the bonny, bonny beard,}
\begin{scverse}
\begin{altverse}
\vleftofline{“}The beard, the beard, the bonny, bonny beard,\\
Oh! it was of wondrous growth;\\
But, eating too fast, his spoon he misplac’d,\\
And scalded it off with the broth.”
\end{altverse}
\end{scverse}

“An excellent new song, entitled The new Song of the Broom of
Cowden Knows,” is in the possession of Mr. David Laing, who dates the
copy “circa 1716.” It commences, “Hard fate that I should banisht
be.” This would be about eight years earlier than the “new words” in
the \textit{Tea Table Miscellany}, “How blyth each morn was I to see.”

p. 464. \textsc{Christmas is my name}.—The flocking of the nobility to
London at Christmas, complained of in the ballad, was the occasion of
a proclamation by James I., which is thus noticed in a letter from
Mr. Chamberlain to Sir Dudley Carleton, bearing date Dec. 21, 1622:
“Divers Lords and personages of quality have made means to be
dispensed withall for going into the country this Christmas according
to the proclamation; but it will not be granted, so that they pack
away on all sides for fear of the worst.” (Nichols’s \textit{Progresses of
James I}.)

p. 478, l. 26. \textsc{Sir W. Davenant’s Siege of Rhodes}.—On the 9th
September, 1653, \textit{The Siege of Rhodes} was entered at Stationers’ Hall,
with many other plays, to “Mr. Mosely.” On 27th August, 1656, it was
again entered as “The Siege of Rhodes by Sir William Davenant, acted
at the back of Rutland House,” \&c., and then printed by Henry
Herringman. In the preface to \textit{The Fairy Queen}, an opera by Purcell,
“represented at the Queen’s theatre, by their Majesties’ servants,”
is the following passage:—“That Sir William Davenant’s \textit{Siege of
Rhodes} was the first opera we ever had in England, no man can deny;
and it is indeed a perfect opera; there being this difference between
an opera and a tragedy; that the one is the story sung with proper
action, the other spoken.” (4to., 1692.) If Dr. Burney had read this
preface, he might have avoided his error about the first operas in
England.

p. 482. \textsc{Row the boat, Norman}.—The missing words to this “roundel”
are supplied by Skelton, in his \textit{Bowge at Court}, where Harvy Hafter
says:—

\settowidth{\versewidth}{Heve and how rombelow, row the bote, Norman, rowe}
\begin{scverse}
\vleftofline{“}I wolde be mery, what wynde that ever blowe,\\
\textit{Heve and how rombelow, row the bote, Norman, rowe}!”
\end{scverse}
(Dyce’s \textit{Skelton}, i. 40.) The commencing with “Heave and ho,
rumbelow,” is the material part. It was evidently the burden or
under-song, and sung on the key-note by each of the three voices in
turn. The words should, therefore, stand thus:—

The second singer begins two bars after the first, and the third
two bars after the second. They continue in that order, without
stopping at the end of the line, but recommencing and singing it over
many times. “Our sailors at Newcastle, in heaving their anchors, have
[still] their \textit{‘Heave and ho, rumbelow},’” says D’Israeli,
in his \textit{Curiosities of literature}.
\end{fixedpage}%783
\pagebreak

\setlength{\fixedpagewidth}{360pt}
\begin{fixedpage}%784
\versoheader

p. 483. \textsc{Now God be with old Simeon}, or \textsc{Hey, Jolly Jenkin}.—This is
printed as an old catch in vol. iv. of Ramsay’s \textit{Tea Table Miscellany},
but thus differing in the latter part:—

\settowidth{\versewidth}{Ho! jolly Jenkin, I spy a knave drinking,}
\begin{dcverse}
\vleftofline{“}To whom drink you, Sir knave, \\
Turn the timber like the lave; \\
Ho! jolly Jenkin, I spy a knave drinking,\\
Come, troll the bowl to me.”
\end{dcverse}

I am informed by the Rev. Thomas Corser, F.S.A., that the catch
is alluded to in Ulpian Fulwell’s \textit{Eyghth Part of the Liberall
Science}, 4to., 1579.

p. 488. \textsc{Conducting by a woman}.—The “woman with a rod in her
hand,” mentioned by Pepys as “keeping time to the musique,” was a
puppet. I mistook the word “motion,” in a hastily-written extract,
for “notion,” although, on collation, the word was corrected in the
text.

p. 490, l. 25. \textsc{Scotch Tunes in the time of Charles II}.—In Mr.
Halliwell’s Collection is a ballad entitled “A loyal subject’s
admonition, or A true song of Brittain’s Civil Wars,” \&c. “To the
the tune of \textit{General Monck’s right March}, that was sounded before him
from Scotland to London, or \textit{The Highlanders’ March}.’’ “Printed for F.
Grove, on Snow Hill.” \textit{The Highlanders’ March} is one of the three tunes
I have named as in \textit{The Dancing Master} of 1665. The words of “Johnny,
cock thy beaver,” are so much in the style of “Jockey is grown a
gentleman,” that I think them rather a good-humoured joke upon the
Scotch, than a genuine Scotch song. The following is Herd’s version
(\textit{Scottish Songs}, ii. 205, 1776):—

\settowidth{\versewidth}{When first my dear Johny came to this town,}
\begin{dcverse}
\vleftofline{“}When first my dear Johny came to this town, \\
He had a blue bonnet, it wanted the crown; \\
But now he has gotten a hat and a feather, \\
Hey, my Johny lad, cock up your beaver: \\
Cock up your beaver, cock up your beaver,\\
Hey, my Johny lad, cock up your beaver;\\
Cock up your beaver, and cock it nae wrang,\\
We’ll a’ to England ere it be lang.”
\end{dcverse}

p. 495. \textsc{The King’s Jig}.—“A new song in praise of the loyal
Company of Stationers, who (after the general forfeit), for their
singular loyalty, obtain’d the first Charter of London, 1684. To the
tune of \textit{Winchester Wedding},”—is printed in the 180 \textit{Loyal Songs}, 1685
and 1694. The worshipful Company of Stationers obtained a restitution
of their Charter, in consequence of their “dutiful submission” to the
Court.

D'Urfey’s song, “The Winchester Wedding,” was also printed on
broadsides with music. The tune (besides the authorities mentioned)
is in Salter’s \textit{Genteel Companion for the Recorder}, 1683: in the
ballad-operas of \textit{Flora}; \textit{The Devil of a Duke}; \textit{The Moch Doctor}; \textit{The
Quakers' Opera}; \textit{The Highland Fair}; \textit{The Jovial Crew}; and many more: in
\textit{The Convivial Songster}, 1782; and the words in various song-books of
the last century.

A tune called “The King’s Jigg,” in Blaikie’s manuscript, is not
the same. It refers, in all probability, to some later King than
Charles II.

p. 498. \textsc{The Wine-Cooper’s Delight}.-—The duty imposed upon French
wines in 1681, met with great opposition at the time, and was deemed
quite prohibitory. A ballad-writer says:—

\settowidth{\versewidth}{Than to poison the subject and beggar the King.”}
\begin{scverse}
\vleftofline{“}French wine’s prohibition meant no other thing \\
Than to poison the subject and beggar the King.”
\end{scverse}
And a writer in the \textit{Poems on Affairs of State}:—

\begin{scverse}
\vleftofline{“}As well may Dutchmen without brandy fight,\\
As English poets without claret write.”
\end{scverse}
\end{fixedpage}%784
\pagebreak
\renewcommand\rectoheadertext{willy was so blithe a lad, etc.}

\setlength{\fixedpagewidth}{360pt}
\begin{fixedpage}%785
\rectoheader

p. 507. \textsc{Willy was so blithe a lad}.—In \textit{Youth's Delight on the
Flagelet}, this is called \textit{Billy was as blyth a lad}.

p. 509. \textsc{Jenny, gin}.—This is a song by Mrs. A. Behn, in her
comedy, \textit{The City Heiress}, commencing, “Ah, Jenny, gin your eyes do
kill.” (1682.) The tune is in Playford’s \textit{Choice Ayres}, v. 25, 1684,
and in all editions of \textit{Pills to purge Melancholy} .

p. 513. \textsc{The leather Bottel}.—The song on Queen Mary here referred
to, written by “William Forrest, \textit{Preest},” was, no doubt, W. Forrest,
of Christ Church, Oxford, who was Chaplain to Queen Mary on her
accession to the throne.

p. 515. \textsc{Turn again, Whittington}.—Some suppose Sir Richard
Whittington to have laid the foundation of his fortune by a coasting
vessel, called a Cat. In the \textit{Encyclopædia Londinensis}, a Cat is thus
defined: “A ship employed in the coal trade, distinguished by a
narrow stern, projecting quarters, and a deck waist. These vessels
are generally built remarkably strong, and carry from four to six
hundred tons (or, in the language of the mariners, from twenty to
thirty keels) of coals.” Cat-water, at Plymouth, is the harbour for
coasters, traders, colliers, \&c., \&c., so called to this day.

p. 524, 1. 3. \textsc{Young Jemmy}.—The song commencing, “Young Jemmy \textit{was}
a lad,” is by Mrs. A. Behn, and included in her \textit{Poems upon several
occasions}, 8vo., 1684. Another song on the Duke of Monmouth was
printed for R. Shuter in 1682, and entitled “Jemmy and Anthony: to
the tune of \textit{Young Jemmy}” (Rox., iii. 917.) It commences:—

\settowidth{\versewidth}{And prosper long those noble peers,}
\begin{dcverse}
\begin{altverse}
\vleftofline{“}Monmonth is a brave lad, \\
The like’s not in our city; \\
He is no Tory blade,—\\
Give ear unto my ditty!\\
Long may he live in happy years, \\
Victorious may he be,\\
And prosper long those noble peers,\\
\textit{Monmouth and Shaftesbury}.”
\end{altverse}
\end{dcverse}

p. 527. \textsc{My lodging is on the cold ground}.—The parody on this
song, which was sung by Nell Gwyn in Howard’s play, \textit{All mistaken},
contains a personal allusion to her rival, Moll Davis, who was short
and fat. (See Cunningham’s \textit{Story of Nell Gwyn}, p. 60, edit. 1852.) It
commences:—

\settowidth{\versewidth}{My lodging is on the cold boards,}
\begin{dcverse}
\begin{altverse}
\vleftofline{“}My lodging is on the cold boards, \\
And wonderful hard is my fare; \\
But that which troubles me most is\\
The fatness of my dear,” \&c.
\end{altverse}
\end{dcverse}

Between 1713 and 1775, the original song may be found in \textit{The
Hive}, 1726 and 1732, and in \textit{Vocal Miscellany}, 1734. Soon after 1775,
the air was introduced by Giordani as the \textit{larghetto} movement of the
third of his first set of \textit{concertos} for the harpsichord, Op. 14, and
on the 15th May, 1794, the song was entered at Stationers’ Hall, as
sung by Mrs. Harrison at Harrison and Knyvett’s concerts; so that it
may be traced in constant favour in England from the time of Charles
II. down to the present day. I cannot find a shadow of reason for
calling it an Irish air. The best Irish authorities disclaim it, and
the air may even have been unknown in Ireland before Giordani went to
reside there, for any proof we have to the contrary. Giordani went to
Dublin in 1779, and the \textit{second} set of his concertos was entered in
London at Stationers’ Hall on the 12th February of that year. After
the failure of his theatrical speculation, Giordani commenced
teaching in Dublin, and, introducing his own music, attained great
repute.
\end{fixedpage}%785
\pagebreak

\setlength{\fixedpagewidth}{360pt}
\begin{fixedpage}%786
\versoheader

p. 536. \textsc{The Northumberland Bagpipes}.—In \textit{Apollo's Banquet}, 1693,
this tune is entitled “A new dance in the play of \textit{The Marriage-Hater
match’d}.” This comedy is by D’Urfey, and was printed in 1692.

\textsc{When busy Fame}.—There are many more ballads than I have named to
this tune. See, for instance, the 2nd vol. of the Roxburghe
Collection, pages 7, 45, 68, 224, 322, and 445.

p. 547. \textsc{The Waits}.—Several instances of holding land by
\textit{wait-service}, or by payments for that service, will he found in
Blount’s \textit{Ancient Tenures}. Thus, in Norfolk, Thomas Spelman held the
manors of Narborough and Wingrave by knight-service, and paying
fourteen shillings annually for wayte-fee and castle-guard; and John
Le Marshall held the manor of Buxton by paying a mark every six weeks
for guarding Norwich Castle, and fifteen shillings quarterly for
wayte-fee at the said castle.

p. 552. \textsc{Jack met his Mother}.—This ballad was re-written in other
metre, under the title of \textit{Hodge of the Mill and buxom Nell}. See \textit{Tea
Table Miscellany},
iv. 379.

p. 559. \textsc{The Northern Lass}.—The Scotch sing the song of \textit{Muirland
Willie} to this tune,—not to the slow version, which is evidently the
original,—but to the air in its abbreviated dancing form. We do not
find \textit{Muirland Willie} sung to it until after it had been turned into a
lively air by D’Urfey, and, although the words of the Scotch song are
old, we have no indication of any tune to which they were to be sung
in early copies. They seem to have been intended for \textit{Green Sleeves},
more likely than any other air. \textit{Muirland Willie} was first printed to
this tune by Thomson, in his \textit{Orpheus Caledonius}, folio, entered at
Stationers’ Hall on 5th January, 1725-6. The tune had then been
published, as \textit{Great Lord Frog}, in Walsh’s 24 \textit{New Country Dances for
the year} 1713; with words in vol. i. of \textit{The Merry Musician}, dated
1716, and in vol. i. of \textit{Pills to purge Melancholy} , 1719.

p. 570. \textsc{Lilliburlero}.—After sifting the evidence as to the origin
of this tune, I have no hesitation in ascribing it to Henry Purcell.
In the preface to Part ii. of \textit{Musick’s Handmaid}, H. Playford says, “I
have accordingly, with much care, compleated this Second Part,
consisting of the newest Tunes and Grounds composed by our ablest
masters, Dr. John Blow, Mr. Henry Purcell, \&c., the impression being
carefully revised and corrected by the said Mr. Henry Purcell.” The
distinction between compositions and arrangements is clearly drawn in
the book. Thus, “A Theater tune” is \textit{set} (i.e., arranged for the
virginals) by Dr. John Blow, who was not a theatrical composer; but
\textit{Lilliburlero} bears the name of H. Purcell without any such
qualification.

p. 597. \textsc{Come and listen to my ditty}.—A claim has been made for
Falconer, the author of \textit{The Shipwreck}, to the song of \textit{Cease, rude
Boreas}, on the ground that G. A. Stevens had access to Falconer’s
manuscripts after his death. This supposition is quite set at rest by
dates, for “Cease, rude Boreas, by Mr. Stevens,—Tune, \textit{Come and listen
to my ditty}” is Song 207, p. 291, of \textit{The Muses Delight}, 8vo.,
Liverpool, 1754. Eight songs by G. A. Stevens are there printed
together. Falconer was in
\end{fixedpage}%786
\pagebreak
\renewcommand\rectoheadertext{fairest jenny, dancing the hay, etc.}

\setlength{\fixedpagewidth}{360pt}
\begin{fixedpage}%787
\rectoheader

the merchant service till 1762, when he first came into notice by
the publication of \textit{The Shipwreck}. He was lost at sea in 1769. Stevens
merely re-introduced his own song in his \textit{Marine Medley} of 1772.

p. 617. \textsc{Fairest Jenny}.—In \textit{The Devil to pay}, this tune is entitled
\textit{Take a kiss or twa}, from the second part of the first stanza.

p. 629. \textsc{Dancing the Hay}.—In Chaucer’s third hook of Fame, among
the Court entertainments were pipers to assist those who chose to
dance either “love-dances, springs, or \textit{rayes},” and in Barclay’s
Eclogues, 1508, a shepherd says, “I can dance the \textit{raye}; I can both
pipe and sing.’’ Quere, is \textit{raye} an earlier name for \textit{hay}?

p. 649. \textsc{O Mother, a hoop}.—Hoops seem to have come into fashion
about 1711. There is an entry at Stationers’ Hall, July 18, 1711, as
follows:—“The Farthingale reviv’d, or More work for the Cooper. A
Panegyrick upon the late, but most admirable invention of the
hoop-pettycoat. Written at the Bath in the year 1711.”

p. 652. \textsc{Vicar of Bray}.—Simon Aleyn has long had the credit of
being the proverbial Vicar of Bray, but it appears from various
authorities quoted in \textit{Athænæ Cantab}, (i. 107), that Simon Symonds
was instituted to the vicarage on the 14th March, 1522-3, and died
about 1551; therefore, the story cannot apply to any vicar of that
time.

p. 700, l. 32. \textsc{Words of Anthems}.—I was mistaken in saying that the
words of anthems “are never in rhyme;” there are many old exceptions to
the rule, and sometimes, although printed as prose, they are really
in rhyme.

p. 715. \textsc{Lovely Nancy}.—I observe that in Book 2 of Oswald’s
\textit{Caledonian Pocket Companion}, this tune is printed as“Lovely Nancy,
by Mr. Oswald.” I have no doubt that he meant to claim the variations
only, for he had previously printed the air, with some difference in
the arrangement, in his \textit{Curious Scots’ Tunes for a violin and flute},\textsuperscript{a}
and then without making any such claim. Oswald has been taken to task
by Mr. G. F. Graham, in the notes on Wood’s \textit{Songs of Scotland}, for
having similarly placed his name before Scotch tunes of which it is
impossible that he can have been the author.

I have seen many half-sheet copies of the song of \textit{Lovely Nancy},
but never with an author’s name, and I doubt whether any one could
properly claim it, for it seems to be only an alteration of \textit{Ye
virgins so pretty} (ante p. 682).

\textit{Lovely Nancy} was turned into a Country Dance in vol. iii. of
Johnson’s Collection, 1744, and the song is included in Walsh’s
\textit{Select Aires for the Guitar}. There are five stanzas, of which the
following are the first three:—

\settowidth{\versewidth}{But which soon, if you frown on, must end in despair.}
\begin{scverse}
\vleftofline{“}How can you, lovely Nancy, thus cruelly slight \\
A swain who is wretched when banish’d your sight?\\
Who for your sake alone thinks life worth his care,\\
But which soon, if you frown on, must end in despair.
\end{scverse}

\begin{dcfootnote}
\textsuperscript{a}  This collection was printed by John Simpson at the
Bass Viol and Flute in Sweeting’s Alley. It is difficult
to know why the tune should have been included in a
collection of \textit{Scotch} tunes, but no one will be surprised
who examines the remainder of the selection, It was
not the fashion of that day to attempt accuracy in the
slightest degree.
\end{dcfootnote}
\end{fixedpage}%787
\pagebreak


\setlength{\fixedpagewidth}{360pt}
\begin{fixedpage}%788
\versoheader

\settowidth{\versewidth}{As they shed such mild influence, they e’er would deceive.}
\begin{scverse}
If you meant thus to torture, O why did your eyes \\
Once express so much fondness, and sweetly surprise?\\
By their lustre inflam’d, I could never believe,\\
As they shed such mild influence, they e’er would deceive.

But, alas! like the pilgrim bewilder’d in night,\\
Who perceives a false splendour at distance invite,\\
O’erjoyed hastens on, pursues it, and dies,\\
A like ruin attends me if away Nancy flies.”
\end{scverse}

p. 715. \textsc{Hearts of Oak}.—Boswell, in his visit to Corsica, says
that the Corsicans requested him to sing them an English song, and he
sang them \textit{Hearts of Oak}. ‘‘Never did I see men so delighted with a
song as the Corsicans were with \textit{Hearts of Oak}. ‘Cuore di querco,’
cried they, ‘Bravo, Inglese.’ It was quite a joyous riot. I fancied
myself to be a recruiting sea-officer—I fancied all my chorus of
Corsicans aboard the British fleet.”—Croker’s edit, of Boswell’s \textit{Life
of Johnson}, x. 233.

\centerrule
\end{fixedpage}%788
\pagebreak

\setlength{\fixedpagewidth}{360pt}
\begin{fixedpage}%789
\headingfour{CHARACTERISTICS OF ENGLISH NATIONAL AIRS,}
\headingfive{AND SUMMARY.}

There are two principal causes which affect the national music of
countries,—the first, the character of the musical instruments in
common use, and the second, the spirit of the songs of the people.
The first is most easily discernible where the employment of one
instrument has predominated greatly over others,—as in Spain, the
guitar, and in Switzerland, the mountain-horn. The Spaniards have
scarcely an air of sustained notes, and Swiss airs are nearly all
composed of the open sounds of the horn.

In England, we had three instruments in general use from very
early times,—the harp, the fiddle (with its variety, the crowd), and
the pipe, both with and without the bag.

The impress of the harp is left upon many of our airs in a bold
stateliness of character, such as is found, from the same cause, in
Welsh music. No one can hear a tune like \textit{Mall Sims}, p. 178, without
being at once reminded of the harp. If this character is not equally
traceable in Irish music, it is in all probability because the Irish
continued to use the small harp, strung with wire, and played upon by
the nails,\textsuperscript{a} when it had fallen into disuse in England and Wales. The
wire strings would vibrate longer than the gut, but their chords
would lack force and decision. The prevalence of the fiddle in
England is shewn in the large proportion of smooth and flowing airs,
and in many spirited dances, like \textit{Roger de Coverley}.\textsuperscript{b} The pipe and
bagpipe are represented in numerous hornpipes, jigs, rounds, and
North-country frisks. There

\begin{dcfootnote}
\textsuperscript{a} Camden, in 1566, and Stanihurst, in 1584, both say that Irish
harps were strung with wire, and the latter (an Irishman who had been
educated in England), that wire strings were not then used elsewhere.
The sense of a long passage ahout harpers and viol-players, in
Stanihurst’s \textit{De Rebus in Hibernia gestis} (4to., Antverpiæ, 1584, p.
38), was misunderstood by Walker, who, in his \textit{Historical Memoirs of
the Irish Bards} (4to., 1786, p. 145), translates “lyra” \textit{harp}, instead
of \textit{viol}. \textit{Cithara} is the word use by Stanihurst for harps of both
kinds, wire and gut. \textit{Lyra} was the conventional Latin for instruments
played upon with a bow, from the ninth century. See the drawing of a
“lyra” in Gerbert’s \textit{De Cantu}, vol. ii., tab. 32. In a note upon
“crowde, instrument of musyke,” in the \textit{Promptorium Parvulorum}, Albert
Way quotes Vocab. Roy. MS. 17, c. xvii., “a crowde, \textit{corus, lira}.” The
English name of \textit{lyra-viol} for a large instrument of the viol kind,
also shews the application of the word. Galilei, who was Stanihurst’s
contemporary, says, “La viola da braccio, detta da non molti anni
indietro \textit{lira}, ad imitatione dell’ antica quanto al nome.” I notice
Walker’s mistake because it is not the only passage in which his
translations are affected (see, for instance,
p. 133), and he has already been copied by Edward Jones, in his
\textit{Welsh Bards}, i. 98; by Bunting, in his \textit{Ancient Music of Ireland},
fol., p. 19; and by others. As the testimony of an adverse witness is
always the best that can be produced, I recommend to the notice of
writers on Irish music a passage of twenty lines in\textit{ The Image of
Irelande, \&c., made and devised by John Derricke in} 1578, 4to.,
London, 1581, which I have not seen quoted. It is accompanied by the
following marginal comments:—“A Barde and a Rimer is all one”—“The
Barde by his rimes hath as great force among woodkarne to persuade, as
the elloquent oration of a learned oratour emongest the civill
people”—“The policie of the Barde to encense the, rebelles to doe
mischefe by repeating their forfathers’ acts.” These three points are
the themes of his song. Whilst on the subject of the harp, I may
remark another mistake in Bunting’s \textit{Ancient Music}. In the last line
of the note at p. 14, he says that Venantius Fortunatus “gives the
harp to the \textit{Germans},” instead of to the “\textit{barbarians}” and in the text
writes of the “\textit{Teutonic} harp,” which is not there mentioned.

\textsuperscript{b} By an oversight, the first four bars of \textit{Roger de Coverley} are
printed, at p. 535, an octave too high.
\end{dcfootnote}

\end{fixedpage}%789
\pagebreak
\renewcommand\versoheadertext{characteristics of english national airs,}

\setlength{\fixedpagewidth}{360pt}
\begin{fixedpage}%790
\versoheader

are also a few song-tunes which, like “Who liveth so merry in all
this land?” (p. 81) seem to require the bagpipe drone.

According to Bunting, the ancient bagpipe had neither fourth nor
seventh in its scale; and others say that some of the pipes were
equally deficient. Never having been fortunate enough to meet with
any directions of early date for playing upon these instruments, I
can add nothing to what has been written, from my own knowledge; but
the omission of the intervals of the fourth and seventh in some
Irish, Scotch, and English North-country tunes, gives a plausibility
to the assertion that such imperfect instruments were in use. It
would indeed have been far more satisfactory to me if I had seen
them, as I am rather incredulous, being unable to account for the
omission of the fourth upon any reasonable hypothesis. Collectors of
Scottish music have laboured hard to prove, in a practical way, that
such imperfect instruments were pre-eminently employed in Scotland,
for their great study has been how to alter tunes to its scale,
without at all troubling themselves for authorities. Every bagpipe
that I can trace had a fourth. The Scotch' Highland bagpipe has not
only a fourth, but also the two sevenths, major and minor, can be
produced upon it. Every scale, under the old system of music, had a
fourth.

Much of the omission of the seventh is, without doubt, to be
attributed to the old modes which were in use before our present
tonality of the minor scale was established. By “seventh,’’ I mean
the semitone below the octave; for whether the seventh is major or
minor (that is, a semitone or a whole tone below the octave),
constitutes the entire difference between our present ascending minor
scale and one of the most popular of the ancient modes, called by
some the Dorian, and by others the first of the old Church modes.
(See p. 12.) I can understand pipes having been made with a minor
seventh, but how the fourth and seventh could both be omitted is the
mystery. Even now we may occasionally hear English tunes sung on the
Dorian scale by untutored singers, and I may instance the version of
\textit{We are poor frozen-out gardeners}, at p. 748, and the Christmas Carol,
at p. 753. The F sharp in the first bar of the former is not what is
termed an “essential” note.

Some persons have called the minor scale “the scale of nature,”
but they have probably not distinguished sufficiently between the old
and new systems. Without doubt, many tunes that were upon old scales
have been since altered to minor keys; nevertheless, the two earliest
of which England can boast are both in major. See pages 24 and 27.

I think no point more likely to strike the hearer in the
preceding collection than the very limited number of airs of a really
melancholy cast. Some few are susceptible of great pathos,—such as,
“In sad and ashy weeds,” p. 202; “O willow, willow,” p. 207; “I sowed
the seeds of love,” p. 522; and “The Northern Lass,” p. 560;— and,
among the narrative ballads, “The Three Ravens,” p. 59; “Near
Woodstock Town,” p. 161; “The Children in the Wood,” p. 201; and “Oh!
the Oak and the Ash,” p. 457: but, even including these, the total
number will scarcely exceed twelve, out of more than four hundred
collected. They who had \textit{deep} sorrows seem rarely to have sung of
them, and six, at least, of the twelve melancholy airs were
afterwards parodied, or turned into quick tunes.

Some persons have written of the sudden changes from major to
minor in the beautiful national music of Ireland, as if these were
the outbursts of grief in their gayest moments. I cannot say that I
have observed any signs of sorrow in Irishmen
\end{fixedpage}%790
\pagebreak
\renewcommand\rectoheadertext{and summary.}

\setlength{\fixedpagewidth}{365pt}
\begin{fixedpage}%791
\rectoheader

when their merry tunes have been played; but if these changes are
proofs of grief, it is at least a grief of very long standing,—for
Giraldus Cambrensis remarked the same peculiarity in the music in the
year 1185. While passing the highest eulogium upon the musicians of
Ireland, he comments upon these rapid and unexpected changes and
modulations in their liveliest airs. The reason now assigned is
perhaps rather more poetical than true.

The characteristic airs of England may be broadly divided into
four classes,—the first and largest division consisting of airs of a
smooth and flowing character—expressive, tender, and sometimes
plaintive, but generally cheerful rather than sad. These are the
ditties, the real pastorals, which are so often mentioned by our
early writers, and in which our poets so constantly expressed their
delight.

The second comprises airs which breathe a frank and manly spirit,
often expanding into rough jollity. Such were many of the songs of
men when not addressed to the fair.

The third consists of the airs to historical and other very long
ballads, some of which airs have probably descended to us from the
minstrels. They are invariably of simple construction, usually
plaintive, and the last three notes often fall gradually to the
key-note at the end. One peculiar feature of these airs is the long
interval between each phrase, so well calculated for recitation, and
for recovering the breath in the lengthy stories to which they were
united. They were rarely, if ever, used for dancing; indeed, they
were not well suited to the purpose, and therein differed from the
carols, and from the ditties, which were usually danced to and sung.
Ditties when accelerated in time, to fit them for dancing, would fall
under the denomination of carols.

In the fourth class may be comprised the numerous hornpipes,
jigs, rounds, and bagpipe-tunes. In the sixteenth and seventeenth
centuries, when villagers assembled every holiday, and on Sunday
evenings after prayers, to dance upon the green, every parish of
moderate population had its piper. “The constable ought not to break
his staff and forswear the watch for one roaring night,” says Ben
Jonson, “nor the piper of the parish to put up his pipes for one
rainy Sunday.” “It was not unusual, I believe,” says Mr. Surtees, “to
amuse labourers on bounty days with music; a piper generally attended
on highway days.” He quotes the following entry in the parish
registers of Gateshead, under the year 1633:—“To workmen, for making
the streets even, at the King’s coming, 18s. 4d.; and paid the piper
for playing to the menders of the highways five several days, 3s.
4d.” Milton, in his speech upon unlicensed printing, says, “The
villagers also must have their visitors, to enquire what lectures the
bagpipe and the rebec reads, even to the ballatry, and the gammuth of
every municipal fiddler, for these are the countryman’s Arcadia, and
his Monte Mayors.’’

The bagpipe was not an instrument in favour with the upper
classes in England; indeed it was generally spoken of with contempt.
When a merry-making was of a mixed character, such distinctions as
the following were usually drawn: “Among all the pleasures provided,
a noise of minstrels and a Lincolnshire bagpipe was prepared; the
minstrels for the great chamber, the bagpipe for the hall, the
minstrels to serve up the knights’ meat, and the bagpipe for the
common dancing.” (\textit{Nest of Ninnies}, 1608.)

Formerly, the bagpipe was in use among the lower classes all over
England, although now happily confined to the North. With it many of
our bagpipe-tunes
\end{fixedpage}%791
\pagebreak

\setlength{\fixedpagewidth}{360pt}
\begin{fixedpage}%792
\versoheader

have travelled Northwards, and thus have become absorbed in
collections of Scottish music. A glance at Daniel Wright’s
“Extraordinary Collection of pleasant and merry humours; containing
Hornpipes, Jiggs, North-Country Frisks, Morrises, Bagpipe-Hornpipes,
and Rounds,” and other works of the same kind, will give evidence of
the migration. I have chosen but few specimens of these tunes, not
from the lack of them, but because limited to one volume to include
all airs from the time of the Commonwealth, and ballad tunes are of
more general interest.

A few extracts about hornpipes have already been given, at pages
544 to 546, 740, and 741. The really old hornpipes, whether for the
fiddle or bagpipe, are all, or nearly all, in triple or compound
triple time; but the measure of jigs is not equally defined. The
greatest number is in compound common time; but some are in simple
common time, while others are not distinguishable from hornpipes.

The jig is now completely associated in the public mind with
Ireland, but English writers of the sixteenth and seventeenth
centuries teem with comments upon it. Although the number of
excellent Irish jigs is now great, I have not found one called Irish
before the latter part of the seventeenth century. Unless evidence
can be given of the existence of the dance in Ireland long anterior
to any that has hitherto been quoted, I submit the probability of its
having extended from “the English pale,” but am not sufficiently
versed in Irish history to give an opinion, with any confidence, as
to its origin in Ireland.

Scotch jigs are noticed by English writers long before those of
the sister country, and Shakespeare’s comparison of “wooing, wedding,
and repenting,” to “a Scotch jig, a Measure, and a
Cinque-pace,” proves that the mode of dancing them was well known in
his time. “The first suit is hot and hasty, like a Scotch jig, and
full as fantastical.” About three years before the publication of
Shakespeare’s play, Morley had written as if English composers were
in the habit of making tunes to this dance; for, in speaking of the
best descanters as but sorry composers, he says, “enjoyne him but to
make a Scottish jygge, he will grossely erre in the true nature and
quality of it.” \textit{Introduction to practicall Musicke}, p. 182, 1597.)
One “Scotish jig” will be found in \textit{Apollo's Banquet}, 1669, but its
genuineness is to be doubted, for it is far more like the rough and
bold style of English music than any other; and I suppose the Scotch
will not claim it, having both fourth and seventh in its scale. It is
the tune to which D’Urfey wrote the song, “Maiden, fresh as a rose,”
in \textit{The Richmond Heiress}, and which, in \textit{The Dancing Master}, is called
\textit{A Trip to Marrowbone}. It proves, however, that Scotch jigs were
danced to tunes in triple or compound triple time; for the second
grows naturally out of the first in the process of division or
variation. Mr. G. F. Graham, in his introduction to \textit{The Dance Music of
Scotland}, says, “The high popularity of the Reel and Strathspey, all
over Great Britain, induces us to dwell more particularly and
minutely upon these dances, which are really the only National Dances
of Scotland; all our other dances of ancient or modern times having
been derived by us from France or from England.” (2nd edit., 8vo.,
Edinburgh, 1854.)

Two of the oldest tunes in compound triple time in this
collection are \textit{Old Sir Simon the King} and \textit{Roger a Cauverley}, or
\textit{Coverley}. The first, from the notice by Laneham, was “ancient” in
1575, and the second, from that of Ralph Thoresby, \textit{may} be of still
earlier date. It has already been shown that the latter is entitled a
jig in one book and a Lancashire hornpipe in another.

I attached formerly greater importance than now to the
terminations of tunes
\end{fixedpage}%792
\pagebreak

\setlength{\fixedpagewidth}{360pt}
\begin{fixedpage}%793
\rectoheader

as national characteristics; for, although certain closes may
prevail over others in a nation, it is very difficult to assign an
exclusive right to any. The fashion of the age,—the character of the
words,—the style of the song, have all their influences. A bass voice
will drop a fifth, and it will be one way on one instrument and
another on another. Certain tunes finish on the second of the
key,—others on the fourth; but it is really because they are
\textit{un}finished,—intended to be repeated. Some end on the third and
fifth, from fancy, or from having a monosyllable at the end, like
“Sir,” in \textit{The Baffled Knight} (p. 520). I do not now think that any
rules are to be given which will not be open to many exceptions.

Moore has claimed several airs as Irish, because they have the
repetition, tum-tum, or tum-tum-tum, on the same note at the end,—and
this when even in opposition to all external evidence. There are
undoubtedly many Irish airs that have that termination, but it is by
no means a peculiarity. Although long out of fashion with English
musicians, there are numberless such tunes still to be heard among
the lower orders. It was a common country-dance ending in the reign
of Queen Elizabeth, and remained so till within the last twenty
years. If we look to the earliest Irish tunes, it is not to be
found,—yet English of the same date have it. For instance, in Queen
Elizabeth’s Virginal. Book, there are three Irish airs which, having
never been quoted or printed, I now submit to my readers. The
manuscript contains but three.

1. “\textsc{The Irish Ho-hoane},” No. 26, p. 42 (no arranger’s name):—

2. “\textsc{The Irish Dumpe},” No. 177, p. 297 (no arranger's name) :—

3. “\textsc{Callino casturame},” No. 157, p. 277, arranged by William
Byrd:—

The first two are rather wild music, and have not the marked
rhythm of English popular tunes of the same date; the third, alluded
to by Shakespeare, is as rhythmical as could be desired.

I now give, from the same manuscript, an English country dance
having the terminations which Moore thought to be exclusively Irish.

“\textsc{Woody Cock},” No. 140, p. 259, arranged by Giles Farnaby:—

One example may suffice to prove the case, but there would be no
difficulty in producing fifty more. I believe that an entire volume
of English tunes might be collected
\end{fixedpage}%793
\pagebreak

\setlength{\fixedpagewidth}{360pt}
\begin{fixedpage}%794
\versoheader

with terminations of this kind. The following, is “Oh! the Oak
and the Ash, and the bonny Ivy Tree,’’ from Queen Elizabeth’s
Virginal Book, under the name of \textit{Quodling’s Delight}. The change of
name caused me to overlook it when giving an account of that air; but
it may also be adduced to shew that these endings are not peculiar to
Ireland.

“\textsc{Quodling’s Delight},” No. 113, p. 213, there arranged by Giles
Farnaby:—

I have alluded to the alteration of tunes by collectors of
Scottish music, to make them upon what they call the Scottish scale.
The following is a case in point; for, although Burns thought the
tune of \textit{Ye banks and braes o’ bonny Doon} to have been made by an
amateur, in trying over the black keys of the pianoforte, with the
aid of Stephen Clarke, the English editor of Johnson’s \textit{Scots’ Musical
Museum}, it is clear that nothing more was effected than the
alteration of a note or two, and the transposition of the symphony of
an older song. The following was printed upon halfsheets, and
included in Dale’s Collection of English Songs (i. 157). Dale
commenced printing in 1780, but I cannot give the date of this
publication, because, the collection consisting exclusively of \textit{old}
songs, he made no entry at Stationers’ Hall, as in other cases. It
is, unquestionably, anterior to “Ye banks and braes o’ bonny Doon.”
\end{fixedpage}%794
\pagebreak

\setlength{\fixedpagewidth}{360pt}
\begin{fixedpage}%795
The alteration was, in all probability, made by Stephen Clarke,
without the intervention of any amateur, for to Clarke only can we
attribute the changes in other well-known airs, to fit them for the
\textit{Scots’ Musical Museum}. Little scruple was shewn in making such
changes, for even the well-known country dance and nursery song,
\textit{Polly, put the kettle on}, was transformed into a Scotch tune for the
\textit{Museum} in 1797. This was about three years after \textit{Polly} had become
very popular with young ladies by means of Dale’s variations for the
pianoforte. The words of \textit{Jenny's bawbee} were adapted to it; although,
as they begin—

\settowidth{\versewidth}{“A’ that e’er my Jenny had, my Jenny had, my Jenny had,” }
\begin{scverse}
“A’ that e’er my Jenny had, my Jenny had, my Jenny had,” 
\end{scverse}
they
were evidently intended for the tune of—

\begin{scverse}
“Sike a wife as Willy had, as Willy had, as Willy had,” 
\end{scverse}
which
will be found in N. Thompson's 180 \textit{Loyal Songs}, 1694. Johnson took
the words of \textit{Jenny's bawbee}, with many others, from Herd’s \textit{Scottish
Songs}, 1776, and, not knowing where to find the right tune,
appropriated the first that came to hand. He professed to include
only Scotch poetry, but even this profession was often very slenderly
cloaked. There was a popular song, which had been sung in a London
pantomime:— 

\settowidth{\versewidth}{If a body meet a body going to the Fair,}
\begin{scverse}
\vleftofline{“}If a body meet a body going to the Fair,\\
If a body kiss a body, need a body care?”
\end{scverse}
This was altered for the \textit{Museum}, into—

\begin{scverse}
\vleftofline{“}Gin a body meet a body \textit{comin’ thro’ the rye,}\\
Gin a body kiss a body, need a body \textit{cry}?”
\end{scverse}
The pantomime came out at Christmas, 1795-6, and the alteration
seems to have been made within about nine months of the publication;
for Broderip and Wilkinson’s entry of the original song at
Stationers’ Hall was on the 29th of June, 1796.\textsuperscript{a}

I have no intention of analyzing the collections of Scottish
music; yet, having, in a few cases, reclaimed tunes that many have
supposed to be Scotch, owing to their having been included in these
publications, it becomes incumbent upon me to shew that popularity
only was considered by the collectors, without any care for accuracy.
Indeed, no stronger proof could be produced than that Johnson should
have included new songs by Hook, Berg, Battishill, and other living
composers, and palmed them

\begin{dcfootnote}
\textsuperscript{a} The entry at Stationers’ Hall is as follows:—“If a
body meet a body, sung by Mrs. Henley, at the Royal
Circus, in the favorite new Pantomime called \textit{Harlequin
Mariner}, the music adapted by J. Sanderson, the words
by Mr. Cross." A copy of the song will be found in the
British Museum (G, 367). Mrs. Henley acted the part of
Market Goody in the pantomime. Cross was the author of
a book called \textit{Circusiana}, and of many pantomimes.

\end{dcfootnote}
\end{fixedpage}%795
\pagebreak

\setlength{\fixedpagewidth}{360pt}
\begin{fixedpage}%796
\versoheader

upon his countrymen as Scotch. Thomson, in his \textit{Orpheus
Caledonius}, did nearly the same,—he appropriated tunes by Purcell,
Daniel Purcell, Farmer, and other English composers; also words by
Martin Parker, Tom D’Urfey, Ambrose Philips, and others that he must
have known not to he Scotch,—and Oswald was even more unscrupulous
than either. The Scotch have a large number of beautiful tunes, but
their collections require a thorough sifting, if they are to be
limited to what is really Scotch.

There are other collections of national music, in the formation
of which the intention may have been good, but the industry or
knowledge has not been commensurate. Such are the so-called \textit{Songs of
Ireland without Words}, by J. T. Surenne, of Edinburgh, and the \textit{Dance
Music of Ireland}, by R. M. Levey, of Dublin. Mr. Surenne, being
unwilling to give the Irish the benefit of \textit{The old Langolee} and other
airs which are also claimed by the Scotch, has thrown in a few
English airs, such as Dibdin’s \textit{Cobbler of Castlebury} (without a
particle of Irish character), to make up the balance. Mr. Levey takes
English airs, even to the late George Macfarren’s popular country
dance, \textit{Off she goes}, but evidently without knowing them to be
English. Collections of this kind require greater care than has
commonly been bestowed upon them.

And now as to the sources from whence national music is derived.
Stafford Smith tells us that “all our early melodies, including those
of Scotland, Ireland, and Wales, are no doubt derived from the
minstrels; and that they have sprung from the minstrel practice of
descanting, or singing extempore, on the plain-chant or plain-song of
the church.” Our old fiddlers and pipers certainly took simple
grounds or bases, and formed tunes upon them by making what was
called division (\ie , variation) upon those bases, but I doubt very
much that they were ever derived from the church. There may be
accidental resemblances between their plain-song or ground, and the
plain-song of the church, but the feelings so naturally revolt
against taking sacred music and applying it to secular purposes, that
I have been unable to trace a single instance of a popular air
derived from such a source. Stafford Smith is peculiarly unfortunate
in his proofs, for three of the six airs that he names are not to be
traced further back than the end of the seventeenth century. If any
of the ancient church \textit{hymns} should he found to resemble secular
music, it is, in all probability, because they were originally
secular tunes; for we can trace the clerical practice of writing
hymns to airs sung by minstrels in every century, from the time of
William the Conqueror to the Reformation,—and the system has
continued to the very present time, not only in England, but also
abroad. The minstrels were far in advance of church music, and no
\textit{melody} was to be obtained at that source.

National music may perhaps be divided into two classes,—the first
to consist of tunes made upon bases, and the second of such as were
made without any base at all.

The first class will be most easily discerned in the hornpipes,
jigs, rounds, and tunes of that kind. See for instance, Cheshire and
Shropshire Rounds, p. 599. The earliest instance is \textit{Summer is icumen
in}, p. 21. The tunes in \timesig{9}{4} or \timesig{9}{8} time (like Old Sir Simon the King and
Roger de Coverley) seem to have grown out of the practice of
ornamenting airs which were originally in simple triple time. It was
a frequent charge against the common pipers and fiddlers, that they
“ran too much into division,” and. the commencement of Hale’s
Derbyshire Hornpipe, printed at p. 741, maybe taken as an example of
this “running into division” after the first four bars. In some of
the arrangements of popular airs in Queen Elizabeth’s Virginal Book,
as well as in various manuscripts of lute-music of the sixteenth
century, the
\end{fixedpage}%796
\pagebreak

\setlength{\fixedpagewidth}{360pt}
\begin{fixedpage}%797
\rectoheader

skeletons only of the tunes are taken—the mere chords, grounds,
or bases upon which the tunes were formed. This was for the purpose
of making division more easy, and in such cases it is often
impossible to extract the melody with any certainty. It becomes
necessary to read through the entire composition, and perhaps even
then, the tune may not be obtained as it was usually sung. For that
reason, I have always preferred fiddle copies (where they were to be
had), if the words and tunes were not to be found in union.

The old musicians used to think of their harmonies while they
were making their tunes, as all real musicians do now. Common
fiddlers and pipers perhaps thought more of their bases than of their
tunes, trusting to their facility in making division or variation for
the latter.

The second class of national airs may be called the amateur
music; for, like most of the amateurs of the present day, the authors
made tunes only, and trusted to others to find out fitting harmonies.

Among these are the “wild and irregular melodies,” with which so
many musicians have been puzzled. Great musical knowledge is often
required to harmonize them; but, when properly fitted, some will
repay, by their excellence, all the trouble that they may have
occasioned. Others are quite unsusceptible of good harmony. I should
say that if, after having been placed in the hands of a thorough
musician—one who knows the character of the tunes, as well as all the
resources of harmony—if these tunes still resist all attempts at
making good bases for them, it is because they are thoroughly
worthless, and ought to be thrown aside. The great test of whether a
tune is good or bad is, will it admit of a good base?

And now to conclude. The reader has found in the preceding pages
most ample proofs of the love the English bore to music. They not
only loved it themselves, but believed that even animals took equal
pleasure in it. “As sheepe loveth pyping,” says a writer of the
fourteenth century, “\textit{therefore} shepherdes usyth pipes whan they walk
wyth their sheepe.” “I am verily persuaded,” says Dr. John Case,
“that the ploughman and carter do not so much please themselves with
their whistling, as they
are delighful to their oxen and horses\dots Every
troublesome and laborious
occupation useth musick for a solace and recreation, and hence it
is that wayfaring men solace themselves with songs, and ease the
wearisomness of their journey; considering that musicke, as a
pleasant companion, is unto them insteed of a waggon on the way. And
hence it is that manual labourers, and mechanical artificers of all
sorts keepe such a chaunting and singing in their shoppes—the tailor
on his bulk— the shoemaker at his last—the mason at his wall—the
ship-boy at his oar—the tinker at his pan—and the tiler on the
house-top.” With such a description of England as the above, and the
multitude of passages of similar purport already quoted, the reader
will not doubt the justice of the title given to our land—\textsc{Merrie
England}.

\headingfive{FINIS.}
\end{fixedpage}%797
\pagebreak

\intentionalemptypage%798

\setlength{\fixedpagewidth}{360pt}
\begin{fixedpage}%799

\headingthree{FIRST INDEX.}

\headingfour{BALLADS, SONGS, AND TUNES,}

\headingfive{EITHER PRINTED HEREIN, OR IN ANY WAY REFERRED TO.}

\centerrule
\scriptsizerr
\indent\textit{Lines or titles beginning with “a,” “an,” or “the,” are indexed
under the word that follows.}

\textit{Ballads of which the first lines are not given in the text are
often indexed under the names of the tunes to which they were sung.
The names of the tunes are therefore the surest mode of reference.}

\begin{multicols}{2}
\setlength{\parindent}{0pt}
A la mode de France, 444.

Abbot of Canterbury (The), 348.

Abraham Newland, 720.

Absence of my mistress (The), 452.

Adieu, Dundee, 611.

Adieu, my dear, 214, and note.

Adieu to the last Earl of Derby, 234.

Admiral Benbow (“Come, all you sailors”), 678. 

Admiral Benbow (“O we sail’d”), 641.

Admiral Vernon’s answer to Hosier’s ghost, 597. 

Advice to bachelors, 427.

Advice to the city, 622.

Advice to the ladies of London, 593.

Adzooks! che’s went the other day, 586.

Aged man’s A-B-C, 93.

Agincourt, Agincourt, 39.

Ah, cruel, bloody fate, 279, 280, 777.

Ah, cruel maid, give o’er, 279.

Ah, dear heart, why do yon rise? 772.

Ah, Jenny, gin your eyes, 785.

Ah, the sighs that come fro’ my heart, 57.

Aim not too high, 162, 167.

Air by Dr. Bull, 697.

Alack and alas! she was dumb, 120.

Alas I how should I sing? 765.

Alas I I am in a rage, 585.

Alas! my love, you do me wrong, 230.

Alas! that I came o’er the moor, 772.

All a green willow, 206, 774.

All flowers of the broom, 91, 116.

All hail to the days that merit more praise, 194.

All in a garden green, 110.

All in a misty morning, 146.

All in a pleasant morning, 139.

All in the Downs the fleet was moor’d, 640.

All in the land of Essex, 779.

All in the merry month of May, 540.

All in the month of May, 184.

All you gallants of city or town, 557.

All you that cry O hone, 174.

All you that desirous are, 336.

All you that do desire to know, 435.

All you that in this house, 176.

All you that lay claim to a good fellow’s name, 193. 

All you that love good fellows, 151.

All you that news would hear, 112.

All you that pass along, 457.

All you that to feasting and mirth, 499.

All you young ranting blades, 462.

Ally Croaker, 713.

Although I am a country lass, 306. 375, 376.

Am I mad, O noble Festus? 333, 335. 

Amarillis told her swain, 283, 778.

Amintas on the new-made hay, 284.

Ancient story I’ll tell you (An), 351.

And how should I your true love? 236.

And ne’er be drunk again, 266, 269.

And was not good King Solomon? 72.

And will he not come again? 237 

And yet methinks I love thee, 182.

Anglers’ song, 284,—and another, 446.

Anne Boleyn’s song, 237.

Anything for a quiet life, 378.

Ar ne kuthe ich sorghe non, 25.

Arise and awake, 93.

Arise, arise, my juggy, 142.

Arraignment of the Devil, 175.

Arthur a Bland, 391.

Arthur of Bradley, 539 and 604.

As at noon Dulcina rested, 143.

As Damon late with Chloe sat, 646.

As down in the meadows, 648.

As from Newcastle I did pass, 441.

As I abroad was walking, 139, 190.

As I came down the Highland town, 616\textsuperscript{a}.

As I came from Tottingham, 219.

As I came thorow the North country, 229.

As I came through Sandgate. 722.

As I in a meadow was walking, 528.

As I lay musing all alone, 274.

As I lay musing in my bed, 778.

As I lay musing, myself alone, 775.

As I lay slumbering in a dream, 619.

As I to Ireland did pass, 128.

As I walkt forth of late, 234.

As I walkt forth to take the air, 190, 509.

As I walk’d forth to view the plain, 616\textsuperscript{a}.

As I walk’d over hills, 185.

As I walk’d the woods, 52, 66, 768.

As I was a driving my waggon one day, 691. 

As I was walking all alone, 260, 367.

As I was wand’ring on the way, 558.
\end{multicols}
\end{fixedpage}%799
\pagebreak
\renewcommand\versoheadertext{first index.}

\setlength{\fixedpagewidth}{372pt}
\begin{fixedpage}%800
\versoheader

\scriptsizerr
\begin{multicols}{2}
\setlength{\parindent}{0pt}
As I went forth one morning fair, 610\textsuperscript{a}.

As I went forth one Summer’s day, 502.

As I went forth to view the Spring, 616\textsuperscript{a}.

As I went through the North country, 259.

As I went to Walsingham, 121.

As it fell on a holiday, 68.

As it fell out on a high holiday, 170.

As it fell out one Whitsunday, 776.

As it fell out on a long Summer’s day, 383.

As Mars and Minerva, 730.

As near Portobello lying, 597.

As one bright, sultry Summer’s day, 675.

As our King lay musing, 39.

As Perkin one morning lay musing, 676.

As pretty Polly Oliver lay musing, 676.

As ye came from the Holy Land, 120.

Ash grove (The), 665.

At Athens, in the market-place, 561.

At home would I be in my own country, 457.

At night, by moonlight, on the plain, 686.

At Rome there is a terrible rout, 231.

At the sign of the horse, old Spintext, 602.

At Winchester was a wedding, 496.

At Wednesbury there was a cocking, 660. 

Attend, my masters, and give ear, 189.

Attend thee, go play thee, 223.

Auld lang syne, 621.

Awake and arise, 93.

Awake, awake, O England, 770.

Away from Romford, away, away, 322.

Away with this rebellion, 436.

Aye, marry, and thank ye too, 584.

\bigskip

Bacchus’s health, 119.

Bad husband turned thrifty, 462.

Baffled knight (The), 519, 793.

Bailiff’s daughter of Islington, 203, 204.

Banbury ale, 483.

Banishment of poverty (The), 612.

Bankers are now such brittle ware, 503.

Barbara Allen, 538.

Barber’s news, or Shields in an uproar, 713.

Bark in tempest toss’d, 584.

Barking barber (The), 717.

Barley-break, 135.

Barley-mow (The), 745.

Barrow Foster’s, or Barrow Faustus’s dream, 240, 775.

Bartholomew Fair, 585. 

Basse his career, 255, 776.

Bateman, 197.

Bath medley (The), 654.

Battle of Agineourt, 39, 199.

Battle of birds over Cork, 502.

Be merry, my friends, 306.

Bear a hand, jolly tars, 715.

Bedlam-boys are bonny, 333.

Bed-making (The), 557.

Beggar-boy (The), 269.

Begging we will go (A), 345.

Beginning of the world (The), 69.

Begone, begone, my Willy, 141.

Begone, \textit{old} Care, or \textit{dull} Care, 689.

Bells of Osney, 517.

Benbow, the brother tar’s song, 678.

Bess a Bell (or Bessy Bell) she doth excel, 366, 459. 

Billy was as blythe a lad, 785.

Binny’s jig, 608.

Bishop of Chester’s jig, 177.

Bishop of Hereford’s entertainment, 395.

Bishop’s last good-night (The), 413.

Black-ey’d Susan, 640.

Blacksmith (The), 231.

Blanket Fair, 124.

Blind beggar’s daughter of Bethnal Green, 158, 160. 

Blink over the burn, sweet Betty, 504.

Blithe and bonny country lass (A), 501.

Blithe jockey, young and gay, 612.

Bloody fate, 279.

Blow thy horne, hunter, 58.

Blowzybella, 307, 646.

Blue bells of Ireland (The), 739.

Blue bells of Scotland (The), 739.

Boar’s head in hand bear I (The), 757.

Boat, a boat, haste to the ferry (A), 492.

Boatman (The), 270.

Bobbing Joe or Joan, 290, 354.

Bonniest lass in all the land (The), 228.

Bonny, bonny bird, 555.

Bonny, bonny broom (The), 460.

Bonny briar (The), 460.

Bonny brow (The), 575.

Bonny Dundee, 568\textsuperscript{a}, 611.

Bonny grey-ey’d morn, 610.

Bonny Kathern Oggy, 616\textsuperscript{a}.

Bonny laddy, Highland laddy, 749.

Bonny milkmaid (The), 297.

Bonny Nell, 501.

Bonny Peggy Ramsey, 218.

Bonny, sweet Robin, 284.

Both young men, maids and lads, 136.

Bothwell banks is bonny, 612.

Bow, wow, wow, 717.

Bowes’s tragedy, 372.

Boys and girls, come out to play, 583.

Brause du, Freiheitsang, 691.

Brave boys, 655, 588.

Brave gallants, now listen, 325.

Brave Lord Willoughby, 114.

Brave Monmouth, England’s glory, 523.

Breast knot (The), 681.

Brethren, I must haste away, 749.

Brewer (The), 231.

Bride’s burial (The), 197.

Bridegroom’s salutation (The), 292.

Brighton camp, 708.

Bring us in good ale, 41 to 43.

Bristol waits, 551.

Britannia, rouse, at Heav’n’s command, 687.

British grenadiers (The), 152, 772.

British sailor’s lament, 684.

Britons, strike home, my boys, 729.

Britons, who dare to claim, 705.

Broom of Cowdon Knowes, 459, 613, 783.

Broom on hill, 459.

Broom, the bonny broom, 233, 458, 613, 783. 

Budgeon it is a delicate trade, 666.

Buff coat has no fellow (The), 342.

Bull-running tune, 37.

Burse of Reformation (The), 817.

Busy Fame, 536, 786.

Butter’d pease, 307.

Buttoned smock (The), 774.

By a bank as I lay, 52, 92.

By force I am fixed, 224.

By the border’s side as I did pass, 439.

By the side of a great kitchen fire, 494.

By the side of a neighbouring stream, 494.

\bigskip

Caller herrin', 621.

Callino casturame, 793
\end{multicols}
\end{fixedpage}%800
\pagebreak
\renewcommand\rectoheadertext{ballads, songs, and tunes.}

\setlength{\fixedpagewidth}{360pt}
\begin{fixedpage}%801
\rectoheader

\scriptsizerr
\begin{multicols}{2}
\setlength{\parindent}{0pt}
Cambridge is a merry town, 219.

Cam’st thou not from Newcastle? 339, 779.

Can love be controll’d by advice? 494.

Can nothing, Sir,move you? 715, 788.

Can you dance the shaking of? 85.

Can you not hit it? 239.

Caper and firk it, 541.

Captain bold in Halifax (A), 714.

Captain Ward, 92.

Caput apri defero, 757.

Card dance (The), 448.

Care, away go thou from me, 689.

Care, thou canker of our joys, 722.

Carlton’s epithalamium, 186.

Carman’s whistle (The), 137 to 140, 428.

Case is altered new (The), 279.

Catching of quails (The), 84, and note.

Catholic ballad, 212.

Cavalier’s complaint (The), 358.

Cavalilly, man, 440.

Caveat for young men, 462.

Cease, rude Boreas, 597, 786.

Cease your funning, 664.

C’est l’amour, l’amour, l’amour, 748.

Chanson Roland, 7, 764, 765.

Charles of Sweden, 657.

Charming Billy, 665.

Chaste, pious, prudent Charles the Second, 569. 

Cheerily and merrily, 285.

Cheshire rounds, 598, 796.

Chester waits, 551.

Chevy chase, 45, 47, 196, 198, 200, 773.

Children in the wood, 200, 265.

Chime tune, 283, 778.

Chips of the old block, 431.

Chirping of the lark (The), 396.

Chirping of the nightingale, 627.

Christmas comes but once a-year, 227

Christmas is my name, 464, 783.

Christmas’s lamentation, 463, 783.

City ramble (The), 553.

City’s feast to the Lord Protector, 161.

Clean contrary way (The), 425, 781.

Clear cavalier (The), 447.

Clear is the air, and the morning is fair, 322.

Clock had struck (The), 778.

Clown’s courtship (The), 87.

Cobbler’s end (The), 352.

Cobbler’s hornpipe (The), 594.

Cobbler’s jig (The), 277.

Cobbler’s last will and testament, 451.

Cobbler of Castlebury (The), 796.

Cobbler there was (A), 348, 352.

Cock Lorrell, 161.

Cocky my cary she, 561, 787.

Coffin for King Charles (A), 439.

Colchester waits, 550.

Cold and raw, 306, 307, 309.

Cold’s the wind and wet's the rain, 278.

Colin’s conceits, 242.

Colin’s complaint, 493.

College hornpipe (The), 740.

Come, all that love good company, 344.

Come, all you caballers, 441.

Come, all you farmers, out of the country, 327. 

Come, all you maidens fair, 524.

Come, all you noble, 327.

Come, all you sailors bold, lend an ear, 678.

Come, all you young blades, 735.

Come, all you young maids as live at a distance, 782.

Come, and do not musing stand, 305.

Come, and listen to my ditty, 597, 786.

Come away, and do not stay, 463.

Come, bachelors and married men, 341.

Come, boys, fill us a bumper, 528.

Come, bring with a noise, 773.

Come, buy my greens and flowers fine, 551.

Come, cheer up, my lads, 716, 788.

Come, come away to the Temple, 436.

Come, come, beloved Londoners, 442.

Come, companions, join your voices, 578.

Come, faith, since I'm parting, 288, 778.

Come, follow, follow me, 272, 273.

Come, gallant Vernon, come, and prove, 658. 

Come, give us a brimmer, 426.

Come, here’s to Robin Hood, 398.

Come hider, love, to me, 34.

Come hither, attend to my ditty, 553.

Come hither, friends, and listen unto me, 430. 

Come, hostess, fill the pot, 365.

Come, Jack, let’s drink a pot of ale, 358.

Come, jolly Bacchus, 657.

Come, landlord, fill a flowing bowl, 670.

Come, lasses and lads, 531.

Come, let’s drink a health to George, our King, 658. 

Come, let us drink a bout, 670.

Come, let us prepare, 664.

Come, listen awhile, tho’ the weather is cold, 124. 

Come, live with me, 213, 214, 215.

Come, loyal Britons, all rejoice, 657.

Come, mad boys, be glad, 284.

Come, my hearts of geld, 269.

Come, my Molly, let’s be jolly, 336.

Come, neighbours, and listen awhile, 540, 604. 

Come o’er the bourn, Bessy, to me, 505\textsuperscript{a}.

Come, open the door, 504.

Come, Robin, Ralph, and little Harry, 64.

Come, shepherds, deck your heads, 261.

Come, sons of Summer, by whose toil, 581.

Come, sweet lass, 600.

Come, sweet love, let sorrow cease, 240, 775. 

Come to the court, and be all made knights, 327. 

Come, Tom, foot it new, 442. 

Come up to my window, 141.

Come, ye merry men all, of Watermen’s Hall, 427. 

Come, ye young men, come along, 126.

Come you not from Newcastle? 339, 779.

Cornin’ thro’ the rye, 795.

Common cries of London town (The), 219. 

Complain, my lute, complain, 210, note.

Conceits of sundry sorts there were, 242. 

Concinamus, O sodales, 578.

Confess, or The court lady, 361.

Conspiracy (The), 455.

Constance of Cleveland, 178, 772.

Constant lover (The), 657.

Constant Penelope, 372.

Cook Lawrell, 160.

Cooper of Norfolk (The), 774.

Coridon and Parthenia, 537.

Corn grinds well (The), 750.

Corn riggs are bonny, 610.

Cotsall shepherds (The), 284.

Council grave our King did hold (A), 39, 199. 

Country bumpkin (A), 659.

Country courtship, 671.

Country farmer and buxom virgin, 562.

Country farmer’s vain glory, 582.

Country garden (The), 652.

Country gentleman came up to town (A), 593.
\end{multicols}
\end{fixedpage}%801
\pagebreak


\setlength{\fixedpagewidth}{390pt}
\begin{fixedpage}%802
\versoheader

\scriptsizerr
\begin{multicols}{2}
\setlength{\parindent}{0pt}
Country hostess’s vindication, 428.

Country lass (The), 806, 375, 647.

Country maiden’s lamentation (The), 593. 

Countryman’s delight (The), 542.

Countryman’s new care-away, 304 

Countryman’s ramble through Bartholomew Fair, 585. 

Countrymen of England, 292. 

Courage crown’d with conquest, 276.

Courageous Betty of Chick Lane, 571.

Courageous soldiers of the West, 571.

Court lady (The), 361.

Courteous health, or Merry boys, 528.

Courtiers, courtiers, think not, 606.

Courtiers scorn us country clowns, 235.

Coy shepherdess (The), 284.

Crafty cracks of East Smithfleld, 551.

Cramp is in my purse (The), 89.

Creditors’ complaint against bankers, 503.

Cries of London (The), 551.

Crimson velvet, 179, 319, 772.

Cripple (The), 158.

Crossed couple (The), 325.

Crown’s far too weighty for shoulders of eighty, 622. 

Cruel black (The), 197.

Cruel, bloody fate, 279.

Cruel parents (The), 190.

Cruel shrew, or Patient man’s woe, 341.

Cruelty of Barbara Allen. 538.

Cruiskeen lawn, 770.

Crums of comfort for the youngest sister, 203. 

Cuckolds all a row, 340.

Cuckoo’s commendation (The), 774.

Cumberland laddy, or Willy and Nelly, 503. 

Cumberland lass (The), 504,

Cumberland Nelly, or North country lovers, 503. 

Cunning Northern beggar, 333.

Cup of old stingo (A), 305, 308.

Cuper’s garden, 727.

Cupid’s conquest, 190.

Cupid’s courtesy, 363, 780.

Cupid’s garden, 727, 735.

Cupid’s revenge, 591.

Cupid’s trepan, 555.

Cushion dance, 153, 156.

Cyclops (The), 431.

\bigskip
Daddy Neptune one day,. 721.

Dainty, come thou to me, 517.

Dal, derra rara, dal dara, 553.

Damask rose, or Lily fair, 780.

Dame, lend me a loaf, 483.

Damsel, I’m told (A), 555.

Dance after my pipe, 84, 164, 182.

Dance of Death, 84, 164, 182.

Dance tune, about 1300, 27.

Daphne, or The shepherdess, 338.

Dargason, 62, 64, 627.

Date obolum Belisario, 717.

Death and the cobbler, 348.

Death and the exciseman, 165.

Death and the lady, 164, 772.

Deel assist the plotting Whigs, 609.

Defence of Hyde Park, 327.

Defence for milkmaids, 349.

Deil’s awa wi’ the exciseman (The), 312.

Delights and the pleasures of a man without care, 498.

Delights of the bottle (The), 498. 

Deptford plum cake (The), 551.

Derbyshire hornpipe (The), 741, 796.

Derbyshire miller (The), 750.

Derry down, 348, 677.

Despairing beside a clear stream, 493.

Deuks dang o’er my daddie, 344.

Devil’s progress (The), 442.

Dialogue between Jack and his mother, 551, 786. 

Dialogue between Tom and Dick, 379.

Diana and her darlings dear, 371.

Did you not hear of a gallant sailor? 452. 

Disloyal favorite, or Unfortunate statesman, 619. 

Distracted Puritan (The), 332. 

Distracted sailor (The), 597.

Dives and Lazarus, 117.

Do, do, nightingale, syng full mery, 765.

Do me no harm, good man, 208, 774.

Dr. Faustus, 162.

Dominion on the sword (The), 431.

Donkin Dargason, 64.

Doubting virgin (The), 557.

Down among the dead men, 643.

Down, down with political fools, 456.

Down in the North country, 279, 281, 777.

Down with the Whigs, we’ll now grow wise, 427. 

Downfall of the Mitre at Cambridge, 446,

Draw near, you country girls, 254.

Dreaded hour, my dear love (The), 605.

Drink, laugh, sing, boys, 344.

Drink to-day, and drown all sorrow, 670,

Drink to me only with thine eyes, 707.

Drive the cold Winter away, 193.

Dublin’s deliverance, or The surrender of Drogheda, 571.

Duchess of Suffolk’s calamity, 371. 

Duke of Norfolk (The), 117.

Duke of York’s delight, 313.

Dulce Domum, 576.

Dulcina, 143, 771.

Dull Sir John, 628.

Dumb maid (The), or Dumb, dumb, dumb, 120, 457. 

Durham stable, 317.

Dusky night rides down the sky, 650.

Dusty miller (The), 608.

\bigskip
Early one morning, just as the sun, 735.

Eighth Henry ruling in this land, 144. 

Eighty-eight (tune of 1588), 211.

Engelsche Klocke Dauns, 283.

England’s darling [Duke of Monmouth], 523. 

England’s honour and London’s glory, 430. 

England’s valour and Holland’s terror, 778. 

English courage, 657. 

Entered apprentice’s song (The), 664.

Essex’s lamentation, 176, 310.

Essex’s last good-night, 174.

Escape of the King of Scots [Charles II.], 435.

Ev’ry man take his glass in his hand, 674. 

Exciseman and Death, 165.

Excuse me, 343.

\bigskip
Fa, la, la, la, la, la, la, la, la (The waits), 549.

Fa, la, lanky down dilly, 274 and 276.

Fading, 235.

Fain I would, if I could, 439, 628.

Fain would I be in the North country, 460.

Fain would I have, 91.

Fair angel of England, 319.

Fair Hebe I left, 676.

Fair lady, lay those costly robes aside, 164.

Fair maid, are you walking? 223.

Fair maid of Bristol (The), 197.

Fair maid of Doncaster (The), 559.

Fair maid of Dunsmore, 371.
\end{multicols}
\end{fixedpage}%802
\pagebreak

\setlength{\fixedpagewidth}{372pt}
\begin{fixedpage}%803
\rectoheader

\scriptsizerr
\begin{multicols}{2}
\setlength{\parindent}{0pt}
Fair maid of Islington, or London vintner, 541, 

Fair maid of London (The), 234.

Fair Margaret and sweet William, 382, 383.

Fair one, if thus kind you be, 257.

Fair one let me in (The), 427, 509, 537.

Fair Rosalind in woeful wise, 717.

Fair Rosamond, 199.

Fairest Jenny, 617, 787.

Fairest mistress, cease your moan, 259.

Fairest nymph the valleys (The), 319.

Fairing for young men and maids, 496.

Faithftul brothers (The), 157.

Faithful lover’s downfall, 280.

Faithful lover’s farewell, 528.

Faithful lover’s resolution (The), 378.

Faithful lovers of the West, 191.

Falconer’s hunting (The), 256.

Fancy’s freedom, 284.

Farewell, both hawk and hind, 271.

Farewell, farewell, my dearest, 367.

Farewell, Manchester, 682.

Farewell, my heart’s delight, 369.

Farewell, rewards and fairies, 182.

Farewell to Gravesend, 428.

Farewell to the woodlands, 725.

Farewell to you, ye fine Spanish ladies, 736. 

Farmer’s daughter of merry Wakefield, 279. 

Farmer’s son (The), 656. 

Father Paul, 650.

Father’s wholesome admonition, 493.

Farthingale reviv’d, or More work for the cooper, 787.

Female quarrel (The), 366. 

Fetching home of May (The), 322.

Fiddlers must be whipt (The), 426.

Fie, nay, prithee, John, 565.

Fife and a’ the lands about it, 617.

Fifteenth day of July, 115.

Fight on, brave soldiers, for the cause, 425.

Fill, fill, fill the glass, 683.

Fire on London Bridge, 199.

First Nowel (The), 756.

Fit’s upon me now (The), 176.

Flora Macdonald, 681.

Fly, Brass, 779.

Fly, merry news, 196.

Flying Fame, 198.

Fond, wanton youth makes love a god, 371.

For bonny, sweet Robin is all my joy, 233.

For that’s the time o’ day, 723.

Forego me now, come to me soon, 771.

Forlorn lover’s lament, 461.

Forth from my sad and darksome cell, 329, 330. 

Forth from the elysian fields, 333.

Fortune, my foe, 162.

Four and-twenty lasses went, 83.

Four merry wives (The), 551.

Four pence-halfpenny-farthing, 366, 367.

Fowls in the frith, 25.

Franklin is fled away, 369, 613, 780.

Freemason’s tune (The), 663.

French levalto, 169.

Friar and the nun, 145.

Friar Bacon walks again, 443.

Friar in the well, 274, 390.

Friar of orders gray, 236.

Frisky Jenny, 657.

Frog and the mouse (The), 88.

Frog he would a wooing go (A), 88.

Frog galliard (The), 127.

From ancient pedigree, 269.

From Cornwall mount to London fair, 315.

From councils of six, 441.

From hunger and cold, 124.

From Oberon in Fairy land, 143.

From the hag and hungry goblin, 332.

From the top of high Caucasus, 332.

Frozen-out gardeners, 747.

\bigskip
Gallant Grahames (The), 612.

Galliard, 153.

Gang (The), or The nine worthies, 391.

Garland (The), 254.

Garter, or King James’s march, 574.

Gather ye rosebuds, 362, 780.

Gathering peascods, 258, 627.

Gaveth me no garland of greene, 765.

Gee ho, Dobbin, 690.

Generous lover (The), 510.

Gently is the fair stream flowing, 453.

George Barnwell, 381.

Gillian of Croydon, 289, 778.

Gin a body meet a body, 795.

Gipsies’round (The), 171, 772.

Girl I left behind me (The), 708.

Girls and boys, come out to play, 584.

Give ear to a frolicsome ditty, 553, 554.

Give me my yellow hose, 218, 220.

Give that wreath to me, 683.

Give us once a drink, gentle butler, 745.

Glancing of her apron, 575.

Glorious first of August (The), 657.

Glory of the North (The), 442,

Glory of the West (The), 444.

Go from my window, 140, 142.

Go home in the morning early, 306.

Go no more a-rushing, 158.

Go to, go to, you Britons all, 144.

God above, for man’s delight, 513.

God bless you, merry gentlemen, 754.

God prosper long our noble King, 199, 773.

God save great James, 699.

God save King Henrie, 698.

God save our Lord, the King, 704.

God save the King (Dr. Bull's piece, so called), 696. 

God save the Queen, 691 to 706, 772.

God send us a happy new year, 232.

God speed the plough, 118.

Goddesses, 456.

Golden days of good Queen Bess (The), 713. 

Golden age (The), 266, 774.

Golden slumbers kiss your eyes, 587.

Gone is Elizabeth, 182.

Good ale for my money, 306.

Good fellow’s consideration (The), 427.

Good fellow’s frolic (The), 427.

Good fellows great and small, 254.

Good fellows mnst go learn to dance, 243.

Good King Cole, and he call’d for his bowl, 634. 

Good-morrow, fair Nancie, 270.

Good-morrow, gossip Joan, 673.

Good-morrow, my neighbours all, 462. 

Good-morrow! ’tis St. Valentine’s Day, 142, 144, 227.

Good old cause (The), 425, 781.

Good Symon, how comes it? 265.

Good, your worship, bestow one token, 269. 

Gosport tragedy (The), 614.

Gowlin (The), 219.

Grammercy, penny, 356, 780.

Gray’s Inn mask, 328.

Great Charles, your English seamen, 292.
\end{multicols}
\end{fixedpage}%803
\pagebreak

\setlength{\fixedpagewidth}{372pt}
\begin{fixedpage}%804
\versoheader

\scriptsizerr
\begin{multicols}{2}
\setlength{\parindent}{0pt}
Great earthquake (The), 770.

Great Lord Frog to Lady Mouse, 561, 786.

Green gown (The), 323, 324.

Green sleeves, 116, 227, 228, 230 to 233.

Green sleeves and pudding pies, 232, 775. 

Greenwich Park, 600. 

Greenwood, 66.

Grim king of the ghosts, 493.

Guardian angels, now protect me, 748.

Guy Fawkes, 717.

Guy of Warwick, 171, 772.

\bigskip
Hag is astride (The), 773.

Hail to thee in the crown of victory, 691.

Hail to the myrtle shades, 493.

Half Hannikin, 73.

Hang sorrow! let’s cast away care, 777.

Hanging tune (The), 163.

Hankin booby, or Hanskin, 73.

Happy clown (The), 675.

Happy Dick, 529, 672.

Happy miller (The), 589.

Hard fate that I should banisht be, 783.

Hardy tars of old England (The), 716.

Hark! how the cries in every street, 565, 566. 

Hark! the bonny Christ-church bells, 565.

Hark! to Winchester, 684.

Harvest home, 583.

Hasty bridegroom (The), 776.

Hathersage cocking (The), 659.

Have among you, good women, 777.

Have at thy coat, old woman, 365.

Have good day, my leman, 765.

Have mercie on me, frere, barefoot that I go, 765. 

Have with you to Walsingham, 121.

Have you any work for the tinker? 187.

Have you heard of a lord of noble descent? 441. 

Have you heard of a frolicsome ditty? 553. 

Hawthorn tree (Merry ballet of), 64. 

He never will come again, 237.

He that a tinker will be, 187.

He that has a little tiny wit, 225.

He that is a clear cavalier, 448.

He that loves sack, doth nothing lack, 528.

He that will not merry, merry be, 588.

Health to Betty, 259, 366.

Health to all honest men, 673.

Healths (The), 288, 778.

Heard you not lately of Hugh? 266.

Heard you not of a valiant trooper? 452. 

Heart’s-ease, 210. 

Hearts of oak, 715, 788.

Heave and ho, rumbelow, 783.

Heigh, ho, 159. (See also \textit{Hey}, ho.)

Heil dir im Sieger Kranz! 691.

Hemp-dresser (The), 312.

Henry, our royal King, 169.

Henry V. at Agineourt, 200.

Henry the Eighth’s song, 56.

Here’s a crew of jovial blades, 427.

Here’s a health to all honest men, 674.

Here’s a health to great Eugene, 551.

Here’s a health to Father Paul, 650.

Here’s a health to jolly Bacchus, 119.

Here’s a health to the barley-mow, 745.

Here’s a health to the King in sack, 417.

Here’s a health to the mem’ry of Queen Ann, 643. 

Here's a health to the Queen, and a lasting peace, 643.

Here’s a health unto his Majesty, 492. 

Here’s a health unto our master, 667.

Here’s to the maiden, 744.

Here must I tell the praise, 517.

Here we go round the mulberry-bush, 719.

Hero and Leander, 378.

Hewson’s lamentation, 451.

Hey, boys, up go we, 425, 428, 781.

Hey, brave Arthur o’ Bradley, 540.

Hey ding, a ding, 263, 776.

Hey down, a down, 391.

Hey down derry, 770.

Hey for a lusty, lively lad, 446.

Hey for Christmas, 64.

Hey for our town, or A fig for Zommersetshire, 513.

Hey, ho, frisca jolly, 53.

Hey, ho! my heart is full of woe, 210.

Hey, ho, my honey, 292, 462.

Hey, ho, the cramp, a, 89.

Hey, how the chevaldoures woke, 765.

Hey, jolly Jenkin, 483, 784.

Hey! my kitten, my kitten, 603.

Hey, then, up go we, 425, 781.

Highlanders’ march (The), 784.

His head as white as milk, 237.

Hit or miss, 147.

Ho! brother Teague, dost hear de decree? 572, 787.

Hobby-horse dance, 601.

Hodge of the mill and buxom Nell, 786.

Home in the morning early, 306.

Honest age (The), 774.

Honest Sir, give me thy hand, 672.

Honest wooer (The), 259.

Honi soit que mal y pense, 286 and 287.

Honour invites you to delights, 327.

Honour of great York and Albany, 503.

Hooky, hooky, we have shorn, 580.

Hoop! do me no harm, 208, 774.

Hosier’s ghost, 597.

Hounds are all out (The), 556.

How blyth ilk morn was I to see, 461.

How can you, lovely Nancy? 788.

How happy are young lovers, 597.

How happy could I be with either, 553.

How happy’s the mortal that lives by his mill. 589.

How now, shepherd? 377. 

How pleasant is it, 258.

How should I with that old man? 765.

How should I your true love know? 236.

How stands the glass around? 669.

How unhappy is Phillis in love, 455.

Hubert’s ghost, 776.

Huggle-duggle, ho, ho, ho, 442.

Humours of the bath (The), 654.

Hunt is up (The), 53, 60 to 62.

Hunter in his career (The), 254 and 255, 776. 

Hunting the hare, 255, 320, 321, 322.

Hunting we will go (A), 346, 651.

Huntsman (The), 66.

Hyde Park, 325, 326, 628.

Hymn to the gentle craft, 451.

\bigskip
I am a jolly huntsman, 346.

I am a jovial cobbler, bold and brave, 589.

I am a lass o’ th’ North country, 503.

I am a lusty beggar, 333.

I am a lusty, lively lad, 447.

I am a poor man, God knows, 267.

I am a poor shepherd undone, 462.

I am a rogue, and a stout one, 779.

I am a senseless thing, 118.

I am a young man that do follow the plough, 556.
\end{multicols}
\end{fixedpage}%804
\pagebreak


\setlength{\fixedpagewidth}{384pt}
\begin{fixedpage}%805
\rectoheader

\scriptsizerr
\begin{multicols}{2}
\setlength{\parindent}{0pt}
I am so deep in love, 183, 364.

I am the Duke of Norfolk, 117, 770.

I am the young lass of Lynn, 584.

I am woe for their wolves so wild, 768.

I cannot come every day to woo, 87, 90, 622.

I cannot eat but little meat, 72.

I fancy none but thee alone, 143.

I had a pretty girl, and a tenant of my own, 595. 

I have a good wife at home, 203.

I have but a mark a year, 356.

I have house and land in Kent, 90.

I have left the world as the world found me, 555. 

I live in the town of Lynn, 584.

I live not where I love, 451, 782.

I lo’e na a laddie but ane, 529.

I lothe that I did love, 216.

I love my love for love again, 776.

I love sixpence, a jolly, jolly sixpence, 738.

I’ll bark against the dog-star, 334.

I’ll go no more to the New Exchange, 317, 318. 

I’ll ne’er be drunk again, 266, 269.

I'll never love thee more, 378, 507.

I’ll show you the captains of Aubrey Vere, 506. 

I’ll tell you an old song, 300.

I’ll tie my mare on thy ground, 738.

I’m a bold recruiting sergeant, 596.

I’m like a skiff on ocean toss’d, 675.

I'm lonesome since I left Blyth camps, 708,

I’m plundered of all my gold, 551.

I made love to Kate, 723.

I married a wife of late, 298.

I marvel, Dick, that having been, 358.

I ne'er yet saw a lovely creature, 291.

I often for my Jenny strove, 591.

I often have known, 194.

I pray, Mr. Jacobite, tell me why, 276.

I pray now attend to this ditty, 553.

I prithee, love, turn to me, 526, 528, 613.

I prithee, Portsmouth, tell me plain, 325\textsuperscript{a}.

I read in ancient times, 354.

I rode from England into France, 588.

I’se go with thee, my sweet Peggy, 306.

I sing a doleful tragedy, 717.

I sing of no heretic, 124.

I sing of sad discords that happen’d of late, 630.

I sing the British seaman’s praise, 654.

I sowed the seeds of love, 521, 522, 735.

I tell thee, Dick, where I have been, 358, 360.

I tell thee, Jack, thou gav’st the King, 359.

I took thee on a sudden, 292.

I will away, and will not stay, 612, 465.

I will never love thee more, 378.

I wish I were those gloves, dear heart, 378.

I would I were in my own country, 456, 782.

If a body meet a body, 795

If all the world and love were young, 214.

If all the world my mind did know, 528.

If all the world were paper, 627.

If Death would come, 182.

If ever I marry, I’ll marry a maid, 96.

If Rosamond, that was so fair, 215.

If the heart of a man be deprest, 639.

If there were employments, 304.

If you want a yonng man with a true, 691. 

Ignoramus justice, 276. 

Ill May-day, 176.

Illustrious George shall come, 436.

In a freshe morninge, among the Flowrys, 775.

In a humour I was of late, 264, 265, 776.

In a Winter’s morning (York waits), 549.

In ancient times, when as plain dealing, 355.

In coming down to Manchester, 734.

In Crete, when Dedimus, 117.

In Cumberland there dwells a maid, 504.

In dole and deep distress, 201.

In eighty-eight, ere I was born, 212.

In England once there reign’d, 144.

In good King Charles’s golden days, 653.

In Gosport of late there a damsel did dwell, 614. 

In January last, 524, 575.

In Kent, I hear, there lately did dwell, 602.

In kingly Stephen’s reign, 381.

In Lime began a rebellion, 496.

In my triumphant chariot, 332.

In Nottingham lives a jolly tanner, 392.

In Nottinghamshire let them boast, 391.

In Paul’s Church-yard in London, 333.

In pescod time, 196.

In Popish times, when bishops proud, 349.

In praise of a dairy, 124.

In Reading town, where I was born. 538 

In Rome there is a most fearful rout, 231, note.

In sad and ashy weeds, 201.

In Scarlet town, where I was born, 538.

In Summer time, when leaves grow green, 392, 393, 541.

In Taunton Dean ich were bore and bred, 262, 570, 672.

In the days of old, 179, 773. 

In the gallant month of June, 143.

In the season of the year, 733.

In the Spring time of the year, 732.

In times of yore, sure men did doat, 357.

In Wakefield there lives a jolly pinder, 394.

In Warwickshire there stands a down, 371.

In Winter time, when flow’rs do fade, 544. 

Inconveniences by marriage, 371.

Inconstancy of the world, 310.

Indian weed, withered quite, 563.

Irish Dumpe (The), 793.

Irish lass’s letter to Teague, 307.

Irish Och hone (The), 793.

Iron chest is still the best (An), 503.

Is there ever a man in all Scotland? 776.

It befell at Martinmas, 226.

It’s better late thrive than never, 780.

It’s Nanny O, it’s Nanny O, 610\textsuperscript{a}.

It is not your Northern Nanny, 354, 779.

It is not yet a fortnight since, 501.

It was a blind beggar, 159, 160.

It was a country lad, 219.

It was a frog in the well, 88.

It was a knight was drunk with wine, 520.

It was a lover and his lass, 204, 774.

It was a maid of my country, 65.

It was a rich merchantman, 381.

It was a young maiden truly, 734.

It was a youthful knight, 179.

It was at the birth of a Winter’s mom, 391. 

Italian nymphs and swains, 596.

\bigskip
Jack's the lad, 741.

Jack met his mother all alone, 552, 786.

Jacobite toss’d in a blanket, 276.

Jamaica, 446.

James the Second’s march, 574.

Jane Shore, 319.

Jemmy and Anthony, 786.

Jenny’s answer to Sawney, 618.

Jenny, gin your eyes, 785.

Jenny’s bawbee, 795.
\end{multicols}
\end{fixedpage}%805
\pagebreak

\setlength{\fixedpagewidth}{414pt}
\begin{fixedpage}%806
\versoheader

\scriptsizerr
\begin{multicols}{2}
\setlength{\parindent}{0pt}
Joan, come kiss me now, 483.

Joan is as good as my lady, 315.

Joan’s ale is new, 187.

Joan’s placket is torn, 518.

Joan stoop’d down, 312.

Joan to the May-pole, 302, 778.

Jock o’ Hazledean, 575.

Jockey and Jenny, 617.

Jockey, away, man! 619\textsuperscript{a}.

Jockey is grown a gentleman, 784.

Jockey rous’d with love, 610.

Jockey’s escape from Dundee, 611.

Jockey to the fair, 712.

Jockey was a dowdy lad, 612\textsuperscript{a}.

Jog on the footpath way, 211.

John Anderson, my jo, 770.

John, come kiss me now, 122, 147, 218, 660, 771. 

John Dory, 67, 359\textsuperscript{a}.

John for the King, 107, 770.

John Had-land’s advice, 460.

John’s earnest request to Betty, 504.

Johnny and Jenny, 774.

Johnny Armstrong, 260, 776, 538.

Johnny, cock thy beaver, 784.

Jolly fellow (The), 670.

Jolly gentleman’s frolic, or City ramble, 553. 

Jolly miller (The), 666.

Jolly Roger Twangdillo, of Plowden Hall, 602. 

Jolly sailor’s resolution (The), 590.

Jolly tradesmen (The), 446.

Jovial cobbler (The), 292, 589.

Jovial beggars (The), 345.

Jovial companions, or Merry travellers, 506. 

Jovial tinker (The), 187, 773, 779.

Joy and sorrow mixt together, 777.

Jubilee (The), or Coronation-day, 436.

\bigskip
Katherine Ogie, 616\textsuperscript{a},

Keel row (The), 721.

Keep a good tongue in your head, 297, 298. 

Kent-street club, 427. 

Kentish hop-pickers’ song, 304.

Keppel's triumph, 715.

Kind husband and imperious wife, 455.

King and Lord Abbot, 350.

King and the bishop, 199.

King Alfred and the shepherd's wife, 199.

King Arthur’s noble acts, 199.

King Charles’s progress [Charles II.], 435.

King Cophetua and the beggar, 591.

King Edward IV. and the tanner of Tamworth, 392, 541.

King Edward, King Edward, 698. 

King Edward wooing the fair maid of London, 234. 

King enjoys his own again (The), 436, 782. 

King Henry II. crowning his son, 242. 

King James’s march, 574.

King John and the Abbot of Canterbury, 350, 630. 

King Lear and his three daughters, 199.

King of Scots and Andrew Browne, 185.

King Olfrey and the old abbot, 185.

King once reigned beyond the seas (A), 591. 

King’s ballad (The), 56.

King’s complaint (The), 439.

King’s delight (The), 257.

King’s hunt-is-up (The), 60.

King’s jig (The), 495, 784.

King’s last good-night (The), 174.

King’s note (The), 34.

King shall enjoy his own (The), 434,782.

King William III., and Queen Mary (Proclamation of), 493.

Knitter’s job (The), 367.

Knot (The), 313.

Know this, my brethren, 428.

\bigskip
Lachrymaa, 92.

Ladies, I think you marvel, 91.

Ladies of London, both wealthy and fair, 593. 

Lady Carey’s dump, 53.

Lady Catherine Ogle, 616\textsuperscript{a}.

Lady Frances Nevill’s delight, 398.

Lady Green Sleeves, 227.

Lady Isabella’s tragedy, 197, 200.

Lady, lie near me, 184.

Lady Marquess (The), and her gang, 781.

Lady’s daughter of Paris properly, 612.

Lady’s fall (The), 148, 196.

Lady Winkfield’s round, 53.

Lament, each one, the blazing fire, 117.

Lament for Walsingham, 122.

Lament, lament, for he is dead, 174.

Lamentation from Rome, 112.

Lamentation of Lord Essex, 176.

Lamentacione of the Kynge of Scots, 775. 

Lancashire hornpipes, 240, 534, 544.

Lancashire lovers (The), 457.

Landlord, he looks very big (The), 661.

Lanthorn for landlords, 118.

Lass of Cumberland (The), 503.

Lass there lives upon the green (A), 685.

Lass that comes to bed to me (The), 503.

Last Sunday, by chance, 602.

Last time I came o’er the moor (The), 772. 

Laugh and lay down, 690.

Law lies bleeding, 431.

Lawyers’ lamentation for Charing Cross, 433.

Lay by your pleading, 431.

Lay by your reason, truly out of season, 782.

Lay the bent to the bonny broom, 530.

Leander’s love to Hero, 367.

Leather apron, 541.

Leather bottèl (The), 513.

Leave lightie love, ladies, 224.

Legend of Sir Guy, 171.

Leicestershire lovers (The), 182.

Lenton stuff is come to town, 89.

Lesley’s march, 615\textsuperscript{c}.

Let every man with tongue and pen, 436.

Let Oliver now be forgotten, 455, 783.

Let Protestants freely allow, 493.

Let’s cast away care and merrily sing, 161.

Let’s sing to the memory of glorious Queen Bess, 568.

Let the cannons roar from shore to shore, 503.

Let the toast pass, 744.

Let us advance the good old cause, 425.

Let ns drink and sing, 292.

Liberty Hall, 677.

Lie lulling beyond thee, 259.

Lie still, my dear, 173.

Life and death of the Duke of Buckingham, 215. 

Life and death of Queen Elizabeth, 197.

Life and death of Mrs. Laurie’s dog, 612.

Life of love (The), 537.

Light o’ love, 221.

Lilliburlero, 568, 786.

Lincolnshire poacher (The), 732.

List, lusty gallants, 92.

Listen awhile to what I shall say, 451.

Little barley-com (The), 305.

Little Musgrave and Lady Barnard, 170.

Little Pegge of Ramsie, 218.
\end{multicols}
\end{fixedpage}%806
\pagebreak

\setlength{\fixedpagewidth}{360pt}
\begin{fixedpage}%807
\rectoheader

\begin{dcindex}
Live, live with me, 214.

Lo! He comes in clouds descending, 748.

Logan water, 620, note.

London gentlewoman (The), 312.

London is a fine town, 219, 774.

London ladies, 592.

London lass’s lamentation, 584.

London prentice (The), 151, 265.

Ditto (another tune), 146.

London’s true character, 340.

London waits, 550.

Long ere the morn, 256.

Long have I liv’d a bachelor’s life, 461.

Long have we been with balladry oppress’d, 632. 

Longing shepherdess (The), 184.

Look up, my lords, 121.

Looking glass for ladies, 371.

Lord Essex’s last good-night, 174.

Lord of Lorne and the false steward, 228.

Lord Lovel, 383.

Lord Thomas he was a bold forester, 145.

Lord Wigmore, 371.

Lord Willoughby’s welcome home, 115, 770. 

Lost is my quiet for ever, 794.

Loth to depart, 173, 708, 772.

Love and innocence, 673.

Love in the blossom, 284.

Love lies bleeding, 431, 782.

Love me little, love me long, 512.

Love’s constancy, 313.

Love’s lunacy, 334.

Love’s solace, 779.

Love-sick maid of Wapping (The), 509.

Love-sick serving-man (The), 592.

Love’s tide, 457.

Love will find out the way, 303.

Lover’s mistake (The), 494.

Lovely Nancy, 715, 787.

Lovely Northern lass (The), 459.

Loyal Feast at Haberdashers’ Hall, 619.

Loyal lovers (The), 284.

Loyal subjects’ exultation (The), 436.

Loyalty respected, and faction confounded, 503. 

Lucina, 270, 285. 

Lull me beyond thee, 259.

Lulling beyond thee, 259.

Lusty friar of Flanders (The), 307.

Lusty gallant, 91, 769.

\bigskip
Macbeth, 614.

Mad Besse’s vagary, 334.

Madman’s Morrice, 334.

Mad Moll, 603.

Mad Robin, 512.

Mad Tom, 330, 778.

Maggie Lawder, 620, note.

Maid peept out of the window, 274.

Maiden fair, 257.

Maiden, fresh as a rose, 792.

Maiden’s joy (The), 117, note.

Maiden’s sad complaint for want of a husband, 462. 

Maid’s answer to the knight’s three questions, 530. 

Make room for an honest red-coat, 334.

Mall Peatly, 298, 778.

Mall Sims, 177, 789.

Malt’s come down, 74.

Man had three sons (A), 189.

Man of Kent (The), 566.

Manchester angel (The), 734.

March, boys, 536.

\hangpara{2em}{1} March, \textit{said} to have been played when Mary, Queen of Scots was
proceeding to execution, 519. 

\hangpara{2em}{1} Margaret’s ghost (on which Mallet founded his ballad), 383.

Marigold (The), on Queen Mary, 178.

Mark! how the blushful morn, 414, 781.

Mark Noble’s frolic, 553.

Mark well my heavy, doleful tale, 197.

Married man's bondage (The), 378.

Married man’s lamentation, 427.

Marriage of the frog and the mouse, 88.

Marriage, or The mouse-trap, 602.

Marry me, quoth the bonny lass, 435.

Mars shall to Cupid now submit, 355.

Martin said to his man, 76.

Mary Ambree, 159.

Mary doth complain, 178.

Mary’s ghost, 383.

Maudlin, the merchant’s daughter, 117.

Maultman comes a-Monday, 535.

May-day country mirth, 301.

May-day, or Mayers’ song at Hitchin, 753. 

May-day dance, 671. 

May-fair, 587.

Maying-time, 377.

May-pole is up (The), 773.

Me have of late been in England, 445.

Meadow brow, (The), 182.

Merchant’s daughter (The), 117, note. 

Merchantman (The), 381. 

Merrily and cherrily, 285.

Merry bagpipes (The), 536, 786.

Merry boys of Christmas, 428.

Merry boys of Kent (The), 359.

Merry, careless lover (The), 356.

Merry, merry, merry, cheary, cheary, 580.

Merry milkmaids (The), 281, 777.

Merry milkmaids’ delight, 297.

Merry milkmaids in green, 295.

Merry wedding, or O brave Arthur of Bradley, 539. 

Midnight messenger (The), 165.

Milking pail (The), 297, 614.

Milkmaids’ dance, 282.

Milkmaid's dumps (The), 295.

Milkmaid’s life (The), 295.

Milkmaid’s new-year’s gift, 428.

Milkmaid’s resolution (The), 556.

Mill goes clack, clack, clack, 589.

Miller and the King’s daughters, 177.

Miller, he caught the fair maid (The), 750.

Miller would I be (A), 483.

Miller’s advice to his sons, 773.

Millfield, 185, 627.

Minstrels play’d their Christmas tune, 752.

Miss Dawson’s hornpipe, 719.

Miss Peachey, 779.

Mock-beggars’ Hall, 354, 355.

Moderation and alteration, 300.

Moll Peatly, 289, 778.

Molly, put the kettle on, 795.

Monarchy triumphant, 436.

Monk, General, 275, 286.

Monmouth is a brave lad, 785.

Montrose’s lines, 379, 507.

Moon shall be in darkness, (The), 739.

Morris dance, 283, 708, 777.

Mother beguil’d the daughter (The), 306, 356, 375.

Mother Watkin’s ale, 137. 

Mouse-trap (The), 602.

Muirland Willie, 786.
\end{dcindex}
\end{fixedpage}%807
\pagebreak

\setlength{\fixedpagewidth}{408pt}
\begin{fixedpage}%808
\versoheader

\begin{dcindex}
Must the absence of my mistress? 452.

My Aunt Margery, 534.

My days have been so wondrous free, 673.

My dear and only love, I pray, 379, 507.

My dear and only love, take heed, 378, 380, 507. 

My dog and I, 291.

My father was born before me, 446.

My freedom is all my joy, 215.

My heart is full of woe, 210, note.

My life and my death are both, 542\textsuperscript{a}.

My little Laud, methinks ’tis strange, 412.

My little pretty one, 79.

My lodging is on the cold boards, 527, 785.

My lodging is on the cold ground, 525, 785.

My man Thomas did me promise, 738.

My masters and friends, and good people, 123. 

My mind to me a kingdom is, 117, note.

My mistress is a shuttlecock, 508.

My mither’s ay glowran o’er me, 609.

My name is honest Harry, 365.

My name is old Hewson, 451.

My noble friends, give ear, 118.

My Patie is a lover gay, 619.

My Robin is to the greenwood gone, 233, 234. 

My scandalous neighbours, 337.

\bigskip
Nancy, 149.

Nancy Dawson, 718.

Nanny, O, 610\textsuperscript{a}.

Near Bristol liv’d a man of fame, 191.

Near the town of Taunton Dean, 746.

Near Woodstock town, 190.

Ned, she that likes thee now, 183.

Ne’er yet was a name, 720.

Neglected tar (The), 654.

Neptune’s raging fury, 292, 778.

Never dally, shall I? shall I? 558.

Never love thee more, 378.

Never will I wed a girl that’s coy, 317.

New Bob-in-Jo, 291.

New Broom [on hill], 458.

Newcastle ale, 573.

New courtier of the King’s, 299.

New dance at the Red Bull, 294.

New England psalm, 442.

New Exchange (The), 140, 317.

New Highland laddie, 772.

New hunting of the hare, 255.

New Ignoramus, 431.

New lusty gallant, 91.

New mad Tom of Bedlam, 328.

New-married Scotch couple (The), 575. 

Newmarket, 562. 

New rant (The), 553.

New Royal Exchange (The), 317.

New soldier (The), 299.

New Wells, 606.

New-year’s gift for the Pope, 336.

News from Court, 508.

News from Hyde Park, 325.

News from Tower Hill, 456.

News, good and new, 770.

Night her blackest sables (The), 509.

Night piece, 84.

No more, fair virgins, boast, 713.

Nobe’s maggot, 595.

Noble fisherman (The), 392.

Noble riddle wisely expounded, 530.

Noble shirve, [shire-reeve], 347.

Noel, Noel, 42, 757.

Nonesuch, 444.

Norfolk gentleman’s last will, 200.

North-country lass (The), 457, 783. 

North-country maids’ resolution, 441. 

North-country mayor (The), 124. 

Northern lad, or Fair maid’s choice (The), 503. 

Northern lass, or To bed to me, 503.

Northern lass (There dwells a maid in Doncaster), 559, 786. 

Northern lass’s lamentation, 456, 783. 

Northern Nancy, 354, 779. 

Northern Nanny, or The loving lass’s lamentation, 575.

Northern turtle (The), 366. 

Northumberland bagpipes (The), 536, 786.

Nose, nose, jolly red nose, 75.

Not light o’ love, lady, 223.

Nothing venture, nothing have, 284.

Nottingham ale, boys, Nottingham ale, 573.

Now all my friends are dead and gone, 357.

Now all you gallants of city or town, 557.

Now, by your good leave, Sirs, 431.

Now, dear lady, if thy will be, 34.

Now farewell to you, ye fine, 736.

Now, God he with old Simeon, 483, 784.

Now, O now, I needs must part, 127.

Now ponder well, 200.

Now Robin, lend to me, 79.

Now the Spring is come, 463.

Now Whitsun holidays they are come, 542.

Now, you young females that follow the mode, 593.

\bigskip
O brave Arthur a Bradley, 539.

O brave Roger a Cauverly, 534.

O come you from Newcastle? 340, 779.

O cruel, bloody fate, 280.

O dear, twelve pence, 737.

O dear, what can the matter be? 732.

O death! rock me asleep, 237.

O, do me no harm, good, 208, 774.

O fly from this place, dear Flora, 504.

Oh, for a husband, 454, 783.

O Fortune! how strangely thy gifts, 717.

O, good ale, thou art my darling, 660.

O grammercy, penny, 780.

O hone, O hone, 369, 780.

Oh! how happy are young lovers, 597.

O how I doat upon the lass, 658.

Oh how they frisk it, 541.

O Jenny, Jenny, where hast thou been? 587.

O leave me to dream and weep, 522.

O let no eyes be dry, 780.

O London is a fine town, 218, 220, 774.

O Lord! we have continual cause, 699.

O man in desperation, 770.

O mistress mine, 209.

O mother, a hoop, a hoop, 649, 787.

O mother! Roger with his kisses, 542\textsuperscript{a}.

O my dainty cavalilly man, 441.

Oh! my kitten, my kitten, 603.

O my sweet Jonathan Swifty, 604.

O no, no, no, not yet, 182, 193\textsuperscript{a}, 378.

O once my thyme was young, 522.

O ponder well, 200.

O rare Botham boy, 747.

O rare Turpin, hero, 661.

O some they will talk of bold Robin Hood, 395. 

O such a good fellow’s true blue, 194.

O such a rogue’s a Roundhead, 410.

O such a rogue should be hang'd, 777.

O that I had never married, 557.

O the broom, the bonny, bonny broom, 459, 461. 

O the golden days of good Queen Bess, 713.

Oh! the mighty innocence of Russell, 565.
\end{dcindex}
\end{fixedpage}%808
\pagebreak

\setlength{\fixedpagewidth}{408pt}
\begin{fixedpage}%809
\rectoheader

\begin{dcindex}
O the oak and the ash, 457, 786.

O this willow tree will twist, 520.

Oh! to bed to me, 503.

O we sail’d to Virginia, 642.

O weel may the keel row, 721.

O what a plague is love, 182.

Oh! when shall I be married? 462.

\hangpara{2em}{1}Oh! where, and oh! where is your Highland laddie? 740.

O willow, willow, 206, 774.

O, you merry souls, 595.

Oak and the ash (The), 457.

Oakerman (The), 779.

Oates well thrashed, 441.

Obsequy of fair Phillida, 319.

Of all delights the earth doth yield, 321.

Of all the birds that ever I see, 75.

Of all the birds that haunt the woods, 775.

Of all the girls in our town, 719.

Of all the girls that are so smart, 646, 647.

Of all the lads that are so smart, 645.

Of all the mauxes in the land, 645.

Of all the recreations, 284, 446.

Of all the simple things we do, 603.

Of all the trades that ever I see, 231.

Of Anna’s charms let others tell, 646.

Of late I chanc’d to be, 235.

Of late did I hear a young damsel complain, 555. 

Of late I did hear a young man domineer, 556.

Of old soldiers the song you shall hear, 299.

Off she goes, 796.

Oft have I heard of many men, 356.

Oft have I ridden, 63.

Old Christmas returned, or Hospitality reviv’d, 498. 

Old courtier of the queen, 299, 778.

Old Hob, or The mouse-trap, 602.

Old Homer, but with him what, 677.

Old Jemmy is a lad [James II.], 524.

Old King Cole, 634,—another, 635.

Old Lancashire hornpipe, 544.

Old langolee (The), 796.

Old lusty gallant, 91.

Old man is a bed full of bones (An), 60.

Old man’s complaint, or Unequal match, 528.

Old Marinet, or Mall Peatly, 289, 778.

Old miser slighted (The), 592.

Old Noll the brewer, 231.

Old Noll’s jig, 449.

Old Roger of Coverlay for evermore, 534.

Old Sir Simon the King, 262, 776, 792, 796.

Old soldier of the Queen, 299.

Old woman clothed in gray (An), 455, 783.

Old woman poor and blind, 551.

Old year now away is fled (The), 232.

Omnia vincit amor, 780.

On a Summer’s day, 254.

On Easter-Monday last, 575.

On Hounslow Heath, as I rode o'er, 662.

On the bank of a river so deep, 493.

On the cold ground, 526.

On yonder high mountains, 304, 681.

On yonder hill there stands, 129.

Once did I love a bonny, brave bird, 555.

Once I lov’d a maiden air, 257.

One evening, a little before it was dark, 326.

One evening, having lost my way, 675.

One holiday, last Summer, 290.

One morning, being free from care, 661.

One morning bright, for my delight, 367.

One morning early, by the break of day, 460.

One Friday morn, when we set sail, 742.

Open the door to three, 555.

Orange (An) on William III., 235.

Organ’s echo (The), 413.

Overthrow of proud Holofernes, 752.

Our King went forth to Normandy, 39.

Our oats they are hoed, 583.

Our ship carried over nine hundred men, 729.

Out, alas, what grief is this, 185,

Over and under, 189.

Over hills and high mountains, 304, 681.

Over hills and over mountains, 782.

Over the mountains and over the waves, 304. 

Over yon hills and yon lofty mountains, 681. 

Oxfordshire tragedy (The), 190, 773. 

\bigskip
Packington’s (or Paggington’s) Pound, 123, 771. 

Painter (The), 161.

Pair of turtle-doves, 452.

Pan, leave piping, 323.

Parliament routed, or A house to be let, 285. 

Parson and Dorothy, 126.

Parthenia, 439.

Parti de mal, 25.

Pass not for ribalds which, 349.

Past three o’clock, and a cold, frosty morning, 550. 

Pastime, with good company, 56.

Patient husband and the scolding wife, 556. 

Pattern of true love, 372.

Paul’s steeple, 117, 770.

Paul’s wharf, 130.

Pavan (The), 156.

Peace, shepherd, cease to moan, 201.

Peerless paragon, 357.

Peg-a-Ramsey, 218.

Peggie, I must love thee, 609.

Peggy is over the sea with the soldier, 614. 

Pepper’s black, 121, 627.

Pescod time, 196.

Peter-penny (The), 773.

Petticoat loose, 724.

Phil. Porter’s lament, 614.

Philander, 279.

Phillida flouts me, 182, 773.

Phillis on the new-made hay, 284.

Phœbus is long over the sea, 775.

Pickaxe and a spade (A), 200.

Pie sat on a pear-tree top (A), 232.

Pinder of Wakefield (The), 393, 394.

Pinnace rigg’d with silken saile (A), 738. 

Ploughman’s art of wooing (The), 555.

Poet’s dream, or Outcry against bailiffs, 619. 

Politick Club (The), 659.

Polly Oliver’s ramble, 676.

Polly, put the kettle on, 795.

Poor Coridon did sometime sit, 458.

Poor English mouths for twenty years, 630.

Poor man’s resolution (The), 266.

Poor Robin's maggot, 639.

Poor soul sate sighing (A), 206, 774.

Poor Tom hath been imprisoned, 334.

Portsmouth, 605.

Power of the sword (The), 431.

Praise of London, 327.

Praise of married life, 378.

Praise of milkmaids, 349.

Praise of nothing, 356.

Praise of sailors and sea affairs, 292, 778.

Pray lend me your ear, 233.

Prepare ye the plough, 121.

Pretty Bessie, 159.

Pretty Betty, now come to me, 558.
\end{dcindex}
\end{fixedpage}%809
\pagebreak

\setlength{\fixedpagewidth}{431pt}
\begin{fixedpage}%810
\versoheader

\begin{dcindex}
Pretty girl of Derby, O, 771.

Pretty Kate of Edinburgh, 622.

Pretty jest I’ll tell (A), 457.

Pretty Polly Oliver, 676.

Pride’s fall, 150.

Prince George at last is come, 524.

Prince Rupert’s march, 433.

Princum, prancum, 153.

Prithee, fill me a glass, 720.

Private news from Chatham, 528.

Proctor, are two and no more, 362.

Prodigal’s son converted, 498.

Prodigal’s resolution (The), 446.

Prognostications on Archbishop Laud, 412.

Proof of the pudding is all in the eating, 269. 

Protestant boys, good tidings I bring, 571. 

Protestant boys, both valiant and stout, 571. 

Protestant’s joy (The), 493. 

Punch cures the gout, 670.

Purcell’s tune like “God save,” 692.

Push about the jorum, 684.

Put on thy smock on Monday, 193.

\bigskip
Quadran Pavan (The), 104.

Queen Ann (Death of), 201.

Queen Dido, 370.

Queen Eleanor’s confession, 174.

Queen Eleanor was a sick woman, 174.

Queen Elizabeth's death, 182.

Queen Elizabeth’s life and death, 197.

Queen Elizabeth (The joyful receiving of), 242. 

Queen Mary’s lamentable complaint, 178.

Queen of May (The), 254.

Queen’s jig (The), 689.

Queen’s new hunt is up, 62.

Queen’s old courtier (The), 299, 636, 778. 

Quodling’s delight, 782, 794.

Quoth John to Joan, 87, 622, 769.

Quho’s at my window, 141,

\bigskip
Ragged, and torn, and true, 265, 267.

Rant (The), 553.

Ranters’ hymns, 748.

Ranting rambler, or Young gentleman’s frolic, 553. 

Ranting young man’s resolution, 555.

Reading skirmish (The), 571.

Rebellion hath broken up house, 266.

Red Bull, 294, 778.

Regret of a true lover, 378.

Remember, O thou man, 373.

Remember us, poor Mayers all, 753.

Remember, ye Whigs, what was formerly done, 622.

Repent, England, repent, 107, 770. 

Repent, therefore, O England, 197.

Restoration of Charles II. (The), 435, 436. 

Rhymer of late in a barber’s shop (A), 393.

Rich merchantman (The), 381.

Riding to Rumford, 111.

Rise, Britannia! Britannia, rise and fight, 687. 

Roast beef of Old England (The), 636.

Robin Hood, 391, 397.

Robin Hood and Allan-a-Dale, 390.

Robin Hood and Arthur-a-Bland, 391.

Robin Hood and Gandeleyn, 388.

Robin Hood and Guy of Gisborne, 396.

Robin Hood and little John, 391.

Robin Hood and Maid Marian, 391.

Robin Hood and Queen Katherine, 390.

Robin Hood and the beggar, 391.

Robin Hood and the Bishop, 391.

Robin Hood and the Bishop of Hereford, 395.

Robin Hood and the butcher, 391.

Robin Hood and the curtal friar, 392, 541.

Robin Hood and the Duke of Lancaster, 390. 

Robin Hood and the four beggars, 391.

Robin Hood and the monk, 388, 541.

Robin Hood and the Pinder of Wakefield, 390,393. 

Robin Hood and the potter, 388.

Robin Hood and the ranger, 391.

Robin Hood and the stranger, 391.

Robin Hood and the tanner, 391.

Robin Hood is to the greenwood gone, 233.

Robin Hood met with his match, 391.

Robin Hood rescuing the widow’s son, 390.

Robin Hood revived, 391.

Robin Hood, Robin Hood, said little John, 398. 

Robin Hood, Scarlet, and John, 393.

Robin Hood’s birth, breeding, valour, \&c., 539. 

Robin Hood’s chase, 391.

Robin Hood’s preferment, 392.

Robin Goodfellow, 142.

Robin is to the greenwood gone, 234.

Robin, lend to me thy bow, 79.

Robin Reddock, 398.

Robin Rowser, 365.

Rochelle, her yielding, 773.

Rock the cradle, John, 189.

\hangpara{2em}{1}Roger de Coverley, or Roger a Cauverley, 534, 789\textsuperscript{a}, 792, 796.

Rogero, 93, 96. 

Rogues’ March (The), 711.

Roland (Song of), 7, 764, 765.

Rome’s A-B-C, 781.

Room for a lusty gallant, 92.

Room, room for a rover, 613.

Room for company, here come good fellows, 322. 

Room for cuckolds, 322.

Room for gentlemen, here comes my Lord Mayor, 322.

Rosalind’s complaint, 493. 

Rosamond’s overthrow, 361.

Rose is white, rose is red, 627.

Rose of England (The), 117.

Round about our coal fire, 266.

Round and round, the mill goes round, 589. 

Round, boys, indeed, 277.

Rousseau’s dream, 749.

Row the boat, Norman, 783.

Row well, ye mariners, 712, 770.

Rowland, 114, 770.

Roy’s wife of Aldivalloch, 621.

Royal Oak (The), 420

Rufst du, mein Vaterland? 691.

Ruined lover (The), 355.

Rule, Britannia! 686.

Rural dance about the maypole, 531.

Russell’s farewell, 565.

\bigskip
Saddle to rags, 553.

Sailor’s complaint (The), 597, 787.

Sailors for my money, 291.

Saint George for England, 92, 265, 286.

Saint George and the Dragon, 286

Saint George and the Dragon (on General Monk), 299. 

Saint’s encouragement (The), 425. 

Sale of Rebellion’s household stuff, 266.

Sally in our alley, 645.

Sally rivall'd by country Molly, 645.

Sally’s lamentation, 645.

Sandgate lass’s lament (The), 734.

Saraband, 497.

Satan, my foe, 163.

Saw ye not Pierce the' piper? 639.

Saw you my father? 731.
\end{dcindex}
\end{fixedpage}%810
\pagebreak

\setlength{\fixedpagewidth}{360pt}
\begin{fixedpage}%811
\rectoheader

\begin{dcindex}
Sawney and Jockey, 619\textsuperscript{a}.

Sawney was a bonny lad, 612.

Sawney was a dawdy lad, 611.

Sawney was tall, 610, 618.

Sawney will ne’er be my love again, 618.

Says old Simon the King, 263, 776.

Scornful maid (The), 774.

Scotch lass’s constancy (The), 619\textsuperscript{a}.

Scotch wedding (The), 575.

Scourge for the Pope, 322.

Seaman’s mournful bride, 427.

Sedany(The), 64.

See the lovely rose, 607.

Sellenger’s round, 69, 541.

Send back my long-stray’d eyes, 640.

Send forth, dear Julian, all thy hooks, 426. 

Seventeenth'day of July (The), 226.

Shackley Hay, 367, 780.

Shaking of the sheet, 84.

Shall I go walk the woods? 66.

Shall I here rehearse the story? 315.

Shall I lie beyond thee? 259.

Shall I, like an hermit, dwell? 315.

Shall I, mother, shall I? 444.

Shall I, shall I? no, no, no, 557.

Shall I, wasting in despair? 315, 316.

Shall the absence of my mistress? 451.

She met with a countryman, 534.

She rose and let me in, 509, 613.

She's gone and left me all alone, 452. 

Shepherd’s daughter (The), 126.

Shepherd’s delight (The), 129.

Shepherd’s dialogue, 377.

Shepherd’s joy (The), 775.

Shepherd’s resolution (The,) 315.

Shepherd sat him under a thorn (A), 536, 786. 

Shepherd, saw thou not my fair? 773. 

Shoemaker of Jerusalem, 197. 

Shoemaker’s holiday (The), 277.

Shore’s wife, 215.

Shropshire rounds, 599, 796.

Shropshire wakes, or Hey for Christmas, 64.

Sic a wife as Willie had, 795.

Sick, and sick, and very sick, 223, 226.

Since bankers are grown so brittle, 503.

Since first I saw your face, 313, 314.

Since Hanover is come, 436.

Since Hodge proves ungrateful, 725.

Since Popery of late, 212.

Since Popish plotters, 431.

Since reformation with Whigs is in fashion, 431. 

Since Sally's charms so long have been, 645. 

Sing, boys, drink, boys, 779. 

Sing care away, 210, 304.

Sing “home again,” Jockey, 612.\textsuperscript{b }

Sir, do not think these lines, 461.

Sir Edward Nowell’s delight, 149.

Sir Eglamore, 275, 276.

Sir Guy, 171.

Sir John Barleycorn, 150, 305.

Sir John got him an ambling nag, 67.

Sir Lancelot du Lake, 199, 271.

Sir Richard Whittington, 517.

Sir Roger de Cnverlcy, 534.

Sir Walter Raleigh's lamentation, 175.

Sit, soldiers, sit, and sing, 342.

Slighted Jockey, or Coy Moggy’s cruelty, 600. 

Slighted maid, or Pining lover, 527.

Slippery Will, or The old bachelor, 461.

Slow men of London, 446.

Smiling Polly, 722.

So, so, the deed is done, 440.

So sweet is the lass that loves me, 780.

Soldier has no fellow (The), 342.

Soldier’s departure (The), 557.

Soldier’s lamentation, 508.

Soldier’s life (A), 144. .

Soldier’s march, 73.

Soldier’s song (A), 669.

Some say drinking does disguise men, 143.

Some talk of Alexander, 152.

Some time I am a tapster, 446.

Some years of late, in eighty-eight. 211, 213. 

Song of harvest-home, 582.

Song of the caps, 85.

Song of the declensions, 367.

Songs of shepherds in rustical, 322, 324.

Spanish Armada (Songs on), 213, 774.

Spanish gipsy, 272, 273.

Spanish lady (The), 186, 773.

Spanish pavan, 240, 776.

Spanish tragedy (The), 371.

Spinning wheel (The), 622.

Spring is coming (The), 654.

Stag chase (The), 346 

Staines Morris dance, 125. 

Stand thy ground, old Harry, 365.

Star that shines by daylight (The), 779.

Stingo, or Oil of barley, 305, 307.

Stop, poor sinner, stop, and think, 749.

Storm (The), “Cease, rude Boreas,” 598.

Stormy winds do blow, 292, 742.

Story strange I will you tell (A), 235.

Stout Cripple (The), 158.

Stow the friar, 147.

Strephon and Cloris, 304.

Strike up, you lusty gallants, 92.

Subject’s satisfaction (The), 493.

Such a rogue would be hang’d, 777.

Such Tories now are we, 427.

Summer isicumen in, 24, 796.

Summer is comen and Winter is gone, 25. 

Summer is comen with love to toune, 25.

Sun had loos’d his weary team (The), 312.

Sure marriage is a fine thing, 596.

Susan Higges, 150.

Susan’s complaint and remedy, 648.

Sussex whistling song (The), 571.

Sweet day, so cool, so calm, 285.

Sweet England’s prize is gone, 176.

Sweet, if you love me, 684.

Sweet is the lass that loves me, 354. 779.

Sweet Lucina, lend me thy aid, 269.

Sweet Margaret, 155.

Sweet Mistress Money, 270.

Sweet Nancy, I do love thee, 294.

Sweet Nelly, my heart’s delight, 656.

Sweet Robert, 114.

Sweet thing is love (A), 381.

Sweet virgin, hath disdain? 363.

Sweet William’s farewell to black-ey’d Susan, 640. 

Sweet, youthful, charming ladies, 361.

Swimming lady (The), 379.

Sword (The), 431.

\bigskip
Take a kiss or twa, dear Jockey, 617, 787.

Take thy old cloak about thee, 505.

Take time while ’tis offered, 279.

Take warning London, and beware, 770.

Tanner and the King, 392.
\end{dcindex}
\end{fixedpage}%811
\pagebreak

\setlength{\fixedpagewidth}{384pt}
\begin{fixedpage}%812
\versoheader

\begin{dcindex}
Tantara rara tantivee, 325.

Taunton Dean, 262, 672.

Teach me, Belissa, what to do, 379.

Teague and Sawney, 571.

Tell me, Daphne, 1.58.

Tenant of my own (A), 595.

Tenth of June (The), 657.

Thames uncas’d (The), 427.

That I am a sailor ’tis very well known, 590. 

Then great Bill Brown, 660.

Then to the may-pole come, 125.

Then up spoke the captain, 743.

There came up a lass from a country town, 593. 

There dwells a maid in Doncaster, 559.

There dwelt a man in fair Westmoreland, 260. 

There dwelt a widow in this town, 446.

There happen’d of late a terrible fray, 601.

There is an old proverh, 730.

There lies a pudding, 483.

There lived a man in Ballymecrazy, 714.

There lives a lass upon the green, 685.

There’s a lusty liquor, 308.

There’s never a maid in all this town, 74.

There’s nothing without money, 356.

There was a bonny, blade, 120, 457.

There was a certain usurer, 146.

There was a countiy gallant, 189.

There was a fair maid of Islington, 541.

There was a frog liv’d in a well, 88.

There was a jolly miller, 666.

There was a jovial beggar, 345.

There was a jovial tinker, 188, 773, 779.

There was a knight, and he was young, 520. 

There was a knight was drunk with wine, 520. 

There was a knight was wine-drunk, 520. 

There was a lady in the North country, 531. 

There was a lass of Cumberland, 503.

There was a little man, and he wooed a little maid, 770.

There was a London gentlewoman, 313. 

There was a maid came out of Kent, 348.

There was a maid in the West, 595.

There was a maid this other day, 136, 454, 783. 

There was a maid went to the mill, 594.

There was a miller, who had three sons, 773. 

There was a pretty lass, 595.

There was a rare ratcatcher, 779.

There was a rich merchantman, 382.

There was a shepherd’s daughter, 126.

There was a youth, and a well-beloved, 203. 

There was an old farmer in Sussex did dwell, 571. 

There was an old fellow, 262.

There was an old prophecy found in a bog, 572. 

There was an old woman liv’d under a hill, 594. 

There was an old woman went up in a basket, 571. 

There were three ravens, 59.

There were three travellers, 506.

There were two sisters, they went a playing, 177. 

This wonder of the Northern star, 559.

Thomas, you cannot, 336.

Thou and I will foot it, Joe, 306.

Thou canst not hit it, 239.

Thou wealthy man of large possessions, 165. 

Though bootless I must needs complain, 378. 

Though I have but a mark a-year, 356. 

Though I live not where I love, 451, 782. 

Though I’m but a farmer’s son, 657.

Though some do wonder why, 356.

Three children sliding on the ice, 199.

Three merry boys are we, 362, 780.

Three merry boys of Kent, 588.

Three merry men of Kent, 588.

Three merry men be we, 216.

Three merry wives of Green-goose Fair, 566. 

Three ravens (The), 59.

Three sheep-skins, 614.

Through all the employments of life, 455. 

Through Liddersdale as lately I went, 460. 

Through the cool, shady woods, 364.

Through the Summer night, 683.

Thus think, and drink tobacco, 563.

Tidings of comfort and joy, 752.

Tight little island (The), 720.

Time’s alteration, 266, 777.

Tinker of Turvey (The), 779.

’Tis good to be meriy and wise, 254.

Titus Oates, 325\textsuperscript{a}, 441.

To all men now I’ll plainly shew, 460.

To all you ladies now at land, 507.

To bed to me, 503.

To carry the milking pail, 296 and 297.

To Christians all I greeting send, 451.

To couple is a custom, 339.

To every faithful lover, 292.

To find my Tom of Bedlam, 333.

To horse, brave boys, to Newmarket, 562.

To marry a widow, 95.

To my muse give attention, 713.

To old Sir Simon the King, 266.

To you, fair ladies, 508.

Tobacco is an Indian weed, 563.

Tobacco, that is withered quite, 563. 

Tobacco’s a musician, and in a pipe delighteth, 779. 

Tom a Lincoln, 566.

Tom and Will were shepherd swains, 609\textsuperscript{a}

Tom Nokes’s jig, 505.

Tom of Bedlam, 255, 330, 332, 417, 779.

Tom Tiler, 187.

Tom Tinker leads a merry life, 187.

Tom Tinker’s my true love, 353.

Tom the tailor, 325\textsuperscript{a}.

Tom Trusty, 336.

Tommy was a lord of high renown, 619. 

To-morrow it is St. Valentine’s Day, 227. 

To-morrow the fox will come, 82.

Tony was small, but of noble race, 619.

Tories’ confession (The), 427.

Tory’s lamentation, 455.

Toothless bride (The), 551.

Touch me lightly, 218.

Touch the thing, you, 684.

Trenchmore, 82, 769.

Trepanned virgin, or Good advice to maidens, 537. 

Trial brings truth to light, 269.

Trifling song you shall hear (A), 267.

Trimmer (The), 233.

Trip and go, 130.

Trip to Marylebone, 792.

Triumph and joy, 229, 775.

Triumph of the Roundheads, 425.

Troll the bowl, 483.

Troy Town, 370.

True blue, 194.

True love’s knot untied, 128.

True love without deceit, 682.

True lover’s happiness, 284.

Trusting heart, though men deceive thee, 722. 

Trusty Dick, 782. 

Truth’s integrity, 304.

Tune of forty-one, 427.

Tune of eighty-eight, 211.
\end{dcindex}
\end{fixedpage}%812
\pagebreak


\setlength{\fixedpagewidth}{390pt}
\begin{fixedpage}%813
\rectoheader

\begin{dcindex}
Turkeyloney, 95.

Turn again, Whittington, 515, 785,

Turn, love, I prithee, 528.

Turpin’s valour, 661.

Twa bonny lads were Sawney and Jockey, 619\textsuperscript{a}. 

Twang lang dildo dee, 236.

’Twas a youthful knight, 178.

’Twas at the gate of Calais, 637.

'Twas down in Cupid’s garden, 727.

'Twas God above that made all things, 513. 

’Twas on the morn of sweet May-day, 712.

’Twas within a furlong of Edinborough town, 611, 612.

Twelfth Night, or King and Queen, 773. 

Twenty-ninth of May (The), 491.

Twitcher (The), 555.

Two faithfhl lovers (The), 369.

Two famous lovers once there were, 367.

Two loving sisters, 259.

Two Toms and Nat in council sat, 446.

\bigskip
Uncomly in cloystre I coure, 26.

Unconstant Roger, 783.

Unconstant woman, 452.

Undaunted Londonderry, 571.

Under and over, 189.

Under the greenwood tree, 541.

Under the holly-bush tree, 544.

Under the rose, 730.

Undone, undone, the lawyers are, 433. 

Unfortunate Miss Bailey, 714.

Unfortunate Stephen! well may1’st thou, 682. 

Union of the red and white rose, 199.

Up tails all, 196, 773,

Up the green forest, 555.

Upon a Summer’s day, 254.

Upon a Summer time. 254.

Upon a sunshine Summer’s day, 622.

Upon a time I chanc’d, 255.

Upon a time when Titan’s steeds, 165.

Upon the meadow brow, 182.

\bigskip
Valiant Jockey’s march’d away, 611, 612.

Valiant seamen’s congratulation to Charles II., 292. 

Valiant trooper and pretty Peggy, 452.

Valiant virgin, or Philip and Mary, 292.

Very good song, and very well sung (A), 262.

Vicar and Moses (The), 602.

Vicar of Bray (The), 653, 787.

Virgin most pure (A), 755.

Virgin Queen (The), 603.

Virgin’s complaint (The), 363.

Virgin unspotted the Prophets foretold (A), 642. 

Vive le Roy, 429, 622, 782.

\bigskip
Waits, 547 to 551.

Wakefield on a green, 394.

Walking in a country town, 182.

Walking in a meadow green, 182.

Walsingham, 121.

Wandering Jew (The), 197.

Wandering maiden, or True love united, 681. 

Wandering Prince of Troy (The), 370. 

Wandering virgin, or Coy lass fitted, 682. 

Warning for married men, 197. 

Warning for maidens, 197.

Warning to false traitors, 228.

Warning to London, 112.

Was ever knight for lady’s sake? 171.

Was ever maiden so scorned? 527.

Was ever man? 171.

Waterman’s song upon the thaw, 427. 

Watkin’s ale, 136. 

Watton town’s end, 219, 220.

We are poor frozen-out gardeners, 747, 790.

We be soldiers three, 77.

We be three poor mariners, 77, 769.

We have brought you a branch of May, 753. 

We’ll drink out of the nut-brown bowl, 745.

We’ll gladden our hearts with the best, 591.

We’ll go no more to Tunbridge Wells, 317.

We’ve cheated the parson, 583.

We will go no more to the Old Exchange, 317. 

Wednesbury cocking (The), 660.

Weep, weep, O Walsingham, 122.

Welcome home, old Rowley, 313.

Welcome, royal May, 491.

Well-a-day, 174.

Well-a-day, where should I stay, 465. 

West-country damosel’s complaint, 776. 

West-country delight, or Hey for Zommersetshire, 542. 

West-country jig, 83. 

West-country lawyer, or Witty maid’s good fortune, 520.

West-countiy lovers (The), 279. 

West-country nymph, or Loyal maid of Bristol, 524.

Western wind, when wilt thou blow? 57. 

Westminster wedding, 186.

What a fine thing have I seen to-day, 649, 650. 

What Booker can prognosticate? 437.

What care I how fair she be? 315.

What creature’s that with his short hairs? 410. 

What doth ail my love? 143.

What if a day, or a month? 310, 311.

What is greater joy or pleasure? 597.

What’s a Protector? 421.

What need we brag or boast at all? 286.

What pity ’tis so bright a thought, 645.

What though I am a country lass? 306, 375, 647. 

What though the zealots? 430.

What woman could do I have tried, 649.

When a man’s a little bit poorly, 659.

When all was grown to dark midnight, 382. 

When Arthur first in court, 199.

When as King Edgar did govern, 768.

When as King Henry rul’d, 199.

When as the Duke of Normandy, 94.

When Britain first, at Heav’n’s command, 687. 

When busy fame, 536, 537, 786.

When Daphne did from Phœbus fly, 338.

When first my dear Johnny came to this town, 784.

When flying fame, 198. 

When from the East, with dappled grey, 651. 

When glistening Phœbus, 327.

When God had taken for our sin, 371.

When grave divines preach up dull rules, 650. 

When Greeks and Trojans fell at strife, 371. 

When Harold was invaded, 567. 

When I call to mind those jovial days, 780. 

When I followed a lass, 518.

When I was a bachelor, 218.

When I was bound apprentice, 733, 734.

When in war on the ocean, 642.

When Isabel, fair England’s Queen, 144.

When love was young, 466.

When mighty roast beef was the Englishman’s food, 636.

When once Master Love, 449. 

When our great prince, with his choice band, 688. 

When Phœbus addrest his course, 191.

When Samson was a tall young man, 241.

When Sawney left me, he had store of gilt, 618.
\end{dcindex}
\end{fixedpage}%813
\pagebreak


\setlength{\fixedpagewidth}{360pt}
\begin{fixedpage}%814
\versoheader

\begin{dcindex}
When shaws are sheen, 397.

When swallows dart from cottage eaves, 713. 

When that I was a little tiny boy, 225.

When the hills and lofty mountains, 665.

When the King comes home in peace again, 437. 

When the King enjoys his own, 434, 613, 782. 

When the Knights of the Bath, 771. 

When the rosy mom appearing, 722.

When the stormy winds, 291, 292, 778.

When this old cap was new, 266, 777.

When Troilus dwelt in Troy Town, 91.

When Troy Town for ten years, 371, 372.

When Venus, the goddess of bcanty and love, 573. 

When you court a young virgin, 639.

When young Paul’s steeple, 117.

Whence comes it, neighbour Dick? 673.

Where gott’st thou the haver-meal bannock? 611. 

Which nobody can deny, 227, 231, 233, 441. 

Which way shall I turn? 353.

Whigs are but small (The), 782.

Whigs are now such precious things, 436.

While others attempt heavy minutes to kill, 690. 

While the raging seas do roar, 742.

Whilom I was, 353.

Whim (The), or Bartholomew Fair, 585.

White hen she cackles (The), 483.

Whittington’s bells, 517.

Who can blame my woe? 201, 612.

Who has e’er been in London? 677.

Who list to lead a soldier’s life? 144.

Who list to read the deeds? 267.

Who liveth so merry? 81, 790.

Who’s the fool now? 76, 768.

Whoop! do me no harm, good man, 208, 774. 

Whoop! Jenny, come down to me, 774.

Why do we boast of Arthur? 286.

Why should not I love my love? 339.

Why should we boast of Arthur? 287.

Why should we so much despise? 563.

Why, soldiers, why? 669.

Widow Brown (The), 446.

Wife, prithee, now give me thy hand, 455. 

Wigmore’s galliard, 242.

Will the shepherd and fair Kate, 190.

Will you hear a German princess? 186.

Will you hear a Spanish lady? 186.

Will you walk the woods? 66.

Willoughby’s welcome home, 115.

Willoughby whim (The), 587.

Willow, willow, 206, 774.

Willow tree (The), 377.

Willow tree (The), “I sow’d the seeds,” 520. 

Willy, prithee, go to bed, 769.

Willy was so blithe a lad, 507, 785.

Wilson’s wild (or wile), 87.

Wiltshire wedding (The), 146.

Wiltshire wonderful tragedy, 194.

Winchester christening (The), 312.

Winchester wedding (The), 495, 784.

Wind blawis cald (The), 193.

Wine-cooper’s delight (The), 498, 784.

Wine cures the gout, 670.

Winifred’s knot, 555.

Winning of Boulogne (The), 148.

Wise men were but seven (The), 363.

Wise sayings of Solomon, 242.

Wit bought at a dear rate, 528.

Wit’s never good till ’tis bought, 776.

With a fading, 234.

With a hie dildo dill, 234.

With a new beard, 299.

With a pudding, 235.

With an old motley coat and a malmsey nose, 299. 

With an old song, 300.

With grateful hearts we’ll drink the health, 651. 

With Jockey to the Fair, 711.

With my flock as walked I, 157, 354.

With never a penny of money, 506.

With sobbing grief, 118.

Within the North country, 279.

Within the town of Walton fair, 367.

Wiving age (The), 774.

Woefull lamentation of Jane Shore, 215. 

Wolsey’s Wild, 87.

Woman’s work is never done, 557.

Women all tell me I'm false (The), 679.

Women are strongest, 62.

Woo’d and married and a’, 723.

Woody Cock (The), 793.

Wooing of Colman, 72.

Wooing of Queen Catherine, by Owen Tudor, 223. 

Wooing song of a yeoman of Kent, 90.

Worksop waits, 551.

World hath allurements (The), 269.

Would you win a young virgin of fifteen? 639.

\bigskip
Ye banks and braes o’ bonny Doon, 794.

Ye Church-ales and ye Morrises, 271.

Ye Commons and Peers, 664.

Ye dainty dames, 118, note.

Ye ladies of London, 593.

Ye maidens and wives, and young widows, 771. 

Ye nymphs and sylvan gods, 297.

Ye sylvan nymphs, 261.

Yellow stockings, 603.

Yeoman of Kent (The), 602.

Yonder comes a courteous knight, 62.

York, York, for my money, 229, 775.

Yorkshire volunteers’ farewell, 684.

You bachelors that single are, 427.

You Calvinists of England, 292.

You coward-hearted citizens, 341.

You dainty dames, so finely fram’d, 197.

You gallant ladies all, 369.

You gallants and you swaggering blades, 356.

You gentlemen of England, 291.

Fou lasses and lads, get leave, 532.

You loyal lovers that are distant, 451.

You’ll think, ere many days, 236, 664.

You parents all attend, 279.

You parents, whose affection fond 242.

You Stockton lads and lasses, 685.

You rural goddesses, 295.

You that have lost your former joys, 371.

You that true Christians be, 773.

You traitors all, that do devise, 229.

You virgins so pretty, 682.

You young men all take pity, 315.

Young Bateman, 197.

Yonng Jemmy is a lad, 523, 525.

Young Jemmy was a lad, 785.

Young man and a lass of late (A), 378.

Young man in our parish (A), 189.

Young man lately did complain (A), 378.

Young man’s opinion (The), 315.

Young men and maids, do lend me, 259.

Young Palmus was a ferryman, 368, 780.

Young virgins love pleasure, 654.

Your hay it is mow’d, 582.

Youthful lawyer fine and gay (A), 520.

\bigskip
Zealous Puritan (The), 779.
\end{dcindex}
\end{fixedpage}%814
\pagebreak

\setlength{\fixedpagewidth}{360pt}
\begin{fixedpage}%815
\headingthree{INDEX OF SUBJECTS.}

\centerrule

\textit{(An index to include every reference to old authors and old books
would be of excessive and unusual length. As names, titles, and
dates, are to be found in the text, it is thought that an index of
the subjects upon which references are made will be more useful. The
names of living authors, and of recently published books, are
included.)}

\centerrule

\begin{dcindex}
%\begin{hangparas}{2em}{1}
A’Beckett (Thomas) preceded by 250 singing boys, 18.

Abell (John), the singer, 616\textsuperscript{a}. 

“Abraham men,” and their songs, 720.

Addison’s character of Tom D’Urfey, 623.

Agincourt (Songs on the battle), 38, 89, 199.

Ailred (Abbot) — Extraordinary description of English singers in
the twelfth century, 404, in note.

Albinus, reputed inventor of the four-stringed fiddle, 763.

Aldhelm (St.) dresses like a minstrel, and sings to obtain a
hearing, 4 and 759. His skill in music, 759. His poems and songs, 759
and 766.

Alfred’s (King) knowledge of music, \&c., 4, 17.

Almanack-makers, time of Charles I., 437\textsuperscript{a}.

Anders (M.), 628\textsuperscript{a}.

Anglo-Saxon fiddle and rote, wood-cuts, 761 and 767.

Anglo-Scottish songs,. 609 \textit{et seq}.

Anne’s (Queen) present to D’Urfey for a lampoon, 622.

Apprentices of London (Songs of), 150. Chaucer’s
description—singing and playing rebec and gittern, 34.

Armstrong (Johnny), Songs about, 260\textsuperscript{a}, 776.

Arne’s masque of Alfred, 686.

Artificers’ habit of singing at work, 109, 797.

Aston (Tony), his mode of life, 654.

Ayrton (W.), 529, 709.

\bigskip
Bacon (R. M.), 142, and note.

\textsc{Bagpipe}.—Reason for its use on pilgrimages, 34\textsuperscript{a}. Tympanum the
mediæval Latin name for bagpipe, 760\textsuperscript{b}. Improbability of its having
lacked the fourth and seventh in its scale, 790. Prevalence of, shown
by extant tunes, 789. Worcestershire, 545. Lincolnshire, 546. Not
approved by the upper classes, 791. Puns upon it, 546. For dancing
bears, 104. Bagpipe-hornpipes, 604.—\textit{See also Piper.}

\textsc{Ballads}.—Against the Lollards, 40. Against the vices of the
Clergy, 40. Earliest printed extant, 54. Infinite multitude of, 106,
157, and 632. “Silly gentlewomen won by,” 487. Influence upon the
public mind, 253, 423, 434, and 568. Acts against, 54, 55, and 417.
Dread of being “balladed,” 253, 422.

\textsc{Ballad-singers}.—Suppressed by the Puritans, 417. Earning 20s.
a-day, 106. Leather jackets, 501.

\textsc{Ballad-writers}.—“Character of a pot-poet,” 150. “Character of a
ballad-monger,” 295. “Uncountable number of rhyming ballet-makers,”
375. Expedient for making words to fit tunes, 375. Exemplified, 243.
“Ballad-mongers hangmen twice,” 422. Power of ballad-writers, 253,
422. Elderton, 91, 107, 121, 125. 185, 229. Deloney, 94, 106, 107,
163, 634, 653, 770. Martin Parker, 265, 269, 272, 279, 291, 295, 306,
322, 356, 366, 375, 418, 419, 420, 434, 452, 773, 777, 779, 781.
Martin Parker’s society of 24 ballad-writers, 419. Thomas Herbert,
419, 781. Lawrence Price, 277, 780. John Wade, 420, 462. Leonard
Gibson, 223. John Taylor, 419, 420. Bichard Johnson’s \textit{Crown Garland
of Golden Roses}, 176, 178, 182, 197, 199, 201.

Bandora, kind of lute strung with wire, 244, 776.

\textsc{Barbers’ shops}, musical instruments in, 101 and 104. Their music,
104, 148.

Barley-break, an ancient game, 136.

\textsc{Bars in music}, when general, 541\textsuperscript{b}.

Bartholomew Fair (Song on), 586. 

Bartholomew’s Hospital (St.) founded by a minstrel, 9.

\textsc{Base-viol} in drawing-rooms for waiting visitors, 100. Also played
by ladies, 101.

Basse (William) songs mentioned by Izaak Walton, 255.

Bede (The Venerable) an able musician, 17.

Beggars’ songs, 270, 333, 345, 720.

Beggars’ Opera, and other ballad-operas, 630.

Bell-ringing, 90.

Benbow (Admiral), Songs on, 641, 678.

Beverley minstrels, 31 and 766. Fair, 135.

Blacksmiths singing rounds, 109.

Boleyn (Anna) songs, 237, 775.

Bombards, large leather bottles, 515.

Boniface, Apostle of Germany, an Englishman, 759.

Braule, or Branle, a dance, 77, and note, 626. Described, 768.

Brooksby (P.), the ballad-printer, 541\textsuperscript{a}.

Brown (Rawdon) on Venetian despatches, 50.

Bull (Dr. John), Anecdote of, 249. Claim to “God save the King”
considered, 695.

Bunting (Edward), 529, 708, 709, 789\textsuperscript{a}.
%\end{hangparas}
\end{dcindex}
\end{fixedpage}%815
\pagebreak
\renewcommand\versoheadertext{index of subjects.}

\setlength{\fixedpagewidth}{360pt}
\begin{fixedpage}%816
\versoheader

\begin{dcindex}
Burdens of songs, 23, 25. Bourdon, 34, and note~\textsuperscript{b}. “Hey, nonny,
nonny, Hey, trolly, lolly,” \&c., 54. Burdens sung in plays, 79, 221,
222, 459.

Burn (J. H.), 263\textsuperscript{c}, 516.

Burney (Dr.)—Criticism on his History of Music, vii. to ix.
Conjecture as to the “Chanson Roland” disproved, 6. His doubts of
Giraldus Cambrensis, 19, 20. On “Sumer is icumen in,” 21, 22. On
numerous inaccuracies and misstatements in his History, 245\textsuperscript{a}. Wrong
account of English opera, 479, and note~\textsuperscript{a}, 783. Mistake about the
rote, 766. Inaccurate account of the first Italian opera, 473, and
note.

Butler (Samuel)—Songs on the Roundheads, 410 and 447.

Byrde (W.)—Eight reasons why every one should sing, 99.

Cambridge, 50 musicians in the College Chapel, 477.

Campion (Dr.), 310, and note. 

Canaries, dance, 368.

Canon in unison, the earliest example, English, 23. General use
of, 108. Contrivances of canon, 484\textsuperscript{b}.

Carey (Henry)—Claim to “God save the King” considered, 694, 701.
His account of “Sally in our alley,” 645.

Carolling, merry singing, ix.

Carriages introduced in England, 138\textsuperscript{b}.

Carters, carmen, and ploughmen, whistling, 53, 138, 579, and 797.

Catches, 108, and note. How to make, 484\textsuperscript{b}. Collections of, 483.
Singing catches, 565.

\textsc{Cathedrals}.—Misappropriation of musical funds, 402. Puritan abuse
of, 405.

Characteristic airs of England, 21, 789, 791. Mere terminations
an unsafe guide, 792. Few airs of melancholy cast, 790. Old ditties
generally cheerful, 791. Manly and jovial airs, 791. Tunes to lengthy
narratives, 791. Hornpipes, jigs, and bagpipe-tunes, 791.

Charles I.—His trial, 164. Love of music, 414. Private band, 468.
Song attributed to, 414. Of doubtful authenticity, 781.

Charles II.—Dancing, 82. Song on his escape from Worcester, 435.
On his restoration, 435, 436. Music for his entry to London, 479.
Coronation, 699. Specimen of his letters, 467. Only liked dance
music, 468, 621. His “four-and-twenty fiddlers,” 468, 469\textsuperscript{b}. Charles
at the theatre, 506, 526. Lampoons on, 569\textsuperscript{a}. Singing with D'Urfey,
622.

Chaucer’s descriptions of music, 32 to 37. “Sing weylaway,” 175.
“Sing a quinible,” 34\textsuperscript{d}. “Douced,” 36, and note.

Chester Fair—Minstrels relieve the castle, 10.

Chettle (Henry) on ballad-singers, 106. On carmen, 138. Ballads
named by, 137, 145.

Chorus, or Corus, a skin stretched in the curve of two
pipes,—also a crowde or fiddle, 760\textsuperscript{b}, 764.

Christmas carols, 41, 42, 48, 64, 111, 151, 194, 197, 232, 234,
279, 319, 373, 541, 750 to 758.

Christmas, Lamentation of, 463, 782.

Christmas returned, 499.

Christmas sports, 64, 70, 195, 548, 601.

Church scales, or modes, 12, 13, 790. Sometimes mistaken for
minor scales, 790. The eleventh mode, or key of C natural, copied
from the “vulgar musicians of towns and villages,” 27. Omission of
the major seventh in old tunes accounted for by the Dorian mode, or first Church scale, 790.

Church singers, 18, 51, 403\textsuperscript{c}.

Cinque pace, a dance, 156\textsuperscript{a}.

Cittern, or Cithren, a kind of guitar strung with wire,
35, 99, 248. Described, 101, and note~\textsuperscript{d}. “As natural, to a barber as
milk to a calf,” 104. Sir J. Hawkins’s description incorrect, 247\textsuperscript{b}.
Facsimile of cittern music, xxiii.

Citole, 35.

Claricorde, 35\textsuperscript{b}, 321\textsuperscript{a}.

Clarion, 35\textsuperscript{b}.

Clavicimbals, 35\textsuperscript{b}.

Clavichord, 49.

Cleiveland (John)—Songs against Puritans, 420. His definition of
a Protector, 421.

Cloth-workers singing catches, 109.

Clubs, earliest notice of, 488.

Cobblers proverbially singers, and merry, 98, 110, 278, 348, 352,
589.—\textit{See also Shoemakers}.

Cock-fighting, 659.

Collier (J. Payne) on the “great variety of entries about music
in household books,” 45. Note on \textit{The Hunt is up}, 60. On the registers
of the Stationers’ Company, 107, 117. On Roxburghe Ballads, 263, 297,
776. On Dodsley’s \textit{Old Plays}, 329, 334.

Colliers singing music in parts, 109.

“Common tunes,” Mace’s description of, 485.

Concerts, origin of public, 479 \textit{et seq}.

Corbet (anecdote of Bishop), 501.

Cork, ballad on burning of the city, 163. Battle of birds over,
502.

Cornet, 248, and note. English excelled in playing, 631.

Cornwall—Three-men’s-songs, 110. Wideslade, the minstrel, 108.
Ballad of flies at Bodmin, 199.

Corser, Rev. Thomas, 784.

Counterpoint, definition of, 15. Origin attributed to England,
21. Of thirteenth century, in six parts, 23.

Country dances at Court, from Elizabeth to James II., 340, 626. “
Transplanted into
almost all the Courts of Europe,” 627\textsuperscript{a}. Italians “fond to a
degree” of them, 307. Mistakes as to the origin of, 626 \textit{et seq}. Names
of, derived from ditties, 424. Round and square country dances, 627.
Sets printed in Paris, 719. In Germany, Denmark, and the Netherlands,
xi. and 628.

Court masques, 473, 474. Music of, reprinted abroad, 424.

Countrymen’s songs and sports, 87, 90, 153,459, 507, 533, 542.

Coverdale (Myles) advises courtiers to forsake their ballads,
ploughmen their whistling, and young women their \textit{hey, nonny, nonny},
and such fantasies, v. and 53, 54.

Craig’s collection of Scotch tunes, 620.

Croker (John Wilson), on country dances, 627.

Cromwell (Oliver)—Love of music, 415. Songs against, 416, 440,
\&c. Psalms to entertain ambassadors, 450.—\textit{See also Old Noll}.

Cromwell (Thomas) introduces \textit{Three-men’s-songs} into Italy, 52.
Ballad on his fall, 54. vii.

Crotch (Dr.), 7, 28, 529. His \textit{Specimens of Music}, vii.

Crowde, cruth, or crwth, described, 762.—\textit{See also Fiddle}.

Crumhorn, Cremorne, or Cremona, 247\textsuperscript{c}.

Cupid’s, or Cuper’s Gardens, opposite Somerset House, 727.

Curtal, a short sackbut or bassoon, 247\textsuperscript{a}.
\end{dcindex}
\end{fixedpage}%816
\pagebreak
\renewcommand\rectoheadertext{index of subjects.}

\setlength{\fixedpagewidth}{360pt}
\begin{fixedpage}%817
\rectoheader

\begin{dcindex}
Cushion dance at Court, 82. Described, 153, 154.

Custom-house duties on music, 321\textsuperscript{a}. Poets’ opposition to the duty
on French wines, 784.

“Dance after my pipe,” proverbial, 84.

\textsc{Dancing}.—“English lofty manner,” 303. English said to excel, 625.
Inns of Court, 424. Dancing schools in London, 625. Puritan
objections, 407. Country people capering and cutting measures,
rounds, and jigs, 153. Dancing to the harp, 35. Sarabands with
castanets, 497. Use and abuse of dancing and minstrelsy, 107.

\textit{Dancing Master} (The)—early editions described, x.\textsuperscript{a}

Daniel (George) — \textit{Merrie England}, 545, 607. Ballads from his
collection, 136, 223, and 243.

Dauney (W.)—Errors in dates, 517\textsuperscript{a}, 613, 616\textsuperscript{a}. “We be three poor
mariners,” \textit{not} a Branle de Poictu, 769.

Davenant (Sir W.)—First English opera, 478.

Davis (Moll), actress and singer, 525.

Dawson (Nancy), “the far-famed toast,” 718.

De Quinccy (T.) on country dances, 627.

Degrees in music conferred only at English Universities, 18.

Derbyshire hornpipes, 545, 741.

Descant, meaning of, 15, 16, 26. The best descanters often sorry
composers, 792.

Devil’s progress (The), original of \textit{The Devil's walk}, 442.

Dibdin (T.), 689, 738.

Ditties, 36 \textit{et passim}. Character of, 791.

Division (\ie, variations on an air) by country fiddlers, 475,
796.

Dixon (J. H.), 564, 656,—and \textit{errata} of 453.

Dorian mode, or scale, 13, 790.

Dowland (John), the great lutenist, 128, 245, 248.

Drayton (Michael) on \textit{Three-men's-songs}, 52. On hornpipes, 545.

Drinking to excess said to have been introduced from the
Netherlands, 679.

Drinking song, “the first of any merit,” 72.

Dulcimer, 35.

Dump, an old dance, 96, 210\textsuperscript{b}.

Dunmow flitch of bacon, in Chaucer’s time, 175, and note.

Dunstan (St.)—His skill in harp and song, 760. His organ, \&c.,
4.

D’Urfey (Tom), 618, 621 \textit{et seq}., 699, 769, 782.

Durham House, in the Strand, 317.

Dyce (Rev. A.), 66, 72.

Edward I.—His harpers and minstrels, 28 \textit{et seq}.

Edward II.—Decree about minstrels, 30. Female minstrel, 31.

Elizabeth—Music in her reign, 98 to 110, 321. Wish to be a
milkmaid, 777. After-celebration of her hirthday, 568\textsuperscript{a} and 713. The
imputed virginal book \textit{not} Queen Elizabeth’s, xiv.

Emigration of musicians at the end of sixteenth century, 249,
698\textsuperscript{a}.

English choral singing (250 boys in groups), 18. Six-men's-songs,
37,—and facsimile facing title-page. Three-men’s-songs, 52. Singing
with under-song, or base, 19. In canon, 18, 23. Chorus of sixty
voices (\ad 1465), 44. A hundred ladies, 766. Venetian Ambassador’s
description of English choristers, 51. Part-music (sixteenth
century), 108, 109. Music to the Watermen’s Roundel (\ad 1453), 783.

English manuscripts of early date in Germany, 759.

Engraving music on metal plates first practised in
England, 630. Originally for instrumental music, 632. Later for
single songs, 632.

Erasmus on English ladies, English music, and feasts, 51. On
pilgrimages to Walsingham, 122. Essex men for dancing the Hay, 135.

Evelyn on 24 fiddlers in the Chapel Royal, 468. On the guitar and
harp, 476 and 477. On Admiral Benbow, {641}.

Exchange (New Royal), in the Strand, 317\textsuperscript{a}.

Fairholt (F. W.) on soldiers in buff, 343.

Falconer \textit{not} the author of “Cease, rude Boreas,” 786.

Fantasies or fancies for instruments, 470, 471, 480. 

Farmer (T.), composer of excellent songs, 618, and note.

Feraboseo (Alfonso), born at Greenwich, 248\textsuperscript{b}. 

Fétis (M.), 23\textsuperscript{a}. On the violin, 760, 763, 764. 

Fielding (Henry) on Squire Western’s favorite . tunes, 265. 
Is the author of “The roast beef of old England,” 636,—and of “The dusky night,” 650.

\textsc{Fiddle}.—Woodcut of Anglo-Saxon fithele, 761. Sketch of the
history, 760 \textit{et seq}. “Fithelyn,” 33. “Fidel,” 35. “Fideler,” 49.
“Fiddyll, or Crowd,” 60. “Fyyele,” 764. “Vièle,” 762. “Vièlle,”
Viol, and Violin, 763. New treble Violin cost 20s., 247\textsuperscript{b}. Four sizes
of, but the same name, 244\textsuperscript{c}. Viols preferred to Violins, and why,
469. Viols, six strings—chest of Viols, 246. Violin and kit preferred
for dancing, 468. Violin in credit, and double notes on, 469. £100
for an old Viol, 486. Names of best English makers, 486. Fiddle and
Violin synonymous, 244\textsuperscript{c}. When first called Violins, 468. Europe
derives the Violin and bow (through the crwth and fithele) from
England, 761 to 764.— \textit{See also Kit}.

\textsc{Fiddlers}.—Bishop Earle’s character of a poor, 67. London full of,
108. Sang catches, 108,—and songs, 110,—political songs, 60. Obtruded
themselves into all companies, 250\textsuperscript{b} and 251\textsuperscript{a}. Silenced by Cromwell's
Parliament, 480. Fiddlers’money, 252. Country fiddlers’ division,
475, 796. Names of Charles the Second’s “four-and-twenty
fiddlers,” 469\textsuperscript{b}.

Flageolet, 481, 487.

\textsc{Flute}.—The English, 36\textsuperscript{b}, 244\textsuperscript{b}. Other notices, 31, 33, 48. Four
sizes of, 246. English said to excel, 631.

Foot of songs. \textit{See Burden.}

Foreigners (English thirst for), 475.

Foreign opinions of English music—Reign of Edward IV., Tetzel,
and Leo von Rozmital, 44, and note. Reign of Henry VIII., Sagudino,
Pasqualigo, and Erasmus, 51. Reign of Elizabeth, Hentzner, 245\textsuperscript{a}.
Reign of James I., Laurencini, 628,—and Orazio Busino, 631. Reign
of Charles I., the Sieur de la Serre, 472. During the Commonwealth,
Giovanni Battista Doni, 631. Reign of Charles II., Count Lorenzo
Magalotti, 472,—and M. Jorevin de Rocheford,
476.

Franco, reputed inventor of characters for time.  15, 16. 

Freemasons’ songs, 663. 

Freemen’s songs, or “King Henry’s mirth,” 52, 55\textsuperscript{b}, 62, 66, 67.

Friars, mendicant, sing and play the rote, 33. Know more of Robin
Hood than their Paternoster, 33. Songs against, 145, 274, 390.

Funeral music, 251. Dirges, 319.
\end{dcindex}
\end{fixedpage}%817
\pagebreak

\setlength{\fixedpagewidth}{360pt}
\begin{fixedpage}%818
\versoheader

\begin{dcindex}
Gale, Dr.—Song on his flogging the St. Paul’s boys, 779.

Galliard, a dance, described, 155 and 157. Allusions to, 127,
147, 242, 626.

Gentle craft (The).—\textit{See Shoemakers}.

Gipsies’songs and dances, 171. 

Glee, derivation of, 2\textsuperscript{a}. Playford’s collections of 484.

Gleemen (Anglo-Saxon), 2, 5.

Gibbons (Orlando)—Analysis of his “Fancies,” 470.

Giraldus Cambrensis on music, 18, 19, 20.

Gittern (The) Mentioned by Chaucer, 34. Strung with gut, 36, 102,
and 248. “Gitterons which are called Spanish vialles,” 764.

Gentlemen expected to play and sing at sight, 99, 100, 489. 
Reason assigned for the diminished cultivation of music among, 625.

George I.—Anti-Jacobite songs, 436.

Gleig (Rev. G. R.)—Note on a march, 519.

Goldsmith (Oliver) — Songs mentioned in his Essays, 772, 779.

“Good old cause” (meaning of), 781.

Graham (G. F.), 255, 615, 787.

Gregorian tones, or scales.—\textit{See Church scales}.

Guido d’Arezzo’s improvement in notation, 14.

Guitar in fashion in the reign of Charles II., 476. In Henry the
Eighth's time, called the Spanish Viol, 764.

Gwynn (Nell), allusions to, 282,325,501,605,785.

\bigskip
Halliwell (J. O.) on “What if a day,” 310.

Halsted (Miss) on Richard III., 200.

\textsc{Harp}.—Root of the word, 5. Handed round at supper by the
Anglo-Saxons for each to play and sing, 4. Allusions to the harp in
the Anglo-Saxon poem of \textit{Beowulf}, and the romance of \textit{Horn-Child}, 6,
20, and 21. Stories of Baldulph entering King Alfred’s camp, of
Alfred entering the Danish camp, and of Aulaff exploring that of King
Athelstan, by means of their haips, 4 and 5. Comparison of man to a
well-strung harp, 17. St. Dunstan’s enchanted harp, 760. Dancing to
the small harp strung with wire, 35. Harps of the sixteenth and
seventeenth centuries, common, double, and triple, and “the
incomparable improvement” of gut strings, 477 and 478. Harp much
esteemed in Charier the Second’s time, and played by gentlemen, 476,
477. Airs characterized by, 789.

\textsc{Harpers}.—Played with the nails in early times, 20 and 36. Their
dress, 44 and 45. Kings’ harpers, 10, 28. Harpers of noblemen,
bishops, and abbots, 11 and 29. Great number of harpers at the
Coronation of Henry V., 38,—and at the Cour plenière of Edward I.,
29. Blind harpers proverbial in sixteenth centuiy, 108.

Harvest songs and rejoicings, 579, 580, 667, and 745.

Hautboy, or Hoboy.—\textit{See Waits}.

Hawkins (Sir John)—Describes instruments chiefly from foreign'
books, 6\textsuperscript{a} and 247\textsuperscript{b}. Conjecture about “Old Simon the King” disproved
by dates, 263. “Mad Tom” \textit{not} by Lawes, 329,330. Tune for Chimes \textit{not}
by Jeffries, 778. His \textit{History of Music} compared with Burney’s, ix.

Hay, or Hey (a dance), 153, 368, and 787. Described, 629. “Hey de
gie,” 82.

Hazlitt (W.)—Contribution of a traditional song, 782.

Henry VII.—His numerous payments for music, 48, 49.

Henry VIII.—His skill in music, 50 and 53. Music of his reign, 50
to 56. “King Henry’s mirth,” or Freemen’s songs, 55 and 483. His
May-day celebrations, 132. Large number of musical instruments at his
death, 247\textsuperscript{b} and \textsuperscript{c}, and 248\textsuperscript{a}.

Heralds and minstrels compared, 30.

Herefordshire Morris dance, 130\textsuperscript{b} and 134.

Herrick—Six songs to one tune, 773.

Hill (Richard)—Songs and carols, fifteenth century, 50.

Hobby-horse dances, 130 and 601.

Hoops for ladies,—when the fashion commenced, 787.

Hornpipes, 544, 545, 585, 595, 626, 740, 741, and 792.

Hospitality,—why decayed, 778.

\textit{Household Words}, on Tom D’Urfey, 622.

Howard papers (The), song found among, 201.

Hunter (Rev. Jos.) on Robin Hood, 387. Beverley minstrels, 766.

Hunt’s-up (A)—Morning music, 61, 100. The custom continued for
the royal family, 251.

Hunting songs, 60, 61, 83, 320, 321, 322, 346.

Hurdy-gurdy \textit{not} the ancient rote, 768.

Husk (W. H.) on musical festivals, 481.

Hyde Park, 326, 327.

\textsc{Hymns to secular tunes}.—Early English, 765.
“Psalms \& songs of Zion to the tunes of a strange “land,” 111.
Church of England, 748. French, 748. Primitive Methodists’, 749. Old
Scotch, to English tunes, 62, 141, 148, 193, 373, 768.

\bigskip
\textsc{Inns or Court}.—Dancing compulsory, 328.
“Sweet and airy activity of the gentlemen,” 423. Dance round the
coal fire, 424. The Chancellorship through good dancing, 625. Rhymes
and revels of the bar, 329.

\textsc{Instrumental Music}.—The English claim of superiority, 472,—and
partial admission of, 631. More cultivated than in Italy, 473.
Opinions of Matthew Locke and Christopher Simpson, 475. Increase of
cultivation in James’s time, 244. Music in six parts, 244. Eight
parts, 471. Lighter kinds, 471. (\textit{See also under Virginal Music}.)
First, second, and third music at theatres, 631.—\textit{See also under
Foreigners' Opinions}.

Irish claims contested to “My lodging is on the cold ground,”
529. “The buff coat hath no
fellow,” 343. “The girl I left behind me,”
708 and 709. “Cruiskeen lawn,”770. “The pretty girl of Derby,”
771,—and “Off she goes,” 796.

Irish harp and Irish bards, 789\textsuperscript{a}.

Irish Hay (The), a dance, 84, 218.

Irish tunes,—the three earliest extant, 793.

\bigskip
Jacobites, 524.

Jacobites (tunes used by)—“Young Jemmy,” 524.

“the King shall enjoy his own again,” 434. “On yonder high,” 681.
“Rule, Britannia,” 687. “God save the King,” 705. “Green sleeves,”
775.

James I., —Fashion for instrumental music in his reign, 244.
General state of music, 244 to 254. Court masques, 473, 474. Song on
his creating 2000 knights, to raise money, 327. Proclamation on
lawful recreations, 406. Counterblast to tobacco, 563. Hunting songs,
321.

James II.—“Old Jemmy,” 524. “Lilliburlero” (and others to the
tune), 568, \textit{et seq}.
\end{dcindex}
\end{fixedpage}%818
\pagebreak

\setlength{\fixedpagewidth}{360pt}
\begin{fixedpage}%819
\rectoheader
 
\begin{dcindex}
Jew’s harp, 775.

Jigs danced by persons of all ranks, 495, 626, and 629. Jig
country-dances, 554. Northern jigs, 585 and 626. Jigging rhymes
called jigs, 495. Irish and Scotch jigs, 792.

John of Gaunt’s Court of Minstrels, 37.

John of Salisbury’s strictures on Church singers, 18.

Jones (Edward)—English airs in his \textit{Welsh Bards}, 64\textsuperscript{a}.

Jonson (Ben)—Laws of the Apollo Club, in verse, 250\textsuperscript{c} and 263.
Quotations from his works, \textit{passim}.

Jordan (Mrs.), composer of “The blue bells of Scotland,” 739.

\bigskip
Kemble (John Philip), 708.

Kent (Songs relating to), 90, 94, 567, 588.

Kiesewetter’s \textit{Geschichte der Musik}, 13 and 23\textsuperscript{a}.

Kissing (English custom of), 407, 408, and 780.

Kit (a small fiddle for dancing, and \textit{not} a cittern), 248, 203.

Kitchener (Dr.), mistake of, 492.

\bigskip
Ladies (Cultivation of music among), 100 and 101. Music a nursery
accomplishment, 487. Every house a pair of virginals, 486.
Tradesmen’s daughters, 486 and 625.

“\textsc{Lancashire} for Hornpipes,” 135, 544. Lancashire pipers and
fiddlers, 546. Lancashire lasses, 545.

Laud (Archbishop), Songs on, 411. Pamphlet against, 419\textsuperscript{b}.

Leather bottles (Various kinds of), 515.

Lent (Ballad on the observance of), 89, 195\textsuperscript{a}.

Lincolnshire bagpipes and round dances, 546.

Lira, the Italian name for a viol or violin, borrowed from the
Latin, and in use till “not many years before 1580,” 763.—\textit{See also
Lyra}.

Locke (John) on Brandenberg Church singing, 773.

Locke (Matthew) on instrumental music, 475. Composed music to the
first English opera, and acted in it, 478. Twenty years anterior to
Lully as a dramatic composer, 479. Notice of, by Pepys, 480. Composed
music for the public entry of Charles II., 479. His Psyche the first
opera printed (with music) in England, 479.

London (City of) advertise the boys in Bridewell and Christ’s
Hospital for husbandry, kitchen service, \&c commending their 
“toward qualities” in music, 98\textsuperscript{a}. Blue-coat boys singing at city
funerals, 251. Waits, 245\textsuperscript{a}, 550, and 618. Cries of London, 565, 589.

Loth-to-depart, a general name for any tunes at parting, 173.

Lovell (Thomas)'—Abuse of dancing and minstrelsy, 107.

\textsc{Lute}.—Described, 102. Mace on the old English lute, 485. Old
better than new, 485. How to keep a lute, 485. Not to be played with
the nails, 485. Price of new lute, £3,—old, up to £100, 485.
Lute-strings for presents, and sold by usurers to cheat spendthrifts,
103. Fame of English lutenists, 248. Theorbo enumerated among
instruments strung with wife, 248.

Lydgate’s songs and ballads, 40.

Lydian measure, 27.

Lyra, or Lira, the conventional Latin for instruments played upon
with a bow, from the ninth to the sixteenth century, 789\textsuperscript{a}.

Lysarden, a base cornet, or serpent, 247\textsuperscript{a}.

\bigskip
Macaulay (Lord), 699.

Macdonald (Flora), 681.

Mackay (Charles), 683, 722.

Macfarren (G. A.), xi. On Orlando Gibbons’s Fancies, 470.

Mad songs, 328 to 335. Why so many English, 334.

Madden (Sir Fred.), xii., 8, 22.

Mallett, 382, 687.

Mary (Queen), Ballad on, 513. Her complaint, 178.

Masses made on popular tunes, 105.

May-day and May-pole celebrations, 126, 131 to 135, 255, 296,
301, 303, 322, 377, 533, 543\textsuperscript{a}, 651, 671, 753, 777.

Measures, slow dances, 626\textsuperscript{a}.

Methodists (Primitive), or Ranters, 748, 749.

Milkmaids—Izaak Walton’s, 260. Overbury’s character, Beaumont and
Fletcher, Pepys, \&c., 281 and 282. Milkmaids’ dance, 282. Misson’s
description, 777. Their songs, 295 to 299, 349, and 428.

Millers’ songs, 589, 594. Chaucer’s miller plays the bagpipes,
34.

Milton’s skill in music, 410. His recommendation of music in
education, 410.

Minikins, small-sized lute and violin strings, 103\textsuperscript{b}.

\textsc{Minstrels}.—Account of, 1 to 11, and 28 to 47. Dress of, 45 and
82. Six kings of the minstrels, 29. Compared with heralds, 30.—\textit{See
also Beverley Minstrels}.

Monk (General), Songs on, 275, 286, 299, 444, 601.

Monmouth (Duke of), Songs on, 444, 523, 524, 785.

Moore (T.).—\textit{See Irish claims}.

Morley (T.) on gentlemen’s singing at sight, 99.

\textsc{Morris Dance}. — Description of, 130 to 135. Staines Morris, 125.

Mug-houses (The) of London and Westminster in 1724 described,
624.

Music a necessary accomplishment of a prince or hero, 3. (\textit{See
also under “Gentlemen,” and under
“Ladies.”}) Universality of music, 98 \textit{et seq}. Taught in charity
schools, 98\textsuperscript{a}. Pepys's servants, whether male or female, all musical,
489. Music a solace and recreation to labour, 579, 791, and 797.

Musical Antiquarian Society, works printed by, 245\textsuperscript{c}.

Musical festivals, origin of, 481.

Musical instruments, 247\textsuperscript{b} and 247\textsuperscript{c}, 248.—\textit{See also under each
name.}

Musical proverbs at Leckington, 35\textsuperscript{b}.

Musicians.—Five to be retained by an Earl, 246. Merchants, from
one to five, 247. Musicians’ right of visiting their patrons’
friends, 250.

\bigskip
Nashe (Thomas), Tunes and ballads named by, 107, 116.

National Music \textit{not} derived from the Church, 796. Divided into two
classes, 796. Characteristics of English, 789 \textit{et seq}. Test of a good
or bad tune, 797.

Neville (Lady)—Her virginal book, xii., 62, and 66. “Lady Frances
Nevill’s delight,” 398.

“New Wells,” various, 606.

Newmarket races in Charles the Second’s time, 562.

Nichols (J. G.), 698.

Northern songs, 379\textsuperscript{a}.

Notation of music, 14, 15, 16. Verses on, 35b. Singers sent from
Rome, to secure uniformity in singing, on account of the
imperfection of early notation, 765.—\textit{See also Facsimiles of MSS., and
Explanation}, xiii. to xv.

\textit{Notes and Queries}, 697, 699, 782.
\end{dcindex}
\end{fixedpage}%819
\pagebreak

\setlength{\fixedpagewidth}{360pt}
\begin{fixedpage}%820
\versoheader

\begin{dcindex}
Nottinghamshire hornpipes, 544. Song on Nottingham ale, 573.

\bigskip
Oates (Titus)—Song, “Oates well thrash’d,” 441.

Old King Cole—Who was he? 633.

Old Noll (Oliver Cromwell). 449.

Old Rowley (Charles II.), 449.

Old scales of music.—\textit{See Church scales}.

Old songs introduced into plays, 72.

Opera, the first English, 478. Not mentioned by Hawkins, and
Burney’s account all wrong, 479. Opera in England thirteen years
before France, 479.—\textit{See also Locke}.

\textsc{Organ}.—St. Dunstan’s, 4. Proverbs on, 35\textsuperscript{b}. £30 for a “pair,” 49.
Meaning of a \textit{pair} of organs, 49\textsuperscript{a}. Destruction of organs by Puritans,
413 and 414.

Oswald (J.)—Airs with his name often only arranged, 787.

Oxford music meetings during the Protectorate, 416. University
degrees in music, 18.

\bigskip
Packington (Sir John)—“Lusty Packington,” 123.

Pandore, a lute strung with wire, 244\textsuperscript{a} and 248.

Parry (John), editor of \textit{Welsh Melodies}, 665.

Part-songs—First printed collection, 55. Part-singing general in
England with all classes, 108, 109.

Pavan, a dance, described, 157 and 772. Spanish pavan, 241 and
776.

Peacham’s “Complete Gentleman” required to sing at sight and play
on the viol or lute, 99. Only men of “brutish stupidity” love not
music, 99. Lines addressed to Dowland, 128.

Pepys (Saml.)—House built by “the blind beggar of Bednall Green,”
159. My lord sings a song on the Rump Parliament, 231. On country
maids with music, 281. Nell Gwynn, 282. Ballads on Genl. Monk, 286.
“Old Courtier of the Queen’s,” 299. New Boyal Exchange, 317\textsuperscript{a}. A ball
at Court, 340. Music at a coffee-house—canon for eight voices, 480. Fire of
London—one boat in three had a pair of virginals, 486. His own
music—violin, lute, flageolet, song, and composition, 487. Music of
taverns and clubs, 488. Mr. Pepys’s four maids, and his servant-boy
all musical, 489. Mrs. Knipp’s songs and first notice of Scotch
music, 490. Much mirth with a ballet, 507.

Percy (Dr.) on Minstrelsy, 1, 5, 8, 9, 38, 39, 44.

Philippi (Pietro)—Peter Phillips, 248.

Pilgrims accompanied by singers and pipers, 34\textsuperscript{a}.

Pills to purge melancholy, \textit{passim}. Edition of 1719 made up by
D’Urfey, 510.

Pipe and tabour, 31, 248. Pipes of corn, 36. Anglo-Saxon pipes,
36\textsuperscript{a}.

Pipers, infinite number of, 96, 108. Piper to each parish, 791.
Paid by the parish for playing to menders of the highway, 791. Often
paid to relieve and quicken labour, 579.

Plackets, when maids began to wear, 518.

Plain-song and prick-song (meaning of), 51\textsuperscript{b}.

Playford (Honest John) and his publications, 423, 424, 482, 483,
484, and x.\textsuperscript{a}

Ploughmen whistling, 53, 138, 579, and 797.

“Poets infinite,” 96. Two hundred and thirty-three English
poets contemporary with Shakespeare, 105. Rejected ballad-metre from
the latter part of the sixteenth century, 621. Therefore their songs
are read, but rarely sung, 621.

Pope’s (Alex.) Letter on the popularity of D’Urfey’s songs, 622.

Portsmouth (Duchess of)—time of CharlesII., 325.

Power (James) on T. Moore’s alteration of tunes, 709.

Prynne on dancing, 407. On the “unloveliness of love-locks,” 409.
On Church singing, 403\textsuperscript{c}.

Psaltery, or Sawtrie, 6, 11, 34, 35. Described, 678.

Purcell (Henry), 570, 777, 786

Puritanism as affecting music, \&c., 401 \textit{et seq}.

Puttenham—On ballad music, 47, 60. On the difference of speech
beyond the Trent, 379\textsuperscript{a}. Song in an Interlude, 769.

\bigskip
Quadrivium of the schools—music included, 11.

\textit{Quarterly Review}, on Memoirs of Madame de Créqui, 692.

Quinible (To sing a),—\ie , descant in fifths, 34\textsuperscript{d}.

\bigskip
Raikes’s \textit{Diary}—An old fable revived, 692.

Raleigh’s (Sir Walter) Lamentation, 175.

Randal, of Chester (Rhymes of), 10, 33.

Rant (The), a dance, 554.

Rastell’s \textit{Four Elements}, on florid Church music, 51\textsuperscript{b}. Burdens of
songs quoted in, 53. Song in score, and barred, 541\textsuperscript{b}.

Ravenscroft’s collections of catches, \&c., 483.

Reading (John), 577\textsuperscript{a}.

Rebec, or Ribible, a three stringed fiddle, 34\textsuperscript{c}.

Recitative in Italy, 473. In England, 474. On the stage, 478.

Recorder (The), 35\textsuperscript{b}, 48, 246, and note \textsuperscript{a}.

Bed Bull play-house, 294.

Reels, 629.

\textit{Reliquiæ Antiquæ}, 25.

Richard I. and his minstrel, 9\textsuperscript{e}.

Richard III. (Song on), 144.

Rimbault (E. F.), xi., 239\textsuperscript{a} \textit{et passim}.

Ritson (Joseph), 1, 2, 8, 22, 59, 64, and 145.

Robin Hood (Songs of), 33, 387 \textit{et seq}.

Rochester (Earl of), Songs by, 427, 441, 506.

Roger of Coverley, or Calverley, 534.

Roland (Song of), 6, 7, 764, 765.

Rote, the Teutonic harp, \textit{not} the hurdy-gurdy, 766. Ancient
drawing of, 767.

Rounds—Descant on, 16. “Sumer is icumen in,” 21. Rounds sung by
all classes, 108, 109, 482, 564. Collections by Ravenscroft, Hilton,
and Playford, 483. Rounds, country dances, 171, 626, 627. Roundelay
synonymous with Round, 483. Roundelays and Ronndelets, 36. Rondel
(Chaucer), 36. Roundel (Gower), 37.

Rowe, and the Countess of Warwick, 493.

Rupert (Prince), 433, 601.

Russell (Lord William), 565, and note.

\bigskip
Sack, a \textit{sweet} wine, 679\textsuperscript{a}. Receipt for sack posset, in verse,
125\textsuperscript{b}.

Sackbut (a trombone), 35, and note.

Sailors’ songs, 78, 291, 292, 597, 598, 685, 715, 716, 729, 741,
and 743.

Salisbury (John of)—Strictures on Church singers, 9.

Saltery.—\textit{See Psaltery}.

Sandy's (W.), 513, 751, 752.

Scotch claims.—To “Cold and raw,” 307, and note. “The broom, the
bonny broom,” 459. “Blink o’er the burn,” 504 and 505\textsuperscript{a}. “I’ll never
love
thee more,” 379. “She rose and let me in,”
510. “Border ditty,” 522. “I lo’e na a laddie,” 529. “The mautman
comes a Monday,” 535.
\end{dcindex}
\end{fixedpage}%820
\pagebreak


\setlength{\fixedpagewidth}{360pt}
\begin{fixedpage}%821
\rectoheader

\begin{dcindex}
\noindent “Jock o’ Hazledean,” 575. Numerous songs, 609 to 620. “Woo’d and
married,’’ 728. “Saw you my father?” 731. “Blue hells of Scotland,”
739. “John Anderson, my jo,” 770. “John, come kiss me,” 771. “Muirland
Willie,” 786. “Ye banks and braes,” 794, “Comin’ thro’ the rye,” 795.
“Jenny’s bawbee,” 795.

Scotch manuscripts (Dates of) mistaken—Skene MSS., 614\textsuperscript{a} and 615.
Blaikie’s, 772.

Scotch tunes in England \textit{not} before the reign of Charles II., 459,
610. First notices of, 490. Scotch national music first printed in
Scotland in the eighteenth century, 490. Necessity of sifting the
collections, if to be limited to what is Scotch, 796.

Scott (Sir Walter), 436, 447, 778.

\textit{Scouring of the White Horse}, 729.

Secular music in sacred offices.—\textit{See Hymns}.

Sempill (Francis), of Beltrees, Songs erroneously attributed to,
511, 620, note.

\textsc{Shakespeare}.—“Hunt's up,” 61. “Good nights,” 138, 174. “Cittern
head,” 101\textsuperscript{c}. Canaries, 368. Recorder, 246. “Broken music,” 246\textsuperscript{c}.
Fading, 234. Bought his house in Blackfriars from a minstrel, 252.
Dr. Wilson, of Oxford, the “Jack Wilson” of his plays, 372. His love
of music, 98. Sonnet, 128. On burdens of songs, 222, 223. Songs or
tunes alluded to:—“Fortune, my foe,” 162; “When Arthur first,” 199;
“King Lear and his three daughters,” 199; “It was a lover,” 205 and
774; “O willow, willow,” 207 and 774; “Whoop, do me no harm, good
man,” 208 and 774; “O mistress mine,” 209; “Heart’s-ease,” 210; “Jog on
the footpath way,” 211; “Farewell, dear love,” 204\textsuperscript{a}; “My heart is
full of woe,” 210, and note; “To shallow rivers,” 213; “Three merry
men be we,” 216; “I loathe that I did love,” 216; Gravedigger’s song,
200; “Peg a Ramsey,” 218; “Light o’ Love,” 221; Fool’s song in
\textit{Twelfth Night}, and in \textit{King Lear}, 225; “The sick tune,” 226; “Good
morrow! ’tis St. Valentine’s Day,” 227; “Green sleeves,” 227; “Come
o’er the bourn, Bessy, to me,” 505\textsuperscript{a}; “Bonny sweet Robin is all my
joy,” 233; “How should I your true love know?” 236; “And will he not
come again?” 237; “Death, rock me asleep,” 237; “Thou canst not hit
it,” 239; “Robin Hood, Scarlet, and John,” 393; “King Cophetua,” 591;
“Oh! for a husband,” 782. Matthew Locke’s Music to \textit{The Tempest}, 479\textsuperscript{b}.
Dr. Rimbault’s proposed publication of ancient music relating to his
plays and poems, 239\textsuperscript{a}.

\textsc{Shawm} (parent of the clarionet), 35\textsuperscript{b}, 246, 768.

Sheale (Richard), the minstrel, 45 to 47, and 198.

Sherwood Forest, 95, 117 acres, 780.

Shoemakers (the “gentle craft”) all supposed to sing, 98, 277.

Shropshire Rounds, 599.

Sidney (Sir Philip)—Defence of poetry, 46.

“Sing well-a-day,” or “Sing well-away,” proverbial,” 175.

Singing, Eight reasons why every one should learn, 99.

Singing servants recommended by Tusser, 99. Pepys acts upon the
advice, 489.

Singing at sight a necessary accomplishment for gentlemen, 99.

Six-men’s-songs, 37.

Skelton—To sing “on the booke,” 52. “Hankin bovy,” 73. “Heve and
how, rombelow,” 783.

Smith (J. Stafford)—Imperfect barring of old music, xiv.
Attributes “I cannot eat but little meat” erroneously to Weelkes, 72.

Soldiers’ songs, 77, 144, 299, 334, 342, 355, 596, 669, 708.
Rogues’ march, 711.

Southey on the asserted decay of hospitality, 778.

Spanish viols, meaning guitars, 764.

Spencer—Allusions to the hornpipe, 545. Rote, 768.

Squire (Chaucer’s) makes songs and plays the flute, 33.

Stenhouse (W.), Mis-statements and blunders of, 343\textsuperscript{a}, 615, 616,
and note, 619, 740, 771, \&c., \&c.

Sternhold’s Psalms, written for the “laying apart ungodly songs
and ballads,” v. Courtiers did not sing them, “only some few
excepted,” 54.

Stubbes, the Puritan, on May games, 133.

Suckling (Sir John), Lampoons upon, 67, 359, and note. His
charming ballad on a wedding, 360.

Suffolk harvest customs, 118, 119.

Symphony sometimes meaning a tabor or drum, 247\textsuperscript{c}.

\bigskip
Tailors.—Never trust one that does not sing, 99. Tailors’catches,
109.

Tantivy boys and Tantivitiers, 781.

Taunton Dean (Virgins of), 444.

Tavern music, 480, 481, 488.—\textit{See also Fiddlers}.

Tetrachords, 13\textsuperscript{a}.

Thomson (James)—Rule, Britannia, 687.

Tympanum sometimes signifying a bagpipe, 760\textsuperscript{b}.

Tinkers singing catches, 110. Their (songs, 187 and 332. Tom
Tinker, 353.

Tobacco, Songs on, 563.

Tory (Origin of the word), 427.

Trench (Rev. R. Chevenix, D.D.), 628.

Trenchmore, a Court dance, 82, 769.

Trombone, 35.

Trumpet, 35.

Tumble-down-Dick (The sign of) means Richard Cromwell, 449.

Turner (Sharon) on “The children in the wood,” 200.

Turpin, the higwayman, 661.

Tutbury court of minstrels, 37.


\bigskip
Venetian Ambassadors on the Court of Henry VIII. 50, 51.

Vernon (Admiral), 597, 657.

Vicar of Bray—The story not historical, 787.

Vièle and Vielle, 762, 763..—\textit{See also Fiddle}.

Viols (6ix strings, tuned by fourths, and with frets). A chest of
viols contained four or five of different sizes, 246. Why once
preferred, for scholastic music, to violins, 469.—\textit{See also Fiddle}.

Violin (four strings, tuned by fifths) Origin of, 76 \textit{et seq}.—\textit{See
also Fiddle}.

Virelay—Chaucer, 36. Gower, 37.

Virginals—time of Henry VII., 35\textsuperscript{b}. Played by both sexes, 100, 101.
Described, 103. Great number in Charles II.’s time, 486. Virginal
jacks, 82, 104. One manuscript of virginal music, containing 296
pieces, of various Ieugths, the first piece with 30 variations, xv.

\bigskip
Wadlow (Simon), 263\textsuperscript{b} and \textsuperscript{c}, 265.

Wait or Waight (the hoboy), 31.

Waits originally musica atchmen, 547 and 786. Afterwards
musicians of towns and corporations, 49, 249, 548, 549.

Wales (Laws of), as to the harp, 5.

Walpole (Horace) on the unpoetical age, 624. On English
country-dances in ltaly, 307.
\end{dcindex}
\end{fixedpage}%821
\pagebreak

\setlength{\fixedpagewidth}{360pt}
\begin{fixedpage}%822
\versoheader

\begin{dcindex}
Walsingham (Pilgrimages to), 121, 122. 

Watchmen singing catches,
110.

Weaver (Thomas)—Songs against Puritans, 420. 

Weavers singing
catches, 109.

Webbe’s \textit{Discourse of English Poetrie}, 375. 

Wedderburn’s \textit{Complaint of Scotland}.—Tunes mentioned, 56, 60.

Weddings.—Music, 250, 251. 

Sir J. Suckling’s song on, 360. 

Welsh bards acted as heralds, 30. Statute against, 38.

\textit{Westminster Review} (The), on Robin Hood, 387.

Westmoreland ballad, 458.

Whittington (Sir Richard), 515, 785.

William III. (Proclaiming of), 493. “Lilliburlero,” \&c., 568 \textit{et seq}.

Willmott (The Rev. R. A.) on George Wither, 422.

Winchester custom, 577.

Wines (Early prices of), 679\textsuperscript{a}.

Wolfe (General), Song erroneously attributed to, 669.

Wolsey (Cardinal)—Masquers surprised, vii. and viii. “Wolsey’s
wild,” 86. Article on Wolsey, 775.

Wood (Anthony) on the Violin, 469.

Worcestershire Morris dance, 130. Worcestershire for bagpipes,
135.

Wright (Thomas), 9\textsuperscript{e}, 22, 41, 387.

\bigskip
Yellow starch,—why the fashion discontinued, 464.

York! for my money, 229 and 775. Yorkshire Volunteers, 684.

\bigskip
Zarlino learnt to write in the key of C, from hearing vulgar
musicians, 27.

\end{dcindex}

\centerrule

\end{fixedpage}%822
\pagebreak


\setlength{\fixedpagewidth}{360pt}
\begin{fixedpage}%


\headingfive{ERRATA.}
\begin{hangparas}{10pt}{1}
Page 5, line 5, for “is crwth,” read “is said to be crwth.”

p. 6, l. 34, for “armée,’’ read “armé.”

p. 6, l. 3, of note \textsuperscript{a}, for “psaltry,” read “rote.’’

p. 7, last line but one, for “apochryphal,” read “apocryphal.”

p. 18, l. 6, after “Saintwix,” add “alias St. Viste or St. Just;”
and for “King’s College,” read “King’s Hall.” 

p. 18, last line of note \textsuperscript{c}, for “indicii,” read “judicii.” 

p. 23, l. 22, cancel from “The rote” to “wheel,” in the following line.

p. 31, l. 33, for “Beverley Minster,” read “St. Mary’s Church, Beverley.” 

p. 45, l. 5, for “1841,” read “1481.”

p. 55, l. 8, for “Songs in parts,” read “Songs in three or four
parts.” 

p. 63, l. 3, for “Robin Hood, Robin Hood, said little John,” read “Heave and ho, rumbelow.”

p. 77, l. 2, of “We be three poor mariners,” for “about 1630,”
read “about 1700.”

p. 87, l. 11, for “1698,” read “1699.”

p. 183, l. 8, for “ii. 142,” read “iii. 142.”

p. 183, cancel “slowly and,” to the time of the tune.

p. 185, cancel “slowly and,” to the time of the tune.

p. 207, l. 4, for “and old thing,” read “an old thing.”

p. 232, l. 3, should be “and all Pretenders shake for \textit{dread},” not
“speed.”

p. 235, last line but one, after “the Fading is,” add “said to
be.”

p. 255, last line, for “The hunting,” read “The new hunting.”

p. 342, l. 2, should be “Her tongue it is so \textit{bold},” not “loud.”

p. 343, note\textsuperscript{a}, last line, first column, for ‘\textit{not even} one,” read
“\textit{only} one” and in last line, second column, for “all \textit{of} his,” read
“all his.” 

p. 379, l. 17, for “dated 1695,” read “date about 1745.” 

p. 435, l. 7, cancel “time of Charles I.” 

p. 443, l. 1, for “gobling crew,” read “Goblin crew.” 

p. 453, l. 5 and 6, for “the late George Macfarren,” read “Mr. J.
W. Dixon.” 

p. 486, l. 9, for “sixteenth,” read “seventeenth.” 

pages 660, 661, and 662, time of the three tunes to be
“jovially,” not “moderate time.” 

p. 779, l. 9, for “The was,” read “There was.”

\centerrule
\headingfive{ERRATA IN MUSIC.}

p. 27, l. 1, treble, last note but three to be F, not G.

p. 56, bar 5, treble, the first two notes of the melody to be G
G, not B B.

p. 58, bar 7, treble, last chord, cancel the upper D.

p. 79, bar 1, treble, the D (under G) to be B, a third lower.

p. 151, line 3, bass, the first note to be C, not B.

p. 159, bar 2, bass, the under note in the third chord to be G,
not F.

p. 233, bar 7, treble, the first note of the melody to be G, not
A.

p. 642, bar 2 of last line, bass, the chord should be C C, not B
B.

p. 644, line 3, treble, first chord to be D B, not E B.

p. 658, bar 3, treble, the last note to be C, not B.

\end{hangparas}
\end{fixedpage}%
\pagebreak



