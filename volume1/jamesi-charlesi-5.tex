%339
\changefontsize{0.96\defaultfontsize}

\settowidth{\versewidth}{But still did neglect him the more he did moan;}
\begin{dcverse}\indentpattern{01010101334}
\begin{patverse}
She gave no ear unto his cry,\\
But still did neglect him the more he did moan;\\
Though he did entreat, she still did deny,\\
And earnestly pray him to leave her alone.\\
Never, never, cries Apollo,\\
Unless to love thou wilt consent,\\
But still, with my voice so hollow,\\
I’ll cry to thee, while life be spent.\\
But if thou turn to me,\\
’Twill prove thy felicity.\\
\textit{Pity, O Daphne, pity me}, \&c.
\end{patverse}

\begin{patverse}
Away, like Venus’s dove she flies,\\
The red blood her buskins did run all adown,\\
His plaintive love she still denies, \\
Crying, Help, help, Diana, and save my renown.\\
Wanton, wanton lust is near me,\\
Cold and chaste Diana, aid!\\
\columnbreak
Let the earth a virgin bear me,\\
Or devour me quick a maid.\\
Diana heard her pray,\\
And turn'd her to a Bay.\\
\textit{Pity, O Daphne, pity me}, \&c.
\end{patverse}

\indentpattern{010101013300}
\begin{patverse}
Amazed stood Apollo then, \\
While he beheld Daphne turn’d as she desir’d,\\
Accurs’d am I, above gods and men,\\
With griefs and laments my senses are tir’d.\\
Farewell! false Daphne, most unkind,\\
My love lies buried in thy grave,\\
Long sought I love, yet love could not find,\\
Therefore is this thy epitaph:\\
\vleftofline{“}This tree doth Daphne cover,\\
That never pitied Lover.” \\
Farewell, false Daphne, that would not pity me,\\
Although not my love, yet art thou my Tree.
\end{patverse}
\end{dcverse}

\backskip{1}

\musictitle{Come You Not From Newcastle?}

This beautiful and very expressive melody is to be found in \textit{The Dancing
Master}, from 1650 to 1690, under the title of \textit{Newcastle}. In \textit{The Grub Street
Opera}, 1731, it is named \textit{Why should I not love my love}? from the burden of the
song. The following fragment of the first stanza is contained in the folio manuscript
formerly in the possession of Bishop Percy, p. 95. See Dr.~Dibdin’s
\textit{Decameron}, vol. 3.
\settowidth{\versewidth}{Come you not from Newcastle?}
\begin{dcverse}\begin{altverse}
\vleftofline{“}Come you not from Newcastle?\\
Come you not there away?\\
O met you not my true love,\\
Ryding on a bonny bay?
\end{altverse}

\begin{altverse}
Why should I not love my love?\\
Why should not my love love me?\\
{\quad *\quad *\quad *\quad *\quad *\quad *\quad }
\end{altverse}
\end{dcverse}
It is quoted in a little black-letter volume, called “The famous Historie of
Fryer Bacon: containing the wonderfull things that he did in his life; also the
manner of his death; with the lives and deaths of the two Conjurers, Bungye
and Vandermast. Very pleasant and delightfull to be read.” 4to., \textit{n.d.} “Printed
at London by A. E., for Francis Grove, and are to be sold at his shop at the
upper-end of Snow Hill, against the Sarazen’s Head:”—

“The second time, Fryer Bungy and he went to sleepe, and Miles alone to watch
the brazen head; Miles, to keepe him from sleeping, got a tabor and pipe, and being
merry disposed, sung this song to a Northern tune of \textit{Cam’st thou not from Newcastle}—
%\settowidth{\versewidth}{To couple is a custome,}
\begin{dcverse}\begin{altverse}
\vleftofline{“}To couple is a custome,\\
All things thereto agree;\\
Why should not I then love?\\
Since love to all is free.
\end{altverse}

\begin{altverse}
But Ile have one that’s pretty,\\
Her cheekes of scarlet dye,\\
For to breed my delight,\\
When that I ligge her by.
\end{altverse}

\begin{altverse}
Though vertue be a dowry,\\
Yet Ile chuse money store:\\
If my love prove untrue,\\
With that I can get more.
\end{altverse}

\begin{altverse}
The faire is oft unconstant,\\
The blacke is often proud;\\
Ile chuse a lovely browne;\\
Come, fidler, scrape thy crowd.
\end{altverse}

\begin{altverse}
Come, fidler, scrape thy crowd,\\
For \textit{Peggie} the browne is she\\
Must be my bride; God guide\\
That Peggie and I agree.”
\end{altverse}
\end{dcverse}
\changefontsize{0.98\defaultfontsize}
\pagebreak
%340


I have been favored by Mr. Barrett with a song, “O come ye from Newcastle?”
as still current in the North of England; but, doubting its antiquity, I have not
thought it desirable to print it in this collection.

\musicinfo{Rather slow and with expression.\scfootnote{
The two last lines are supplied from a song written to complete the fragment, by the late Mr. George Macfarren.}}{}

\includemusic{chappellV1179.pdf}

\musictitle{Cuckolds All A Row.}

This tune is to be found in every edition of \textit{The Dancing Master}. Pepys
mentions it in the following account of a court ball, in the reign of Charles II.:—

“31 Dec., 1662. By and bye comes the King and Queene, the Duke and
Duchess, and all the great ones; and after seating themselves, the King takes out the
Duchesse of York; and the Duke, the Duchesse of Buckingham; the Duke of
Monmouth, my Lady Castlemaine; and so on, other lords other ladies; and they
danced the Bransle. After that, the King led a lady a single Coranto: and then the
rest of the lords, one after another, other ladies: very noble it was, and great pleasure
to see. Then to Country-dances; the King leading the first, which he called for,
which was \textit{Cuckolds all a row}, the old dance of England.”

\changefontsize{1.00\defaultfontsize}
It became a party tune of the Cavaliers, \pagebreak who sang the songs of \textit{Hey, boys, up
%341
go we}, and \textit{London’s true character}, to it. The latter, abusing the Londoners for
taking part against the King, and commencing, “You coward-hearted citizens,”
is contained in \textit{Rats rhimed to death, or The Rump Parliament hanged in the
Shambles},~1660; and in both editions of \textit{Loyal Songs written against the Rump
Parliament}.

The tune is mentioned in the old song, \textit{O London is a fine town}; and one with
the burden is contained in \textit{Wit and Drollery}, 1661. The latter is reprinted (to
the tune of \textit{London is a fine town}) in \textit{ Pills to purge Melancholy}, ii. 77, 1700, and
iv. 77, 1719.

The following, on the miseries of married life, is from a black-letter ballad,
“printed by M.P. for Henry Gosson, on London Bridge, neere the gate,” and
signed Arthur Halliarg. A copy is in the Roxburghe Collection, i. 28; and
it is reprinted in Evans’ \textit{Old Ballads} i. 170 (1810). I haye omitted four stanzas,
the remainder being sufficient to tell the story. “The cruel Shrew; or The
Patient Man’s Woe:
\settowidth{\versewidth}{Straightway she such a noise will make}
\begin{scverse}
Declaring the misery and great pain,\\
By his unquiet wife, he doth daily sustain.”
\end{scverse}

To the tune of \textit{Cuckolds all a row}.

\musicinfo{Moderate time.}{}

\smallskip %tight to music

\includemusic{chappellV1180.pdf}

%\changefontsize{\defaultfontsize}
\pagebreak
%342

\settowidth{\versewidth}{Her morning’s draught well spiced straight }
\begin{dcverse}\begin{altverse}
She never lins her bawling,\\
Her tongue it is so loud,\\
But always she’ll be railing,\\
And will not he controlled:\\
For she the breeches still will wear,\\
Although it breeds my strife;\\
If I were now a bachelor,\\
I’d never have a wife.
\end{altverse}

\begin{altverse}
Sometimes I go in the morning\\
About my daily work,\\
My wife she will be snorting,\\
And in her bed she’ll lurk,\\
Until the chimes do go at eight,\\
Then she’ll begin to wake,\\
Her morning’s draught well spiced straight \\
To clear her eyes she’ll take.
\end{altverse}

\begin{altverse}
As soon as she is out of bed,\\
Her looking-glass she takes,\\
(So vainly is she daily led),\\
Her morning’s work she makes\\
In putting on her brave attire,\\
That fine and costly be;\\
While I work hard in dirt and mire:\\
Alack what remedy?
\end{altverse}

\begin{altverse}
Then she goes forth a gossiping\\
Amongst her own comrades;\\
And then she falls a boosing\\
With all her merry blades;\\
When I come from my labour hard,\\
Then she’ll begin to scold,\\
And call me rogue without regard;\\
Which makes my heart full cold.
\end{altverse}

\begin{altverse}
When I, for quiet’s sake, desire\\
My wife for to be still,\\
She will not grant what I require,\\
But swears she’ll have her will;\\
Then if I chance to heave my hand,\\
Straightway she’ll murder cry;\\
Then judge all men that here do stand,\\
In what a case am I.
\end{altverse}

\begin{altverse}
And if a friend by chance me call\\
To drink a pot of beer,\\
Then she’ll begin to curse and brawl,\\
And fight, and scratch, and tear;\\
And swears unto my work she’ll send\\
Me straight without delay;\\
Or else with the same cudgel’s end,\\
She will me soundly pay.
\end{altverse}

\begin{altverse}
Then is not this a piteous cause,\\
Let all men now it try,\\
And give their verdicts, by the laws,\\
Between my wife and I;\\
And judge the cause, who is to blame.\\
I’ll to their judgment stand,\\
And be contented with the same,\\
And put thereto my hand.
\end{altverse}

\begin{altverse}
If I abroad go anywhere,\\
My business for to do,\\
Then will my wife anon be there\\
For to increase my woe;\\
Straightway she such a noise will make\\
With her most wicked tongue,\\
That all her mates, her part to take,\\
About me soon will throng.
\end{altverse}

\begin{altverse}
Thus am I now tormented still\\
With my most wretched wife;\\
All through her wicked tongue so ill,\\
I am weary of my life:\\
I know not truly what to do,\\
Nor how myself to mend,\\
This lingering life doth breed my woe,\\
I would ’twere at an end.
\end{altverse}

\begin{altverse}
O that some harmless honest man,\\
Whom death did so befriend.\\
To take his wife from off his hand,\\
His sorrows for to end,\\
Would change with me to rid my care.\\
And take my wife alive,\\
For his dead wife, unto his share!\\
Then I would hope to thrive.
\end{altverse}
\end{dcverse}

\musictitle{The Buff Coat Has No Fellow.}

In Fletcher’s play, \textit{The Knight of Malta}, act iii., sc. 1, there is a “Song by
the Watch,” commencing thus:—
\settowidth{\versewidth}{Sit, soldiers, sit and sing, the round is clear,}
\begin{scverse}
\vleftofline{“}Sit, soldiers, sit and sing, the round is clear,\\
And cock-a-loodle-loo tells us the day is near;\\
Each toss his can until his throat be mellow,\\
Drink, laugh, and sing \textit{The soldier has no fellow}.”
\end{scverse}

The last line is repeated in three out of the four verses or parts, and I suppose
\textit{The soldier has no fellow} to have been then a well-known song.
\changefontsize{1.06\defaultfontsize}
\pagebreak
%343

Various ballads were written to a tune called \textit{The buff coat has no fellow} (see,
for instance, Pepys Coll., iii. 150; Roxburghe, i. 536; \&c.), and as the buff
coat was a distinguishing mark of the soldier of the seventeenth century, if the
words could be recovered, it might prove to be the song in question.

“In the reign of King James I.,” says Grose, “no great alterations were made
in the article of defensive armour except that the \textit{buff coat}, or jerkin, which was
originally worn under the cuirass, now became frequently a substitute for it, it
having been found that a good \textit{buff leather} would of itself resist the stroke of a
sword; this, however, only occasionally took place among the light-armed cavalry
and infantry, complete suits of armour being still used among the heavy horse.”—
\textit{Military Antiquities}, 1801, 4to., ii., 323. I have been favored with the following
note on the same subject by F. W. Fairholt, F.S.A.:—“The buff coat was
peculiarly indicative of the soldier. It first came into use in the early part of
the seventeenth century, when the heavier defensive armour of plate was discarded
by all but cavalry regiments. The infantry, during the great civil wars
of England, were all arrayed in buff coats; and in Rochester Cathedral are
still preserved some of these defensive coverings, as worn by Oliver’s soldiers
in their unwelcome visits there; as well as the bandoleers worn over them, to hold
the charges for muskets. The officers and cavalry at this time only added the
cuirass; the leather coat was frequently very thick and tough, and a defence
against a sword cut. The foreign, as well as the English armies, about this time,
discarded heavier armour; and the prints by Gheyn, of Low-Country troopers, as
well as those by Ciartes, of the soldiers of the French King, are all habited in
the buff coat, which displays, in the rigidity of its form, its innate strength.”
Grose gives an engraving of those that were worn over corslets, from one that
belonged to Sir Francis Rhodes, Part., of Balbrough Hall, Derbyshire, in the
time of Charles I.

The tune, \textit{The buff coat has no fellow}, is to be found in the fourth and every
subsequent edition of \textit{The Dancing Master};\footnote{\textit{}
Mr. Stenhouse, in his notes to Johnson’s \textit{Scot’s Musical
Museum}, asserts that this air is to be found in Playford’s
\textit{Dancing Master} of 1657, a book which he quotes constantly,
and which, I am convinced, he never saw. Having
tested all his references to that work, I have no hesitation
in saying that not even one of the airs he mentions
is to he found in it. Mr. Stenhouse had before him one
of the last editions of vol. i. of \textit{The Dancing Master},
printed by Pearson and Young, between 1713 and 1725,
and consisting of 358 pages, to which only can \textit{all} of his
quotations be referred.}
 in the earlier editions as \textit{Buff coat},
and afterwards as \textit{Buff coat, or Excuse me}. The following list of ballad-operas, in
all of which songs may be found that were written to the tune, sufficiently proves
its former popularity:—\textit{Polly; The Lottery; An Old Man taught Wisdom; The
Intriguing Chambermaid; The Lovers’ Opera; The Bay’s Opera; The Lover his
own Rival; The Grub Street Opera; The Devil of a Duke, or Trapolin’s Vagaries;
The Fashionable Lady, or Harlequin’s Opera, The Generous Freemason; and
The Footman}.

\changefontsize{0.96\defaultfontsize}
This popularity extended to Ireland and Scotland; and although, in its old
form, purely English in character, the air has been claimed both as Irish and as
Scotch. T. Moore appropriated it, under the name of \textit{My husband’s a journey to 
Portugal gone}, although in the opinion of \pagebreak Dr. Crotch, Mr. Wade, and others, “it is
%344
not at all like an Irish tune.” In Scotland it has been claimed as \textit{The Deuks
dang o'er my Daddie}, and again disclaimed by Mr. George Farquhar Graham,
editor of Wood’s \textit{Songs of Scotland}, who “freely confesses his belief that the air
is not of Scottish origin.” iii. 165.

All the oldest copies of \textit{Buff coat} begin with three long notes, which seem to
require corresponding monosyllables for the commencement of the words. The
line I have quoted from \textit{The Knight of Malta} suggests a commencement somewhat
in the following manner:—

\musicinfo{Boldly.}{}

\includemusic{chappellV1181.pdf}

I should add, that in some copies of \textit{The Dancing Master} the tune is in common
time.

In later versions, where the long notes at the commencement are split into
quavers (as in many of the ballad-operas), the bold character of the tune is lost,
and it becomes rather a pretty than a spirited air. This change seems to be
owing to the monosyllabic commencement having been discarded in the ballads
which were written to it: as, for instance, in the following, from the Roxburghe
Collection, i. 536:—“The merry Hostess; or—
\settowidth{\versewidth}{A pretty new ditty, compos’d on an hostess that lives in the city.}
\begin{scverse}
A pretty new ditty, compos’d on an hostess that lives in the city.\\
To wrong such an hostess it were a great pity,\\
By reason she caused this pretty new ditty.
\end{scverse}
To the tune of \textit{Buff coat has no fellow}.”
\settowidth{\versewidth}{Come all that love good company,}
\begin{dcverse}\begin{altverse}
\vleftofline{“}Come all that love good company,\\
And hearken to my ditty;\\
’Tis of a lovely hostess fine,\\
That lives in London city;
\end{altverse}

\begin{altverse}
Who sells good ale, nappy and stale,\\
And always thus sings she:\\
My ale was tunn’d when I was young,\\
And but little above my knee,” \&c.
\end{altverse}
\end{dcverse}
The above is printed in Evans’ Collection, i. 150 (1810).
\changefontsize{1.01\defaultfontsize}
\pagebreak
%345

In several of the ballad-operas, the tune, whether under the name of \textit{Buff
coat}, or \textit{Excuse me}, commences thus (see, for instance, \textit{The Generous Freemason},
1731):—

\includemusic{chappellV1182.pdf}

And in some more modern versions thus:--

\includemusic{chappellV1183.pdf}

When the key-note is heard three times in equal succession at the end of a tune,
it is considered to be characteristic of Irish music; but that peculiarity often arises,
as in the last example, from too many syllables in the words adapted to the air.

\musictitle{A Begging We Will Go.}

In the Bagford Collection, a copy of this song, in black-letter, is entitled “The
Beggars’ Chorus in \textit{The Jovial Crew, to an excellent new tune}.” Brome’s comedy,
\textit{The Jovial Crew, or The Merry Beggars}, was acted at the Cockpit in Drury
Lane, in 1641, and I suppose the song to have been introduced, as it is
not contained in the printed copy of the play. One of the Cavaliers’ ditties,
“Col. John Okie’s Lamentation, or a Rumper cashiered,” is to the tune of
\textit{A begging we will go}. This was published on the 28th March, 1660, and a copy
may be seen among the King’s Pamphlets, Brit. Mus.

\textit{A begging we will go} is printed, with the music, in book v. of \textit{Choice Ayres
and Songs to sing to the Theorlo or Bass Viol}, fol. 1684; in \textit{180 Loyal Songs},
3rd edit., 1685; in \textit{ Pills to purge Melancholy}; \&c. It is sometimes entitled
\textit{The Jovial Beggars}.
\settowidth{\versewidth}{And a begging, we will go, we’ll go, we’ll go,}
\begin{dcverse}\begin{patverse}
\indentpattern{121200}
\vin \vleftofline{“}There was a jovial beggar,\\
He had a wooden leg,\\
Lame from his cradle,\\
And forced for to beg.\\
And a begging, we will go, we’ll go, we’ll go,\\
And a begging we will go!
\end{patverse}

\indentpattern{12125}
\begin{patverse}
\vin A bag for his oatmeal,\\
Another for his salt;\\
And a pair of crutches\\
To show that he can halt;\\
And a begging, \&c.
\end{patverse}

\begin{patverse}
\vin A bag for his wheat,\\
Another for his rye;\\
A little bottle by his side\\
To drink when he’s a dry, \&c.
\end{patverse}

\begin{patverse}
\vin Seven years I begg’d\\
For my old master \textit{Wild},\\
He taught me to beg\\
When I was but a child, \&c.
\end{patverse}

\begin{patverse}
\vin I begg’d for my master,\\
And got him store of pelf;\\
But now, Jove be praised,\\
I’m begging for myself, \&c.
\end{patverse}

\begin{patverse}
\vin In a hollow tree\\
I live, and pay no rent;\\
Providence provides for me,\\
And I am well content, \&c.
\end{patverse}

\begin{patverse}
\vin Of all the occupations,\\
A beggar’s life’s the best;\\
For whene’er he’s weary.\\
He’ll lay him down and rest, \&c.
\end{patverse}

\indentpattern{121200}
\begin{patverse}
\vin I fear no plots against me,\\
I live in open cell;\\
Then who would be a king\\
When beggars live so well.\\
And a begging we will go, we’ll go, we’ll go,\\
And a begging we will go!”
\end{patverse}
\end{dcverse}
\changefontsize{1.05\defaultfontsize}
\pagebreak
%346

The tune was introduced into the ballad-operas of \textit{Polly, The Lovers, The
Quakers’ Opera, Don Quixote in England, The Court Legacy, The Rape of Helen,
The Humours of the Court, The Oxford Act, The Sturdy Beggars}, \&c.; and the
song is the prototype of many others, such as, “A bowling we will go,” “A fishing
we will go,” “A hawking we will go,” and “A hunting we will go.” The
last-named is printed in the sixth vol. of \textit{The Musical Miscellany}, 8vo., 1731.
It is still popular with those who take delight in hunting; and as the air is now
scarcely known by any other title, I have printed the words to the tune. In
\textit{The Musical Miscellany} it is entitled \textit{The Stag Chace}, and there are twenty-nine
verses; twelve are here omitted, being principally a description of the dogs,
and a catalogue of their names; indeed, it is presum’d that seventeen stanzas
will suffice.

\musicinfo{Gaily.}{}

\includemusic{chappellV1184.pdf}

\settowidth{\versewidth}{Thro’ bush and brake, o’er hedge and stake,}
\begin{dcverse}\begin{altverse}
I leave my bed betimes,\\
Before the morning’s grey;\\
Let loose my dogs, and mount my horse,\\
And halloo “come away!” \&c.
\end{altverse}

\begin{altverse}
The game’s no sooner rous’d,\\
But in rush the cheerful cry,\\
Thro’ bush and brake, o’er hedge and stake,\\
The noble beast does fly, \&c.
\end{altverse}

\begin{altverse}
In vain he flies to covert,\\
A num’rous pack pursue,\\
That never cease to trace his steps,\\
Even tho’ they’ve lost the view, \&c.
\end{altverse}

\begin{altverse}
Now sweetly in full cry\\
Their various notes they join;\\
Gods! what a concert’s here, my lads!\\
’Tis more than half divine, \&c.
\end{altverse}

\begin{altverse}
The woods, the rocks, and mountains,\\
Delighted with the sound,\\
To neighb’ring dales and fountains\\
Repeating, deal it round, \&c.
\end{altverse}

\begin{altverse}
A glorious chace it is,\\
We drive him many a mile,\\
O’er hedge and ditch, we go thro’ stitch,\\
And hit off many a foil, \&c.
\end{altverse}
\end{dcverse}
\pagebreak
%347

\begin{dcverse}\begin{altverse}
And yet he runs it stoutly,\\
How wide, how swift he strains!\\
With what a skip he took that leap,\\
And scours o’er the plains! \&c.
\end{altverse}

\begin{altverse}
See how our horses foam!\\
The dogs begin to droop;\\
With winding horn, on shoulder borne,\\
’Tis time to cheer them up, \&c.
\end{altverse}

\begin{altverse}
Hark! Leader, Countess, Bouncer!\\
Cheer up my good dogs all;\\
To Tatler, hark! he holds it smart,\\
And answers ev’ry call, \&c.
\end{altverse}

\begin{altverse}
Up yonder steep I’ll follow.\\
Beset with craggy stones;\\
My lord cries, “Jack, you dog, come back,\\
Or else you’ll break your bones!” \&c.
\end{altverse}

\begin{altverse}
See, now he takes the moors,\\
And strains to reach the stream!\\
He leaps the flood, to cool his blood,\\
And quench his thirsty flame, \&c.
\end{altverse}

\begin{altverse}
He scarce has touch’d the bank,\\
The cry bounce finely in,\\
And swiftly swim across the stream,\\
And raise a glorious din, \&c.
\end{altverse}

\begin{altverse}
His legs begin to fail,\\
His wind and speed are gone,\\
He stands at bay, and gives ’em play,\\
He can no longer run, \&c.
\end{altverse}

\begin{altverse}
But vain are heels and antlers.\\
With such a pack set round,\\
Spite of his heart, they seize each part,\\
And pull him fearless down, \&c.
\end{altverse}

\begin{altverse}
Ha! dead, ’ware dead! whip off,\\
And take a special care;\\
Dismount with speed, and pray take heed,\\
Lest they his haunches tear! \&c.
\end{altverse}

\begin{altverse}
The sport is ended now,\\
We’re laden with the spoil;\\
As home we pass, we talk o’ th’ chace,\\
O’erpaid for all our toil, \&c.
\end{altverse}
\end{dcverse}

Many songs to the tune will be found in the publications enumerated above.
Others in the \textit{Songs sung at the Mug-houses in London and Westminster}, 1716; in
\textit{120 Loyal Songs}, 1684; and in the various collections of ballads. “The Church
Scuffle, or News from St. Andrew’s” is one of these; and contained in the collection
given to the Cheetham Library by Mr. Halliwell (No. 366).

\musictitle{The Noble Shirve.}

This tune is taken from a manuscript volume of virginal music, formerly in
the possession of Mr. Windsor, of Bath, and now in that of Dr. Rimbault.

Although the transcript is of the seventeenth, the tunes are generally traceable
to the sixteenth century, and perhaps the latest are of the reign of James I.

I regret very much not having been able to find the ballad from which it
derives its name, for I imagine it would prove an interesting, and, probably, a
very early one.

“Shirve” is a very old form of “Shire-reeve,” or Sheriff; and I have not
been able to trace any other instance of its use so late as the seventeenth century.
It was then, almost universally, written “Shrieve.” The tune is one that—
like \textit{The Three Ravens} (ante p. 59), and \textit{The Friar in the Well} (p. 274)—
requires a burden at the end of the first and second lines of words, as well as at
the end. The third and fourth bars of music seem almost to speak the words
“dōwn-ă-down,” and “hĕy dōwn-ă-dōwn” (or some similar burden); and the
seventh and eighth, “dōwn, ă-dōwn, ă-dōwn-ā.”

These repeated burdens were more common in the sixteenth than in the
seventeenth century.

As every ballad-tune sounds the better for having words to it, I have taken one 
of the snatches of old songs sung by \pagebreak Moros, the fool, or jester, in Wager’s
%348
interlude, \textit{The longer thou livest the more fool thou art}, 1568. It is not in the
precise measure—there should be two long syllables, instead of “out of Kent,”
in the second bar, \&c.—but I cannot find any old ballad, with similar burdens,
that corresponds exactly.

\musicinfo{Moderate time, and smoothly.}{}

\includemusic{chappellV1185.pdf}

\musictitle{Derry Down.}

This tune is referred to as \textit{The Abbot of Canterbury}; as \textit{Derry down}; as
\textit{A Cobbler there was}; and as \textit{Death and the Cobbler}.

Henry Carey, in his \textit{Musical Century}, 1740, i. 53, gives a song commencing—
\settowidth{\versewidth}{King George he was born in the month of October—}
\begin{scverse}
\vleftofline{“}King George he was born in the month of October—\\
’Tis a sin for a subject that month to be sober;”
\end{scverse}
which is to this tune; and he says, “The melody stolen from an \textit{old} ballad,
called \textit{Death and the Cobbler}.”

In Watts’ \textit{Musical Miscellany}, 1729, i. 94, is “A ballad to the \textit{old} tune, \textit{The
Abbot of Canterbury};” and, in the second volume of the same collection,
“\textit{Cobbler there was}, set by Mr. Leveridge,” who was then living. The tunes
are essentially the same, but Leveridge altered a few notes in the second part.

\changefontsize{0.99\defaultfontsize}
Dr. Percy remarks that “the common popular \pagebreak ballad of \textit{King John and the
%349
Abbot of Canterbury} seems to have been abridged and modernized about the time
of King James I., from one much older, entitled \textit{King John and the Bishop of
Canterbury}.” He adds that “the archness of the questions and answers hath
been much admired by our old ballad-makers; for, besides the two copies above
mentioned, there is extant another ballad on the same subject, entitled \textit{King
Olfrey and the Abbot}.” “Lastly, about the time of the civil wars, when the cry
ran against the bishops, some Puritan worked up the same story into a very
doleful ditty to a solemn tune, concerning \textit{King Henry and a Bishop}, with this
stinging moral”—
\settowidth{\versewidth}{Unlearned men hard matters out can find,}
\begin{scverse}
\vleftofline{“}Unlearned men hard matters out can find,\\
When learned bishops princes’ eyes do blind.”
\end{scverse}

A copy of the last is in the Douce Collection, fol. 110, entitled \textit{The King and
the Bishop}; another in the Pepys, i. 472; and a third in the Roxburghe, iii. 170.
It commences thus:—
\settowidth{\versewidth}{In Popish times, when bishops proud}
\begin{scverse}
\begin{altverse}
\vleftofline{“}In Popish times, when bishops proud\\
In England did bear sway,\\
Their lordships did like princes live,\\
And kept all at obey.”
\end{altverse}
\end{scverse}

The ballad of \textit{The old Abbot and King Olfrey} is in the Douce Collection, fol. 169.
Olfrey is supposed to be a corruption of Alfred.

Mr. Payne Collier, in his \textit{Extracts from the Registers of the Stationers’
Company}, i.~90, prints a ballad entitled \textit{The praise of Milkemaydes}, from
a manuscript of the time of James I., now in his possession. It is evidently the
same as \textit{A defence for Mylkemaydes against the terme of Mawken}, which was
entered on the Registers in 1563–4. Unfortunately neither the entry, nor Mr.
Collier’s manuscript, gives the name of the tune to which that ballad was sung.
I have a strong persuasion that it was to this air, for it has all the character of
antiquity, and I can find no other that would suit the words. The ballad
commences thus:—

\settowidth{\versewidth}{Passe not for rybaldes which mylkemaydes defame,}
\begin{scverse}
“Passe not for rybaldes which mylkemaydes defame,\\
And call them but Malkins, poore Malkins by name;\\
Their trade is as good as anie we knowe,\\
And that it is so I will presently showe.\\
\attribution Downe, a-downe, \&c.”
\end{scverse}

If, instead of “downe, a-downe, \&c.” we had the burden complete, “downe,
a-downe, downe, hey derry down,” I should feel no doubt of its being the air;
but the burden is not given at length in the manuscript. The second and sixth
stanzas allude to the singing of milkmaids—

\begin{scverse}
“They rise in the morning to heare the larke sing,\\
\textit{And welcome with ballettes the summer’s comming}\\
They goe to their kine, and their milking is done\\
Before that some sluggardes have lookt at the sunne.\\
\attribution Downe, a-downe, \&c.

In going to milking, or comming awaie,\\
They sing mery ballettes, or storyes they saye;\\
Their mirth is as pure and as white as their milke;\\
You cannot say that of your velvett and silke.\\
\attribution Downe, a-downe,” \&c.
\end{scverse}
\pagebreak
%350

There are numberless songs and ballads to the tune, under one or other of its
names. Political songs will be found in the collections written against the Rump
Parliament; in those of the time of James II.; and again in “A Collection of
State Songs, \&c., that have been published since the Rebellion, and sung in the
several mug-houses in the cities of London and Westminster” (1716). One of
Shenstone’s ballads, \textit{The Gossiping}, is to the tune of \textit{King John and the Abbot of
Canterbury}, and is printed in his works, Oxford, 1737. Again, in \textit{The Asylum
for Fugitive Pieces}, 1789, there are several; and the tune is in common use at
the present day.

Dr. Rimbault, in his \textit{Musical Illustrations to Percy’s Reliques of Ancient
Poetry}, prints from a MS. of the latter part of the seventeenth century, which
agrees with the copy in Watts’ \textit{Musical Miscellany}. Other copies will be found
in \textit{The Beggars’ Opera}, third edit., 1729; \textit{The Village Opera}, 1729; \textit{Penelope},
1728; \textit{The Fashionable Lady}, 1730; \textit{The Lover his own Rival}, 1736; \textit{The
Boarding-School, or The Sham Captain}, 1733; \textit{The Devil to pay}, 1731; \textit{The
Oxford Act; The Sturdy Beggars; Love and Revenge; The Jew decoy’d}; \&c.

I have printed two copies of the tune; the first being the commonly received
version, and the second taken from Watts’ \textit{Musical Miscellany}. These differ
materially, but intermediate versions will be found in \textit{The Beggars’ Opera}, and
some other of the above-mentioned works.

Both \textit{The King and the Abbot}, and \textit{The King and the Bishop}, are in the
catalogue of ballads, printed by Thackeray, in the reign of Charles II. The
copy of the former in the Bagford Collection is entitled “King John and the
Abbot of Canterbury, to the tune of \textit{The King and Lord Abbot}.” The story,
upon which these ballads are founded, can be traced back to the fifteenth
century.

\musicinfo{Moderate time.}{}

\bigskip

\includemusic{chappellV1186.pdf}

%\changefontsize{0.93\defaultfontsize}
\pagebreak
%351

\settowidth{\versewidth}{—Now from the third question thou must not shrink,}
\begin{dcverse}\footnotesizerr
And I’ll tell you a story, a story so merry,\\
Concerning the Abbot of Canterbury;\\
How for his housekeeping, and high renown,\\
The king he sent for him to fair London town.

An hundred men, the king did hear say,\\
The abbot did keep in his house every day;\\
And fifty gold chains, without any doubt,\\
In velvet coats waited the lord abbot about.

How now, father abbot, I hear it of thee,\\
Thou keepest a far better house than me;\\
And from thy housekeeping and high renown,\\
I fear thou work’st treason against my crown.

My liege, quo’ the abbot, I would it were known,\\
I never spend nothing but what is my own;\\
And I trust that your grace will do me no dere,\\
For spending of my own true-gotten gear.

Yes, yes, father abbot, thy fault it is high,\\
And now for the same thou needest must die;\\
For, except thou canst answer me questions three,\\
Thy head shall be smitten from off thy bodỳ.

And first, quo’ the king, when I’m in this stead,\\
With my crown of gold so fair on my head,\\
Among all my liegemen so noble of birth,\\
Thou must tell me to one penny what I am worth.

And, secondly, tell me, without any doubt,\\
How soon I may ride the whole world about.\\
And at the third question thou must not shrink,\\
But tell me here truly what I do think.

O, these are hard questions for my shallow wit,\\
And I cannot answer your grace as yet:\\
But if you will give me but three weeks space,\\
I’ll do my endeavour to answer your grace.

Now three weeks’ space to thee will I give,\\
And that is the longest time thou hast to live;\\
For if thou dost not answer my questions three,\\
Thy lands and thy livings are forfeit to me.

Away rode the abbot all sad at that word,\\
And he rode to Cambridge and Oxenford;\\
But never a doctor there was so wise,\\
That could with his learning an answer devise.

Then home rode the abbot of comfort so cold,\\
And he met his shepherd a going to fold:\\
“How now, my lord abbot, you are welcome home, \\
What news do you bring us from good King John?”

“Sad news, sad news, shepherd, I must give;\\
That I have but three days longer to live;\\
For if I do not answer him questions three,\\
My head will be smitten from off my bodỳ.

The first is to tell him there in that stead,\\
With his crown of gold so fair on his head,\\
Among all his liegemen so noble of birth,\\
To within one penny of what he is worth.

The second, to tell him, without any doubt,\\
How soon he may ride this whole world about:\\
And at the third question I must not shrink,\\
But tell him there truly what he does think.”

“Now cheer up, sire abbot, did you never hear yet,\\
That a fool he may learn a wise man wit?\\
Lend me horse, and serving men, and your apparel,\\
And I’ll ride to London to answer your quarrel.

Nay frown not, if it hath been told unto me\\
I am like your lordship, as ever may be;\\
And if you will only but lend me your gown,\\
There’s none that shall know us at fair London town.”

“Now horses and serving-men thou shalt have,\\
With sumptuous array most gallant and brave;\\
With crozier, and mitre, and rochet, and cope,\\
Fit to appear ’fore our father the pope.”

“Now welcome, sire abbot, the king he did say,\\
’Tis well thou’rt come back to keep to thy day;\\
For and if thou canst answer my questions three,\\
Thy life and thy living both saved shall be.

And first, when thou seest me here in this stead,\\
With my crown of gold so fair on my head,\\
Among all my liegemen so noble of birth,\\
Tell me to one penny what I am worth.”

“For thirty pence our Saviour was sold\\
Among the false Jews, as I have been told;\\
And twenty-nine is the worth of thee,\\
For I think thou art one penny worser than he.”

The king he laughed, and swore by St. Bittel,\\
“I did not think I had been worth so little!\\
—Now, secondly tell me, without any doubt,\\
How soon I may ride this whole world about.”

“You must rise with the sun, and ride with the same,\\
Until the next morning he riseth again;\\
And then your grace need not make any doubt,\\
But in twenty-four hours you’ll ride it about.”

The king he laughed, and swore by St. Jone,\\
“I did not think it could be gone so soon!\\
—Now from the third question thou must not shrink,\\
But tell me here truly what I do think.”

\end{dcverse}
%\changefontsize{0.94\defaultfontsize}
\pagebreak
%352

\begin{dcverse}\footnotesizerr
“Yea, that shall I do, and make your grace merry,\\
You think I’m the Abbot of Canterbury; \\
But I’m his poor shepherd, as plain you may see,\\
That am come to beg pardon for him and for me.”

The king he laughed, and swore by the mass,\\
“I’ll make thee lord abbot this day in his place.\\
\columnbreak
“Now nay, my liege, be not in such speed,\\
For, alack, I can neither write ne read.”

“Four nobles a week, then, I will give thee,\\
For this merry jest thou hast shown unto me;\\
And tell the old abbot, when thou comest home,\\
Thou hast brought him a pardon from good King John.”
\end{dcverse}

The following is a very different version of the tune, as printed in Watts’
\textit{Musical Miscellany}.

\musicinfo{Moderate time.}{}

\medskip

\includemusic{chappellV1187.pdf}

The following punning prototype of the late T. Hood’s comic songs, should not
be omitted. It is entitled \textit{The Cobbler's End}:—

\begin{dcverse}A cobbler there was, and he liv’d in a stall,\\
Which serv’d him for parlour, for kitchen, and all;\\
No coin in his pocket, nor care in his pate,\\
No ambition had he, nor duns at his gate.\\
\vin Derry down, down, down, derry down.

Contented he work’d, and he thought himself happy, \\
If at night he could purchase a jug of brown nappy;\\
How he’d laugh then, and whistle, and sing, too, most sweet,\\
Saying just to a hair I have made both ends meet. \\
\vin Derry down, down, \&c.

But love the disturber of high and of low,\\
That shoots at the peasant as well as the beau,\\
He shot the poor cobbler quite thorough the heart:\\
I wish he had hit some more ignoble part.\\
\vin Derry down, down, \&c.

It was from a cellar this archer did play,\\
Where a buxom young damsel continually lay;\\
Her eyes shone so bright when she rose ev’ry day, \\
That she shot the poor cobbler quite over the way,\\
\vin Derry down, down, \&c.
\end{dcverse}
%\changefontsize{\defaultfontsize}
\pagebreak
%353

\begin{dcverse}He sang her love songs as he sat at his work:\\
But she was as hard as a Jew or a Turk;\\
Whenever he spake, she would flounce and would fleer,\\
Which put the poor cobbler quite into despair.\\
\vin Derry down, down, \&c.

He took up his awl that he had in the world,\\
And to make away with himself was resolv’d;\\
%\columnbreak
He pierc’d through his body instead of his sole,\\
So the cobbler he died, and the bell it did toll.\\
\vin Derry down, down, \&c.

And now in good-will I advise as a friend,\\
All cobblers take warning by this cobbler’s end; \\
Keep your hearts out of love, for we find by what’s past,\\
That love brings us all to an end at the last.\\
\vin Derry down, down, \&c.
\end{dcverse}

\musictitle{Tom Tinker’s My True-Love.}

The tune of \textit{Tom Tinker's my true love} is mentioned in a black-letter tract,
called \textit{The World's Folly}, which was reprinted in \textit{The British Bibliographer},
edited by Sir Egerton Brydges:—“A pot of strong ale, which was often at his nose,
kept his face in so good a coulour, and his braine in so kinde a heate, as, forgetting
part of his forepassed pride, (in the good humour of grieving patience,) made him,
with a hemming sigh, ilfavourdly singe the ballad of \textit{Whilom I was}, to the tune of
\textit{Tom Tinker}.” (ii. 559). The tune is in \textit{The Dancing Master} from 1650 to
1698. About the latter period it seems to- have been rejected for another air
(under the same name), which is printed with the words in \textit{Pills to purge
Melancholy}, vi. 265; and was introduced in \textit{The Beggars’ Opera} for the song
\textit{Which way shall I turn me}?

The following tune is from\textit{The Dancing Master}:—

\musicinfo{Moderate time.}{}

\includemusic{chappellV1188.pdf}

%\changefontsize{\defaultfontsize}
\pagebreak
%354

The \textit{Tom Tinker} of \textit{The Beggars’ Opera}, and to which D’Urfey prints the
above words, is subjoined.

\musicinfo{Moderate time, and Smoothly.}{}

\includemusic{chappellV1189.pdf}

\musictitle{Northern Nancy.}

This tune is contained in every edition of \textit{The Dancing Master}, after 1665.
It is evidently only another version of \textit{With my flock as walked I }(ante p. 157).\footnote{\textit{}
I had not observed the identity of these tunes when
the former sheet went to press; otherwise I should have
compressed the account of them under one head. The
difference is chiefly in the two first bars, but even that
variation is diminished in the copy called\textit{ The faithful
Brothers}, to which I have referred at the former page.}
\settowidth{\versewidth}{Then plump \textit{Bobbing Joan} straight call’d for her own,}
\begin{scverse}
\begin{altverse}
\vleftofline{“}Then plump \textit{Bobbing Joan} straight call’d for her own,\\
And thought she frisk’d better than any,\\
Till Sisly, with pride, took the fiddler aside,\\
And hade him strike up Northern Nanny.”\\
\attribution \textit{Pills to purge Melancholy}, ii. 232, 1719.
\end{altverse}
\end{scverse}

In the Roxburghe Collection, i. 252, is a black-letter ballad, entitled “The
Map of Mock-Begger Hall, with his scituation in the spacious countrey called
\textsc{Anywhere}. To the tune of \textit{It is not your Northern Nanny}; or \textit{Sweet is the
lass that loves me}.” It commences thus:
\settowidth{\versewidth}{While \textit{Mock-Beggar Hall stands empty}.”}
\begin{dcverse}\begin{altverse}
\vleftofline{“}I read in ancient times of yore\\
That men of worthy calling\\
Built alms-houses and spittles store,\\
Which now are all down falling:
\end{altverse}

\begin{altverse}
And few men seek them to repair,\\
Nor is there one among twenty\\
That for good deeds will take any care,\\
While \textit{Mock-Beggar Hall stands empty}.”
\end{altverse}
\end{dcverse}

It consists of twelve stanzas, and “Printed at London for Richard Harper, neere
to the Hospitall Gate in Smithfield.”

In the same Collection, iii. 218, is another version of the same ballad, issued
by the same printer, but with variations in the imprint, in the number of stanzas,
and in the woodcut.

%\changefontsize{\defaultfontsize}
The first has a woodcut of a country mansion; the second of a castle. The
second has three additional stanzas, and variations in the remaining twelve.
The title commences, “Mock-Begger’s-Hall,” \pagebreak instead of “The Map of;” and
%355
at the end, “London: Printed for Richard Harper, \textit{at the Bible and Harp} in
Smithfield.”

Mr. Payne Collier, who has reprinted the latter in his \textit{Roxburghe Ballads}, is of
opinion that, although Richard Harper printed during the Commonwealth, the
ballad itself is of the early part of the seventeenth century. (It contains the
same complaints of the decay of hospitality that are to be found in \textit{The Queen's
Old Courtier}.) The first stanza of the second ballad is here printed to the tune.

In the Roxburghe Collection, ii. 390, is another ballad, called \textit{The ruined
Lover}, \&c., “to the tune of \textit{Mock-Begger's Hall stands empty},” beginning—
\settowidth{\versewidth}{Mars shall to Cupid now submit,}
\begin{dcverse}
\begin{altverse}
\vleftofline{“}Mars shall to Cupid now submit,\\
For he hath gain’d the glory;\\
You that in love were never yet,\\
Attend unto my story;
\end{altverse}

\begin{altverse}
For it is new, ’tis strange and true,\\
As ever age afforded;\\
A tale more sad you never had\\
In any books recorded.”
\end{altverse}
\end{dcverse}

This was printed by W. Thackeray, temp. Charles II.

\textit{Northern Nancy} is one of the tunes called for by “the hob-nailed fellows” in
\textit{The Second Tale of a Tub}, 8vo., 1715.

\musicinfo{Rather slowly.}{}

\includemusic{chappellV1190.pdf}

\changefontsize{0.93\defaultfontsize}
\pagebreak
%356

\musictitle{I Have But A Mark A Year.}

This tune is to be found in \textit{ Pills to purge Melancholy}, ii. 116, 1700 and 1707;
or iv.~116,~1719. The ballad is by Martin Parker, and a copy is contained in
the Roxburghe Collection, i. 122. In the preface to the \textit{Pills}, Playford tells us
that the words of the songs “which are old have their rust generally filed from
them, which cannot but make them very agreeable.” This is one that has
undergone the process of “filing;” it is abbreviated, but certainly not improved,
by the operation. The copy in the Roxburghe Collection is entitled “A fair
portion for a fair Maid; or—
\settowidth{\versewidth}{The thrifty maid of Worcestershire,}
\begin{dcverse}The thrifty maid of Worcestershire,\\
Who lives at London for a mark a year;

This mark was her old mother’s gift.\\
She teaches all maids how to thrift.
\end{dcverse}
To the tune of \textit{Grammercy, Penny}.” (The first stanza is here printed with the
music.) \textit{Grammercy} (or \textit{God-a-mercy}), \textit{Penny}, derives its name from the burden
of another ballad, also in the Roxburghe Collection (i. 400), entitled “There’s
nothing to be had without money; or—
\begin{dcverse}He that brings money in his hand,\\
Is sure to speed by sea and land;\\
But he that hath no coin in’s purse,

His fortune is a great deal worse;\\
Then happy are they that always have\\
A penny in purse, their credit to save.
\end{dcverse}
To \textit{a new Northern tune}, or\textit{ The mother beguil’d the daughter}.” It commences thus:
\begin{dcverse}“You gallants and you swagg’ring blades,\\
Give ear unto my ditty;\\
I am a boon-companion known\\
In country, town, and city;

I always lov'd to wear good clothes,\\
And ever scorned to take blows;\\
I am belov’d of all me knows,\\
But \textit{God-a-mercy penny}.”
\end{dcverse}

This was “printed at London for H[enry] G[osson].” Six stanzas in the
first, and eight in the second part.

Another ballad, from the same press, is “The Praise of Nothing: to the tune
of \textit{Though I have but a marke a yeare}, \&c.” A copy in the Roxburghe Collection,
i. 328, and reprinted in Payne Collier’s \textit{Roxburghe Ballads}, p. 147. The
following lines are added to the title of the ballad:—
\settowidth{\versewidth}{Though some do wonder why I write the praise}
\begin{scverse}
\vleftofline{“}Though some do wonder why I write the praise\\
Of Nothing in these lamentable days,\\
When they have read, and will my counsel take,\\
I hope of Nothing they will Something make!”
\end{scverse}

The above contains much excellent advice.

Having traced the tune from \textit{I have but a mark a year} to \textit{God-a-mercy, Penny},
and from the latter to “\textit{a new Northern tune}, or \textit{The mother beguil’d the daughter},”
the following ballads may also be referred to it:—

Roxburghe, i. 238—“The merry careless lover: Or a pleasant new ditty, called
I love a lass since yesterday, and yet I cannot get her. To the tune of \textit{The
mother beguilde the daughter}."
\settowidth{\versewidth}{Oft have I heard of many men}
\begin{dcverse}\begin{altverse}
\vleftofline{“}Oft have I heard of many men\\
Whom love hath sore tormented,\\
With grief of heart, and bitter smart,\\
And minds much discontented;\\
Such, love to me shall never be,\\
Distasteful, grievous, bitter!
\end{altverse}

\begin{altverse}
I have lov’d a lass since yesterday,\\
And yet I cannot get her.\\
But let her choose—if she refuse,\\
And go to take another,\\
I will not grieve, but still will be\\
\textit{The merry careless lover},” \&c.
\end{altverse}
\end{dcverse}

\changefontsize{0.96\defaultfontsize}
\pagebreak
%357

\noindent Signed Robert Guy. Twelve stanzas. Printed at London for F. Coules, and
reprinted in Evans’ \textit{Old Ballads}, i. 176, 1810.

Roxburghe, i. 314, “A Peerless Paragon; or—
\settowidth{\versewidth}{Few so chaste, so beauteous, or so fair;}
\begin{scverse}
Few so chaste, so beauteous, or so fair;\\
For with my love I think none can compare.
\end{scverse}
To the tune of \textit{The mother beguild the daughter}.”
\settowidth{\versewidth}{In times of yore sure men did doat,}
\begin{dcverse}\begin{altverse}
\vleftofline{“}In times of yore sure men did doat,\\
And beauty never knew,\\
Else women were not of that note,\\
As daily come to view:
\end{altverse}

\begin{altverse}
For, read of all the faces then\\
That did most brightly shine,\\
Be judg’d by all true-judging men,\\
They were not like to mine.”
\end{altverse}
\end{dcverse}
This has no burden. It consists of thirteen stanzas. “Printed at London for
Thomas Lambert.”

Martin Parker’s ballad, “The Countrey Lasse,” to the tune of \textit{The mother
beguild the daughter}, has been quoted at p. 306, but it appears also to have had a
separate tune, which will be given hereafter.

\musicinfo{Cheerfully.}{}

\includemusic{chappellV1191.pdf}

\changefontsize{1.00\defaultfontsize}
\pagebreak
%358

\musictitle{I Tell Thee, Dick, Where I Have Been.}

This celebrated ballad, by Sir John Suckling, was occasioned by the marriage
of Roger Boyle, the first Earl of Orrery (then Lord Broghill), with Lady
Margaret Howard, daughter of the Earl of Suffolk. The words are in the first
edition of Sir John Suckling’s works, 1646; in \textit{Wit’s Recreation}, 1654; in
\textit{Merry Drollery Complete}, 1661; \textit{Antidote to Melancholy}, 1661; in\textit{ The Convivial
Songster}, 1782; in Ritson’s \textit{Ancient Songs}, p. 223; and Ellis’ \textit{Specimens of Early
English Poets}, iii. 248.

The tune is in \textit{A Choice Collection of 180 Loyal Songs}, third edit., 1685; in
\textit{Pills to purge Melancholy}, vol. i., 1699 and 1707; in \textit{The Convivial Songster},
1782, \&c.

The following were written to the tune:—

1.\textit{ The Cavalier’s Complaint}. A copy in the Bagford Collection (643, m. 11,
p. 23) dated 1660; and one in the King’s Pamphlets, No. 19, fol., 1661; others
in \textit{Antidote to Melancholy}; \textit{Merry Drollery}, 1670; \textit{The New Academy of Compliments},
1694 and 1713; and Dryden’s \textit{Miscellany Poems}, vi. 352; \&c.

\settowidth{\versewidth}{Come, Jack, let’s drink a pot of ale,}
\begin{dcverse}\indentpattern{001001}
\begin{patverse}
\vleftofline{“}Come, Jack, let’s drink a pot of ale,\\
And I will tell thee such a tale,\\
Shall make thine ears to ring;\\
My coin is spent, my time is lost,\\
And I this only fruit can boast—\\
That once I saw my King.
\end{patverse}

\begin{patverse}
But this doth most afflict my mind—\\
I went to court in hope to find\\
Some of my friends in place;\\
And, walking there, I had a sight\\
Of all the crew—but, by this light,\\
I hardly knew one face!
\end{patverse}

\begin{patverse}
S’life, of so many noble sparks,\\
Who on their bodies bear the marks\\
Of their integrity,\\
And suffer’d ruin of estate,\\
It was my damn’d unhappy fate\\
That I not one could see.
\end{patverse}

\begin{patverse}
Not one, upon my life, among\\
My old acquaintance, all along\\
At Truro, and before;\\
And I suppose the place can shew\\
As few of those whom thou didst know\\
At York, or Marston-Moor.
\end{patverse}

\begin{patverse}
But, truly, there are swarms of those\\
Whose chins are beardless, yet their hose\\
And buttocks still wear muffs;\\
Whilst the old rusty Cavalier\\
Retires, or dares not once appear.\\
For want of coin and cuffs.
\end{patverse}

\begin{patverse}
When none of these I could descry,\\
(Who better far deserv’d than I,)\\
Calmly did I reflect;\\
Old services, by rule of state,\\
Like almanacks, grow out of date;\\
What then can I expect?
\end{patverse}

\begin{patverse}
Troth, in contempt of fortune’s frown,\\
I’ll get me fairly out of town,\\
And in a cloister pray\\
That since the stars are yet unkind\\
To Royalists, the King may find\\
More faithful friends than they.’’
\end{patverse}
\end{dcverse}

2. \textit{An Echo to the Cavalier’s Complaint}. Copies in The \textit{Antidote to Melancholy},
1661; \textit{Merry Drollery Complete}, 1670; \textit{New Academy of Compliments}; \&c.

\begin{dcverse}\indentpattern{001001}
\begin{patverse}
\vleftofline{“}I marvel, Dick, that having been\\
So long abroad, and having seen\\
The world, as thou hast done,\\
Thou shouldst acquaint me with a tale\\
As old as Nestor, and as stale\\
As that of priest and nun.
\end{patverse}

\begin{patverse}
Are we to learn what is a court?\\
A pageant made for Fortune’s sport,\\
Where merits scarce appear;\\
For bashful merit only dwells\\
In camps, in villages, and cells;\\
Alas! it dwells not there.
\end{patverse}
\end{dcverse}
\changefontsize{0.96\defaultfontsize}
\pagebreak
%359

\begin{dcverse}\indentpattern{001001}
\begin{patverse}
Desert is nice in its address,\\
And merit oft-times doth oppress,\\
Beyond what guilt would do;\\
But they are sure of their demands\\
That come to court with golden hands,\\
And brazen faces too.
\end{patverse}

\begin{patverse}
The King, they say, doth still profess\\
To give his party some redress,\\
And cherish honesty;\\
But his good wishes prove in vain,\\
Whose service with his servant’s gain\\
Not always doth agree.
\end{patverse}

\begin{patverse}
All princes, be they ne’er so wise,\\
Are fain to see with others’ eyes,\\
But seldom hear at all;\\
And courtiers find’t their interest\\
In time to feather well their nest,\\
Providing for their fall.
\end{patverse}

\begin{patverse}
Our comfort doth on time depend;\\
Things, when they are at worst, will mend:\\
And let us but reflect\\
On our condition th’other day,\\
When none but tyrants bore the sway,\\
What, then, did we expect?
\end{patverse}

\begin{patverse}
Meanwhile, a calm retreat is best;\\
But discontent, if not supprest,\\
Will breed disloyalty.\\
This is the constant note I sing,—\\
I have been faithful to my king,\\
And so shall ever be.
\end{patverse}
\end{dcverse}

3. \textit{Upon Sir John Suckling’s 100 Horse}. Contained in \textit{Le Prince d’Amour}, or
\textit{The Prince of Love}, 1660, p. 148. Sir John raised a magnificent regiment of
cavalry at his own expense (12,000\textit{l}.), in the beginning of our civil wars, which
became equally conspicuous for cowardice and finery. They rendered him the
subject of much ridicule; and although he had previously served in a campaign
under Gustavus Adolphus—during which he was present at three battles, five
sieges, and as many skirmishes—his military reputation did not escape.
\indentpattern{001001}
\begin{dcverse}\begin{patverse}
“I tell thee, Jack, thou gav’st the King\\
So rare a present, that nothing\\
Could welcomer have been;\\
A hundred horse! beshrew my heart,\\
It was a brave heroic part,\\
The like will scarce be seen.
\end{patverse}

\begin{patverse}
For ev’ry horse shall have on’s back\\
A man as valiant as Sir Jack,\\
Although not half so witty:\\
Yet I did hear the other day\\
Two tailors made seven run away,\\
Good faith, the more’s the pity.” \&c.
\end{patverse}
\end{dcverse}
There are seven stanzas, and then “An Answer” to it.\footnote{\textit{}
These were not the only satires Sir John Suckling
had to bear. There were, at least, two others. One, to
the tune of \textit{John Dory}, begins—
\settowidth{\versewidth}{Had you seen but his look, you would sweare by the book,}
\indentpattern{0323}
\begin{fnverse}
\begin{patverse}
\vin\vin\vleftofline{“}Sir John got on a bonny brown beast,\\
To Scotland for to ride-a;\\
A brave buff coat upon his back,\\
And a short sword by his side-a.”
\end{patverse}
\end{fnverse}
The other—
\indentpattern{232323253}
\begin{fnverse}
\begin{patverse}
\vin\vin\vleftofline{“}Sir John got him an ambling nag,\\
To Scotland for to go,\\
With a hundred horse, without remorse,\\
To keep ye from the foe;\\
No carpet knight ever went to fight\\
With half so much bravado;\\
Had you seen but his look, you would\\ 
sweare by the book,\\
He’d ha’ conquer’d the whole Armado.”
\end{patverse}
\end{fnverse}
There are also two other versions of the latter; the one
beginning, “Then as it fell out on a holiday,” (see “Censura
Literaria,” vol. vi., p. 269) and the other in Percy's
\textit{Reliques of Ancient Poetry}, vol. ii., p. 326.}


4 and 5. \textit{A ballad on a Friend’s Wedding}, and \textit{Three Merry Boys of Kent},
in \textit{Folly in Print, or a Book of Rhymes}, 1667.

6. \textit{A new ballad, called The Chequers Inn}, in \textit{Poems on State Affairs}, iii. 57,
1704, It begins:—
\settowidth{\versewidth}{I tell thee, Dick, where I have been,}
\begin{scverse}
\vleftofline{“}I tell thee, Dick, where I have been,\\
Where I the Parliament have seen,” \&c.
\end{scverse}

7. \textit{A Christmas Song}, when the Rump Parliament was first dissolved, \textit{Loyal
Songs}, ii. 99, 1731.

Besides these, there is one in Carey’s \textit{Trivial Poems}, 1651; three in \textit{180 Loyal
Songs}, 1685; \&c.

%\changefontsize{0.90\defaultfontsize}
“The grace and elegance of Sir John Suckling’s songs and ballads are inimitable.”
“They have a touch,” says Phillips, “of a gentle spirit, and seem
\pagebreak
%360
to savour more of the grape than the lamp.” The author of the song above
quoted from \textit{Folly in Print}, says—
\textit{I do not write to get a name,}

\begin{dcverse}\vleftofline{“}I do not write to get a name,\\
At best this is but ballad-fame;\\
And Suckling hath shut up that door,\\
To all hereafter, as before.”
\end{dcverse}


Sir John died in 1641, at the early age of twenty-eight. The ballad is a
countryman’s description of a wedding.

\musicinfo{Smoothly.}{}

\includemusic{chappellV1192.pdf}

\backskip{1}
\begin{dcverse}\indentpattern{001}
\begin{patverse}
“At Charing Cross, hard by the way\\
Where we, thou know’st, do sell our hay,\\
There is a house with stairs;
\end{patverse}

\begin{patverse}
And there did I see, coming down,\\
Such folk as are not in our town,\\
Forty, at least, in pairs.”
\end{patverse}
\end{dcverse}

There, are twenty-two stanzas, but some lines of the ballad might now be
considered objectionable. I have, therefore, extracted the following—a part of
the description of the bride:—
\settowidth{\versewidth}{Would not stay on which they did bring,}
\begin{dcverse}\indentpattern{001001}
\begin{patverse}
The maid—and thereby hangs a tale—\\
For such a maid no Whitsun-ale\\
Could ever yet produce:\\
No grape that’s kindly ripe could be\\
So round, so plump, so soft as she,\\
Nor half so full of juice.
\end{patverse}

\begin{patverse}
Her finger was so small, the ring\\
Would not stay on which they did bring,\\
It was too wide a peck:\\
And, to say truth, (for out it must,)\\
It lookt like the great collar (just)\\
About our young colt’s neck.
\end{patverse}

\indentpattern{0010010001001}
\begin{patverse}
Her feet beneath her petticoat,\\
Like little mice stole in and out,\\
As if they fear’d the light;\\
But, oh! she dances such a way,\\
No sun upon an Easter-day\\
Is half so fine a sight.\\
\quad\quad *\quad\quad *\quad\quad *\quad\quad *\quad\quad *\quad\quad *\\
Her cheeks so rare a white was on,\\
No daisy makes comparison;\\
(Who sees them is undone;)\\
For streaks of red were mingled there,\\
Such as are on a Kath’rine pear,\\
The side that’s next the sun.
\end{patverse}

\indentpattern{001001}
\begin{patverse}
Her lips were red, and one was thin,\\
Compar’d to that was next her chin;\\
Some bee had stung it newly:\\
But, Dick, her eyes so guard her face,\\
I durst no more upon them gaze.\\
Than on the sun in July.
\end{patverse}
\end{dcverse}

%\changefontsize{\defaultfontsize}
\pagebreak