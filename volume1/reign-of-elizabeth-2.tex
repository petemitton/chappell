%121
\changefontsize{1.01\defaultfontsize}
\musictitle{Pepper is Black}

This tune is to be found in \textit{The Dancing Master}, from 1650 to 1690. It is
mentioned as a dance tune by Nashe in\textit{ Have with you to Saffron-Walden}, 1596.
(See ante p. 116.) A copy of the following ballad by Elderton is in the collection
of Mr. George Daniel, of Canonbury: “Prepare ye to the plough, to the tune
of \textit{Pepper is black}.”

\settowidth{\versewidth}{“The Queen holds the plough, to continue good seed,}
\begin{scverse}
\vleftofline{“}The Queen holds the plough, to continue good seed,\\
Trusty subjects, be ready to help if she need.”
\end{scverse}


\musicinfo{Moderate time.}{}

\includemusic{chappellV1045.pdf}

\settowidth{\versewidth}{Can bring about that I found out,}
\begin{dcverse}
Parnaso hill, not all the skill\\
Of nymphs, or muses feigned,

Can bring about that I found out,\\
By Christ himself ordained, \&c.
\end{dcverse}

There are twelve stanzas, each of eight lines, subscribed W. Elderton. Printed
by Wm. How, for Richard Johnes.

\musictitle{Walsingham.}

This tune is in Queen Elizabeth’s, and Lady Neville’s, Virginal Books (with
thirty variations by Dr. John Bull); in Anthony Holborne’s \textit{Cittharn Schoole},
1597; in Barley’s \textit{New Booke of Tablature}, 1596, \&c. It is called “\textit{Walsingham},”
“\textit{Have with you to Walsingham}” and “\textit{As I went to Walsingham}.”

It belongs, in all probability, to an earlier reign, as the Priory of Walsingham,
in Norfolk, which was founded during the Episcopate of William, Bishop of
Norwich (1146 to 1174), was dissolved in 1538.

Pilgrimages to this once famous shrine commenced in or before the reign of
Henry III., who was there in 1241. Edward I. was at Walsingham in 1280, and
again in 1296; and Edward II. in 1315. The author of \textit{The Vision of Piers
Ploughman}, says—

\settowidth{\versewidth}{Wenten to Walsyngham, and her [their] wenches after.}
\begin{scverse}
\vleftofline{“}Heremytes on a hepe, with hooked staves,\\
Wenten to Walsyngham, and her [their] wenches after.”
\end{scverse}

A curious reason why pilgrims should have both singers and pipers to accompany
them, will be found in note \textit{a}, at page 34.

Henry VII., having kept his Christmas of 1486-7, at Norwich, “from thence
went in manner of pilgrimage to Walsingham, where he visited Our Lady’s Church, 
famous for miracles;  and made his prayers \pagebreak and vows for help and deliverance.”
%122
And in the following summer, after the battle of Stoke, “he sent his banner to
be offered to Our Lady of Walsingham, where before he made his vows.”

“Erasmus has given a very exact and humorous description of the superstitions
practised there in his time. See his account of the \textit{Virgo Parathalassia}, in his
colloquy, intitled \textit{Peregrinatio Religionis ergo}. He tells us, the rich offerings in
silver, gold, and precious stones, that were shewn him, were incredible; there being
scarce a person of any note in England, but what some time or other paid a visit,
or sent a present, to Our Lady of Walsingham. At the dissolution of the monasteries
in 1538, this splendid image, with another from Ipswich, was carried to
Chelsea, and there burnt in the presence of commissioners; who, we trust, did not
burn the jewels and the finery.”—\textit{Percy’s Reliques}.

The tune is frequently mentioned by writers of the sixteenth and seventeenth
centuries. In act v. of Fletcher’s \textit{The Honest Man’s Fortune,} one of the servants
says, “I’ll renounce my five mark a year, and all the hidden art I have in carving,
to teach young birds to whistle \textit{Walsingham}.” A verse of “As you came from
Walsingham,” is quoted in \textit{The Knight of the Burning Pestle}, and in \textit{Hans Beerpot,
his invisible Comedy}, 4to., 1618.

In \textit{The weakest goes to the wall}, 1600, the scene being laid in Burgundy, the
following lines are given:—
\begin{scverse}
\vleftofline{“}King Richard’s gone to Walsingham, to the Holy Land,\\
To kill Turk and Saracen, that the truth do withstand;\\
Christ his cross be his good speed, Christ his foes to quell,\\
Send him help in time of need, and to come home well.”
\end{scverse}
In the Bodleian Library is a small quarto volume, apparently in the hand-writing
of Philip, Earl of Arundel (eldest son of the Duke of Norfolk, who suffered in
Elizabeth’s time), containing \textit{A lament for Walsingham}. It is in the ballad style,
and the two last stanzas are as follows:—
\settowidth{\versewidth}{Weep, weep, 0 Walsingham!}
\begin{dcverse}
\vleftofline{“}Weep, weep, O Walsingham!\\
Whose days are nights;\\
Blessings turn’d to blasphemies—\\
Holy deeds to despites.

Sin is where Our Lady sat,\\
Heaven turned is to hell;\\
Satan sits where Our Lord did sway:\\
Walsingham, Oh, farewell!”
\end{dcverse}
In Nashe’s \textit{Have with you to Saffron-Walden}, 1596, sign. L, “As I went to
Walsingham” is quoted, which is the first line of the ballad in the Pepysian
Collection, vol. i., p. 226, and a verse of which is here printed to the music.

One of the \textit{Psalmes and Songs of Sion, turned into the language, and set to the
tunes of a strange land}, 1642, is to the tune of \textit{Walsingham}; and Osborne, in his
\textit{Traditional Memoirs on the Reigns of Elizabeth and James}, 1653, speaking of the
Earl of Salisbury, says:—
\settowidth{\versewidth}{Many a hornpipe he tuned to his Phillis,}
\begin{scverse}
\vleftofline{“}Many a hornpipe he tuned to his Phillis,\\
And sweetly sung \textit{Walsingham} to’s Amaryllis.”
\end{scverse}
In \textit{Don Quixote}, translated by J. Phillips, 1687, p. 278, he says, “An infinite
number of little birds, with painted wings of various colours, hopping from branch
to branch, all naturally singing Walsingham, and whistling \textit{John, come kiss
me now}.”

Two of the ballads are reprinted in Percy’s \textit{Reliques of Ancient Poetry}; the
one beginning, “Gentle herdsman, tell \pagebreak to me;” the other, “As ye came from the
%123
Holy Land.” The last will also be found in Deloney’s \textit{Garland of Goodwill},
reprinted by the Percy Society.

\musicinfo{Slow and plaintive.}{}
\smallskip

\includemusic{chappellV1046.pdf}

\changefontsize{0.97\defaultfontsize}

This ballad is on one of the affairs of gallantry that so frequently arose out of
pilgrimages.

\backskip{1}

\musictitle{Packington’s, or Paggington's Pound.}

This tune is to be found in Queen Elizabeth’s Virginal Book; in \textit{A New Book
of Tablature}, 1596; in the \textit{Collection of English Songs}, printed at Amsterdam, in
1634; in \textit{Select Ayres}, 1659; in \textit{A Choice Collection of 180 Loyal Songs}, 1685;
in Playford’s \textit{Pleasant Musical Companion}, Part II., 1687; in \textit{The Beggars’
Opera}, 1728; in \textit{The Musical Miscellany}, vol. v.; and in many other collections.

It probably took its name from Sir John Packington, commonly called “lusty
Packington,” the same who wagered that he would swim from the Bridge at
Westminster, \ie, Whitehall Stairs, to that at Greenwich, for the sum of 3,000\textit{l}.
“But the good Queen, who had particular tenderness for handsome fellows, would
not permit Sir John to run the hazard of the trial.” His portrait is still preserved
at Westwood, the ancient seat of the family.

In Queen Elizabeth’s Virginal Book it is called Packington’s Pound; by Ben
Jonson, \textit{Paggington’s Pound}; and, in a MS. now in Dr. Rimbault’s possession,
\textit{A Fancy of Sir John Pagington}.

Some copies, viz., that in the Virginal Book, and in the Amsterdam Collection,
have the following difference in the melody of the first four bars:—

\includemusic{chappellV1047.pdf}

\noindent and it is probably the more correct reading, as the other closely resembles the
commencement of “Robin Hood, Robin Hood, said Little John.”

The song in Ben Jonson’s comedy of \textit{Bartholomew Fair}, commencing, “My
masters and friends, and good people, draw near,” was written to this air, and is
thus introduced:—
\settowidth{\versewidth}{\textit{Cokes.} (Sings) Fa, la la la, la la la, fa la la, la! Nay, I’ll put thee in tune and all!}
\begin{scverse}
\textit{Night}. To the tune of Paggington's Pound, Sir?\\
\textit{Cokes.} (Sings) Fa, la la la, la la la, fa la la, la! Nay, I’ll put thee in tune and all!\\
\vin\vin\vin Mine own country dance! Pray thee begin."—\textit{Act} 3.
\end{scverse}
\pagebreak
%124
\changefontsize{\defaultfontsize}

The songs written to the tune are too many for enumeration. Besides those
in the various Collections of Ballads in the British Museum, in D’Urfey’s \textit{Pills},
and in the \textit{Pill to purge State Melancholy}, 1716,—in one Collection alone, viz.,
\textit{The Choice Collection of 180 Loyal Songs}, there are no fewer than thirteen. The
following are curious:—

No. 1. A popular Beggars’ Song, by which the tune is often named, commencing:—
\settowidth{\versewidth}{Or who is so richly cloathed as we.”—\textit{Select Ayres}, 1659.}
\begin{scverse}\vleftofline{“}From hunger and cold who liveth more free?\\
Or who is so richly cloathed as we.”—\textit{Select Ayres}, 1659.
\end{scverse}

No. 2. “Blanket Fair, or the History of Temple Street. Being a relation of
the merry pranks plaid on the river Thames during the great Frost.”
\begin{scverse}“Come, listen awhile, though the weather be cold.”
\end{scverse}

No. 3. “The North Country Mayor,” dated 1697, from a manuscript volume
of Songs by Wilmot, Earl of Rochester, and others, in the Harleian Library:—
\indentpattern{000022000}
\begin{scverse}\begin{patverse}
\vleftofline{“}I sing of no heretic Turk, or of Tartar,\\
But of a suffering Mayor who may pass for a Martyr;\\
For a story so tragick was never yet told\\
By Fox or by Stowe, those authors so old;\\
How a vile Lansprasado\\
Did a Mayor bastinado,\\
And played him a trick worse than any Strappado:\\
O Mayor, Mayor, better ne’er have transub’d, [turn’d Papist]\\
Than thus to be toss’d in a blanket and drubb’d,” \&c.
\end{patverse}
\end{scverse}

The following song, in praise of milk, is from Playford’s \textit{Musical Companion},
Part II., 1687:—

\musicinfo{Moderate time and smoothly.}{}

\includemusic{chappellV1048.pdf}

\pagebreak
%125

\indentpattern{0011100}
\begin{scverse}\begin{patverse}
The first of fair dairy-maids, if you’ll believe,\\
Was Adam’s own wife, our great-grandmother Eve,\\
Who oft milk’d a cow,\\
As well she knew how;\\
Though butter was not then so cheap as ’tis now,\\
She hoarded no butter nor cheese on her shelves,\\
For butter and cheese in those days made themselves.
\end{patverse}

\begin{patverse}
In that age or time there was no horrid money,\\
Yet the children of Israel had both milk and honey:\\
No queen you could see,\\
Of the highest degree,\\
But would milk the brown cow with the meanest she;\\
Their lambs gave them clothing, their cows gave them meat,\\
And in plenty and peace all their joys were compleat.
\end{patverse}

\begin{patverse}
Amongst the rare virtues that milk does produce,\\
For a thousand of dainties it’s daily in use;\\
Now a pudding I’ll tell thee,\\
Ere it goes in the belly,\\
Must have from good milk both the cream and the jelly:\\
For a dainty fine pudding, without cream or milk,\\
Is a citizen's wife without satin or silk.
\end{patverse}

\begin{patverse}
In the virtues of milk there is more to be muster’d,\\
The charming delights both of cheese-cake and custard,\\
For at Tottenham Court,\\
You can have no sport,\\
Unless you give custards and cheese-cake too for’t;\\
And what's the jack-pudding that makes us to laugh,\\
Unless he hath got a great custard to quaff?
\end{patverse}

\begin{patverse}
Both pancake and fritter of milk have good store,\\
But a Devonshire whitepot\dcfootnote{
Devonshire white-pot, or hasty-pudding, consisting of
flour and milk boiled together.} must needs have much more;\\
No state you can think,\\
Though you study and wink,\\
From the lusty sack-posset\dcfootnote{
The following is a receipt for sack-posset:—
\settowidth{\versewidth}{Fetch sugar, half a pound; fetch sack, from Spain,}
\begin{fnverse}\scriptsize
\vleftofline{“}From fair Barbadoes, on the western main,\\
Fetch sugar, half a pound; fetch sack, from Spain,\\
A pint; then fetch, from India’s fertile coast,\\
Nutmeg, the glory of the British toast."\\
\vin\vin\vin\textit{Dryden's Miscellany Poems}, vol. v., p, 138.
\end{fnverse}} to poor posset drink,\\
But milk’s the ingredient, though sack’s ne’er the worse,\\
For 'tis sack makes the man, though 'tis milk makes the nurse.
\end{patverse}
\end{scverse}

Elderton’s ballad, called “News from Northumberland,” a copy of which is in
the Library of the Society of Antiquaries, was probably written to this tune.

\musictitle{The Staines Morris Tune.}

This tune is taken from the first edition of \textit{The Dancing Master}.	It is also in
William Ballet’s Lute Book (time of Elizabeth); and was printed as late as about
1760, in a Collection of Country Dances, by Wright.

The Maypole Song, in \textit{Actæon and Diana}, seems so exactly fitted to the air,
that, haying no guide as to the one intended, I have, on conjecture, printed it
with this tune.
\pagebreak
%126

\musicinfo{Boldly and rather quick.}{}
\medskip

\includemusic{chappellV1049.pdf}

\indentpattern{000011}
\settowidth{\versewidth}{What kisses you your sweethearts gave,}
\begin{dcverse}\begin{patverse}
It is the choice time of the year,\\
For the violets now appear;\\
Now the rose receives its birth,\\
And pretty primrose decks the earth.\\
Then to the May-pole come away,\\
For it is now a holiday.
\end{patverse}

\begin{patverse}
Here each batchelor may chuse\\
One that will not faith abuse;\\
Nor repay with coy disdain\\
Love that should be loved again.\\
Then to the May-pole come away,\\
For it is now a holiday.
\end{patverse}

\begin{patverse}
And when you well reckoned have\\
What kisses you your sweethearts gave,\\
Take them all again, and more,\\
It will never make them poor.\\
Then to the May-pole come away,\\
For it is now a holiday.
\end{patverse}

\begin{patverse}
When you thus have spent the time\\
Till the day be past its prime,\\
To your beds repair at night,\\
And dream there of your day’s delight.\\
Then to the May-pole come away,\\
For it is now a holiday.
\end{patverse}
\end{dcverse}

\backskip{1}

\musictitle{The Shepherd’s Daughter.}
\small %cramped page
This is in every edition of \textit{The Dancing Master}, except the first, either under
the name of \textit{The Shepherd’s Daughter}, or \textit{Parson and Dorothy}. It is also under
the latter title in several of the ballad operas. Percy says the ballad of \textit{The
Knight and Shepherd’s Daughter}, “was popular in the time of Queen Elizabeth,
being usually printed with her picture before it, as Hearne informs us in his preface
to \textit{Gul. Neubrig. Hist. Oxon}., vol. i., 70.

Four lines are quoted in Fletcher’s comedy \textit{The Pilgrim}, act iv., sc. 2: “He
called down his merry men all,” \&c.; and in \textit{The Knight of the Burning Pestle}:
“He set her on a milk-white steed,”~\&c.


\begin{center}\scriptsize 
$^{a}$ In William Ballet’s Lute Book, the third note of the melody is E; in the 2nd edition of \textit{The Dancing Master}, B.
\end{center}

%127
Copies of the ballad will be found in the Roxburghe Collection, vol. ii., 30;
and in the Douce Collection, with the burden or chorus, “Sing, trang, dildo dee,”
at the end of each verse, which is not given by Percy. The two last bars are
here added for the burden. In some copies the four first bars are repeated.

\musicinfo{Rather slow.}{}

\includemusic{chappellV1050.pdf}

The ballad will be found in Percy’s \textit{Reliques of Ancient Poetry}, series 3, book i.

\musictitle{The Frog Galliard, or Now, O Now!}

This is the only tune, composed by a well-known musician of the age, that
I have found employed as a ballad tune.

In Dowland’s \textit{First Book of Songes}, 1597, it is adapted to the words, “Now,
O now, I needs must part” (to be sung by one voice with the lute, or by four
without accompaniment); but in his Lute Manuscripts it is called \textit{The Frog
Galliard}, and seems to have been commonly known by that name.

In Morley’s \textit{Consort Lessons}, 1599 and 1611, it is called \textit{The Frog Galliard};
in Thomas Robinson’s \textit{New Citharen Lessons}, 1609, The \textit{Frog}; and in the Skene
Manu\-script, \textit{Froggis Galziard}.

In \textit{Nederlandtsche Gedenck-Clanck}, printed at Haerlem in 1626, it is called
\textit{Nou, nou} [for Now, O now]; but all the ballads I have seen, that were written
to it, give the name as \textit{The Frog Galliard}.

In Anthony Munday’s \textit{Banquet of daintie Conceits}, 1588, there is a song to the
tune of \textit{Dowland’s Galliard}, but it could not be sung to this air.

It seems probable that \textit{Now, O now}, was originally a dance tune, and the
composer finding that others wrote songs to his galliards, afterwards so adapted
it likewise.

The latest Dutch copy that I have observed is in Dr. Camphuysen’s \textit{Stichtelycke
Rymen}, printed at Amsterdam in 1647.

Dowland is celebrated in the following sonnet, which, from having appeared in
\textit{The Passionate Pilgri}m, has been attributed to Shakespeare, but was published
previously in a Collection of Poems by Richard Barnfield.
\pagebreak
%128

\settowidth{\versewidth}{One knight loves both, and both in thee remain!}
\begin{scverse}\vleftofline{“\textit{To his}}\textit{ friend,, Master R. L., in praise of Music and Poetry}!”\\
\indentpattern{01010101010100}
\begin{patverse}
\vleftofline{“}If music and sweet poetry agree,\\
As they must needs, (the sister and the brother,)\\
Then must the love be great ’twixt thee and me,\\
Because thou lov’st the one, and I the other.\\
\textit{Dowland} to thee is dear, whose heavenly touch\\
Upon the lute doth ravish human sense;\\
\textit{Spenser} to me, whose deep conceit is such,\\
As, passing all conceit, needs no defence;\\
Thou lov’st to hear the sweet melodious sound\\
That Phœbus’ lute, the queen of music, makes,\\
And I, in deep delight am chiefly drown’d,\\
When as himself to singing he betakes;\\
One God is good to both, as poets feign,\\
One knight loves both, and both in thee remain!”
\end{patverse}
\end{scverse}

Anthony Wood says of Dowland, that “he was the rarest musician that the
age did behold.” In \textit{No Wit, no Help, like a Woman’s}, a comedy by Thomas
Middleton (1657), the servant tells his master bad news; and is thus answered:
“Thou plaiest Dowland’s \textit{Lachrimæ} to thy master.”

In Peacham’s \textit{Garden of Heroical Devices}, are the following verses, portraying
Dowland’s forlorn condition in the latter part of his life:—
\settowidth{\versewidth}{So since (old friend) thy years have made thee white,}
\indentpattern{010100}
\begin{scverse}\begin{patverse}
\vleftofline{“}Here Philomel in silence sits alone\\
In depth of winter, on the bared briar,\\
Whereon the rose had once her beauty shown,\\
Which lords and ladies did so much desire!\\
But fruitless now, in winter’s frost and snow,\\
It doth despis’d and unregarded grow.
\end{patverse}

\begin{patverse}
So since (old friend) thy years have made thee white,\\
And thou for others hast consum’d thy spring,\\
How few regard thee, whom thou didst delight,\\
And far and near came once to hear thee sing!\\
Ungrateful times, and worthless age of ours,\\
That lets us pine when it hath cropt our flowers.”
\end{patverse}
\end{scverse}

The device which precedes these stanzas, is a nightingale sitting on a bare
brier, in the midst of a wintry storm.

The following ballads were sung to the tune under the title of \textit{The Frog
Galliard}:\ —“The true love’s-knot untyed: being the right path to advise princely
virgins how to behave themselves, by the example of the renouned Princess, the
Lady Arabella, and the second son to the Lord Seymore, late Earl of Hertford;”
commencing—
\begin{scverse}\begin{altverse}
“As I to Ireland did pass,\\
I saw a ship at anchor lay,\\
Another ship likewise there was,\\
Which from fair England took her way.\\
This ship that sail’d from fair England,\\
Unknown unto our gracious King,\\
The Lord Chief Justice did command,\\
That they to London should her bring,” \&c.
\end{altverse}
\end{scverse}
\pagebreak
%129

A copy in the British Museum Collection, and printed by Evans in \textit{Old Ballads},
1810, vol. iii., 184.

Also, “The Shepherd’s Delight,” commencing—
\settowidth{\versewidth}{And by that flower there stands a bower,}
\begin{scverse}
\begin{altverse}
“On yonder hill there stands a flower,\\
Fair befall those dainty sweets;\\
And by that flower there stands a bower,\\
Where all the heavenly muses meet,” \&c.
\end{altverse}
\end{scverse}

A copy in the Roxburghe Collection, vol, i., 388, and Evans, vol. i., 388.

\musicinfo{Slowly and smoothly.}{}

\includemusic{chappellV1051.pdf}

\backskip{1}

\settowidth{\versewidth}{Dear, when I from thee am gone,}
\begin{dcverse}
\begin{altverse}
Dear, when I from thee am gone,\\
Gone are all my joys at once!\\
I loved thee, and thee alone,\\
In whose love I joyed once.\\
While I live I needs must love,\\
Love lives not when life is gone:
\end{altverse}

\begin{altverse}
Now, at last, despair doth prove\\
Love divided loveth none.\\
And although your sight I leave,\\
Sight wherein my joys do lie,\\
Till that death do sense bereave,\\
Never shall affection die.
\end{altverse}
\end{dcverse}
\pagebreak
%130

\musictitle{Paul’s Wharf}

This tune is in Queen Elizabeth’s Virginal Book, and in \textit{The Dancing Master},
from 1650 to 1665.

Paul’s Wharf was, and still is, one of the public places for taking water, near
to St. Paul’s Cathedral. In “The Prices of Fares and Passages to be paide to
Watermen,” printed by John Cawood, (n.d.,) is the following: “Item, that no
Whyry manne, with a pare of ores, take for his fare from Pawles Wharfe, Queen
hithe, Parishe Garden, or the blacke Fryers to Westminster, or White hall, or
lyke distance to and fro, above iij\textit{d}.

\musicinfo{Gracefully.}{}

\includemusic{chappellV1052.pdf}


\musictitle{Trip and Go}
This was one of the favorite Morris-dances of the sixteenth and seventeenth
centuries, and frequently alluded to by writers of those times.

Nashe, in his Introductory Epistle to the surreptitious edition of Sidney’s
\textit{Astrophel and Stella}, 4to., 1591, says, “Indeede, to say the truth, my stile is
somewhat heavie gated, and cannot daunce \textit{Trip and goe} so lively, with ‘Oh my
love, ah my love, all my love gone,’ as other shepheards that have beene \textit{Fooles in
the morris}, time out of minde.” He introduces it more at length, and with a
description of the Morris-dance, in the play of \textit{Summer’s last Will and Testament},
1600:

\settowidth{\versewidth}{“\textsc{Ver}.-- \textit{goes in and fetcheth out the Hobby-horse and the Morris-dance, who}.}
\begin{scverse}
“\textsc{Ver}.-- \textit{goes in and fetcheth out the Hobby-horse and the Morris-dance, who
dance about}.
\end{scverse}

\textit{Ver}.—“About, about! lively, put your horse to it; rein him harder; jerk him with
your wand. Sit fast, sit fast, man! Fool, hold up your ladle\dcfootnote{\textit{}
The ladle is still used by the sweeps on May-day.}
 there.”

\textit{Will Summer}.—“0 brave Hall!\dcfootnote{\textit{}
The tract of “Old Meg of Herefordshire for a Mayd
Marian, and Hereford towne for a Morris-dance,” 4to,
1609, is dedicated to \textit{old Hall}, a celebrated Taborer of
Herefordshire; and the author says,—“The People of
Herefordshire are beholding to thee; thou givest the men
light hearts by thy pipe, and the women light heeles by
thy tabor, O wonderful piper! O admirable tabor-man!”
\dots “The wood of this olde Hall’s tabor should
have beene made a paile to carie water in at the beginning
of King Edward the Sixt’s reigne; but Hall (being wise,
because hee was even then reasonably well strucken in
years) saved it from going to the water, and converted it
in these days to a tabor.” For more about old Hall and
his pipe and tabor, see page 134.}
 O well said, butcher! Now for the credit of
Worcestershire. The finest set of Morris-dancers that is between this and Streatham. 
Marry, methinks there is one of them danceth \pagebreak like a clothier’s horse, with a wool-pack
%131
upon his back. You, friend, with the hobby-horse, go not too fast, for fear of wearing
out my lord’s tile-stones with your hob-nails.”

\textit{Ver}.—“So, so, so; trot the ring twice over, and away.”

After this, three clowns and three maids enter, dancing, and singing the song
which is here printed with the music.

\textit{Trip and go} seems to have become a proverbial expression. In Gosson’s \textit{Schoole
of Abuse}, 1579: “\textit{Trip and go}, for I dare not tarry.” In \textit{The two angrie Women
of Abington}, 1599: “Nay, then, \textit{trip and go}.” In Ben Jonson’s \textit{Case is altered}:
“O delicate \textit{trip and go}.” And in Shakespeare’s \textit{Love's Labour Lost}: “\textit{Trip
and go}, my sweet.”

The tune is taken from \textit{Musick's Delight on the Cithren}, 1666. It resembles
another tune, called \textit{The Boatman}. (See Index.)

\musicinfo{Moderate time and trippingly.}{}

\includemusic{chappellV1053.pdf}

The Morris-dance was sometimes performed by itself, but was much more
frequently joined to processions and pageants, especially to those appointed for
the celebration of May-day, and the games of Robin Hood. The festival, instituted
in honour of Robin Hood, was usually solemnized on the first and
succeeding days of May, and owes its original establishment to the cultivation
and improvement of the manly exercise of archery, which was not, in former
times, practised merely for the sake of amusement.

\changefontsize{1.05\defaultfontsize}

“I find,” says Stow, “that in the month of May, the citizens of London, of all
estates, lightly in every parish, or sometimes two or three parishes joining
together, had their several \textit{Mayings}, and did fetch in May-poles, with divers
\textit{warlike shews}, with good \textit{archers, Morris-dancers}, \pagebreak and other devices for pastime all
%132
\markboth{morris dance and may-day.}{reign of elizabeth.}
the day long: and towards the evening they had stage-plays and bonfires in the
streets\ldots These great Mayings and May-games, made by the governors and
masters of this city, with the triumphant setting up of the great shaft (a principal
Maypole in Cornhill, before the parish church of St. Andrew, which, from the pole
being higher than the steeple itself, was, and still is, called St. Andrew Undershaft), 
by means of an insurrection of youths against aliens on May-day, 1517,\dcfootnote{\textit{}
 The “story of {\blackletter M} May-day, in the time of Henry the
Eight, and why it is so called; and how Queen Catherine
begged the lives of two thousand London apprentices,” is
the subject of an old ballad in Johnson’s \textit{Crown Garland
of Golden Roses}, and has been reprinted in Evans’ \textit{Old
Ballads}, vol. iii. p. 76, edition of 1810.}
 the
ninth of Henry the Eighth, have not been so freely used as afore.”—\textit{Survey of
London}, 1598, p. 72.

The celebration of May-day may be traced as far back as Chaucer, “who, in
the conclusion of his \textit{Court of Love}, has described the Feast of May, when—”

\begin{scverse}
\vleftofline{“}Forth go’th all the court, both most and least.\\
To fetch the floures fresh, and braunch and bloom—\\
And namely hawthorn brought, both page and groom;\\
And they rejoicen in their great delight;\\
Eke each at other throw the floures bright,\\
The primerose, the violete, and the gold,\\
With freshe garlants party blue and white.’’
\end{scverse}

Henry the Eighth appears to have been particularly attached to the exercise of
archery, and the observance of May. “Some short time after his coronation,”
says Hall, “he came to Westminster, with the queen, and all their train: and on
a time being there, his grace, the Earls of Essex, Wiltshire, and other noblemen,
to the number of twelve, came suddenly in a morning into the queen’s chamber,
all appareled in short coats of Kentish Kendal, with hoods on their heads, and
hosen of the same, every one of them his bow and arrows, and a sword and
buckler, like outlaws or Robin Hood’s men; whereof the queen, the ladies, and
all other there, were abashed, as well for the strange sight, as also for their
sudden coming: and, after certain dances and pastime made, they departed.”—
\textit{Hen. VIII}., fo. 6, b. The same author gives a curious account of Henry and
Queen Catherine going a Maying.

Bourne, in his \textit{Antiquitates Vulgares}, says, “On the Calends, or first-day of
May, commonly called May-day, the juvenile part of both sexes were wont to rise
a little before midnight and walk to some neighbouring wood, accompanied with
music, and the blowing of horns, where they brake down branches from the trees,
and adorn them with nosegays and crowns of flowers. When this is done, they
return with their booty homewards, about the rising of the sun, and make their
doors and windows to triumph in the flowery spoil. The after part of the day is
chiefly spent in dancing round a tall pole, they call a May-pole; which being
placed in a convenient part of the village, stands there, as it were consecrated
to the goddess of flowers, without the least violence offered it in the whole circle
of the year.” Borlase, in his \textit{Natural History of Cornwall}, tells us, “An ancient
custom, still retained by the Cornish, is that of decking their doors and porches,
on the first of May, with green sycamore and hawthorn boughs, and of planting
trees, or rather stumps of trees, before \pagebreak their houses: and on May-eve, they from
%133
towns make excursions into the country, and having cut down a tall elm, brought
it into town, fitted a straight and taper pole to the end of it, and painted the
same, erect it in the most public places, and on holidays and festivals adorn it
with flower garlands, or insigns and streamers.”

Philip Stubbes, the puritan, who declaims as vehemently against May-games as
against dancing, minstrelsy, and other sports and amusements, thus describes
“the order of their May-games” in this reign. “Against May, Whitsuntide, or
some other time of the year, every parish, town, and village, assemble themselves
together, both men, women, and children; and either all together, or dividing
themselves into companies, they go, some to the woods and groves, some to the
hills and mountains, some to one place, some to another, and in the morning they
return, bringing with them birch, boughs, and branches of trees, to deck their
assemblies withal\ldots  But their chiefest jewel they bring from thence is their
May-pole, which they bring home with great veneration, as thus: they have
twenty or forty yoke of oxen, every ox having a sweet nosegay of flowers tied to
the tip of his horns; and these oxen draw home this May-pole, (this stinking
idol rather), which is covered all over with flowers and herbs, bound round
about with strings, from the top to the bottom, and sometime painted with
variable colours, with two or three hundred men, women, and children, following
it with great devotion. And thus, being reared up, with handkerchiefs
and flags streaming on the top, they strew the ground about, bind green boughs
about it, set up summer halls, bowers, and arbours, hard by it; and then fall
they to banquet and feast, to leap and dance about it, as the heathen people
did at the dedication of their idols, whereof this is a perfect pattern, or rather
the thing itself.”—(\textit{Anatomie of Abuses}, reprint of 1585 edit., p. 171.)

Browne, also, has given a similar description of the May-day rites, in his
\textit{Britannia's Pastorals}, book ii., song 4:—
\settowidth{\versewidth}{When envious night commands them to be gone,}
\begin{scverse}
“As I have seen the Lady of the May\\
Sit in an arbour,\ldots\\
Built by a May-pole, where the jocund swains\\
Dance with the maidens to the bagpipe’s strains,\\
When envious night commands them to be gone,\\
Call for the merry youngsters one by one,\\
And, for their well performance, ‘she’ disposes\\
To this a garland interwove with roses;\\
To that a carved hook, or well-wrought scrip;\\
Gracing another with her cherry lip:\\
To one her garter; to another, then,\\
A handkerchief, cast o’er and o’er again;\\
And none returneth empty, that hath spent\\
His pains to fill their rural merriment.”
\end{scverse}

The Morris-dance, when performed on May-day, and \textit{not} connected with the
Games of Robin Hood, usually consisted of the Lady of the May, the fool or jester,
a piper, and two, four, or more, morris-dancers. But, on other occasions, the hobby-horse, 
and sometimes a dragon, with Robin Hood, Maid Marian, Friar Tuck, Little 
John, and other characters supposed to have \pagebreak been the companions of that famous
%134
outlaw, were added to the dance. Maid Marian was sometimes represented by a
smooth-faced youth, dressed in a female garb; Friar Tuck, Robin Hood’s chaplain,
by a man of portly form, in the habit of a Franciscan friar; the hobby-horse was a
paste-board resemblance of the head and tail of a horse, on a wicker frame, and
attached to the body of a man, whose feet being concealed by a foot-cloth hanging
to the ground, he was to imitate the ambling, the prancing, and the curveting of
the horse; the dragon (constructed of the same materials) was made to hiss, yell,
and shake his wings, and was frequently attacked by the man on the hobby-horse,
who then personated St. George.

The garments of the Morris-dancers were adorned with bells, which, were not
placed there merely for the sake of ornament, but were sounded as they danced.
These, which were worn round the elbows and knees, were of unequal sizes,
and differently denominated; as the fore bell, the second bell, the treble, the mean
or countertenor, the tenor, the great bell or base, and sometimes double bells were
worn.\dcfootnote{\textit{}
For the bells of the Morris, see Ford’s play,\textit{ The Witch
of Edmonton}, act 2, sc. 1. Weber is mistaken as to
“mean” meaning tenor.}
 The principal dancer in the Morris was more superbly habited than his
companions; as appears from a passage in \textit{The blind Beggar of Bethnall Green}
(dramatised from the ballad of the same name), by John Day, 1659: “He wants
no clothes, for he hath a cloak laid on with gold lace, and an embroidered jerkin;
and thus he is marching hither \textit{like the foreman of a morris}.”

In \textit{The Vow-breaker}, or \textit{Fair Maid of Clifton}, by William Sampson, 1636,
we find, “Have I not practised my reins, my careers, my prankers, my ambles,
my false trots, my smooth ambles, and Canterbury paces—and shall the mayor
put me, besides the hobby-horse? I have borrowed the fore-horse bells, his
plumes, and braveries; nay, I have had the mane new shorn and frizzled. Am
I not going to buy ribbons and toys of sweet Ursula for the Marian—and shall
I not play the hobby-horse? Provide thou the dragon, and let me alone for the
hobby-horse.” And afterwards: “Alas, sir! I come only to borrow a few
ribbands, bracelets, ear-rings, wire-tiers, and silk girdles, and handkerchers, for a
Morris and a show before the queen; I come to furnish the hobby-horse.”

There is a curious account of twelve persons of the average age of a hundred
years, dancing the Morris, in an old book, called “Old Meg of Herefordshire for
a Mayd Marian, and Hereford towne for a Morris-dance; or twelve Morris-dancers
in Herefordshire of 1200 years old,”\dcfootnote{\textit{}
Brand, in his Popular Antiquities, vol.2, p.208, 1813,
gives an account of a May-game, or Morris-dance, by
\textit{eight} persons in Herefordshire, whose ages, computed
together, amounted to 800 years; probably the same as
mentioned by Lord Bacon, as happening “a few years
since in the county of Hereford.” See \textit{History, Natural
and Experimental, of Life and Death}, 1638.}
 quarto, 1609. It is dedicated to the renowned
old Hall, taborer of Herefordshire, and to “his most invincible weatherbeaten
nut-brown tabor, which hath made bachelors and lasses dance round
about the May-pole, three-score summers, one after another in order, and is not
yet worm-eaten.” Hall, who had then “stood, like an oak, in all storms, for
ninety-seven winters,” is recommended to “imitate that Bohemian Zisca, who at
his death gave his soldiers a strict command to flay his skin off, and cover a drum
with it, that alive and dead he might sound like a terror in the ears of his enemies: 
so thou, sweet Hereford Hall, bequeath \pagebreak in thy last will, thy vellum-spotted skin
%135
to cover tabors; at the sound of which to set all the shires a dancing\dots The
court of kings is for stately \textit{measures}; the city for light heels and nimble footing;
western men for, gambols; Middlesex men for tricks above ground; Essex men
for the \textit{Hey}; Lancashire for \textit{Hornpipes}; Worcestershire for bagpipes; but Herefordshire
for a Morris-dance, puts down not only all Kent, but very near (if one
had line enough to measure it) three quarters of Christendom. Never had Saint
Sepulchre’s a truer ring of bells; never did any silk-weaver keep braver time;
never could Beverley Fair give money to a more sound taborer; nor ever had
Robin Hood a more deft Maid Marian.”

Full particulars of the Morris-dance and May-games may be found by referring
to Strutt’s \textit{Sports and Pastimes}; to Ritson’s \textit{Robin Hood}; to an account of a
painted window, appended to part of Henry IV., in Steevens’ \textit{Shakespeare}, the
xv. vol. edition; to Gifford’s \textit{Ben Jonson}, vol. i., pages 50, 51, 52, vol. iv., p. 405,
and vol.~vii., p. 397; to \textit{The British Bibliographer}, vol. iv., p. 326; Brand’s
\textit{Popular Antiquities}; Douce’s \textit{Illustrations of Shakespeare}; and Dr. Drake’s
\textit{Shakespeare and his Times}, vol. i., \&c., \&c.

\musictitle{BARLEY BREAK}

From Lady Neville's Virginal Book, which was transcribed in 1591.

\musicinfo{Stately.}{}

\includemusic{chappellV1054.pdf}

\pagebreak
%136
\markboth{english song and ballad music.}{reign of elizabeth.}
\changefontsize{1.0\defaultfontsize}

Gifford has given the following description of the sport called Barley-break, in
a note upon Massinger’s \textit{Virgin Martyr}, act v., sc. 1:—“Barley-break was
played by six people\dcfootnote{\textit{}
Rather, perhaps, by \textit{not less} than six people.
“Heyday! there are a \textit{legion} of young cupids at Barlibreak."
—\textit{The Guardian}, act i., sc. 1.}
 (three of each sex), who were coupled by lot. A piece of
ground was then chosen and divided into three compartments, of which the middle
one was called Hell. It was the object of the couple condemned to this division,
to catch the others, who advanced from the two extremities; in which case a
change of situation took place, and hell was filled by the couple who were excluded
by pre-occupation, from the other places: in this ‘catching,’ however, there was
some difficulty, as, by the regulations of the game, the middle couple were not
to separate before they had succeeded, while the others might break hands whenever
they found themselves hard pressed. When all had been taken in turn, the
last couple was said \textit{to be in hell}, and the game ended.” In this description,
Gifford does not in any way allude to it as a dance, but Littleton explains \textit{Chorus
circularis}, barley-break, when they dance, taking their hands round. See Payne
Collier’s note on Dodsley’s \textit{Old Plays}, vol. iii., p. 316. Strutt, in his \textit{Sports and
Pastimes}, quotes only two lines from Sidney, which he takes from Johnson’s
Dictionary:—
\begin{scverse}
“By neighbours prais’d, she went abroad thereby,\\
At barley-brake her sweet swift feet to try.”
\end{scverse}
In the Roxburghe Collection, vol. i., 344, is a ballad called “The Praise of our
Country Barley-brake, or—
\begin{scverse}
Cupid’s advisement for young men to take\\
Up this loving old sport, called Barley-brake."
\end{scverse}
“To the tune of \textit{When this old cap was new}.” It commences thus:—
\begin{scverse}
\begin{altverse}
\vleftofline{“}Both young men, maids, and lads,\\
Of what state or degree,\\
Whether south, east, or west,\\
Or of the north country;\\
I wish you all good health,\\
That in this summer weather\\
Your sweet-hearts and yourselves\\
Play at barley-break together.” \&c.
\end{altverse}
\end{scverse}

Allusions to \textit{Barley-break} occur repeatedly in our old writers. Mr. M. Mason
quotes a description of the pastime with allegorical personages, from Sir John
Suckling:—
\begin{scverse}
\vleftofline{“}Love, Reason, Hate, did once bespeak\\
Three mates to play at Barley-break;\\
Love Folly took, and Reason Fancy;\\
And Hate consorts with Pride; \textit{so dance they},” \&c.
\end{scverse}

\musictitle{Watkin’s Ale}

The tune from Queen Elizabeth’s Virginal Book, where it is arranged by Byrd.
Ward, in his \textit{Lives of the Gresham Professors}, states that it is also contained in
one of the MSS. formerly belonging to Dr. John Bull. A copy of the original
ballad is in the collection of Mr. George Daniel, of Canonbury. \textit{Watkin’s Ale} is
referred to in a letter prefixed to \pagebreak Anthony Munday’s translation of \textit{Gerileon in
\normalsize
%137
England}, part ii., 1592, and in Henry Chettle’s pamphlet, \textit{Kind-harts Dreame},
printed in the same year. The ballad is entitled:

\begin{scverse}“A ditty delightful of Mother Watkin’s ale\\
A warning well weighed, though counted a tale.”
\end{scverse}

\musicinfo{Moderate time.}{}

\includemusic{chappellV1055.pdf}

Each part of the tune is to be repeated for words. The following stanzas
is the seventh:--
\settowidth{\versewidth}{He made thereof a country dance.}
\begin{dcverse}\indentpattern{00002224}
\begin{patverse}
Thrice scarcely changed hath the moon\\
Since first this pretty trick was done;\\
Which being heard of one by chance,\\
He made thereof a country dance.\\
And as I heard the tale,\\
He called it Watkin’s Ale,\\
Which never will be stale\\
I do believe;
\end{patverse}

\indentpattern{22240000}
\begin{patverse}
\vin\vin This dance is now in prime,\\
And chiefly us’d this time,\\
And lately put in rhime:\\
Let no man grieve,\\
To hear this merry jesting tale,\\
The which is called Watkin’s Ale:\\
It is not long since it was made,\\
The finest flower will soonest fade.
\end{patverse}
\end{dcverse}

\musictitle{The Carman’s Whistle}

This tune is in Queen Elizabeth’s and Lady Neville’s Virginal Books (arranged
by Byrd), as well as in several others of later date. The ballad is mentioned in a
letter, bearing the signature of T. N., addressed to his good friend A[nthony]
M[unday], prefixed to the latter’s translation of \textit{Gerileon of England}, part ii.,
quarto, 1592; and by Henry Chettle in his \textit{Kind-harts Dreame}, printed in the
same year.
\pagebreak
%138
\changefontsize{1.08\defaultfontsize}
\normalsize

The Carmen of the sixteenth and seventeenth centuries appear to have been
singularly famous for their musical abilities; but especially for whistling their
tunes. Falstaff’s description of Justice Shallow is, that “he came ever in the
rear-ward of the fashion,” and “sang the tunes he heard the carmen whistle,
and sware they were his Fancies, or his Good-nights.”\dcfootnote{\textit{}Good-nights are “Last dying speeches” made into
ballads. See Essex’s last Good-night.}
—(\textit{Henry IV}., Part ii.,
act 3.) In Ben Jonson’s \textit{Bartholomew Fair}, Waspe says, “I dare not let him
walk alone, for fear of learning vile tunes, which he will sing at supper, and in
the sermon times! If he meet but a carman in the street, and I find him not
talk to keep him off on him, he will whistle him all his tunes over at night, in his
sleep.”—(Act i., sc. 1.) In the tract called “The World runnes on Wheeles,”\dcfootnote{\textit{}
Taylor's tract was written against coaches, which injured
his trade as a waterman. He says, “In the year
1564, one William Boonen, a Dutchman, brought first the
use of coaches hither, and the said Boonen was Queen
Elizabeth’s coachman, for indeed a coach was a strange
monster in those days, and the sight of them put both
horse and man into amazement. Some said it was a great
crab-shell, brought out of China, and some imagined it
to be one of the Pagan temples, in which the cannibals
adored the devil.” He argues that the cart-horse is a
more learned beast than a coach-horse, “for scarce any
coach-horse in the world doth know any letter in the book;
when as every cart-horse doth know the letter \textit{G} most
understandingly.”}
by Taylor, the Water-poet, he says, “If the carman’s horse be melancholy or
dull with hard and heavy labour, then will he, like a kind piper, whistle him a
fit of mirth to any tune, from above. Eela to below Gammoth;\dcfootnote{\textit{}
Gamut, then the lowest note of the scale, as Eela was
the highest.}
 of which generosity
and courtesy your \textit{coachman} is altogether ignorant, for he never whistles,
but all his music is to rap out an oath.” And again he says, “The word \textit{carmen},
as I find it in the [Latin] dictionary, doth signify a verse, or a song; and betwixt
car\textit{men} and car\textit{man}, there is some good correspondence, for versing, singing, and
whistling, are all three musical.” Burton, in his \textit{Anatomy of Melancholy}, says,
“A carman’s whistle, or a boy singing some ballad early in the street, many
times alters, revives, recreates a restless patient that cannot sleep;” and again,
“As carmen, boys, and prentices, when a new song is published with us, go singing
that new tune still in the streets.” Henry Chettle, in his \textit{Kind-hart’s
Dreame}, says, “It would be thought the carman, that was wont to whistle to his
beasts a comfortable note, might as well continue his old course, whereby his
sound served for a musical harmony in God’s ear, as now to follow profane
jigging vanity.” In\textit{ The Pleasant Historie of the two angrie Women of Abington},
quarto, 1599, Mall Barnes asks, “But are ye cunning in the carman’s lash, and
can ye whistle well?” In \textit{The Hog hath lost its Pearl}, Haddit, the poet, tells the
player shortly to expect “a notable piece of matter; such a jig, whose tune, with
the natural whistle of a carman, shall be more ravishing to the ears of shopkeepers
than a whole concert of barbers at midnight.”--(\textit{Dodsley’s Old Plays},
vol. vi.) So in Lyly’s \textit{Midas}, “A carter with his whistle and his whip, in \textit{true}
ears, moves as much as Phœbus with his fiery chariot and winged horses.” In
Heywood’s \textit{A Woman hill’d with Kindness}, although all others are sad, the stage
direction is, “Exeunt, except Wendall and Jenkin; \textit{the carters whistling}.“And
Playford, in his \textit{Introduction to the skill of Music}, 1679, says, “Nay, the poor
labouring beasts at plough and cart are cheered by the sound of music, though it
be but their master’s whistle.”
\pagebreak
%139

The following ballads were sung to the tune:—“The Comber’s Whistle, or The
Sport of the Spring,” commencing—
\settowidth{\versewidth}{“All in a pleasant morning;”}
\begin{scverse}
“All in a pleasant morning;”
\end{scverse}
a copy in Pepys’ Collection, vol. iii., 291, and Roxburghe Collection, vol. ii., 67.
“All is ours and our husbands’, or the Country Hostesses’ Vindication;” a copy
in the Roxburghe Collection, vol. ii., 8.

“The Courteous Carman and the Amorous Maid: or the Carman’s Whistle,”\dcfootnote{\textit{}
There are twelve stanzas in the ballad, of which five
are here omitted. A black-letter copy in the Douce
Collection, fol. 33, and one in Mr. Payne Collier’s Collection.}
\&c., “To the tune of \textit{The Carman's Whistle}; or \textit{Lord Willoughby's March}.”

\musicinfo{Gracefully.}{}

\includemusic{chappellV1056.pdf}

\settowidth{\versewidth}{So comely was her countenance,}
\begin{dcverse}\begin{altverse}
So comely was her countenance,\\
And ‘winning was her air,’\\
As though the goddess Venus\\
Herself she had been there;\\
And many a smirking smile she gave\\
Amongst the leaves so green,\\
Although she was perceived,\\
She thought she was not seen.
\end{altverse}

\begin{altverse}
At length she chang’d her countenance,\\
And sung a mournful song,\\
Lamenting her misfortune\\
She staid a maid so long;\\
Sure young men are hard-hearted,\\
And know not what they do,\\
Or else they want for compliments\\
Fair maidens for to woo.
\end{altverse}
\end{dcverse}
\pagebreak
%140

\changefontsize{1.02\defaultfontsize}

\begin{dcverse}\begin{altverse}
Why should young virgins pine away\\
And lose their chiefest prime;\\
And all for want of sweet-hearts,\\
To cheer us up in time?\\
The young man heard her ditty,\\
And could no longer stay,\\
But straight unto the damosel\\
With speed he did away.
\end{altverse}

\begin{altverse}
When he had played unto her\\
One merry note or two,\\
Then was she so rejoiced,\\
She knew not what to do:\\
O God-a-mercy, carman,\\
Thou art a lively lad;\\
Thou hast as rare a whistle\\
As ever carman had.
\end{altverse}

\begin{altverse}
Now, if my mother chide me\\
For staying here so long;\\
What if she doth, I care not.\\
For this shall be my song:\\
‘Pray, mother, be contented,\\
Break not my heart in twain;\\
Although I have been ill a-while,\\
I now am well again.’
\end{altverse}

\begin{altverse}
Now fare thee well, brave carman,\\
I wish thee well to fare,\\
For thou didst use me kindly,\\
As I can well declare:\\
Let other maids say what they will,\\
The truth of all is so,\\
The bonny Carman’s whistle\\
Shall for my money go.
\end{altverse}
\end{dcverse}

The following is the old arrangement of the tune of \textit{The Carman’s Whistle},
by Byrd, taken from Queen Elizabeth’s Virginal Book.

\musicinfo{Gracefully.}{}

\includemusic{chappellV1057.pdf}

\backskip{1}

\musictitle{Go From My Window.}

This tune is arranged both by Morley and by John Munday, in Queen Elizabeth’s 
Virginal Book; it is in \textit{A new Book of Tablature}, 1596; in Morley’s \textit{First
Booke of Consort Lessons}, 1599 and 1611; and in Robinson’s \textit{Schoole of Musick},
1603. In \textit{The Dancing Master}, from 1650 to 1686, it appears under the title of
“The new Exchange, or Durham Stable;” but the tune is there altered into
\timesig{6}{4} time, to fit it for dancing.

On the 4th March, 1587-8, John Wolfe had a license to print a ballad called
“Goe from \textit{the} windowe.” Nash, in his controversial tracts with Harvey, 1599,
mentions a song, “Go from my \textit{garden}, go.” In Beaumont and Fletcher’s
\textit{Knight of the Burning Pestle}, Old Merrythought sings—
\pagebreak
%141

\musicinfo{Slowly and smoothly.}{}

\includemusic{chappellV1058.pdf}

\backskip{1}

\settowidth{\versewidth}{Begone, begone, my juggy, my puggy,}
\begin{scverse}
Begone, begone, my juggy, my puggy,\\
Begone, my love, my dear;\\
The weather is warm,\\
’Twill do thee no harm:\\
Thou canst not be lodged here.”
\end{scverse}
In Fletcher’s \textit{Monsieur Thomas}, we find—
\settowidth{\versewidth}{Come up to my window, love, come, come, come,}
\begin{scverse}
\indentpattern{00110}
\begin{patverse}
\vleftofline{“}Come up to my window, love, come, come, come,\\
Come to my window, my dear;\\
The wind nor the rain\\
Shall trouble thee again:\\
But thou shalt be lodged here.”
\end{patverse}
\end{scverse}
It is again quoted by Fletcher in \textit{The Woman’s Prize, or the Tamer tamed}, act i.,
sc. 3; by Middleton in \textit{Blurt, Master Constable}; and by Otway in \textit{The Soldier’s
Fortune}.

It is one of the ballads that were parodied in “Ane compendious booke of
Godly and Spiritual! Songs... with sundrie of other ballates, chainged out of
prophaine Songes, for avoiding of Sinne and Harlotrie;” printed in Edinburgh
in 1590 and 1621. There are twenty-two stanzas in the Godly Song, the following
are the two first:—
\settowidth{\versewidth}{Quho [who] is at my windo, who, who?}
\begin{scverse}
\vleftofline{“}Quho [who] is at my windo, who, who?\\
Goe from my windo; goe, goe.\\
Quha calles there, so like ane strangere?\\
Go from my windo, goe.

Lord, I am here, ane wratched mortall,\\
That for thy mercie dois crie and call\\
Unto Thee, my Lord celestiall;\\
See who is at my windo, who?”
\end{scverse}

At the end of Heywood’s \textit{The Rape of Lucrece}, a song is printed beginning—
\settowidth{\versewidth}{The weather is warme, ’twill doe thee no harme,}
\begin{scverse}
\begin{altverse}
\vleftofline{“}Begone, begone, my Willie, my Billie,\\
Begone, begone, my deere;\\
The weather is warme, ’twill doe thee no harme,\\
Thou canst not be lodged here.”
\end{altverse}
\end{scverse}

which is also in \textit{Wit and Drollery, Jovial Poems}, 1661, p. 25.
\pagebreak
\origpage{}%142

In \textit{Pills to purge Melancholy}, 1707, vol. ii., 44, or 1719, vol. iv., 44, is another
version of that song, beginnings “Arise, arise, my juggy, my puggy;” but in
both editions it is printed to the tune of “Good morrow, ’tis St. Valentine’s day,”
and not to the original music.

I received the following \textit{traditional} version of “Go from my window” from a
very kind friend of former days, the late R. M. Bacon, of Norwich.\dcfootnote{\textit{}
Mr. Bacon was for many years the well-known editor,
as well as principal proprietor, of \textit{The Norwich Mercury},
and editor of \textit{The Quarterly Musical Review}. His memory
was so stored with traditional songs, learnt in boyhood,
that, having accepted a challenge at the tea-table to sing
a song upon any subject a lady would mention, I have
heard him sing verse after verse upon tea-spoons, and
other such themes, proposed as the most unlikely for
songs to have been written upon. He had learnt a number
of sea songs, principally from one old sailor, and some
were so descriptive, that, it was almost thrilling to hear
them sung by him. Seventeen years ago, these appeared
to me too irregular and declamatory to be reduced
to rhythm, but I have since greatly regretted the loss of
an opportunity that can never recur.}
 The tune is
very like that of Ophelia’s Song, “And how should I your true love know;” the
first and last strains being the same in both. The words promise an improvement
of the original, and it is to be regretted that my informant had only heard
the first stanza, which is here printed to the music.

\musicinfo{Rather slow.}{}

\includemusic{chappellV1059.pdf}

\musictitle{Dulcina.}

This tune is referred to under the names of “Dulcina;” “As at noon Dulcina
rested;” “From Oberon in fairy-land;” and “Robin Goodfellow.”

The ballad of “The merry pranks of Robin Goodfellow” (attributed to Ben
Jonson) commences with the line, “From Oberon in fairy-land;” and in the old
black-letter copies, is directed to be sung to the tune of \textit{Dulcina}. The ballad of
“As at noon Dulcina rested,” is said, upon the authority of Cayley and Ellis, to
have been written by Sir Walter Raleigh. Both are printed in Percy’s \textit{Reliques
of Ancient Poetry}, series iii., book 2.

The Milk-woman in Walton’s \textit{Angler}, says, “What song was it, I pray
you? Was it, “Come, shepherds, deck your heads,” or “As at noon Dulcina
rested,” \&c.
\pagebreak
%143

The following ballads were also sung to the tune:--

“The downfall of dancing; or the overthrow of three fiddlers and three bagpipers,”
\&c., “to the tune of \textit{Robin Goodfellow}. Copies in the Douce and Pepys
Collections.

“A delicate new ditty, composed upon the posie of a ring, being, ‘I fancy none
but thee alone:’ sent as a new year’s gift by a lover to his sweet-heart. To the
tune of \textit{Dulcina}.” Roxburghe Collection, vol. i., 80.

“The desperate damsel’s tragedy, or the faithless young man;” beginning,
“In the gallant month of June.”

“A pleasant new song, betwixt a sailor and his love. To the tune of \textit{Dulcina};”
beginning, “What doth ail my love so sadly.” In the Bagford and Roxburghe
Collections, where several more will be found.

A Cavalier’s drinking-song, by Matt. Arundel, to the tune of \textit{Robin Goodfellow},
commencing, “Some say drinking does disguise men,” is printed in \textit{Tixall
Poetry}, quarto, 1813. The last verse dates this after the Restoration.

\textit{Dulcina} was also one of the tunes to the “Psalms and Songs of Sion; turned
into the language and set to the tunes of a strange land,” 1642.

\musicinfo{Cheerfully.}{Tune of Dulcina}%TODO check this!

\includemusic{chappellV1060.pdf}

\pagebreak
