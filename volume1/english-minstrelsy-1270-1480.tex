%26
\markboth{national songs not on church scales.}{church music always in arrear.}


In the Arundel Collection (No. 292), there is a song in “a handwriting of
the time of Edward II.,” beginning—
\settowidth{\versewidth}{“Uncomly in cloystre I coure [cower] ful of care,”}
\begin{scverse}
“Uncomly in cloystre I coure [cower] ful of care,”
\end{scverse}
which is on the comparative difficulties of learning secular and church music,
but, except in the line, “Thou bitest asunder bequarre for bemol” (B natural
for B flat), there is no reference to the practice of music.

Secular music must have made considerable progress before the end of the
thirteenth century, for even Franco had spoken of a sort of composition called
“Conductus,” in which, instead of merely adding parts to a plain song, the student
was first to compose as pretty a tune as he could, and then to make descant
upon it;\dcfootnote{
“In Couductis aliter est operandum, quia qui vult
facere Conductum, primum cantum invenire debet pulchriorem
quam potest, deinde uti debet illo, ut de tenore,
faciendo discantura.”
} and he further says, that in every other case, some melody already made
is chosen, which is called the tenor, and governs the descant originating from it:
but it is different in the Conductus, where the cantus (or melody) and the descant
(or harmony) are both to be produced. This was evidently applied to secular
composition, since, about 1250, Odo, Archbishop of Rheims, speaks of Conducti et
Motuli as “jocose and scurrilous songs.”

Accidental sharps, discords and their resolutions, and even chromatic counterpoint,
are treated on by Marchetto of Padua (in his Romerium Artis Musicæ
Mensurabilis) in 1274, and the Dominican Monk, Peter Herp, mentions in
Chronicle of Frankfort, under the year 1300, that new singers, composers, and
harmonists had arisen, who used other scales or modes than those of the Church.\dcfootnote{
“Novi cantores surrexere, et componistæ, et figuristæ.
qui inceperunt alios modos assuere.” When music deviated
from the Church scales, it was called by the old
writers generally, \textit{Musica falsa}, and by Franchinus,
\textit{Musica ficta, seu colorata}, from the chromatic semitones
used in it.
}
Pope John XXII. (in his decree given at Avignon in 1322) reproves those who,
“attending to the \textit{new notes and new measures of the disciples of the new school},
would rather have their ears tickled with semibreves and minims, and \textit{such frivolous}
inventions, than hear the ancient ecclesiastical chant.” White minims, with tails,
to distinguish them from semibreves, seem first to have been used by John de
Muris, about 1330, retaining the lozenge-shaped head to the note. He also used
signs to distinguish triple from common time. These points should be borne in
mind in judging of the age of manuscripts.

It will be observed that “Sumer is icumen in” is not within the compass of
any Church scale. It extends over the octave of F, and ends by descending to the
seventh below the key note for the close, which, indeed, is one of the most common
and characteristic terminations of English airs. The dance tune which follows
next in order has the same termination, and extends over a still greater compass
of notes. I shall therefore quit the subject of Church scales, relying on the
practical refutation which a further examination of the tunes will afford. Burney
has remarked that at any given period secular music has always been at least a
century in advance of Church music. And notwithstanding the improvements
in musical \textit{notation} made by monks, the Church still adhered to her imperfect 
system, as well as to bad harmony, for \pagebreak
 centuries after better had become general.
%027
Even in the sixteenth century, modulation being still confined to the ecclesiastical
modes, precluded the use of the most agreeable keys in music. Zarlino, who
approved of the four modes added by Glareanus, speaks of himself, and a few
others, having composed in the eleventh mode, or key of C natural (which was not
one of the original eight), to which they were led \textit{by the vulgar musicians of the
streets and villages}, who generally accompanied rustic dances with tunes in this
key, and which was then called, \textit{Il modo lascivo}—The wanton key. I suppose it
acquired this name, because, like the “sweet Lydian measure” of old, the interval
from the seventh to the octave is only a semitone.

\noindent\begin{minipage}{\textwidth}
\musictitle{Dance Tune}
\vspace{-2\baselineskip}
\musicinfo{}{About 1300.}
\vspace{\baselineskip}

\includemusic{chappellV1004.pdf}
\end{minipage}

\changefontsize{1.07\defaultfontsize}
The above dance tune is taken from the Musica Antiqua by John Stafford
Smith. He transcribed it from a manuscript then in the possession of Francis
Douce, Esq. (who bequeathed the whole of his manuscripts to the Bodleian
Library), and calls it, “a dance tune of the reign of Edward II., or earlier.”
The notation of the MS. is the same \pagebreak
as in that which contains \textit{Sumer is icumen in},
%028
\markboth{english minstrelsy resumed.}{edward i.}
and I do not think it can be dated later than 1300. Dr: Crotch remarks:—
“The abundance of appoggiaturas in so ancient, a melody, and the number of bars
in the phrases, four in one and five in another—nine in each part, are its most
striking peculiarities. It is formed on an excellent design, similar to that of
several fine airs of different nations. It consists of three parts, resembling each
other excepting in the commencement of their phrases, in which they tower above
each other with increasing energy, and is altogether a curious and very favorable
specimen of the state of music at this very early period.”

The omission of the eighth bar in each phrase would make it strictly in modern
rhythm.


\headingfour{CHAPTER III.}

\headingfive{English Minstrelsy from 1270 to 1480, and the gradual extinction
of the old Minstrel.}

Edward the First, according to the Chronicle of Walter Hemmingford, about the
year 1271, a short time before he ascended the throne, took his harper with him
to the Holy Land, who must have been a close and constant attendant on his
master, for when Edward was wounded at Ptolemais, the harper (Citharæda
suus), hearing the struggle, rushed into the royal apartment, and, striking the
assassin on the head with a tripod or trestle, beat out his brains.

“That Edward ordered a massacre of the Welsh bards,” says Sharon Turner,
“seems rather a vindictive tradition of an irritated nation than an historical fact.
The destruction of the independent sovereignties of Wales abolished the patronage
of the bards, and in the cessation of internal warfare, and of external ravages,
they lost their favorite subjects, and most familiar imagery. They declined
because they were no longer encouraged.” The Hon. Daines Barrington could
find no instances of severity against the Welsh in the laws, \&c. of this monarch,\footnote{
See his observations on the statutes, 4to. 4th Ed.
}
and that they were not extirpated is proved by the severe law which we find in
the Statute Book, 4 Henry IV. (1402), c. 27, passed against them during the
resentment occasioned by the outrages committed under Owen Glendour. In that
act they are described as Rymours and Ministralx, proving that our ancestors
could not distinguish between them and our own minstrels.

In May, 1290, was celebrated the marriage of Queen Eleanor’s daughter Joan,
surnamed of Acre, to the Earl of Gloucester, and in the following July, that of
Margaret, her fifth daughter, to John, son of the Duke of Brabant. Both ceremonies
were conducted with much splendour, and a multitude of minstrels flocked
from all parts to Westminster: to the first came King Grey of England, King
Caupenny from Scotland, and Poveret, the minstrel of the Mareschal of Champagne.
The nuptials of Margaret, however, seem to have eclipsed those of her sister. 
Walter de Storton, the king’s harper, \pagebreak 
distributed a hundred pounds, the gift of
%029
the bridegroom, among 426 minstrels, as well English as others.\dcfootnote{
Pages lxix. and lxx. Introduction to Manners and
Household Expenses of England in the 13th and 15th
centuries, illustrated by original records. 4to. London.
Printed for the Roxburghe Club, 1841, and quoted from
Wardrobe Book, 18 Edward I. Rot. Miscell. in Turr.
Lond. No. 56.
} In 1291, in the
accounts of the executors of Queen Eleanor, there is an entry of a payment of
39\textit{s}., for a cup purchased to be given to one of the king’s minstrels.

The highly valuable roll, preserved among the records in the custody of the
Queen’s Remembrancer, which has been printed for the Roxburghe Club, marks
the gradations of rank among the minstrels, and the corresponding rewards
bestowed upon them. It contains the names of those who attended the \textit{cour
plenière} held by King Edward at the Feast of Whitsuntide, 1306, at Westminster,
and also at the New Temple, London; because “the royal palace,
although large, was nevertheless small for the crowd of comers.” Edward then
conferred the honor of knighthood upon his son, Prince Edward, and a great
number of the young nobility and military tenants of the crown, who were summoned
to receive it, preparatory to the King’s expedition to Scotland to avenge
the murder of John Comyn, and the revolt of the Scotch.

On this occasion there were six kings of the minstrels, five of whom, viz.,
Le Roy de Champaigne, Le Roy Capenny, Le Roy Boisescue, Le Roy Marchis,
and Le Roy Robert, received each five marks, or 3\textit{l}. 6\textit{s}. 8\textit{d}., the mark being
13\textit{s}. 4\textit{d}. It is calculated that a shilling in those days was equivalent to fifteen
shillings of the present time; according to which computation, they received 50\textit{l}.
each. The sixth, Le Roy Druet, received only 2\textit{l}. The list of money \textit{given} to
minstrels is principally in Latin; but that of \textit{payments} made to them being in
Norman French, it is difficult to distinguish English minstrels from others. Le
Roy de Champaigne was probably “Poveret, the minstrel of the Mareschal of
Champagne,” of 1290, Le Roy Capenny, “King Caupenny from Scotland,” and
Le Roy Robert, whom we know to have been the English king of the minstrels
by other payments made to him by the crown (see Anstis’ Register of the Order
of the Garter, vol. ii. p. 303), was probably the “King Grey of England” of
the former date. Among the names we find, Northfolke, Carletone, Ricard de
Haleford, Adam de Werintone (Warrington?), Adam de Grimmeshawe, Merlin,
Lambyn Clay, Fairfax, Hanecocke de Blithe, Richard Wheatacre, \&c. The
harpers are generally mentioned only by their Christian names, as Laurence,
Mathew, Richard, John, Robert, and Geoffrey, but there are also Richard de
Quitacre, Richard de Leylonde, William de Grimesar, William de Duffelde, John
de Trenham, \&c., as well as Adekyn, harper to the Prince, who was probably
a Welsh bard. In these lists only the principal minstrels are named, the remaining
sum being divided, by the kings and few others, among the \textit{menestraus de la
commune}. Harpers are in the majority where the particular branch of minstrelsy
is specified. Some minstrels are locally described, as Robert “de Colecestria,”
John “de Salopia,” and Robert “de Scardeburghe;” others are distinguished
as the harpers of the Bishop of Durham, Abbot of Abyngdon, Earls of Warrenne,
Gloucester, \&c.; one is Guillaume sans manière; another, Reginald le menteur;
a third is called Makejoye; and a fourth, Perle in the eghe.
\pagebreak

%030
\markboth{english minstrelsy.}{edward ii.}
\changefontsize{1.04\defaultfontsize}
The total sum expended was about 200\textit{l}., which according to the preceding
estimate would be equal to about 3,000\textit{l}. of our money.

The minstrels seem to have been in many respects upon the same footing as the
heralds; and the King of the Heralds, like the King at Arms, was, both here and
on the Continent, an usual officer in the courts of princes. Heralds seem even to
have been included with minstrels in the preceding account, for Carletone, who
occupies a fair position among them, receiving 1\textit{l}. as a payment, and 5\textit{s}. as a
gratuity, is in the latter case described as Carleton “Haralde.”

In the reign of Edward II., besides other grants to “King Robert,” before
mentioned, there is one in the sixteenth year of his reign to William de Morlee,
“The king’s minstrel, styled \textit{Roy de North}” of houses that had belonged to
John le Boteler, called Roy Brunhaud. So, among heralds, \textit{Norroy} was usually
styled \textit{Roy d'Armes de North} (Anstis. ii. 300), and the Kings at Arms in general
were originally called Reges Heraldorum, as these were Reges Minstrallorum.\dcfootnote{
Heralds and minstrels seem to have been on nearly
the same footing abroad. For instance, Froissart tells us
“The same day th’ Erle of Foix gave to \textit{Heraudes} and
\textit{Minstrelles} the somme of fyve hundred frankes: and
gave to the Duke of Tourayn’s Minstrelles gowns of
Cloth of Gold, furred with Ermyns, valued at two hundred
franks.”—\textit{Chronicle Ed}. 1525, book 3, ch. 31.
}
—\textit{Percy's Essay}.

The proverbially lengthy pedigrees of the Welsh were registered by their bards,
who were also heralds.\dcfootnote{
“The Welshman’s pedigree was his title-deed, by
which he claimed his birthright in the country. Every
one was obliged to shew his descent through nine generations,
in order to be acknowledged a free native, and by
which right he claimed his portion of land in the community. 
Among a people, where surnames were not in
use, and where the right of property depended on descent,
an attention to pedigree was indispensable. Hence arose
the second order of Bards, who were the \textit{Arwyddvierdd}, or
Bard-Heralds, whose duty it was to register arms and
pedigrees, as well as undertake the embassies of state.
The \textit{Arwyddvardd}, in early Cambrian history, was an
officer of national appointment, who, at a later period,
was succeeded by the \textit{Prydydd}, or Poet. One of these was
to attend at the birth, marriage, and death of any man of
high descent, and to enter the facts in his genealogy.
The \textit{Marwnad}, or Elegy, composed at the decease of such
a person, was required to contain truly and at length his
genealogy and descent; and to commemorate the survivor,
wife or husband, with his or her descent and progeny.
The particulars were registered in the books of the
\textit{Arwyddvardd}, and a true copy therefrom delivered to the
heir, to be placed among the authentic documents of the
family. The bard’s fee, or recompense, was a stipend
out of every plough land in the district; and he made a
triennial Bardic circuit to correct and arrange genealogical
entries.”—\textit{Extruded from Meyrick's Introduction to his
edition of Lewis Durm's Heraldic Visitations of Wales
2 vols. 4to. Llandovery}. 1846.
}

In the reign of Edward II., \ad 1309, at the feast of the installation of Ralph,
Abbot of St. Augustin’s, at Canterbury, seventy shillings was expended on
minstrels, who accompanied their songs with the harp.— \textit{Warton}, vol. i., p. 89.

In this reign such extensive privileges were claimed by these men, and by dissolute
persons assuming their character, that it became a matter of public grievance, 
and a royal decree was issued in 1315 to put an end to it, of which the
following is an extract:—

“Edward by the grace of God, \&c. to sheriffes, \&c. greetyng, Forasmuch as\ldots  many
idle persons, under colour of Mynstrelsie, and going in messages, and other faigned
business, have ben and yet be receaved in other mens houses to meate and drynke, and
be not therwith contented yf they be not largely consydered with gyftes of the lordes
of the houses: \&c\ldots We wyllyng to restrayne suche outrageous enterprises and idleness,
\&c. have ordeyned\ldots that to the houses of prelates, earles, and barons, none
resort to meate and drynke, unlesse he be a Mynstrel, and of these Minstrels that there
come none except it be three or four \textsc{Minstrels of honour} at the most in one day,
unlesse he be desired of the lorde of the house. \pagebreak 
And to the houses of meaner men
%031
that none come unlesse he be desired, and that such as shall come so, holde themselves
contented with meate and drynke, and with such curtesie as the maister of the house
wyl shewe unto them of his owne good wyll, without their askyng of any thyng.
And yf any one do agaynst this Ordinaunce, at the firste tyme he to lose his \textit{Minstrelsie},
and at the second tyme to forsweare his craft, and never to be receaved for
a Minstrel in any house\ldots .Geven at Langley the vi. day of August, in the ix yere of
our reigne.”—\textit{Hearne’s Append. ad Leland Collect}., vol. vi., p. 36.

Stow, in his Survey of London, in an estimate of the annual expenses of
the Earl of Lancaster about this time, mentions a large disbursement for the
liveries of the minstrels. That they received vast quantities of money and costly
habiliments from the nobles, we learn from many authorities; and in a poem on
the times of Edward II., knights are recommended to adhere to their proper
costume lest they be mistaken for minstrels.

\settowidth{\versewidth}{“Kny[gh]tes schuld weare clothes}
\indentpattern{010101012003}
\begin{dcverse}
\begin{patverse}
\vleftofline{“}Kny[gh]tes schuld weare clothes\\
I-schape in dewe manere,\\
As his order wo[u]ld aske,\\
As wel as schuld a frere [friar]:\\
Now thei beth [are] disgysed,\\
So diverselych i-digt [bedight],\\
That no man may knowe\\
A mynstrel from a knyg[h]t\\
Well ny:\\
So is mekenes[s] falt adown\\
And pride aryse an hye."\\
\textit{Percy Soc., No}. 82, \textit{p}. 23.
\end{patverse}
\end{dcverse}

That minstrels were usually known by their dress, is shown by the following
anecdote, which is related by Stowe:—“When Edward II. this year (1316)
solemnized the feast of Pentecost, and sat at table in the great hall of Westminster,
attended by the peers of the realm, a certain woman, \textit{dressed in the habit
of a Minstrel, riding on a great horse, trapped in the Minstrel fashion}, entered the
hall, and going round the several tables, acting the part of a Minstrel, at length
mounted the steps to the royal table, on which she deposited a letter. Having
done this, she turned her horse, and, saluting all the company, she departed.”
The subject of this letter was a remonstrance to the king on the favors heaped
by him on his minions to the neglect of his faithful servants. The door-keepers
being called, and threatened for admitting such a woman, readily replied, “that
it never was the custom of the king’s palace to deny admission to Minstrels,
especially on such high solemnities and feast days.”

On the capital of a column in Beverley Minster, is the inscription, “Thys
pillor made the meynstyrls.” Five men are thereon represented, four in short
coats, reaching to the knee, and one with an overcoat, all having chains round
their necks and tolerably large purses. The building is assigned to the reign of
Henry VI., 1422 to 1460, when minstrelsy had greatly declined, and it cannot
therefore be considered as representing minstrels in the height of their prosperity.
They are probably only instrumental performers (with the exception, perhaps, of
the lute player); but as one holds a pipe and tabor, used only for rustic dances,
another a crowd or treble viol, a third what appears to be a bass flute, and a
fourth either a treble flute or perhaps that kind of hautboy called a wayght, or
wait, and there is no harper among them—I do not suppose any to have been of
that class called minstrels of honour, \pagebreak 
who rode on horseback, with their servants
%32
to attend them, and who could enter freely into a king’s palace. Such distinctions
among minstrels are frequently drawn in the old romances. For instance, in the
romance of Launfel we are told, “They had menstralles of moche honours,” and
also that they had “Fydelers, sytolyrs (citolers), and trompoteres.” It is not,
however, surprising that they should be rich enough to build a column of a
Minster, considering the excessive devotion to, and encouragement of, music which
characterised the English in that and the two following centuries.

No poets of any country make such frequent and enthusiastic mention of minstrelsy
as the English. There is scarcely an old poem but abounds with the
praises of music. Adam Davy, or Davie, of Stratford-le-Bow, near London,
flourished about 1312. In his Life of Alexander, we have several passages like
this:—
\settowidth{\versewidth}{The mynstrall synge, the jogelour carpe” (recite).}
\begin{scverse}
\vleftofline{“}Mer[r]y it is in halle to he[a]re the harpe,\\
The mynstrall synge, the jogelour carpe” (recite).
\end{scverse}
And again,—
\settowidth{\versewidth}{“Mery is the twynkelyng of the harpour.”}
\begin{scverse}
“Mery is the twynkelyng of the harpour.”
\end{scverse}
The fondness of even the most illiterate, to hear tales and rhymes; is much
dwelt on by Robert de Brunne, or Robert Mannyng, “the first of our vernacular
poets who is at all readable now.” All rhymes were then sung with accompaniment, 
and generally to the harp. So in 1338, when Adam de Orleton, bishop of
Winchester, visited his Cathedral Priory of St. Swithin, in that city, a minstrel
named Herbert was introduced, who sang the \textit{Song of Colbrond}, a Danish Giant,
and the tale of \textit{Queen Emma delivered from the plough-shares}, or trial by fire, in
the hall of the Prior. A similar festival was held in this Priory in 1374, when
similar gestes or tales were sung. Chaucer’s Troilus and Cresseide, though almost
as long as the Æneid, was to be “redde, or else songe,” and Warton has printed
a portion of the Life of St. Swithin from a manuscript, with points and accents
inserted, both over the words and dividing the line, evidently for the purposes of
singing or recitation (\textit{History of English Poetry}, vol. i., p. 15. 1840). We have
probably by far more tunes that are fitted for the recitation of such lengthy stories
than exist in any other country.

In the year 1362, an Act of Parliament passed, that “all pleas in the court
of the king, or of any other lord, shall be pleaded and adjudged in the English
tongue” (stat. 36 Edw. III., cap. 15); and the reason, which is recited in the
preamble, was, that the French tongue was so unknown in England that the
parties to the law-suits had no knowledge or understanding of what was said for
or against them, because the counsel spoke French. This was the era of Chaucer,
and of the author of Pierce Plowman—two poets whose language is as different as
if they had been born a century apart. Longland, instead of availing himself of the
rising and rapid improvements of the English language, prefers and adopts the style
of the Anglo-Saxon poets, even prefering their perpetual alliteration to rhyme.
His subject—a satire on the vices of the age, but particularly on the corruptions
of the clergy and the absurdities of superstition—does not lead him to say much
of music, but he speaks of ignorance of the art as a just subject of reproach.
\settowidth{\versewidth}{“They kennen [know] no more mynstralcy, ne musik, men to gladde,}
\begin{scverse}
\vleftofline{“}They kennen [know] no more mynstralcy, ne musik, men to gladde,\\
Than Mundy the muller [miller], of \textit{multa fecit Deus}!’’
\end{scverse}
\pagebreak
%033
\changefontsize{1.01\defaultfontsize}
\markright{pierce plowman---chaucer.}
He says, however, of himself, in allusion to the minstrels:—

\settowidth{\versewidth}{Nother sailen [leap or dance], ne sautrien, ne singe with the giterne.”}
\begin{scverse}\vleftofline{“}Ich can nat tabre, ne trompe, ne telle faire gestes,\\
Ne fithelyn, at fe[a]stes, ne harpen:\\
Japen ne jagelyn, ne gentilliche pipe;\\
Nother sailen [leap or dance], ne sautrien, ne singe with the giterne.”
\end{scverse}
He also describes his Friar as much better acquainted with the “\textit{Rimes of
Robinhode} and of \textit{Randal, erle of Chester},” than with his Paternoster.

Chaucer, throughout his works, never loses an opportunity of describing or
alluding to the general use of music, and of bestowing it as an accomplishment
upon the pilgrims, heroes, and heroines of his several tales or poems, whenever
propriety admits. We may learn as much from Chaucer of the music of his day,
and of the estimation in which the art was then held in England, as if a treatise
had been written on the subject.

Firstly, from the Canterbury Tales, in his description of the Squire (line 91
to~96), he says:—
\begin{scverse}\vleftofline{“}\textit{Syngynge he mas, or flowtynge} [fluting] \textit{al the day};\\
He was as fresh as is the moneth of May:\\
Short was his gonne, with sleevès long and wyde;\\
Well cowde he sitte on hors, and faire ryde.\\
\textit{He cowde songes wel make and endite,}\\
Juste (fence) and eke daunce, and wel p[o]urtray and write.”
\end{scverse}

Of the Nun, a Prioress (line 122 to 126), he says:—
\begin{scverse}\vleftofline{“}\textit{Ful wel sche sang the servise devyne},\\
\textit{Entuned in hire nose ful seemyly};\\
And Frensch sche spak ful faire and fetysly [neatly],\\
Aftur the schole of Stratford attè Bowe,\\
For Frensch of Parys was to hire unknowe” [unknown].
\end{scverse}

The Monk, a jolly fellow, and great sportsman, seems to have had a passion for
no music but that of hounds, and the bells on his horse’s bridle (line 169 to~171):
\begin{scverse}\vleftofline{“}And whan he rood [rode], men might his bridel heere\\
Gyngle in a whistlyng wynd so cleere,\\
And eke as lowde as doth the chapel belle.”
\end{scverse}

Of his Mendicant Friar, whose study was only to please (lines 235-270),
he says:
\begin{scverse}\vleftofline{“}And certayn he hadde a mery note;
\textit{Wel couthe he synge and playe on a rote }[hurdy-gurdy]\ldots \\
Somewhat he lipsede [lisped] for wantounesse,\\
To make his Englissch swete upon his tunge;\\
And \textit{in his harpyng, whan that he had sunge},\\
His eyghen [eyes] twynkeled in his he[a]d aright,\\
As don the sterrès [do the stars] in the frosty night.”
\end{scverse}

Of the Miller (line 564 to 568), he says:—
\settowidth{\versewidth}{Was never trompe [trumpet] of half so gre[a]t a soun” (sound).}
\begin{scverse}\vleftofline{“}Wel cowde he ste[a]le corn, and tollen thries [take toll thrice];\\
And yet he had a thomhe of gold,\footnote{
Tyrwhitt says there is an old proverb—“Every \textit{honest}
miller has a thumb of gold.” Perhaps it means that
nevertheless he was as honest as his brethren. There are
many early songs on thievish millers and bakers.
} pardé,\\
A whight cote and blewe hood we[a]red he;\\
\end{scverse}
%34
\pagebreak
\begin{scverse}
\textit{A baggepipe cowde he blowe and sowne} [sound],\\
And therewithal he brought us out of towne.”\dcfootnote{
A curious reason for the use of the Bagpipe in Pilgrimages
will he found in State Trials—Trial of William
Thorpe. Henry IV., an. 8, shortly after Chaucer’s death.
“I say to thee that it is right well done, that Pilgremys
have with them both Syngers, and also Pipers, that whan
one of them, that goeth bar[e]fo[o]te, striketh his too upon
a stone, and hurteth hym sore, and maketh hym to blede;
it is well done that he or his fel[l]ow begyn than a Songe,
or else take out of his bosome a \textit{Baggepype} for to drive
away with soche myrthe the hurte of his felow.”
}
\end{scverse}
Of the Pardoner (line 674 to 676):—
\begin{scverse}\vleftofline{“}\textit{Ful lowde he sang, ‘Come hider, love, to me}.’\\
This Sompnour bar[e] to him a \textit{stif burdoun},\dcfootnote{
This Sompnour (Sumner or Summoner to the Ecclesiastical
Courts, now called Apparitor) supported him by
singing the \textit{burden}, or \textit{bass}, to his song in a deep loud
voice. \textit{Bourdon} is the French for \textit{Drone}; and \textit{Foot},
\textit{Under-song}, and \textit{Burden} mean the same thing, although
Burden was afterwards used in the sense of Ditty, or
any line often recurring in a song, as will be seen here-
after.
}\\
Was never trompe [trumpet] of half so gre[a]t a soun” (sound).
\end{scverse}
Of the poor scholar, Nicholas (line 3213 to 3219):—
\begin{scverse}\vleftofline{“}And al above ther lay a gay \textit{sawtrye} [psaltry],\\
On which he made, a-nightes, melodye\\
So swetely, that al the chambur rang:\\
And \textit{Angelus ad Virginem} he sang.\\
And after that he sang \textit{The Kynge’s note};\\
Ful often blessed was his mery throte.”
\end{scverse}
Of the Carpenter’s Wife (lines 3257 and 8):--
\begin{scverse}\vleftofline{“}But of her song, it was as lowde and yerne [brisk]\\
As eny swalwe [swallow] chiteryng on a berne” [barn].
\end{scverse}
Of the Parish Clerk, Absolon (lines 3328 to 3335):—
\begin{scverse}\vleftofline{“}In twenty manners he coude skip and daunce,\\
After the schole of Oxenfordè tho,\\
And with his leggès casten to and fro;\\
\textit{And pleyen songes on a small Rubible}\dcfootnote{
Ribible (the diminutive of Ribibe or Rebec) is a small
fiddle with three strings.
} [Rebec],\\
\textit{Ther-to he sang som tyme a lowde quynyble};\dcfootnote{
To sing a “quinible” means to descant by singing
fifths on a plain-song, and to sing a “quatrible” to descant
by fourths. The latter term is used by Cornish in
his Treatise between Trowthe and Enformacion. 1528.
}\\
\textit{And as wel coude he pleye on a giterne}:\\
In al the toun nas [nor was] brewhous ne taverne\\
That he ne visited with his solas” [solace].
\end{scverse}
He serenades the Carpenter’s Wife, and we have part of his song (lines 3352—64):%
\begin{scverse}\vleftofline{“}The moone at night ful cleer and brightè schoon,\\
And Absolon his giterne hath i-take,\\
For paramours he seyde he wold awake\ldots \\
He syngeth in hys voys gentil and smal—\\
Now, deere lady, if thi wille be,\\
I pray you that ye wol rewe [have compassion] on me.’\\
Full wel acordyng to his gyternyng,\\
This carpenter awook, and herde him syng.”
\end{scverse}
Of the Apprentice in the Cook’s Tale, who plays both on the ribible and gitterne:
\begin{scverse}\vleftofline{“}At every brideale wold he synge and hoppe;\\
He loved bet [better] the taverne than the schoppe.”
\end{scverse}
\pagebreak
%35
\markright{notices of music by chaucer.}

\settowidth{\versewidth}{\ldots . “some, for they can synge and daunce,}

The Wife of Bath says (lines 5481 and 2, and 6039 and 40), that wives were
chosen—
%\vspace{-\baselineskip}
\begin{scverse}\ldots . “	some, for they \textit{can synge and daunce},\\
And some for gentilesse or daliaunce\ldots \\
How couthe I dannce to an harpe smale,\\
And synge y-wys as eny nightyngale.”
\end{scverse}
I shall conclude Chaucer’s inimitable descriptions of character with that of his
Oxford Clerk, who was so fond of books and study, that he loved Aristotle better
%\vspace{-\baselineskip}
\begin{scverse}\vleftofline{“}Than robès riche, or fidel or sautrie\ldots \\
Souning in moral virtue was his speech,\\
And gladly would he lerne and gladly teche.”
\end{scverse}
We learn from the preceding quotations, that country squires in the fourteenth
century could pass the day in singing, or playing the flute, and that some could
“Songès well make and indite:” that the most attractive accomplishment in
a young lady was to be able to sing well, and that it afforded the best chance of
her obtaining an eligible husband; also that the cultivation of music extended
to every class. The Miller, of whose education Pierce Plowman speaks so slightingly,
could play upon the bagpipe; and the apprentice both on the ribible and
gittern. The musical instruments that have been named are the harp, psaltry,
fiddle, bagpipe, flute, trumpet, rote, rebec, and gittern. There remain the lute,
organ, shalm (or shawm), and citole, the hautboy (or wayte), the horn, and
shepherd’s pipe, and the catalogue will be nearly complete, for the cittern or
cithren differed chiefly from the gittern, in being strung with wire instead of gut,
or other material. The sackbut was a bass trumpet with a slide,\dcfootnote{
“As he that plaies upon a Sagbut, by pulling it up
and down alters his tones and tunes.”—\textit{Burton’s Anatomy
of Melancholy}, 8vo. Edit, of 1800, p. 379.
} like the modern
trombone; and the dulcimer differed chiefly from the psaltry in the wires being
struck, instead of being twitted by a plectrum, or quill, and therefore requiring
both hands to perform on it.

In the commencement of the Pardoner’s Tale he mentions lutes, harps, and
gitterns for dancing, as well as singers with harps; in the Knight’s Tale he represents
Venus with a citole in her right hand, and the organ is alluded to both in
the History of St. Cecilia, and in the tale of the Cock and the Fox.

In the House of Fame (Urry’s Edit., line 127 to 136), he says:
\settowidth{\versewidth}{How that he dorstè not his sorwe [sorrow] telle,}
\begin{scverse}
\vleftofline{“}That madin loude Minstralsies\\
In Cornmuse [bagpipe] and eke in \textit{Shalmies},\footnotemark\\
\end{scverse}
\footnotetext{\scriptsize
A very early 1414 drawing of the Shalm, or Shawm, is in
one of the illustrations to a copy of Froissart, in the Brit.
Mus.—\textit{Royal} MSS. 18, E. Another in Commenius’
Visible World, translated by Hoole, 1650, (he translates
the Latin word \textit{gingras}, shawm,) from which it is copied
into Cavendish’s Life of Wolsey, edited by Singer, vol. i.
p. 114., Ed. 1825. The modern clarionet is an improvement
upon the shawm, which was played with a reed,
like the wayte, or hautboy, but being a bass instrument,
with about the compass of an octave, had probably more
the tone of a bassoon. It was used on occasions of state.
“What \textit{stately} music have you? You have shawms?
Ralph plays a stately part, and he must needs have
shawms.”—\textit{Knight of the Burning Pestle}. Drayton speaks
of it as shrill-toned: “E’en from the \textit{shrillest shawm}, unto
the cornamute."—\textit{Polyolbion}, vol. iv., p. 376. I conceive
the shrillness to have arisen from over-blowing, or else
the following quotation will appear contradictory:—
\settowidth{\versewidth}{It mountithe not to hye, but \textit{kepithe rule and space}.}
\begin{fnverse}
“A \textit{Shawme} maketh a \textit{swete} sounde, \\
\hspace{\vgap} for he \textit{tunythe the basse},\\
It mountithe not to hye, but \textit{kepithe rule and space}.\\
Yet \textit{yf it be blowne withe to vehement a wynde,}\\
It \textit{makithe it to mysgoverne out of his kynde}.”
\end{fnverse}
This is one of the “proverbis” that were written about
the time of Henry VII., on the walls of the Manor House
at Leckingfield, near Beverley, Yorkshire, anciently belonging
to the Percys, Earls of Northumberland, but uow
destroyed. There were many others relating to music,
and musical instruments (harp, lute, recorder, claricorde,
clarysymballis, virgynalls, clarion, organ, singing, and
musical notation,) and the inscribing them on the walls
adds another to the numberless proofs of the estimation
in which the art was held. A manuscript copy of them
is preserved in Bib, Reg. 18. D. 11. Brit. Mus.
}
\pagebreak
%36

\begin{scverse}
And in many an othir pipe,\\
That craftely began to pipe\\
Bothe in \textit{Douced} and eke in \textit{Rede},\footnotemark\\
That bin at feastes with the brede [bread]:\\
And many a \textit{Floite} and litlyng \textit{Horne}\\
And \textit{Pipes made of grenè corne}.\\
As have these little Herdègroomes\\
That kepin Beastes [keep oxen] in the broomes.”
\end{scverse}

\footnotetext{\scriptsize
Tyrwhitt thinks \textit{Doucete} an Instrument, and quotes
Lydgate—
\settowidth{\versewidth}{“Ther were trumpes and trumpetes,}
\begin{fnverse}
\vleftofline{“}Ther were trumpes and trumpetes,\\
Lowde shall[m]ys and doucetes.”
\end{fnverse}
but it seems to me only to mean soft pipes in opposition
to loud shalms. By the distinction Chaucer draws, “both
in douced and in reed” (the shalm being played on by
a reed), I infer by “douced” that flutes are intended; the
tone of which, especially the large flute, is extremely soft.
I had a collection of English flutes, of which one was
nearly a yard and a half long. All had mouth-pieces like
the flageolet, and were blown in the same manner; the
tone very pleasing, but less powerful and brilliant than
the modern or “German” flute.
}

As to the songs of his time, see the Frankeleyne’s Tale (line 11,254 to 60):—
\begin{scverse}
\vleftofline{“}He was dispeired, nothing dorst he seye\\
Sauf [save] in his songès somewhat wolde he wreye [betray]\\
His woo, as in a general compleyning;\\
He said he loved, and was beloved nothing.\\
Of suche matier made he many \textit{Layes},\\
\textit{Songes, Compleyntes, Roundelets, Virelayes}:\\
How that he dorstè not his sorwe [sorrow] telle,\\
But languisheth as doth a fuyr in helle.”
\end{scverse}
and he speaks elsewhere of \textit{Ditees, Rondils, Balades}, \&c.

The following passages relate to minstrelsy, and to the manner of playing the
harp, pointing and performing with the nails, as the Spaniards do now with the
guitar. The first is from the House of Fame (Urry, line 105 to 112):—
\begin{scverse}
\ldots “Stoden\ldots the castell all aboutin\\
Of all manir of Minstralis\\
And gestours that tellen tales\\
Both of wepyng and of game,\\
And all that ’longeth unto fame;\\
There herde I playin on an \textit{Harpe}\\
\textit{That ysounid bothe well and sharpe}”
\end{scverse}
and from Troylus, lib. 2, 1030:—
\begin{scverse}
“For though that the best barper upon live\\
Would on the bestè sounid jolly harpe\\
That evir was, with all his fingers five\\
Touch aie o (one) string, or aie o warble harpe,\\
\textit{Were his nailes poincted nevir so sharpe}\\
It shoulde makin every wight to[o] dull\\
To heare [h]is Glee, and of his strokes ful.”
\end{scverse}

Even the musical gamut is mentioned by Chaucer. In the supplementary tale
he makes the host give “an hid[e]ouse cry in ge-sol-re-ut the haut,” and there is
scarcely a subject connected with the art as practised in his day, that may not be
illustrated by quotation from his works;
\begin{scverse}
“For, gif he have nought sayd hem, leeve [dear] brother,\\
In o bo[o]k, he hath seyd hem in another.”
\end{scverse}
%\normalsize

%037
\pagebreak
\markright{gower.---richard ii.}

I shall conclude these numerous extracts with one of the song of nature, from
the Knighte’s Tale, (line 1493 to 98):—

\indentpattern{The silver dropès, hongyng on the leeves.”}
\begin{scverse}
\vleftofline{“}The busy larkè, messager of daye,\\
Salueth in hire song the morwe [morning] gray;\\
And fyry Phebus ryseth up so bright,\\
That al the orient laugheth of the light,\\
And with his stremès dryeth in the greves [groves]\\
The silver dropès, hongyng on the leeves.”
\end{scverse}
Having quoted so largely from Chaucer, whose portraiture of character and
persons has never been excelled, it will be unnecessary to refer to his contemporary, Gower, further than to say that in his \textit{Confessio Amantis}, Venus greets
Chaucer as her disciple and poet, who had filled the land in his youth with
dittees and “songès glade,” which he had made for her sake; and Gower says of
himself:—
\begin{scverse}
\vleftofline{“}And also I have ofte assaide\\
Roundel, Balades, and Virelaie\\
For her on whom myn hert laie.”
\end{scverse}
But about the same time, in the Burlesque Romance, The T[o]urnament of
Tottenham (written in ridicule of chivalry), we find a notice of songs in six parts
which demands attention. In the last verse:—
\begin{scverse}
\vleftofline{“}Mekyl mirth was them among;\\
In every corner of the hous\\
Was melody delycyous\\
For to he[a]re precyus\\
\vin\vin Of six menys song.”
\end{scverse}

It has been supposed that this is an allusion to \textit{Sumer is icumen in}, which
requires six performers, but in all probability there were many such songs,
although but one of so early a date has descended to us. We find in the Statutes
of New College, Oxford (which was founded about 1380), that William of
Wykeham ordered his scholars to recreate themselves on festival days with songs
in the hall, both after dinner and supper; and as part-music was then in common
use, it is reasonable to suppose that the founder intended the students thereby to
combine improvement and recreation, instead of each singing a different song.

In the fourth year of king Richard II. (1381), John of Gaunt erected at
Tutbury, in Staffordshire, a \textit{Court of Minstrels} similar to that annually kept at
Chester; and which, like a court-leet, or court-baron, had a legal jurisdiction,
with full power to receive suit and service from the men of this profession within
five neighbouring counties, to determine their controversies and enact laws; also
to apprehend and arrest such of them as should refuse to appear at the said court,
annually held on the 16th of August. For this they had a charter, by which
they were empowered to appoint a King of the Minstrels, with four officers to
preside over them. They were every year elected with great ceremony; the
whole form of which, as observed in 1680, is described by Dr. Plot in his History
of Staffordshire. That the barbarous diversion of bull-running was no part of the
original institution, is fully proved by the Rev. Dr. Pegge, in Archæologia, vol. ii.,
No. xiii., p. 86. The bull-running tune, however, is still popular in Staffordshire.

%038
\pagebreak
\markboth{english minstrelsy.}{henry v.}

Du Fresne in his Glossary (art. Ministrelli), speaking of the King of the
Minstrels, says, “His office and power are defined in a French charter of
Henry~IV., king of England, in the Monasticon Anglicanum, vol. i., p. 355;”
but though I have searched through Dugdale’s Monasticon, I find no such
charter.


In 1402, we find the before-mentioned statute against the Welsh bards,%\linebreak
(4 Henry~IV., c.~27).\footnote{
It runs in these terms: “Item, pour eschuir plusieurs
diseases et mischiefs qont advenuz devaunt ces heures en
la terre de Gales par plusieurs Westours Rymours,
Minstralx et autres Vacabondes, ordeignez est, et
establiz, que nul Westour, Rymour Minstral, ne Vacabond
soit aucunemeut sustenuz en la terre de Gales pur
faire kymorthas ou coillage sur la commune poeple
iliocques.”
} As they had excited their countrymen to rebellion
against the English government, it is not to he wondered (says Percy) that the
Act is conceived in terms of the utmost indignation and contempt against this
class of men, who are described as Rymours, Ministralx, which are apparently
here used as only synonymous terms to express the Welsh bards, with the usual
exuberance of our Acts of Parliament; for if their Ministralx had been mere
musicians, they would not have required the vigilance of the English legislature
to suppress them. It was their songs, exciting their countrymen to insurrection,
which produced “les diseases et mischiefs en la terre de Gales.”

At the coronation of Henry V., which took place in Westminster Hall (1413),
we are told by Thomas de Elmham, that “the number of harpers was exceedingly
great; and that the sweet strings of their harps soothed the souls of the guests
by their soft melody.” He also speaks of the dulcet sounds of the united
music of other instruments, in which no discord interrupted the harmony,
as “inviting the royal banqueters to the full enjoyment of the festival.”—
(Vit. et. Gest. Henr. V., c. 12, p. 23.) Minstrelsy seems still to have
flourished in England, although it had declined so greatly abroad; the Provençals
had ceased writing during the preceding century. When Henry was preparing
for his great voyage to France in 1415, an express order was given for his
minstrels to attend him.—(Rymer, ix.,~255.) Monstrelet speaks of the English
camp resounding with the national music (170) the day preceding the battle of
Agincourt, but this must have been before the king “gave the order for silence,
which was afterwards strictly observed.”

When he entered the City of London in triumph after the battle, the gates and
streets were hung-with tapestry representing the histories of ancient heroes; and
boys with pleasing voices were placed in artificial turrets, singing verses in his
praise. But Henry ordered this part of the pageantry to cease, and commanded
that for the future no “ditties should be made and sung by Minstrels\footnote{
Hollinshed, quoting from Thomas de Elmham, whose
words are, “Quod cantus de suo triumpho fieri, sen per
\textit{Citharistas} vel alios quoscunque cantari penitus prohibebat.”
It will be observed that Hollinshed translates
Citharistas (literally harpers) minstrels.
} or others,”
in praise of the recent victory; “for that he would whollie have the praise
and thankes altogether given to God.”

Nevertheless, among many others, a minstrel-piece soon appeared on the
\textit{Seyge of Harflett} (Harfleur), and the \textit{Battayle of Agynkourte}, “evidently,” says
Warton, “adapted to the harp,” and of \pagebreak
 which he has printed some portions.
%39
(Hist. Eng. Poet., vol. ii. p. 257.) Also the following song, which Percy has
printed in his Reliques of Ancient Poetry, from a M.S. in the Pepysian Library,
and Stafford Smith, in his Collection of English Songs, 1779 fol., in fac-simile of
the old notation, as well as in modern score, and with a chorus in three parts to
the words, “Deo gratias, Anglia, redde pro victoria.” The tune is here given
with the first verse of the words,\footnote{
I do not intend to reprint songs or Ballads that are
contained in Percy's Reliques of Ancient Poetry, without
some particular motive, for that delightful book can be
purchased in many shapes and at a small cost. As a
general rule, the versions given by Percy are best suited
to music, because more metrical than others, although
they maybe less exactly and minutely in accordance
with old copies, which are often very carelessly printed
or transcribed.
} for although the original is a regular composition
in three parts, it serves to shew the state of melody at an early period, and
the subject is certainly a national one.

\musictitle{Song on the Victory of Agincourt.}

\musicinfo{Slowly and Majestically.}{1415}
\bigskip

\includemusic{chappellV1005.pdf}

\changefontsize{1.07\defaultfontsize}
There are also two well-known ballads on the Battle of Agincourt; the one
commencing “A council grave our king did hold;” the other “As our king lay
musing in his bed,” which will be noticed under later dates; and a three-men’s
song, which was sung by the tanner and his fellows, to amuse the guests, in
Heywood’s play, \textit{King Edward IV}., beginning—

\indentpattern{“Agincourt! Agincourt! know ye not Agincourt?}
\begin{scverse}\vleftofline{“}Agincourt! Agincourt! know ye not Agincourt?\\
Where the English slew or hurt\\
\vin All the French foemen?” \&c.
\end{scverse}

Although Henry had forbidden the minstrels to celebrate his victory, the order
evidently did not proceed from any disregard for the professors of music or of
song, for at the Feast of Pentecost, which he celebrated in 1416, having the
Emperor and the Duke of Holland as his guests, he ordered rich gowns for sixteen 
of his minstrels. And having before his \pagebreak
 death orally granted an annuity of an
%40
 hundred shillings to each of his minstrels, the grant was confirmed in the first
year of his son, Henry VI. (\ad 1423), and payment ordered out of the exchequer.
Both the biographers of Henry declare his love for music.\dcfootnote{
“Musicis delectabatur.”—Tit. Liv., p.~5. “lnstrumentis
organicis plurimum deditus”—Elmham.
} Lydgate
and Occleve, the poets whom he patronized, attest also his love of literature, and
the encouragement he gave to it.

John Lydgate, Monk of Bury St. Edmunds, describes the minstrelsy of his
time less completely, but in nearly the same terms as Chaucer.

Lydgate was a very voluminous writer. Ritson enumerates 251 of his pieces,
and the list is far from being complete. Among his minor pieces are many songs
and ballads, chiefly satirical, such as “On the forked head-dresses of the ladies,”
on “Thievish Millers and Bakers,” \&c. A selection from these has been recently
printed by the Percy Society.

Among the devices at the coronation banquet of Henry VI. (1429), were, in
the first course, a “sotiltie” (subtlety) of St. Edward and St. Lewis, in coat
armour, holding between them a figure like King Henry, similarly armed, and
standing with a \textit{ballad under his feet}. “In the second, a device of the Emperor
Sigismund and King Henry V., arrayed in mantles of garter, and a figure like
Henry VI. kneeling before them with \textit{a ballad against the Lollards};\dcfootnote{
Ritson has printed one of these ballads against the
Lollards, in his Ancient Songs, p. 63, 1790, taken, from
\textit{MS. Cotton, Vespasian, B. 16. Brit. Mus}.
} and in the
third, one of our Lady, sitting with her child in her lap, and holding a crown in
her hand, St. George and St. Denis kneeling on either side, presenting to her
King Henry \textit{with a ballad in his hand}.\dcfootnote{
Quoted by Sharon Turner, from Fab. 419.
} These subtleties were probably devised
by the clergy, who strove to smother the odium which, as a body, their vices had
excited, by turning public attention to the further persecution of the Lollards.\dcfootnote{
Sir John Oldcastle, Lord Cohham, bad been put to
death in the preceding reign.
}
In a discourse which was prepared to be delivered at the Convocation of the
Clergy, ten days after the death of Edward IV., and which still exists in MS.
(MS. Cotton Cleopatra, E. 3), exhorting the clergy to amendment, the writer
complains that “The people laugh at us, and make us their songs all the day
long.” Vicious persons of every description had been induced to enter the church
on account of the protection it afforded against the secular power, and the facilities
it provided for continued indulgence in their vices.

In that age, as in more enlightened times, the people loved better to be pleased
than instructed, and the minstrels were often more amply paid than the clergy.
During many of the years of Henry VI., particularly in the year 1430, at the
annual feast of the fraternity of the \textsc{Holie Crosse}, at Abingdon, a town in
Berkshire, twelve priests each received four pence for singing a dirge: and the
same number of minstrels were rewarded each with two shillings and four pence,
besides diet and horse-meat. Some of these minstrels came only from Maydenhithe, 
or Maidenhead, a town at no great distance, in the same county. (Liber
Niger, p. 598.) In the year 1441, eight priests were hired from Coventry,
to assist in celebrating a yearly obit in the church \pagebreak 
of the neighbouring priory of
 \markright{henry vi.}%041
%41
Maxtoke; as were six minstrels (\textsc{Mimi}) belonging to the family of Lord Clinton,
who lived in the adjoining Castle of Maxtoke, to sing, harp, and play in the hall
of the monastery, during the extraordinary refection allowed to the monks on that
anniversary. Two shillings were given to the priests, and four to the minstrels:
and the latter are said to have supped in \textit{camera picta}, or the painted chamber of
the convent, with the sub-prior, on which occasion the chamberlain furnished
eight massive tapers of wax. (Warton, vol. ii., p. 309.) However, on this occasion,
the priests seem to have been better paid than usual, for in the same year
(1441) the prior gave no more than sixpence to a preaching friar.

As late as in the early part of the reign of Elizabeth, we find an entry in the
books of the Stationers’ Company (1560) of a similar character: Item, payd to
the preacher, 6\textit{s}. 2\textit{d}. Item, payd to the minstrell, 12\textit{s}.; so that even in the
decline of minstrelsy, the scale of remuneration was relatively the same.

A curious collection of the songs and Christmas carols of this reign (Henry~VI.) 
have been printed recently by the Percy Society. (Songs and Carols, No. 73.)

The manuscript book from which they are taken, had, in all probability, belonged
to a country minstrel who sang at festivals and merry makings, and it has been,
most judiciously, printed entire, as giving a general view of the classes of poetry
then popular. A proportion of its contents consists of carols and religious songs,
such as were sung at Christmas, and perhaps at other festivals of the Church.
Another class, in which the MS. is, for its date, peculiarly rich, consists of
drinking songs. It also contains a number of those satirical songs against the
fair sex, and especially against shrews, which were so common in the middle ages,
and have a certain degree of importance as showing the condition of private
society among our forefathers. The larger number of the songs, including some
of the most interesting and curious, appear to be unique, and the others
are in general much better and more complete copies than those previously
known (viz. in MS. Sloane, No. 2593, Brit. Mus). The editor of the MS.
(Mr. T. Wright) observes that “The great variations in the different copies of
the same song, show that they were taken down from oral recitation, and had
often been preserved by memory among minstrels, who were not unskilful at
composing, and who were not only in the habit of, voluntarily or involuntarily,
modifying the songs as they passed through their hands, and adding or omitting
stanzas, but of making up new songs by stringing together phrases and lines, and
even whole stanzas from the different compositions which were imprinted on their
memories.” But what renders the manuscript peculiarly interesting, is, that it
contains the melodies of some of the songs as well as the words. From this it
appears that the same tune was used for different words. At page 62 is a note,
which in modern spelling is as follows: “This is the tune for the song following;
if so be that ye will have another tune, it may be at your pleasure, for I have set
all the song.” The words of the carol, “Nowell, Nowell,” (Noel) are written
under the notes, but the wassail song that follows, and for which the tune was also
intended, is of a very opposite character, “Bryng us in good ale.” I have
printed the first verse of each under the tune, but it requires to be sung more
quickly for the wassail song than for the carol.
\pagebreak %042

 \noindent\begin{minipage}{\textwidth}
\musictitle{Christmas Carol.\footnotemark}
\musicinfo{The Burden or Chorus}{About 1460.}
\vspace{\baselineskip}

\begin{picture}(375,100)(0,0)
\drawline(210,0)(210,80)(370,80,)(370,0)
%\drawline(0,100)(65,100)(65,10)
\end{picture}
\vspace{-85pt}

\includemusic{chappellV1006.pdf}
\end{minipage}

\footnotetext{\scriptsize
The two bars marked off by a line are added, because
there would not otherwise be music enough for the \textit{Wassail
Song}. They are a mere repetition of the preceding,
and can be omitted at pleasure. The only way in
which the latter could have been sung to the music as
written in the manuscript, would be by omitting the line
“And bring us in good ale;” but, as it is \textit{merely} a repetition,
it \textit{could} be omitted.}

\pagebreak
%043
\normalsize
\markright{christmas carol and wassail song.}

The notation of the original is in semibreves, minims, and crotchets, which
are diminished to crotchets, quavers, and semiquavers, as became necessary in
modernizing the notation; for the quickest note then in use was the crotchet.\dcfootnote{
After the Percy Society had printed the Songs, I was
to have had the opportunity of transcribing \textit{all} the Music;
but, in the mean time, the bookbinder to whom this rare
MS. was entrusted, disappeared, and with him the manuscript, 
which is, perhaps, already in some library in the
United States.
}
The Christmas carol partakes so much of the character of sacred music, that it is
not surprising it should be in an old scale. If there were not the flat at the signature,
which takes off a little of the barbarity, it would be exactly in the eighth
Gregorian tone.

There are seven verses to the carol, but as they are not particularly interesting,
perhaps the words of the wassail song will be preferred, although we should not
now sing of “our blessed lady,” as was common in those days.
\settowidth{\versewidth}{Bring us in no brown bread, for that is made of bran,}
\begin{scverse}
Bring us in no brown bread, for that is made of bran,\\
Nor bring us in no white bread, for therein is no gain,\\
\hspace{\vgap}But bring us in good ale, and bring us in good ale;\\
\hspace{\vgap}For our blessed Lady’s sake, bring us in good ale.

Bring us in no beef, for there is many bones,\\
But bring us in good ale, for that go’th down at once. And bring, \&c.

Bring us in no bacon, for that is passing fat,\\
But bring us in good ale, and give us enough of that. And bring, \&c.

Bring us in no mutton, for that is passing lean,\\
Nor bring us in no tripes, for they be seldom clean. But bring, \&c.

Bring us in no eggs, for there are many shells,\\
But bring us in good ale, and give us nothing else. But bring, \&c.

Bring us in no butter, for therein are many hairs,\\
Nor bring us in no pig’s flesh, for that will make us bears. But bring, \&c.

Bring us in no puddings, for therein is all God’s good,\\
Nor bring us in no venison, that is not for our blood. But bring, \&c.

Bring us in no capon’s flesh, for that is often dear,\\
Nor bring us in no duck’s flesh, for they slobber in the mere, [mire]\\
\hspace{\vgap}But bring us in good ale, and bring us in good ale,\\
\hspace{\vgap}For our blessed lady’s sake, bring us in good ale.
\end{scverse}

An inferior copy of this song, without music, is in Harl. M.S., No. 541, from
which it has been printed in Ritson’s Ancient Songs, p. xxxiv. and xxxv.

With the reign of Edward IV. we may conclude the history of the \textit{old wandering}
minstrel. In 1469, on a complaint that persons had collected money in different
parts of the kingdom by assuming the title and livery of the king’s minstrels, he
granted to Walter Halliday, \textit{Marshal}, and to seven others whom he names,
a charter of incorporation. They were to be governed by a marshal appointed for
life, and two wardens to be chosen annually, who were authorized to admit members; 
also to examine the pretensions of all who exercised the minstrel profession,
and to regulate, govern, and punish them \pagebreak
 throughout the realm (those of Chester
%44
excepted). “This,” says Percy, “seems to have some resemblance to the Earl
Marshal’s court among the heralds, and is another proof of the great affinity and
resemblance which the minstrels bore to the College of Arms.” Walter Halliday,
above mentioned, had been retained in the service of the two preceding monarchs,
and Edward had granted him an annuity of ten marks for life, in 1464.

\changefontsize{1.06\defaultfontsize}
In this reign we find also mention of a \textit{Serjeant} of the minstrels, who upon
one occasion did his royal master a singular service, and by which his ready access
to the king at all hours is very apparent: for “as he [K. Edward IV.] was in
the north contray, in the Monneth of Septembre, \textit{as he lay in his bedde}, one
named Alexander Carlile, that was Sarjaunt of the Mynstrellis, cam to him
in grete hast, and badde hym aryse, for he hadde enemyes cumming for to take
him, the which were within six or seven miles,” \&c.

Edward seems to have been very liberal to his minstrels. He gave to several
annuities of ten marks a year (6 Parl. Rolls, p. 89), and, besides their
regular pay, with clothing and lodging for themselves and \textit{their horses}, they had
two servants to carry their instruments, four gallons of ale per night, wax candles,
and other indulgences. The charter is printed in Rymer, xi. 642, by Sir
J. Hawkins, vol. iv., p. 366, and Burney, vol. ii., p. 429. All the minstrels
have English names.

When Elizabeth, his queen, went to Westminster Abbey to be church\-ed (1466),
she was preceded by troops of choristers, chanting hymns, and to these succeeded
long lines of the noblest and fairest women of London and its vicinity, attended by
bands of musicians and trumpeters, and forty-two royal singers. After the banquet
and state ball, a state concert commenced, at which the Bohemian ambassadors
were present, and in their opinion as well as that of Tetzel, the German who accompanied
them, and who has also recounted their visit to England, no better
singers could be found in the whole world,\footnote{
Tetzel says, “Nach dem Tantz do muosten des
Kunigs Cantores kumen und muosten singen\ldots ich
mein das, in der Welt, nit besser Cantores sein.” “\textit{Des
böhmischen Herrn Leo’s von Rozmital Ritter,—Hof und
Pilger—Reise, 1465-1467,” \&c., 8vo., Stuttgart}, 1844, p. 157.

Again Tetzel says, “Do hörten wir das aller kostlichst
Korgesang, das alls gesatzt was, das lieblich zu hören
was.”—\textit{Ib}. p. 158.

Leo Von Rozmital, brother of the Queen of Bohemia
says "Musicos nullo uspiam in loco jucundiores et
suaviores audivimus, quam ibi: eorum chorus sexaginta,
circiter cantoribus constat,” —\textit{Ib}. p. 42.
} than those of the English king.
These ambassadors travelled through France, Belgium, Spain, Portugal, Italy,
and parts of Germany, as well as England, affording them, therefore, the widest
field for comparison with the singers of other countries.

At this time every great family had its establishment of musicians, and among
them the harper held a prominent position. Some who were less wealthy retained
a harper only, as did many bishops and abbots. In Sir John Howard’s expenses
(1464) there is an entry of a payment as a new year’s gift to Lady Howard’s
grandmother’s harper, “that dwellyth in Chestre.” When he became Lord
Howard he retained in his service, Nicholas Stapylton, William Lyndsey, and
“little Richard,” as singers, besides “Thomas, the harperd,” (whom he provided
with a “lyard,” or grey “gown”), and children of the chapel, who were successively
four, five, and six in number at different dates. Mr. Payne Collier, who
edited his Household Book from 1481 to \pagebreak
 1485 for the Roxburghe Club, remarks
%45
\markright{edward iv.} %045
on “the great variety of entries in connection with music and musical performers,”
as forming “a prominent feature” of the hook. “Not only were the musicians
attached to noblemen, or to private individuals, liberally rewarded, but also those
who were attached to particular towns, and who seem to have been generally
required to perform before Lord Howard on his various journies. On the 14th of
October, 1841, he entered into an agreement with William Wastell, harper of
London, that he should teach the son of John Colet, of Colchester, harper, for
a year, in order, probably, to render him competent afterwards to fill the post of
one of the family musicians.”

Here also a part of the stipulation was that, at the end of the year, Lord
Howard should give Wastell a \textit{gown}, which seems to have been the distinguishing
feature of a harper’s dress. In Laneham’s letter from Kenilworth (1575),
describing the “device of an \textit{ancient} minstrel and his song,” which was to have
been proffered for the amusement of queen Elizabeth, this “Squire minstrel, of
Middlesex, who travelled the country this summer season, unto worshipful men’s
houses,” is represented as a harper with a long gown of Kendal green, gathered
at the neck with a narrow gorget, and fastened before with a white clasp; his
gown having long sleeves down to mid-leg, but slit from the shoulders to the
hand, and lined with white. His harp was to be “in good grace dependent before
him,” and his “wrest,” or tuning-key, “tied to a green lace, and hanging by.”
He wore a red Cadiz girdle, and the corner of his handkerchief, edged with blue
lace hung from his bosom. Under the gorget of his gown hung a chain, “resplendent
upon his breast, of the ancient arms of Islington.” The acts of king
Arthur were the subject of his song.

The Romances which still remained popular [1480] are mentioned by William
of Nassyngton [in a MS. which Warton saw in the library of Lincoln Cathedral],
who gives his readers fair notice that \textit{he} does not intend to amuse them.


\settowidth{\versewidth}{“I warne you first at the begynnynge}

\begin{dcverse}
“I warne you first at the begynnynge\\
That I will make no vayne carpynge,\\
Of dedes of armes, ne of amours,\\
As does Mynstrellis and Gestours,\\
That maketh carpynge in many a place\\
Of \textsc{Octaviane} and \textsc{Isenbrace},

And of many other \textit{Gestes},\\
As namely, when they come to festes;\\
Ne of the lyf of \textsc{Bevys of Hamptoune},\\
That was a Knyght of grete renowne;\\
Ne of \textsc{Syr Gye of Warwyke}, \&c.\\
\hfill\textit{Warton}, vol. iv., p. 368.
\end{dcverse}


The invention of printing, coupled with the increased cultivation of poetry and
music by men of genius and learning, accelerated the downfall of the Minstrels.
They could not long withstand the superior standard of excellence in the sister
arts, on the one hand, and the competition of the ballad-singer (who sang without
asking remuneration, and sold his songs for a penny) on the other. In little more
than fifty years from this time they seem to have fallen into utter contempt. We
have a melancholy picture of their condition, in the person of Richard Sheale,
which it is impossible to read without sympathy, if we consider that to him we
are indebted for the preservation of the celebrated heroic ballad of \textit{Chevy Chace},
at which Sir Philip Sidney’s heart was \pagebreak
 wont to beat, “as at the sound of a
%46
trumpet;”\footnote{
“I never heard the old song of Percy and Douglas, that
I found not my heart moved more than with a trumpet: and
yet it is sung but by some blind crowder, with no rougher
voice than rude style; which being so evil aparelled
in the dust and cobweb of that uncivil age, what would it
work, trimmed in the gorgeous eloquence of Pindare!”—
\textit{Sir Philip Sidney’s Defence of Poetry}.
} and of which Ben Jonson declared he would rather have been the
author, than of all he had ever written. This luckless Minstrel had been robbed
on Dunsmore Heath, and, shame to tell, he was unable to persuade the public
that a son of the Muses had ever been possessed of sixty pounds, which he
averred he had lost on the occasion. The account he gives of the effect upon his
spirits is melancholy, and yet ridiculous enough. [As the preservation of the
old spelling is no longer essential to the rhyme or metre, I venture to give it in
modern orthography.]

\settowidth{\versewidth}{\textit{From the “Chant of Richard Sheale,”—British Bibliographer},}
\begin{scverse}
\vleftofline{“}After my robbery my memory was so decay’d\\
That I could neither sing, nor talk, my wits were so dismay’d.\\
My audacity was gone, and all my merry talk,\\
There are some here have seen me as merry as a hawk;\\
But now I am so troubled with fancies in my mind,\\
I cannot play the merry knave, according to my kind.\\
Yet to take thought, I perceive, is not the next way\\
To bring me out of debt,—my creditors to pay.\\
I may well say that I had but evil hap\\
For to lose about threescore pounds at a clap.\\
The loss of my money did not grieve me so sore,\\
But the talk of the people did grieve me much more.\\
Some said I was not robb’d, I was but a lying knave,\\
\textit{It was not possible for a Minstrel so much money to have}.\\
Indeed, to say the truth, it is right well known\\
That I never had so much money of my own,\\
But I had friends in London, whose names I can declare,\\
That at all times would lend me two hundred pounds of ware,\\
And with some again such friendship I found,\\
That they would lend me in money nine or ten pound.\\
The occasion why I came in debt I shall make relation—\\
My wife, indeed, is a silk-woman, by her occupation;\\
In linen cloths, most chiefly, was her greatest trade,\\
And at fairs and markets she sold sale-ware that she made,\\
As shirts, smocks, and partlets, head-clothes, and other things,\\
As silk thread and edgings, skirts, bands, and strings.\\
At Lichfield market, and Atherston, good customers she found,\\
Also at Tamworth, where I dwell, she took many a pound.\\
When I had got my money together, my debts to have paid,\\
This sad mischance on me did fall, that cannot be denay’d; [denied]\\
I thought to have paid all my debts and to have set me clear,\\
And then what evil did ensue, ye shall hereafter hear:\\
Because my carriage should be light I put my money into gold,\\
And without company I rode alone—thus was I foolish bold;\\
\textit{I thought by reason of my harp no man would me suspect,\\
For Minstrels oft with money, they be not much infect."\\
\hfill From the “Chant of Richard Sheale,”—British Bibliographer}, vol. iv., p. 100.
\end{scverse}
\pagebreak
%047
\markright{richard sheale.---extinction of minstrelsy.}

Sheale was a Minstrel in the service of Edward, Earl of Derby, who died in
1574, celebrated for his bounty and hospitality, of whom Sheale speaks most
gratefully, as well as of his eldest son, Lord Strange. The same MS. contains an
“Epilogue” on the Countess of Derby, who died in January, 1558, and his
version of Chevy Chace must have been written at \textit{least} ten years before the
latter date, if it be the one mentioned in the Complaynte of Scotland, which was
written in 1548.

In the thirty-ninth year of Elizabeth, an act was passed by which “Minstrels,
wandering abroad” were held to be “rogues, vagabonds, and sturdy beggars,”
and were to be punished as such. This act seems to have extinguished the profession
of the Minstrels, who so long had basked in the sunshine of prosperity.
The name, however, remained, and was applied to itinerant harpers, fiddlers,
and other strolling musicians, who are thus described by Puttenham, in his \textit{Arte
of English Poesie}, printed in 1589. Speaking of ballad music, he says, “The
over busy and too speedy return of one manner of tune, doth too much annoy,
and, as it were, glut the ear, unless it be in small and popular musicks sung by
these \textit{Cantabanqui} upon benches and barrels’ heads, where they have none other
audience than boys or country fellows that pass by them in the street; or else by
blind harpers, or such like tavern minstrels, that give a fit of mirth for a groat;
and their matter being for the most part stories of old time, as the Tale of Sir
Topas, Bevis of Southampton, Guy of Warwick, Adam Bell and Clym of the
Clough, and such other old romances or historical rhimes, \textit{made purposely for} the
recreation of the common people at Christmas dinners and bride-ales, and in
taverns and alehouses, and such other places of base resort. Also they” [these
short times] “be used in Carols and Rounds, and such like light and lascivious
poems, which are commonly more commodiously uttered by these buffons, or vices
in plays than by another person.”

Ritson, whose animosity to Percy and Warton seems to have extended itself
to the whole minstrel race, quotes, with great glee, the following lines on their
downfall, which were written by Dr. Bull, a rival musician:—

\settowidth{\versewidth}{He turned the Minstrels out of doors,}
\begin{scverse}
“When Jesus went to Jairus’ house,\\
(Whose daughter was about to die)\\
He turned the Minstrels out of doors,\\
Among the rascal company:\\
Beggars they are with one consent,—\\
And rogues, \textit{by act of Parliament.}”
\end{scverse}

\vfill

\centerrule
\pagebreak
\vfill
