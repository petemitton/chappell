%201
\changefontsize{\defaultfontsize}

\musicinfo{Slowly and smoothly}{}

\includemusic{chappellV1100.pdf}

\musictitle{In Sad And Ashy Weeds.}

The four first stanzas of this song were found among the Howard papers in
the Heralds’ College, in the handwriting of Anne, Countess of Arundel, widow of
the Earl who died in confinement in the Tower of London in 1595. They were
written on the cover of a letter. Lodge, who printed them in his \textit{Illustrations of
British History} (iii. 241, 8vo., 1838), thought they “were probably composed”
by the Countess; and that “the melancholy exit of her lord was not unlikely to
have produced these pathetic effusions.” She could not, however, have been the
\textit{author} of verses, in her transcript of which the rhymes between the first and third
lines of every stanza have been overlooked. \footnote{\textit{}
In the Countess’s transcript, as printed by Lodge,
the first four lines stand thus—
\settowidth{\versewidth}{In sad and ashy weeds I sigh,}
\begin{fnverse}
\begin{altverse}
\vleftofline{“}In sad and ashy weeds I \textit{sigh},\\
I groan, I pine, I mourn;\\
My oaten yellow reeds\\
I all to jet and ebon turn;”
\end{altverse}
\end{fnverse}
instead of—
\begin{fnverse}
\begin{altverse}
“In sad and ashy weeds\\
I sigh, I groan, I pine, I mourn;”
\end{altverse}
\end{fnverse}
as “weeds” should rhyme with “reeds” in the third line,
and so in each verse.}
They were evidently written from
memory, and rendered more applicable to her case by a few trifling alterations,
such as “Not I, poor I, alone,” instead of “Now, a poor lad alone,” at the
commencement of the fourth stanza.

The tune is contained in a MS. volume of virginal music, transcribed by Sir
John Hawkins; the words in the \textit{Crown Garland of Golden Roses}, edition of
1659 (Percy Society reprint, p. 6.). It is there entitled “The good Shepherd’s
sorrow for the loss of his beloved son.”

Among the ballads to the tune of \textit{In sad and ashy weeds}, are “A servant’s
sorrow for the loss of his late royal mistress, Queen Anne” (wife to James I.),
“who died at Hampton Court” (May 2, 1618), beginning—
\settowidth{\versewidth}{In dole and deep distress,}
\begin{scverse}
\vleftofline{“}In dole and deep distress,\\
Poor soul, I, sighing, make my moan.”
\end{scverse}
It will be found in the same edition of the \textit{Crown Garland}; as well as an answer
to \textit{In sad and ashy weeds}, entitled “Coridon’s Comfort: the second part of the
good Shepherd commencing, “Peace, Shepherd, cease to moan.”

The tune is quoted under the title of “In sadness, or Who can blame my woe,”
as one for the \textit{Psalmes or Songs of Sion}, \&c., 1642.

\pagebreak
%202

\musicinfo{Slowly and smoothly.}{}

\includemusic{chappellV1101.pdf}

\vspace{-2\baselineskip}

\settowidth{\versewidth}{\vin But more than men make moan with me:}
\indentpattern{0101221221}
\begin{dcverse}\begin{patverse}
In sable robes of night\\
My days of joy consumed be,\\
My sorrow sees no light,\\
My lights through sorrow nothing see.\\
For now my sun\\
His course hath run,\\
And from my sphere doth go,\\
To endless bed\\
Of folded lead;\\
And who can blame my woe?
\end{patverse}

\begin{patverse}
My flocks I now forsake,\\
That so my sheep my grief may know,\\
The lilies loathe to take,\\
That since his death presum’d to grow.\\
I envy air,\\
Because it dare\\
Still breathe, and he not so;\\
Hate earth, that doth\\
Entomb his youth;\\
And who can blame my woe?
\end{patverse}

\begin{patverse}
Not I, poor I, alone,\\
(Alone, how can this sorrow be?)\\
Not only men make moan,\\
But more than men make moan with me:\\
The gods of greens, \\
The mountain queens,\\
The fairy-circled row,\\
The muses nine,\\
And powers divine,\\
Do all condole my woe.
\end{patverse}
\end{dcverse}

In the above lines I have chiefly followed the Countess of Arundel’s transcript.
There are three more verses in the \textit{Crown Garland of Golden Roses}, besides seven
in the second part.
\pagebreak
%203

\musictitle{The Bailiff’S Daughter Of Islington.}

Copies of this ballad are in the Roxburghe, Pepys, and Douce Collections; it is
printed by Ritson among the ancient ballads in his \textit{English Songs}, and by Percy
(\textit{Reliques}, series iii., book 2, No. 8).

In the Roxburghe, ii. 457, and Douce, 230, it is entitled “True love requited,
or The Bailiff’s Daughter of Islington: to a \textit{North-country tune}, or \textit{I have a good
old mother at home}.'” In other copies it is to “I have a good \textit{old woman} at home,”
and “I have a good \textit{wife} at home.”

In the Douce, 32, is a ballad called “Crums of comfort for the youngest sister,
\&c., to a pleasant new \textit{West-country} tune;” beginning—
\settowidth{\versewidth}{I have a good old \textit{father} at home,}
\begin{scverse}\begin{altverse}
\vleftofline{“}I have a good old \textit{father} at home,\\
An ancient man is he:\\
But he has a mind that ere he dies\\
That I should married be.”
\end{altverse}
\end{scverse}

Dr. Rimbault found the first tune in a lute MS., formerly in the possession of
the Rev. Mr. Gostling, of Canterbury, under the name of \textit{The jolly Pinder}. It is
in the ballad-opera of \textit{The Jovial Crew}, 1731, called “The Baily’s Daughter of
Islington.”

The second is the traditional tune to which it is commonly sung throughout the
country.

\musicinfo{Rather slow.}{First tune.}

\includemusic{chappellV1102.pdf}

\vspace{-1.5\baselineskip}

\settowidth{\versewidth}{Yet she was coy, and would not believe}
\begin{dcverse}\begin{altverse}
Yet she was coy, and would not believe\\
That he did love her so,\\
No, nor at any time would she\\
Any countenance to him show.
\end{altverse}

\begin{altverse}
But when his friends did understand\\
His fond and foolish mind,\\
They sent him up to fair London,\\
An apprentice for to bind.
\end{altverse}

\begin{altverse}
And when he had been seven long years,\\
And never his love could see:\\
Many a tear have I shed for her sake,\\
When she little thought of me.
\end{altverse}

\begin{altverse}
Then all the maids of Islington\\
Went forth to sport and play,\\
All but the bailiff’s daughter dear;\\
She secretly stole away.
\end{altverse}

\begin{altverse}
She pulled off her gown of green,\\
And put on ragged attire,\\
And to fair London she would go,\\
Her true love to enquire.
\end{altverse}

\begin{altverse}
And as she went along the high road,\\
The weather being hot and dry,\\
She sat her down upon a green bank,\\
And her true love came riding by.
\end{altverse}
\end{dcverse}
\origpage{}%204

\settowidth{\versewidth}{\vin For now I have found mine own true love,}
\begin{dcverse}\begin{altverse}
She started up with a colour so red,\\
Catching hold of his bridle-rein;\\
One penny, one penny, kind sir, she said,\\
Will ease me of much pain.
\end{altverse}

\begin{altverse}
Before I give you one penny, sweet-heart,\\
Pray tell me where you were born:\\
At Islington, kind Sir, said she,\\
Where I have had many a scorn.
\end{altverse}

\begin{altverse}
I prythee, sweetheart, tell to me,\\
O tell me whether you know\\
The bailiff’s daughter of Islington?\\
She is dead, Sir, long ago.
\end{altverse}

\begin{altverse}
If she he dead, then take my horse,\\
My saddle and bridle also;\\
For I will into some far country,\\
Where no man shall me know.
\end{altverse}

\begin{altverse}
O stay, O stay, thou goodly youth,\\
She standeth by thy side;\\
She is here alive, she is not dead,\\
And ready to be thy bride.
\end{altverse}

\begin{altverse}
O farewell grief, and welcome joy,\\
Ten thousand times therefore;\\
For now I have found mine own true love,\\
Whom I thought I should never see more.
\end{altverse}
\end{dcverse}

\vspace{-\baselineskip}

\musicinfo{Rather slowly and very smoothly.}{Second tune.}

\includemusic{chappellV1103.pdf}

\vspace{-2\baselineskip}

\musictitle{It Was A Lover And His Lass.}

From a quarto MS., which has successively passed through the hands of
Mr. Cranston, Dr. John Leyden, and Mr. Heber; and is now in the Advocates’
Library, Edinburgh. It contains about thirty-four songs with words,\footnote{\textit{}
Among these are Wither’s song, “Shall I, wasting
in despair,” and “Farewell, dear love,” quoted in \textit{Twelfth
Night}, the music of which, by Robert Jones (twelfth from
his first book, published in 1601) is reprinted in \textit{Musica
Antiqua: a Selection of Music from the commencement of
the twelfth to the beginning of the eighteenth century}, \&c.
edited by John Stafford Smith.}
 and sixteen
song and dance tunes without. The latter part of the manuscript, which bears
the name of a former proprietor, William Stirling, and the date of May, 1639,
consists of Psalm Tunes, evidently in the same handwriting, and written about
the same time as the earlier portion. This song is in the comedy of \textit{As you
like it}, the first edition of which was printed in 1623; and the inaccuracies in
that copy, which have given much trouble to commentators on Shakespeare, are
not to be found in this. In the printed copy, the last verse stands in the place of
the second: this was first observed and remedied by Dr. Thirlby; and the words
“ring time,” there rendered “\textit{rang} time,” and by commentators altered to “\textit{rank}
time,” were first restored to the proper meaning by Steevens, who explains them
as signifying the aptest season for marriage. \pagebreak The words are here printed from the
\markright{illustrating shakespeare.}
%205
manuscript in the Advocates’ Library, (fol. 18), and other variations will be
found on comparing them with the published copies of the~play.

\musicinfo{Moderate time.}{}

\includemusic{chappellV1104.pdf}

\vspace{-2\baselineskip}

\settowidth{\versewidth}{With a hey, with a ho, with a hey, non ne no,}
\indentpattern{000022222}
\begin{dcverse}\begin{patverse}
Between the acres of the rye,\\
With a hey, with a ho, with a hey, non ne no,\\
And a hey non ne, no ni no.\\
These pretty country fools did lie,\\
In Spring time, in Spring time,\\
The only pretty ring time,\\
When birds do sing\\
Hey ding, a ding, a ding,\\
Sweet lovers love the Spring.
\end{patverse}

\settowidth{\versewidth}{This carol they began that hour,}
\indentpattern{0202}
\begin{patverse}
This carol they began that hour,\\
With a hey, \&c.\\
How that life was but a flow’r,\\
In Spring time, \&c.
\end{patverse}

\begin{patverse}
Then, pretty lovers, take the time,\\
With a hey, \&c.,\\
For love is crowned with the prime,\\
In Spring time, \&c.
\end{patverse}
\end{dcverse}
\pagebreak
%206

\musictitle{Willow, Willow!}

The song of \textit{Oh! willow, willow}, which Desdemona sings in the fourth act of
\textit{Othello}, is contained in a MS. volume of songs, with accompaniment for the lute,
in the British Museum (Addit. MSS. 15,117). Mr. Halliwell considers the
transcript to have been made about the year 1633; Mr. Oliphant (who catalogued
the musical MS.) dates it about 1600; but the manuscript undoubtedly contains
songs of an earlier time, such as—
\settowidth{\versewidth}{O death! rock me asleep,}
\begin{scverse}
\vleftofline{“}O death! rock me asleep,\\
Bring me to quiet rest,” \&c.
\end{scverse}
attributed to Anne Boleyn, and which Sir John Hawkins found in a MS. of the
reign of Henry VIII.

The song of \textit{Willow, willow}, is also in the Roxburghe Ballads, i. 54; and was
printed by Percy from a copy in the Pepys Collection, entitled “A Lover’s
Complaint, being forsaken of his Love: to a pleasant tune.”

\textit{Willow, willow}, was a favorite burden for songs in the sixteenth century.
There is one by John Heywood, a favorite dramatist and court musician of the
reigns of Henry VIII. and Queen Mary, beginning—
\settowidth{\versewidth}{Alas! by what mean may I make ye to know}
\begin{scverse}
\vleftofline{“}Alas! by what mean may I make ye to know\\
The unkindness for kindness that to me doth grow?”
\end{scverse}
which has for the burden—
\settowidth{\versewidth}{All a green willow; willow, willow, willow;}
\begin{scverse}
\vleftofline{“}All a green willow; willow, willow, willow;\\
All a green willow, is my garland.”
\end{scverse}
It has been printed by Mr. Halliwell, with others by Heywood, Redford, \&c., for
the Shakespeare Society, in a volume containing the moral play of \textit{Wit and
Science}.

Another with the burden—
\begin{scverse}
\vleftofline{“}Willow, willow, willow; sing all of green willow;\\
Sing all of green willow, shall be my garland,”
\end{scverse}
will be found in \textit{A Gorgious Gallery of Gallant Inventions} (1578). It commences
thus:
\settowidth{\versewidth}{My love, what misliking in me do you find,}
\begin{scverse}
\indentpattern{02020202}
\begin{patverse}
\vleftofline{“}My love, what misliking in me do you find,\\
Sing all of green willow;\\
That on such a sudden you alter your mind?\\
Sing willow, willow, willow.\\
What cause doth compel you so fickle to be,\\
Willow, willow, willow, willow;\\
In heart which you plighted most,loyal to me?\\
Willow, willow, willow, willow.”--\textit{Heliconia}, i. 32.
\end{patverse}
\end{scverse}

In Fletcher’s \textit{The two Noble Kinsmen}, when the Jailer’s daughter went mad
for love, “She sung nothing but \textit{Willow, willow, willow}.”—Act iv., sc. 1.

In the tragedy of \textit{Othello}, Desdemona introduces the song “in this pathetic
and affecting manner:”
\pagebreak
%207
\markright{illustrating shakespeare.}

\settowidth{\versewidth}{And did forsake her: she had a song of Willow;}
\begin{scverse}
\vleftofline{“}My mother had a maid call’d Barbara;\\
She was in love; and he she lov’d prov’d mad,\\
And did forsake her: she had a song of \textit{Willow};\\
And \textit{old thing} ’twas, but it express’d her fortune,\\
And she died singing it. That song to-night\\
Will not go from my mind; I have much to do,\\
But to go hang my head all at one side,\\
And sing it like poor Barbara.”
\end{scverse}

\musicinfo{Rather slow and smoothly.}{}

\includemusic{chappellV1105.pdf}

\pagebreak
%208

\settowidth{\versewidth}{The true tears fell from him would have melted the stones, Sing,}
\begin{scverse}
He sigh’d in his singing, and made a great moan, Sing, \&c.;\\
I am dead to all pleasure, my true love he-is gone, \&e.

The mute bird sat by him was made tame by his moans, \&c.;\\
The true tears fell from him would have melted the stones, Sing, \&c.

Come, all you forsaken, and mourn you with me, Sing, \&c.;\\
Who speaks of a false love, mine’s falser than she, \&c.

Let love no more boast her in palace nor bower, Sing, \&c.;\\
It buds, but it blasteth ere it be a flower, \&c.

Though fair, and more false, I die with thy wound, Sing, \&c.;\\
Thou hast lost the truest lover that goes upon the ground, \&c.'

Let nobody chide her, her scorns I approve [though I prove];\\
She was born to be false, and I to die for her love, \&c.

Take this for my farewell and latest adieu, Sing, \&c.;\\
Write this on my tomb, that in love I was true, \&c.
\end{scverse}

The above copy of the words is from the same manuscript as the music. It
differs from that in Percy’s \textit{Reliques of Ancient Poetry}; and Shakespeare has
somewhat varied it to apply to a female character.

\musictitle{Whoop! Do Me No Harm, Good Man.}

This is twice alluded to by Shakespeare, in act iv., sc. 3, of \textit{A Winter’s Tale};
and by Ford, in act iii., sc. 3, of \textit{The Fancies chaste and noble}, where Secco,
applying it to Morosa, sings “Whoop! do me no harm, good \textit{woman}.”

The tune was transcribed by Dr. Rimbault, from a MS. volume of virginal
music, in the possession of the late John Holmes, Esq., of Retford. A song with
this burden will be found in Fry’s \textit{Ancient Poetry}, but it would not be desirable
for republication.

\musicinfo{Cheerfully.}{}

\includemusic{chappellV1106.pdf}

\pagebreak
%209

\musictitle{O Mistress Mine!}

This tune is contained in both the editions of Morley’s \textit{Consort Lessons}, 1599
and 1611. It is also in Queen Elizabeth’s Virginal Book, arranged by Byrd.

As it is to be found in print in 1599, it proves either that Shakespeare’s \textit{Twelfth
Night} was written in or before that year, or that, in accordance with the then prevailing
custom, \textit{O Mistress mine} was an old song, introduced into the play.

Mr. Payne Collier has proved \textit{Twelfth Night} to have been an established
favorite in February, 1602 (\textit{Annals of the Stage}, i. 327), but we have no evidence
of so early a date as 1599.

In act ii., sc. 3., the Clown asks, “Would you have a love-song, or a song of
good life?”

\textit{Sir Toby}.—“A love-song, a love-song.”

\musicinfo{Moderate time and very smoothly.}{}

\includemusic{chappellV1107.pdf}

\vspace{-1.5\baselineskip}

\settowidth{\versewidth}{Then come kiss me, sweet-and-twenty,}
\indentpattern{001001}
\begin{scverse}\begin{patverse}
\vleftofline{“}What is love?—’tis not hereafter;\\
Present mirth hath present laughter;\\
What’s to come is still unsure:\\
In delay there lies no plenty;\\
Then come kiss me, sweet-and-twenty,\\
Youth’s a stuff will not endure.”
\end{patverse}
\end{scverse}

\vspace{-1.5\baselineskip}

\musictitle{Heart’s-Ease.}

The tune of \textit{Heart’s-ease} is contained in a MS. volume of lute music, of the
sixteenth century in the Public Library, Cambridge (D. d., ii. 11), as well as in
\textit{The Dancing Master}, from 1650 to 1698. It belongs, in all probability, to an
earlier reign than that of Elizabeth, as it was sufficiently popular about the year
1560 to have a song written to it in the interlude of \textit{Misogonus}. Shakespeare
thus alludes to it in \textit{Romeo and Juliet}, 1597 (act iv., sc. 5.)—
\pagebreak
%210

\textit{Peter}.—“Musicians, O musicians, \textit{Heart’s-ease, heart’s-ease}: O an you will have
me live, play \textit{Heart’s-ease}.

\textit{1st Mus}.—Why \textit{Heart’s-ease}?

\textit{Peter}.—O musicians, because my heart itself plays \textit{My heart is full of woe}:\footnote{\textit{}
This is the burden of “A pleasant new Ballad of two
Lovers: to a pleasant new tune;“beginning—

\settowidth{\versewidth}{Complain my lute, complain on him}
\begin{fnverse}
\begin{altverse}
\vleftofline{“}Complain my lute, complain on him\\
That stays so long away;\\
He promised to be here ere this,\\
But still unkind doth stay.\\
But now the proverb true I find,\\
Once out of sight then out of mind.\\
Hey, ho! \textit{my heart is full of woe},” \&c.
\end{altverse}
\end{fnverse}
It has been reprinted by Mr. Andrew Barton, in the first
volume of the Shakespeare Society's Papers, 1844.}
O play me some merry dump,\footnote{\textit{}
A dump was a slow dance. \textit{Queen Mary's Dump} is
one of the tunes in William Ballet’s Lute Book, and My
\textit{Lady Carey's Dompe} is printed in Stafford Smith’s \textit{Musica
Antiqua}, ii. 470, from a manuscript in the British
Museum, temp. Henry VIII.}
 to comfort me.”

The following song is from \textit{Misogonus}, by Thomas Rychardes; and, as Mr.
Payne Collier remarks, “recollecting that it was written about the year 1560,
may be pronounced quite as good in its kind as the drinking song\footnote{\textit{}
“I cannot eat but little meat,” see page 72.}
 in \textit{Gammer
Gurtons Needle}.”

\musicinfo{Moderate time.}{}

\includemusic{chappellV1108.pdf}

\pagebreak
%211

\settowidth{\versewidth}{The miser’s wealth doth hurt his health;—}

\begin{dcverse}\begin{altverse}
“What doth’t avail far hence to sail,\\
And lead our life in toiling?\\
Or to what end should we here spend\\
Our days in irksome moiling? [labour]\\
It is the best to live at rest,\\
And take’t as God doth send it;\\
To haunt each wake, and mirth to make,\\
And with good fellows spend it.
\end{altverse}

\begin{altverse}
Nothing is worse than a full purse\\
To niggards and to pinchers;\\
They always spare, and live in care,\\
There’s no man loves such flinchers.\\
The merry man, with cup and can,\\
Lives longer than do twenty;\\
The miser’s wealth doth hurt his health;—\\
Examples we have plenty.
\columnbreak
\end{altverse}

\begin{altverse}
’Tis a beastly thing to lie musing\\
With pensiveness and sorrow;\\
For who can tell that he shall well\\
Live here until the morrow?\\
We will, therefore, for evermore,\\
While this our life is lasting,\\
Eat, drink, and sleep, and ‘merry’ keep,\\
’Tis Popery to use fasting.
\end{altverse}

\begin{altverse}
In cards and dice our comfort lies,\\
In sporting and in dancing,\\
Our minds to please and live at ease,\\
And sometimes to use prancing.\\
With Bess and Nell we love to dwell\\
In kissing and in ‘talking;’\\
But whoop! ho holly, with trolly lolly,\\
To them we’ll now be walking.”
\end{altverse}

\end{dcverse}
{\footnotesize\hfill Collier’s \textit{History of Early Dramatic Poetry}, ii. 470.}

\musictitle{Jog On, Jog On.}

This tune is in \textit{The Dancing Master}, from 1650 to 1698, called \textit{Jog on}; also in
Queen Elizabeth’s Virginal Book, under the name of \textit{Hanskin}. The words of
\textit{Jog~on}, of which the first verse is sung by Autolycus, in act iv., sc. 2, of
Shakespeare’s \textit{A Winter’s Tale}, are in \textit{The Antidote against Melancholy}, 1661.
Another name for the tune is \textit{Sir Francis Drake}, or \textit{Eighty-eight}.

The following is the song from \textit{The Antidote against Melancholy}:—
\settowidth{\versewidth}{Your merry heart goes all the day}
\begin{dcverse}
\begin{altverse}
\vleftofline{“}Jog on, jog on the footpath way,\\
And merrily hent\footnote{\textit{}
To hent or hend is to hold or seize. At the head of
one of the chapters of Sir Walter Scott’s novels, this is
misquoted “bend.”

\begin{fnverse}% vins needed due to us already being in a verse env (?)
\vin\vleftofline{“}And in his hand a battle-axe he \textit{hent}.”—\textit{Honor of the Garter}, by George Peele.\\
\vleftofline{“}Upon the sea, till Jhesu Crist him \textit{hente}.”—Chaucer, line~700.\\
\vin\vleftofline{“}Till they the reynes of his bridel \textit{henten}.”—Chaucer, line~906.\\
\vleftofline{“}Or reave it out of the hand that did it \textit{hend}.” —Spenser’s \textit{Faery Queen}.
\end{fnverse}
}
 the stile-a;\\
Your merry heart goes all the day;\\
Your sad tires in a mile-a.

Your paltry money-bags of gold,\\
What need have we to stare for,\\
When little or nothing soon is told.\\
And we have the less to care for.
\end{altverse}
\end{dcverse}

\begin{scverse}
\begin{altverse}
Cast care away, let sorrow cease,\\
A fig for melancholy;\\
Let’s laugh and sing, or, if you please,\\
We’ll frolic with sweet Dolly.”
\end{altverse}
\end{scverse}

In the \textit{Westminster Drollery}, 3rd edit., 1672, is “An \textit{old} song on the Spanish
Armado,” beginning, “Some years of late, in eighty-eight;” and in MSS. Harl.,
791, fol. 59, and in \textit{Merry Drollery complete}, 1661, a different version of the same,
commencing, “In eighty-eight, ere I was born.” Both have been reprinted for
the Percy Society in Halliwell’s \textit{Naval Ballads of England}. The former is also
in \textit{Pills to purge Melancholy}, 1707, ii. 37, and 1719, iv. 37, or Ritson’s \textit{Ancient
Songs}, 1790, p. 271.

In the Collection of Ballads in the Cheetham Library, \pagebreak Manchester, fol. 30, is
%212
“The Catholick Ballad, or an Invitation to Popery, upon considerable grounds and
reasons, to the tune of \textit{Eighty-eight}.” It is in black-letter, with a bad copy of the
tune, and another (No. 1103), dated 1674. It will also be found in \textit{Pills to purge
Melancholy}, 1707, ii. 32, or 1719, iv. 32. It commences thus:—
\settowidth{\versewidth}{Since Popery of late is so much in debate,}
\begin{scverse}
\begin{altverse}
\vleftofline{“}Since Popery of late is so much in debate,\\
And great strivings have been to restore it,\\
I cannot forbear openly to declare\\
That the ballad-makers are for it.”
\end{altverse}
\end{scverse}

This song attained some popularity, because others are found to the tune of
\textit{The Catholic Ballad}.

The following are the two ballads on the Spanish Armada; the first (with the
tune) as in the Harl. MS., and the second from \textit{Westminster Drollery}.

\musicinfo{Moderate time.}{}

\includemusic{chappellV1109.pdf}

\settowidth{\versewidth}{Don Pedroa  hight, [called] as good a knight}
\begin{dcverse}

\begin{altverse}
Spain, with Biscay and Portugal,\\
Toledo and Grenada;\\
All these did meet, and made a fleet,\\
And call’d it the Armada.
\end{altverse}

\begin{altverse}
Where they had got provision,\\
As mustard, pease, and bacon;\\
Some say two ships were full of whips,\\
But I think they were mistaken.
\end{altverse}

\begin{altverse}
There was a little man of Spain\\
That shot well in a gun-a,\\
Don Pedro\footnote{\textit{}
The person meant by Don Pedro was the Duke of
Medina Sidonia, commander of the Spanish fleet. His
name was not Pedro, but Alonzo \textit{Perez} di Guzman.}
  hight, [called] as good a knight\\
As the Knight of the Sun-a.
\end{altverse}

\begin{altverse}
King Philip made him admiral,\\
And charg’d him not to stay-a,\\
But to destroy both man and boy,\\
And then to run away-a.
\end{altverse}

\begin{altverse}
The King of Spain did fret amain,\\
And to do yet more harm-a;\\
He sent along, to make him strong,\\
The famous Prince of Parma.
\end{altverse}

\begin{altverse}
When they had sail’d along the seas,\\
And anchor’d upon Dover,\\
Our Englishmen did board them then,\\
And cast the Spaniards over.
\end{altverse}

\begin{altverse}
Our Queen was then at Tilbury,\\
What could you more desire-a?\\
For whose sweet sake Sir Francis Drake\\
Did set them all on fire-a.
\end{altverse}

\begin{altverse}
But let them look about themselyes,\\
For if they come again-a,\\
They shall be serv’d with that same sauce\\
As they were, I know when-a.
\end{altverse}

\end{dcverse}
\pagebreak
%213

“An old song of the Spanish Armado,” called, also, in \textit{Pills to purge Melancholy},
“Sir Francis Drake: or Eighty-Eight.” To the same tune. (The words
from \textit{Westminster Drollery}, 1672.)


\settowidth{\versewidth}{They brought two ships well fraught with whips,}
\begin{dcverse}\begin{altverse}
Some years of late, in eighty-eight,\\
As I do well remember;\\
It was, some say, the nineteenth of May,\\
And some say in September.
\end{altverse}

\begin{altverse}
The Spanish train, launch’d forth amain,\\
With many a fine bravado,\\
Their (as they thought, but it proved not)\\
Invincible Armado.
\end{altverse}

\begin{altverse}
There was a little man that dwelt in Spain,\\
Who shot well in a gun-a,\\
Don Pedro hight, as black a wight\\
As the Knight of the Sun-a.
\end{altverse}

\begin{altverse}
King Philip made him admiral,\\
And bid him not to stay-a,\\
But to destroy both man and boy,\\
And so to come away-a.
\end{altverse}

\begin{altverse}
Their navy was well victualled\\
With biscuit, pease, and bacon;\\
They brought two ships well fraught with whips,\\
But I think they were mistaken.
\end{altverse}

\begin{altverse}
Their men were young, munition strong,\\
And to do us more harm-a,\\
They thought it meet to join their fleet,\\
All with the Prince of Parma.
\end{altverse}

\begin{altverse}
They coasted round about our land,\\
And so came in to Dover;\\
But we had men, set on them then,\\
And threw the rascals over.
\end{altverse}

\begin{altverse}
The Queen was then at Tilbury,\\
What could we more desire-a,\\
And Sir Francis Drake, for her sweet sake,\\
Did set them all on fire-a.
\end{altverse}

\begin{altverse}
Then straight they fled by sea and land,\\
That one man kill’d three score-a;\\
And had not they all run away,\\
In truth he had kill’d more-a.
\end{altverse}

\begin{altverse}
Then let them neither brag nor boast.\\
But if they come again-a,\\
Let them take heed they do not speed,\\
As they did, you know when-a.
\end{altverse}
\end{dcverse}

\musictitle{Come, Live With Me, And Be My Love.}

This tune, which was discovered by Sir John Hawkins, “in a MS. as old as
Shakespeare’s time,” and printed in Steevens’ edition of Shakespeare, is also contained
in “The Second Booke of Ayres, some to sing and play to the Base-Violl
alone: others to be sung to the Lute and Base-Violl,” \&c., by W. Corkine,
fol.~1612.

In act iii., sc. 1, of \textit{The Merry Wives of Windsor}, 1602, Sir Hugh Evans sings
the following lines, which form part of the song:—
\settowidth{\versewidth}{Thou, in whose groves, by Dis above,}
\begin{scverse}
\vleftofline{“}To shallow rivers, to whose falls\\
Melodious birds sing madrigals;\\
There will we make our beds of roses,\\
And a thousand fragrant posies.”
\end{scverse}
In Marlow’s tragedy, \textit{The Jew of Malta}, written in or before 1591, he introduces
the first lines of the song in the following manner:—
\begin{scverse}
“Thou, in whose groves, by Dis above,\\
Shall live with me and be my love.”
\end{scverse}
In \textit{England's Helicon}, 1600, it is printed with the name “Chr. Marlow” as the
author. It is also attributed to Marlow in the following passage from Walton’s
\textit{Angler}, 1653:—“It was a handsome milkmaid, that had not attained so much
age and wisdom as to load her mind with any fears of many things that will never
be, as too many men often do; but she cast away all care and sung like a nightingale: 
her voice was good, and the ditty fitted for it: it was that smooth song
which was made by Kit. Marlow, now at least fifty years ago.”
\pagebreak
%214

On the other hand, it was first printed by W. Jaggard in “The passionate
Pilgrim and other sonnets by Mr. William Shakespeare,” in 1599; but Jaggard
is a very bad authority, for he included songs and sonnets by Griffin and Barnfield
in the same collection, and subsequently others by Heywood.

\textit{England’s Helicon} contains, also, “The Nimph’s reply to the Shepheard,”
beginning—
\settowidth{\versewidth}{And we will some new pleasures prove.}
\begin{scverse}
\vleftofline{“}If all the world and love were young,\\
And truth in every shepherd’s tongue;”
\end{scverse}
which is there subscribed “Ignoto,” but which Walton attributes to Sir Walter
Raleigh, “in his younger days;” and “Another of the same nature made since,”
commencing—
\begin{scverse}
\vleftofline{“}Come, live with me, and be my deere,\\
And we will revel all the yeere,”
\end{scverse}
with the same subscription.

Dr. Donne’s song, entitled “The Bait,” beginning—
\begin{scverse}
\vleftofline{“}Come, live with me, and be my love,\\
And we will some new pleasures prove.\\
Of golden sands and crystal brooks,\\
With silken lines and silver hooks,” \&c.
\end{scverse}
which, as Walton observes, he “made to shew the world that he could make soft
and smooth verses, when he thought smoothness worth his labour,” is also in
\textit{The Complete Angler}; and the three above quoted from \textit{England’s Helicon}, are
reprinted in Ritson’s \textit{English Songs and Ancient Songs}; and two in Percy’s
\textit{Reliques of Ancient Poetry}, \&c., \&c.

In \textit{Choice, Chance, and Change, or Conceits in their colours}, 4to., 1606,
Tidero, being invited to live with his friend, replies, “Why, how now? do you
take me for a woman, that you come upon me with a ballad of \textit{Come, live with me,
and be my love}?”

Nicholas Breton, in his \textit{Poste with a packet of Mad Letters}, 4to., 1637, says,
“You shall hear the old song that you were wont to like well of, sung by the
black brows with the cherry cheek, under the side of the pied cow, \textit{Come, live with
me, and be my love}, you know the rest.”

Sir Harris Nicholas, in his edition of Walton’s \textit{Angler}, quotes a song in imitation
of \textit{Come, live with me}, by Herrick, commencing—
\begin{scverse}
\vleftofline{“}Live, live with me, and thou shalt see;”
\end{scverse}
and Steevens remarks that the ballad appears to have furnished Milton with the
hint for the last lines of \textit{L’Allegro and Penseroso}.

From the following passage in \textit{The World’s Folly}, 1609, it appears that there
may have been an older name for the tune:—“But there sat he, hanging his
head, lifting up the eyes, and with a deep sigh, singing the ballad of \textit{Come, live
with me, and be my love}, to the tune of \textit{Adew, my deere}.”\footnote{\textit{}
A song in Harl. MSS. 2252, of the early part of Henry
the Eighth’s reign, “Upon the inconstancy of his mistress,” begins thus;—
\settowidth{\versewidth}{Mornyng, mornyng, thus may I sing,}
\begin{fnverse}
“Mornyng, mornyng, thus may I sing, \\
\vin\vin Adew, my dere, adew.”
\end{fnverse}
It is reprinted in Ritson's \textit{Ancient Songs} (p. 98), but the
metre differs from that of \textit{Come, live with me}, and with
out repeating words, could not have been sung to	the
same air.
}

In Deloney’s \textit{Strange Histories}, 1607, is the ballad of “The Imprisonment of 
Queen Eleanor,” \&c., to the tune of \pagebreak \textit{Come, live with me, and be my love}, but it has
%215
six lines in each stanza; and “The woefull lamentation of Jane Shore,” beginning,
“If Rosamond that was so fair” (copies of which are in the Pepys, Bagford, and
Roxburghe Collections), “to the tune of \textit{Live with me}” has four lines and a
burden of two— 
\settowidth{\versewidth}{Then maids and wives in time amend,}
\begin{scverse}
\vleftofline{“}Then maids and wives in time amend,\\
For love and beauty will have end.”
\end{scverse}
From this it appears that either the half of the tune was repeated, or that there
were two airs to which it was sung. In \textit{Westminster Drollery}, 1671 and 1674, a
parody on \textit{Come, live with me}, is to the tune of \textit{My freedom is all my joy}. That
has also six lines, and the last is repeated.

Other ballads, like “A most sorrowful song, setting forth the miserable end of
Banister, who betrayed the Duke of Buckingham, his lord and master: to the tune
of \textit{Live with me};” and the Life and Death of the great Duke of Buckingham, who
came to an untimely end for consenting to the depositing of two gallant young
princes,” \&c.: to the tune of \textit{Shore’s Wife}, have, like \textit{Come, live with me}, only
four lines in each stanza. (See \textit{Crown Garland of Golden Roses}, 1612; and
Evans’ \textit{Old Ballads}, iii. 18 and 23.)

\musicinfo{Rather slow\hspace{2em} \footnote{\textit{}In Sir John Hawkins’ copy, this note is written an
octave lower, probably because taken from a lute arrangement,
in which the note, being repeated, was to be played
on a lower string. In the second bar of the melody, his
copy, if transposed into this key, would be B \underline{A} D, instead
of B \underline{C} D; which latter seems right by the analogy of that
and the other phrases, although the difference is not very
material.}}{}

\includemusic{chappellV1110.pdf}

\settowidth{\versewidth}{And see the shepherds feed their flocks,}
\begin{dcverse}There will we sit upon the rocks,\\
And see the shepherds feed their flocks,\\
By shallow rivers, to whose falls\\
Melodious birds sing madrigals.

There will I make thee beds of roses,\\
And twine a thousand fragrant posies;\\
A cap of flowers, and a kirtle,\\
Embroider’d all with leaves of myrtle.

A gown made of the finest wool,\\
Which from our pretty lambs we pull;\\
Slippers lined choicely for the cold,\\
With buckles of the purest gold.

A belt of straw and ivy buds,\\
With coral clasps and amber studs:\\
And if these pleasures may thee move,\\
Come, live with me, and be my love.
\end{dcverse}


\begin{scverse}The shepherd swains shall dance and sing,\\
For thy delight, each May morning;\\
If these delights thy mind may move,\\
Then live with me, and be my love.
\end{scverse}

\pagebreak
%216

\musictitle{Three Merry Men Be We.}

This is quoted in the same passage in\textit{ Twelfth Night} as \textit{Peg-a-Ramsey}. The tune
is contained in a MS. common-place book, in the handwriting of John Playford,
the publisher of \textit{The Dancing Master}, in the possession of the Hon. George
O’Callaghan.” The words are also in Peele’s \textit{The Old Wives’ Tale}, 1595 (Dyce,
i.~208), where it is sung instead of the' song proposed, \textit{O man in desperation}.

In the comedy of \textit{Laugh and lie down}, 1605, “He plaied such a song of the
\textit{Three Merry Men}.” In Fletcher’s \textit{The Bloody Brother}, the Cook, who is about to
be hung with two others, says:
\settowidth{\versewidth}{This hasty work was ne’er done well: give us so much time}
\begin{scverse}
\vleftofline{“}Good Master Sheriff, your leave too;\\
This hasty work was ne’er done well: \textit{give us so much time}\\
\textit{As but to sing our own ballads}, for we’ll trust no man,\\
Nor no tune but our own; ’twas done in ale too,\\
And therefore cannot be refus’d in justice:\\
\textit{Your penny-pot poets are such pelting thieves,\\
They ever hang men twice}.”
\end{scverse}
Each then sings a song, and they join in the chorus of—
\settowidth{\versewidth}{Three merry boys, and three merry boys,}
\begin{scverse}
\vleftofline{“}Three merry boys, and three merry boys,\\
And three merry boys are we,\\
As ever did sing in a hempen string\\
Under the gallow tree.”—\textit{Act iii., sc. 2, Dyce},--- x. 428.
\end{scverse}
“Three merry men be we” is also quoted in \textit{Westward Hoe}, by Dekker and
Webster, 1607; and in \textit{Ram Alley}, 1611.

\musicinfo{Moderate time and gaily.}{}

\smallskip

\includemusic{chappellV1111.pdf}

\musictitle{I Loathe That I Did Love.}

On the margin of a copy of the Earl of Surrey’s poems, in the possession of
Sir W. W. Wynne, some of the little airs to which his favorite songs were sung
are written in characters of the times. Dr. Nott printed them from that copy in
his edition of Surrey’s \textit{Songs and Sonnets},\scfootnote{\textit{}
The music was added after a portion of the edition had been circulated.}
 4to., 1814. From this the first tune
for “I loathe that I did love” is taken. The second is from a MS. containing
songs to the lute, in the British Museum (Addit. 4900), but it is more like the
regular composition of a musician than the former.
\pagebreak
%217



Three stanzas from the poem are sung by the grave-digger in \textit{Hamlet}; but
they are much corrupted, and in all probability designedly, to suit the character
of an illiterate clown. On the stage the grave-digger now sings them to the tune
of \textit{The Children in the Wood}.

In the \textit{Gorgious Gallery of Gallant Inventions}, 1578, “the lover complaineth
of his lady’s inconstancy; to the tune of \textit{I lothe that I did love},” therefore a tune
was formerly known by that name, and probably one of the two here printed.

The song will be found among the ballads that illustrate Shakespeare, in Percy’s
\textit{Reliques of Ancient Poetry}.

\includemusic{chappellV1112.pdf}

\pagebreak%218

\musictitle{Peg A Ramsey, Or Peggie Ramsey.}

In \textit{Twelfth Night}, act ii., sc. 3, Sir Toby says, “Malvolio’s a \textit{Peg-a-Ramsey},
and \textit{Three merry men be we}.” There are two tunes under the name of \textit{Peg-a-Ramsey},
and both as old as Shakespeare’s time. The first is called \textit{Peg-a-Ramsey}
in William Ballet’s Lute Book, and is given by Sir John Hawkins as the tune
quoted in \textit{Twelfth Night}. (See Steevens’ edition of Shakespeare.) He says,
“Peggy Ramsey is the name of some old song;” but, as usual, does not cite his
authority. It is mentioned as a dance tune by Nashe (see the passage quoted at
p. 116), and in \textit{The Shepheard’s Holiday}—


\settowidth{\versewidth}{And I am sure thou there shall find}
\begin{dcverse}
\vleftofline{“}Bounce it Mall, I hope thou will,\\
For I know that thou hast skill;\\
And I am sure thou there shall find\\
Measures store to please thy mind.\\
Roundelays—Irish hayes;\\
Cogs and Rongs, and \textit{Peggie Ramsy};\\
Spaniletto—The Venetto;\\
\textit{John come kiss me}—Wilson’s Fancy.\\
But of all there’s none so sprightly\\
To my ear, as \textit{Touch me lightly}.”\\
\vin\vin\vin \textit{Wit’s Recreations}, 1640.
\end{dcverse}

“Little Pegge of Ramsie” is one of the tunes in a manuscript by Dr. Bull, which
formed a part of Dr. Pepusch’s, and afterwards of Dr. Kitchener’s library. Ramsey,
in Huntingdonshire, was formerly an important town, and called “Ramsey the
rich,” before the destruction of its abbey.

Burton, in his \textit{Anatomy of Melancholy}, says, “So long as we are wooers, we
may kiss at our pleasure, nothing is so sweet, we are in heaven as we think; but
when we are once tied, and have lost our liberty., marriage is an hell. ‘\textit{Give me
my yellow hose again}:’ a mouse in a trap lives as merrily.”

“Give me my yellow hose” is the burden of a ballad called—
\settowidth{\versewidth}{A merry jest of John Tomson, and Jackaman his wife,}
\begin{scverse}
\vleftofline{“}A merry jest of John Tomson, and Jackaman his wife,\\
Whose jealousy was justly the cause of all their strife;”
\end{scverse}
to the tune of \textit{Pegge of Ramsey}; beginning thus—

\settowidth{\versewidth}{But now I am a married man}
\begin{dcverse}
\begin{altverse}
\vleftofline{“}When I was a bachelor\\
I led a merry life,\\
But now I am a married man\\
And troubled with a wife,

I cannot do as I have done,\\
Because I live in fear;\\
If I go but to Islington,\\
My wife is watching there.
\end{altverse}
\end{dcverse}

\settowidth{\versewidth}{For now my wife she watcheth me,}
\begin{scverse}
\begin{altverse}
\textit{Give me my yellow again}.\\
Give me my yellow hose,\\
For now my wife she watcheth me,\\
See yonder where she goes.”
\end{altverse}
\end{scverse}
It has been reprinted in Evans’ \textit{Old Ballads}, i. 187 (1810.)

In \textit{Wit and Mirth, or\textit{ Pills to purge Melancholy}} (1707, iii. 219, or 1719,
v. 139), there is a song called “\textit{Bonny Peggy Ramsey},” to the second tune,
which in earlier copies is called \textit{O London is a fine town}, and \textit{Watton Town’s End}.

The original song, “Oh! London is a fine town,” is probably no longer extant.
A ballad to be sung to the tune was written on the occasion of James the First’s
visit to Cambridge, in March, 1614—
\pagebreak
%219
\changefontsize{10.4}

\settowidth{\versewidth}{The Mayor and some few Aldermen}
\begin{scverse}\begin{altverse}
\vleftofline{“}Cambridge is a merry town,\\
And Oxford is another,\\
The King was welcome to the one,\\
And fared well at the other,” \&c.\\
\vin\vin\vin\vin See Hawkins’ \textit{Ignoramus}, xxxvi.
\end{altverse}
\end{scverse}
A second with the burden—
\begin{scverse}
\begin{altverse}
\vleftofline{“}London is a fine town,\\
Yet I their cases pity;\\
The Mayor and some few Aldermen\\
Have clean undone the city,”
\end{altverse}
\end{scverse}
will he found in the King’s Pamphlets, British Museum (fol. broadsides, vol. v.).
It begins, “Why kept your train-bands such a stir,” and is dated Aug. 13,1647.
(Reprinted in Wright’s \textit{Political Ballads}, for the Percy Society.)

In \textit{Le Prince d’Amour}, 12m., 1660, is a third, commencing thus:—
\settowidth{\versewidth}{Governed with scarlet gowns; give ear unto my ditty:}
\begin{scverse}\vleftofline{“}London is a fine town, and a brave city,\\
Governed with scarlet gowns; give ear unto my ditty:\\
And there is a Mayor, which Mayor he is a Lord,\\
That governeth the city by righteous record.\\
Upon Simon and Jude’s day their sails then up they hoist,\\
And then he goes to Westminster with all the galley foist.\\
\vin\vin\vin\vin London is a fine town,” \&c.
\end{scverse}
A fourth song beginning, “\textit{Oh!} London is a fine town,” will he found in \textit{Pills to
purge Melancholy}, 1707, ii. 40, or 1719, iv. 40; and in the same volume another
to the tune, beginning—
\settowidth{\versewidth}{As I came from Tottingham,}
\begin{dcverse}\begin{altverse}
\vleftofline{“}As I came from Tottingham,\\
Upon a market day,\\
There I met a bonny lass\\
Clothed all in gray.
\end{altverse}

\begin{altverse}
Her journey was to London\\
With buttermilk and whey,\\
\textit{To come down, a down,\\
To come down, down, a down-a}.”
\end{altverse}
\end{dcverse}
The burden, to this song suggests the possibility of its being the tune of a snatch
sung by Ophelia in \textit{Hamlet}—
\settowidth{\versewidth}{You must sing down, a down,}
\begin{scverse}\vleftofline{“}You must sing down, a down,\\
An you call him a down-a.”
\end{scverse}
One of D’Urfey’s “Scotch” Songs, called \textit{The Gowlin}, in his play of \textit{Trick for
Trick}, was also sung to this tune.

In \textit{The Dancing Master}, 1665 and after, it is called \textit{Watton Town’s End}; and
in the second part of \textit{Robin Goodfellow}, 1628, there is a song “to the tune of
\textit{Watton Town’s End},” beginning—
\begin{scverse}\vleftofline{“}It was a country lad,\\
That fashions strange would see,” \&c.
\end{scverse}
It is reprinted in Evans’\textit{Old Ballads} 1810, i. 200. Another entitled—
\begin{scverse}\vleftofline{“}\vleftofline{“}The common cries of London town,\\
Some go up street, some go down,”
\end{scverse}
is to the tune of \textit{Watton Townes End}, black-letter, 1662.

Many others will he found to these tunes, under their various names.

The following is a verse from the ballad quoted in Burton’s \textit{Anatomy of
Melancholy}. It consists of eighteen stanzas, each of eight lines, and a ditty of
four (“Give me my yellow hose again,” \&c.). See Evans’ \textit{Old Ballads}.
\pagebreak
%220

\musicinfo{Moderate time.}{}

\includemusic{chappellV1113.pdf}

There are slight differences in the copies of the tune called \textit{Watton Town's End}
in \textit{The Dancing Master}, and \textit{Oh! London is a fine town} in \textit{Pills to purge Melancholy},
and in \textit{The Beggars’ Opera}. The following is \textit{The Beggars’ Opera} version:—

\musicinfo{Lively.}{}

\includemusic{chappellV1114.pdf}

\pagebreak
%221

\musictitle{Light O’love.}

\textit{Light of Love} is so frequently mentioned by writers of the sixteenth century,
that it is much to be regretted that the words of the original song are still
undiscovered. When played slowly and with expression the air is beautiful. In
the collection of Mr. George Daniel, of Canonbury, is “A \textit{very proper} dittie: to
the tune \textit{Lightie Love};” which was printed in 1570. The original may not have
been quite so “proper,” if “Light o’Love” was used in a sense in which it was
occasionally employed, instead of its more poetical meaning:—
\settowidth{\versewidth}{One of your \textit{London Light o'Loves}, a right one.}
\begin{scverse}
\vleftofline{“}One of your \textit{London Light o'Loves}, a right one.\\
Come over in thin pumps, and half a petticoat.”\\
\attribution Fletcher’s \textit{Wild Goose Chase}, act iv., sc. 2.
\end{scverse}

Or in the passage quoted by Douce: “There be wealthy housewives and good
housekeepers that use no starch, but fair water; their linen is as white, and they
look more Christian-like in small ruffs than \textit{Light of Love} looks in her great
starched ruffs, look she never so high, with her eye-lids awry.”—\textit{The Glasse of
Man's Follie},~1615.

Shakespeare alludes twice to the tune. Firstly in \textit{The Two Gentlemen of Verona},
act i., sc. 2—

\begin{scverse}
\vleftofline{“\textit{Julia.} }Some love of yours hath writ to you in rhime.\\
\vleftofline{\textit{Lucetta.} }That I might sing it, madam, to a tune:\\
\textit{Give me a note:—your ladyship can set}.\\
\vleftofline{\textit{Jul. }}As little by such toys as may be possible:\\
Best sing it to the tune of \textit{Light o'Love}.\\
\vleftofline{\textit{Luc. }}It is too heavy for so light a tune.\\
\vleftofline{\textit{Jul. }}Heavy? belike it hath some \textit{burden} then.\\
\vleftofline{\textit{Luc. }}Ay; and melodious were it would you sing it.\\
\vleftofline{\textit{Jul. }}And why not you?\\
\vleftofline{\textit{Luc. }}I cannot reach so high.\\
\vleftofline{\textit{Jul. }}Let’s see your song:—How now, minion?\\
\vleftofline{\textit{Luc. }}Keep tune there still, so you will sing it out:\\
And yet, methinks, I do not like this tune.\\
\vleftofline{\textit{Jul.}} You do not?\\
\vleftofline{\textit{Luc.}} No, madam; ’tis too sharp.\\
\vleftofline{\textit{Jul.}} You, minion, are too saucy.\\
\vleftofline{\textit{Luc.}} Nay, now you are too flat,\\
And mar the concord with too harsh a \textit{descant}:\\
There wanteth but a \textit{mean} to fill your song.\\
\vleftofline{\textit{Jul.}} The mean is drown’d with your unruly \textit{base}.’’
\end{scverse}

I have quoted this passage \textit{in extenso} as bearing upon the state of music at the
time, beyond the mere mention of the tune. Firstly, when Lucetta says, “Give
me a note [to sing it to]: your ladyship can set” [a song to music,] it adds one
more to the many proofs of the superior cultivation of the science in those days.
We should not now readily attribute to ladies, even to those who are generally
considered to be well educated and \pagebreak accomplished, enough knowledge of
%222
harmony to enable them to set a song correctly to music, however agile their
fingers may be. Secondly—
\settowidth{\versewidth}{Heavy? belike it hath some burden then!}
\begin{scverse}
\vleftofline{“}It is too heavy for so light a tune,\\
Heavy? belike it hath some burden then!”
\end{scverse}

The burden of a song, in the old acceptation of the word, was the base, foot, or
under-song. It was sung throughout, and not merely at the end of the verse.
Burden is derived from \textit{bourdoun}, a drone base (French, \textit{bourdon}.)

\begin{scverse}
“This Sompnour bare to him a stiff burdoun,\\
Was never trompe of half so gret a soun.”—\textit{Chaucer}.
\end{scverse}

We find as early as 1250, that \textit{Somer is icumen in} was sung with a foot, or burden,
in two parts throughout ( “Sing cuckoo, sing cuckoo” ); and in the preceding
century Giraldus had noticed the peculiarity of the English in singing under-parts
to their songs.


That burden still bore the sense of an under-part or base, and not merely of a
ditty,\scfootnote
{“Ditties, they are the ends of old ballads.”—Rowley’s \textit{A Match at Midnight}, act iii., sc. 1.}
 see \textit{A Quest of Inquirie}, \&c., 4to., 1595, where it is compared to the music
of a tabor:—“Good, people, beware of wooers’ promises, they are like the musique
of a tabor and pipe: the pipe says golde, giftes, and many gay things; but performance
is moralized in the tabor, which \textit{bears the burden} of ‘I doubt it, I doubt it.’—
(\textit{British Bibliographer}, vol. i.) In Fletcher’s \textit{Humorous Lieutenant}, act v., sc. 2,
“H’as made a thousand rhymes, sir, and plays the \textit{burden} to 'em on a Jew’s-
trump” (\textit{Jeugd-tromp}, the Dutch for a child’s horn). So in \textit{Much Ado about
Nothing}, in the scene between Hero, Beatrice, and Margaret, the last says, “Clap
us into \textit{Light o'Love}, that goes without a burden” [there being no man or men
on the stage to sing one]. “Do you sing it and I’ll dance it.” \textit{Light o'Love}
was therefore strictly a \textit{ballet}, to he sung and danced.

In the interlude of \textit{The Four Elements}, about 1510, Ignorance says—
\settowidth{\versewidth}{\textit{But there is a bordon, thou must bear it},}
\begin{scverse}
\vleftofline{“}But if thou wilt have a song that is good,\\
I have one of Robin Hood,\\
The best that ever was made.\\
\vleftofline{\textit{Humanity}. }Then i’ fellowship, let us hear it.\\
\vleftofline{\textit{Ign}. }\textit{But there is a bordon, thou must bear it},\\
\textit{Or else it will not be}.\\
\vleftofline{\textit{Hum}. }Then begin and care not to \dots\\
Downe, downe, downe, \&c.\\
\vleftofline{\textit{Ign}. }Robin Hood in Barnsdale stood,” \&c.
\end{scverse}

%\setlength{\DFNcolumnwidth}{0.5\textwidth}
%\addtolength{\DFNcolumnwidth}{-0.5\columnsep}

Here Humanity starts with the burden, giving the key for the other to sing in.
So in old manuscripts, the burden is generally found at the head of the song, and
not at the end of the first verse.

Many of these burdens were short proverbial expressions, such as— .
\begin{scverse}
“ ‘Tis merry in hall when beards wag all;”
\end{scverse}
which is mentioned as the “under-song or holding” of one in \textit{The Serving-man's
Comfort}, 1598, and the line quoted by Adam Davy, in his \textit{Life of Alexander}, as
early as about 1312. Peele, in his \pagebreak \textit{Edward I}., speaks of it as “the old
%223
English proverb but he uses the word “proverb” also in the sense of song, for
in his \textit{Old Wives’ Tale}, 1595, Antick says, “Let us rehearse the old proverb—
\settowidth{\versewidth}{‘Three merry men and three merry men,}
\begin{scverse}
‘Three merry men and three merry men,\\
And three merry men be we,’” \&c.
\end{scverse}
Shakespeare puts the following four lines into the mouth of Justice Silence when
in his cups:—
\begin{scverse}
“Be merry, be merry, my wife has all,\\
For women are shrews, both short and tall;\\
\textit{’Tis merry in hall, when beards wag all}.\\
\vin And welcome merry Shrovetide.”
\end{scverse}
See also Ben Jonson, v. 235, and note; and vii. 273, Gifford’s edit.

Other burdens were mere nonsense words that went glibly off the tongue, giving
the accent of the music, such as \textit{hey nonny, nonny no; hey derry down}, \&c. The
“foot” of the first song in \textit{The pleasant Comedy of Patient Grissil} is—
\begin{scverse}
“Work apace, apace, apace, apace,\\
Honest labour bears a lovely face;\\
Then hey noney, noney; hey noney, noney.”
\end{scverse}
I am aware that “Hey down, down, derry down,” has been said to be “a modern
version of ‘Hai down, ir deri danno,’ the burden of an old song of the Druids,
signifying, ‘Come, let us hasten to the oaken grove’ (Jones’ \textit{Welsh Bards}, i. 128);
but I believe this to be mere conjecture, and that it would now be impossible to
prove that the Druids had such a song.

The last comment I have to make upon the passage from Shakespeare is on the
word \textit{mean}. The mean in music was the intermediate part between the tenor and
treble; not the tenor itself, as explained by Steevens. Descant has already been
explained at p. 15.

Reverting to \textit{Light o’Love}: it is also quoted as a tune by Fletcher in \textit{The Two
Noble Kinsmen}, The air was found by Sir J. Hawkins in an “ancient manuscript;” 
it is also contained in William Ballet’s MS. Lute Book, and in \textit{Musick’s
Delight on the Cithren}, 1666.

In the volume of transcripts made by Sir John Hawkins there is a tune entitled
\textit{Fair Maid are you walking}, the first four bars of which are identical with \textit{Light
o'Love}; and in the Music School, Oxford, one of the manuscripts presented by
Bishop Fell, with a date 1620, has \textit{Light o'Love} under the name of \textit{Sicke and sicke
and very sicke}; but this must be a mistake, as that ballad could not be sung to it.
See \textit{Captain Car} in Ritson’s \textit{Ancient Songs}, 1790, p. 139.

In \textit{A Gorgious Gallery of Gallant Inventions}, 1578, the lover exhorteth his
lady to be constant: to the tune of \textit{Attend thee, go play thee};\footnote{\textit{}
“Attend thee, go play thee,” is a song in A Hand\-efull
of Pleasant Delites, 1584, and is also the tune of one sung
by Wantonness in the interlude of \textit{The Marriage of Wit
and, Wisdom}. See Shakespeare Society’s Reprint, p. 20.}
 “and begins with
the line, “Not \textit{Light o'Love}, lady.” The ballad, “The Banishment of Lord Maltravers
and Sir Thomas Gurney,” in Deloney’s \textit{Strange Histories}, \&c., 1607, and of
“A song of the wooing of Queen Catherine by Owen Tudor, a young gentleman
of Wales” are also to the tune of \textit{Light o'Love}. See \textit{Old Ballads}, 1727, iii. 32;
or Evans, ii. 356.

The following is the ballad by Leonard Gybson, a copy of which is in Mr.
George Daniel’s Collection.
\pagebreak
%224

\musictitle{A Very Proper Dittie: To The Tune Of Lightie Love.}

\vspace{-\baselineskip}

\settowidth{\versewidth}{Leave lightie love, ladies, for fear ofyll name:}
\begin{scverse}\vleftofline{“}Leave lightie love, ladies, for fear of yll name:\\
And true love embrace ye, to purchase your fame.”
\end{scverse}

\backskip{1}

\musicinfo{Very slow and smoothly}{}

\smallskip

\includemusic{chappellV1115.pdf}

\vspace{-1.5\baselineskip}

\begin{dcverse}\footnotesize
Deceit is not dainty, it comes at each dish;\\
Fraud goes a fishing with friendly looks;\\
Through friendship is spoiled, the silly poor fish\\
That hover and shower upon your false hooks,\\
With bait you lay wait, to catch here and there,\\
Which causeth poor fishes their freedom to lose;\\
Then lout ye, and flout ye;—whereby doth appear,\\
Your lighty love, ladies, still cloaked with glose.

With Dian so chaste you seem’d to compare,\\
When Helens you be, and hang on her train;\\
Methinks faithful Thisbes be now very rare,\\
But one Cleopatra, I doubt, doth remain.\\
You wink, and you twink, until Cupid have caught,\\
And forceth through flames your lovers to sue:\\
Your lighty love, ladies, too dear they have bought, \\
When nothing will move you their causes to rue.

I speak not for spite, nor do I disdain\\
Your beauty, fair ladies, in any respect;\\
But one’s ingratitude doth me constrain,\\
As child hurt with fire, the flame to neglect.\\
For, proving in loving, I find by good trial,\\
When Beauty had brought me unto her beck,\\
She staying, not weighing, but making denial,\\
And shewing her lighty love, gave me the check.

Thus fraud for friendship did lodge in her breast;\\
Such are most women, that when they espy\\
Their lovers inflamed, with sorrows opprest,\\
They stand then with Cupid against their reply.\\
They taunt, and they vaunt, they smile when they view\\
How Cupid hath caught them under his train;\\
But warned, discerned, the proof is most true,\\
That lighty love, ladies, amongst you does reign.

Ye men that are subject to Cupid his stroke,\\
And therein seem now to have your delight,\\
Think, when you see bait, there is hidden a hook,\\
Which surely will have you, if that you do bite.\\
Such wiles, and such guiles by women are wraught,\\
That half of their mischiefs men cannot prevent;\\
When they are most pleasant, unto your thought,\\
Then nothing but lighty love is their intent.

Consider that poison doth lurk often time\\
In shape of sugar, to put some to pain;\\
And fair wordès painted, as dames can define,\\
The old proverb saith, doth make some fools fain.\\
Be wise and precise, take warning by me,\\
Trust not the crocodile, lest you do rue;\\
To women’s fair words do never agree,\\
For all is but lighty love;—this is most true.
\end{dcverse}
\pagebreak
%225

\begin{dcverse}\footnotesize
I touch no such ladies as true love embrace,\\
But such as to lighty love daily apply;\\
And none will be grieved, in this kind of case,\\
Save such as are minded true love to deny.\\
Yet friendly and kindly I shew you my mind:\\
Fair ladies, I wish you to use it no more;\\
But say what you list, thus I have defin’d\\
That lighty love, ladies, you ought to abhor.

To trust women’s words, in any respect,\\
The danger by me right well it is seen;\\
And Love and his laws, who would not neglect,\\
The trial whereof hath most perilous been?\\
Pretending, the ending, if I have offended,\\
I crave of you, ladies, an answer again:\\
Amend, and what’s said shall soon be amended,\\
If case that your light love no longer do reign.
\end{dcverse}

\backskip{2}

\musictitle{When That I Was A Little Tiny Boy.}

The Fool’s song which forms the Epilogue to \textit{Twelfth Night} is still sung on the
stage to this tune. It has no other authority than theatrical tradition. A song
of the same description, and with the same burden, is sung by the Fool in \textit{King
Lear}, act iii., sc.~2—

\vspace{-0.5\baselineskip}

\begin{scverse}\begin{altverse}
“He that has a little tiny wit,\\
\textit{With a heigh ho! the wind and the rain},\\
Must make content with his fortunes fit,\\
\textit{For the rain it raineth every day}.”
\end{altverse}
\end{scverse}

\vspace{-0.5\baselineskip}

The following is the song in \textit{Twelfth Night}:—

\musicinfo{In moderate time.}{}

\smallskip

\includemusic{chappellV1116.pdf}

\backskip{2}

\indentpattern{0404}
\begin{dcverse}\begin{patverse}
But when I came to man’s estate,\\
With a heigh ho! \&c.,\\
’Gainst knaves and thieves men shut their gate,\\
For the rain, \&c.
\end{patverse}

\begin{patverse}
But when I came, alas! to wive,\\
With a heigh ho! \&c.,\\
By swaggering I could never thrive,\\
For the rain, \&c.
\end{patverse}

\begin{patverse}
But when I came unto my bed,\\
With a heigh ho! \&c.,\\
With toss-pots still I’d drunken head,\\
For the rain, \&c.
\end{patverse}

\indentpattern{0101}
\begin{patverse}
A great while ago the world begun,\\
With a heigh ho! the wind and the rain;\\
But that is all one, our play is done,\\
And we’ll strive to please you every day.
\end{patverse}
\end{dcverse}
\pagebreak
%226

\musictitle{Sick, Sick, And Very Sick.}

This tune is contained in Anthony Holborne’s \textit{Cittharn Schoole}, 4to., 1597, and
in one of the Lute MSS. in the Public Library, Cambridge. (D. d. iv. 23.) In
\textit{Much Ado about Nothing}, Hero says, “Why, how now! do you speak in the \textit{sick
tune}?” and Beatrice answers, “I am out of all other tune, methinks.” In
Nashe’s \textit{Summer’s last Will and Testament}, Harvest says, “My mates and fellows,
sing no more \textit{Merry, merry}, but weep out a lamentable \textit{Hooky, hooky}, and let your
sickles cry—
\settowidth{\versewidth}{For Harvest, your master, is}
\begin{scverse}
\begin{altverse}
Sick, sick, and very sick,\\
And sick and for the time;\\
For Harvest, your master, is\\
Abus’d without reason or rhyme.”
\end{altverse}
\end{scverse}

On 24th March, 1578, Richard Jones, had licensed to him “a ballad intituled
\textit{Sick, sick}, \&c., and on the following 19th June, “A new songe, intituled—
\settowidth{\versewidth}{For grief to see this wicked world, that will not mend, I fear.”}
\begin{scverse}
\textit{Sick, sick, in grave I would I were}.\\
For grief to see this wicked world, that will not mend, I fear.”
\end{scverse}
This was probably a moralization of the former.

In the \textit{Harleian Miscellany}, 4to, 10. 272, is “A new ballad, declaring the
dangerous shooting of the gun at the court (1578), to the tune of \textit{Sicke and sicke};
commencing—
\settowidth{\versewidth}{The seventeenth day of July last,}
\begin{dcverse}\settowidth{\versewidth}{And had the watermen to row,}
\begin{altverse}
\vleftofline{“}The seventeenth day of July last,\\
At evening toward night,\\
Our noble Queen Elizabeth\\
Took barge for her delight;\\
And had the watermen to row,\\
Her pleasure she might take,
\end{altverse}

\settowidth{\versewidth}{To think upon the gun was shot}
\indentpattern{012323}
\begin{patverse}
About the river to and fro,\\
As much as they could make.\\
Weep, weep, still I weep,\\
And shall do till I die,\\
To think upon the gun was shot\\
At court so dangerously.”
\end{patverse}
\end{dcverse}

The ballad from which the tune derives its name is probably that printed in
Ritson’s \textit{Ancient Songs}, (1793, p. 139) from a manuscript in the Cotton Library
(Vespasian, A 25), and entitled \textit{Captain Car}. The event which gave rise to it
occurred in the year 1571. The first stanza is here printed to the tune:—

\includemusic{chappellV1117.pdf}

\pagebreak
%227
\changefontsize{10.8}

\musictitle{To-Morrow Is St. Valentine’s Day.}

This is one of Ophelia’s songs in \textit{Hamlet}. It is found in several of the ballad
operas, such as \textit{The Cobblers’ Opera} (1729), \textit{The Quakers’ Opera} (1728), \&c.,
under this name. In \textit{Pills to purge Melancholy} (1707, ii. 44) it is printed to a
song in \textit{Heywood’s Rape of Lucrece}, beginning, “Arise, arise, my juggy, my
puggy.” Other versions will be found under the names of “Who list to lead
a soldier’s life,” and “Lord Thomas and Fair Ellinor.” See pages 144 and 145.

\musicinfo{Cheerfully.}{}

\includemusic{chappellV1118.pdf}

\musictitle{Green Sleeves.}

\textit{Green Sleeves}, or \textit{Which nobody can deny}, has been a favorite tune, from the
time of Elizabeth to the present day; and is still frequently to be heard in the
streets of London to songs with the old burden, “Which nobody ean deny.” It
will also be recognised as the air of \textit{Christmas comes but once a year}, and many
another merry~ditty.

“And set our credits to the tune of \textit{Greene Sleeves}.”—\textit{The Loyal Subject}, by
Beaumont and Fletcher.

\textit{Falstaff}.—“Let the sky rain potatoes! let it thunder to the tune of \textit{Green Sleeves},
hail kissing comfits, and snow eringoes, let there come a tempest of provocation, I will
shelter me here.” (\textit{Embracing her}.)—\textit{Merry Wives of Windsor}, act v., sc. 5.

“\textit{Mrs. Ford}.—“I shall think the worse of fat men, as long as I have an eye to
make difference of men’s liking. And yet he would not swear; praised women’s
modesty; and gave such orderly and well-behaved reproof to all uncomeliness, that
I would have sworn his disposition would have gone to the truth of his words: but
they do no more adhere and keep pace together, than the Hundredth Psalm to the
tune of \textit{Green Sleeves}.”—\textit{Merry Wives of Windsor}, act ii., sc. 1.

The earliest mention of the ballad of \textit{Green Sleeves} in the Registers of the
Stationers’ Company is in September, 1580, when Richard Jones had licensed to
him, “A new Northern Dittye of the \textit{Lady Greene Sleeves}.” The date of the
entry, however, is not always the date of the ballad; and this had evidently
attained some popularity before that time, \pagebreak because on the same day Edward
%228
White had a license to print, “A ballad, being the Ladie Greene Sleeves \textit{Answere}
to Donkyn his frende.” Also Edward Guilpin in his \textit{Skialethia, or a Shadow of
Truth}, 1598, says:
\settowidth{\versewidth}{“Yet like th’ olde ballad of the \textit{Lord of Lorne},}
\begin{scverse}
\vleftofline{“}Yet like th’ olde ballad of the \textit{Lord of Lorne},\\
Whose last line\footnote{\textit{}
The last lines of the Lord of Lorne are—
\settowidth{\versewidth}{Let Rebels therefore warned be,}
\begin{fnverse}
\begin{altverse}
\vleftofline{“}Let Rebels therefore warned be,\\
How mischief once they do pretend;

For God may suffer for a time,\\
But will disclose it at the end.”
\end{altverse}
\end{fnverse}

Perhaps Guilpin may mean that this formed part of an
older balled.}
 in King Harries days was borne.”
\end{scverse}

As the ballad of \textit{The Lord of Lorne and the False Steward}, which was entered on
the 6th October, 1580, was sung to the tune of \textit{Green Sleeves}, it would appear that
\textit{Green Sleeves} must be a tune of Henry’s reign. Copies of \textit{The Lord of Lorne} are in
the Pepys Collection (i. 494), and the Roxburghe (i. 222).

Within twelve days of the first entry of \textit{Green Sleeves} it was converted to a
pious use, and we have, “\textit{Greene Sleves} moralised to the Scripture, declaring the
manifold benefites and blessings of God bestowed on sinful man;” and on the
fifteenth day Edward White had “tollerated unto him by Mr. Watkins, a
ballad intituled Greene Sleeves and Countenance, in Countenance is Greene
Sleeves.” By the expression “tolerated” instead of “licensed,” we may infer
it to have been of questionable propriety.

Great, therefore, was the popularity of the ballad immediately after its publication, 
and this may he attributed rather to the merry swing of the tune, than to the
words, which are neither remarkable for novelty of subject, nor for its treatment.

An attempt was speedily made to improve upon them, or to supply others of
more attractive character, for in December of the same year, Jones, the original
publisher, had “tolerated to him A merry newe Northern Songe of \textit{Greene
Sleeves},” beginning, \textit{The bonniest lass in all the land}. This was probably the ballad
that excited William Elderton to write his “Reprehension against Greene Sleeves”
in the following February, for there appears nothing in the original song to have
caused it. The seventh entry within the year was on the 24th of August, 1581,
when Edward White had licensed “a ballad intituled—
\settowidth{\versewidth}{Greene Sleeves is worne awaie,}
\begin{scverse}
\vleftofline{“}Greene Sleeves is worne awaie,\\
Yellow Sleeves come to decaie.\\
Blaeke Sleeves I holde in despite,\\
But White Sleeves is my delight.”
\end{scverse}

Nashe, speaking of Barnes’ \textit{Divine Centurie of Sonets}, says they are “such
another device as the goodly ballet of John Careless, or the song of Green Sleeves
Moralized.” Fletcher says, “And, by my Lady Greensleeves, am I grown so
tame after all my triumphs?” and Dr. Rainoldes, in his \textit{Overthrow of Stage
Plays}, 1599, says, “Now if this were lawfully done because he did it, then
William, Bishop of Ely, who, to save his honour and wealth, became a \textit{Green
Sleeves}, going in women’s raiment from Dover Castle to the sea-side, did therein
like a man;—although the women of Dover, when they found it out, by plucking
down his muffler and seeing his new shaven heard, called him a monster for it.”

In Mr. Payne Collier’s Collection, and in that of the Society of Antiquaries,
are copies of “A Warning to false Traitors, by example of fourteen; whereof six 
were executed in divers places neere about \pagebreak London, and two near Braintford, the
%229
28th day of August, 1588; also at Tyborne were executed the 30th day six;
viz., five men and one woman: to the tune of \textit{Green Sleeves},” beginning—
\settowidth{\versewidth}{To hurt our Queen in treacherous wise,}
\indentpattern{00010001}
\begin{scverse}
\begin{patverse}
\vleftofline{“}You traitors all that do devise\\
To hurt our Queen in treacherous wise,\\
And in your hearts do still surmise\\
Which way to hurt our England;\\
Consider what the end will be\\
Of traitors all in their degree:\\
Hanging is still their destiny\\
That trouble the peace of England.”
\end{patverse}
\end{scverse}

The conspirators were treated with very little consideration by the ballad-monger
in having their exit chaunted to a merry tune, instead of the usual
lamentation, to the hanging-tune of \textit{Fortune my foe}.

Elderton’s ballad, \textit{The King of Scots and Andrew Brown}, was to be sung to
the tune of \textit{Mill-field}, or else to \textit{Green Sleeves} (see p. 185), but the measure suits
the former and not the latter. However, his “New Yorkshire Song, intituled—
\settowidth{\versewidth}{For merry pastime and companie,}
\begin{scverse}
\vleftofline{“}Yorke, Yorke, for my monie,\\
Of all the cities that ever I see.\\
For merry pastime and companie,\\
\vin Except the cittie of London;”
\end{scverse}
which is dated “from Yorke, by W. E., and imprinted at London by Richard
Jones,” in 1584, goes so trippingly to \textit{Green Sleeves}, that, although no tune is
mentioned on the title, I feel but little doubt of its having been intended for that
air. It was written during the height of its popularity, and not long after his
own “Reprehension.”

The song of \textit{York for my money} is on a match at archery between the Yorkshire
and the Cumberland men, backed by the Earls of Essex and Cumberland,
which Elderton went to see, and was delighted with the city and with his
reception; especially by the hospitality, of Alderman Maltby of York.

Copies will be found in the Roxburghe Collection, i. 1, and Evans’ \textit{Old Ballads},
i. 20,. It begins, “As I come thorow the North countrey,” and is refered to in
Heywood’s \textit{King Edward IV}., 1600.

In Mr. Payne Collier’s \textit{Old Ballads}, printed for the Percy Society, there is one
of Queen Elizaheth at Tilbury Fort (written shortly anterior to the destruction of
the Spanish Armada) to the tune of \textit{Triumph and Joy}. The name of the air is
probably derived from a ballad which was entered on the Stationers’ books in
1581, of “The Triumpe shewed before the Queene and the French Embassadors,”
who preceded the arrival of the Duke of Anjou, and for whose entertainment
jousts and triumphs were held. The tune for this ballad is not named in the
entry at Stationers’ Hall, but if a copy should be found, I imagine it will prove
also to have been written to \textit{Green Sleeves}, from the metre, and the date
coinciding with the period of its great popularity.

Richard Jones, to whom \textit{Green Sleeves} was first licensed, was also the printer
of \textit{A Handefull of Pleasant Delites}, 1584, in which a copy of the ballad will be
found. Also in Ellis’ \textit{Specimens}, \pagebreak ii. 394, (1803). A few verses are subjoined,
%230
as affording an insight into the dress and manners of an age with which we cannot
be too well acquainted.

The tune is contained in several of Dowland’s lute manuscripts; in William
Ballet’s Lute Book; in Sir John Hawkins’ transcripts of virginal music; in \textit{The
Dancing Master}; \textit{The Beggar's Opera}; and in many other books.

As the second part differs in the oldest copies, from others of later date, both
versions are subjoined.

The first is from William Ballet’s Lute Book compared with another in Sir
John Hawkins’ transcripts of virginal music; both having the older second part.

\musicinfo{Smoothly and in moderate time.}{Tune of Green Sleeves. Oldest Copy.}

\medskip

\includemusic{chappellV1119.pdf}

\indentpattern{01012}
\settowidth{\versewidth}{I kept thee booth at board and bed,}
\begin{dcverse}\begin{patverse}
I have been ready at your hand\\
To grant whatever you would crave,\\
I have both waged life and land,\\
Your love and good-will for to have.\\
Greensleeves was all my joy, \&c.
\end{patverse}

\begin{patverse}
I bought thee kerchers to thy head,\\
That were wrought fine and gallantly,\\
I kept thee booth at board and bed,\\
Which cost my purse well favoredly.\\
Greensleeves was all my joy, \&c.
\end{patverse}
\end{dcverse}
\pagebreak
%231

\indentpattern{01012}
\settowidth{\versewidth}{I bought thee petticoats of the best,}
\begin{dcverse}\begin{patverse}
I bought thee petticoats of the best,\\
The cloth so fine as might be;\\
I gave thee jewels for thy chest,\\
And all this cost I spent on thee.\\
Greensleeves was all my joy, \&c.
\end{patverse}

\begin{patverse}
Thy smock of silk, both fair and white,\\
With gold embroidered gargeously;\\
Thy petticoat of sendal right, [thin silk]\\
And these I bought thee gladly.\\
Greensleeves was all my joy, \&c.
\end{patverse}
\end{dcverse}
\noindent He then describes her girdle of gold, her purse, the crimson stockings all of silk,
the pumps as white as milk, the gown of grassy green, the satin sleeves, the
gold-fringed garters; all of which he gave her, together with his gayest gelding,
and his men decked all in green to wait upon her:
\settowidth{\versewidth}{Thy foot might not once touch the ground.}
\begin{scverse}\begin{patverse}
They set thee up, they took thee down,\\
They serv’d thee with humility;\\
Thy foot might not once touch the ground.\\
And yet thou wouldst not love me.\\
Greensleeves was all my joy, \&c.
\end{patverse}
\end{scverse}
She could desire no earthly thing without being gratified:

\begin{dcverse}\begin{patverse}
Well I will pray to God on high,\\
That thou my constancy mayst see,\\
And that yet once before I die\\
Thou wilt vouchsafe to love me.\\
Greensleeves was all my joy, \&c.
\end{patverse}

\begin{patverse}
Greensleeves, now farewell! adieu!\\
God I pray to prosper thee!\\
For I am still thy lover true,\\
Come once again and love me.\\
Greensleeves was all my joy, \&c.
\end{patverse}
\end{dcverse}


At the Revolution \textit{Green Sleeves} became one of the party tunes of the Cavaliers;
and in the “Collection of Loyal Songs written against the Rump Parliament,”
there are no less than fourteen to be sung to it. It is sometimes referred to under
the name of \textit{The Blacksmith}, from a song (in the Roxburghe Collection, i. 250)
to the tune of \textit{Green Sleeves}, beginning—

\begin{scverse}\vleftofline{“}Of all the trades that ever I see\\
There is none with the blacksmith’s compared may be,\\
For with so many several tools works he,\\
\vin \textit{Which nobody can deny”}
\end{scverse}

Pepys, in his diary, 22nd April, 1660, says that, after playing at nine-pins,
“my lord fell to singing a song upon the Rump, to the tune of \textit{The Blacksmith}.”

It was also called \textit{The Brewer}, or \textit{Old Noll, the Brewer of Huntingdon}, from a
satirical song about Oliver Cromwell, which is to be found in \textit{The Antidote to
Melancholy}, 1661, entitled “The Brewer, a ballad made in the year 1657, to the
tune of \textit{The Blacksmith};” also in \textit{Wit and Drollery, Jovial Poems}, 1661.

In \textit{The Dancing Master}, 1686, the tune first appears under the name of \textit{Green
Sleeves and Pudding Pies}; and in some of the latest editions it is called \textit{Green
Sleeves and Yellow Lace}. Percy says, “It is a received tradition in Scotland that
\textit{Green Sleeves and Pudding Pies} was designed to ridicule the Popish clergy,” but
the tradition most probably refers to a song of James the Second’s time called
\textit{At Rome there is a terrible rout},\footnote{\textit{}
This is entitled “Father Peters’ Policy discovered; or
the Prince of Wales proved a Popish Perkin.” London:
printed for R. M., ten stanzas, of which the following is
the first:—
\medskip
\settowidth{\versewidth}{Because the birth of the babe’s come out,}
\indentpattern{0001}
\begin{fnverse}
\begin{patverse}
\vleftofline{“}In Rome there is a most fearful rout;\\
And what do you think it is about?\\
Because the birth of the babe’s come out,\\
Sing Lullaby Baby, by, by, by.”
\end{patverse}
\end{fnverse}}
 which was sung to the tune, and attained some
popularity, since in the ballad-opera of \textit{Silvia}, or \textit{The Country Burial}, 1731,
it appears under that name. Boswell, in his Journal, 8vo., 785, p. 319, prints
the following Jacobite song:—
\origpage{}%232

\indentpattern{0001}
\begin{dcverse}\settowidth{\versewidth}{And I’ll be with her before she rise,}
\begin{patverse}
\vleftofline{“}Green Sleeves and Pudding Pies,\\
Tell me where my mistress lies,\\
And I’ll be with her before she rise,\\
Fiddle and aw together.
\end{patverse}

\settowidth{\versewidth}{And may our King come home with speed,}
\begin{patverse}
May our affairs abroad succeed,\\
And may our King come home with speed,\\
And all Pretenders shake for speed,\\
And let his health go round.
\end{patverse}
\end{dcverse}
\settowidth{\versewidth}{Let all Pretenders shake for dread,}
\begin{scverse}To all our injured friends in need;\\
This side and beyond the Tweed,\\
Let all Pretenders shake for dread,\\
\vin And let his health go round.”
\end{scverse}

There is no apparent connection between the subject of the first and that of the
remaining stanzas; and although the first may have been the burden of an older
song, it bears no indication of having refered to the clergy of any denomination.

There is scarcely a collection of old English songs in which at least one may
not be found to the tune of \textit{Green Sleeves}. In the West of England it is still
sung at harvest-homes to a song beginning, “A pie sat on a pear-tree top;” and
at the Maypole still remaining at Ansty, near Blandford, the villagers still dance
annually round it to this tune.

The following “Carol for New Year’s Day, to the tune of \textit{Green Sleeves},” is
from a black-letter collection printed in 1642, of which the only copy I have seen
is in the Ashmolean Library, Oxford.

\indentpattern{00010001}
\settowidth{\versewidth}{And now with new year’s gifts each friend}
\begin{dcverse}\begin{patverse}
The old year now away is fled,\\
The new year it is entered;\\
Then let us now our sins down tread,\\
And joyfully all appear.\\
Let’s merry be this holiday,\\
And let us run with sport and play,\\
Hang sorrow, let’s cast care away—\\
God send you a happy new year.
\end{patverse}

\begin{patverse}
And now with new year’s gifts each friend\\
Unto each other they do send;\\
God grant we may our lives amend,\\
And that the truth may appear.\\
Now like the snake cast off your skin\\
Of evil thoughts and wicked sin,\\
And to amend this new year begin—\\
God send us a merry new year.
\end{patverse}

\begin{patverse}
And now let all the company\\
In friendly manner all agree,\\
For we are here welcome all may see\\
Unto this jolly good cheer.\\
\columnbreak
I thank my master and my dame,\\
The which are founders of the same,\\
To eat to drink now is no shame—\\
God send us a merry new year.
\end{patverse}

\begin{patverse}
Come lads and lasses every one.\\
Jack, Tom, Dick, Bess, Mary, and Joan,\\
Let’s cut the meat unto the bone,\\
For welcome you need not fear.\\
And here for good liquor we shall not lack.\\
It will whet my brains and strengthen my back,\\
This jolly good cheer it must go to wrack—\\
God send us a merry new year.
\end{patverse}

\begin{patverse}
Come, give us more liquor when I do call,\\
I’ll drink to each one in this hall,\\
I hope that so loud I must not bawl,\\
But unto me lend an ear.\\
Good fortune to my master send,\\
And to my dame which is our friend,\\
God bless us all, and so I end—\\
And God send us a happy new year.
\end{patverse}
\end{dcverse}

The following version of the tune, from \textit{The Beggars' Opera}, 1728, is that
now best known. I have not found any lute or virginal copy which had this
second part. The earliest authority for it is \textit{The Dancing Master}, 1686, and it
may have been altered to suit the violin, as the older second part is rather low,
and less effective, for the instrument.
\pagebreak
%233

I have selected a few lines from a political song called \textit{The Trimmer}, to print
with this copy, because it has the burden, “Which nobody can deny.” It is one
of the many songs to the tune in \textit{Pills to purge Melancholy}.

\musicinfo{Boldly.}{Tune of Green Sleeves. Later copy.}

\medskip

\includemusic{chappellV1120.pdf}

\musictitle{My Robin Is To The Greenwood Gone; Or, Bonny Sweet Robin.}

This is contained in Anthony Holborne’s \textit{Cittharn Schoole}, 1597; in Queen
Elizabeth’s Virginal Book; in William Ballet’s Lute Book; and in many other
manuscripts and printed books.

There are two copies in William Ballet’s Lute Book, and the second is entitled
“Robin \textit{Hood} is to the greenwood gone;” it is, therefore, probably the tune of a
ballad of Robin Hood, now lost.

Ophelia sings a line of it in \textit{Hamlet}—
\settowidth{\versewidth}{“For bonny sweet Robin is all my joy;”}
\begin{scverse}
“For bonny sweet Robin is all my joy;”
\end{scverse}
and in Fletcher’s \textit{Two Noble Kinsmen}, the jailer’s daughter, being mad, says,
“I~can sing twenty more\dots I can sing \textit{The Broom} and \textit{Bonny Robin}.” In
Robinson’s \textit{Schoole of Musicke} (1603), \pagebreak and in one of Dowland’s Lute Manuscripts,
%234
(D.~d.,~2.~11, Cambridge), it is entitled, “Robin is to the greenwood gone; in
Addit. MSS. 17,786 (Brit. Mus.), “\textit{My} Robin,” \&c.

A ballad of “A dolefull adieu to the last Erle of Darby, to the tune of \textit{Bonny
sweet Robin}” was entered at Stationers’ Hall to John Danter on the 26th April,
1593; and in the \textit{Crown Garland of Golden Roses} is “A courtly new ballad of
the princely wooing of the fair Maid of London by King Edward;” as well as
“The fair Maid of London’s answer,” to the same tune. The two last were also
printed in black-letter by Henry Gosson, and are reprinted in Evans’ \textit{Old
Ballads}, iii. 8.

In “Good and true, fresh and new Christmas Carols,” \textsc{b.l.}, 1642, there is a
“Carol for St. Stephen’s day: tune of \textit{Bonny sweet Robin}” beginning—
\settowidth{\versewidth}{“Come, mad boys, be glad, boys, for Christmas is here,”}
\begin{scverse}
\vleftofline{“}Come, mad boys, be glad, boys, for Christmas is here,\\
And we shall be feasted with jolly good cheer,” \&c.
\end{scverse}

\vspace{-0.5\baselineskip}

\musicinfo{Slowly and ad libitum.}{}

\includemusic{chappellV1121.pdf}

\vspace{-1.5\baselineskip}

\musictitle{With A Fading.}

In act iv., sc. 3, of Shakespeare’s \textit{Winter’s Tale}, the servant says of Autolycus,
“He hath songs for man or woman, of all sizes; no milliner can so fit his
customers with gloves: he has the prettiest love-songs for maids;\ldots with such
delicate burdens of \textit{dildos} and \textit{fadings}.”

In the Roxburghe Collection, i. 12, there is a ballad by L. P. (Laurence Price?),
entitled “The Batchelor’s Feast; or—
\settowidth{\versewidth}{Betwixt the batchelor’s pleasure and the married man’s trouble.}
\begin{scverse}The difference betwixt a single life and a double;\\
Betwixt the batchelor’s pleasure and the married man’s trouble.
\end{scverse}
To a pleasant new tune, called \textit{With a hie dildo dill}.” It begins thus:—
\indentpattern{00005656}
\begin{scverse}\begin{patverse}
“As I walkt forth of late, where grass and flowers spring,\\
I heard a batchelor within an harbour sing.\\
The tenor of his song contain’d much melodie:\\
It is a gallant thing to live in liberty.\\
\textit{With a hie, dildo, dill,\\
Hie do, dil dur lie};\\
It is a delightful thing\\
To live at liberty.”
\end{patverse}
\end{scverse}
There are six stanzas; and six more in \pagebreak a second part (at p. 17 of the same
%235
volume), “printed at London for I. W.” (either I. Wright or I. White, who were
both ballad printers of the reigns of James I. and Charles I.)

In \textit{Choice Drollery}, 1656, p. 31, is another, which would require a different
tune, commencing—
\settowidth{\versewidth}{Of a woman that danc’d upon the rope.}
\begin{scverse}\begin{altverse}
\vleftofline{“}A story strange I will you tell,\\
But not so strange as true,\\
Of a woman that danc’d upon the rope.\\
And so did her husband too.\\
\textit{With a dildo, dildo, dildo,\\
With a dildo, dildo dee}.’’
\end{altverse}
\end{scverse}

In the Pepys Collection of Ballads, i. 224, is one by Robert Guy, printed for
H.~Gosson, and with the following title:—

\begin{scverse}\vin\vin\vin\vin “The Merry Forester.\\
Young men and maids, in country or in city\\
I crave your aids with me to tune this ditty;\\
Both new and true it is, no harm in this is,\\
But is composed of the word call’d kisses;\\
Yet meant by none, abroad loves to be gadding:\\
It goes unto the tune of \textit{With a fadding}.”
\end{scverse}
The first line is “Of late I chanc’d to be where I,” \&c.

Another song, which has the burden “with a fading,” will he found in
Shirley’s \textit{Bird in a Cage}, act iv., sc. 1 (1633). A third in \textit{Sportive Wit}, \&c.,
1656, p. 58. The last is also printed in \textit{Pills to purge Melancholy}, ii. 99 (1707),
with the tune, of which there are other copies in the same work.

There are also ballads to it, under the name of \textit{An Orange}, and \textit{With a
Pudding}. See Roxburghe Collection, ii. 16; \textit{Pills to purge Melancholy}, i. 90
(1707), \&c.

The \textit{Fading} is the name of an Irish dance, but\textit{ With a fading} (or \textit{fadding})
seems to he used as a nonsense-burden, like \textit{Derry dawn, Hey nonny, nonny no,} \&c.

\musicinfo{Trippingly and in moderate time.}{}

\includemusic{chappellV1122.pdf}

\pagebreak
%236
\changefontsize{10.4}

\indentpattern{01018}
\begin{dcverse}\footnotesize
\begin{patverse}
You hawk, you hunt, you lie upon pallets,\\
You eat, you drink (the Lord knows how!);\\
We sit upon hillocks, and pick up our sallets,\\
And drink up a syllabub under a cow.\\
\textit{With a fading.}
\end{patverse}

\begin{patverse}
Your masks are made for knights and lords,\\
And ladies that go fine and gay;\\
We dance to such music the bagpipe affords,\\
And trick up our lasses as well as we may.\\
\textit{With a fading.}
\end{patverse}

\begin{patverse}
Your clothes are made of silk and satin,\\
And ours are made of good sheep’s gray;\\
You mix your discourse with pieces of Latin,\\
We speak our English as well as we may.\\
\textit{With a fading.}
\end{patverse}

\begin{patverse}
You dance Corants and the French Braul,\\
We jig the Morris upon the green,\\
And make as good sport in a country hall,\\
As you do before the King and the Queen.\\
\textit{With a fading.}
\end{patverse}
\end{dcverse}

\vspace{-1.5\baselineskip}

\musictitle{How Should I Your True Love Know?}

\vspace{-0.5\baselineskip}

The late W. Linley (an accomplished amateur, and brother of the highly-gifted
Mrs. Sheridan) collected and published “the wild and pathetic melodies of
Ophelia, as he remembered them to have been exquisitely sung by Mrs. Forster,
when she was Miss Field, and belonged to Drury Lane Theatre;” and he says,
“the impression remained too strong on his mind to make him doubt the
correctness of the airs, agreeably to her delivery of them.” Dr. Arnold also
noted them down from the singing of Mrs. Jordan, and Mr. Ayrton has followed
that version in his Annotations to Knight’s \textit{Pictorial Edition of Shakespeare}.
The notes of this air are the same in both; but in the former it is in \timesig{3}{4} time,
in the latter in common time. The melody is printed in common time in
\textit{The Beggars’ Opera} (1728) to “You’ll think, ere many days ensue,” and in
\textit{The Generous Freemason}, 1731.

Dr. Percy selected some of the fragments of ancient ballads which are
dispersed through Shakespeare’s plays, and especially those sung by Ophelia in
\textit{Hamlet}, and connected them by a few supplemental stanzas into his charming
ballad, \textit{The Friar of Orders Gray}, the first line of which is taken from one, sung
by Petruchio, in \textit{The Taming of the Shrew}.

The following is the tune; but in singing Ophelia’s fragments, each line should
begin on the first of the bar, and not with the note before it. In the ballad-operas
it has the burden, \textit{Twang, lang, dildo dee} at the end, with two additional
bars of music, the same as to \textit{The Knight and Shepherd’s Daughter}. See p. 127.


\musicinfo{Moderate time and smoothly.}{}

\smallskip

\includemusic{chappellV1123.pdf}

\backskip{1}
\settowidth{\versewidth}{At his head a green grass turf,}
\begin{dcverse}\footnotesize
\begin{altverse}
He is dead and gone, lady,\\
He is dead and gone;\\
At his head a green grass turf,\\
At his heels a stone.
\end{altverse}

\begin{altverse}
White his shroud as mountain snow,\\
Larded with sweet flowers,\\
Which bewept to the grave did go\\
With true love showers.
\end{altverse}
\end{dcverse}
\pagebreak
%237
\changefontsize{10.2}

A parody on this song seems to be intended in Rowley’s \textit{A Match at Midnight},
1633, where the Welshman sings—
\settowidth{\versewidth}{Did hur not see hur true love-a}
\begin{scverse}
\vleftofline{“}Did hur not see hur true love-a\\
\vin As hur come from London?” \&c.
\end{scverse}

\musictitle{And Will He Not Come Again?}

This fragment, sung by Ophelia, was also noted down by W. Linley. It
appears to be a portion of the tune entitled \textit{The Merry Milkmaids} in \textit{The Dancing
Master}, 1650, and \textit{The Milkmaids’ Dumps} in several ballads. The following lines
in \textit{Eastward Roe}, 1605, resemble, and are probably a parody on, Ophelia’s song:—
\settowidth{\versewidth}{His head as white as milk,}
\indentpattern{00110}
\begin{scverse}
\begin{patverse}
\vleftofline{“}His head as white as milk,\\
All flaxen was his hair;\\
But now he is dead,\\
And lain in his bed,\\
And never will come again.”—\textit{Dodsley}, iv., 223.
\end{patverse}
\end{scverse}

\musicinfo{Very slowly and ad libitum.}{}

\includemusic{chappellV1124.pdf}

\begin{scverse}
\begin{patverse}
His beard was white as snow,\\
All flaxen was his hair,\\
He is gone, he is gone,\\
And we cast away moan;\\
God ’a mercy on his soul.
\end{patverse}
\end{scverse}

\musictitle{O Death! Rock Me Asleep.}

In the second part of Shakespeare’s \textit{King Henry IV}., act ii., sc. 4, Pistol
snatching up his sword, exclaims—
\settowidth{\versewidth}{\textit{Then death rock me asleep, abridge my doleful days}!”}
\begin{scverse}
\vleftofline{“}What! shall we have incision? shall we imbrue?\\
Then \textit{death rock me asleep, abridge my doleful days}!”
\end{scverse}

This is in allusion to the following song, which is supposed to have been written
by Anne Boleyn. The words were printed by Sir John Hawkins in his \textit{History
of Music}, having been “communicated to him by a very judicious antiquary,”
then “lately deceased,” whose opinion was that they were written either by, or in
the person of, Anne Boleyn; “a conjecture,” he adds, “which her unfortunate
history renders very probable.” On this Ritson remarks, “It is, however, but a
conjecture: any other state prisoner of \pagebreak that period having an equal claim.
%238
George, Viscount Rochford, brother to the above lady, and who suffered on her
account, ‘hath the fame,’ according to Wood, ‘of being the author of several
poems, songs, and sonnets, with other things of the like nature,’ and to him he
(Ritson) is willing to refer them.”—(\textit{Ancient Songs}, 1790, p. 120.)

The first stanza of the words, with the tune, is contained in a manuscript of
the latter part of Henry’s reign, formerly in the possession of Stafford Smith,
and now in that of Dr. Rimbault. It is a single-voice part, in the diamond-headed
note, and without accompaniment. Another copy, with an accompaniment for the
lute, will be found in Addit. MSS. 4900, British Museum.

\musicinfo{Moderate time, and like recitative.}{}

\smallskip

\includemusic{chappellV1125.pdf}

\pagebreak
%239

\settowidth{\versewidth}{My dolour will not suffer strength}
\begin{dcverse}
\indentpattern{01014}
\begin{patverse}
My pains who can express?\\
Alas! they are so strong;\\
My dolour will not suffer strength\\
My life for to prolong.\\
Toll on, \&c.
\end{patverse}

\begin{patverse}
Alone in prison strong\\
I wail my destiny;\\
Woe worth this cruel hap that I\\
Should taste this misery.\\
Toll on, \&c.
\end{patverse}

\indentpattern{0101000222}
\begin{patverse}
Farewell my pleasures past,\\
Welcome my present pain;\\
I feel my torments so increase,\\
That life cannot remain.\\
Cease now the passing bell,\\
Rung is my doleful knell,\\
For the sound my death doth tell.\\
Death doth draw nigh,\\
Sound my end dolefully,\\
For now I die.
\end{patverse}
\end{dcverse}

\backskip{1}

\musictitle{Can You Not Hit It, My Good Man?}

The following lines are sung by Rosaline and Boyet in act iv., sc. 1, of \textit{Love’s
Labour Lost}. The tune was transcribed by Dr. Rimbault from one of the MSS.
presented by Bishop Fell to the Music School at Oxford, and bearing a date of
1620. \textit{Canst thou not hit it} is mentioned as a dance in the play of \textit{Wily Beguiled},
written in the reign of Elizabeth. In 1579, “a ballat intytuled \textit{There is better
game, if you could hit it}” was licensed to Hughe Jaxon.

\musicinfo{Trippingly and moderately fast.}{}

\includemusic{chappellV1126.pdf}

\backskip{1}

\begin{center}{\makebox[1in]{\hrulefill}}\end{center}

The list of music illustrating Shakespeare might be largely increased, by
including in it catches, part-music, and the works of known composers, which do
not fall within the scope of the present collection. The admirers of Shakespeare
will be gratified to know that a work is in progress which will include not only
those, but also such of the original music to his dramas as can still be found.\footnote{\textit{}
This work (to which Dr. Rimbault has devoted many
years of zealous research) will be entitled “A Collection
of Ancient Music, illustrating the plays and poems of
Shakespeare.” The first portion will contain all that now
remains of the original music to his dramas, or which, if
not composed for the first representation of them, was
written during the life-time of the poet. The whole of
the music of \textit{The~Tempest} will be included in this part.
Another division will contain the old songs, ballads,
catches, \&c., inserted, or alluded to, by Shakespeare. The
dances will form the third part. It was owing to researches
on a subject so much akin to that of the present
Collection, that Dr. Rimbault’s aid has been so peculiarly
valuable in this work.}

The three following ballads, with which I close the reign of Elizabeth, were
popular in the time of Shakespeare, \pagebreak but are not mentioned by the great poet.
%240

\musictitle{Bara Faustus’ Dream.}

In the instrumental arrangements of this tune it is usually entitled \textit{Bara
Faustus} (or \textit{Barrow Foster’s}) \textit{Dream}; and when found as a song, it is generally
as, “\textit{Come, sweet love, let sorrow cease}.”

It will be found under the former name in Queen Elizabeth’s Virginal Book
(twice); in Rossiter’s \textit{Lessons for Consort}, 1609; and in Nederlandtsche Gedenck-
Clanck, 1626, under the latter in “Airs and Sonnets,”, MS., Trin. Col., Dublin
(F.~v.~13); in the MS. containing “It was a lover and his lass,” described at
p. 204; and in Forbes’ \textit{Cantus}, 1682.

\textit{Bara Faustus’ Dreame} was one of the tunes chosen for the \textit{Psalmes or Songs
of Sion}, \&c., 1642.


\musicinfo{Smoothly, and with expression.}{}

\medskip

\includemusic{chappellV1127.pdf}

\musictitle{The Spanish Pavan.}

Dekker, in his \textit{Knight’s Conjuring} (1607) thus apostrophises his opponent:
“Thou, most clear-throated singing man, with thy harp, to the twinkling of
which inferior spirits skipp’d like goats over the Welsh mountains, hadst privilege
(because thou wert a fiddler) to be saucy? Inspire me with thy cunning, and
guide me in true fingering, that I may strike those tunes which thou playd’st!
Lucifer himself danced a \textit{Lancashire Hornpipe} whilst thou wert there. If I can
but harp upon \textit{thy} string, he shall now, for my pleasure, tickle up \textit{The Spanish
Pavan}.” The tune of \textit{The Spanish Pavan} was very popular in the reigns of
Elizabeth and James. One of the songs in \pagebreak Anthony Munday’s \textit{Banquet of Daintie Conceits}
\markright{reign of elizabeth.}
%241
1588, is “to the note of \textit{The Spanish Pavin};” another in
part ii. of \textit{Robin Goodfellow}, 1628; and there are many in the Pepys and Roxburghe
Collections of Ballads.

It is mentioned as a dance in act iv., sc. 2, of Middleton’s \textit{Blurt, Master Constable},
1602; and in act i., sc. 2, of Ford’s ’\textit{Tis Pity}, 1633. In the former the
tune is played for Lazarillo to dance \textit{The Spanish Pavan}. The figure, which
differed from other Pavans, is described in Thoinot Arbeau’s \textit{Orchesographie}, 1589;
but as the tune there printed is wholly different from the following (which is
found in Queen Elizabeth’s Virginal Book, William Ballet’s Lute Book, Sir
J. Hawkins’ transcripts of Virginal Music, \&c.), I suppose this to be English,
although not a characteristic~air.

The ballad, “When Samson was a tall young man,” (of which the first stanza
is here printed) is in the Pepys Collection, i. 32; in the Roxburghe, i. 366; and
in Evans’ \textit{Old Ballads}, i. 283 (1810).\footnote{\textit{}
The copies in the Pepys and Roxburghe Collections
differ. The former has no printer’s name; the latter
(which is followed by Evans) was printed “for the
assigns of T.~Symcocke.”}
 It is parodied in \textit{Eastward Hoe}, the joint
production of Ben Jonson, Marston, and Chapman, act ii., sc. 1. The two first
lines are the same in the parody and the ballad.

\musicinfo{Moderate time.}{}

\includemusic{chappellV1128.pdf}

\pagebreak
%242

\musictitle{Wigmore’s Galliard.}

The tune from William Ballet’s Lute Book. In Middleton’s \textit{Your five Gallants},
Jack says, “This will make my master leap out of the bed for joy, and dance
\textit{Wigmore’s Galliard} in his shirt about his chamber!” It is frequently mentioned
by other early writers, and there are many ballads to the tune. Among them
are “A most excellent new Dittie, wherein is shewed the wise sayings and wise
sentences of Solomon, wherein each estate is taught his dutie, with singular
counsell to his comfort and consolation” (a copy in the collection of the late
Mr. W. H. Miller, from Heber’s Library). “A most famous Dittie of the joyful
receiving of the Queen’s most excellent Majestie by the worthie citizens of
London, the 12th day of November, 1584, at her Grace’s coming to St. James’”
(a copy in the Collection of Mr. George Daniel). In the Pepys Collection, i. 455,
is “A most excellent Ditty called Collin’s Conceit,” beginning—
\settowidth{\versewidth}{“Conceits of sundry sorts there are.”}
\begin{scverse}
“Conceits of sundry sorts there are.”
\end{scverse}
Others are in the second volume of the Pepys Collection; in the Roxburghe; in
Anthony Munday’s \textit{Banquet of Daintie Conceits}; in Deloney’s \textit{Strange Histories},~1607,~\&c.

The following stanza is from the ballad of “King Henry the Second crowning
his son Henry, in his life-time,” \&c., by Deloney. The entire ballad is reprinted
by Evans (ii. 63), from \textit{The Garland of Delight}, but he omits the name of the tune.

\includemusic{chappellV1129.pdf}

\pagebreak
%243

\musictitle{Good Fellows Must Go Learn To Dance.}

The following ballad is from a copy (probably unique) in the Collection of
Mr. George Daniel, of Canonbury. It may be sung to several of the foregoing
airs, but the name of the proper tune is not given on the copy.

\begin{center}\small\textsc{A New Ballad Intituled}
\end{center}

\musictitle{Good Fellows Must Go Learn To Dance.}

\begin{dcverse}
\settowidth{\versewidth}{The bridegroom would give 20 pound}
\begin{altverse}
Good fellows must go learn to dance,\\
The bridal is full near-a,\\
There is a Braule come out of France,\\
The trick’st you heard this year-a;\\
For I must leap, and thou must hop,\\
And we must turn all three-a,\\
The fourth must bounce it like a top,\\
And so we shall agree-a;\\
I pray thee, Minstrel, make no stop,\\
For we will merry be-a.
\end{altverse}

\begin{altverse}
The bridegroom would give 20 pound\\
The marriage-day were past-a;\\
You know while lovers are unbound,\\
The knot is slipp’ry fast-a.\\
A better man may come in place,\\
And take the bride away-a;\\
God send or Wilkin better grace,\\
Our pretty Tom doth say-a;\\
Good Vicar, axe the banns apace,\\
And haste the marriage-day-a.
\end{altverse}

\begin{altverse}
A band of bells in bawdrick wise\\
Would deck us in our kind-a;\\
A shirt after the Morris guise,\\
To flounce it in the wind-a;\\
A Whiffler for to make the way,\\
And May brought in with all-a,\\
Is braver than the sun, I say,\\
And passeth Round or Braule-a,\\
For we will trip so trick and gay,\\
That we will pass them all-a.
\end{altverse}

\begin{altverse}
Draw to dancing, neighbours all,\\
Good fellows, hip is best-a;\\
It skills not if we take a fall,\\
In honoring this feast-a.\\
The bride will thank us for our glee,\\
The world will us behold-a;\\
O where shall all this dancing be?\\
In Kent or in Cotswold-a?\\
Our lord doth know, then axe not me,\\
And so my tale is told-a.
\end{altverse}
\end{dcverse}

Imprinted at London in Flete Strete at the signe of the Faucon, by Wylliam
Gryffith, and are to be solde at his shoppe in S. Dunstones Church Yearde, 1569.

\centerrule
\pagebreak
