\changefontsize{1.03\defaultfontsize}
\origpage{}%279

\musictitle{Down In The North Country.}

This tune was formerly very popular, and is to be found under a variety of
names, and in various shapes. In the second vol. of \textit{The Dancing Master} it is
entitled \textit{The Merry Milkmaids}. In \textit{The Merry Musician, or a Cure for the
Spleen}, i. 64, it is printed to the ballad, “The Farmer’s Daughter of merry
Wakefield.” That ballad begins with the line, “Down in the North Country;”
and the air is so entitled in the ballad-opera, \textit{A Cure for a Scold}, 1738. In
\textit{180 Loyal Songs}, third and fourth editions, 1684 and 1694, there are two songs,
and the tune is named \textit{Philander}. The first of the songs begins, “\textit{Ah, cruel
bloody fate},” and the second is “to the tune of \textit{Ah, cruel bloody fate};” by which
name it is also called in \textit{The Genteel Companion for the Recorder}, 1683, and
elsewhere.

One of M[artin] P[arker’s] ballads is entitled “Take time while ’tis offer’d;”
\settowidth{\versewidth}{For Tom has broke his word with his sweeting,}
\begin{scverse}
\vleftofline{“}For Tom has broke his word with his sweeting,\\
And lost a good wife for an hour’s meeting;\\
Another good fellow has gotten the lass,\\
And Tom may go shake his long ears like an ass.”
\end{scverse}
to the tune \textit{Within the North Country}”. (Roxburghe, i. 396.) It begins with
the line, “When Titan’s fiery steeds,” and the last stanza is—
\settowidth{\versewidth}{And I have raised my fortunes well—}
\begin{scverse}
\begin{altverse}
“Thus Tom hath lost his lass,\\
Because he broke his vow;\\
And I have raised my fortunes well—\\
\textit{The case is alter’d now}.’’
\end{altverse}
\end{scverse}

There are many ballads to the tune \textit{The case is altered}, and probably this is
intended.

In the Bagford Collection is “The True Lover’s lamentable Overthrow; or
The Damosel’s last Farewell,” \&c.: “to the tune of \textit{Cruel bloody fate};”
commencing—
\begin{scverse}
\vleftofline{“}You parents all attend\\
To what of late befell;\\
It is to you I send\\
These lines, my last farewell.” \&c.
\end{scverse}

In the Douce Collection, p. 245, “The West Country Lovers—
\begin{scverse}
\begin{altverse}
See here the pattern of true love,\\
Amongst the country blades,\\
Who never can delighted be,\\
But when amongst the maids:
\end{altverse}
\end{scverse}
tune of \textit{Philander}.”

The last is in black-letter, printed by J. Bonyers, at the Black Raven in Duck
Lane. A former possessor has written “Cruel bloody fate” under “Philander,”
as being the other name of the tune.

In the Roxburghe Collection, ii. 105,—“The Deceiver Deceived; or The
Virgin’s Revenge: to the tune of \textit{Ah, cruel bloody fate},” begins, “Ah, cruel maid,
give o’er.”

In \textit{A Cabinet of Choice Jewels}, 1688 (Wood’s Library, Oxford)—a “Carol for
Innocents’ Day: tune of \textit{Bloody fate}.”
\pagebreak
%280

The song of \textit{Philander} is in \textit{ Pills to purge Melancholy}, ii. 252 (1707), or
iv. 284 (1719); in \textit{Wit and Drollery}, 1682; and a black-letter copy in the
Douce Collection, p. 74, entitled “The Faithfull Lover’s Downfall; or The
Death of fair Phillis, who killed herself for the loss of her Philander,” \&c.: to a
pleasant new play-house tune, or \textit{O cruel bloody fate}.” (Printed by T. Vere, at
the Angel in Giltspur Street.)

\musicinfo{Smoothly and in moderate time.}{}

\includemusic{chappellV1146.pdf}

\settowidth{\versewidth}{Ah, I come! she cried, with a wound so wide, to need no second blow.}
\begin{scverse}Her poniard then she took, and graspt it in her hand,\\
And with a dying look, cried, Thus I fate command:\\
Philander, ah, my love! I come to meet thy shade below;\\
Ah, I come! she cried, with a wound so wide, to need no second blow.

In purple waves her blood ran streaming down the floor;\\
Unmov’d she saw the flood, and bless’d her dying hour:\\
Philander, ah Philander, still the bleeding Phillis cried;\\
She wept awhile, and forc’d a smile, then clos’d her eyes and died.
\end{scverse}

The following is the version called “Down in the North Country,” of which
there are also copies in Halliwell’s Collection (Cheetham Library, 1850), and in
Dr. Burney’s Collection, Brit. Mus.
\pagebreak
%281

\musicinfo{Cheerfully.}{}

\includemusic{chappellV1147.pdf}


The following is the version of the same tune, which is entitled \textit{The Merry Milkmaids}
in the second volume of \textit{The Dancing Master}. It was formerly the custom
for milkmaids to dance before the houses of their customers in the month of May,
to obtain a small gratuity; and probably this tune, and \textit{The Merry Milkmaids in
green}, were especial favorites, and therefore named after them. To be a milkmaid
and to be merry were almost synonymous in the olden time. Sir Thomas
Overbury’s \textit{Character of a Milkmaid}, and some allusions to their songs, will be
found with the tune entitled \textit{The Merry Milkmaids in green}. The following
quotations relate to their music and dancing.

In Beaumont and Fletcher’s play, \textit{The Coxcomb}, Nan, the milkmaid, says—
\settowidth{\versewidth}{And we serve a very good woman, and a gentlewoman;}
\begin{scverse}\vleftofline{“}Come, you shall e’en home with us, and be our fellow;\\
Our house is so honest!\\
And we serve a very good woman, and a gentlewoman;\\
And we live as merrily, and dance o’ good days\\
After even-song. Our wake shall be on Sunday:\\
Do you know what a wake is?—we have mighty cheer then,” \&c.
\end{scverse}

Pepys, in his Diary, 13th Oct., 1662, says, “With my father took a melancholy
walk to Portholme, seeing the country-maids, milking their cows there,
they being there now at grass; and to see with what mirth they come all home
together in pomp with their milk, and sometimes they have music go before them.”
\pagebreak
%282
Again, on the 1st May, 1667, “To Westminster; on the way meeting many
milkmaids with their garlands upon their pails, dancing with a fiddler before
them; and saw pretty Nelly” [Nell Gwynne] “standing at her lodgings’ door in
Drury Lane, in her smock sleeves and bodice, looking upon one: she seemed a
mighty pretty~creature.”

In a set of prints, called \textit{Tempest's Cryes of London}, one is called “The Merry
Milkmaid, whose proper name was Kate Smith. She is dancing with her milkpail
on her head, decorated with silver cups, tankards, and salvers, borrowed for
the purpose, and tied together with ribbands, and ornamented with flowers. Of
late years, the plate, with other decorations, were placed in a pyramidical form,
and carried by two chairmen upon a wooden horse. The mikmaids walked before
it, and performed the dance without any incumbrance. Strutt mentions having
seen “these superfluous ornaments, with much more propriety, substituted by a
cow. The animal had her horns gilt, and was nearly covered with ribbands of
various colours, formed into bows and roses, and interspersed with green oaken
leaves and bunches of flowers.” \textit{Sports and Pastimes}, edited by Hone, p. 358.

\musicinfo{Lively.}{The Milkmaids' Dance.}

\includemusic{chappellV1148.pdf}

\pagebreak
%283

\musictitle{Morris Dance.}

This is entitled \textit{Engelsche Klocke-Dans} in three of the Collections published in
Holland: viz., in \textit{Bellerophon} (Amsterdam, 1622); \textit{Nederlandtsche GedenckClanck}
(Haerlem, 1626); and \textit{Friesche Lust-Hof} (Amsterdam, 1634.)

As “klok” signifies “bell,” and bells were worn in the morris, I suppose it to
have been a morris-dance. In the above-named collections, Dutch songs are
adapted to it, but I have no clue to the English words.

\musicinfo{Moderate time.}{}

\includemusic{chappellV1149.pdf}

\musictitle{Amarillis Told Her Swain.}

This is found, under the name of \textit{Amarillis}, among the violin tunes in \textit{The
Dancing Master} of 1665, and in all later editions; in \textit{Musick’s Delight on the
Cithren}, 1666; in \textit{Apollo’s Banquet}, 1670; in the \textit{Pleasant Companion for the
Flageolet},~1680;~\&c.

The song, “Amarillis told her swain,” is in \textit{Merry Drollery complete}, 1670 (p.3).

The air is sometimes referred to as \pagebreak “Phillis on the new-made hay,” from a
%284
ballad entitled “The coy Shepherdess; or Phillis and Amintas;” which was sung
to the tune of \textit{Amarillis}. See Roxburghe Collection, ii. 85.

Among the ballads to the air, are also the following:—
\settowidth{\versewidth}{Attend on human nature,” \&c.}
“The Royal Recreation of Jovial Anglers;” beginning—
\begin{scverse}
\vleftofline{“}Of all the recreations which\\
Attend on human nature,” \&c.\\
\attribution Roxburghe Collection.
\end{scverse}
Collier’s Roxburghe Ballads, p. 232; and \textit{Merry Drollery complete}, 1661 and
1670. It is also in \textit{ Pills to purge Melancholy}; but there set to the tune of
\textit{My father was born before me}.

“Love, in the blossom; or Fancy in the bud: to the tune \textit{Amarillis told her
swain}.” Roxburghe, ii. 315.

“Fancy’s Freedom; or true Lovers’ bliss: tune of \textit{Amarillis}, or \textit{Phillis on the
new-made hay}.” Roxburghe, iii. 114.

“The true Lovers’ Happiness; or Nothing venture, nothing have,” \&c.: tune
of \textit{Amintas on the new-made hay}; or \textit{The Loyal Lovers}.” Douce Collection, and
Roxburghe, ii. 486.

“The Cotsall Shepherds: to the tune of \textit{Amarillis told her swain},” in \textit{Folly in
print, or a Book of Rhymes}, 1667.

The following stanza, set to the tune, is the. first of the above-named ballad,
“The coy Shepherdess; or Phillis and Amintas:”—

\musicinfo{Smoothly, and in moderate time.}{}

\includemusic{chappellV1150.pdf}

\pagebreak
\changefontsize{0.99\defaultfontsize}
%285

\musictitle{Cherrily And Merrily.}

In \textit{The Dancing Master} of 1652, this is entitled \textit{Mr. Webb’s Fancy}; and in
later editions \textit{Cherrily and merrily}.

In vol. xi. of the King’s Pamphlets, folio, there is a copy of a ballad written on
the violent dissolution of the Long Parliament by Cromwell, entitled “The Parliament
routed; or Here’s a house to be let:
\settowidth{\versewidth}{Shall be at peace, and give no way to warres:}
\begin{scverse}I hope that England, after many jarres,\\
Shall be at peace, and give no way to warres:\\
O Lord, protect the generall, that he\\
May be the agent of our unitie:
\end{scverse}
to the tune of \textit{Lucina}, or \textit{Merrily and cherrily}.” [June 3, 1653.] It has been
reprinted in \textit{Political Ballads}, Percy Society, No. 11, p. 126. The first stanza
is as follows:—
\settowidth{\versewidth}{Hot spirits are quenched, the tempest is layd,}
\begin{scverse}\begin{altverse}
\vleftofline{“}Cheer up, kind countrymen, be not dismay’d,\\
True news I can tell ye concerning the nation:\\
Hot spirits are quenched, the tempest is layd,\\
And now we may hope for a good reformation.”
\end{altverse}
\end{scverse}

The above is more suited to the tune of \textit{Lucina} (\ie, \textit{The Beggar Boy}, p. 269)
than to this air; I have therefore adapted a song from \textit{Universal Harmony}, 1746,
an alteration of the celebrated one by George Herbert.

\musicinfo{Smoothly and in moderate time.}{}

\includemusic{chappellV1151.pdf}

\backskip{1}

\settowidth{\versewidth}{Storehouse where sweets unnumher’d lie,}
\begin{dcverse}\begin{altverse}
Sweet rose, so fragrant and so brave,\\
Dazzling the rash beholder’s eye,\\
Thy root is ever in its grave,\\
And thou, with all thy sweets, must die.
\end{altverse}

\begin{altverse}
Sweet Spring, so beauteous and so gay,\\
Storehouse where sweets unnumber’d lie,\\
Not long thy fading glories stay,\\
But thou, with all thy sweets, must die.
\end{altverse}

\begin{altverse}
Sweet love, alone, sweet wedded love,\\
To thee no period is assign’d;\\
Thy tender joys by time improve,\\
In death itself the most refin’d.
\end{altverse}
\end{dcverse}
\pagebreak
%286

\musictitle{St. George For England.}

There are black-letter copies of this ballad in the Pepys and Bagford Collections. 
It is also in An \textit{Antidote to Melancholy}, 1661; in part ii. of \textit{Merry
Drollery Complete}, 1661 and 1670; in \textit{Wit and Drollery}, 1682; \textit{Pills to purge
Melancholy}, 1707 and 1719; \&c.

It is one of those offered for sale by the ballad-singer in Ben Jonson’s
comedy of \textit{Bartholomew Fair}.

Pepys, in his Diary, tells us of “reading a ridiculous ballad, made in praise of
the Duke of Albemarle, to the tune of \textit{St. George}—the tune being printed too;”
and adds, “I observe that people have great encouragement to make ballads of
him, of this kind. There are so many, that hereafter he will sound like Guy of
Warwick.” (6th March, 1667.)

Fielding, in his novel of \textit{Tom Jones}, speaks of \textit{St. George he was for England}
as one of Squire Western’s favorite tunes.

The ballad in the Pepys Collection (i. 87) is entitled “Saint George’s Commendation
to all Souldiers; or Saint George’s Alarum to all that profess martiall
discipline, with a memoriall of the Worthies who have been borne so high on the
wings of Fame for their brave adventures, as they cannot be buried in the pit of
oblivion: to \textit{a pleasant new tune}.” It was “imprinted at London, by W. W.,” in
1612, and is the copy from which Percy printed, in his \textit{Reliques of Ancient
Poetry}. It begins—“Why do we boast of Arthur and his Knightes.’’

In Anthony Wood’s Collection, at Oxford, No. 401, there is a modernization
of this ballad, entitled—
\settowidth{\versewidth}{St. George for England, and St. Dennis for France;}
\begin{scverse}
\vleftofline{“}St. George for England, and St. Dennis for France;\\
O hony soite qui mal y pance:
\end{scverse}
to an excellent new tune.” (Wood’s Ballads, ii. 118.) It is subscribed S. S., and
“printed for W. Gilbertson, in Giltspur Street;” from which it may be dated
about 1659.

As a specimen of the comparative modernization, I give the first stanza:—
\settowidth{\versewidth}{His sword with fame was crown’d;}
\begin{dcverse}
\begin{altverse}
\vleftofline{“}What need we brag or boast at all\\
Of Arthur and his Knights,\\
Knowing how many gallant men\\
They have subdued in fights.\\
For bold Sir Launcelot du Lake\\
Was of the table round;\\
And fighting for a lady’s sake,\\
His sword with fame was crown’d;\\
Sir Tarquin, that great giant,\\
His vassal did remain;\\
But St. George, St. George,\\
The Dragon he hath slain.\\
St. George he was for England,\\
St. Dennis was for France;\\
O hony soite qui mal y pance.”
\end{altverse}
\end{dcverse}

A copy of the \textit{old} ballad in the Bagford Collection is entitled “A new ballad
of St. George and the Dragon,” but there is also another of St. George and the
Dragon, which, Percy has printed in the \textit{Reliques}.

In \textit{180 Loyal Songs}, 1685 and 1694, there is “a new song on the instalment,of
Sir John Moor, Lord Mayor of London: tune, \textit{St. George for England}.” And in
\textit{Pills to purge Melancholy}, iii. 20 (1707), “A new ballad of King Edward and
Jane Shore,” to the same.
\pagebreak
%287

As the ballad is contained in Percy’s \textit{Reliques}, as well as a witty second part,
written by John Grubb, and published in 1688, the first stanza only is here
printed with the music.

\musicinfo{Moderate time.}{}

\includemusic{chappellV1152.pdf}
\pagebreak
%288

\musictitle{The Healths.}

This tune is in \textit{The Dancing Master}, from 1650 to 1690, and in \textit{Musick’s
Delight on the Cithren}, 1666.

In the first editions of \textit{The Dancing Master} it is entitled \textit{The Health}; in the
seventh and eighth, \textit{The Healths}, or \textit{The Merry Wassail}.

The following song, “Come, faith, since I’m parting,” was written by Patrick
Carey, a loyal cavalier, on bidding farewell to his hospitable entertainers at Wickham, 
in 1651. It is “to the tune of \textit{The Healths}.”

\musicinfo{Moderate time.}{}

\includemusic{chappellV1153.pdf}

\backskip{1}

\settowidth{\versewidth}{He’ll make a brave man, you may see’t in his face;}
\indentpattern{0001}
\begin{dcverse}\begin{patverse}
And first to Sir William, I’ll take’t on my knee;\\
He well doth deserve that a brimmer it be:\\
More brave entertainments none ere gave than he;\\
Then let his health go round.
\end{patverse}

\begin{patverse}
Next to his chaste lady, who loves him as life;\\
And whilst we are drinking to so good a wife,\\
The poor of the parish will pray for her life;\\
Be sure her health go round.
\end{patverse}

\begin{patverse}
And then to young Will, the heir of this place;\\
He’ll make a brave man, you may see’t in his face;\\
I only could wish we had more of the race;\\
At least let his health go round.
\end{patverse}

\begin{patverse}
To well-grac’d Victoria the next room we owe;\\
As virtuous she’ll prove as her mother, I trow,\\
And somewhat in huswifry more she will know;\\
O let her health go round!
\end{patverse}
\end{dcverse}
\pagebreak
%289

\settowidth{\versewidth}{The most are good fellows, and love to carouse;}
\begin{dcverse}\indentpattern{0001}
\begin{patverse}
To plump Bess, her sister, I drink down this cup:\\
Birlackins, my masters, each man must take’t up;\\
’Tis foul play, I bar it, to simper and sup,\\
When such a health goes round.
\end{patverse}

\begin{patverse}
And now, helter-skelter, to th rest of the house:\\
The most are good fellows, and love to carouse;\\
Who’s not, may go sneck-up;\footnote{\textit{}
Sir Walter Scott prints this “sneake-up:” I suppose
it should be “snecke-up”—a common expression,
equivalent to “go and be hanged.”}
 he’s not worth a louse\\
That stops a health i’ th’ round.
\end{patverse}

\begin{patverse}
To th’ clerk, so he’ll learn to drink in the morn;\\
To Heynous, that stares when he has quaft up his horn;\\
To Philip, by whom good ale ne’er was forlorn;\\
These lads can drink a round.
\end{patverse}

\begin{patverse}
John Chandler! come on, here’s some warm beer for you;\\
A health to the man that this liquor did brew:\\
Why Hewet! there’s for thee; nay take’t, ’tis thy due,\\
But see that it go round.
\end{patverse}

\begin{patverse}
Hot Coles is on fire, and fain would be quench’d;\\
As well as his horses, the groom must be drench’d:\\
Who’s else? let him speak, if his thirst he’d have stench’d,\\
Or have his health go round.
\end{patverse}

\begin{patverse}
And now to the women, who must not be coy;\\
A glass, Mistress Cary, you know’s but a toy;\\
Come, come, Mistress Sculler, no \textit{perdonnez moy},\\
It must, it must go round.
\end{patverse}

\begin{patverse}
Dame Nell, so you’ll drink, we’ll allow you a sop;\\
Up with’t, Mary Smith, in your draught never stop ;\\
Law, there now, Nan German has left ne’er a drop,\\
And so must all the round.
\end{patverse}

\begin{patverse}
Jane, Joan, Goody Lee, great Meg, and the less,\\
You must not he squeamish, but do as did Bess:\\
How th’ others are nam’d, if I could but guess,\\
I’d call them to the round.
\end{patverse}
\end{dcverse}

\settowidth{\versewidth}{And now, for my farewell, I drink up this quart,}
\begin{scverse}And now, for my farewell, I drink up this quart,\\
To you, lads aud lasses, e’en with all my heart;\\
May I find you ever, as now when we part,\\
\vin Each health still going round.
\end{scverse}

\musictitle{Mall Peatly.}

This tune is contained in \textit{Bellerophon, of Lust tot Wyshed}, Amsterdam, 1622;
in the seventh and later editions of \textit{The Dancing Master}; in \textit{Apollo's Banquet};
and in several of the ballad-operas.

In \textit{Bellerophon}, the first part is in common time, and the second in triple, like
a cushion dance; but it is not so in any of the above-named English copies,
which, however, are of later date.

D’Urfey wrote to it a song entitled \textit{Gillian of Croydon} (see \textit{Pills to purge
Melancholy}, ii. 46), and it is to be found under that name in some of the ballad-operas, 
such as \textit{The Fashionable Lady}, or \textit{Harlequin's Opera}, 1730; \textit{Sylvia}, or
\textit{The Country Burial}, 1731; \textit{The Jealous Clown}, 1730; \&c. There are also several
songs to it in the \textit{Collection of State Songs sung at the Mug-houses in London and
Westminster}, 1716. In \textit{Apollo’s Banquet}, the tune is entitled \textit{The Old Marinett},
or \textit{Mall Peatly}; in Gay’s \textit{Achilles}, \textit{Mo}ll Peatly.

Mall is the old abbreviation, of Mary. (See Ben Jonson’s \textit{English Grammar}.)

In \textit{Round about our coal-fire}, or \textit{Christmas Entertainments} (7th edit., 1734), it
is said, in allusion to Christmas, \pagebreak “This time of year being cold and frosty,
%290
generally speaking, or when Jack-Frost commonly takes us by the nose, the
diversions are within doors, either in exercise or by the fire-side. Dancing is one
of the chief exercises—\textit{Moll Peatly} is never forgot;—this dance stirs the blood
and gives the males and females a fellow-feeling for each other’s activity, ability,
and agility: Cupid always sits in the corner of the room where these diversions
are transacting, and shoots quivers full of arrows at the dancers, and makes his
own game of them.”

\musicinfo{Gaily.}{}

\smallskip

\includemusic{chappellV1154.pdf}

\changefontsize{1.02\defaultfontsize}

\musictitle{Bobbing Joe, Oh Bobbing Joan.}

The tune of \textit{Bobbing Joe} will be found in every edition of \textit{The Dancing Master};
in \textit{Musick's Delight on the Cithren}, 1666; \&c.

It is sometimes entitled \textit{Bobbing Joan}, as by Carey in his \textit{Ballades} (1651); in
\textit{Polly}, 1729; in \textit{The Bay's Opera}, 1730; \textit{The Mad House}, 1737; \textit{A Cure for a
Scold}, 1738;~\&c.
\pagebreak
%291

“New Bob-in-Jo” is mentioned as a tune in No. 38 of \textit{Mercurius Democritus,
or a True and Perfect Nocturnall}, December, 1652. (See King’s Pamphlets,
Brit. Mus.)

The song, “My dog and I,” is to the tune of \textit{My dog and I}, or \textit{Bobbing Joan}.
(A~copy in Mr. Halliwell’s Collection.)

The following is the ballad by Patrick Carey, “to the tune of \textit{Bobbing Joane}.”

\musicinfo{Cheerfully.}{}

\includemusic{chappellV1155.pdf}

\settowidth{\versewidth}{Those beauties so, which were ensh}
\begin{dcverse}\indentpattern{0101220}
\begin{patverse}
It still was mine and others’ wonder\\
To see me court so eagerly;\\
Yet, soon as absence did me sunder\\
From those I lov’d, quite cured was I.\\
The reason was,\\
That my breast has,\\
Instead of heart, a looking-glass.
\end{patverse}

\begin{patverse}
And as those forms that lately shined\\
I’ th’ glass, are easily defac’d;\\
Those beauties so, which were enshrined\\
Within my breast, are soon displac’d:\\
Both seem as they\\
Would ne’er away;\\
Yet last but while the lookers stay.
\end{patverse}

\begin{patverse}
Then let no woman think that ever\\
In absence I shall constant prove;\\
Till some occasion does us sever\\
I can, as true as any, love:\\
But when that we\\
Once parted be,\\
Troth, I shall court the next I see.
\end{patverse}
\end{dcverse}

\musictitle{When The Stormy Winds Do Blow.}

The ballad, now known as \textit{You Gentlemen of England}, is an alteration of one
by M[artin] P[arker], a copy of which is in the Pepys Collection, i. 420; printed
at London for C. Wright. It is in black-letter, and entitled “Saylers for my
money: a new ditty composed in the praise of Saylers and Sea Affaires; briefly
shewing the nature of so worthy a calling, \pagebreak and effects of their industry: to the
%292
tune of \textit{The~Joviall Cobler.}” Instead of “You gentlemen of England,” it begins,
“Countriemen of England,” \&c.

Ritson prints from a copy entitled “Neptune’s raging fury; or The Gallant
Seaman’s Sufferings. Being a relation of their perils and dangers, and of the
extraordinary hazards they undergo in their noble adventures: together with
their undaunted valour and rare constancy in all their extremites; and the
manner of their rejoycing on shore, at their return home. Tune of \textit{When the
stormy winds do blow}” (the burden of the song). A black-letter copy of this
version is in the Bagford Collection, printed by W. O[nley], temp. Charles II.;
and in one of the volumes of the Douce Collection, p. 168, printed by C. Brown
and T. Norris, and sold at the Looking Glass on London Bridge. A third in the
Roxburghe Collection, ii. 543. “\textit{Stormy winds}” is also in the list of ballads
printed by W.~Thackeray, about~1660.

On the accession of Charles II., we have, “The valiant Seaman’s Congratulation
to his Sacred Majesty King Charles the Second,” \&c.: to the tune of
\textit{Let us drink and sing, and merrily troul the bowl}, or \textit{The stormy winds do blow}, or
\textit{Hey, ho, my honey}.” (Black-letter, twelve stanzas; F. Grove, Snow Hill.)
It commences thus:--
\settowidth{\versewidth}{Great Charles, your English seamen,}
\begin{scverse}
\begin{altverse}
\vleftofline{“}Great Charles, your English seamen,\\
Upon our bended knee.\\
Present ourselves as freemen\\
Unto your Majesty.\\
Beseeching God to bless you\\
Where ever that you go;\\
So we pray, night and day,\\
When the stormy winds do blow.”
\end{altverse}
\end{scverse}

Although the option of singing it to three tunes is given, it is evident, from the
two last lines, that it was written to this.

Among the other ballads to the tune are, “The valiant Virgin, or Philip and
Mary: In a description of a young gentlewoman of Worcestershire (a rich gentleman’s 
daughter) being in love with a farmer’s son, which her father despising,
because he was poor, caus’d him to be press’d for sea: and how she disguised
herself in man’s apparel and follow’d him,” \&c. “To the tune of \textit{When the stormy
winds do blow};” (Roxburghe, ii. 546) beginning—
\begin{scverse}
\vleftofline{“}To every faithful lover\\
\vin That’s constant to her dear,” \&c.
\end{scverse}

In \textit{Poems by Ben Jonson, junior}, 8vo., 1672, is “The Bridegroom’s Salutation:
to the tune \textit{When the stormy winds do blow};” beginning—
\settowidth{\versewidth}{In all thy glories drest,” \&c.}
\begin{scverse}
\vleftofline{“}I took thee on a suddain,\\
\vin In all thy glories drest,” \&c.
\end{scverse}

In \textit{180 Loyal Songs}, 1686 and 1694, a bad version of the tune is printed to
“You Calvinists of England.”

There are fourteen stanzas in the copy of “You gentlemen” printed by Ritson,
in his \textit{English Songs}. The following shorter version is from one of the broadsides
with music, compared with another copy in \textit{Early Naval Ballads} (Percy Society,
No. 8, p. 34.)
\pagebreak
%293

\musicinfo{Boldly.}{}

\includemusic{chappellV1156.pdf}

\settowidth{\versewidth}{When the stormy winds do blow.}
\begin{dcverse}
\begin{altverse}
The sailor must have courage.\\
No danger he must shun;\\
In every kind of weather\\
His course he still must run;\\
Now mounted on the top-mast,\\
How dreadful ’tis below:\\
Then we ride, as the tide,\\
When the stormy winds do blow.
\end{altverse}

\begin{altverse}
If enemies oppose us,\\
And England is at war\\
With any foreign nation,\\
We fear not wound nor scar.\\
To humble them, come on, lads,\\
Their flags we’ll soon lay low;\\
Clear the way for the fray,\\
Tho’ the stormy winds do blow.
\end{altverse}

\begin{altverse}
Sometimes in Neptune’s bosom\\
Our ship is toss’d by waves,\\
And every man expecting\\
The sea to be our graves;\\
Then up aloft she’s mounted,\\
And down again so low,\\
In the waves, on the seas,\\
When the stormy winds do blow.
\end{altverse}

\begin{altverse}
But when the danger’s over,\\
And safe we come on shore,\\
The horrors of the tempest\\
We think of then no more;\\
The flowing bowl invites us,\\
And joyfully we go,\\
All the day drink away,\\
Tho’ the stormy winds do blow.
\end{altverse}
\end{dcverse}
\pagebreak
%294
\musictitle{Red Bull.}

This tune is named after the Red Bull Playhouse, which formerly stood in
St. John Street, Clerkenwell. It was in use throughout the reigns of James I.
and Charles I., and perhaps before. At the Restoration, the King’s actors, under
Thomas Killigrew, played there until they removed to the new Theatre in Drury
Lane; and when Davenant produced his \textit{Playhouse to be Let}, in 1663, it was
entirely abandoned. (See Collier’s \textit{Annals of the Stage}.)

In the Roxburghe Collection, i. 246, is a ballad entitled “A mad kind of
wooing; or A Dialogue between Will the simple, and Nan the subtle, with their
loving agreement: to the tune of \textit{The new Dance at the Red Bull Playhouse}.”
It is black-letter, printed for the assigns of T. Symcocke, whose patent for
“printing of paper and parchment on the one side” was granted in 1620, and
assigned in the same year. Another copy of the ballad will be found in the
Pepys Collection, i.~276, “printed for H[enry] G[osson] on London Bridge.

The tune is contained in \textit{Apollo’s Banquet for the Treble Violin}, entitled \textit{The
Damsel’s Dance}; and in \textit{The Dancing Master}(1698), \textit{Red Bull}.

\musicinfo{Rather slow.}{}

\includemusic{chappellV1157.pdf}

The last eight bars are repeated for four more lines in the stanza. The whole
is reprinted in Evans’ \textit{Old Ballads}, i. 312 (1810).
\pagebreak
%295

\musictitle{The Merry Milkmaids In Green.}

This is evidently the same air as \textit{And will he not come again}, one of the snatches
sung by Ophelia in \textit{Hamlet}, but in a different form (see p. 237). It is contained
in every edition of \textit{The Dancing Master}. In the eighteenth edition it is entitled
“The merry Milkmaids in \textit{green},” to distinguish it from another air of similar
name.

In Sir Thomas Overbury’s \textit{Character of a Milkmaid}, he says, “She dares go
alone, and unfold her sheep in the night, and fears no manner of ill, because she
means none: yet, to say truth, she is never alone, \textit{she is still accompanied with old
songs}, honest thoughts, and prayers, but short ones.”

In the “Character of a Ballad-monger,” in \textit{Whimzies, or a new Cast of
Characters}, 12mo., 1631, we find, “Stale ballad news, cashiered the city, must
now ride post for the country, where it is no less admired than a giant in a
pageant: till at last it grows so common there too, as every poor milkmaid\textit{ can
chant and chirp it under her cow}, which she useth, as a harmless charm, to make
her let down her milk.”

Maudlin, the milkmaid, in Walton’s \textit{Angler}, sings (among others) portions of
two ballads by Martin Parker, a well-known ballad-writer of the latter part of the
reign of James I., and during that of Charles and the Protectorate, and both are
to this tune. The first is—

“The Milkemaid’s Life; or—
\settowidth{\versewidth}{The praise of the milking paile to defend:}
\begin{scverse}
A pretty new ditty, composed and pen’d\\
The praise of the milking paile to defend:
\end{scverse}
to a curious new tune, called \textit{The Milkemaid’s Dumps}.” (Roxburghe Coll., i. 244,
or Collier’s \textit{Roxburghe Ballads}, 243.) Mr. Payne Collier remarks that the last
stanza but one proves it to have been written before “the downfall of May-games”
under the Puritans.

\settowidth{\versewidth}{That woods and fields possess,}
\begin{dcverse}
\indentpattern{0011011011110}
\begin{patverse}
You rural goddesses,\\
That woods and fields possess,\\
Assist me with your skill,\\
That may direct my quill\\
More jocundly to express\\
The mirth and delight.\\
Both morning and night.\\
On mountain or in dale,\\
Of those who choose\\
This trade to, use,\\
And through cold dews\\
Do never refuse\\
To carry the milking pail.
\end{patverse}

\begin{patverse}
The bravest lasses gay\\
Live not so merry as they;\\
In honest civil sort\\
They make each other sport,\\
As they trudge on their way.\\
Come fair or foul weather,\\
They’re fearful of neither—\\
Their courages never quail;\\
In wet and dry,\\
Though winds be high,\\
And dark’s the sky.\\
They ne’er deny\\
To carry the milking pail.
\end{patverse}

\begin{patverse}
Their hearts are free from care,\\
They never will despair;\\
Whatever may befall,\\
They bravely bear out all.\\
And Fortune’s frowns out-dare.\\
They pleasantly sing\\
To welcome the Spring—\\
’Gainst heaven they never rail;\\
If grass well grow.\\
Their thanks they show;\\
And, frost or snow,\\
They merrily go\\
Along with the milking pail.
\end{patverse}
\end{dcverse}
\pagebreak
%296

\indentpattern{0011011011110}
\begin{dcverse}
\begin{patverse}
Bad idleness they do scorn;\\
They rise very early i’ th’ morn.\\
And walk into the field,\\
Where pretty birds do yield\\
Brave music on ev’ry thorn:\\
The linnet and thrush\\
Do sing on each bush.\\
And the dulcet nightingale\\
Her note doth strain\\
In a jocund vein,\\
To entertain\\
That worthy train\\
Which carry the milking pail.
\end{patverse}

\begin{patverse}
Their labour doth health preserve,\\
No doctors' rules they observe;\\
While others, too nice\\
In taking their advice,\\
Look always as though they would starve;\\
Their meat is digested.\\
They ne’er are molested,\\
No sickness doth them assail;\\
Their time is spent\\
In merriment;\\
While limbs are lent,\\
They are content\\
To carry the milking pail.
\end{patverse}

\begin{patverse}
Those lasses nice and strange,\\
That keep shops in the Exchange,\\
Sit pricking of clouts;\\
And giving of flouts;\\
They seldom abroad do range:\\
Then comes the green sickness\\
And changeth their likeness,\\
All this for want of good sale;\\
But ’tis not so,\\
As proof doth show,\\
By those that go\\
In frost and snow\\
To carry the milking pail.
\end{patverse}

\indentpattern{0011011011100}

\begin{patverse}
If they any sweethearts have\\
That do affection crave,\\
Their privilege is this,\\
Which many others miss:—\\
They can give them welcome brave.\\
With them they may walk,\\
And pleasantly talk,\\
With a bottle of wine or ale;\\
The gentle cow\\
Doth them allow,\\
As they know how.\\
God speed the plough,\\
And bless the milking pail.
\end{patverse}

\indentpattern{0011011011110}

\begin{patverse}
Upon the first of May,\\
With garlands fresh and gay;\\
With mirth and music sweet,\\
For such a season meet.\\
They pass their time away:\\
They dance away sorrow,\\
And all the day \textit{thorow},\\
Their legs do never fail;\\
They nimblely\\
Their feet do ply,\\
And bravely try\\
The victory,\\
In honour o’ th’ milking pail.
\end{patverse}

\begin{patverse}
If any think that I\\
Do practice flattery,\\
In seeking thus to raise\\
The merry milkmaids’ praise,\\
I’ll to them thus reply:\\
It is their desert\\
Inviteth my art\\
To study this pleasant tale;\\
In their defence,\\
Whose innocence\\
And providence\\
Gets honest pence\\
Out of the milking pail.
\end{patverse}
\end{dcverse}

There is another version of the above ballad in the Roxburghe Collection
(ii.~230), entitled “The innocent Country Maid’s Delight; or a Description of
the lives of the Lasses of London: set to \textit{an excellent Country Dance}.” It commences
with the lines quoted by the milkmaid from the above sixth stanza:
\settowidth{\versewidth}{That keep shop in the Exchange.”}
\begin{scverse}
\vleftofline{“}Some lasses are nice and strange\\
That keep shop in the Exchange.”
\end{scverse}
\pagebreak
%297

The second ballad quoted by Maudlin is entitled “Keep a good tongue in your
head; or—
\settowidth{\versewidth}{But only her tongue breeds all her defect:}
\begin{scverse}
Here’s a good woman, in every respect,\\
But only her tongue breeds all her defect:
\end{scverse}

to the tune of \textit{The Milkmaids},” \&c. (Roxburghe Coll., i. 510, or Collier’s \textit{Roxburghe
Ballads}, 237.) From this I have selected a few stanzas to print with the
tune. It is sometimes referred to under its name, as in the following:—

“Hold your hands, honest men: for—
\settowidth{\versewidth}{Here’s a good wife hath a husband that likes her,}
\begin{scverse}
Here’s a good wife hath a husband that likes her,\\
In every respect, but only he strikes her;\\
Then if you desire to be held men complete.\\
Whatever you do, your wives do not beat.
\end{scverse}

To the tune of \textit{Keepe a good tongue},” \&c. (Roxburghe, i. 514.) The following
song by D’Urfey, entitled \textit{The Bonny Milkmaid}, was also written to the tune, but
had afterwards music composed to it for his play of \textit{Don Quixote}, and is so printed
in both editions of \textit{ Pills to purge Melancholy}, and in \textit{The Merry Musician, or
A Cure for the Spleen}, ii. 116. It is a rifacimento of Martin Parker’s song
printed above.

\settowidth{\versewidth}{That love green fields and woods,}
\begin{dcverse}\begin{patverse}
Ye nymphs and sylvan gods,\\
That love green fields and woods,\\
Where Spring, newly blown,\\
Herself does adorn\\
With flow’rs and blooming buds:\\
Come sing in the praise,\\
Whilst flocks do graze\\
In yonder pleasant vale,\\
Of those that choose\\
Their sleep to lose,\\
And in cold dews,\\
With clouted shoes,\\
Do carry the milking pail.
\end{patverse}

\begin{patverse}
The goddess of the morn\\
With blushes they adorn,\\
And take the fresh air,\\
Whilst linnets prepare\\
A concert in each green thorn.\\
The blackbird and thrush\\
On every bush,\\
And charming nightingale,\\
In merry vein\\
Their throats do strain\\
To entertain\\
The jolly train\\
That carry the milking pail.
\end{patverse}

\begin{patverse}
When cold bleak winds do roar\\
And flow’rs can spring no more.\\
The fields that were seen\\
So pleasant and green\\
By Winter all candied o’er:\\
Oh! how the town lass\\
Looks, with her white face\\
And lips so deadly pale;\\
But it is not so\\
With those that go\\
Through frost and snow,\\
With cheeks that glow,\\
To carry the milking pail.
\end{patverse}

\begin{patverse}
The country lad is free\\
From fear and jealousy,\\
When upon the green\\
He is often seen\\
With a lass upon his knee;\\
With kisses most sweet\\
He does her greet,\\
And swears she'll ne’er grow stale;\\
While the London lass,\\
In every place.\\
With her brazen face,\\
Despises the grace\\
Of those with the milking pail.
\end{patverse}
\end{dcverse}

“The Merry Milkmaid’s Delight” was one of the ballads printed by
W. Thackeray, in the time of Charles II.

The following stanzas are selected from the ballad above-mentioned, “Keep
a good tongue in your head.”
\pagebreak
%298

\musicinfo{Cheerfully.}{}

\includemusic{chappellV1158.pdf}

\indentpattern{0011011011110}
\settowidth{\versewidth}{Between her round chin and her n}
\begin{dcverse}\begin{patverse}
Her cheeks are red as the rose\\
Which June for her glory shows;\\
Her teeth in a row\\
Stand like a wall of snow\\
Between her round chin and her nose;\\
Her shoulders are decent,\\
Her arms white and pleasant,\\
Her fingers are small and long.\\
No fault I find,\\
But, in my mind,\\
Most womenkind\\
Must come behind:\\
O that she could rule her tongue!
\end{patverse}

\begin{patverse}
With eloquence she will dispute;\\
Few women can her confute.\\
\textit{She sings and she plays,\\
And she knows all the keys}\\
Of the vial de gambo, or lute.\\
She’ll dance with a grace,\\
Her Measures she’ll trace\\
As doth unto art belong;\\
She is a girl\\
Fit for an earl,\\
Not for a churl:\\
She were worth a pearl,\\
If she could but rule her tongue.
\end{patverse}
\end{dcverse}
\pagebreak