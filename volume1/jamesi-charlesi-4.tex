%319
\changefontsize{1.03\defaultfontsize}

\musictitle{The Fairest Nymph The Valleys.}

This, like \textit{In sad and ashy weeds} (p. 202), or like \textit{Fear no more the heat of the
sun}, in Shakespeare’s \textit{Cymbeline}, is a sort of dirge, a mourning or funeral song.
The copy in the Roxburghe Collection, i. 330, is entitled “The Obsequy of
Faire Phillida: with the Shepherds’ and Nymphs’ Lamentation for her losse.
\textit{To a new court tune}.” The music is contained in a MS. volume of virginal
music transcribed by Sir John Hawkins, and in Starter’s \textit{Friesche Lust-Hof},
1634, under its English name. In the library of the British Museum there is a
copy of “Psalmes or Songs of Sion, turned into the language and set to the tunes
of a Strange Land, by W[illiam] S[latyer], intended for Christmas Carols, and
fitted to divers of the most noted and common, but solemne tunes, every where
in this land familiarly used and knowne.” 1642. Upon this copy a former
possessor had written the names of the tunes to which they were designed to be
sung. These are, \textit{The fairest Nymph the valleys; All in a garden green; Bara
Faustus’ Dreame; Crimson velvet; What if a day, or a month, or a year? Fair
Angel of England; Dulcina; Walsingham}; and \textit{Jane Shore}.\scfootnote
{All the tunes here named will be found in this Collection.}

\musicinfo{With expression.}{}

\medskip

\includemusic{chappellV1169.pdf}

\pagebreak
%320

\settowidth{\versewidth}{Come, fatal sisters, leave your spools,}
\indentpattern{0122101221011221}
\begin{dcverse}
\begin{patverse}
The sheep for woe go bleating,\\
That they their goddess miss,\\
And sable ewes,\\
By their mourning, shew\\
Her absence, cause of this.\\
The nymphs leave off their dancing,\\
Pan’s pipe of joy is cleft,\\
For great his grief,\\
He shunneth all relief,\\
Since she from him is reft.\\
Come, fatal sisters, leave your spools,\footnote{\textit{}
A spool to wind yarn upon.}\\
Leave ‘weaving’ altogether,\\
That made this flower to wither.\\
Let envy, that foul vipress,\\
Put on a wreath of cypress,\\
Sing sad dirges altogether.
\end{patverse}

\begin{patverse}
Diana was chief mourner\\
At these sad obsequies,\\
Who with her train\\
Went tripping o’er the plain,\\
Singing doleful elegies.\\
Menalchus and Amintas,\\
And many shepherds moe,\footnote{\textit{}
More.}\\
With mournful verse,\\
Did all attend her hearse,\\
And in sable saddles go.\\
Flora, the goddess that us’d to beautify\\
Fair Phillis’ lovely bowers\\
With sweet fragrant flowers,\\
Now her grave adorned,\\
And with flowers mourned,\\
Tears thereon in vain she pours.
\end{patverse}

\begin{patverse}
Venus alone triumphed\\
To see this dismal day,\\
Who did despair\\
That Phillida the fair\\
Her laws would ne’er obey.\\
The blinded boy his arrows\\
And darts were vainly spent;\\
Her heart, alas,\\
Impenetrable was,\\
And to love would ne’er assent.\\
At which affront, Citharea repining,\\
Caus’d Death with his dart\\
To pierce her tender heart;\\
But her noble spirit\\
Doth such joys inherit,\\
‘As’ from her shall ne’er depart.
\end{patverse}
\end{dcverse}

\musictitle{Hunting The Hare.}

\settowidth{\versewidth}{Was al his lust, for no cost wolde he spare.”}
\begin{scverse}
\vleftofline{“}Of prikyng and of hunting for the Hare\\
Was al his lust, for no cost wolde he spare.”\\
\attribution \textit{Chaucer’s Description of a Monk}.
\end{scverse}

Hunting has always been so favorite an amusement with the English, that the
great variety of songs upon the subject will excite no surprise. Those I have
printed, of the reign of Henry VIII., relate either to deer or fox-hunting; but
Henry was no less careful of the minor sport, as may be seen by an act of
Parliament (passed anno 14-15 of his reign), entitled “An Act concerning
the Hunting of the Hare.” It recites that, “For as muche as oure Soveraigne
Lorde the Kinge, and other noblemen of this realme, before this time hath
used and exercised the game of huntynge the hare, for their disporte and
pleasure, which game is now decayed and almost utterly dystroied for that
divers parties of this realme, by reason of the \textit{trasinge in the snow}, have killed
and destroied, and dayly do kille and distroy the same hares, by fourteen or sixteen
upon a daye, to the dyspleasure of our Soveraigne Lorde the Kinge and
other nobleman,” \&c.; therefore the act fixes a penalty of six shillings and eightpence
(a large sum in comparison with the value of the hares in those days) for 
every one so killed. Henry seems, also, \pagebreak to have considered the sale of hunting-horns 
%321
of sufficient importance, as a source of revenue, to affix an export duty of
four shillings per dozen upon them.\footnote{\textit{}
This will be found in “The Rates of the Custome
House, both inwarde and outwarde, very necessarye
for all Merchantes to knowe. Imprinted at London, by
me, Rycharde Kele, dwellynge at the longe shoppe in the
Poultrye, under Saynt Myldreds Churche.” 1545. Among
the \textit{import} duties relating to music, will be found—
“Clarycordes, the payre, 2\textit{s}.; Harpe Strynges, the boxe,
10\textit{s}.; Lute Strynges, called Mynikins, the groce, 22\textit{d}.;
Orgons, the payre, \textit{ut sint in valore}; Wyer for Clarycordes,
the pound, 4\textit{d}.; Virginales, the payre, 3\textit{s}. 4\textit{d}.;
Whisteling Bellowes, the groc, 8\textit{s.}}


“A Songe of the huntinge and killinge of the Hare” was entered on the
registers of the Stationers’ Company, to Richard Jones, on June 1, 1577, but the
entry contains no clue to the words, or to the air.

The tune of the present song may be traced back to the reign of James I.;
but, both in his reign, and in that of his predecessor, hunting was so favorite a
sport, and hunting songs so generally popular, that the introduction of either on
the stage was thought a good means of assisting the success of a play.

Wood tells us that in Richard Edwardes’ comedy of \textit{Palæmon and Arcyte}
(which was performed before Queen Elizabeth, in Christ Church Hall, Oxford, on
the 2nd and 3rd September, 1566) “A cry of hounds was acted in the quadrant
upon the train of a fox, in the hunting of Theseus; with which the young
scholars, who stood in the remoter part of the stage and windows, were so much
taken and surprised, supposing it to be real, that they cried out, ‘There, there—he’s 
caught, he’s caught! ’ All which the Queen, merrily beholding, said,
‘Oh, excellent! These boys, in very truth, are ready to leap out of the windows
to follow the hounds.’”

James was passionately fond of hunting; and Anthony Munday, in his play,
\textit{The Downfall of Robert, Earl of Huntington}, thus deprecates his displeasure and
that of the audience for not having introduced hunting songs, or resorted to the
other usual expedients to ensure applause. In act iv., sc. 2, Little John says—
\settowidth{\versewidth}{No pleasant skippings up and down the wood;}
\begin{scverse}
\begin{altverse}
\vleftofline{“}Methinks I see no jests of Robin Hood;\\
No merry Morrices of Friar Tuck;\\
No pleasant skippings up and down the wood;\\
No hunting songs; no coursing of the buck.\\
Pray God this play of ours may have good luck,\\
And the King’s Majesty mislike it not.”
\end{altverse}
\end{scverse}

I have printed one song on hare-hunting, of James’ reign (\textit{Master Basse his
Careere}, or \textit{The New Hunting of the Hare}), at p. 256. Another song, entitled
“The Hunting of the Hare, with her last will and testament,
\settowidth{\versewidth}{As it was performed on Bamstead Downs,}
\begin{scverse}
As it was performed on Bamstead Downs,\\
By coney-catchers and their hounds,”
\end{scverse}
was printed by Coles, Vere, and Wright, and will be found in Anthony à Wood’s
Collection. It commences thus—
\settowidth{\versewidth}{Whose echo shall, throughout the sky,}
\begin{scverse}
\vleftofline{“}Of all delights that earth doth yield,\\
Give me a pack of hounds in field,\\
Whose echo shall, throughout the sky,\\
Make Jove admire our harmony,\\
And wish that he a mortal were,\\
To share the pastime we have here.”
\end{scverse}
No tune is indicated in the copy, and \pagebreak it could not have been sung to this air.
%322

In \textit{Wit and Drollery}, and in several other publications, is a song, entitled
\textit{The Hunt}, commencing—
\settowidth{\versewidth}{Sweet is the breath, and fresh is the earth}
\begin{scverse}
\begin{altverse}
\vleftofline{“}Clear is the air, and the morning is fair,\\
Fellow huntsmen, come wind me your horn;\\
Sweet is the breath, and fresh is the earth\\
That melteth the rime from the thorn.”
\end{altverse}
\end{scverse}
\textit{Hunting the Hare} is also in the list of the songs and ballads printed by William
Thackeray, at the Angel in Duck Lane, in the early part of the reign of
Charles II., and it is, in all probability, the song to this tune (commencing—
\settowidth{\versewidth}{“Songs of shepherds, and rustical roundelays”),}
\begin{scverse}
“Songs of shepherds, and rustical roundelays”),
\end{scverse}
because the tune was then popular, and the words are to be found near that time
in \textit{Westminster Drollery}, part ii. (1672); as well as afterwards in \textit{Wit and
Drollery},~1682; in the \textit{Collection of Old Ballads}, 8vo., 1727; in \textit{Miscellany
Poems}, edited by Dryden, iii. 309 (1716); in Ritson’s, Dale’s, and other
Collections of English~Songs.

The first copy of the tune that I have discovered is in Playford’s \textit{Musick’s
Recreation on the Lyra Viol}, 1652; the second is in \textit{Musick’s Recreation on the
Viol, Lyra-way}, 1661. In both publications it is entitled \textit{Room for Cuckolds}.

Pennant speaking of Rychard Middleton (father of Sir Hugh Middleton), says,
“Thomas, the fourth son, became Lord Mayor of London, and was the founder of
the family of Chirk Castle. It is recorded that having married a young wife in
his old age, the famous song of \textit{Room for Cuckolds, here comes my Lord Mayor}!
was invented on the occasion.”—\textit{Pennant’s Tours in Wales}, ii. 152 (1810).
Thomas Middleton was Lord Mayor of London in 1614. Pennant gives the
Sebright MSS. as his authority for the anecdote.

In the Pepys Collection, i. 60, will be found, “A Scourge for the Pope;
satyrically scourging the itching sides of this obstinate brood, in England. To
the tune of \textit{Room for Cuckolds}.” It is one of Martin Parker’s early songs:
“Printed by John Trundle, at his shop in Smithfield,” and signed, “Per me,
Martin Parker.” Another song, which bears this title of the tune, is contained
in vol. xvi. of the King’s Pamphlets Brit. Mus., and dated in MS., 1659. It is
also quoted, by the same name, in \textit{Folly in print, or A Book of Rhymes}, 1667, in
the song, “Away from Romford, away, away.”

A third, and perhaps the earliest name for the air, is \textit{Room for Company};
apparently derived from a ballad in the Pepys Collection, i. 168, entitled and
commencing, “Room for Company, here comes good fellowes. \textit{To a pleasant new
tune}.” Imprinted at London for E. W. This was perhaps Edward White, a
ballad-printer of Elizabeth’s reign, and of the earliest part of that of James I.

In \textit{ Pills to purge Melancholy}, vi. 136, there is a song about the twelve great
Companies of the city of London, printed to this tune, and commencing—
\settowidth{\versewidth}{Room for gentlemen, here comes my Lord Mayor.”}
\begin{scverse}
\vleftofline{“}Room for gentlemen, here comes my Lord Mayor.”
\end{scverse}

In the Roxburghe Collection, i. 538, is, “The fetching home of May; or—
\begin{scverse}
\vleftofline{“}A pretty new ditty, wherein is made known,\\
How each lass doth strive for to have a green gown.
\end{scverse}

To the tune of \textit{Room for Company}.” \pagebreak Printed for J. Wright, jun., dwelling
%323
at the upper end of the Old Bailey (about 1663). It is also contained in the \textit{Antidote to Melancholy}, 1661; and in \textit{ Pills to purge Melancholy}, ii. 26 (1707),
or iv. 26 (1719).

The first stanza is subjoined, with the earlier version of the tune.

\musicinfo{Smoothly, and in moderate time.}{}

\includemusic{chappellV1170.pdf}

In the \textit{Antidote to Melancholy}, and in \textit{ Pills to purge Melancholy}, the above song
is printed under the title of \textit{The Green Grown}, a name derived from the last line of 
each stanza of the song. In \textit{Musick à-la-Mode}; \pagebreak or \textit{The young Maid’s Delight:
%324
containing five excellent new songs sung at the Drolls in Bartholomew Fair}, 1691,
there is another song, under the name of \textit{The Green Grown}, “to an excellent playhouse
tune.”

The tune of \textit{Hunting the Hare} is now in common use for comic songs, or for
such as require great rapidity of utterance; but it has also been employed as a
slow air. For instance, in Gay’s ballad-opera of \textit{Achilles}, 1733, it is printed
in \timesig{3}{4} time, and entitled “A Minuet.”

\musicinfo{Fast.}{Hunting the Hare}

\bigskip

\includemusic{chappellV1171.pdf}

\settowidth{\versewidth}{The earth old and ample, they soon leave the air;}
\indentpattern{0101330330}
\begin{dcverse}\begin{patverse}
Stars quite tir’d with pastimes Olympical,\\
Stars and planets which beautiful shone,\\
Could no longer endure that men only shall\\
Swim in pleasures, and they but look on;\\
Round about horned\\
Lucina they swarmed,\\
And her informed how minded they were,\\
Each god and goddess,\\
To take human bodies,\\
As lords and ladies, to follow the hare.
\end{patverse}
\columnbreak

\begin{patverse}
Chaste Diana applauded the motion,\\
While pale Proserpina sat in her place,\\
To light the welkin, and govern the ocean,\\
While she conducted her nephews in chase:\\
By her example,\\
Their father to trample,\\
The earth old and ample, they soon leave the air;\\
Neptune the water,\\
And wine Liber Pater,\\
And Mars the slaughter, to follow the hare.
\end{patverse}
\end{dcverse}
\pagebreak
%325
\changefontsize{0.96\defaultfontsize}

\settowidth{\versewidth}{Light god Cupid was mounted on Pegas}
\begin{dcverse}\begin{patverse}
Light god Cupid was mounted on Pegasus,\\
Lent by the Muses, by kisses and pray’rs;\\
Strong Alcides, upon cloudy Caucasus,\\
Mounts a centaur, which proudly him bears;\\
Postilion of the sky,\\
Light-heeled Mercury\\
Soon made his courser fly, fleet as the air;\\
Tuneful Apollo,\\
The kennel did follow,\\
And whoop and halloo, boys, after the hare.
\end{patverse}

\begin{patverse}
Drown’d Narcissus from his metamorphosis,\\
Rous’d by Echo, new manhood did take;\\
Snoring Somnus upstarted from Cimmeris,\\
Before, for a thousand years, he did not wake;\\
There was club-footed \\
Mulciber booted,\\
And Pan promoted on Corydon’s mare;\\
Proud Pallas pouted,\\
Loud Æolus shouted,\\
And Momus flouted, yet followed the hare.
\end{patverse}

\begin{patverse}
Hymen ushers the lady Astræa,\\
The jest took hold of Latona the cold;\\
Ceres the brown, with bright Cytherea;\\
Thetis the wanton, Bellona the bold;\\
Shame-fac’d Aurora,\\
With witty Pandora,\\
And Maia with Flora did company hear;\\
But Juno was stated\\
Too high to be mated,\\
Although she hated not hunting the hare.
\end{patverse}

\begin{patverse}
Three brown bowls to th’ Olympical rector,\\
The Troy-born boy presents on his knee;\\
Jove to Phoebus carouses in nectar,\\
And Phœbus to Hermes, and Hermes to me;\\
Wherewith infused, \\
I piped and I mused,\\
In language unused, their sports to declare:\\
Till the house of Jove\\
Like the spheres did move:—\\
Health to those who love hunting the hare!
\end{patverse}
\end{dcverse}

\backskip{1}

\musictitle{The Crossed Couple.}

This tune is referred to under three names, viz,, \textit{The Crossed Couple}, \textit{Hyde
Park}, and \textit{Tantara rara tantivee}.

The ballad of “The Crost Couple: to a new Northern tune much in fashion,”
is in the Roxburghe Collection, ii. 94. In the same volume, at p. 379, is “News
from Hide Park,” \&c., “to the tune of \textit{The Crost Couple}.”

The burden of “News from Hide Park” (as will be seen by the verse printed
below with the music) is \textit{Tantara rara tantive}e; and in the Bagford Collection
(p. 170), the tune is quoted under that name, in “A pleasant Dialogue betwixt
two wanton Ladies of Pleasure; or, The Duchess of Portsmouth’s woful farewell
to her former felicity.” This ballad is a supposed conversation between Nell
Gwyn and Louise Renée de Penencourt de Quérouaille (vulgarly, Madame
Carwell), whom Charles II. created Duchess of Portsmouth.

Nell Gwynn was as popular with the ballad-singers, from her many redeeming
qualities, as the Duchess of Portsmouth (being a Roman Catholic, and supposed
to send large sums of money to her relations in France) was out of favour with
them.\footnote{\textit{}
On the following page, in the same collection, there
is another Dialogue hetween the Duchess of Portsmouth
and Nell Gwyn, on the supposed intention of the former
to retire to France with the money she had acquired. It
is entitled, “Portsmouth’s Lamentation: Or a Dialogue
between two amorous Ladies, E. G. and D. P.
\settowidth{\versewidth}{Dame Portsmouth was design’d for France}
\begin{fnverse}
\vin \vleftofline{“}Dame Portsmouth was design’d for France\\
\vin But therein was prevented;\\
\vin Who mourns at this unhappy chance,\\
\vin And sadly doth lament it.\\
\vin\vin To the tune of \textit{Tom the Taylor}, or \textit{Titus Oates}’’
\end{fnverse}
It commences thus:—
\begin{fnverse}
\vin\vleftofline{“}I prithee, Portsmouth, tell me plain,\\
\vin\vin  Without dissimulation,\\
\vin  When dost thou home return again,\\
\vin\vin  And leave this English nation?\\
\vin Your youthful days are past and gone,\\
\vin\vin  You plainly may perceive it,\\
\vin Winter of age is coming on,\\
\vin\vin ’Tis true—you may believe it.”
\end{fnverse}
Nine stanzas, “Printed for C. Dennisson, at the Stationers
Arms, within Aldgate.”}
 The ballad commences thus:—
 
 \backskip{0.5}
 
\settowidth{\versewidth}{Of a pleasant discourse that I heard at Pell-Mell,}
\begin{scverse}\begin{altverse}
“Brave gallants, now listen, and I will you tell,\\
With a fa la la, la fa, la la,\\
Of a pleasant discourse that I heard at Pell-Mell,\\
With a fa la la, la fa, la la, \&c.
\end{altverse}
\end{scverse}

\pagebreak
%326
\changefontsize{0.98\defaultfontsize}

The ballad of \textit{News from Hide Park} is also printed, with the tune, in \textit{Pills to
purge Melancholy}, ii. 138 (1700 and 1707). Cunningham, in his \textit{Hand-book of
London}, says of Hyde Park:—“In 1550, the French Ambassador hunted there
with the King; in 1578, the Duke Casimer ‘killed a barren doe with his piece,
in Hyde Park, from amongst 300 other deer.’ In Charles the First’s reign, it
became celebrated for its foot and horse races round the Ring; in Cromwell’s
time, for its musters and coach races; in Charles the Second’s reign, for its drives
and promenades—a reputation which it still retains.” (Edit. 1850, p. 241.)
This ballad was printed in the reign of Charles II. The following are the three
first stanzas.

\musicinfo{Gaily.}{}

\includemusic{chappellV1172.pdf}

\settowidth{\versewidth}{The Park shone brighter than the skies,}
\indentpattern{01010001}
\begin{dcverse}\begin{patverse}
The Park shone brighter than the skies,\\
Sing tantara rara tantivee,\\
With jewels, and gold, and ladies’ eyes,\\
That sparkled and cried, “Come see me;”\\
Of all parts of England Hyde Park hath the name\\
For coaches, and horses, and persons of fame;\\
It look’d, at first sight, like a field full of flame,\\
Which made me ride up tantivee.
\end{patverse}

\begin{patverse}
There hath not been such a sight since Adam’s,\\
For perriwig, ribbon, and feather;\\
Hyde Park may be termed the market for madams,\\
Or lady-fair, choose yon whether. \\
Their gowns were a yard too long for their legs,\\
They show’d like the rainbow cut into rags,\\
A garden of flowers, or navy of flags,\\
When they did all mingle together.
\end{patverse}
\end{dcverse}

Another tune called \textit{Hide Park} is to be found in the earliest editions of \textit{The
Dancing Master}, and there are ballads in a different metre, such as “A new ditty
of a Lover, tost hither and thither, that cannot speak his mind when they are
together,” by Peter Lowberry (Roxburghe, i. 290); commencing thus:—
\pagebreak
%327
\settowidth{\versewidth}{Alas! I am in love,}
\begin{dcverse}\begin{altverse}
\vleftofline{“}Alas! I am in love,\\
And cannot speak it;\\
My mind I dare not move,\\
Nor ne’er can break it.
\end{altverse}

\begin{altverse}
She doth so far excel\\
All and each other,\\
My mind I cannot tell,\\
When we’re together.”
\end{altverse}
\end{dcverse}

In the Pepys Collection, i. 197, is a ballad, “The Defence of Hide Parke from
some aspersions cast upon her, tending to her great dishonour: \textit{To a curious new
Court tune}” It is in ten-line stanzas, and commences, “When glistering Phœbus.”
“Printed at London for H[enry] G[osson].” Also, at i. 188, “The praise of
London: or, A delicate new Ditty, which doth invite you to faire London City.
To the tune of the \textit{second part of Hide Parke}.”

In \textit{Westminster Drollery}, 1671, there is another song called “\textit{Hide Park}: the
tune, \textit{Honour invites you to delights—Come to the Court, and be all made Knights};”
commencing—

\settowidth{\versewidth}{Come, all you noble,}
\begin{scverse}\vleftofline{“}Come, all you noble,\\
You that are neat ones,” \&c.
\end{scverse}

A copy of the ballad, \textit{Come to the Court, and be all made Knights}, will be found in
Addit. MSS., Brit. Mus., No. 5,832, fol. 205, entitled “Verses upon the Order
for making Knights of such persons who had 40\textit{l}. per annum, in King James
the First’s time.” Both James I. and Charles I. resorted to this obnoxious expedient
for raising money. According to John Philipot, Somerset Herald, in his
\textit{Perfect Collection or Catalogue of all Knights Batchelours made by King James,
since his coming to the Crown of England}, 1660, James I. created 2,323 Knights,
of whom 900 were made the first year of his reign.
\settowidth{\versewidth}{Carters, ploughmen, hedgers, and all;}
\begin{dcverse}\footnotesizerr
\indentpattern{01010010}
\begin{patverse}
\vleftofline{“}Come all you farmers out of the country,\\
Carters, ploughmen, hedgers, and all;\\
Tom, Dick, and Will, Ralph, Roger, and Humphrey,\\
Leave off your gestures rusticall.\\
Bid all your home-spun russets adieu,\\
And suit yourselves in fashions new;\\
\textit{Honour invites you to delights—\\
Come all to Court, and be made Knights.}
\end{patverse}

\begin{patverse}
He that hath forty pounds per annum\\
Shall be promoted from the plough;\\
His wife shall take the wall of her grannum,\\
Honour is sold so dog-cheap now.\\
Though thou hast neither good birth nor breeding.\\
If thou hast money thou’rt sure of speeding.\\
\textit{Honour invites you}, \&c.
\end{patverse}

\begin{patverse}
Knighthood, in old time, was counted an honour,\\
Which the blest spirits did not disdain;\\
But now it is used in so base a manner,\\
That it’s no credit, but rather a stain.\\
Tush, it’s no matter what people do say,\\
The name of a Knight a whole village will sway.\\
\textit{Honour invites you}, \&c.
\end{patverse}
\columnbreak

\begin{patverse}
Shepherds, leave singing your pastoral sonnets,\\
And to learn compliments shew your endeavours;\\
Cast off for ever your two shilling bonnets,\\
Cover your coxcombs with three pound beavers.\\
Sell cart and tar-box, new coaches to buy,\\
Then, ‘Good, your worship,’ the vulgar will cry.\\
\textit{Honour invites you}, \&c.
\end{patverse}

\begin{patverse}
And thus unto worship being advanced,\\
Keep all your tenants in awe with your frowns,\\
And let your rents be yearly enhanced,\\
To buy your new-moulded madams new gowns.\\
Joan, Siss, and Nell, shall all be ladyfied,\\
Instead of hay-carts, in coaches shall ride.\\
\textit{Honour invites you}, \&c.
\end{patverse}

\begin{patverse}
Whatever you do, have a care of expences;\\
In hospitality do not exceed;\\
Greatness of followers belongeth to princes,\\
A coachman and footman are all that you need.\\
And still observe this—Let your servants meat lack,\\
To keep brave apparel upon your wife’s back.\\
\textit{Honour invites you},” \&c
\end{patverse}.
\end{dcverse}

\pagebreak
%328
\changefontsize{1.02\defaultfontsize}

Another version of this ballad is printed in the Rev. Joseph Hunter’s \textit{History
of Sheffield} (p. 104), from “a small volume of old poetry in the Wilson Collections.”
It is there entitled, “Verses on account of King \textit{Charles the First} raising
money by Knighthood, 1630.” Shepherds are said to wear ten-penny, instead of
“two shilling,” bonnets in that version; and it has the following concluding
stanza;—
\settowidth{\versewidth}{Now to conclude and shut up my sonnet,}
\begin{scverse}
\begin{altverse}
\vleftofline{“}Now to conclude and shut up my sonnet,\\
Leave off the cart, whip, hedge-hill, and flail;\\
This is my counsel, think well upon it,\\
Knighthood and honour are now put to sale.\\
Then make haste quickly, and let out your farms,\\
And take my advice in blazing your arms.\\
\vin\vin\vin \textit{Honour invites you},” \&c.
\end{altverse}
\end{scverse}

The above would suit the tune of \textit{Hunting the Hare}.

\musictitle{New Mad Tom Of Bedlam, or Mad Tom.}

The earliest printed copy hitherto discovered of the music of this celebrated
song, which retains undiminished popularity after a lapse of more than two centuries, 
is to be found in the first edition of \textit{The English Dancing Master}, 1650-51.
This is one of the earliest known publications by Playford, before whose time music
was sparingly printed, and small pieces, such as songs, ballad and dance tunes, or
lessons for the virginals, were chiefly to be bought in manuscript, as they are in
many parts of Italy at the present time. In the first edition of \textit{The Dancing
Master} the tune is called \textit{Gray's-Inne Maske}, and in later editions (for instance,
the fourth, printed in 1670) \textit{Gray's-lnne Maske}; or, \textit{Mad Tom}. The blackletter
copies of the ballad, in the Pepys Collection (i. 502); in the Bagford
(643, m. 9, p. 52); and the Roxburghe (i. 299), are entitled \textit{New Mad Tom of
Bedlam}; or,—
\settowidth{\versewidth}{The Man in the Moone drinks claret}
\begin{scverse}
\vleftofline{“}The Man in the Moone drinks claret,\footnote{\textit{}
The ballad is usually printed with another, which is also
entitled “The New Mad Tom; or, The Man in the Moon
drinks Claret, as it was lately sung at the Curtain, Holywell, 
to the same tune.” The Curtain Theatre (according
to Malone and Collier) was in disuse at the commencement
of the reign of Charles I. (1625). This ballad has
three long verses, in the same measure, and evidently intended
to be sung to the same music. The first is as
follows:—
\settowidth{\versewidth}{Bacchus, the father of drunken nowls,}
\indentpattern{00000000101220220112212200000}
\begin{fnverse}
\vleftofline{“}Bacchus, the father of drunken nowls,\\
\begin{patverse}
Full mazers, beakers, glasses, bowls,\\
Greezie flap-dragons, Flemish upsie freeze,\\
With health stab’d in arms upon naked knees;\\
Of all his wines he makes you tasters.\\
So you tipple like bumbasters;\\
Drink till you reel, a welcome he doth give;\\
O bow the boon claret makes you live;\\
Not a painter purer colours shows\\
Then what’s laid on by claret.\\
Pearl and ruby doth set out the nose,\\
When thin small beer doth mar it;\\
Rich wine is good,\\
It heats the blood,\\
It makes an old man lusty.\\
The young to brawl,\\
And the drawers up call,\\
Before being too much musty.\\
Whether you drink all or little,\\
Pot it so yourselves to wittle;\\
Then though twelve\\
A clock it be,\\
Yet all the way go roaring.\\
If the band\\
Of bills cry stand,\\
Swear that you must a ---\\
Such gambols, such tricks, such fegaries,\\
We fetch though we touch no canaries;\\
Drink wine till the welkin roars,\\
And cry out a --- of your scores.”
\end{patverse}
\end{fnverse}}\\
With powder'd beef, turnip, and carret,” \&c.\\
\attribution “The tune is \textit{Gray's-Inn Maske}”
\end{scverse}
 \pagebreak% moved as most of this footnote is on the next page
\changefontsize{0.99\defaultfontsize} 

It was formerly the custom of gentlemen of the Inns of Court to hold revels
four times a year,\footnote{\textit{}
Another curious custom, of obliging lawyers to \textit{dance}
four times a year, is quoted from Dugdale by Sir John
Hawkins. (\textit{History of Music}, vol. ii., p. 137.) “It is not
many years since the judges, in compliance with ancient
custom, danced annually on Candlemas-day. And, that
nothing might be wanting for their encouragement in this
excellent study (the law), they have very anciently had
dancings for their recreations and delight, commonly
called Revels, allowed at certain seasons; and that, by
special order of the society, as appeareth in 9 Hen. VI.,
there should be four Revels that year, and no more,” \&c.
And again he says, “Nor were these exercises of dancing
merely permitted, but thought very necessary, as it seems,
and much conducing to the making of gentlemen more fit
for their books at other times; for, by an order made 6th
Feb. 7 Jac., it appears that the under-barristers were by
decimation put out of Commons for example’s sake, because
the whole bar offended by not dancing on the
Candlemas-day preceding, according to the ancient order
of this society, when the judges were present; with this,
that if the like fault were afterwards committed, they
should be fined or disbarred.”}
 and to represent masks and plays in their own Halls, or elsewhere.
 %329 break above because...
A curious letter on the subject of a mask, which for some unexplained
reason did not take place, may he seen in Collier’s \textit{History of Early Dramatic
Poetry and Annals of the Stage}, vol. i., p.~268. It is addressed to Lord
Burghley, by “Mr. Frauncis Bacon” (afterwards Lord Bacon), who in 1588 discharged
the office of Reader of Gray’s Inn. Many other curious particulars of
their masks may he found in the same work, and some in Sir J. Hawkins’ \textit{History
of Music}. For the Christmas Revels of the bar, see Mr. Payne Collier’s note to
Dodsley’s Old Plays, vol. vii., p. 311. Lawyers are now, generally speaking, a
music-loving class. The enjoyment of sweet sounds is to many the most acceptable
recreation after long study. They were also famous in former days for
songs and squibs. Some, too, were tolerable composers, for every one claiming to
be a gentleman learnt music. As their compositions are rather out of my present
subject, I will refer only to their rhyming propensities; and, although much more
ample illustration might be given, two passages from letters of John Chamberlain
to Sir Dudley Carleton, printed in \textit{The Court of James I}. (1849), will probably
suffice. On May 20, 1615, Chamberlain says, “On Saturday last the King went
again to Cambridge to see the play, \textit{Ignoramus}, which hath so nettled the lawyers,
that they are almost out of all patience; and the Lord Chief Justice [Sir E.
Coke] both openly at the King’s Bench, and divers other places, hath galled and
glanced at scholars with much bitterness; and \textit{there be divers Inns at Court have
made rhymes and ballads against them}, which they have answered sharply enough.”
(i. 363.) Again in the letter of Nov. 23, 1616, “Here is a bold rhyme of
our young gallants of Inns of Court against their old benchers, and a pretty
epigram upon the Lord Coke, and no doubt more will follow; for when men are
down, the very drunkards make rhymes and songs upon them.” (i. 444.)

The authorship of the music of this song has been a subject of contention; and
so little have dates been regarded, that it has long passed as the composition of
Henry Purcell, and is still published with his name. Walsh paved the way to
this error (in which Ritson and many others followed), by including it in
a collection of “Mr. Henry Purcell’s Favourite Songs, out of his most celebrated
\textit{Orpheus Britannicus,} and the rest of his works.” It is \textit{not} contained in
the \textit{Orpheus Britannicus} (which was published by Purcell’s widow), and the music
may still be seen as \textit{printed} eight years before Purcell’s birth.

In a note upon the passage before quoted from Walton’s \textit{Angler}, Sir J.
Hawkins adds, “This song, beginning, ‘Forth from my dark and dismal cell,’
with the music to it, \textit{set by Henry Lawes}, is printed in a book, entitled \textit{Choice
Ayres, Songs, and Dialogues to sing to the Theorbo-Lute and Bass Viol}, fol. 1675;
and in Playford’s \textit{Antidote against Melancholy}, 8vo.,~1669.”

\pagebreak
%330

Sir John Hawkins must have had some reason, which he does not assign, for
attributing the composition to Henry Lawes. It is not contained in either of the
printed collections of Lawes’ songs, nor have I been able to find any copy with his
name attached to it. Sir John seems to be mistaken, because Lawes did not
enter the Chapel Royal until 1626, and the Curtain Theatre, at which \textit{one of the
songs to the tune were sung},\footnote{\textit{}
Mr. Payne Collier, in a note to Heber’s Catalogue,
Part iv., p. 92, says that this song was sung at the Curtain
Theatre, about 1610. In \textit{Choice Ayres}, 2nd edition, fol.,
1675, the composer’s name is not given, and it is printed
without any base.}
 was in disuse at the commencement of the reign of
Charles I. (1625). We must therefore look to an earlier composer.

One of the Addit. MSS., Brit. Mus. (No. 10,444) is a collection of Mask-
tunes, and there are several in that collection entitled “Gray’s Inn.” See
Nos. 50, 51, 91, 99, \&c. If Nos. 50 and 99 are from the same Mask (which is
not improbable), Mad Tom may be the composition of Lawes’ master, John
Cooper, called “Cuperario” after his visit to Italy. No. 50, the first of the
above tunes, is there called “\textit{Cuperaree}, or Gray’s Inn;” No. 51, “Gray’s In
Anticke Masque;” and No. 99 (the tune in question), “Gray’s Inne Masque.”

There is an equal uncertainty about the authorship of the words. In Walton’s
Angler, 1653, Piscator says, “I’ll promise you I’ll sing a song that was lately
made at my request by Mr. William Basse, one that made the choice songs of
\textit{The Hunter in his career}, and \textit{Tom of Bedlam}, and many others of note.” There
are, however, so many \textit{Toms of Bedlam}, that it is impossible to determine, from
this passage, to which of them Isaak Walton refers.

In addition to the broadsides, and a copy in \textit{Le Prince d’Amour}, 1660, there is
in MSS. Harl., No. 7,332, a version in the handwriting of “Fearegod Barebone, of
Daventry, in the county of Northampton,” who, “beinge at many times idle, and
wanting imployment, bestoed his time with his penn and incke wrighting thease
sonnets, songes, and epigrames, thinkinge that it weare bettar so to doe for the
mendinge of his hand in wrighting, then worse to bestow his time.” Master
Fearegod Barebone was, no doubt, a puritanical hypocrite; and wrote this excuse
about improving his handwriting, to be prepared in case the book should fall into
“ungodly hands.” No other inference can be drawn from his selection of some
of the songs in the manuscript. \textit{Mad Tom}, however, is not one of those objectionable
ditties, and, as being the oldest copy, I have here followed his manuscript.
The tune is from \textit{The Dancing Master}, and differs somewhat from later versions.

\textit{Mad Tom} was employed as a ballad tune in \textit{Penelope}, 1728; and \textit{The Bay’s
Opera}, 1730.

\musicinfo{Majestically.}{}

\includemusic{chappellV1173.pdf}

\pagebreak
%331

\includemusic{chappellV1174.pdf}

\pagebreak
%332
\changefontsize{1.0\defaultfontsize}

\settowidth{\versewidth}{Last night I heard the dog-star bark;}
\begin{dcverse}\indentpattern{0000000001012212211101010000}
\begin{patverse}
Last night I heard the dog-star bark;\\
Mars met Venus in the dark;\\
Limping Vulcan het an iron bar,\\
And furiously he ran at the god of war.\\
Mars with his weapons beset him about,\\
But Vulcan’s temples had the gout,\\
And his horns did hang so in his light,\\
He could not see to aim his blows aright.\\
Mercury, the nimble post of heaven,\\
Came to see the quarrel;\\
Gor-bellied Bacchus, giant-like,\\
Bestrid a strong-beer barrel.\\
To me he drank,\\
I did him thank,\\
But I could get no cider;\\
He drank whole buts,\\
Till he brake his guts,\\
But mine be never the wider.\\
Poor Tom is very dry:\\
A little drink for charity!\\
Now, hark! I bear Actæon’s hounds,\\
The huntsman whoops and halloos;\\
Ringwood, Roister, Bowman, Jowler,\\
And all the troop do follow.\\
The Man in the Moon drinks claret,\\
Eats powder’d beef, turnip, and carrot,\\
But a cup of old Malaga sack\\
Will fire the bush at his back.
\end{patverse}
\end{dcverse}

It will be observed that the second verse of the above is not now sung.
Another \textit{Mad Tom}, composed by George Hayden, and commencing, “In my
triumphant chariot hurl’d,” has been added to the first, to make a bravura. There
are even different copies of George Hayden’s song, some having a \timesig{9}{4} movement at
the close, which others have not. Hayden was the author of the still favorite
duet, “As I saw fair Clora.” He flourished in the early part of last century.

\musictitle{Tom A Bedlam.}

In \textit{Le Prince d’Amour}, 1660, there are no less than three songs entitled
\textit{Tom of Bedlam}; also Bishop Corbet’s song, \textit{The distracted Puritan}, which is to
the tune of \textit{Tom of Bedlam}.

The first song (at p. 164) consists of eight stanzas, and commences thus:—
\settowidth{\versewidth}{And them I bore twelve leagues and more,}
\begin{dcverse}\begin{altverse}
\vleftofline{“}From the top of high Caucasus,\\
To Paul’s Wharf near the Tower,\\
In no great haste, I easily pass’d\\
In less than half an hour.\\
The gates of old Byzantium\\
I took upon my shoulders,
\end{altverse}

\begin{altverse}
And them I bore twelve leagues and more,\\
In spite of Turks and soldiers, \\
\textit{Sing, sing, and sob; sing, sigh, and be merry;\\
Sighing, singing, and sobbing;\\
Thus naked Tom away doth run,\\
And fears no cold nor robbing}.
\end{altverse}
\end{dcverse}

The second is at p. 167, and consists also of eight stanzas, of which the two
first are as follows:—
\settowidth{\versewidth}{And the spirits, that stand by the naked man}
\begin{dcverse}\begin{altverse}
\vleftofline{“}From the hag and hungry goblin,\\
That into rags would rend you, \\
And the spirits, that stand by the naked man\\
In the book of moons, defend you;\\
That of your five sound senses\\
You never be forsaken,\\
Nor travel from yourselves with Tom\\
Abroad to beg your bacon.\\
\textit{While I do sing, ‘Any food, any feeding,\\
Feeding, drink, or clothing!\\
Come, dame or maid, be not afraid,\\
Poor Tom will injure nothing}.’
\end{altverse}

\begin{altverse}
Of thirty bare years have I\\
Twice twenty been enraged;\\
And, of forty, been three times fifteen\\
In durance soundly caged;\\
On the lordly lofts of Bedlam,\\
With stubble soft and dainty,\\
Brave bracelets strong, and whips, ding-dong\\
And wholesome hunger plenty.\\
\textit{Yet did I sing, ‘Any food, any feeding,\\
Feeding, drink, or clothing!\\
Come, dame or maid, be not afraid,\\
Poor Tom will injure nothing}.’”
\end{altverse}
\end{dcverse}

Ritson, who has reprinted the above two songs, supposes them “to have been
written by way of burlesque on such sort of things.” (\textit{Ancient Songs}, p. 261, 1790.)
\pagebreak
%333
\changefontsize{0.92\defaultfontsize}

The third song (p. 169) is now commonly known as \textit{Mad Tom}. It is in
another metre, and has a separate tune. (Ante p. 330.)

The fourth, commencing, “Am I mad, O noble Festus,” (p. 171), is here
printed to this~tune.

In the Roxburghe Collection, i. 42, there is a song on the tricks and disguises
of beggars, entitled “The cunning Northerne Begger:
\settowidth{\versewidth}{Who all the bystanders doth earnestly pray,}
\begin{scverse}Who all the bystanders doth earnestly pray,\\
To bestow a penny upon him to-day:
\end{scverse}
to the tune of \textit{Tom of Bedlam}.” The first stanza is as follows:—
\settowidth{\versewidth}{And weare all ragged garments!xxxxxx}
\begin{dcverse}\indentpattern{010010100101001}
\begin{patverse}
“I am a lusty begger,\\
And live by others giving;\\
I scorne to worke,\\
But by the highway lurke,\\
And beg to get my living.\\
I’ll i’ th’ wind and weather,\\
And weare all ragged garments!\\
\columnbreak
Yet, though I’m bare,\\
I’m free from care,\\
A fig for high preferments, \\
\textit{But still will I cry, ‘Good, your worship, good sir.\\
Bestow one poor denier, sir;\\
Which, when I’ve got,\\
At the pipe and the pot,\\
I soon will it cashier, sir}.’”
\end{patverse}
\end{dcverse}

This copy of the ballad was printed “at London” for F. Coules, and may be
dated as of the reign of Charles, or James I.

In \textit{Wit and Drollery}, 1656 (p. 126), there is yet another \textit{Tom of Bedlam},
beginning—
\settowidth{\versewidth}{Forth from the Elysian fields, a place of restless souls,}
\begin{scverse}\vleftofline{“}Forth from the Elysian fields, a place of restless souls,\\
\vin Mad Maudlin is come to seek her naked Tom,\\
Hell’s fury she controls,” \&c.
\end{scverse}
This is printed in an altered form, and with an imperfect copy of the tune, in \textit{
Pills to purge Melancholy}, ii. 192 (1700 and 1707), under the title of “Mad
Maudlin to find out Tom of Bedlam:”
\settowidth{\versewidth}{To find my Tom of Bedlam, ten thousand years I’ll travel;}
\begin{scverse}
\vleftofline{“}To find my Tom of Bedlam, ten thousand years I’ll travel;\\
Mad Maudlin goes, with dirty toes, to save her shoes from gravel.\\
\textit{Yet will I sing, Bonny boys, bonny mad boys, Bedlam boys are bonny;\\
They still go bare, and live by the air, and want no drink nor money}.”
\end{scverse}

The tune is again printed in \textit{ Pills to purge Melancholy}, iii. 13 (1707), to a song
“On Dr. G[ill?], formerly master of St. Paul’s School,” commencing—
\settowidth{\versewidth}{In Paul’s Churchyard in London,xxxx}
\begin{scverse}\indentpattern{0011023223}
\begin{patverse}
\vleftofline{“}In Paul’s Churchyard in London,\\
There dwells a noble firker,\\
Take heed, you that pass,\\
Lest you taste of his lash,\\
For I have found him a jerker:\\
\textit{Still doth he cry, take him up, take him up, sir,\\
Untruss with expedition;\\
O the birchen tool\\
Which he winds in the school\\
Frights worse than the Inquisition}.”
\end{patverse}
\end{scverse}

In \textit{Loyal Songs written against the Rump Parliament}, 1731, ii. 272, we have
“The cock-crowing at the approach of a Free Parliament; or—
\settowidth{\versewidth}{Than fig, raisin, or stewed prune is:}
\begin{dcverse}\indentpattern{010110}
\begin{patverse}
\vin Good news in a ballat\\
More sweet to your pallat\\
Than fig, raisin, or stewed prune is:\\
A country wit made it,\\
Who ne’er got the trade yet,\\
And \textit{Mad Tom of Bedlam} the tune is.”
\end{patverse}
\end{dcverse}

\pagebreak
%334
\changefontsize{0.98\defaultfontsize}

Among the King’s Pamphlets in the British Museum there are two songs to
this tune. The first (by a loyal Cavalier) is “Mad Tom a Bedlam’s desires of
Peace: Or his Benedicities for distracted England’s Restauration to her wits
again. By a constant though unjust sufferer (now in prison) for His Majesties
just Regality and his Country’s Liberty. S.F.W.B.” (Sir Francis Wortley,
Bart.) This is in the sixth vol. of folio broadsides, and dated June 27, 1648.
\settowidth{\versewidth}{Yet still he cries for the King, for the good King;}
\begin{dcverse}\begin{altverse}
\vleftofline{“}Poor Tom hath been imprison’d,\\
With strange oppressions vexed;\\
He dares boldly say, they try'd each way\\
Wherewith Job was perplexed.
\end{altverse}

\begin{altverse}
Yet still he cries for the King, for the good King;\\
Tom loves brave confessors; \\
But he curses those that dare their King depose,\\
Committees and oppressors.” \&c.
\end{altverse}
\end{dcverse}
This has been reprinted in Wright’s \textit{Political Ballads}, for the Percy Society,
p.~102; and in the same volume, p. 183, is another, taken from the fifteenth vol.
of broadsides, entitled “A new Ballade, to an old tune,—\textit{Tom of Bedlam},” dated
January 17, 1659, and commencing, “Make room for an honest red-coat.”

Besides these, we have, in \textit{Wit and Drollery}, 1682, p. 184, \textit{Loving Mad Tom},
commencing, “I’ll bark against the dog-star;” and many other mad-songs in the
Roxburghe Collection, such as “\textit{The Mad Man’s Morrice};” “\textit{Love’s Lunacie, or
Mad Besse’s Vagary};” \&c., \&c.

Bishop Percy has remarked that “the English have more songs on the subject
of madness, than any of their neighbours.” For this the following reason has
been assigned by Mr. Payne Collier, in a note to Dodsley’s Collection of Old
Plays, ii.~4:—

“After the dissolution of the religious houses, where the poor of every denomination
were provided for, there was for many years no settled or fixed provision made to
supply the want of that care which those bodies appear always to have taken of their
distressed brethren. In consequence of this neglect, the idle and dissolute were
suffered to wander about the country, assuming such characters as they imagined were
most likely to insure success to their frauds, and security from detection. Among
other disguises, many affected madness, and were distinguished by the name of
\textit{Bedlam Beggars}. These are mentioned by Edgar, in \textit{King~Lear}:
\settowidth{\versewidth}{The country gives me proof and precedent,}
\begin{scverse}\vleftofline{“}The country gives me proof and precedent,\\
Of \textit{Bedlam} beggars, who, with roaring voices,\\
Stick in their numb’d and mortify’d bare arms\\
Pins, wooden pricks, nails, sprigs of rosemary;\\
And, with this horrible object, from low farms,\\
Poor pelting villages, sheep-cotes, and mills,\\
Sometime with lunatic bans, sometime with prayer,\\
Inforce their charity.”
\end{scverse}

In Dekker’s \textit{Bellman} of London, 1616, all the different species of beggars are
enumerated. Amongst the rest are mentioned \textit{Tom of Bedlam}’s band of mad caps,
otherwise called Poor Tom’s flock of wild geese (whom here thou seest by his black
and blue naked arms to be a man beaten to the world), and those wild geese, or hair
brains, are called Abraham men. An Abraham man is afterwards described in this
manner: ‘Of all the mad rascals (that are of this wing) the \textit{Abraham man} is the
most fantastick. The fellow (quoth this old Lady of the Lake unto me) that sate
half naked (at table to-day) from the girdle upward, is the best \textit{Abraham man} that 
ever came to my house, and the notablest villain: \pagebreak he swears he hath been in Bedlam,
%335
and will talk frantickly of purpose: you see pins stuck in sundry places of his naked
flesh, especially in his arms, which pain he gladly puts himself to (being, indeed, no
torment at all, his skin is either so dead with some foul disease, or so hardened with
weather, only to make you believe he is out of his wits): he calls himself by the name
of \textit{Poor Tom}, and coming near any body, cries out, Poor Tom is a cold. Of these
\textit{Abraham men}, some be exceeding merry, and do nothing but sing songs, fashioned
out of their own brains, some will dance; others will do nothing but either laugh or
weep; others are dogged, and are sullen both in look and speech, that, spying but a
small company in a house, they boldly and bluntly enter, compelling the servants
through fear to give them what they demand, which is commonly \textit{Bacon}, or something
that will yield ready money.’”

The song of \textit{Tom of Bedlam} is alluded to in Ben Jonson’s \textit{The Devil is an Ass},
1616, act v., sc. 2. When Pug wishes to be thought mad, he says, “Your best
song’s Thom o’Bet’lem.”

The following copy of the tune is from a manuscript volume of virginal music,
formerly in the possession of Mr. Windsor, of Bath, and now in that of
Dr. Rimbault. It is entitled \textit{Tom a Bedlam}. The words are from Bishop
Corbet’s song, \textit{The distracted Puritan}, which is printed entire in Percy’s \textit{Reliques
of Ancient Poetry}.

\musicinfo{Pompously.}{}

\includemusic{chappellV1175.pdf}


\pagebreak
%336


\settowidth{\versewidth}{In the house of pure Emanuel}
\begin{dcverse}\indentpattern{001104}
\begin{patverse}
In the house of pure Emanuel\scfootnote
{Emanuel College, Cambridge, was originally a seminary of Puritans.}\\
I had my education,\\
Where my friends surmise\\
I dazzled my eyes\\
With the sight of revelation.\\
\textit{Boldly I preach}, \&c.
\end{patverse}

\begin{patverse}
They bound me like a bedlam,\\
They lash’d my four poor quarters;\\
Whilst this I endure,\\
Faith makes me sure\\
To be one of Fox’s martyrs.\\
\textit{Boldly I preach}, \&c.
\end{patverse}

\begin{patverse}
These injuries I suffer\\
Through antichrist’s persuasion:\\
Take off this chain,\\
Neither Rome nor Spain\\
Can resist my strong invasion.\\
\textit{Boldly I preach}. \&c.
\end{patverse}

\begin{patverse}
Of the beast’s ten horns (God bless us!)\\
I have knock’d off three already;\\
If they let me alone\\
I’ll leave him none:\\
But they say I am too heady.\\
\textit{Boldly I preach}, \&c.
\end{patverse}

\begin{patverse}
When I sack’d the seven hill’d city,\\
I met the great red dragon;\\
I kept him aloof\\
With the armour of proof,\\
Though here I have never a rag on.\\
\textit{Boldly I preach}, \&c.
\end{patverse}

\begin{patverse}
With a fiery sword and target,\\
There fought I with this monster:\\
But the sons of pride\\
My zeal deride,\\
And all my deeds misconster.\\
\textit{Boldly I preach}, \&c.
\end{patverse}

\begin{patverse}
I un-hors’d the Whore of Babel,\\
With the lance of Inspiration;\\
I made her stink,\\
And spill the drink\\
In her cup of abomination.\\
\textit{Boldly I preach}, \&c.
\end{patverse}

\begin{patverse}
I appear’d before the archbishop,\\
And all the high commission;\\
I gave him no grace,\\
But told him to his face,\\
That he favour’d superstition.\\
\textit{Boldly I preach}. \&c.
\end{patverse}
\end{dcverse}

\musictitle{Thomas, You Cannot.}

This tune is contained in Sir John Hawkins’ Transcripts of Virginal Music; in
the fourth and later editions of \textit{The Dancing Master}; in \textit{The Beggars’ Opera};
\textit{The Mock Doctor}; \textit{An Old Man taught Wisdom}; \textit{The Oxford Act}; and other
ballad-operas.

In some of the earlier editions of \textit{The Dancing Master}, it is entitled\textit{ Thomas,
you cannot}; in others, \textit{Tumas, I cannot}, or \textit{Tom Trusty}; in some of the ballad-operas
(for instance, \textit{The Generous Freemason}, and \textit{The Lover his own Rival}),
\textit{Sir Thomas, I cannot}.

In the Pepys Collection, i. 62, is a black-letter ballad (one of the many written
against the Roman Catholics after the discovery of the Gunpowder Plot, in~1605),
entitled “A New-yeeres-Gift for the Pope; O come see the difference plainly
decided between Truth and Falsehood:
\settowidth{\versewidth}{Not all the Pope’s trinkets, which here are brought forth,}
\begin{scverse}
Not all the Pope’s trinkets, which here are brought forth,\\
Can balance the bible, for weight, or for worth,” \&c.
\end{scverse}
“To the tune of \textit{Thomas you cannot}.” It commences thus:—
\begin{scverse}
\vleftofline{“}All you that desirous are to behold\\
The difference ’twixt falsehood and faith,” \&c.
\end{scverse}

In \textit{Grammatical Drollery}, by W. H (Captain Hicks), 1682, p. 75, is a song
commencing, “Come, my Molly, let us be jolly:” to the tune of \textit{Thomas,
I cannot}; and in Chetwood’s \textit{History of the Stage}, 8vo., 1749, a song on a
theatrical anecdote, by Mr. John Leigh (an actor, who died in 1726), of which
the following is the first stanza:—
\pagebreak
%337

\musicinfo{Gaily.}{}

\includemusic{chappellV1176.pdf}

I have not been successful in finding the song of \textit{Thomas, you cannot}, from
which the tune derives its name. In some copies (when there are no words), the
second part of the tune consists only of eight bars, instead of ten. See the
following from Sir J. Hawkins’ Transcripts of Virginal Music.

\musicinfo{Gaily.}{}

\includemusic{chappellv1177.pdf}

\pagebreak
%338

\musictitle{When Daphne Did From Phœbus Fly.}

This tune is to be found in \textit{Nederlandtsche Gedenck-Clanck}, 1626; in\textit{ Friesche
Lust-Hof}, 1634; and in \textit{The Dancing Master}, from 1650 to 1690.

In the first named it is entitled \textit{Prins Daphne}; in the second,\textit{ When Daphne
did from Phœbus fly}; and in the last, \textit{Daphne, or The Shepherdess}.

A copy of the words will he found in the Roxburghe Collection, i. 388, entitled
“A pleasant new Ballad of Daphne: To a new tune.” Printed by the assignees
of Thomas Symcocke. It is on the old mythological story of Daphne turned into
a~laurel.

\musicinfo{Gracefully and not too slow.}{}

\includemusic{chappellV1178.pdf}

\pagebreak