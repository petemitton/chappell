%257
\changefontsize{1.02\defaultfontsize}
\markright{reigns of james i. and charles i.}

\musictitle{Once I Loved A Maiden Fair.}

A copy of this ballad is in the Roxburghe Collection, i. 350, printed for the
assigns of Thomas Symcock. The tune is in \textit{The Dancing Master}, from 1650 to
1698; in Playford’s Introduction, 1664; in \textit{Musick’s Delight on the Cithren}, 1666;
in \textit{Apollo’s Banquet for the Treble Violin}, 1670; in the \textit{Pleasant Companion for
the Flageolet}, 1680;~\&c.

The first song in Patrick Carey’s \textit{Trivial Poems}, written in 1651 (“Fair one!
if thus kind you be”), is to the tune \textit{Once I lov’d a maiden fair}. It is also
alluded to in \textit{The Fool turn’d Critic}, 1678—“We have now such tunes, such
lamentable tunes, that would make me forswear all music. \textit{Maiden fair} and \textit{The
King’s Delight} are incomparable to some of these we have now.”

The ballad consists of twelve stanzas, from which the following are selected.

\musicinfo{Smoothly and in moderate time.}{}

\includemusic{chappellV1132.pdf}

\backskip{1}

\settowidth{\versewidth}{That the church should make us one,}
\begin{dcverse}\begin{altverse}
Three times I did make it known\\
To the congregation,\\
That the church should make us one,\\
As priest had made relation.\\
Married we straight must be.\\
Although we go a begging;\\
Now, alas! ’tis like to prove\\
A very hopeless wedding.
\end{altverse}

\begin{altverse}
Happy he who never knew\\
What to love belonged;\\
Maidens wavering and untrue\\
Many a man have wronged.\\
Fare thee well! faithless girl,\\
I’ll not sorrow for thee;\\
Once I held thee dear as pearl,\\
Now I do abhor thee.
\end{altverse}
\end{dcverse}
\pagebreak
%258

\musictitle{Gathering Peascods.}
{\small
This beautiful air is contained in all editions of \textit{The Dancing Master}, from
1650 to 1690. The two first bars are the same as “All in a garden green” (see
p. 111); but the resemblance continues no further, and that air is in phrases of
eight, and this of six bars.

Not having been able to discover the original words,, the following lines were
written to it by the late Mr. J. A. Wade; retaining the pastoral character, which
is indicated by its name.
}%end \small

\musicinfo{Moderate time, and sustained.}{}

\includemusic{chappellV1133.pdf}

\pagebreak
%259
\changefontsize{10.0}

\settowidth{\versewidth}{Flow’rs sweet to gaze on, as the songs of birds to hear,}
\begin{dcverse}\indentpattern{010111}
\begin{patverse}
And as I wander in the blossom of the year,\\
By crystal waters’ flow,\\
Flow’rs sweet to gaze on, as the songs of birds to hear,\\
Spring up where e'er I go!	 \\
The violet agrees,\\
With the honey-suckle trees,
\end{patverse}

\columnbreak

\indentpattern{0110}
\begin{patverse}
To shed their balms around!—\\
Thus from the busy throng,\\
I careless roam along,\\
\vleftofline{’}Mid perfume and sweet sound.
\end{patverse}
\end{dcverse}

\musictitle{Lull Me Beyond Thee.}

This tune is in \textit{The Dancing Master}, from 1650 to 1690.

In the Pepys Collection, i. 372, there is a black-letter ballad entitled “The
Northern Turtle, wailing his unhappy fate in being deprived of his sweet mate:
to a new Northern tune, or \textit{A health to Betty}.” This is not the air of \textit{A health of
Betty}, and therefore I suppose it to be the “new Northern tune.” The first
stanza is here arranged to the music. The same ballad is the Roxburghe Collection, 
i. 319, as the second part to one entitled “The paire of Northerne Turtles:
\settowidth{\versewidth}{Whose love was firm till cruel death}
\begin{scverse}Whose love was firm till cruel death\\
Depriv’d them both of life and breath.”
\end{scverse}
That is also to “a new Northern tune,” and printed “for F. Coules, dwelling in
the Old Baily.” Coules printed about 1620 to 1628.

The following ballads are also to the tune:—

Pepys, i. 390—
\settowidth{\versewidth}{Which gives content unto a man’s life.}
\begin{scverse}\vleftofline{“}A constant wife, a kind wife,\\
Which gives content unto a man’s life.
\end{scverse}
To the tune of \textit{Lie lulling beyond thee}.'’ Printed for F. C[oules]. It begins—
\settowidth{\versewidth}{“Young men and maids, do lend me your aids.”}
\begin{scverse}“Young men and maids, do lend me your aids.”
\end{scverse}

Pepys i., and Roxburghe, i. 156—“The Honest Wooer,
\begin{scverse}His mind expressing, in plain and few terms,\\
By which to his mistris his love he confirms:”
\end{scverse}
to the tune of \textit{Lulling beyond her}, begins—
\begin{scverse}\begin{altverse}
\vleftofline{“}Fairest mistris, cease your moane,\\
Spoil not your eyes with weeping,\\
For certainly if one be gone,\\
You may have another sweeting.\\
I will not compliment with oaths,\\
Nor speak you fair to prove you;\\
But save your eyes, and mend your clothes,\\
For it is I that love you.”
\end{altverse}
\end{scverse}

Roxburghe, i. 416—“The two fervent Lovers,” \&c., “to the tune of \textit{The two
loving Sisters}, or \textit{Lulling beyond thee}.” Signed L.P.

Pepys, i. 427—
\begin{scverse}\vleftofline{“}A pleasant new ballad to sing both even and morn,\\
Of the bloody murther of Sir John Barley-Corne.
\end{scverse}
To the tune of \textit{Shall I lie beyond thee}.” Printed at London for H[enry] G[osson].
It commences thus:—
\begin{scverse}\vleftofline{“}As I went through the North country,\\
I heard a merry greeting,” \&c.
\end{scverse}

This excellent ballad has been reprinted by Evans (\textit{Old Ballads} iv. 214,
ed. 1810), from a copy in the Roxburghe Collection, “printed for John Wright.”
\pagebreak
%260

\musicinfo{Smoothly and rather slow.}{}

\includemusic{chappellV1134.pdf}

\musictitle{Come, Shepherds, Deck Your Heads.}

This is also one of the songs mentioned by old Isaak Walton.

\textit{Milkwoman}. “What song was it, I pray? was it ‘Come, shepherds, deck your
heads;’ or, ‘As, at noon, Dulcina rested;’ or, ‘Philida flouts me;’ or, Chevy
Chace;’ or, ‘Johnny Armstrong;’ or, ‘Troy Town?’”\footnote{\textit{}
All will he found in this collection except “Johnny
Armstrong” of which (although an English song, and of
a Westmoreland man) I have not yet found the tune. The
words are in \textit{Wit restored}, 1658, and in \textit{Wit and Drollery,
Jovial Poems}, 1682, called “A Northern Ballet,” beginning—
\settowidth{\versewidth}{There dwelt a man in fair Westmorland,}
\begin{fnverse}
\begin{altverse}
\vleftofline{“}There dwelt a man in fair Westmorland,\\
Johnny Armstrong men did him call;\\
He had neither lands nor rents coming in,\\
Yet he kept eight score men in his hall.”
\end{altverse}
\end{fnverse}
There is also a Scotch ballad ahout the same hero.}

Izaak Walton was born in 1593, and married first Rachel Cranmer, niece of
that distinguished prelate, Thomas Cranmer, Archbishop of Canterbury, in 1624.

The air is found, under its English name, in \textit{Bellerophon, of Lust tot Wÿshed},
Amsterdam, 1622; and in \textit{Gesangh der Zeeden}, Amsterdam, 1662.\footnote{\textit{}
There is another English tune under the same name,
to be found in two other collections, \textit{Nederlandtsche Gedenck-Clanck},
 1626, and \textit{Friesche Lust-Hof}, 1634. I printed
it in \textit{National English Airs}, 1839, but think this rather
more like a ballad-tune, and it is of somewhat earlier
authority.}


The words (which Ritson said “are not known”) will be found in the Pepys
Collection, i. 366, entitled “The Shepherd’s \pagebreak Lamentation: to the tune of
%261
\textit{The plaine-dealing Woman}” On the other half of the sheet is “The second part
of \textit{The plaine-dealing Woman}” beginning—
\settowidth{\versewidth}{“Ye Sylvan Nymphs, come skip it,” \&c.}
\begin{scverse}
“Ye Sylvan Nymphs, come skip it,” \&c.
\end{scverse}

\changefontsize{10.4}

Imprinted at London for J. W. Sir Harris Nicolas prints the song. \textit{Come,
shepherds}, in his edition of Walton’s \textit{Angler}, from a MS. formerly in the possession
of Mr. Heber. A third copy will be found in MSS. Ashmole, No. 38,
art. 164.

\musicinfo{Moderate time.}{}

\includemusic{chappellV1135.pdf}

\settowidth{\versewidth}{The Satyrs strove to have her;}
\begin{dcverse}
\begin{altverse}
All ye forsaken wooers,\\
That ever care oppressed,\\
And all you lusty dooers,\\
That ever love distressed.\\
That losses can condole,\\
And altogether summon;\\
Oh! mourn for the poor soul\\
Of my plain-dealing woman.
\end{altverse}

\begin{altverse}
Fair Venus made her chaste,\\
And Ceres beauty gave her;\\
Pan wept when she was lost,\\
The Satyrs strove to have her;\\
Yet seem’d she to their view\\
So coy, so nice, that no man\\
Could judge, but he that knew\\
My own plain-dealing woman.
\end{altverse}

\begin{altverse}
At all her pretty parts\\
I ne’er enough can wonder;\\
She overcame all hearts,\\
Yet she all hearts came under;\\
Her inward mind was sweet,\\
Good tempers ever common;\\
Shepherd shall never meet\\
So plain a dealing woman.
\end{altverse}
\end{dcverse}
\pagebreak
%262

\musictitle{There Was An Old Fellow At Waltham Cross.}

This is quoted as an old song in Brome’s play, \textit{The Jovial Crew}, which was
acted at the Cock-pit in Drury Lane, in 1641—“T’other old song for that.”
It is also in the \textit{Antidote to Melancholy}, 1661.

\textit{The Jovial Crew} was turned into a ballad-opera in 1731, and this song
retained. The tune was then printed under the name of \textit{Taunton Dean};
perhaps from a song commencing, “In Taunton Dean I was born and bred.”

The four last bars of the air are the prototype of \textit{Lilliburlero}, and still often
sung to the chorus,—

\begin{scverse}
“A very good song, and very well sung;\\
Jolly companions every one.”
\end{scverse}

The first part resembles \textit{Dargason} (see p. 65), and an air of later date, called
\textit{Country Courtship} (see Index).

\musicinfo{Boldly and moderate time.}{}

\includemusic{chappellV1136.pdf}

\musictitle{Old Sir Simon The King.}

This tune is contained in Playford’s \textit{Musick’s Recreation on the Lyra Viol},
1652; in \textit{Musick’s Handmaid for the Virginals}, 1678; in \textit{Apollo’s Banquet for
the Treble Violin}; in \textit{The Division Violin}, 1685; in \textit{180 Loyal Songs}, 1684
and 1694; and in the seventh and all later editions of \textit{The Dancing Master}.

It it also in\textit{\textit{ Pills to purge Melancholy}}; in the \textit{Musical Miscellany}, 1721; in
many ballad-operas, and other works of later date.
\pagebreak
%263

Some of the ballads written to the tune have the following burden, which
appears to be the original:—
\indentpattern{3303}
\settowidth{\versewidth}{With his ale-dropt hose, and his malmsey nose,}
\begin{scverse}
\begin{patverse}
\vin\vin\vin \vleftofline{“}Says old Simon the king,\\
Says old Simon the king,\\
With his ale-dropt hose, and his malmsey nose,\\
Sing, hey ding, ding a ding, ding.”
\end{patverse}
\end{scverse}

From its last line, Ritson conjectured that the “Hey ding a ding” mentioned
in Laneham’s \textit{Letter from Kenilworth}, 1575, as one of the ballads “all ancient,”
then in the possession of Captain Cox, the Coventry mason, was \textit{Old Sir Simon}
under another name. So far as internal evidence can weigh, the tune may be of
even much greater antiquity, but we have no direct proof.

Mr. Payne Collier is of opinion that the ballad entitled \textit{Ragged and torn and
true}, was “first published while Elizabeth was still on the throne.” (See Collier’s
\textit{Roxburghe Ballads}, p. 26.) As it was sung to the tune of \textit{Old Simon the King},
the latter necessarily preceded it. This adds to the probability of Ritson’s conjecture. 
But, although we have ballads printed during the reign of James I., to
the tune of \textit{Old Simon}, I have not succeeded in discovering one of earlier date.

Sir John Hawkins, in the additional notes to his \textit{History of Music}, says, “It is
conjectured that the subject of the song was Simon Wadloe, who kept the Devil
(and St. Dunstan) Tavern, at the time when Ben Jonson’s Club, called the
Apollo Club,\footnote{\textit{}
For the excellent rules of this Club, see Note, p.~250.}
 met there.” The conjecture rests upon two lines of the inscription
over the door of the Apollo room—
\settowidth{\versewidth}{Cries Old Sym, the King of Skinkers.”}
\begin{scverse}
\vleftofline{“}Hang up all the poor hop-drinkers,\\
Cries Old Sym, the King of Skinkers.”
\end{scverse}
A skinker meaning one who serves drink. Sir John quotes the song in \textit{Pills to
purge Melancholy}, iii. 144. It has but one line of burden,—
\settowidth{\versewidth}{“Says old Simon the King; ”}
\begin{scverse}
“Says old Simon the King;”
\end{scverse}
and instead of the Devil tavern, the Crown is the tavern named in it. It appears
to be of too late a date for the original song. The Simon Wadloe\footnote{\textit{}
A Latin “Epitaph upon Simon Wadloe, vintner,
dwelling at the Signe of the Devil and St. Dunstan,” will
he found in MS. Ashmole, No. 38 fol., art. 328; and in
Camden’s \textit{Remains}. It commences thus:—
\settowidth{\versewidth}{“Apollo et cohors Musarum}
\begin{fnverse}
“Apollo et cohors Musarum\\
Bacchus vini et uvarum,” \&c.
\end{fnverse}}
 whom Ben
Jonson dubbed “King of Skinkers,” was buried in March, 1627,\footnote{\textit{}
See Descriptive Catalogue of the Beaufoy Tokens, by
Jacob Henry Burn, 8vo.,1855. From the same book we learn
that \textit{John} Wadlow was proprietor of the Devil Tavern at
the Restoration. He is mentioned twice in Pepys’ Diary
(22nd April, 1661, and 25th Feb., 1664-5). The second
time as having made a fortune—gone to live like a prince
in the country,—there spent almost all he had got, and
finally returned to his old trade again.}
 and more
probably owed his title to having the same Christian name as the Simon of the
earlier song.

As there are two times, which differ considerably, it seems desirable, in the
case of a song once so popular, to print both. The first is from \textit{Musick’s
Recreation on the Lyra Viol}, 1652; and the viol was tuned to what was
termed the “bagpipe tuning,” to play it. To this I have adapted the song quoted
by Hawkins, but completing the burden as the music requires. I have no doubt
that “Old Simon the King” was changed to “Old \textit{Sir} Simon the King,” from
the want of another syllable to correspond with accent of the tune.
\pagebreak
%264

\musicinfo{Cheerfully.}{First Tune.}

\includemusic{chappellV1137.pdf}



\settowidth{\versewidth}{\vin He may hang up himself for shame,}
\indentpattern{0101010101013}
\begin{dcverse}\begin{patverse}
Considering in my mind,\\
I thus began to think:\\
If a man be full to the throat,\\
And cannot take off his drink,\\
If his drink will not go down,\\
He may hang up himself for shame,\\
So the tapster at the Crown;\\
Whereupon this reason I frame:\\
Drink will make a man drunk,\\
Drunk will make a man dry,\\
Dry will make a man sick,\\
And sick will make a man die,\\
Says Old Simon the King.
\end{patverse}

\begin{patverse}
If a man should he drunk to-night,\\
And laid in his grave to-morrow,\\
Will you or any man say\\
That he died of care or sorrow?\\
Then hang up all sorrow and care,\\
’Tis able to kill a cat,\\
\columnbreak
And he that will drink all right,\\
Is never afraid of that;\\
For drinking will make a man quaff,\\
And quaffing will make a man sing,\\
And singing will make a man laugh,\\
And laughing long life doth bring,\\
Says Old Simon the King.
\end{patverse}

\begin{patverse}
If a Puritan skinker do cry,\\
Dear brother, it is a sin\\
To drink unless you be dry,\\
Then straight this tale I begin:\\
A Puritan left his can,\\
And took him to his jug,\\
And there he played the man\\
As long as he could tug;\\
And when that he was spied,\\
Did ever he swear or rail?\\
No, truly, dear brother, he cried,\\
Indeed all flesh is frail.\\
Says Old Simon the King.
\end{patverse}
\end{dcverse}
\pagebreak
%265


The above song dates before the Restoration, because there is a political parody
upon it among the King’s Pamphlets, Brit. Mus., dated January 19th, 1659,
commencing thus:—
\settowidth{\versewidth}{Not a man stands up for the Rump,” \&c.}
\begin{scverse}
\begin{altverse}
\vleftofline{“}In a humour of late I was\\
Ycleped a doleful dump;\\
Thought I, we’re at a fine pass,\\
Not a man stands up for the Rump,” \&c.
\end{altverse}
\end{scverse}
I suppose it to have been written only a short time before the return of Charles,
and that this \textit{Old Simon the King} is the same person alluded to in one of the
Catches in the \textit{Antidote to Melancholy}, 4to, 1661, beginning—

\settowidth{\versewidth}{Sure the heat of the toast your nose did so roast}
\begin{scverse}
\begin{altverse}
\vleftofline{“}Good Symon, how comes it your nose is so red,\\
And your cheeks and your lips look so pale?\\
Sure the heat of the toast your nose did so roast\\
When they were both soused in ale,” \&c.
\end{altverse}
\end{scverse}
And perhaps also in “An Epitaph on an honest citizen and true friend to all
claret drinkers,” contained in part ii. of Playford’s \textit{Pleasant Musical Companion},
4to,~1687—
\begin{scverse}
\vleftofline{“}Here lyeth Simon, cold as clay,\\
Who whilst he liv’d cried Tip away,” \&c.
\end{scverse}
At the end of this epitaph it is said—
\begin{scverse}
\vleftofline{“}Now although this same epitaph was long since given,\\
Yet Simon’s not dead more than any man living.”
\end{scverse}
He was, perhaps, an old man whose death had been long expected.

The tune was in great favour at, and after, the Restoration. Many of the
songs of the Cavaliers were sung to it; many by Martin Parker, and other
ballad-writers of the reigns of James and Charles; several by Wilmott, Earl of
Rochester; and others of still later date.

Penkethman, the actor, wrote a comedy called \textit{Love without Interest, or
The Man too hard for the Master} (1699), in which one of the characters says
satirically, “Who? he! why the newest song he has is \textit{The Children in the Wood},
or \textit{The London Prentice}, or some such like ditty, set to the \textit{new} modish tune of
\textit{Old Sir Simon the King}.”

The name of the tune, \textit{Old Simon the King}, is printed in much larger letters
than any other of the collection, on the title-page of “A Choice Collection of
Lessons, being excellently sett to the Harpsichord, by the two great masters,
Dr. John Blow, and the late Mr. Henry Purcell,” printed by Henry Playford in
1705: it was evidently thought to be the great attraction to purchasers.

Fielding, in his novel of \textit{Tom Jones}, makes it Squire Western’s favorite tune.
He tells us, “It was Mr. Western’s custom every afternoon, as soon as he was
drunk, to hear his daughter play upon the harpsichord.\ldots He never relished
any music but what was light and airy; and, indeed, his most favorite tunes were
\textit{Old Sir Simon the King}, \textit{St. George he was for England}, and some others\dots The
Squire declared, if she would give him t’other bout of \textit{Old Sir Simon}, he would
give the gamekeeper his deputation the next morning. \textit{Sir Simon} was played
again and again, till the charms of music soothed Mr. Western to sleep.”—i. 169.

It was the tune rather than the words, that gave it so lengthened a popularity.
I have found the air commonly quoted under \pagebreak five other names; viz., as \textit{Ragged}
%266
\textit{and torn, and true}; as \textit{The Golden Age}; as \textit{I’ll ne’er be drunk again}; as \textit{When
this old cap was new}; and as \textit{Round about our coal-fire}. The first is from the
ballad called “Ragged and torn, and true; or The Poor Man’s Resolution: to
the tune of \textit{Old Simon the King}.” See Roxburghe Collection, i. 352; or Payne
Collier’s Roxburghe Ballads, p. 26.

The second from “The Newmarket Song, to the tune of \textit{Old Simon the King};”
and beginning with the line, “The Golden Age is come.” See \textit{180 Loyal Songs},
4th edition, 1694, p. 152.

The third from a song called “The Reformed Drinker;” the burden of which
is, “And ne’er be drunk again.” See \textit{ Pills to purge Melancholy}, ii. 47, 1707, or
iv. 47, 1719; also Ritson’s \textit{English Songs}, ii. 59, 1813.

The fourth from one entitled “Time’s Alteration:
\settowidth{\versewidth}{The old man’s rehearsal what brave things he knew,}
\begin{scverse}
\vleftofline{“}The old man’s rehearsal what brave things he knew,\\
A great while agone, when this old cap was new;
\end{scverse}
to the tune of \textit{Ile nere be drunke againe}.” Pepy’s Collection, i. 160; or Evans’
\textit{Old Ballads}, iii. 262. (The name of the tune omitted, as usual, by Evans.)

The fifth is the name commonly given to it in collections of country dances,
printed during the last century.

One of the best political songs to the tune is “The Sale of Rebellion’s
Household Stuff;” a triumph over the downfall of the Rump Parliament,
beginning—
\settowidth{\versewidth}{Which was warm and pleasant to sit in?” \&c.}
\begin{scverse}
\begin{altverse}
\vleftofline{“}Rebellion hath broken up house,\\
And hath left me old lumber to sell;\\
Come hither and take your choice,\\
I’ll promise to use you well.\\
Will you buy the old Speaker’s chair,\\
Which was warm and pleasant to sit in?” \&c.
\end{altverse}
\end{scverse}

This song has the old burden at full length. The auctioneer, finding no purchasers,
offers, at the end, to sell the whole “for an old song.” It has been reprinted
in Percy’s \textit{Reliques of Ancient Poetry}.

I have seen a song beginning—
\settowidth{\versewidth}{And young Sir Simon the Squire,”}
\begin{scverse}
\vleftofline{“}To old Sir Simon the King,\\
And young Sir Simon the Squire,”
\end{scverse}
but have mislaid the reference. The tune is called “\textit{To} old Sir Simon the King,”
in the first edition of the \textit{Beggars’ Opera}, 1728.

In the Roxburghe Collection, i. 468, one of the ballads by Martin Parker, to
the tune of \textit{Ragged and torn, and true}, is entitled “Well met, Neighbour, or—
\settowidth{\versewidth}{A dainty discourse, between Nell and Sis,}
\begin{scverse}
\vleftofline{“}A dainty discourse, between Nell and Sis,\\
Of men that do use their wives amiss.”
\end{scverse}
This might be revived with some benefit to the lower classes at the present day,
especially if the last line of the burden could be impressed upon them—
\settowidth{\versewidth}{\textit{Oh! such a rogue should be hang’d.}}
\begin{scverse}
\begin{altverse}
\vleftofline{“}Heard you not lately of Hugh,\\
How soundly his wife he bang’d?\\
He beat her all black and blue:\\
\textit{Oh! such a rogue should be hang’d.}”
\end{altverse}
\end{scverse}
\pagebreak
%267
Farquhar’s song in the \textit{Beaux's Stratagem}, beginning—
\settowidth{\versewidth}{A trifling song you shall hear,}
\begin{scverse}
\begin{altverse}
\vleftofline{“}A trifling song you shall hear,\\
Begun with a trifle and ended;\\
All trifling people draw near,\\
And I shall be nobly attended,”
\end{altverse}
\end{scverse}
was written to this tune, and is printed to it in \textit{The Musical Companion, or Lady's
Magazine}, 8vo., 1741.

“The praise of St. David’s day: shewing the reason why the Welshmen honour
the leek on that day.” Beginning—
\settowidth{\versewidth}{Who list to read the deeds}
\begin{scverse}
\vleftofline{“}Who list to read the deeds\\
By valiant Welshmen done,” \&c.,
\end{scverse}
is also to the tune, under the name of \textit{When this old cap was new}.

The following is the ballad of “Ragged and torn, and true; or The Poor Man’s
Resolution,” set to the tune as it is found in \textit{The Dancing Master}, and other
violin copies, but omitting the variations.

\musicinfo{Cheerfully.}{Second Tune.}

\includemusic{chappellV1138.pdf}

\pagebreak
%268

\settowidth{\versewidth}{Though my doublet be rent i’ th’ sleeves,}
\begin{dcverse}\begin{altverse}
I scorn to live by the shift,\\
Or by any sinister dealing;\\
I’ll flatter no man for a gift,\\
Nor will I get money by stealing.\\
I’ll be no knight of the post,\\
To sell my soul for a bribe;\\
Though all my fortunes be cross’d,\\
Yet I scorn the cheater’s tribe.\\
Then hang up sorrow and care,\\
It never shall make me rue;\\
What though my cloak be threadbare,\\
\textit{I'm ragged, and torn, and true}.
\end{altverse}

\begin{altverse}
A boot of Spanish leather\\
I’ve seen set fast in the stocks,\\
Exposed to wind and weather,\\
And foul reproach and mocks;\\
While I, in my poor rags,\\
Can pass at liberty still—\\
O, fie on these brawling brags,\\
When money is gotten so ill!\\
O, fie on these pilfering knaves!\\
I scorn to be of that crew;\\
They steal to make themselves brave—\\
\textit{I'm ragged, and torn, and true}.
\end{altverse}

\begin{altverse}
I’ve seen a gallant go by\\
With all his wealth on his back;\\
He looked as loftily\\
As one that did nothing lack.\\
And yet he hath no means\\
But what he gets by the sword,\\
Which he consumes on queans,\\
For it thrives not, take my word.\\
O, fie on these highway thieves!\\
The gallows will be their due—\\
Though my doublet be rent i’ th’ sleeves,\\
\textit{I’m raggedy and torn, and true}.
\end{altverse}

\begin{altverse}
Some do themselves maintain\\
With playing at cards and dice;\\
O, fie on that lawless gain,\\
Got by such wicked vice!\\
They cozen poor country-men\\
With their delusions vilde; [vile]\\
Yet it happens now and then\\
That they are themselves beguil’d;\\
For, if they be caught in a snare,\\
The pillory claims its due;—\\
Though my jerkin be worn and bare,\\
\textit{I'm ragged, and torn, and true}.
\end{altverse}

\begin{altverse}
I have seen some gallants brave\\
Up Holborn ride in a cart,\\
Which sight much sorrow gave\\
To every tender heart;\\
Then have I said to myself\\
What pity is it, for this,\\
That any man for pelf\\
Should do such a foul amiss.\\
O, fie on deceit and theft!\\
It makes men at the last rue;\\
Though I have but little left,\\
\textit{I’m raggedy and torn, and true}.
\end{altverse}

\begin{altverse}
The pick-pockets in a throng,\\
Either at market or fair,\\
Will try whose purse is strong,\\
That they may the money share;\\
But if they are caught i’ th’ action,\\
They’re carried away in disgrace,\\
Either to the House of Correction,\\
Or else to a worser place.\\
O, fie on these pilfering thieves?\\
The gallows will be their due;\\
What need I sue for reprieves?\\
\textit{I’m raggedy and torn, and true}.
\end{altverse}

\begin{altverse}
The ostler to maintain\\
Himself with money in’s purse,\\
Approves the proverb true,\\
And says, Grammercy horse;\\
He robs the travelling beast,\\
That cannot divulge his ill,\\
He steal's a whole handful, at least,\\
From every half-peck he should fill.\\
O, fie on these cozening scabs,\\
That rob the poor jades of their due!\\
I scorn all thieves and drabs—\\
\textit{I’m ragged, and torn, and true}.
\end{altverse}

\begin{altverse}
’Tis good to be honest and just,\\
Though a man be never so poor;\\
False dealers are still in mistrust,\\
They’re afraid of the officer’s door:\\
Their conscience doth them accuse,\\
And they quake at the noise of a bush;\\
While he that doth no man abuse,\\
For the law needs not care a rush.\\
Then well fare the men that can say,\\
I pay every man his due;\\
Although I go poor in array,\\
\textit{I’m ragged, and torn, and true}.
\end{altverse}

\end{dcverse}
\pagebreak
%269
\changefontsize{10.2}

The following is the before-mentioned song, “The Reformed Drinker, or I’ll
ne’er be drunk again,” also to the tune of \textit{Old Sir Simon the King}.
\settowidth{\versewidth}{’Twill cherish and comfort the blood}
\begin{dcverse}\begin{altverse}
Come, my hearts of gold,\\
Let us be merry and wise;\\
It is a proverb of old,\\
Suspicion hath double eyes.\\
Whatever we say or do,\\
Let’s not drink to disturb the brain;\\
Let’s laugh for an hour or two,\\
And ne’er be drunk again.
\end{altverse}

\begin{altverse}
A cup of old sack is good\\
To drive the cold winter away;\\
’Twill cherish and comfort the blood\\
Most when a man’s spirits decay:\\
But he that drinks too much,\\
Of his head he will complain;\\
Then let’s have a gentle touch,\\
And ne’er be drunk again.
\end{altverse}

\begin{altverse}
Good claret was made for man,\\
But man was not made for it;\\
Let’s be merry as we can,\\
So we drink not away our wit;\\
Good fellowship is abus'd,\\
And wine will infect the brain:\\
But we’ll have it better us’d,\\
And ne’er be drunk again.
\end{altverse}

\begin{altverse}
When with good fellows we meet,\\
A quart among three or four,\\
’Twill make us stand on our feet,\\
While others lie drunk on the floor.\\
Then, drawer, go fill us a quart,\\
And let it be claret in grain;\\
’Twill cherish and comfort the heart—\\
But we’ll ne’er be drunk again.
\end{altverse}

\begin{altverse}
Here’s a health to our noble King,\\
And to the Queen of his heart;\\
Let’s laugh and merrily sing,\\
And he’s a coward that will start.\\
Here’s a health to our general,\\
And to those that were in Spain;\\
And to our colonel—\\
And we’ll ne’er be drunk again.
\end{altverse}

\begin{altverse}
Enough’s as good as a feast,\\
If a man did but measure know;.\\
A drunkard’s worse than a beast,\\
For he’ll drink till he cannot go.\\
If a man could time recall,\\
In a tavern that’s spent in vain,\\
We’d learn to be sober all,\\
And we’d ne’er be drunk again.
\end{altverse}
\end{dcverse}

\musictitle{The Beggar Boy.}

This tune is contained in \textit{The Dancing Master}, from 1650 to 1690.

The following ballads were sung to it:—

Roxburghe Collection, i. 528—“Trial brings Truth to light; or—
\settowidth{\versewidth}{A dainty new ditty of many things treating:}
\begin{scverse}The proof a pudding is all in the eating;\\
A dainty new ditty of many things treating:
\end{scverse}
to the tune of \textit{The Beggar Boy}” by Martin Parker; and beginning—
\settowidth{\versewidth}{Her tricks and devices he’s wise that well knows—}
\begin{scverse}\begin{altverse}
\vleftofline{“}The world hath allurements and flattering shows,\\
To purchase her lovers’ good estimation;\\
Her tricks and devices he’s wise that well knows—\\
The learn’d in this science are taught by probation,” \&c.
\end{altverse}
\end{scverse}
The burden is, “The proof of the pudding is all in the eating.”

In the Roxburghe Collection, i. 542—“The Beggar Boy of the North—
\settowidth{\versewidth}{Whose lineage and calling to the world is proclaim’d,}
\begin{scverse}Whose lineage and calling to the world is proclaim’d,\\
Which is to be sung to the tune so nam’d;”
\end{scverse}
beginning—
\begin{scverse}“From ancient pedigree, by due descent,\\
I well can derive my generation,” \&c.;
\end{scverse}
and the burden, “And cry, Good, your worship, bestow one token.”

In the Roxburghe, i. 450, and Pepys, i. 306—“The witty Western Lasse,” \&c.,
“to a new tune called \textit{The Beggar Boy}:” subscribed Robert Guy. This begins,
“Sweet Lucina, lend me thy ayde;” \pagebreak and in the Pepys Collection, i. 310, there is
%270
a ballad to the tune of \textit{Lucina}, entitled “A most pleasant Dialogue, or a merry
greeting hetween two Lovers,” \&c.; beginning, “Good morrow, fair Nancie,
whither so fast;” which I suppose to be also to the tune. It is subscribed C.R.
Printed at London for H[enry G[osson.]

The following is also from the Roxburghe Collection (i. 462), and is reprinted
in Collier’s \textit{Roxburghe Ballads}, p. 7.

\musicinfo{Slow \& very smoothly.}{}

\includemusic{chappellV1139.pdf}

\musictitle{The Boatman.}

This is a bagpipe tune, and might be harmonized with a drone base. In
\textit{Musick’s Recreation on the Viol, Lyra-way}, 1661, the viol is strung to the, “bagpipe
tuning,” to play it. It is to be found in every edition of \textit{The Dancing Master},
from the first to that of 1698. I have not discovered the song of \textit{The Boatman},
but have adapted a stanza from Coryat’s \textit{Crambe}, 1611, to the air. It resembles
\textit{Trip and go} (see p. 131), and the same words might be sung to it. The accent
of the tune seems intended to imitate the turning of the scull in boating.

In the Roxburghe Collection, ii. 496, \pagebreak is a ballad entitled “The wanton wife of
%271
Castle-gate, or The Boatman’s Delight: to its own proper new tune;” but it
appears, from the following, which is the first stanza, that this air cannot have
been intended.
\settowidth{\versewidth}{For its neither grief nor sorrow}
\begin{dcverse}
\begin{altverse}
\vleftofline{“}Farewell both hawk and hind,\\
Farewell both shaft and bow,\\
Farewell all merry pastimes,\\
And pleasures in a row:
\end{altverse}

\begin{altverse}
Farewell, my best beloved.\\
In whom I put my trust;\\
For its neither grief nor sorrow\\
Shall harbour in my breast.”
\end{altverse}
\end{dcverse}

There is an air in Thomson’s \textit{Orpheus Caledonius} called \textit{The Boatman}, but wholly
different from this.

\musicinfo{In rowing time.}{}

\includemusic{chappellV1140.pdf}


\musictitle{Sir Launcelot Du Lake.}

This second tune to the ballad, “When Arthur first in court began” (which
the black-letter copies, \textit{The Garland of Good-will}, \&c., direct to be sung to the
tune of \textit{Flying Fame}—see p. 199), was transcribed by Dr. Rimbault, from the fly-leaf
of a rare book of Lessons for the Virginals, entitled \textit{Parthenia Inviolata}.

The ballad is quoted by Shakespeare, by Beaumout and Fletcher, by Marston,
\&c. It is founded on the romance of \textit{Sir Launcelot du Lake}, than which none
was more popular. Chaucer, in “The Nonne Prest his tale,” says—
\settowidth{\versewidth}{As the book is of Launcelot the Lake; ’’}
\begin{scverse}
\vleftofline{‘}This story is al so trewe, I undertake,\\
As the book is of Launcelot the Lake;’’
\end{scverse}
\pagebreak
%272
and Churchard, in his “Replication to Camel’s objection,” says to him—
\small %begin cramped page 272
\settowidth{\versewidth}{The most of your study hath been of Robyn Hood}
\begin{scverse}\vleftofline{“}The most of your study hath been of Robyn Hood:\\
And Bevis of Hampton and Syr Launcelet de Lake\\
Hath taught you, full oft, your verses to make.”
\end{scverse}
The ballad, entitled “The noble acts of Arthur of the Round Table, and of Sir
Launcelot du Lake,” is printed in Percy’s \textit{Reliques of Ancient Poetry}.

\musicinfo{Boldly and slow.}{}

\includemusic{chappellV1141.pdf}

\vspace{-1.5\baselineskip}

\musictitle{The Spanish Gipsy.}

\vspace{-0.5\baselineskip}

This is in every edition of \textit{The Dancing Master}, and in \textit{Musick’s Delight on
the Cithren}, 1666.

It is found in the ballad-operas, such as \textit{The Bays’ Opera}, 1730, and \textit{The
Fashionable Lady}, 1730, under the name of \textit{Come, follow, follow me}.

The name of \textit{The Spanish Gipsy} is probably derived from a gipsies’ song in
Rowley and Middleton’s play of that name. It begins, “Come, follow your
leader, follow,” and the metre is suitable to the air.

In the Roxburghe Collection, i. 544, is a black-letter ballad, entitled “The
brave English Jipsie: to the tune of \textit{The Spanish Jipsie}. Printed for John
Trundle,” \&c. It consists of eighteen stanzas, and commencing—
\settowidth{\versewidth}{’Tis English Jipsies’ call.”}
\begin{scverse}\vleftofline{“}Come, follow, follow all,\\
’Tis English Jipsies’ call.”
\end{scverse}
And in the same volume, p. 408, one by M[artin] P[arker], called “The three
merry Cobblers,” of which the following are the first, eighth, fourteenth, and last
stanzas. (Printed at London for F. Grove.)
\settowidth{\versewidth}{Our trade excels most trades i’the land,}
\indentpattern{111100}
\begin{dcverse}\begin{patverse}
\vin Come, follow, follow me,\\
To the alehouse we’ll march all three.\\
Leave awl, last, thread, and leather,\\
And let’s go all together.\\
Our trade excels most trades i’the land,\\
For we are still on the mending hand.
\end{patverse}

\begin{patverse}
\vin Whatever we do intend,\\
We bring to a perfect \textit{end};\\
If any offence be past,\\
We make all well at \textit{last}.\\
We sit at work when others stand,\\
And still we are on the mending hand.
\end{patverse}

\begin{patverse}
\vin All day we merrily sing,\\
And customers to us do bring\\
Or unto us do send\\
Their boots and shoes to mend.\\
We have our money at first demand;\\
Thus still we are on the mending hand.
\end{patverse}

\begin{patverse}
\vin We pray for dirty weather,\\
And money to pay for leather,\\
Which if we have, and health,\\
A fig for worldly wealth.\\
Till men upon their heads do stand,\\
We still shall be on the mending hand.
\end{patverse}
\end{dcverse}
\pagebreak
%273

The most popular Song to this tune was—
\settowidth{\versewidth}{Ye fairy elves that be,” \&c.}
\begin{scverse}
\vleftofline{“}Come, follow, follow me,\\
Ye fairy elves that be,” \&c.
\end{scverse}
It is the first in a tract entitled “A Description of the King and Queene of
Fayries, their habit, fare, abode, pompe, and state: being very delightful to the
sense, and full of mirth. London: printed for Richard Harper, and are to be
sold at his shop at the Hospitall Gate, 1635;” and the song was to be “sung
like to the \textit{Spanish~Gipsie}.”

The first stanza is here printed to the tune. The song will be found entire in
Percy’s \textit{Reliques of Ancient Poetry}, or Ritson’s \textit{English Songs}.

\musicinfo{Lightly and in moderate time.}{}

\medskip

\includemusic{chappellV1142.pdf}

\musictitle{The Friar In The Well.}

In Anthony Munday’s \textit{Downfall of Robert, Earl of Huntington} (written in
1597), where Little John expresses his doubts of the success of the play;
saying—
\settowidth{\versewidth}{No pleasant skippings up and down the wood;}
\begin{scverse}
\begin{altverse}
\vleftofline{“}Methinks I see no jests of Robin Hood;\\
No merry Morrices of Friar Tuck;\\
No pleasant skippings up and down the wood;\\
No hunting songs,” \&c.
\end{altverse}
\end{scverse}
The Friar answers, that “merry jests” have been shewn before, such as—
\begin{scverse}
\vleftofline{“}How the Friar fell into the well,\\
For love of Jenny, that fair, bonny belle,” \&c.
\end{scverse}
The title of this ballad is “The Fryer well fitted; or—
\begin{scverse}
\textit{A pretty jest} that once befell;\\
How a maid put a Fryer to cool in a well:
\end{scverse}
to a merry tune.”
\pagebreak
%274

The tune is in \textit{The Dancing Master}, from 1650 to 1686, entitled \textit{The Maid
peept out at the window}, or \textit{The Frier in the Well}.

The ballad is in the Bagford Collection; in the Roxburghe (ii. 172); the
Pepys (iii. 145); the Douce (p. 85); and in \textit{Wit and Mirth, an Antidote to
Melancholy}, 8vo., 1682. Also, in an altered form, in\textit{ Pills to purge Melancholy},
1707, i. 340; or 1719, iii. 325. But not one of these contains the line, “The
maid peept out of the window.” I suppose, therefore, that the present has been
modelled upon some earlier version of the ballad, which I have not seen. The
story is a very old one, and one of the many against monks and friars, in which,
not only England but all Europe delighted.

\musicinfo{Gracefully.}{}

\includemusic{chappellV1143.pdf}


The story of the ballad may be told, with slight abbreviation. Firstly, the
Friar makes love to the Maid:---
\settowidth{\versewidth}{Tush, quoth the Friar, thou needst not doubt,}
\begin{scverse}
\vleftofline{“}But she denyed his desire,\\
And told him that she fear’d Hell-fire.\\
Tush, quoth the Friar, thou needst not doubt,\\
If thou wert in Hell, I could sing thee out.”
\end{scverse}
\pagebreak
%275

The Maid pretends to be persuaded by his arguments, but stipulates that he shall
bring her an angel of money.

\begin{dcverse}\settowidth{\versewidth}{How she the Friar might beguile;}
\vleftofline{“}Tush, quoth the Friar, we shall agree,\\
No money shall part my love and me;\\
Before that I will see thee lack,\\
I’ll pawn my grey gown from my back.\\
The Maid bethought her of a wile,\\
How she the Friar might beguile;

While he was gone (the truth to tell),\\
She hung a cloth before the well.\\
The Friar came, as his covenant was,\\
With money to his bonny lass.\\
Good morrow, fair Maid, good morrow, quoth he,\\
Here is the money I promised thee.”
\end{dcverse}
The Maid thanks him, and takes the money, but immediately pretends that her
father is coming.

\begin{dcverse}\vleftofline{“}Alas! quoth the Friar, where shall I run,\\
To hide myself till he be gone?\\
Behind the cloth run thou, quoth she,\\
And there my father cannot thee see.\\
Behind the cloth the Friar crept,\\
And into the well on a sudden he leapt.\\
Alas! quoth he, I am in the well;\\
No matter, quoth she, if thou wert in Hell:\\
Thou sayst thou couldst sing me out of Hell,\\
Now, prythee, sing thyself out of the well.\\
The Friar sung on with a pitiful sound,\\
O help me out! or I shall he drown'd.\\
I trow, quoth she, your courage is cool’d;\\
Quoth the Friar, I never was so fool’d;\\
I never was served so before.\\
Then take heed, quoth she, thou com’st here no more;

Quoth he, for sweet St. Francis’ sake,\\
On his disciple some pity take;\\
Quoth she, St. Francis never taught\\
His scholars to tempt young maids to naught.\\
The Friar did entreat her still\\
That she would help him out of the well;\\
She heard him make such piteous moan,\\
She help’d him out, and bid him begone.\\
Quoth he, shall I have my money again,\\
Which from me thou hast before-hand ta’en?\\
Good sir, quoth she, there’s no such matter.\\
I’ll make you pay for fouling the water.\\
The Friar went all along the street,\\
Dropping wet, like a new-wash'd sheep;\\
Both old and young commended the Maid\\
That such a witty prank had play’d.”
\end{dcverse}

\musictitle{Sir Eglamore.}

This “merry tune” is another version of \textit{The Friar in the Well} (see the preceding). 
the ballad of \textit{Sir Eglamor}e is a satire upon the narratives of deeds
of chivalry in old romances. It is contained in \textit{The Melancholie Knight}, by
S[amuel] R[owlands], 1615; in the \textit{Antidote to Melancholy}, 1661; in \textit{Merry
Drollery Complete}, 1661; in Dryden’s \textit{Miscellany Poems}, iv. 104; in the Bagford
and Roxburghe Collections of Ballads; in Ritson’s \textit{Ancient Songs}; Evans’ \textit{Old
Ballads}; \&c., \&c.

It appears, with music, in part ii. of Playford’s \textit{Pleasant Musical Companion},
1687; in\textit{ Pills to purge Melancholy}; in Stafford Smith’s \textit{Musica Antiqua}; and the
tune, with other words, in \textit{180 Loyal Songs}, \&c.

The title of the ballad is, “Courage crowned with Conquest; or A brief relation
how that valiant Knight, and heroic Champion, Sir Eglamore, bravely fought
with and manfully slew a terrible, huge, great, monstrous Dragon. To a pleasant
new tune.” There are many variations in the copies from different presses.

The following songs were sung to \textit{Sir Eglamore}:—

“Sir Eglamore and the Dragon, or a relation how General Monk slew a most
cruel Dragon, Feb. 11, 1659.” \textit{See Loyal Songs written against the Rump
Parliament}.
\pagebreak
%276

“Ignoramus Justice; or—
\settowidth{\versewidth}{To let knaves out and keep honest men in:}
\begin{scverse}
The English laws turn’d into a gin.\\
To let knaves out and keep honest men in:
\end{scverse}

to the tune of \textit{Sir Egledemore}.” London: printed for Allen Bancks, 1682.

“The Jacobite toss’d in a Blanket,” \&c. (Pepys Coll., ii. 292); beginning—
\settowidth{\versewidth}{You’ve brought yourselves and your cause to nought.}
\begin{scverse}
\vleftofline{“}I pray, Mr. Jacobite, tell me why, Fa la, \&c.,\\
You on our government look awry, Fa la, \&c.\\
With paltry hat, and threadbare coat,\\
And jaws as thin as a Harry groat.\\
You’ve brought yourselves and your cause to nought.\\
Fa-la, fa-la-la-la, Fa-la, lanky down dilly.”
\end{scverse}

In Rowland’s \textit{Melancholie Knight}, the ballad is thus prefaced:—
\begin{scverse}
“But that I turn, and overturn again.\\
Old books, wherein the worm-holes do remain;\\
Containing acts of ancient knights and squires\\
That fought with dragons, spitting forth wild fires.\\
The history unto you shall appear,\\
Even by myself, verbatim, set down here.”
\end{scverse}

\musicinfo{Gracefully.}{}

\includemusic{chappellV1144.pdf}

\pagebreak
%277

\settowidth{\versewidth}{The Knight did tremble, horse did quake;}
\begin{dcverse}A Dragon came out of his den,\\
Had slain, God knows how many men:\\
When he espied Sir Eglamore,\\
Oh! if you had but heard him roar!

Then the trees began to shake,\\
The Knight did tremble, horse did quake;\\
The birds betake them all to peeping,\\
It would have made you fall a weeping.

But now it is vain to fear,\\
For it must be fight dog, fight bear;\\
To it they go, and fiercely fight\\
A live-long day, from morn till night.

The Dragon had a plaguey hide,\\
And could the sharpest steel abide;\\
No sword would enter him with cuts,\\
Which vext the Knight unto the guts;

But, as in choler he did burn,\\
He watched the Dragon a good turn,\\
And as a yawning he did fall,\\
He thrust the sword in, hilt and all.

Then like a coward he [did] fly\\
Unto his den that was hard by,\\
And there he lay all night and roar’d:\\
The Knight was sorry for his sword;
\end{dcverse}
\settowidth{\versewidth}{He that will fetch it, let him take it.”}
\begin{scverse}But riding thence, said, I forsake it.\\
He that will fetch it, let him take it.”
\end{scverse}

Instead of the two last lines, in many of the copies, are the three following
stanzas:—
\settowidth{\versewidth}{When all was done, to the alehouse he went,}
\begin{dcverse}The sword, that was a right good blade,\\
As ever Turk or Spaniard made,\\
I, for my part, do forsake it,\\
And he that will fetch it, let him take it.

When all was done, to the alehouse he went,\\
And by and by his two-pence he spent;\\
%\columnbreak
For he was so hot with tugging with the Dragon,\\
That nothing would quench him but a whole flagon,

Now God preserve our King and Queen,\\
And eke in London may he seen\\
As many knights, and as many more,\\
And all so good as Sir Eglamore.
\end{dcverse}

\musictitle{The Cobbler’s Jigg.}

This tune first appears in \textit{The Dancing Master}, in the seventh edition, printed
in 1686. It is there entitled \textit{The Cobbler’s Jigg}. More than sixty years before
it had been published in Holland, as an English song-tune, in \textit{Bellerophon}, 1622;
and in \textit{Nederlandtsche Gedenck-Clanck}, 1626. In the index to the latter, among
the “Engelsche Stemmen,” it is entitled “Cobbeler, of: Het Engelsch Lapperken.” 
All the English airs in these Dutch books have Dutch words adapted to
them; but as I do not know the English words which belong to this, I have
adapted an appropriate song from \textit{The Shoemaker's Holiday}, 1600.

In the Pepys Collection of Ballads, vol. i., No. 227, is one entitled “Round,
boyes, indeed! or \textit{The Shoomaker's Holy-day}:
\settowidth{\versewidth}{To fit both country, towne, and cittie, \&c.}
\begin{scverse}Being a very pleasant new ditty,\\
To fit both country, towne, and cittie, \&c.
\end{scverse}
To a pleasant new tune.” It is signed L.P. (Laurence Price?), and printed
for J.~Wright, who printed about 1620. This may prove to be the ballad.
I noted that it was in eighteen stanzas, but omitted to copy it.

Shoemakers called their trade “the gentle craft,” from a tradition that King
Edward IV., in one of his disguises, once drank with a party of shoemakers, and
pledged them. The story is alluded to in the old play, \textit{George a Greene, the
Pinner of Wakefield} (1599), when Jenkin says—
\settowidth{\versewidth}{Marry, because you have drank with the King,}
\begin{scverse}\vleftofline{“}Marry, because you have drank with the King,\\
And the King hath so graciously pledg’d you,\\
You shall no more be called shoemakers;\\
But you and yours, to the world’s end,\\
Shall be called the trade of the gentle graft.’’\\
\attribution Dodsley’s \textit{Old Plays}, v. iii., p. 45.
\end{scverse}

\pagebreak
%278
\changefontsize{1.03\defaultfontsize}

“Would I had been created a shoemaker,” (says the servant in a play of Dekker’s)
“for all the \textit{gentle craft} are gentlemen every Monday by their copy, and scorn
then to work one true stitch.”—Dodsley’s \textit{Old Plays}, v. iii., p. 282.

Cobblers, too, were proverbially a merry set. In the opening scene of Ben
Jonson’s play, \textit{The case is altered}, Juniper, the cobbler, is discovered sitting at
work in his shop, and singing; and Onion, who is sent for him, has great difficulty
in stopping his song. When told that he must slip on his coat and go to
assist, because they lack waiters, he exclaims, “A pityful hearing! for now must I,
of a \textit{merry cobbler}, become a mourning creature.” (The family were in mourning).
Juniper is also represented as a small poet; and when, in the third act, Onion
goes to him again (the cobbler being in his shop, and singing, as usual), and
explains his distress because Valentine had not written the ditty he ordered of
him, Juniper says, “No matter, I’ll hammer out a ditty myself.”

\musicinfo{Jovially, and in moderate time.}{}

\smallskip

\includemusic{chappellV1145.pdf}

\settowidth{\versewidth}{Let’s sing a dirge for Saint Hugh’s soul.}
\indentpattern{01013434}
\begin{scverse}\begin{patverse}
Troll the bowl, the nut-brown bowl,\\
And here, kind mate, to thee!\\
Let’s sing a dirge for Saint Hugh’s soul.\\
And down it merrily.\\
Hey down a down, hey down a down,\\
Hey derry, derry, down a down;\\
Ho! well done, to me let come,\\
Ring compass, gentle joy.
\end{patverse}
\end{scverse}

\pagebreak