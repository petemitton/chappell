%009
\markright{william i. to richard i.}
\changefontsize{1.04\defaultfontsize}

After the Conquest, the first notice we have relating to the Minstrels is the 
founding of the Priory and Hospital of St. Bartholo\-mew,\dcfootnote{
Vide the \textit{Monasticon}, tom. ii. pp. 166-67, for a curious
history of this priory and its founder. Also \textit{Stowe's Survey}. 
In the \textit{Pleasaunt History of Thomas of Reading}, 4to.
1662, he is likewise mentioned. His monument, in good
preservation, may yet be seen in the parish church of
St. Bartholomew, in Smithfield, London.
} in Smithfield, by
Royer, or Raherus, the King’s Minstrel, in the the third year of King Henry I.,
\ad 1102. Henry’s conduct to a luckless Norman minstrel who fell into his power,
tells how keenly the minstrel’s sarcasms were felt, as well as the ferocity of Henry’s
revenge. “Luke de Barre,” said the king, “has never done me homage, but he has
fought against me. He has composed facetiously indecent songs upon me; he has
sung them openly to my prejudice, and often raised the horse-laughs of my malignant
enemies against me.” Henry then ordered his eyes to be pulled out. The
wretched minstrel rushed from his tormentors, and dashed his brains against
the wall.\dcfootnote{
Quoted from Ordericus Vitalis. Hist. Eccles. in Sharon
Turner's Hist. England.
}

In the reign of King Henry II., Galfrid or Jeffrey, a harper, received in 1180
an annuity from the Abbey of Hide, near Winchester; and as every harper was
expected to sing,\dcfootnote{
So in Horn-Child, K. Allof orders his steward,
Althebrus to “teche him of harpe and song.” And
Chaucer, in his description of the Limitour or Mendicant
Friar, speaks of harping as inseparable from singing—“in
his harping, when that he had sung.” Also in 1481, see
Lord Howard's agreement with William Wastell, Harper
of London, to teach a boy named Colet “to harp and to sing.”
} we cannot doubt that this reward was bestowed for his music
and his songs, which, as Percy says, if they were for the solace of the monks there,
we may conclude would be in the English language. The more rigid monks,
however, both here and abroad, were greatly offended at the honours and rewards
lavished on Minstrels. John of Salisbury, who lived in this reign, thus declaims
against the extravagant favour shown to them: “For \textit{you} do not, like the fools of
this age, pour out rewards to Minstrels (Histriones et Mimos\dcfootnote{
Histrio, Mimus, Joculator, and Ministrallus, are all
nearly equivalent terms for Minstrels in Mediaeval Latin.
“Incepit \textit{more Histrionico}, fabulas dicere, et plerumque
cantare.” “Super quo \textit{Histriones cantabant}, sicut modo
cantatur de Rolando et Oliverio.” “Dat sex Mimis
Domini Clynton, \textit{cantantibus}, \textit{citharisantibus}, ludentibus,” 
\&c. 4 s. Geoffrey of Monmouth uses \textit{Joculator} as
equivalent to \textit{Citharista}, in one place, and to \textit{Cantor} in
another. See Notes to Percy’s Essay.
}) and monsters of
that sort, for the ransom of your fame, and the enlargement of your name.”
—(\textit{Epist}. 247.)

“Minstrels and Poets abounded under Henry’s patronage: they spread the love
of poetry and literature among his barons and people, and the influence of the
royal taste soon became visible in the improved education of the great, in the
increasing number of the studious, and in the multiplicity of authors, who wrote
during his reign and the next.”—\textit{Sharon Turner’s Hist. Eng}.

In the reign of Richard I. (1189.) minstrelsy flourished with peculiar splendour.
His romantic temper, and moreover his own proficiency in the art, led him to be
not only the patron of chivalry, but also of those who celebrated its exploits.
Some of his poems are still extant. The romantic release of this king from the
castle of Durrenstein, on the Danube, by the stratagem and fidelity of his Minstrel
Blondel, is a story so well known, that it is needless to repeat it here.\dcfootnote{
The best authority for this story, which has frequently
been doubted, is the Chronique de Rains, written in the
13th Century.—See \textit{Wright's Biograph,Brit., Anglo Norman
Period}, p.~325.
}

\changefontsize{1.06\defaultfontsize}
Another circumstance which proves how easily Minstrels could always gain
admittance even into enemies’ camps and prisons, occurred in this reign. The
young heiress of D’Evreux, Earl of Salisbury,
\pagebreak
“was carried abroad, and secreted
%010 
by her French relations in Normandy. To discover the place of her concealment, 
a knight of the Talbot family spent two years in exploring that province, at first
under the disguise of a pilgrim; till having found where she was confined, in
order to gain admittance he assumed the dress and character of a harper, and
being a \textit{jocose} person, exceedingly skilled in ‘the Gests of the Ancients,’—so they
called the romances and stories which were the delight of that age,—he was gladly
received into the family, whence he took an opportunity to carry off the young
lady, whom he presented to the king; and he bestowed her on his natural brother,
William Longespee (son of fair Rosamond), who became, in \textit{her} right, Earl of
Salisbury.

In the reign of king John (\ad 1212) the English Minstrels did good service
to Ranulph, or Randal, Earl of Chester. He, being beseiged in his Castle of
Rothelan (or Rhuydland), sent for help to De Lacy, Constable of Chester, who,
“making use of the Minstrels of all sorts, then met at Chester fair, by the allurements
of their music, assembled such a vast number of people, who went forth
under the conduct of a gallant youth, named Dutton (his steward and son-in-law)
that he intimidated the Welsh, who supposed them to be a regular body of armed
and disciplined soldiers, so that they instantly raised the siege and retired.”

For this deed of service to Ranulph, both De Lacy and Dutton had, by
respective charters, patronage and authority over the Minstrels and others, who,
under the descendants of the latter, enjoyed certain privileges and protection for
many ages.

Even so late as the reign of Elizabeth, when this profession had fallen into such
discredit that it was considered in law a nuisance, the Minstrels under the jurisdiction
of the family of Dutton are expressly excepted out of all acts of Parliament
made for their suppression; and have continued to be so excepted ever since.\dcfootnote{
See the statute of Eliz. anno. 39. cap. iv. entitled an
Act for punishment of rogues, vagabonds, \&c.; also a renewal
of the same clauses in the last act on this subject,
passed in the reign of George III. The ceremonies
attending the exercise of this jurisdiction are described
by Dugdale (Bar i.. p. 101), and from him, by Percy.
}

“We have innumerable particulars of the good cheer and great rewards given to
the Minstrels in many of the convents, which are collected by Warton and others.
But one instance, quoted from Wood’s Hist. Antiq. Ox., vol. i. p. 67, during the reign
of king Henry III. (sub. an. 1224), deserves particular mention. Two itinerant
priests, on the supposition of their being Minstrels, gained admittance. But the
cellarer, sacrist, and others of the brethren, who had hoped to have been entertained
by their diverting arts, \&c., when they found them to be only two indigent ecclesiastics,
and were consequently disappointed of their mirth, beat them, and turned
them out of the monastery.”

In the same reign (\ad 1252) we have mention of Master Richard, the king’s
Harper, to whom that monarch gave not only forty shillings and a pipe of wine,
but also a pipe of wine to Beatrice, his wife. Percy remarks, that the title of
Magister, or Master, given to this Minstrel, deserves notice, and shows his
respectable situation.

“The learned and pious Grosteste, bishop
\pagebreak
of Lincoln who died in 1253, is said,
%011
\markright{king john to edward i.}
in some verses of Robert de Brunne, who flourished about the beginning of the 
next century, to have been very fond of the metre and music of the Minstrels.
The good prelate had written a poem in the Romanse language, called \textit{Manuel
Peche}, the translation of which into English, Robert de Brunne commenced in
1302, with a design, as he tells us himself, that it should be sung to the harp at
public entertainments.”


\settowidth{\versewidth}{Hys Harper’s chaumbre was fast ther}
\begin{dcverse}For lewde [unlearned] men I undertoke\\
In Englysshe tunge to make thys boke,\\
For many ben of swyche manere\\

That talys and rymys wyl blithly here,\\
Yn gamys and festys, and at the ale\\
Love men to listene trotevale. [triviality]
\end{dcverse}


The following anecdote concerning the love which his author, bishop
Gros\-teste, had for music, seems to merit a place here, though related in rude
rhymes.


\settowidth{\versewidth}{Hys Harper’s chaumbre was fast ther}

\begin{dcverse}I shall yow telle as I have herde\\
Of the bysshope Seynt Roberde,\\
Hys toname [surname] is Grostest\\
Of Lynkolne, so seyth the gest,\\
He loved moche to here the Harpe,\\
For mannes wytte it makyth sharpe.\\
Next hys chaumbre, besyde his study,\\
Hys Harper’s chaumbre was fast therby.\\
Many tymes, by nightes and dayes,\\
He had solace of notes and layes,\\
One askede hym the resun why\\
He hadde delyte in Mynstralsy?\\
He answerde hym on thys manere\\
Why he helde the Harpe so dere:\\
\columnbreak

\vleftofline{“}The vertu of the Harpe, thurgh\\
\vin\vin\vin [through] skylle and ryght,\\
\vleftofline{“}Wyll destrye the fendys [fiends] myght;\\
\vleftofline{“}And to the Cros by gode skylle\\
\vleftofline{“}Is the Harpe ylykened weyl.\\
\vleftofline{“}Tharefore, gode men, ye shall lere, [learn]\\
\vleftofline{“}Whan ye any Gleman here,\\
\vleftofline{“}To wurschep God at your powere,\\
\vleftofline{“}As Davyd seyth in the Santere. [Psalter]\\
\vleftofline{“}In harpe and tabour and symphan\scfootnote{
Either part-singing, or the instrument called the symphony.
} gle\\
\vleftofline{“}Wurschep God: in trumpes and sautre,\\
\vleftofline{“}In cordes, in organes, and bells ringyng:\\
\vleftofline{“}In all these wurschepe the hevene Kyng, \&c.”
\end{dcverse}\normalsize


Before entering on the reign of Edward I., I quit the Minstrels for awhile, to
endeavour to trace the progress of music up to that period. It will be necessary
to begin with the old Church Scales, it having been asserted that all national
music is constructed upon them—an assertion that I shall presently endeavour
to confute; and by avoiding, as far as possible, all obsolete technical, as well
as Greek terms, which render the old treatises on Music so troublesome a study,
I hope to convey such a knowledge of those scales as will answer the purpose of
such general readers as possess only a slight knowledge of music.

\headingfour{CHAPTER II.}

\headingfive{Music of the Middle Ages.—Music in England to the end of
the Thirteenth Century.}

\changefontsize{1.05\defaultfontsize}
During the middle ages Music was always ranked, as now, among the seven
liberal arts, these forming the \textit{Trivium} and \textit{Quadrivium}, and studied by all
those in Europe who aspired at reputation for learning. The Trivium comprised
Grammar, Rhetoric, and Logic; the
\pagebreak
Quadrivium comprehended Music,
%012
\markboth{music of the middle ages.}{gregorian tones.}
Arithmetic, Geometry, and Astronomy. Sharon Turner remarks, that these 
comprised not only all that the Romans knew, cultivated, or taught, but
embodied “the whole encyclopaedia of ancient knowledge.” If we may trust
the following jargon hexameters, which he quotes as “defining the subjects
they comprised,” Music was treated as an art rather than as a science, and
a practical knowledge of it was all that was required:—

\settowidth{\versewidth}{\textit{Mus. canit}; Ar. numerat; Geo. ponderat; Ast. colit astra.}
\begin{quotation}
\noindent Gramm, loquitur; Dia. vera docet; Rhet. verba colorat\\
\textit{Mus. canit}; Ar. numerat; Geo. ponderat; Ast. colit astra.
\end{quotation}

But the methods of teaching both the theory and the practice of music were so
dark, difficult, and tedious, before its notation, measure, and harmonial laws were
settled, that we cannot wonder when we hear of youth having spent nine or ten
years in the study of scholastic music, and apparently to very little purpose.

In the latter part of the fourth century (\ad 374 to 397), Ambrose, bishop of
Milan, introduced a model of Church melody, in which he chose four series
or successions of notes, and called them simply the first, second, third, and fourth
tones, laying aside, as inapplicable, the Greek names of Doric, Phrygian, Lydian,
Æolic, Ionic, \&c. These successions distinguished themselves only by the position
of the semitones in the degrees of the scale, and are said to be as follows:

\begin{center}
\begin{tabular}{llllllllllllll}
1st tone, & d & e & f & g & a & b & c & d \\
2nd tone, &\raisebox{1.5pt}{\rule{1em}{1pt}} & e & f & g & a & b & c & d & e \\
3rd tone, &\multicolumn{2}{c}{\raisebox{2pt}{\rule{3em}{1pt}}}& f & g & a & b & c & d & e & f \\
4th tone, &\multicolumn{3}{c}{\raisebox{2pt}{\rule{5em}{1pt}}} & g & a & b & c & d & e & f & g \\
\end{tabular}
\end{center}

These, Pope Gregory the Great (whose pontificate extended from 590 to 604)
increased to eight. He retained the four above-mentioned of Ambrose, adding to
them four others, which were produced by transposing those of Ambrose a fourth
lower; so that the principal note (or key-note, as it may be called) which formerly
appeared as the first in that scale, now appeared in the middle, or strictly
speaking, as the fourth note of the succession, the four additional scales being
called the plagal, to distinguish them from the four more ancient, which received
the name of authentic.

In this manner their order would of course be disarranged, and, instead of being
the first, second, third, and fourth tones, they became the first, third, fifth, and
seventh.

The following are the eight ecclesiastical tones (or scales) which still exist as such
in the music of the Romish church, and are called Gregorian, after their founder:

\medskip
\noindent\small\begin{tabular}{lclcrlrrcrlrrr}
1st&tone&Authentic,&\multicolumn{2}{c}{\raisebox{2pt}{\rule{4em}{1pt}}}&D&e\tie f&g&A&b\tie c&D\\
2d&do.&Plagal,&A&b\tie c&D&e\tie f&g&A\\
3rd&do.&Authentic,&\multicolumn{3}{c}{\raisebox{2pt}{\rule{6em}{1pt}}}&E\tie f&g&a&B\tie c&d&E\\
4th&do.&Plagal,&\raisebox{1.5pt}{\rule{0.8em}{1pt}}&B\tie c&d&E\tie f&g&a&B\\
5th&do.&Authentic,&\multicolumn{3}{c}{\raisebox{2pt}{\rule{6em}{1pt}}}&F&g&a&b\tie C&d&e\tie F\\
6th&do.&Plagal,&\raisebox{1.5pt}{\rule{0.8em}{1pt}}&C&d&e\tie F&g&a&b\tie C\\
7th&do.&Authentic,&\multicolumn{4}{c}{\raisebox{2pt}{\rule{10em}{1pt}}}&G&a&b\tie c&D&e\tie f&G\\
8th&do.&Plagal,&\multicolumn{2}{c}{\raisebox{2pt}{\rule{4em}{1pt}}}&D&e\tie f&G&a&b\tie c&D
\end{tabular}
\medskip
\normalsize

It will be perceived at the first glance,
\pagebreak
that these Gregorian tones have only
%013
the intervals of the diatonic scale of C, such as are the white keys of the pianoforte,
without any sharps or flats. The only allowable accidental note in the Canto
fermo or plain song of the Romish church is B flat, the date of the introduction
of which has not been correctly ascertained.\dcfootnote{
It was probably derived from the tetrachords of the
Greek scale, which admitted both \textit{b} flat and \textit{b} natural, but
which it is not necessary to discuss here.
} No sharp occurs in genuine chants
of high antiquity. In some modern books the flat is placed at the clef upon \textit{b}, for
the fifth and sixth modes, but the strict adherents to antiquity do not admit this
innovation. These tones only differ from one another in the position of the half
notes or semitones, as from \textit{b} to \textit{c}, and from \textit{e} to \textit{f}. In the four plagal modes, the
final or key note remains the same as in the relative authentic; thus, although in the
sixth mode we have the \textit{notes} of the scale of C, we have not in reality the key of
C, for the fundamental or key note is \textit{f}; and although the first and eighth tones
contain exactly the same notes and in the same position, the fundamental note of
the first is \textit{d}, and of the eighth \textit{g}. There is no other difference than that the
melodies in the four authentic or principal modes are generally (and should
strictly speaking be) confined within the compass of the eight notes above the key
note, while the four plagal go down to a fourth below the key note, and only
extend to a fifth above it.

No scale or key of the eight ecclesiastical modes is to us complete. The first
and second of these modes being regarded, according to the modern rules of
modulation, as in the key of D minor, want a flat upon \textit{b}; the third and fourth
modes having their termination in E, want a sharp upon \textit{f}; the fifth and sixth
modes being in F, want a flat upon \textit{b}; and the seventh and eighth, generally
beginning and ending in G major, want an \textit{f} sharp.

The names of Dorian, Phrygian, Lydian, Mixolydian, \&c., have been applied to
them with equal impropriety (more particularly since Glareanus, who flourished
in the sixteenth century); they bear no more resemblance to the Greek scales than
to the modern keys above cited.

Pope Gregory made an important improvement by discarding the thoroughly
groundless system of the tetrachord, adopted by the ancient Greeks,\dcfootnote{
In the old Greek notation there were 1620 tone characters, 
with which Musicians were compelled to burthen
their memories, and 990 marks actually different from
each other.
} and by
founding in its place that of the octave, the only one which nature indicates. And
another improvement no less important, in connexion with his system of the
octave, was the introduction of a most simple nomenclature of the seven sounds of
the scale, by means of the first seven letters of the alphabet. Burney says that the
Roman letters were first used as musical characters between the time of Boethius,\dcfootnote{
It appears from Burney, that Boethius used the first
fifteen letters of the alphabet, but only as marks of
reference in the divisions of the monochord, not as
musical notes or characters.
}
who died in 526, and St. Gregory; but Kiesewetter\dcfootnote{
“History of the Modern Music of Western Europe,
from the first century of the Christian era, to the present
day,” \&c., by R. G. Kiesewetter, translated by Robert
Müller, 8vo., 1848. It is a very clearly and concisely
written history, and contains in an appendix within the
compass of a few pages, as much of the Greek music as
any modern can require to know.
} attributes this improvement
in notation entirely to Gregory, in whose time the scale consisted only of two 
octaves, the notes of the lower octave \pagebreak being expressed by capital letters, and the
%014
higher by small letters. Eventually a third octave was added to the scale, four 
notes of which are attributed to Guido, and one to his pupils; the two remaining 
notes still later. The highest octave was then expressed by double letters; as, \textit{aa},
\textit{bb}, \&c. These three octaves in modern notes would constitute the following scale:

\noindent\begin{minipage}{\textwidth}
\includemusic{chappellV1002.pdf}

\vspace{-\baselineskip}
{\small\noindent\hspace*{4em} First Octave\hfill Second Octave\hfill Third Octave\hspace*{4em}}
\bigskip
\end{minipage}

This is the alphabetical system of names for the notes which we, in England,
still retain for every purpose but that of exercising the voice, for which solfaing
on vowels is preferred.

Gregory’s alphabetical system of notation was, however, only partially adopted.
Some wrote on lines varying from seven to fifteen in number, placing dots, like
modern crotchet-heads, upon them, but making no use of the spaces. Others used
spaces only, and instead of the dots wrote the words themselves in the spaces, disjointing
each syllable to place it in the position the note should occupy. A third
system was by points, accents, hooks, and strokes, written over the words, and they
were intended to represent to the singer, by their position, the height of the note,
and by their upward or downward tendency, the rising or falling of the voice. It
was, however, scarcely possible for the writer to put down a mark so correctly,
that the singer could tell exactly which note to take. It might be one or two
higher or lower. To remedy this, a red line was drawn over, and parallel to the
words of the text, and the marks were written above and below it. A further
improvement was the use of two lines, one red and the other yellow, the red for F,
the yellow for C, as it only left three notes (G, A, and B) to be inserted between
them.\dcfootnote{
Specimens of this notation, with red and yellow lines,
will be found in Martini’s Storia della Musica, vol. i.
p. 184; in Burney’s History, vol. ii. p. 37; in Hawkins’s
History, p. 947 (8vo. edition); and in Kiesewetter’s
p. 280. Also of other systems mentioned above.
}

Such was the notation before the time of Guido, a monk of Arezzo, in Tuscany,
who flourished about 1020. He extended the number of lines by drawing one
line under F, and another between F and C, and thus obtained four lines and
spaces, a number, which in the Rituals of the Romish Church has never been
exceeded.

The clefs were originally the letters F and C, used as substitutes for those red
and yellow lines. The Base clef still marks the position of F, and the Tenor
clef of C, although the forms have been changed.

Guido, in his Antiphonarium, gives the hymn
%\pagebreak% natural breakpoint
\textit{Ut queant laxis}\dcfootnote{
Hymn for St. John the Baptist’s day, written by
Paul the Deacon, about 774.

\settowidth{\versewidth}{UT queant laxis}
\begin{fnverse}\scriptsize
UT queant laxis\\
REsonare fibris,\\
MIra gestorum\\
FAmuli tuorum:\\
SOLve polluti,\\
LAbii reatum,\\
\vin\vin Sancte Johannes.
\end{fnverse}

SI was not the settled name for B until nearly the end
of the seventeenth century; and, although it was proposed
in 1547, Butler in his Principles of Musick, 1636, gives 
the names of the notes as Ut, re, mi, fa, sol, la, \textit{pha}. In
1673, Gio. Maria Bononcini, father of Handel's pseudorival,
used Do in place of Ut, but the French still retain Ut.
} \pagebreak
(from the
%015
\markright{scales, notation, clefs, and descant.}
initial lines of which the names of the notes, Ut, re, mi, fa, sol, la, were taken), in 
old ecclesiastical notation, and in the Chronicle of Tours, under the year 1033, he
is mentioned as the first who applied those names to the notes. He did not add
the Greek gamma (our G) at the bottom of the scale,\dcfootnote{
To distinguish G on the lowest line of the Base from
the G in the fifth space, the former was marked with the
Greek Ґ, and hence the word gammut, applied to the
whole scale.
} as was long supposed,
for Odo, Abbot of Cluny, in Burgundy, had used it as the lowest note, in his
Enchiridion, a century before.

\changefontsize{1.08\defaultfontsize}
To Franco, of Cologne (who, by the testimony of Sigebert, his cotemporary,
had acquired great reputation for his learning in 1047, and lived at least till 1083,
when he filled the office of Preceptor of the Cathedral of Liege), is to be ascribed
the invention of characters for \textit{time}.\dcfootnote{
John de Muris, who flourished in 1330, in giving a list
of anterior musicians, who had merited the title of
inventors, names Guido, who constructed the gammut, or
scale, for the monochord, and placed notes upon lines and
spaces; after whom came Magister Franco, who invented
the figures, or notes, of the Cantus mensurabilis (qui
invenit in cantu mensuram figurarum). Marchetto da
Padova, who wrote in 1274, calls Franco the inventor of the
four first musical characters; and Franchinus Gaffurius
twice quotes him as the author of the time-table.
} By this he conferred the most important
benefit on music, for, till then, \textit{written} melody was entirely subservient to syllabic
laws, and music in parts must have consisted of simple counterpoint, such, says
Burney, as is still practised in our parochial psalmody, consisting of note against
note, or sounds of equal length.

The first ecclesiastical harmony was called Descant, and by the Italians, Mental
Counterpoint (Contrappunto alia mente). It consisted of extemporaneous singing
in fourths, fifths, and octaves, above and below the plain song of the Church; and
although in its original sense, it implied only singing in two parts, it had made
considerable advances in the ninth century, towards the end of which we find
specimens, still existing, of harmony in three and four parts. When Descant was
reduced to writing, it was called Counterpoint, from \textit{punctum contra punctum},
point against point, or written notes placed one against the other.

Hubald, Hucbald, or Hugbald, as he is variously named, and who died in~930,
at nearly ninety years of age, has left us a treatise, called Musica Enchiriadis,
which has been printed by the Abbé Gerbert, in his Scriptores Ecclesiastici. In
chapters X. to XIV., De Symphoniis, he says: “There are three kinds of
symphony (harmony), in the fourth, fifth, and octave, and as the combination of
some letters and syllables is more pleasing to the ear than others, so is it with
sounds in music. All mixtures are not equally sweet.” In the fifteenth chapter
he uses a transient second and third, both major and minor; and in the eighteenth
he employs four thirds in succession. Burney says: “Hubald’s idea that one
voice might wander at pleasure through the scale, while the other remains fixed,
shows him to have been a man of genius and enlarged views, who, disregarding
rules, could penetrate heyond the miserable practice of his time, into our Points
d’Orgue, Pedale, and multifarious harmony upon a holding note, or single base,
and suggests the principal, at least, of the boldest modern harmony.” It is in
this last sense of amplifying a point, that we still retain the verb to \textit{descant} in
common use. Guido describes the Descant
\pagebreak
existing in his time, as consisting of 
%016
fourths, fifths, and octaves under the plain-song or chant, and of octaves (either 
to the plain song or to this base) above it. He suggests what he terms a smoother
and more pleasing method of under-singing a plain-song, in admitting, besides the
fourth and the tone, the major and minor thirds; rejecting the semitone and the
fifth. “No advances or attempts at variety seem to have been made in counterpoint,
from the time of Hubald, to that of Guido, a period of more than a hundred
years; for with all its faults and crudities, the counterpoint of Hubald is at least
equal to the best combinations of Guido;” but the monk, Engelbert, who wrote in
the latter end of the thirteenth century, tells us that all “regular descant” consists
of the union of fourths, fifths, and octaves, so that these uncouth and barbarous
harmonies, in that regular succession which has been since prohibited,
continued in the Church for four centuries.

Before the use of lines, there were no characters or signs for more than two
kinds of notes in the Church; nor since ecclesiastical chants have been written
upon four lines and four spaces, have any but the square and lozenge characters,
commonly called Gregorian notes, been used in Canto fermo: and, although the
invention of the time-table extended the limits of ingenuity and contrivance to
the utmost verge of imagination, and became all-important to secular music,
the Church made no use whatever of this discovery.

That melody received no great improvement from the monks, need excite
no wonder, as change and addition were alike forbidden; but not to have
improved harmony more than they did for many centuries after its use was
allowed, is a matter of just surprise, especially since the cultivation of music
was a necessary part of their profession.

We have occasional glimpses of secular music through their writings; for
instance, Guido, who gives a fair definition of harmony in the sense it is now
understood (Armonia est diversarum vocum apta coadunatio), says that he
merely writes for the Church, where the pure Diatonic genus was first used, but
he was aware of the deficiency as regards other music. “Sunt prœterea et alia
musicorum genera aliis mensuris aptata.” Franco (about 1050) just mentions
Discantum in Cantilenis Rondellis—“Descant to Rounds or Roundelays,”—but
no more.

When Franco writes in four parts, he sometimes gives five lines to each part,
the five lowest for the Tenor or plain song, the next five for the Medius, five for
the Triplum Discantus, and the highest for the Quadruplum. Each has a clef
allotted to it. Although many changes in the form of musical notes have been
made since his time, the lines and spaces have remained without augmentation or
diminution, four for the plain song of the Romish Church, and five for secular
music.

He devotes one chapter to characters for measuring silence, and therein gives
examples of rests for Longs, Breves, Semibreves, and final pauses. He also
suggests dots, or points of augmentation. Bars are placed in the musical examples,
as pauses for the singers to take breath at the end of a sentence, verse, or phrase
of melody. And this is the only use made of bars in Canto fermo.
\pagebreak
%017
\markright{anglo-saxon music.}

\changefontsize{1.05\defaultfontsize}
Turning to England, Milton tells us, from the Saxon annals, that in 668, 
Pope Vitalian sent singers into Kent, and in 680, according to the Venerable
Bede,\dcfootnote{
As a proof of the veneration in which Bede was held,
and the absurd legends relating to him, I quote from
a song of the fifteenth century:—

\settowidth{\versewidth}{\vin\textit{Songs and Carols. Percy Soc}. No. 73, p. 31.}

\begin{fnverse}\scriptsize
\vleftofline{“}When Bede had prechd to the stonys dry\\
The my[gh]t of God made [t]hem to cry\\
Amen:--certys this no ly[e]!”\\
\vin\textit{Songs and Carols. Percy Soc}. No. 73, p. 31.
\end{fnverse}
} Pope Agatho sent John, the Præcentor of St. Peter’s at Pome, to
instruct the monks of Weremouth in the manner of performing the ritual, and he
opened schools for teaching music in other parts of the kingdom of Northumberland.
Bede was also an able musician, and is the reputed author of a short
musical tract in two parts, de \textit{Musica theorica}, and \textit{de Musica practica, seu mensurata};
but Burney says, although the first may have been written by him, the
second is manifestly the work of a much more modern author, and he considers it
to have been produced about the twelfth century, \ie, between the time of Guido
and the English John de Muris. There must always be a difficulty in identifying
the works of an author who lived at so remote a period, without the aid of
contemporary authority, or of allusions to them of an approximate date; and when
he has written largely, such difficulties must be proportionably increased. But,
rejecting both the treatises on music, if he be the author of the Commentary on
the Psalms, which is included in the collected editions of his works of 1563
and 1688, sufficient evidence will remain to prove, not only his knowledge of
music, but of all that constituted the “regular” descant of the church from the
ninth to the thirteenth century. I select one passage from his Commentary on
the 52nd Psalm. “As a skilful harper in drawing up the cords of his instrument,
tunes them to such pitches, that the higher may agree in harmony with the lower,
some differing by a semitone, a tone, or two tones, others yielding the consonance
of the fourth, fifth, or octave; so the omnipotent God, holding all men predestined
to the harmony of heavenly life in His hand like a well-strung harp, raises some
to the high pitch of a contemplative life, and lowers others to the gravity of active
life.” And he thus continues:—“Giving the consonance of the octave, which
consists of eight strings;”\ldots “the consonance of the fifth, consisting of five
strings; of the fourth, consisting of four strings, and then of the smaller vocal
intervals, consisting of two tones, one tone, or a semitone, and of there being
semitones in the high as well as the low strings.”\dcfootnote{
“Sient peritus citharæda chordas plures tendens in
cithara, temperat eas acumine et gravitate tali, ut
superiores inferioribus conveniant in melodia, quædam
semitonii, quædam unius toni, quædam duorum tonorum
differentiam gerentes, aliæ vero diatesseron, aliæ autem
diapente, vel etiam diapason consonantiam reddentes: ita
et Deus omnipotens omnes homines ad cœlestis vitæ
harmoniam prædestinatos in manu sua, quasi eitharam
quandam, chordis convenientibus ordinatam, habens,
quosdam quidem ad acutum contemplativæ vitæ sonum
intendit, alios verò ad activæ vitæ gravitatem temperando
remittit.”—“ut ad alios comparati quasi diapason consonantiam,
quæ oeto chordis constat, reddant..Quos
autem ad diapente consonantiam, quinque chordis constantem,
eligit, illi possunt intelligere qui tantæ jam perfectionis
sunt\ldots Diatesseron quatuor chordis constans.
\ldots Per minora vero vocum intervalla quæ duos tonos
aut unum, vel semitonium sonant\ldots Sed quia tam in
altisonis quam in grandisonis chordis habetur semitonium.”
\&c.—\textit{Bedæ Presbyteri Opera}, vol. 8, p. 1070, fol.
\textit{Basilæ}, 1563, or \textit{Coloniæ Agrippinæ}, vol. 8, p. 908,
fol. 1688.
} Our great king, Alfred,
according to Sir John Spelman, “provided himself of musitians, not common, or
such as knew but the practick part, but men skilful in the art itself;” and in 866, 
according to the annals of the Church of
\pagebreak
Winchester, and the testimony of many
%018
\markboth{music in england, time of henry ii.}{giraldus cambrensis’ account.}
ancient writers, he founded a Professorship at Oxford,\dcfootnote{
The earliest express mention of the University of
Oxford, after the foundation of the schools there by
Alfred, is from the historian Ingulphus, whose youth
coincided with the early part of the reign of Edward the
Confessor. He tells us that, having been born in the City
of London, he was first sent to school at Westminster,
and that from Westminster he proceeded to Oxford,
where he studied the Aristotelian Philosophy, and the
rhetoritical writings of Cicero.
} for the cultivation of music 
as a science. The first who filled the chair was Friar John, of St. David’s, who
read not only lectures on Music, but also on Logic and Arithmetic. Academical
honors in the faculty of music have only been traced back to the year 1463, when
Henry Habington was admitted to the degree of Bachelor of Music, at Cambridge,
and Thomas Saintwix, Doctor of Music, was made Master of King’s College, in
the same university; but it is remarkable that music was the only one of the
seven sciences that conferred degrees upon its students, and England the only
country in which those degrees were, and are still conferred.

About 1159, when Thomas à Becket conducted the negociations for the
marriage of Henry the Second’s eldest son with the daughter of Louis VII., and
went to Paris, as chancellor of the English Monarch, he entered the French towns,
his retinue being displayed with the most solicitous ostentation, “preceded by two
hundred and fifty boys on foot, in groups of six, ten, or more together, singing
English songs, according to the custom of their country.”\dcfootnote{
“In ingressu Gallicanarum villarum et castrorum,
primi veuiebant garciones pedites quasi ducenti quinquaginta, 
gregatim euntes sex vel deni, vel plures simul,
aliquid lingua sua pro more patriæ suæ cantantes.”—
\textit{Stephanides, Vita S. Thomæ Cantuar}, pp.~20,~21.
} This singing in groups
resembled the “turba canentium,” of which Giraldus afterwards speaks; and the
following passage from John of Salisbury, about 1170, shows at least the delight
the people had in listening to part-singing, or descant. “The rites of religion
are now profaned by music; and it seems as if no other use were made of it than
to corrupt the mind by wanton modulations, effeminate inflexions, and frittered
notes and periods, even in the \textit{Penetralia}, or sanctuary, itself. The senseless
crowd, delighted with all these vagaries, imagine they hear a concert of sirens,
in which the performers strive to imitate the notes of nightingales and parrots,
not those of men, sometimes descending to the bottom of the scale, sometimes
mounting to the summit; now softening, and now enforcing the tones, repeating
passages, mixing in such a manner the grave sounds with the more grave, and
the acute with the most acute, that the astonished and bewildered ear is unable
to distinguish one voice from another.”\dcfootnote{
Musica cultum religionis incestat, quod ante conspectum
Domini, in ipsis penetralibus sanctuarii, lascivientis
vocis luxu, quadam ostentatione sui, muliebribus
modis notularum articulorumque cæsuris, stupentes
animulas emollire nituntur. Cum præcinentium, et succinentium, 
canentium, et decinentium, intercinentium,
et occinentium, præmolles modulationes audieris, Sirenarum
concentus credas esse, non hominum et de vocum
facilitate miraberis, quibus philomela vel psittacus, aut
si quid sonorius est, modos suos nequeunt coæquare. Ea
siquidem est, ascendeudi descendendique facilitas; ea
sectio vel geminatio notularum, ea replicatio articulorum,
singulorumque consolidatio; sic acuta vel acutissima,
gravibus et subgravibus temperantur, ut auribus sui
indicii fere subtrahetur auturitas.—\textit{Policraticus, sive de
Nugis Curialium}, lib. i., c.~6.
} It was probably this abuse of descant
that excited John’s opposition to music, and his censures on the minstrels, as
shown in the passage before quoted. It proves also, that descant in England did
not then consist merely of singing in two parts, but included the licenses and
ornaments of florid song. Even singing in canon seems to be comprised in the
words, “præcinentium et succinentium, canentium et decinentium.”

About 1185, Gerald Barry, or Giraldus Cambrensis, 
\pagebreak
archdeacon, and afterwards
%019
bishop, of St. David’s, gave the following description of the peculiar manner 
of singing of the Welsh, and the inhabitants of the North of England: “The
Britons do not sing their tunes in unison, like the inhabitants of other countries,
but in different parts. So that when a company of singers meets to sing, as is
usual in this country, as many different parts are heard as there are singers, who
all finally unite in consonance and organic melody, under the softness of B flat.\dcfootnote{
“Uniting under the softness of B flat,” is not very
intelligible, but one thing may be inferred from it, that
they sang in the natural scale, such as the fifth mode
became by the use of B flat in the scale of F, and not in
the modes that were peculiar to the church. B flat was
only used in the fifth mode and its plagal.
}
In the Northern parts of Britain, beyond the Humber, and on the borders of
Yorkshire, the inhabitants make use of a similar kind of symphonious harmony
in singing, but with only two differences or varieties of tone and voice, the one
murmuring the under part, the other singing the upper in a manner equally soft
and pleasing. This they do, not so much by art, as by a habit peculiar to themselves,
which long practice has rendered almost natural, and this method of singing
has taken such deep root among this people, that hardly any melody is accustomed
to be uttered simply, or otherwise than in many parts by the former, and in two
parts by the latter. And what is more astonishing, their children, as soon as they
begin to sing, adopt the same manner. But as not all the English, but only those
of the North sing in this manner, I believe they had this art at first, like their
language, from the Danes and Norwegians, who were more frequently accustomed
to occupy, as well as longer to retain, possession of those parts of the island.”\dcfootnote{
In musico modulamine non uniformiter ut alibi,
sed multipliciter multisque modis et modulis cantilenas
emittunt, adeò ut in turba canentium, sicut huic genti
mos est, quot videas capita tot audias carmina discriminàque
vocum varia, in unam denique sub B mollis
dulcedine blanda consonantiam etorganicam convenientia
melodiam. In borealibus quoque majoris Britanniæ partibus
trans Humbrum, Eboracique finibus Anglorum
populi qui partes illas inhabitant simili canendo symphonica
utuntur harmonia: binis tamen solummodo
tonorum differentiis et vocum modulando varietatibus,
una inferius sub murmurante altera verò supernè demulcente
pariter et delectante. Nec arte tantum sed usu
longævo et quasi in naturam mora diutina jam converso,
hæc vel illa sibi gens hanc specialitatem comparavit.
Qui adeò apud utramque invaluit et altas jam radices
posuit, ut nihil hic simpliciter, ubi multipliciter ut apud
priores, vel saltem dupliciter ut apud sequentes, mellitè
proferri consuaverit. Pueris etiam (quòd magis admirandum) 
et ferè infantibus, (cum prinum à fletibus in
cantus erumpunt) eandem modulationem observantibus,
Angli verò quoniam non generaliter omnes sed boreales
solùm hujusmodi vocum utuntur modulationibus, credo
quòd a Dacis ct Norwagiensibus qui partes illas insulæ
frequentiùs occupare ac diutiùs obtinere solebant, sicut
loqueudi affinitatem, sic canendi proprietatem contraxerunt.—\textit{Cambriæ
Descriptio}, cap. xiii.
}
Now, allowing a little for the hyperbolic style so common with writers of that age,
this may fairly be taken as evidence that part-singing was common in Wales, or
that at least they made descant to their tunes, in the same way that singers did
to the plain-song or Canto fermo of the Church at the same period; also that
singing in two parts was common in the North of England, and that children tried
to imitate it. Burney and Hawkins think that what Giraldus says of the singing
of the people of Northumberland, in two parts, is reconcileable to probability,
because of the schools established there in the time of Bede, but Burney doubts
his account of the Welsh singing in many parts, and makes this “turba
canentium” to be \textit{of the common people}, adding, “we can have no exalted idea
of the harmony of an \textit{untaught} crowd.” These, however, are his own inferences;
Giraldus does not say that the singers were untaught, or that they were of
the common people. As he is describing 
\pagebreak
what was the custom in his own time,
%020
\markboth{harpers not taught by monks.}{character of tunes often derived from instruments.}
not what had taken place a century before, there seems no sufficient ground 
for disbelieving his statement,\dcfootnote{
Dr. Percy says, “The credit of Giraldus, which hath
been attacked by some partial and bigoted antiquaries,
the reader will find defended in that learned and curious
work, ‘Antiquities of Ireland,’ by Edward Ledwich,
LL.D. Dublin, 1790, 4to., p. 207. et seq.”
} and least of all, should they who are of the opinion
that all musical knowledge was derived from the monasteries, call it in question,
since, as already shown, part music had then existed in the Church, in the form
of descant, for three centuries.

\changefontsize{1.09\defaultfontsize}
“If, however,” says Burney, “incredulity could be vanquished with respect to
the account which Giraldus Cambrensis gives of the state of music in Wales
during the twelfth century, it would be a Welsh MS. in the possession of Richard
Morris, Esq., of the Tower, which contains pieces for the harp, that are in \textit{full
harmony} or \textit{counterpoint}; they are written in a peculiar notation, and supposed
to be as old as the year 1100; at least, such is the known antiquity of many of
the songs mentioned in the collection,” \&c. It is not necessary here to enter
into the defence of Welsh music, but the specimens Dr. Burney has printed from
that manuscript, which he describes as in full harmony and counterpoint, are
really nothing more than the few simple chords which must fall naturally under
the hand of any one holding the instrument, and such as would form a child’s
first lessons. First the chord, G C E, and then that of B D F, form the entire
bass of the only two lessons he has translated; and though from B to F is
a “false fifth,” it must be shown that the harper derived his knowledge of
the instrument from the Church, before the assertion that it is more modern
harmony than then in use can have any weight. In England, at least, not
only the evidence of Giraldus, but all other that I can find, is against such a
supposition. I have before alluded to the Romance of Horn-Child, (note \textit{c}, to
page 9), and here give the passage, to prove that such knowledge was not
derived from the Church, as well as to show what formed a necessary part of
education for a knight or warrior. It is from that part of the story where
Prince Horn appears at the court of the King of Westnesse.

\settowidth{\versewidth}{His steward, and [to] him said thus}

\bigskip
\noindent\small\begin{tabular}{ll}
\hspace*{\fill}\textsc{Original Words.}\hspace*{\fill}&\hspace*{\fill}\textsc{Words Modernized.}\hspace*{\fill}\\
“The kyng com in to halle,&The king came into [the] hall\\
Among his knyhtes alle,&Among his knights all,\\
Forth he clepeth Athelbrus,&Forth he calleth Athelbrus,\\
His stiward, and him seide thus:&His steward, and [to] him said thus\\
‘Stiward, tac thou here&“Steward, take thou here\\
My fundling, for to lere&My foundling, for to teach\\
Of thine, mestere&Of thy mystery\\
Of wode and of ryuere,&Of wood and of river,\\
\textit{Ant toggen o the harpe}&And to play on the harp\\
\textit{With is nayles sharpe}.&With his nails sharp.\\
Ant tech him alle the listes&And teach him all thou listest,\\
That thou euer wystest,&That thou ever knewest,\\
Byfore me to keruen.&Before me to carve\\
And of my coupe to seruen:&And my cup to serve:\\
\end{tabular}
\pagebreak
%021

\changefontsize{1.08\defaultfontsize}
\noindent\small\begin{tabular}{ll}
Ant his feren deuyse&And devise for his fellows\\
With ous other seruise;&With us other service;\\
Horn-Child, thou vnderstond&Horn-Child, thou understand\\
\textit{Tech him of harpe and of song}.’”&Teach him of harp and of song.”
\end{tabular}
\bigskip
\normalsize



\noindent In another part of the poem he is introduced playing on his harp. 

\bigskip
\noindent\small\begin{tabular}{ll}
Horn sette him abenche,&Horn seated himself on a bench,\\
Is harpe he gan clenche&His harp he began to clench;\\
He made Rymenild a lay&He made Rymenild a lay\\
Ant hue seide weylaway, \&c.\footnotemark &And he said wellaway! \&c.
\end{tabular}

\footnotetext{\scriptsize Warton’s History of English Poetry, vol. i., p.~38, 8vo., 1840.}
\bigskip
\normalsize




In searching into the early history of the music of any country, the first
subject of inquiry should be the nature and character, as well as the peculiarities
of scale, of the musical instruments they possessed. If the musical instruments
in general use had an imperfect scale, the national music would generally, if not
universally, have retained the peculiarities of that scale. Hence the characteristics
of Scottish music, and of some of the tunes of the North of England, which resemble
it. In the following collection many can he pointed out as bagpipe tunes,
such as “Who liveth so merry in all this land, as doth the poor widow that selleth
the sand,” and “By the border’s side as I did pass,” both of which seem to
require the accompaniment of the drone, while others, like “Mall (or Moll)
Sims,” strictly retain the character of harp music. Where, however, the harp
was in general use, the scale would be more perfect than if some other instruments
were employed, and hence the melodics would exhibit fewer peculiarities,
unless, indeed, the harp was tuned to some particular scale, which, judging by
the passage above quoted from Bede, does not seem to have been the case in
England.

About 1250 we have the song, \textit{Sumer is icumen in}, the earliest secular composition, 
in parts, known to exist in any country. Sir John Hawkins supposed that
it could not he earlier than the fifteenth century, because John of Dunstable, to
whom the invention of figurative music has heen attributed, died in 1455. But
Dr. Burney remarks that Dunstable could not have been the inventor of that art,
concerning which several treatises were written before John was born, and shows
that mistake to have originated in a passage from Proportionales Musices, by John
Tinctor, a native of Flanders, and the “most ancient composer and theorist of
that country, whose name is upon record.” It is as follows: “Of which new art,
as I may call it (counterpoint), the fountain and source is said to have been among
the English, of whom Dunstable was the \textit{chief}.”\dcfootnote{
“Cujus, ut ita dicam, novæ artis (Contrapunctis), fons
et origo apud Anglos, quorum caput Dunstaple extitit,
fuisse perhibetur.” From Proportionale Musices, dedicated
to Ferdinand, king of Sicily, Jerusalem, and
Hungary (who reigned from 1458 to 1494), by John
Tinctor, Chaplain and Maestro di Capella to that Prince.
} “Caput,” literally meaning
“head,” had been understood in its secondary sense of “originator or beginner.”
Dr. Burney’s opinion with respect to the age of this canon seems to have been
very unsettled (if indeed he can be said to have formed one at all). He first
presents it as a specimen of the harmony \pagebreak 
in our country, “about the fourteenth 
%022
\markboth{manuscripts---thirteenth century.}{sumer is icumen in.}
and fifteenth century.” On the same page he tells us that the notes of the 
MS. resemble those of Walter Odington’s Treatise\dcfootnote{
The Best summary of the state of music in England,
about 1230, is contained in Walter Odington’s Treatise,
which is fully described in Burney’s History of Music,
vol. ii., p. 155, et seq. Burney considers it the most
complete of all the early treatises, whether written here
or abroad.
} (1230), and seem to be of the
thirteenth or fourteenth century, and he can hardly imagine the canon much
more modern. Then he is “sometimes inclined to imagine” it to have been
the production of the Northumbrians, (who, according to Giraldus Cambrensis,
used a kind of natural symphonious harmony,) \textit{but with additional parts}, and a
second drone-base of later times. By “additional parts” I suppose Burney
to mean \textit{adding to the length} of the tune, and so continuing the canon. Next
in reviewing “the most ancient musical tract that has been preserved in our
vernacular tongue” (by Lyonel Power), he says, this rule (a prohibition of
taking fifths and octaves in succession) seems to have been so much unknown
or disregarded by the composer of the canon, “Sumer is icumen in,” as to
excite a suspicion that it is “much more ancient than has been imagined.”
And finally, “It has been already shown that counterpoint, in the Church,
began by adding parts to plain chant; and in secular music, by harmonizing
old tunes, as florid melody did by variations on these tunes. It was long
before men had the courage to invent new melodies. It is a matter of surprise
that so little plain counterpoint is to be found, and of this little, none
correct, previous to attempts at imitation, fugue, and canon; contrivances to which
there was a very early tendency, in all probability, during times of extemporary
descant, before there was any such thing as written harmony: for we find in the
most ancient music in parts that has come down to us, that fugue and canon had
made considerable progress at the time it was composed. The song, or round,
‘Sumer is icumen in,’ is a very early proof of the cultivation of this art.” He
then proceeds to show how, according to Martini, from the constant habit of
descanting in successive intervals, new melodies would be formed in harmony with
the original, and whence imitations would naturally arise.

\changefontsize{1.05\defaultfontsize}
Ritson, who knew more of the age of manuscripts than of musical history, is
of opinion that Burney and Hawkins were restrained by fear from giving their
opinion of its date, and says it may be referred to as early a period (\textit{at least}) as
the year 1250. Sir Frederick Madden,\dcfootnote{
Keeper of the Manuscripts in the British Museum.
} in a note to the last edition of Warton’s
English Poetry, says: “Ritson justly exclaims against the ignorance of those who
refer the song to the fifteenth century, when the MS. itself is certainly of the
middle of the thirteenth.” Mr. T. Wright, who has devoted his attention
almost exclusively to editing Anglo-Norman, Anglo-Saxon, and early English
manuscripts, says: “The latter part of this manuscript, containing, among others,
the long political song printed in my Pol. Songs, p. 72, was certainly written
during the interval between the battle of Lewes, in May, 1264, and that of
Evesham, in the year following, and most probably immediately after the firstmentioned
event. The earlier part of the MS., \pagebreak 
which contains the music, was
%023
evidently written at an earlier period—perhaps by twenty or thirty years—and
the song with its music must therefore be given to the first half of the thirteenth 
century, at latest.” I have thus entered into detail concerning this song
(though all the judges of manuscripts, whom I have been enabled to consult, are
of the same opinion as to its antiquity), because it is not only one of the first
English songs with or without music, but the first example of counterpoint in six
parts, as well as of fugue, catch, and canon; and at least a century, if not two
hundred years, earlier than any composition of the kind produced out of
England.\dcfootnote{
The earliest specimen of secular part-music that has
yet been discovered on the Continent, is an old French
song, for three voices, the supposed production of a singer
and poet, by name Adam de la Hale, called Le Boiteux
d'Arras, who was in the service of the Comte de Provence.
The discovery has been recently made and communicated
by M. Fétis, in his Revue Musicale. “It may be placed
about the year~1280, if a dilettante of the discantus of \textit{the
following age} has not experimentalised on the melody left
by De la Hale, as on a tenor or Canto fermo; since the
other songs, in similar notation, are not in counterpoint;
and the manuscript may be assigned to the \textit{fourteenth}
century.” It is given in Kiesewetter’s History of Music.
}


The antiquity of the words has not been denied, the progress of our language
having been much more studied than our music, but the manuscript deserves much
more attention from musicians than it has yet received.\dcfootnote{
The Musical Notation in this MS, (Harl. 978) is
throughout the same. Only two forms of note are used
with occasional ligatures. “Sumer is icumen in” is on the
back of page 9, and just after it is an Antiphon in praise
of Thomas à Becket. At page 12 we have the musical
scale in letters, exactly corresponding with the scale of
Guido, with the ut, re, mi, fa, \&c., but only extending to
two octaves and four notes, without even the “\textit{e e},” said
to have been added by his pupils. At the back of that
page is an explanation of the intervals set to music, to
impress them on the memory by singing, and examples of
the ligatures used in the notation of the manuscript. At
page 8 is a hymn, “Ave gloriosa mater Salvatoris,” with
Latin and Norman French words, in score in three parts,
on fifteen red lines undivided, and with three clefs for the
voices. The remainder of the musical portion of the
manuscript consists of hymns, \&c., in one or two parts.
} It is not in Gregorian
notation, which might have been a bar to all improvement, but very much resembles
that of Walter Odington, in 1230. All the notes are black. It has neither
marks for time, the red note, nor the white open note, all of which were in use in
the following century.

The chief merit of this song is the airy and pastoral correspondence between
the words and music, and I believe its superiority to be owing to its having been
a national song and tune, selected, according to the custom of the time, as a basis
for harmony, and that it is not entirely a scholastic composition. The fact of its
having a natural drone bass would tend rather to confirm this view than otherwise.
The bagpipe, the true parent of the organ, was then in use as a rustic instrument
throughout Europe. The rote, too, which was in somewhat better estimation, had
a drone, like the modern hurdy-gurdy, from the turning of its wheel. When the
canon is sung, the key note may be sustained \textit{throughout}, and it will be in accordance
with the rules of modern harmony. But the foot, or burden, as it stands
in the ancient copy, will produce a very indifferent effect on a modern ear,\dcfootnote{
We ought, perhaps, to except the lover of Scotch
Reels.
} from its
constantly making fifths and octaves with the voices, although such progressions
were not forbidden by the laws of music in that age. No subject would be more
natural for a pastoral song than the approach of Summer; and, curiously enough,
the late Mr. Bunting noted down an Irish song from tradition, the title of
which he translated “Summer is coming,” and the tune begins in the same way.
That is the air to which Moore adapted the words, “Rich and rare were the gems 
she wore.” Having given a fac-simile of
\pagebreak
“Sumer is icumen in,” taken from the
%024
\markboth{sumer is icumen in.}{songs with music---thirteenth century.}
manuscript, and as it may be seen in score in Burney and Hawkins’ Histories, 
the tune is here printed, harmoilized by Mr. Macfarren, as the first of National
English Airs. A few obsolete words have been changed, but the original are
given below.
%\backskip{1}

\musictitle{Sumer Is Icumen In}
\backskip{2}

\noindent\begin{minipage}{\textwidth}
\musicinfo{Rather slow and smoothly.}{About 1250.}
\vspace{\baselineskip}

\includemusic{chappellV1003.pdf}
\end{minipage}


\noindent\small\begin{tabular}{ll}
\hspace*{\fill}\textsc{Original Words.}{\hspace*{\fill}}&\hspace*{\fill}\textsc{Words Modernized.}{\hspace*{\fill}}\\
 &\\
 Sumer is icumen\footnotemark in,&Summer is come in,\\
\vin Lhude\footnotemark sing Cuccu,&\vin Loud sing Cuckoo!\\
 Groweth sed, and bloweth med&Groweth seed, and bloweth mead\\
\vin And springth the wde nu&\vin And spring’th the wood now\\
\vin\vin Sing Cuccu!&\vin\vin Sing Cuckoo.\\
\end{tabular}

\footnotetext[1]{
“\textit{icumen}” come (from the Saxon verb, \textit{cuman}, to
come); so in Robert of Gloucester, \textit{i\/}paied for paid.
} 
\footnotetext[2]{
Lhude, wde, awe, and calve, are all to be pronounced as
of two syllables.
}
\pagebreak
%025

\noindent\begin{tabular}{ll}
Awe bleteth after lomb&Ewe bleateth after lamb,\\
\vin Lhouth after calve cu;&\vin Loweth after calf [the] cow.\\
 Bulluc sterteth, bucke verteth&Bullock starteth,\footnotemark buck verteth\footnotemark \\
\vin Murie sing Cuccu,&\vin Murie sing Cuccu,\\
\vin\vin Cuccu, Cuccu.&\vin\vin Cuckoo, Cuckoo!\\
 Wel singes thu Cuccu&Well sing’st thou Cuckoo\\
 Ne swik thu naver nu.&Nor cease thou never now.\\
\end{tabular}

\footnotetext[1]{
Jumps.
}
\footnotetext[2]{
Frequents the green fern.
}
\medskip

\normalsize

In the original, the “Foot,” or Burden, is sung, as an under part by two 
voices, to the words, “Sing Cuccu, nu, sing Cuccu,” making a rude base to it.

Two other songs of the thirteenth century on the approach of Summer are
printed in Reliquiæ Antiquæ (8vo. Bond. 1841), but without music. The first
is taken from MSS. Egerton, No. 613, Brit. Mus., and begins thus:—

\settowidth{\versewidth}{“Somer is comen, and winter is gon, this day beginniz to longe [lengthen],}

\begin{scverse}
\vleftofline{“}Somer is comen, and winter is gon, this day beginniz to longe [lengthen],\\
And this foules everichon [birds every one] joy [t]hem wit[h] songe.”
\end{scverse}

The other from MSS. Digby, No. 86, Oxford, of the Thrush and the Nightingale:

\settowidth{\versewidth}{With blostme [blossom], and with brides roune [birds’ songs]}

\begin{scverse}
\indentpattern{001}
\begin{patverse}
\vleftofline{“}Somer is comen with love to toune\\
With blostme [blossom], and with brides roune [birds’ songs]\\
The note [nut] of hasel springeth,’' \&c.
\end{patverse}
\end{scverse}

In the Douce Collection (Bod. Lib., Ox., MS. No. 139), there is an English
song with music, beginning—
\settowidth{\versewidth}{Foweles in the frith, the fisses in the flod.}
\begin{scverse}
“Foweles in the frith, the fisses in the flod.”
\end{scverse}
and the MS., which contains it, is of the thirteenth century, but it is only in
two parts; and in Harl. MSS. No. 1717, is a French or Anglo-Norman song,
“Parti de Mal,” which seems to have been cut from an older manuscript to form
the cover of a Chronicle of the Dukes of Normandy, written by order of Henry II.
It is only for one voice, and a sort of hymn, but a tolerable melody. Both these
may be seen in Stafford Smith’s Musica Antiqua, Vol. 1.

Another very early English song, with music, is contained in a manu\-script,
“Liber de Antiquis Legibus,” now in the Record Room, Town Clerk’s Office,
Guildhall. It contains a Chronicle of Mayors and Sheriffs of London, and of the
events that occurred in their times, from the year 1188 to the month of August,
1274, at which time the manuscript seems to have been completed. It is the
Song of a Prisoner. The first four lines are more Saxon than modern English:—
\medskip

\noindent\small\begin{tabular}{ll}
\hspace*{\fill}\textsc{Original Words.}{\hspace*{\fill}}&\hspace*{\fill}\textsc{Words Modernized.}{\hspace*{\fill}}\\
&\\
Ar ne kuthe ich sorghe non&Ere [this] knew I sorrow none\\
Nu ich mot manen min mon&Now I must utter my moan\\
Karful wel sore ich syche&Full of care well sore I sigh\\
Geltles ihc sholye muchele schame&Guiltless I suffer much shame\\
Help, God, for thin swete name&Help, God, for thy sweet name,\\
\vin Kyng of Hevene riche.&\vin King of Heaven-Kingdom.\\
\end{tabular}
\normalsize
\medskip

\pagebreak
