%299
\changefontsize{0.99\defaultfontsize}

\settowidth{\versewidth}{O that she could rule her tongue!}
\begin{dcverse}
\begin{patverse}
Her needle she can use well,\\
In that she doth most excel;\\
She can spin and knit,\\
And every thing fit,\\
As all her neighbours can tell.\\
Her fingers apace\\
At weaving bone-lace\\
She useth all day long.\\
All arts that be\\
To women free,\\
Of each degree,\\
Performeth she:\\
O that she could rule her tongue!
\end{patverse}

\begin{patverse}
For huswifery she doth exceed;\\
She looks to her business with heed;\\
She’s early and late\\
Employ’d, I dare say’t,\\
To see all things well succeed.\\
She is very wary\\
To look to her dairy,\\
As doth to her charge belong;\\
Her servants all\\
Are at her call,\\
But she’ll so brawl\\
That still I shall\\
Wish that she could hold her tongue.
\end{patverse}
\end{dcverse}

\musictitle{The Queen’s Old Courtier.}

This ballad, which obtained a long and extensive popularity, seems to have
been first printed in the reign of James I. (by T. Symcocke).

Pepys thus refers to it in his Diary, under the date of 16th of June, 1668.
“Came to Newbery, and there dined, and music: a song of the Old Courtier of
Queen Elizabeth’s, and how he was changed upon the coming in of the King, did
please me mightily, and I did cause W. Hewer to write it out.” There are many
other versions of the ballad (sometimes entitled “The Old and New Courtier”),
and some are of greater length than others. Besides those in the great collections,
copies will be found in \textit{Le Prince d’Amour}, 1660; \textit{Antidote to Melancholy}, 1661;
\textit{Wit and Drollery}, 1682; Dryden’s \textit{Miscellany Poems}, iv., 108 (1716), \&c.

In \textit{Le Prince d’Amour}, and in \textit{Merry Drollery Complete}, 1661 and 1670, there
is a song of “An old \textit{Soldier} of the Queen’s commencing—
\settowidth{\versewidth}{With an old motley coat and a malmsey nose,”}
\begin{scverse}
\vleftofline{“}Of an old Soldier of the Queen’s,\\
With an old motley coat and a malmsey nose,”
\end{scverse}
and in \textit{Wit and Drollery}, 1682, p. 165, one entitled “Old Soldiers;” commencing—
\settowidth{\versewidth}{And we old fiddlers have forgot who they were,”}
\begin{scverse}
\vleftofline{“}Of old soldiers the song you would hear.\\
And we old fiddlers have forgot who they were,”
\end{scverse}
and at p. 282, “The new Soldier” (“With a new beard,” \&c.).

A ballad, written on the occasion of the overthrow of the Rump Parliament,
by General Monck, and dated Feb. 28, 1659, is amongst the King’s Pamphlets,
Brit. Mus. (folio broadsides, vol. xvi.). It“is entitled “Saint George and the
Dragon, anglice Mercurius Poeticus.” To the tune of “\textit{The old Souldier of the
Queen’s};” commencing—
\settowidth{\versewidth}{A dialogue between Haselrigg the baffled, and Arthur the furious}
\begin{scverse}
\vleftofline{“}News, news,—here’s the occurrences and a new Mercurius,\\
A dialogue between Haselrigg the baffled, and Arthur the furious,\\
With Ireton’s readings upon legitimate and spurious, \&c.”
\end{scverse}
It is reprinted in Wright’s Political Ballads (Percy Soc., No. 11).

In the reign of Charles II., “T. Howard, Gent.,” wrote and published “An
old song of the Old Courtiers of the \textit{King’s}, \pagebreak with a new song of a New Courtier of
%300
the King’s: to the tune of \textit{The Queen’s Old Courtier}.” A copy of this latter,
“printed for F. Coles,” is among the Roxburghe Ballads.

Dr. King, in his “Preface to the Art of Cookery, in imitation of Horace’s Art
of Poetry,” declares his love “to the old British Hospitality, charity and valour,
when the arms of the family, the old pikes, muskets, and halberts, hung up in
the hall over the long table, and \textit{Chevy Chase}, and \textit{The Old Courtier of the Queen’s}
were placed over the carved mantle-piece, and beef and brown bread were carried
every day to the poor.” (Dr. King’s Works, vol. iii.)

About the middle of the last century the ballad was revived and sung by
Mr. Vernon in Shadwell’s comedy, \textit{The Squire of Alsatia}, the burden being altered
to “Moderation and Alteration,” and, when comparing the young courtier to
the old, to—
\settowidth{\versewidth}{’Tis a wonderful alteration.”}
\begin{scverse}
\vleftofline{“}Alteration, alteration,\\
’Tis a wonderful alteration.”
\end{scverse}

Finally, it has been again revived, with further “alteration,” in the present
century, under the title of “The old English Gentleman.”

The ballad is to be chanted, \textit{ad libitum}, upon one note, except the final syllable
of each stanza, and the burden “Like an old Courtier,” \&c.

\musicinfo{To be sung ad. lib. upon one note.}{}

\includemusic{chappellV1159.pdf}

\settowidth{\versewidth}{Who every quarter pays her old servants their}
\begin{dcverse}With an old lady whose anger a good word assuages,\\
Who every quarter pays her old servants their wages\\
Who never knew what belonged to coachman, footmen, nor pages;\\
But kept twenty old fellows with blue coats and badges.\\
Like an old Courtier, \&c.

With an old study fill’d full of learned old books,\\
With an old reverend parson, you may judge him by his looks.\\
With an old buttery hatch worn quite off the hooks\\
And an old kitchen, that maintains half-a-dozen old cooks.\\
Like an old, \&c.

\end{dcverse}
\pagebreak
%301

\begin{dcverse}\footnotesizerr
With an old hall hung about with guns, pikes, and bows,	\\
With old swords and bucklers that have stood many shrewd blows,\\
And an old frieze coat to cover his worship’s trunk hose,\\
And a cup of old sherry to comfort his copper nose. Like an old, \&c.

With an old fashion when Christmas was come,\\
To call in his neighbours with bagpipe and drum;\\
And good cheer enough to furnish every old room,\\
And old liquor able to make a cat speak and a man dumb. Like an old, \&c.

With an old huntsman, a falconer, and a kennel of hounds;	\\
Which never hunted nor hawked but in his own grounds;\\
Who like an old wise man kept himself within his own bounds,\\
And when be died, gave every child a thousand old pounds. Like an old, \&c.

But to his eldest son, his house and land he assigned,\\
Charging him in his will to keep the old bountiful mind,\\
To love his good old servants and to his neighbours be kind;\\
But in the ensuing ditty you shall hear how he was inclin’d. Like a young Courtier, \&c.

Like a young gallant newly-come to his land,\\
That keeps a brace of creatures at his command,\\
And takes up a thousand pound upon his own land\\
And lies drunk in a new tavern, ’till he can neither go nor stand. Like a young, \&c.

With a neat lady that is brisk and fair,\\
That never knew what belonged to good house-keeping or care,\\
But buys several fans to play with the wanton air,\\
And seventeen or eighteen dressings of other women’s hair. Like a young, \&c.

With a new hall built where the old one stood,\\
Wherein is burned neither coal nor wood.\\
And a shovelboard-table whereon meat never stood,\\
Hung round with pictures that do the poor no good. Like a young, \&c.

With a new study stuft full of pamphlets and plays;\\
With a new chaplain that swears faster than he prays ;\\
With a new buttery hatch that opens once in four or five days;\\
With a new French cook to make kickshaws and toys. Like a young, \&c.

With a new fashion when Christmas is come,\\
With a new journey up to London we must be gone,\\
And leave nobody at home but our new porter John,\\
Who relieves the poor with a thump on the back with a stone. Like a young, \&c.

With a gentleman usher whose carriage is complete;\\
With a footman, coachman, and page to carry meat;\\
With a waiting-gentlewoman whose dressing is very neat;\\
Who, when the master has dined, lets the servants not eat. Like a young, \&c.
\end{dcverse}

\settowidth{\versewidth}{With a new honour bought with the old gold,}
\begin{scverse}With a new honour bought with the old gold,\\
That many of his father’s old manors had sold,\\
And this is the occasion that most men do hold,\\
That good house-keeping is now grown so cold. Like a young, \&c.
\end{scverse}

\musictitle{Joan, To The Maypole.}

This ballad is in the Roxburghe Collection, ii. 354, and Douce Collection,
p. 152. It is entitled “May-day Country Mirth; or The young Lads’ and
Lasses’ innocent Recreation, which is to be prized before courtly pomp and pastime: 
to an excellent new tune.” Dr. Rimbault, in his “Little Book of Songs and
Ballads, gathered from Ancient Music-books,” prints a version “from a MS.
volume of old songs and music, formerly in the possession of the Rev. H. J.
Todd, dated 1630.” The same is in Evans’ \textit{Old Ballads} i. 245 (1810). Another
version will be found with the tune in \textit{ Pills to purge Melancholy}, ii. 145 (1707),
or iv. 145 (1719), with many more stanzas.
\pagebreak
%302

\musicinfo{Gaily.}{}

\includemusic{chappellV1160.pdf}

\pagebreak
%303

\settowidth{\versewidth}{Joan, shall we have a Hay or a Round,}
\indentpattern{00000101}
\begin{dcverse}\footnotesizerr
\begin{patverse}
Joan, shall we have a Hay or a Round,\\
Or some dance that is new-found?\\
Lately I was at a Masque in the Court,\\
Where I saw of every sort,\\
Many a dance made in France,\\
Many a Braule, and many a Measure;\\
Gay coats, sweet notes,\\
Brave wenches—O ’twas a treasure.
\end{patverse}


But now, methinks, these courtly toys\\
Us deprive of better joys:\\
Gown made of gray, and skin soft as silk,\\
Breath sweet as morning milk;\\
O, these more please;\\
{[All]} these hath my Joan to delight me:\\
\vin False wiles, court smiles,\\
None of these hath my Joan to despite me.

\end{dcverse}

In \textit{ Pills to purge Melancholy}, the above second and third stanzas are replaced
by others, such as the following:—

\begin{dcverse}\footnotesizerr
\begin{patverse}
Did you not see the Lord of the May\\
Walk along in his rich array?\\
There goes the lass that is only his;\\
See how they meet, ,and how they kiss!\\
Come Will, run Gill,\\
Or dost thou list to lose thy labour;\\
Kit, Crowd, scrape aloud,\\
Tickle up Tom with a pipe and tabor.
\end{patverse}

\begin{patverse}
Lately I went to a Masque at the Court,\\
Where I saw dances of every sort;\\
There they did dance with time and measure,\\
But none like a country-dance for pleasure;\\
They did dance as in France,\\
Not like the English lofty manner;\\
And every she must furnished be\\
With a feathered knack, when she’s hot for to fan her.
\end{patverse}

\begin{patverse}
But we, when we dance, and do happen to sweat,\\
Have a napkin in hand for to wipe off the wet;\\
And we with our lasses do jig it about,\\
Not like at Court, where they often are out;\\
If the tabor play, we jump away,\\
And turn, and meet our lasses to kiss 'em;\\
Nay, they will he as ready as we,\\
That hardly at any time can we miss 'em.
\end{patverse}
\columnbreak

\begin{patverse}
Come, sweet Joan, let us call a new dance,\\
That we before ’em may advance;\\
Let it be what you desire and crave,\\
And sure the same sweet Joan shall have.\\
She cried, and replied,\\
If to please me thou wilt endeavour,\\
Sweet Pig, the Wedding Jig,\\
Then, my dear, I’ll love thee for ever.
\end{patverse}

\begin{patverse}
There is not any that shall outvie\\
My litttle pretty Joan and I;\\
For I am sure I can dance as well\\
As Robin, Jenny, Tom, and Nell:\\
Last year we were here,\\
When rough Ralph he played us a Boree,\\
And we merrily\\
Thump’d it about, and gain’d the glory.
\end{patverse}

\begin{patverse}
And if we hold on as we begin,\\
Joan, thou and I the garland shall win;\\
Nay, if thou live till another day,\\
I’ll make thee Lady of the May.\\
Dance about, in and out,\\
Turn and kiss, and then for greeting;\\
Now, Joan, we have done,\\
Fare thee well till next merry meeting.
\end{patverse}
\end{dcverse}

\musictitle{Love Will Find Out The Way.}

This tune is contained in Playford’s \textit{Musick’s Recreation on the Lyra Viol},
1652; in \textit{Musick’s Delight on the Cithren}, 1666; in the Skene and several other
MSS.; also in \textit{ Pills to purge Melancholy}, vi. 86 (1719).

The words are in Percy’s \textit{Reliques}; Evans’ \textit{Old Ballads}, iii. 282 (1810); and
Rimbault’s \textit{Little Book of Songs and Ballads}, p. 137. All these versions differ.

Evans prints from a black-letter copy by F. Coules (whose ballads occasionally
bear dates which vary from 1620 to 1628); Rimbault from Forbes’ \textit{Cantus}, 1662,
with the second part from Coules’ copy; and Percy from a comparatively modern
edition.

The ballad is quoted in Brome’s \textit{Sparagus Garden}, acted in 1635, and its
popularity was so great, that “Love will find out the way” was taken as the
title to a play printed in 1661. Although stated on the title-page to be a
comedy by T. B., it was only Shirley’s \textit{Constant Maid}, under a new name.
\pagebreak
%304

The air is still current, for in the summer of 1855, Mr. Jennings, Organist of
All Saints’ Church, Maidstone, noted it down from the wandering hop-pickers
singing a song to it, on their entrance into that town.

The title of the ballad, as printed by Coules, is “Truth’s Integrity; or
A curious Northern ditty, called \textit{Love will find out the way}: to a pleasant new
tune.” A later copy in the Douce Collection, p. 232, is entitled “A curious
Northern ditty, called \textit{Love will find out the way}.”

In the Roxburghe Collection, ii. 436, is a black-letter ballad of “Stephen and
Cloris; or The coy Shepherd and the kind Shepherdess: to a new play-house
tune, or \textit{Love will find out the way}.”

I suppose ballads which are said to be “to the tune of \textit{Over hills and high
mountains},” are also intended for this air; because the words of that ballad are
almost a paraphrase of this, and in the same measure. See the following stanza
from a copy in the Pepys Collection, iii. 165:—

\settowidth{\versewidth}{Ah! and down by the fountains,}
\begin{dcverse}\begin{altverse}
\vleftofline{“}Over hills and high mountains\\
Long time have I gone;\\
Ah! and down by the fountains,\\
By myself all alone;
\end{altverse}

\begin{altverse}
Through bushes and briers,\\
Being void of all care;\\
Through perils and dangers\\
For the loss of my dear.”
\end{altverse}
\end{dcverse}

There is, however, an air, entitled \textit{On yonder high mountains}, which may be intended,
and which will be found in this collection, under a later date.

Another black-letter ballad to the tune of \textit{Love will find out the way}, is entitled
“The Countryman’s new Care away;” commencing—

\begin{dcverse}\begin{altverse}
\vleftofline{“}If there were employments\\
For men, as have been;\\
And drums, pikes, and muskets,\\
I’ the field to he seen;
\end{altverse}

\begin{altverse}
And every worthy soldier\\
Had truly his pay;\\
Then might they be bolder\\
To sing Care away.”
\end{altverse}
\end{dcverse}

As the version of \textit{Love will find out the way} printed by Percy is the shortest,
consisting in all of but five stanzas, it is here coupled with the tune.

\musicinfo{Smoothly and not too fast.}{}

\includemusic{chappellV1161.pdf}

\pagebreak
%305 

\settowidth{\versewidth}{Where the midge dares not venture,}
\begin{dcverse}\begin{altverse}
Where there is no place\\
For the glow-worm to lie;\\
Where there is no space\\
For receipt of a fly;\\
Where the midge dares not venture,\\
Lest herself fast she lay;\\
If Love come, he will enter,\\
And soon find out his way.
\end{altverse}

\begin{altverse}
You may esteem him\\
A child for his might;\\
Or you may deem him\\
A coward from his flight.\\
But if she, whom Love doth honour,\\
Be conceal’d from the day,\\
Set a thousand guards upon her,\\
Love will find out the way.
\end{altverse}

\begin{altverse}
Some think to lose him,\\
By having him confin’d;\\
And some do suppose him,\\
Poor thing, to be blind;\\
But if ne’er so close ye wall him,\\
Do the best that you may,\\
Blind Love, if so ye call him,\\
Soon will find out his way.
\end{altverse}

\begin{altverse}
You may train the eagle\\
To stoop to your fist;\\
Or you may inveigle\\
The phœnix of the east;\\
The lioness, ye may move her\\
To give o’er her prey;\\
But you’ll ne’er stop a lover:\\
He will find out his way.
\end{altverse}
\end{dcverse}

\backskip{2}

\musictitle{Stingo, Or Oil Of Barley.}

This tune is contained in every edition of \textit{The Dancing Master}, and in many
other publications. It is often quoted under three, if not more, names.

In \textit{The Dancing Master}, from 1650 to 1690, it appears as \textit{Stingo}, or \textit{The Oyle
of Barley}.

The song, “A cup of old stingo” (\ie, old strong beer), is contained in \textit{Merry
Drollery Complete}, 1661 and 1670, and, if it be the original song, must be of a
date from thirty to forty (and perhaps more) years earlier than the book.

Traces of that doughty hero, Sir John Barleycorn, so famous in the days of
ballad-singing, are to be found as far back as the time of the Anglo-Saxons. In
the Exeter MS. (fol. 107) is an enigma in Anglo-Saxon verse, of which the
following is a literal translation;—

{\footnotesizerr “A part of the earth is prepared beautifully with the hardest, and with the sharpest,
and with the grimest of the productions of men, cut and\ldots . (sworfen), turned and
dried, bound and twisted, bleached and awakened, ornamented and poured out, carried
afar to the doors of people; it is joy in the inside of living creatures, it knocks and
slights those, of whom before, while alive, a long while it obeys the will, and expostulateth
not; and then after death it takes upon it to judge, to talk variously. It is
greatly to seek by the wisest man, what this creature is.”—\textit{Essay on the State of
Literature and Learninq under the Anglo-Saxons, by Thomas Wright, Esq., M.A.,
F.S.A.}, p. 79, 8vo., 1839.}

In the Roxburghe Collection, i. 214, there is a black-letter ballad “to the tune
of \textit{Stingo},” which was evidently written in the reign of Charles I., by its
allusions to “the King’s great porter,” “Bankes’ Horse,” \&c. It is entitled,
“The Little Barley-Corn:
\settowidth{\versewidth}{Whose properties and vertues here}
\begin{scverse}Whose properties and vertues here\\
Shall plainly to the world appeare;\\
To make you merry all the yeere.”
\end{scverse}

As it has been reprinted in Evans’ \textit{Old Ballads} i. 156 (1810), the first stanza
only is subjoined:—
\begin{dcverse}\begin{altverse}
“Come, and do not musing stand,\\
If thou the truth discern;\\
But take a full cup in thy hand,\\
And thus begin to learn:\\
Not of the earth, nor of the air,\\
At evening or at morn,\\
But, jovial boys, your Christmas keep\\
With the little barley-corn.”
\end{altverse}
\end{dcverse}
\noindent The ballad is divided into two parts, each consisting of eight stanzas.
\pagebreak
%306

A second name for the tune is \textit{The Country Lass}, which it derived from a
ballad by Martin Parker. Copies of that ballad are in the Pepys Collection
(i. 268), and in the Roxburghe (i. 52). The former bears Martin Parker’s
initials, but no printer’s name; the latter was printed for the assigns of Thomas
Symcocke.

The copy in the Pepys Collection is entitled “The Countrey Lasse:
\settowidth{\versewidth}{To a dainty new note: which if you cannot hit,}
\begin{scverse}
To a dainty new note: which if you cannot hit,\\
There’s another tune which doth as well fit—\\
That’s \textit{The Mother beguil’d the Daughter}.”
\end{scverse}

\settowidth{\versewidth}{Although I am a countrey lasse,}
\begin{dcverse}\begin{altverse}
\vleftofline{“}Although I am a countrey lasse,\\
A loftie minde I beare-a;\\
I thinke myselfe as good as those\\
That gay apparrell weare-a.\\
My coat is made of comely gray,\\
Yet is my skin as soft-a,\\
As those that with the choicest wines\\
Do bathe their bodies oft-a.
\end{altverse}

\begin{altverse}
Downe, downe derry, derry downe,\\
Heigh downe, a downe, a downe-a,\\
A derry, derry, derry downe,\\
Heigh downe, a downe, a derry.”
\end{altverse}
\end{dcverse}

This is reprinted in Evans’ \textit{Old Ballads}, i. 41, and an altered copy will be found,
with the music, in \textit{ Pills to purge Melancholy}, ii. 165 (1707), or iv. 152 (1719).

The tune is referred to, under the above name, in a ballad by Laurence Price,
entitled “Good Ale for my money:
\settowidth{\versewidth}{The good fellowes resolution of strong ale,}
\begin{scverse}
The good fellowes resolution of strong ale,\\
That cures his nose from looking pale.
\end{scverse}
To the tune of \textit{The Countrey Lasse}.

\settowidth{\versewidth}{Be merry, my friends, and list awhile}
\begin{dcverse}\begin{altverse}
Be merry, my friends, and list awhile\\
Unto a merry jest,\\
It may from you produce a smile,\\
When you hear it exprest;\\
Of a young man lately married,\\
Which was a boone good fellow,
\end{altverse}

\begin{altverse}
This song in’s head he always carried,\\
When drink had made him mellow:\\
I cannot go home, nor will I go home,\\
It’s long of the \textit{oyle of barley};\\
I’ll tarry all night for my delight.\\
\textit{And go home in the morning early}.”
\end{altverse}
\end{dcverse}

A copy will be found in the Roxburghe Collection, i. 138.

Hilton wrought this tune into a catch for three voices, and published it in his
\textit{Catch that catch can}, in 1652; and it was afterwards reprinted in that form by
Playford in his \textit{Musical Companion}, 1667, 1673, \&c.

The first line of the catch is “I’se goe with thee, my sweet Peggy, my honey.”
The third part is to the tune of \textit{Stingo}, with the following words:—

\settowidth{\versewidth}{And what we doe neene shall know;}
\indentpattern{00101}
\begin{scverse}
\begin{patverse}
\vleftofline{“}Thou and I will foot it, Joe,\\
And what we doe neene shall know;\\
But taste the \textit{juice of barley}.\\
We’ll sport all night for our delight,\\
And \textit{home in the morning early}.”
\end{patverse}
\end{scverse}

The air is somewhat altered to harmonize with the other parts.

In the editions of \textit{The Dancing Master} which were printed \textit{after} 1690, the
name is changed from \textit{Stingo}, or \textit{The Oyle of Barley}, to \textit{Cold and raw}. This new
title was derived from a (so called) “New Scotch Song,” written by Tom
D’Urfey, which first appeared in the second \pagebreak book of \textit{Comes Amoris}, or \textit{The}
%307
\textit{Companion of Love}, printed by John Carr in 1688;\footnote{\textit{}
a A few pages further in the same book there is another
“new Scotch song,” set by Mr. Akeroyd.

Ritson, in his \textit{Historical Essay on Scotish Song}, 1794,
says, “An inundation of \textit{Scotch songs}, so called, appears
to have been poured upon the town by Tom D’Urfey and
his Grub-street brethren, toward the end of the seventeenth
and in the beginning of the eighteenth century; of
which it is hard to say whether wretchedness of poetry,
ignorance of the Scotish dialect, or nastiness of ideas, is
most evident, or most despicable. In the number of
these miserable caricatures, the reader may be a little surprised
to find the favorite songs of \textit{De’ill take the Wars
that hurry'd Willy from me}; \textit{O Jenny, Jenny, where hast thou
been}? \textit{Young Philander wooed me lang; Farewell, my
bonny, witty, pretty Moggy; In January last; She rose and
let me in; Pretty Kate of Edinburgh; As I sat at my spinning
wheel; Fife, and a’ the lands about it; Bonny lad,
prithee lay thy pipe down; The bonny grey-ey'd morn;
’Twas within a furlong of Edinburgh town; Bonny Dundee;
O'er the hills and far away; By moonlight on the green;
What’s that to you}? and several others, which he has
been probably used to consider as genuine specimens of
Scotish song; as, indeed, most of them are regarded
even in Scotland.” Ritson's list might be very greatly
extended.}
 and, as frequently the case,
the air was a little altered for the words.

Of this song Sir John Hawkins relates the following anecdote in his \textit{History of
Music} (8vo., ii. 564):—

\begin{quotation}\small
“This tune was greatly admired by Queen Mary, the consort of King William;
and she once affronted Purcell by requesting to have it sung to her, he being present.
The story is as follows: The Queen having a mind one afternoon to be entertained
with music, sent to Mr. Gosling, then one of her Chapel, and afterwards Sub-Dean of
St. Paul’s, to Henry Purcell, and to Mrs. Arabella Hunt, who had a very fine voice,
and an admirable hand on the lute, with a request to attend her; they obeyed her
commands; Mr. Gosling and Mrs. Hunt sung several compositions of Purcell, who
accompanied them upon the harpsichord; at length, the Queen beginning to grow
tired, asked Mrs. Hunt if she could not sing the ballad of ‘Cold and raw;’\footnote{\textit{}
Sir John Hawkins, who relates the anecdote \textit{traditionally},
and who had evidently seen no older copy of the tune
than that contained in the Catch (as he elsewhere mentions
Hilton’s \textit{Catches} as Playford’s \textit{first} publication) calls
it “the old Scot's ballad,” but from the allusion to “the
next birth-day song,” it must have happened within four
years of the first publication. The term “old,” could
therefore only be applied, with propriety, to the music.}
 Mrs.
Hunt answered, Yes, and sung it to her lute. Purcell was all the while sitting at the
harpsichord unemployed, and not a little nettled at the Queen’s preference of a vulgar
ballad to his music; but seeing Her Majesty delighted with this tune, he determined
that she should hear it upon another occasion; and, accordingly, in the next birthday
song, viz., that for the year 1692, he composed an air ta the words, ‘May her
bright example chace vice in troops out of the land,’ the bass whereof is the tune to
‘Cold and raw.’”
\end{quotation}

In Anthony à Wood’s collection of broadsides (Ashmolean Library, vol. 417)
there are two ballads with music, bearing the date of December, 1688, and
printed to this tune. The first is “The Irish Lasses Letter; or her earnest
request to Teague, her dear joy: \textit{to an excellent new tune}.” The second is the
famous song of \textit{Lilliburlero}.

In the Douce Collection is a ballad called “The lusty Friar of Flanders: to
the tune of \textit{Cold and raw}.”

Horace Walpole mentions it under the same name in a letter to Richard West,
Esq., dated from Florence (Feb. 27, 1740), where, in speaking of the Carnival,
he says, “The Italians are fond to a degree of our Country Dances.\footnote{\textit{}
This agrees with what I have been told about the book
entitled \textit{The Dancing Master} (the early editions of which
are extremely scarce in England), viz., that it is very well
known to the dealers in Italy, and that it may be procured
there with comparatively little trouble.}
 \textit{Cold and
raw} they only know by the tune; \textit{Blowzybella} is almost Italian, and \textit{Butter'd
Peas} is \textit{Pizelli al buro}.” (\textit{Letters of Walpole}, in vi. vols, 1840; vol. i. p. 32.)

The following is the song of “A cup of old stingo,” from \textit{Merry Drollery
Complete}, with the tune from \textit{The Dancing Master} of 1650.
\pagebreak
%308

\musicinfo{Jovially.}{}

\includemusic{chappellV1162.pdf}

\settowidth{\versewidth}{’Twill make him dance about a cross,}
\begin{dcverse}\begin{altverse}
’Twill make a man indentures make,\\
’Twill make a fool seem wise,\\
’Twill make a Puritan sociate,\\
And leave to be precise:\\
’Twill make him dance about a cross,\\
And eke to run the ring too,\\
Or anything he once thought gross,\\
Such virtue hath old stingo.
\end{altverse}

\begin{altverse}
’Twill make a constable oversee\\
Sometimes to serve a warrant,\\
’Twill make a bailiff lose his fee,\\
Though he be a knave-arrant;\\
’Twill make a lawyer, though that he\\
To ruin oft men brings, too,\\
Sometimes forget to take his fee,\\
If his head be lin’d with stingo.
\end{altverse}

\begin{altverse}
’Twill make a parson not to flinch,\\
Though he seem wondrous holy,\\
And for to kiss a pretty wench,\\
And think it is no folly;\\
’Twill make him learn for to decline\\
The verb that’s called \textit{Mingo},\\
’Twill make his nose like copper shine,\\
If his head be lin’d with stingo.
\end{altverse}

\begin{altverse}
’Twill make a weaver break his yarn,\\
That works with right and left foot,\\
But he hath a trick to save himself,\\
He’ll say there wanteth woof to’t;\\
’Twill make a tailor break his thread,\\
And eke his thimble ring too,\\
’Twill make him not to care for bread,\\
If his head be lin’d with stingo.
\end{altverse}

\begin{altverse}
’Twill make a baker quite forget\\
That ever corn was cheap,\\
’Twill make a butcher have a fit\\
Sometimes to dance and leap;\\
’Twill make a miller keep his room,\\
A health for to begin, too,\\
’Twill make him shew his golden thumb.\\
If his head be lin’d with stingo.
\end{altverse}

\begin{altverse}
’Twill make an hostess free of heart,\\
And leave her measures pinching,\\
’Twill make an host with liquor part\\
And bid him hang all flinching;\\
It’s so belov’d, I dare protest,\\
Men cannot live without it,\\
And where they find there is the best,\\
The most will flock about it.
\end{altverse}
\end{dcverse}
\pagebreak
%309

\settowidth{\versewidth}{Though he be lame, he’ll prove his crutch,}
\begin{dcverse}\begin{altverse}
And, finally, the beggar poor,\\
That walks till he be weary,\\
Craving along from door to door,\\
With \textit{pre-commiserere};\\
If he do chance to catch a touch,\\
Although his clothes be thin, too,\\
Though he be lame, he’ll prove his crutch,\\
If his head be lin’d with stingo.
\end{altverse}

\begin{altverse}
Now to conclude, here is a health\\
Unto the lad that spendeth,\\
Let every man drink off his can,\\
And so my ditty endeth;\\
I willing am my friend to pledge,\\
For he will meet me one day;\\
Let’s drink the barrel to the dregs.\\
For the \textit{malt-man comes a Monday}.
\end{altverse}
\end{dcverse}

The last line has furnished the subject for a Scotch song.

The following is a later version of the tune. The copies in \textit{The Beggars’
Opera}, \textit{ Pills to purge Melancholy}, \textit{The Dancing Master}, and \textit{Midas} (1764), have
all slight differences, such as would occur from writing down a familiar tune from
memory. The words are Tom D’Urfey’s “last new Scotch song.” (See
\textit{Comes Amoris}, or \textit{The Companion of Love}, ii. 16, fol. 1688.)

\musicinfo{Gracefully.}{}

\includemusic{chappellV1163.pdf}

\begin{dcverse}\begin{altverse}
Down I veil’d my bonnet low,\\
Thinking to show my breeding;\\
She returned a graceful bow—\\
A village far exceeding.
\end{altverse}

\begin{altverse}
I ask’d her where she went so soon,\\
I long’d to begin a parley,\\
She told me to the next market town\\
On purpose to sell her barley.\footnote{\textit{}
However unobjectionable this song may have been in
Queen Mary’s time, the three remaining stanzas would
not be very courteously received in Queen Victoria's
\textit{Tempora mutantur}.}
\end{altverse}
\end{dcverse}
\pagebreak
%310


\musictitle{What If A Day, Or A Month, Or A Year?}

Copies of this song are in the Roxburghe Collection, i. 116 and ii. 182, and
in \textit{The Golden Garland of Princely Delights}, third edition, 1620. In the
Roxburghe Ballads it is entitled “A Friend’s Advice, in an excellent ditty,
concerning the variable changes in this world” (printed by the assigns of Thomas
Symcocke); in \textit{The Golden Garland}, “The inconstancy of the world.”

The music is in a volume of transcripts of virginal music, by Sir John Hawkins;
in \textit{Logonomia Anglica}, by Alexander Gil, 1619; in \textit{Friesche Lust-Hof}, 1634; in
D. R. Camphuysen’s \textit{Stichtelycke Rymen}, 4to., Amsterdam, 1647; in the Skene
MS.; in Forbes’ \textit{Cantus}; \&c. The same words are differently set by Richard
Allison, in his \textit{Howre’s Recreation in Musicke}, 1608.

Gil (or Gill), who was Master of St. Paul’s School, refers to the song twice in his
\textit{Logonomia}. Firstly, “Hemistichium est, duobus constans dactylis, et choriambo;”
and secondly, “Ut in illo perbello cantico Tho. Campaiani, cujus mensuram, ut
rectius agnoscas, exhibeo cum notis.”

Thomas Campian, or Campion, to whom the poetry, and perhaps also the
music, is here ascribed, was by profession a physician; but he was also an eminent
poet and admirable musician. He flourished during the latter part of the
reign of Elizabeth and the greater portion of that of James I. Neither the words
nor music are, however, to be found in his printed collections.

According to the registers of St. Dunstan’s in the West, “Thomas Campion,
Doctor of Physicke,” was buried there on the 1st of March, 1619.\footnote{\textit{}
Haslewood supposed him to have died in 1621, It
is strange that the name of so eminent a man should
have been omitted in the usual Biographical Dictionaries
and Universal Biographies. A short account of him is
given, with the reprint of his “Observations in the art
of English Poetry,” in Haslewood's “Ancient Critical
Essays upon English Poets and Poësy.” Haslewood
does not notice his four books of “Ayres,” printed in
1610 and 1612, which, with some others, are described in
Rimbault’s \textit{Bibliothica Madrigaliana}. He composed the
Psalm tune, called “Babylon’s streams,” which is still
in use. His \textit{Art of Descant} is contained in Playford’s
\textit{Introduction}.
}


In Camphuysen’s \textit{Stichtelycke Rymen} the song is entitled “\textit{Essex's Lamentation},
or \textit{What if a day}.”

Ritson, in a note to his \textit{Historical Essay on Scotish Song}, p. 57, says, “In a
curious dramatic piece, entitled \textit{Philotus}, printed at Edinburgh in 1603, by way
of finale, is \textit{Ane sang of the foure lufearis }(lovers), though little deserving that
title. It is followed by the old English song, beginning, ‘What if a day, or
a month, or a year?’ alluded to in \textit{Hudibras}, which appears to have been sung at
the end of the play, and was probably, at that time, new and fashionable.”

Mr. Halliwell, in a paper read before the Society of Antiquaries in Dec., 1840,
says, “It is a curious fact that one of the songs in Ryman’s well-known collection
of the fifteenth century, in the Cambridge Public Library, commences—
\settowidth{\versewidth}{Crowne my desyres wythe every delyghte; ’}
\begin{scverse}
\vleftofline{‘}What yf a daye, or nyghte, or howre,\\
Crowne my desyres wythe every delyghte; ’
\end{scverse}
and that in Sanderson’s Diary in the British Museum, MSS. Lansdowne 241,
fol.~49, temp. Elizabeth, are the two first stanzas of the song, more like the copy
in Ryman, and differing in its minor arrangements from the later version.
Moreover, that the tune in Dowland’s Musical Collection, in the Public Library,
Cambridge, is entitled ‘What if a day, or a \textit{night}, or an \textit{hour}?’ agreeing with
Sanderson’s copy.” Mr. Halliwell has reverted to the subject in \textit{Reliquæ Antiquæ},
i. 323, and ii. 123.
\pagebreak
%311

“What if a day, or a month, or a year?” is mentioned as one of the tunes for
\textit{Psalms and Songs of Sion}, by W[illiam] S[atyer], 1642. See p. 319.

\musicinfo{Rather slow.}{}

\includemusic{chappellV1164.pdf}

\pagebreak
%312

\settowidth{\versewidth}{Th’ earth’s but a point of the world, and a man}
\indentpattern{000011111111}
\begin{dcverse}\begin{patverse}
Th’ earth’s but a point of the world, and a man\\
Is but a point of the earth’s compared centre:\\
Shall then the point of a point be so vain,\\
As to triumph in a silly point’s adventure?\\
All is hazard that we have,\\
Here is nothing biding;\\
Days of pleasure are as streams\\
Through fair meadows gliding.\\
Weal or woe, time doth go,\\
Time hath no returning;\\
Secret Fates guide our states\\
Both in mirth and mourning.
\end{patverse}

\begin{patverse}
What if a smile, or a beck, or a look,\\
Feed thy fond thoughts with many vain conceivings:\\
May not that smile, or that beck, or that look,\\
Tell thee as well they are all but false deceivings?\\
Why should beauty be so proud,\\
In things of no surmounting?\\
All her wealth is but a shroud,\\
Nothing of accounting.\\
Then in this there’s no bliss,\\
Which is vain and idle,\\
Beauty’s flow’rs have their hours,\\
Time doth hold the bridle.
\end{patverse}

\begin{patverse}
What if the world, with a lure of its wealth,\\
Raise thy degree to great place of high advancing;\\
May not the world, by a check of that wealth,\\
Bring thee again to as low despised changing?\\
While the sun of wealth doth shine\\
Thou shalt have friends plenty;\\
But, come want, they repine,\\
Not one abides of twenty.\\
Wealth (and friends), holds and ends,\\
As thy fortunes rise and fall:\\
Up and down, smile and frown,\\
Certain is no state at all.
\end{patverse}

\begin{patverse}
What if a grip, or a strain, or a fit,\\
Pinch thee with pain of the feeling pangs of sickness:\\
May not that grip, or that strain, or that fit,\\
Shew thee the form of thine own true perfect likeness?\\
Health is but a glance of joy,\\
Subject to all changes;\\
Mirth is but a silly toy,\\
Which mishap estranges.\\
Tell me, then, silly man,\\
Why art thou so weak of wit.\\
As to be in jeopardy.\\
When thou mayst in quiet sit?
\end{patverse}
\end{dcverse}

\musictitle{The Hemp-Dresser, or The London Gentlewoman.}

This tune has attained a long-enduring popularity. It is to be found in every
edition of \textit{The Dancing Master}, as well as in many other publications, and is
commonly known at the present day.

The name of \textit{The Hemp-dresser}, or \textit{The London Gentlewoman}, is derived from
an old song which was translated into Latin (together with \textit{Chevy Chace} and many
others) by Henry Bold, and published, after his death, in “Latine Songs with
their English,” 1685.

One of D’Urfey’s songs, commencing, “The sun had loos’d his weary team,”
was written to this air. It is printed, with music, in his third book of songs,
1685; in Playford’s third book of “Choice Ayres and Songs;” and in vol. i.
of all the editions of \textit{Pills to purge Melancholy}. In the first, it is entitled “A new
song set to a pretty country dance, called \textit{The Hemp-dresser}:” in the second, it
has the further prefix of “The Winchester Christening; The Sequel of the
Winchester Wedding. A new song,” \&c.

In \textit{The Beggars’ Opera}, 1728; \textit{The Court Legacy}, 1733; \textit{The Sturdy Beggars},
1733; and \textit{The Rival Milliners}, 1737, the time is named “The sun had loos’d
his weary team,” from D’Urfey’s song. In other ballad-operas, such as \textit{Penelope},
1728; and \textit{Love and Revenge}, or \textit{The Vintner outwitted}, n.d., it takes the name
of one beginning, “Jone stoop’d down.” Burns also wrote a song to it—“The
Deil’s awa wi’ the Exciseman.”
\pagebreak
%313

In the “History of Robert Powel, the puppet-showman,” 8vo., 1715, \textit{The
Duke of York’s Delight; Welcome home, Old Rowley; The Knot;} and \textit{The Hemp-dressers}, 
are mentioned as favorite tunes called for by the company.

The song of \textit{The Hemp-dresser} consists of four stanzas, of which the two first
are as follows:—

\settowidth{\versewidth}{There was a London gentlewoman}
\indentpattern{01018}
\begin{dcverse}
\begin{patverse}
There was a London gentlewoman\\
That lov’d a country man-a;\\
And she did desire his company\\
A little now and then-a.\\
Fa la, \&c.
\end{patverse}

\begin{patverse}
This man he was a hemp-dresser,\\
And dressing was his trade-a;\\
And he did kiss the mistress, sir,\\
And now and then the maid-a.\\
Fa la, \&c.
\end{patverse}
\end{dcverse}

The first verse of D’Urfey’s song is here printed with the music.

\musicinfo{Gracefully.}{}

\includemusic{chappellV1165.pdf}

\backskip{1}

\musictitle{Since First I Saw Your Face.}

The following tune is by Thomas Ford, one of the musicians in the suite of
Prince Henry, the eldest son of James I. It is a song for one voice to the lute,
or for four without accompaniment, and contained in his \textit{Musicke of sundrie
Kindes} (fol. 1607.) The second part of a popular tune called \textit{Jamaica}, or \textit{My
father was born before me}, bears a resemblance to the second part of this.

In the \textit{Golden Garland of Princely Delight}, third edition, 1620, the song is
entitled, “Love’s Constancy.”
\pagebreak
%314
\changefontsize{1.03\defaultfontsize}

Ford was not a great harmonist, but this song (now miscalled a madrigal) has
survived the works of many more learned composers, and is probably as popular
at the present day as when first written. The harmony of the modern copies is
not by Ford.

\musicinfo{Slow.}{}

\includemusic{chappellV1166.pdf}

\settowidth{\versewidth}{And your sweet beauty, past compare,}
\begin{dcverse}\begin{altverse}
If I admire or praise you too much,\\
That fault you may forgive me;\\
Or if my hands had stray’d to touch,\\
Then justly might you leave me.\\
I ask'd you leave, you bade me love,\\
Is’t now a time to chide me?\\
No, no, no, I’ll love you still,\\
What fortune e’er betide me.
\end{altverse}

\begin{altverse}
The sun, whose beams most glorious are,\\
Rejecteth no beholder;\\
And your sweet beauty, past compare,\\
Made my poor eyes the bolder.\\
When beauty moves, and wit delights,\\
And signs of kindness bind me,\\
There, O there, where’er I go,\\
I’ll leave my heart behind me.
\end{altverse}

\begin{altverse}
{[}If I have wronged you, tell me wherein,\\
And I will soon amend it;\\
In recompense of such a sin,\\
Here is my heart, I’ll send it.\\
If that will not your mercy move,\\
Then, for my life I care not;\\
Then, O then, torment me still,\\
And take my life, and spare not.{]}
\end{altverse}
\end{dcverse}

I have only found the last stanza in late copies, such as \textit{Wit's Interpreter},
third edition, 8vo., 1671.
\pagebreak
%315

\musictitle{What Care I How Fair She Be?}

A copy of this song is in the Pepys Collection, i. 230, entitled “A new song of
a young man’s opinion of the difference between good and bad women. \textit{To a
pleasant new tune}.” (Printed at London for W. I.) It is also in the second part
of \textit{The Golden Garland of Princely Delights}, third edition, 1620, entitled “The
Shepherd’s Resolution. To the tune of \textit{The Young Man's Opinion}.” As the
name of the tune is here derived from the title of the ballad, it must have been
printed in ballad form before 1620, when it was published among T\textit{he Workes of
Master George Wither}.

The tune is in Heber’s Manuscript (described at p. 204), but, except for the
popularity of the words, it would scarcely be worth preserving. They were afterwards
reset by Mr. King, and are printed to his tune in \textit{ Pills to purge Melancholy}.

The first line of the copy in the Pepys Collection (unlike that in \textit{The Golden
Garland}) is, “Shall I \textit{wrestling} in dispaire.” In the same volume are the
following:—

Page 200.—“The unfortunate Gallant gull’d at London. To the tune of
\textit{Shall I wrastle in despair}.” (Printed for T. L.) Beginning—
\settowidth{\versewidth}{“From Cornwall Mount to London fair.”}
\begin{scverse}
“From Cornwall Mount to London fair.”
\end{scverse}

Page 316.—“This maid would give tenne shillings for a kisse. To the tune
of \textit{Shall I wrassle in despair}.” (Printed at London by I. White.) Beginning—
\begin{scverse}
“You young men all, take pity on me.”
\end{scverse}

Page 236.—“Jone is as good as my lady. To the tune of \textit{What care I how
fair she be}?” (Printed at London for A. M[ilbourn].) Beginning—
\begin{scverse}
“Shall I here rehearse the story.”
\end{scverse}

The following (which has been attributed, upon insufficient evidence, to Sir
Walter Raleigh) is in the same metre, and has the same burden as George
Wither’s song:—

\settowidth{\versewidth}{Calling home the smallest part}
\indentpattern{00000011}
\begin{dcverse}
\begin{patverse}
Shall I, like an hermit, dwell\\
On a rock or in a cell?\\
Calling home the smallest part\\
That is missing of my heart,\\
To bestow it where I may\\
Meet a rival every day?\\
If she undervalues me,\\
What care I how fair she be.
\end{patverse}

\begin{patverse}
Were her tresses angel-gold;\\
If a stranger may be bold,\\
Unrebuked, unafraid,\\
To convert them to a braid,\\
And, with little more ado,\\
Work them into bracelets too;\\
If the mine be grown so free,\\
What care I how rich it be.
\end{patverse}

\begin{patverse}
Were her hands as rich a prize\\
As her hairs or precious eyes;\\
If she lay them out to take\\
Kisses, for good manners sake;\\
And let every lover skip\\
From her hand unto her lip;\\
If she seem not chaste to me.\\
What care I how chaste she be.
\end{patverse}

\begin{patverse}
No, she must be perfect snow,\\
In effect as well as show,\\
Warming but as snow-balls do,\\
Not, like fire, by burning too;\\
But when she by chance hath got\\
To her heart a second lot;\\
Then, if others share with me,\\
Farewell her, whate’er she be.
\end{patverse}
\end{dcverse}
\pagebreak
%316

\musicinfo{Moderate time.}{}

\includemusic{chappellV1167.pdf}

\backskip{1}

\begin{dcverse}
\begin{patverse}
Shall my foolish heart be pin’d,\\
’Cause I see a woman kind?\\
Or a well-disposed nature,\\
Joined with a lovely feature?\\
Be she kind, or meeker than\\
Turtle-dove or pelican;\\
If she be not so to me,\\
What care I how kind she be.
\end{patverse}

\begin{patverse}
Shall a woman’s virtues move\\
Me to perish for her love?\\
Or, her well-deservings known,\\
Make me quite forget mine own?\\
Be she with that goodness blest,\\
Which may gain her name of Best;\\
If she be not so to me,\\
What care I how good she be.
\end{patverse}

\begin{patverse}
’Cause her fortune seems too high,\\
Shall I play the fool, and die?\\
He that bears a noble mind\\
If no outward help he find,\\
Think what with them he would do,\\
That without them dares to woo:\\
And, unless that mind I see,\\
What care I how great she be.
\end{patverse}

\begin{patverse}
Great, or good, or kind, or fair,\\
I will ne’er the more despair:\\
If she love me, this believe,\\
I will die ere she shall grieve.\\
If she slight me when I woo,\\
I can slight and let her go:\\
If she be not fit for me,\\
What care I for whom she be.
\end{patverse}
\end{dcverse}
\pagebreak
%317
\changefontsize{0.92\defaultfontsize}

\musictitle{The New Royal Exchange.}

In \textit{The Dancing Master} of 1665 there are two tunes under very similar titles.
The first is \textit{The New Exchange}; the second, \textit{The New New-Exchange}. The first
is sometimes called \textit{Durham Stable};\footnote{
Strype, in his edition of Stow’s \textit{London}, book vi., p. 75,
says “In the place where certain old stables stood, belonging
to this house [Durham House], is the New Exchange;
being furnished with shops on both sides the walls, both
below and above stairs, for milliners, sempstresses, and
other trades, and is a place of great resort and trade for
the nobility and gentry, and such as have occasion for
such commodities.” It was opened April 11th, 1609, in
the presence of James I. and his Queen, and taken down
in 1737. Coutts’ Banking House now stands upon the
site. Pepys, in his Diary, 15th April, 1662, says,‘“With
my wife by coach to the New Exchange, to buy her some
things; where we saw some new-fashion pettycoats of
sarcenet, with a black broad lace printed round the bottom
and before; very handsome, and my wife had a mind to
one of them.”
}
 the second, which was more frequently
used as a ballad tune, is, in other editions, named \textit{The New Royal Exchange}.

In \textit{Wit and Drollery}, 1656, p. 110, is a song to this tune—“On the Souldiers
walking in the new Exchange to affront the Ladies.” It consists of four stanzas,
the first of which is here printed with the music.

In the same book, at p. 60, is another song of six stanzas beginning—

\backskip{0.5}

\settowidth{\versewidth}{We’ll go no more to Tunbridge Well,}
\begin{dcverse}\begin{altverse}
\vleftofline{“}We’ll go no more to Tunbridge Wells,\\
The journey is too far;\\
Nor ride in Epsom waggon, where\\
Our bodies jumbled are.\\
But we will all to the westward waters go,\\
The best that e’er you saw,\\
And we will have them henceforth call’d\\
The Kentish new-found Spa.\\
Then go, lords and ladies, whate’er you ail;\\
Go thither all that pleases;\\
For it will cure you, without fail,\\
Of old and new diseases.”
\end{altverse}
\end{dcverse}

\backskip{0.5}

In \textit{Westminster Drollery}, part ii, 1671, is a third song, “to the tune of \textit{I'll go
no more to the New Exchange};” beginning—

\backskip{0.5}

\begin{dcverse}\begin{altverse}
\vleftofline{“}Never will I wed a girl that's coy,\\
Nor one that is too free;\\
But she alone shall be my joy\\
That keeps a mean\footnote{\textit{}
Mean, \ie, a middle course; the mean being the intermediate
part, or parts, between the treble and tenor. If
there were two means, as in the lute, the lower was called
the greater: the upper, the lesser mean.}
 to me.\\
For, if too coy, then I must court\\
For a kiss as well as any;\\
And if too free, I fear o’ th’ sport\\
I then may have too many,” \&c.
\end{altverse}
\end{dcverse}

\backskip{0.5}

In \textit{Wit Restored, in severall select Poems, not formerly publisht}, 1658, there are
two songs, \textit{The Burse of Reformation}, and \textit{The Answer}. The first commencing—

\backskip{0.5}

\begin{dcverse}\begin{altverse}
\vleftofline{“}We will go no more to the Old Exchange,\\
There’s no good ware at all;\\
Their bodkins, and their thimbles, too,\\
Went long since to Guildhall.\\
But we will go to the New Exchange,\\
Where all things are in fashion;\\
And we have it henceforth call’d\\
The Burse of Reformation.\\
Come, lads and lasses, what do you lack?\\
Here is ware of all prices;\\
Here’s long and short, here’s wide and straight;\\
Here are things of all sizes.
\end{altverse}
\end{dcverse}

\backskip{0.5}

\noindent and the Answer—

\backskip{0.5}

\begin{dcverse}\begin{altverse}
\vleftofline{“}We will go no more to the New Exchange,\\
Their credit’s like to fall,\\
Their money and their loyalty\\
Is gone to Goldsmiths’ Hall.\footnote{\textit{}
The place appointed for the reception of fines imposed
upon the Royalists; and for loans, etc., to the Puritanic
party.}\\
But we will keep our Old Exchange,\\
Where wealth is still in fashion,\\
Gold chaines and ruffes shalt beare the bell,\\
For all your reformation.\\
Look on our walls, and pillars too,\\
You’ll find us much the sounder:\\
Sir Thomas Gresham stands upright,\\
But Crook-back was your founder.”
\end{altverse}
\end{dcverse}

\backskip{0.5}

These have been reprinted in “Satirical Songs and Poems on Costume,” for the
Percy Society, by F. W. Fairholt, F.S.A. \pagebreak Another equally curious song for the
%318
manners and fashions of the day, is “The New Exchange,” in \textit{Merry Drollery
Complete}, 1670, p. 134; commencing—

\normalsize
\changefontsize{1.0\defaultfontsize}

\settowidth{\versewidth}{I'll go no more to the Old Exchange,}
\begin{dcverse}\begin{altverse}
\vleftofline{“}I'll go no more to the Old Exchange,\\
There's no good ware at all;\\
But I will go to the New Exchange,\\
Call’d Haberdashers’ Hall:\\
For there are choice of knacks and toys,\\
The fancy for to please;\\
For men and maids, for girls and boys,\\
And traps to catch the fleas.
\end{altverse}

\begin{altverse}
There you may buy a Holland smock,\\
That’s made without a gore,” \&c.
\end{altverse}
\end{dcverse}

\musicinfo{Lively.}{}

\includemusic{chappellV1168.pdf}

\pagebreak