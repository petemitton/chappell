%244
\changefontsize{\defaultfontsize}

\headingfour{REIGN OF JAMES I.}

The most distinguishing feature of chamber music, in the reign of James I.,
from that of his predecessor, was the rapidly-increasing cultivation of instrumental
music, especially of such as could be played in concert; and, coevally, the incipient
decline of the more learned, but less melodious descriptions of vocal music,
such as madrigals and motets.

During the greater part of the reign of Elizabeth, vocal music held an almost
undivided sway, and the practice of instrumental music, in private life, was
generally confined to solo performances, and to accompaniments for the voice.

The change of fashion, so far as I have been able to trace it, may be dated from
1599, in which year Morley printed a “First Booke of Consorte Lessons, made
by divers exquisite authors,” for six instruments to play together; and Anthony
Holborne a collection of “Pavans, Galliards, Almaines, and other short airs, both
grave and light, in five parts.” Morley’s publication consisted of favorite
subjects arranged for the Treble Lute, the Pandora,\footnote{\textit{}
There was a foreign instrument of the lute description,
with a great number of strings, called the Pandura,
but I imagine the English Pandora to be the same instrument
as the \textit{B}andora. In Thomas Robinson’s “School
of Musicke, the perfect fingering of the Lute, \textit{Pandora},
Orpharion and Viol da Gamba” the music is noted on six
lines, for an instrument of six strings like the Lute. In
1613, Drayton and Sir William Leighton severally enumerated
the instruments in use in England. Drayton
names the “Pandore” among instruments strung with
wire. Sir William Leighton speaks of the “Bandore,”
but neither of \textit{both}. In 1609, Philip Rosseter printed a
set of “Lessons for Consort,” like Morley’s, and for the
same six instruments, if the Bandora be not an exception. 
It was a large instrument of the lute kind,
with the same number of strings (but in all probability of
wire), and invented in 1562 by John Rose, citizen of
London, dwelling in Bridewell. It was much used in
this reign, especially with the Cittern, to which it formed
the appropriate base.}
 the Cittern, the (English)
Flute,\footnote{\textit{}
The English flute, described by Mersenne as the
\textit{Fistula dulcis, seu Anglica}, and by some as the \textit{Flute
à bec}, has eight holes for the fingers, and a mouth-piece
at the end like a flageolet. Of the eight holes, six are in
a row in front, one at the end for the little finger
(added afterwards), and one at the back for the
thumb. The tone is soft, rich, and melodious, but less
brilliant than the present flute. The ordinary length is
rather more than two feet. I had three or four of different
sizes, the largest exceeding four feet in length. The
base flute must have been still longer. The modern
flute is blown like the old fife; or as in the ancient
sculpture of The Piping Fawn.}
 and the Treble and Bass Viols. Holborne’s was for Viols, for Violins,\footnote{\textit{}
Under the name of “Violins“ the four different sizes
of the instrument are here comprehended. The word
Violoncello is of comparatively modem use. In Ben
Jonson’s \textit{Bartholomew Fair}, we find, “A set of these
Violins I would buy, too, for a delicate \textit{young noise}” (i.e
company of young musicians) “I have in the country;
they are every one a size less than another;—\textit{just like your
fiddles}—Act iii., sc. 1. Charles the Second’s famous
band of “four-and-twenty \textit{fiddlers}, all in a row,” consisted
of six violins, six counter-tenors, six tenors,
and six bases. The counter-tenor violin has become obsolete, 
because all the notes of its scale could be played
upon the violin or tenor.}
or for wind-instruments.

I know of no set of Madrigals printed during the reign of Elizabeth, which is
described on the title-page as “apt \textit{for Viols} and Voices”—it was fully understood
that they were for voices only;—but, from 1603, when James ascended the
throne, that mode of describing them became so general, that I have found but
two sets printed without it.\footnote{
The exceptions are Bateson’s \textit{First Set of Madrigals},
1604, and Pilkington’s \textit{First Set}, 1613, but the second sets
of both authors are described as “apt for viols and voices.”
So are Wilbye’s \textit{Second Set}, 1609; Michael Este’s \textit{Eight Sets},
of various dates, and the Madrigals of Orlando Gibbons,
Robert Jones, John Ward, Henry Lichfield, Walter Porter,
as well as Byrd's \textit{Psalmes, Songs, and Sonnets}, 1611: Peerson’s
\textit{Motets or Grave Chamber Music}, 1630; and many
lighter kinds of music. See Rimbault’s \textit{Bibliotheca
Madrigaliana}, 8vo.,~1847.}
\markright{reign of james i.}

\pagebreak
%245

Between 1603 and 1609, Dowland printed his “Lacrimæ, or Seven Teares
figured in seven passionate Pavans, with divers other Pavans, Galliards, and
Almands.” This work, to which there are so many allusions by contemporary
Dramatists, was in five parts, for the Lute, Viols, or Violins. In 1609, Rossiter
printed his “Lessons for Consort” for the same six instruments as Morley. In
1611, Morley’s work was reprinted,\footnote{\tinyrrr
Twelve volumes of Dr. Burney’s MS. extracts for his
\textit{History of Music} were formerly in my possession, and are
now in the British Museum. In one of them (Add. MSS.
11,587) are his extracts from Morley’s \textit{Consort Lessons}.
To “O mistress mine” (which I have printed at p. 209)
he appends the following note:—“If any melody or movement, 
besides the Hornpipe (\textit{a tune played by the Cornish
pipe, or pipe of Cornwall}), be truly native, it seems to be
this; \textit{which has the genuine drawl of our country clowns
and ballad singers in sorrowful ditties}, as the hornpipe has
the coarse and vulgar jollity of their mirth and merriment.” 
This criticism is a curiosity, and not less curious
is the judgment he passes on the Consort Lessons, after
scoring two out of the six parts (the Treble Viol and
Flute), and adding \textit{his own} base. Morley dedicates them
to the Lord Mayor and Aldermen, and Dr. Burney says,
“Master Morley, supposing that the harmony which was
to be heard through the clattering of knives, forks, spoons,
and plates, with the jingling of glasses, and clamorous
conversation of a city feast, need not be very accurate and
refined, was not very nice in setting parts to these tunes,
\textit{if we may judge of the rest by what passes between the viol
and flute},” \&c. The whole of this passage is transferred
to his \textit{History of Music} (iii. 102, Note D, 1789), except
the qualification, “\textit{if} we may judge,” \&c. It was not
advisable to tell the reader \textit{how} he had formed his opinion
of a work that had formerly passed through two editions.\linebreak
\indent Among Dr. Burney’s other criticisms of English Music
(for his History is essentially a critical one, and he has been
commonly quoted as an authority) are the following, which
are also directly connected with the subject of this book;—
In vol. ii., p. 553, he says, “It is related by Gio. Battista
Donado that the Turks have a limited number of tunes, to
which the poets of their country have continued to write
for ages; and the vocal music of our own country seems
long to have been \textit{equally circumscribed}; for, till the last
century, it seems as if the number of our secular and
popular melodies did not greatly exceed that of the Turks.”
In a note it is stated that the tunes of the Turks were in
all twenty-four; which were to depict melancholy, joy, or
fury; to be mellifluous or amorous. It may not, I hope,
be too presumptuous to say that Dr. Burney knew very
little of the subject. In vol. iii., 143, after criticising a
work printed in 1614, and saying, “The Violin was now
\textit{hardly known by the English, in shape or name}” (although
Ben Jonson describes the instrument, at that very time,
as commonly sold with roast pigs in Bartholomew Fair,
and violins had certainly been used on the English
stage from its infancy. See, for instance, the tragedy of
\textit{Gorboduc, or Ferrex and Porrex}, acted by the gentlemen
of the Inner Temple before Queen Elizaheth, in 1561);
he adds, “And the low state of our regal music
in the time of Henry VIII., 1530, may be gathered
from the accounts given in Hall and Hollinshed’a
Chronicles, of a Masque at Cardinal Wolsey’s palace,
Whitehall, where the King was entertained with ‘\textit{a concert
of drums and fifes}’” He then says, “But this was
soft music compared with that of his heroic daughter
Elizaheth, who, according to Hentzner, \textit{used to be regaled
during dinner} “with twelve trumpets, and two kettle-drums; 
which, \textit{together with fifes, cornets, and side-drums},
made the hall ring for half an hour together.” I find
nothing of the kind in Hall’s Chronicle (there is a short
notice of a similar Masque at Cardinal Wolsey’s, in the
tenth year of Henry VIII., fol. 65, b. 1548, but no drums
and fifes); and Hollinshed, who takes the account from
Cavendish’s \textit{Life of Wolsey}, is speaking not of a “concert”
at the Cardinal’s, but of the manner of receiving the King
and some of his nobles, who came by water to a Masque;
firstly by firing off “divers chambers” (short guns that
make a loud report) at his landing, and then conducting
him up into the chamber “with such a noise of drums
and fleutes, as seldom had been heard the like.” Cavendish
says, “with such a number of drums and fifes as
I have seldom seen together at one time in any masque”
(Singer’s edit., 8vo., 1825); and, describing the masques
generally, says, “Then was there all kind of music and
harmony set forth, with excellent voices both of men and
children.” Sagudino, the Venetian Ambassador, who
describes a banquet given by Henry VIII., in honor of
the Flemish envoys, on the 7th July, 1517, says, “during
the dinner there were boys on a stage in the centre of the
hall, some of whom sang, and others played the flute, rebeck, 
and virginals, making the sweetest melody.” As to
Queen Elizabeth, I quote Hentzner’s words from the copy
used by Dr. Burney: “During the time this guard, which
consists of the tallest and stoutest men that can he found
in all England, \textit{were bringing dinner}, twelve trumpets and
two kettle-drums made the hall ring for half an hour together.” 
(This was the loud music to give notice to prepare
for dinner, like the gong, or dinner-bell of the present
day, but the fifes, cornets, and side-drums, are of Dr.
Burney's invention.) “\textit{At the end} of all this ceremonial
a number of unmarried ladies appeared, who with particular
solemnity lifted the meat off the table, and conveyed
it into the Queen's inner and more private chamber, where,
after she had chosen for herself, the rest goes to the ladies
of the Court. The Queen dines and sups alone, with very
few attendants,” \&c. Hentzner also says, “Without the
city” (of London) “are some theatres where English
actors represent almost every day tragedies and comedies
to very numerous audiences: these are concluded with
\textit{excellent music}, variety of dances, and the excessive
applause of those that are present.” The original words
are “quas variis etiam saltationibus, suavissimâ adhibitâ
musicâ, magno cum populi applausu finire solent.” Again,
in summing up the character of the English in a few
lines, he says, “\textit{They excel in dancing and music, for they
are active and lively}, though of a thicker make than the
French.” Dr. Burney, throughout his History, writes in
a similarly disparaging strain about English music and
English musicians, for which I am unable to account.}
and about the same time Orlando Gibbons
published his \textit{Fantasies of three parts for Viols}. \footnote{\tinyrrr
For the republication of these, and many other works of
the sixteenth and seventeenth centuries, the world is indebted
to the Musical Antiquarian Society. The Madrigals
of Wilbye, Weelkes, Bennet, Bateson, and Gibbons;
the Ballets of Morley and Hilton; the four-part songs of
Dowland, and four Operas by Purcell; besides the first
music printed for the Virginals, the four-part Psalms by
Este, and various Anthems, \&c., \&c.
}
\pagebreak
%246
\changefontsize{0.99\defaultfontsize}

Viols had six strings, and the position of the fingers was marked on the fingerboard
by frets, as in guitars of the present day. The “Chest of Viols” consisted
of three, four, five, or six of different sizes; one for the treble, others for the mean,
the counter-tenor, the tenor, and perhaps two for the base. Old English musical
instruments were commonly made of three or four different sizes, so that a player
might take any of the four parts that were required to fill up the harmony. So
Violins, Lutes, Recorders, Flutes, Shawms, \&c., have been described by some
writers in a manner which (to those unacquainted with this peculiarity) has
appeared irreconcileable with other accounts. Shakespeare (in \textit{Hamlet}) speaks of
the Recorder as a little pipe, and says, in \textit{A Midsummer Night's Dream}, “he hath
played on his prologue like a \textit{child} on a recorder;” but in an engraving of the
instrument,\footnote{\textit{}
See “The Genteel Companion for the Recorder,” by
Humphrey Salter, 1683. Recorders and (English) Flutes
are to outward appearance the same, although Lord Bacon,
in his \textit{Natural History}, cent, iii., sec. 221, says the Recorder
hath a less bore, and a greater above and below.
The number of holes for the fingers is the same, and the
scale, the compass, and the manner of playing, the same.
Salter describes the \textit{recorder} from which the instrument
derives its name, as situate in the upper part of it, \ie,
between the hole below the mouth and the highest hole
for the finger. He says, “Of the kinds of music, vocal
has always had the preference in esteem, and in consequence, 
the Recorder, \textit{as approaching nearest to the
sweet delightfulness of the voice}, ought to have first place
in opinion, as we see by the universal use of it confirmed.”
The Hautboy is considered now to approach most nearly
to the human voice, and Mr. Ward, the military instrument
manufacturer, informs me that he has seen “old
English Flutes” with a hole bored through the side, in
the upper part of the instrument, the holes being covered
with a thin piece of skin, like gold-beater's skin. I suppose
this would give somewhat the effect of the quill or
reed in the Hautboy, and that these were Recorders. In
the proverbs at Leckingfield (quoted ante Note \textsuperscript{b}, p. 35),
the Recorder is described as “desiring” the mean part,
but manifold fingering and stops bringeth high (notes)
from its clear tones. This agrees with Salter’s book. He
tells us the high notes are produced by placing the thumb
\textit{half} over the hole at the back, and blowing a little stronger.
Recorders were used for teaching birds to pipe.}%end footnote
it reaches from the lip to the knee of the performer; and among
those left by Henry VIII. were Recorders of box, oak, and ivory, great and small,
two base recorders of walnut, and one \textit{great} base recorder. In the same catalogue
we find “flutes called Pilgrims’ staves,” which were probably six feet long.

Richard Braithwait, a writer of this reign, has “set down \textit{Some Rules for the
Government of the House of an Earl},” in which the Earl was to keep “five
musitions skillfull in that commendable sweete science,” and they were required
to teach the Earl’s children to sing, and to play upon the base-viol, the virginals,
the lute, and the bandora, or cittern. When he gave “great feasts,” the musicians
were to play, whilst the service was going to the table, upon Sackbuts,
Cornets, Shawms, and “such other instruments going with wind;”\footnote{\textit{}
In Middleton’s play, \textit{The Spanish Gipsy}, act ii., sc.~1,
is another allusion to the loud music while dinner was
being carried in, as well as a common pun upon sackbuts
and sack.

\textit{Alv}. “You must not look to \textit{have your dinner served in
with trumpets}”

\textit{Car}. “No, no, \textit{sack-buts} shall serve us.”}%end footnote
and upon
“Viols, Violins, or other \textit{broken}\footnote{\textit{}
“Broken Music,” as is evident from this and other
passages, means what we now term “a string band.”
Shakespeare plays with the term twice: firstly in\textit{ Troilus
and Cressida}, act iii., sc.~1, proving that the musicians then,
on the stage were performing on stringed instruments;
and secondly in \textit{Henry V}., act v., sc. 2, where he says to
the French Princess Katherine, “Come, your answer in
broken music; for thy voice is music and thy English
broken.” The term originated probably from harps, lutes,
and such other stringed instruments as were played without
a bow, not having the capability to sustain a long note
to its full duration of time.}
musicke,” during the repast.

The custom of retaining musicians in the service of families continued to the
time of the Protectorate. It was not confined to men of high rank (either in this
or the preceding century), but was general \pagebreak with the wealthy of all classes.
%247
So the old merchant in Shirley’s \textit{Love Tricks} (licensed 1625) says, “I made a
ditty, and my musician, \textit{that I keep in my house to teach my daughter}, hath set it
to a very good air, he tells me.” At least one wealthy merchant of the reign of
Henry VIII. retained as many musicians in his service as are prescribed for the
household of an Earl in James’ reign. Sir Thomas Kytson, citizen and mercer,
built Hengrave Hall, in Suffolk, between the years 1525 and 1538, and at the
death of his son (towards the close of Elizabeth’s reign) inventories of all the furniture
and effects were taken, including those of “the chamber where the musicyons
playe,” and of the “instruments and books of musicke” it contained.\footnote{\textit{}
\textit{History and Antiquities of Hengrave}, by John Gage,
F.S.A., fol., 1822. There are six viols in a chest; six
violins in a chest (in 1572 a treble violin cost 20s.); seven
recorders in a case; besides lutes, cornets, bandoras,
citterns, sackbuts, flutes, hautboys, a curtall (or short sort
of bassoon), a lysarden (base cornet, or serpent), a pair of
little virginals, a pair of double virginals, “a wind instrument
like a virginal,” and a pair of double organs.}
With the
exception of those for the lute, all the books of instrumental music were in sets of
five (for music in five or more parts), as well as those containing the vocal music,
described as “old.” The number of musicians was perhaps increased by his son,
for in the household expenses of the year 1574, we find, “seven cornets bought
for the musicians;” and the viols, violins, and recorders, in the inventory, are
(like those of Henry VIII) in chests or cases containing six or seven of each;
whilst much of the vocal music required six, and some seven and eight, voices
to sing it. In 1575 he lent the services of Robert Johnson, Mus. Bac., one of
his musicians, to the Earl of Leicester, on the occasion of the pageants at
Kenilworth.

Although we have no old English book written for the purpose of describing the
musical instruments in use in former days, like those of Mersenne and Kircher
for France and Germany, we find in our translations of the Bible and the
Metrical Psalms, the names of all in general use at the times those translations
were made, for the Hebrew instruments are all rendered by the names of such as
were then commonly known. We are so accustomed to picture David playing
on the harp, that we are not easily reconciled to the French version of the
Psalms, in which, in translations of the same passages, the violin is the instrument
assigned to him; and what we translate lute, they render bagpipe (\textit{musette}).
It is not my purpose to enter upon a detailed account of musical instruments,\footnote{\textit{}
Sir John Hawkins’ descriptions of musical instruments
are too much drawn from foreign sources. English
instruments often differed materially from those in use
abroad, as many do at the present day. I cannot agree
with his description of the Cittern (it has too many strings)
or of some others. The catalogue of musical instruments
left by Henry VIII. (Harl. MSS. 1419, fol. 200)
was unfortunately unknown to him, or it would have
explained many~difficulties.}
but the curious in such matters will find in Sir William Leighton’s “Teares or
Lamentations of a sorrowful soule,” a long catalogue of those known at this period.
It is contained in “A thanksgiving to God, with magnifying of his holy name \textit{upon
all instruments}.\footnote{\textit{}
A copy with music in the British Museum. Among
the instruments not mentioned by Drayton are the following, 
which I give in Sir William Leighton’s spelling:—
“Regalls, Simballs, Timbrell, Syrons, Crowdes, Claricoales, 
Dulsemers, Crouncorns, and Simfonie.” He mentions
the Drum after the Simphony, thereby apparently
drawing a distinction between them, but according to
Bartholomeus \textit{De Proprietatibus Rerum}, printed by
Wynken de Worde, the Simphony is “an instrument
of musyke... made of an holowe tree, closed in lether
in eyther syde, and mystrels betyth it wyth styckes.”
“Crouncorn” means, perhaps, Krumhorn or Cromhorn, a
crooked horn, in imitation of which we have a reed stop in
old organs called the Cromhorn, which is now corrupted
into Cremona. Henry VIII., at his death, left several
cases containing from four to seven Crumhorns in each.}
In the following lines from Song IV. in Drayton’s \textit{Poly-olbion},
printed in the same year (1613), many of those in common use are cited:—
\pagebreak
%248
\changefontsize{1.03\defaultfontsize}

\settowidth{\versewidth}{On which the practic’d hand with perfect’st fing’ring strikes,}
\begin{scverse}
\vleftofline{“}When now the British side scarce finished their song,\\
But th’ English, that repin’d to be delay’d so long.\\
All quickly at the hint, as with one free consent,\\
Struck up at once and sung, each to the instrument\\
(Of sundry sorts that were, as the musician likes).\\
On which the practic’d hand with perfect’st fing’ring strikes,\\
Whereby their height of skill might liveliest be exprest.\\
The trembling Lute some touch, some strain the Viol best.\\
In sets that there were seen, the music wondrous choice.\\
Some, likewise, there affect the Gamba with the voice,\\
To shew that England could variety afford.\\
Some that delight to touch the sterner wiry chord,\\
The Cithren, the Pandore, and the Theorbo strike:\\
The Gittern and the Kit the wand’ring fiddlers like.\\
So were there some again, in this their learned strife,\\
Loud instruments that lov’d, the Cornet\footnote{\textit{}
Among Henry the Eighth’s instruments were “Gitteron
Pipes of ivory or wood, called Cornets” The Cornet
described by Mersenne is of a bent shape, like the segment
of a large circle, gradually tapering from the bottom to
the mouth-piece, The cornet was of a loud sound, but
in skilful hands could be modulated so as to resemble the
tones of the human voice. In Ben Jonson’s Masque of
\textit{Neptune’s Triumph}, the instruments employed were five
Lutes and three Cornets. In several other Masques, Lutes
and Cornets were the only instruments used. At the
Restoration, Cornets supplied the deficiency of boys'
voices in Cathedral Service. The base Cornet was of a
more serpentine form, and from four to five feet in length;
but Mersenne says, the Serpent (contorted to render it
more easy of carriage, as its length was six feet one inch)
was the genuine base of that instrument.}
and the Fife,\\
The Hoboy, Sackbut deep, Recorder, and the Flute;\\
E’en from the shrillest Shawm unto the Cornamute.\\
Some blow the Bagpipe up, that plays the Country-Round;\\
The Tabor and the Pipe some take delight to sound.”\\
\attribution \textit{ The Sundry Musiques of England}.
\end{scverse}

In consequence of the almost universal cultivation of music in the sixteenth
century, and of the great employment and encouragement of musicians, so many
persons embraced music as a profession, that England overflowed with them.
Many travelled, and some were tempted by lucrative engagements to settle abroad.
Dowland, whose “touch upon the lute” was said to “ravish human sense,”
travelled through Italy, France, Germany, and the Netherlands, and about the
year 1600 became lutenist to the King of Denmark. On Dowland’s return to
England in 1607, Christian IV. begged of Lady Arabella Stuart (through the
Queen and Prince Henry) to allow Thomas Cutting, another famous lutenist, then
in her service, to replace him. Peter Phillips, better known on the continent
(where the greater part of his works were printed) as Pietro Philippi, accepted an
engagement as organist to the Arch-duke and Duchess of Austria, governors of
the Low Countries, and settled there. John Cooper spent much of his life in
Italy, and was called Coprario, or Cuperario, There were few, if any, Italian
composers or singers then in England,\footnote{\textit{}
Alfonso Ferabosco, the elder, was born, of Italian
parents, at Greenwich. As he was brought up and lived
in England, he can scarcely he considered as an Italian
musician. Nicholas Lanier was an Italian by birth, and
came to England as an engraver. He settled here, and
became an eminent musician.}
and the music of Italy was chiefly known
by the Madrigal, for the sacred music, as being for the service of the Mass, was
strictly~prohibited.
\pagebreak
%249

Anthony à Wood tells the following story of Dr. John Bull:—While
travelling \textit{incognito} through France and Germany for the recovery of his health,
he heard of a famous musician belonging to the Cathedral of St. Omer, and
applied to him to see his works. The musician having conducted Bull to a vestry
or music-school adjoining the Cathedral, shewed him a lesson or song of forty
parts, and then made a vaunting challenge to any person in the world to add one
more part, supposing it so complete that it was impossible to correct or add to it.
Dr.~Bull having requested to be locked up for two or three hours, speedily added
forty more parts, whereupon the musician declared that “he that added those
forty parts must either be the devil or Dr.~John Bull.”\footnote{\textit{}
Such exercises of learned ingenuity were common in
that day. Tallis wrote a Motet in forty parts, a copy of
which is now before me. It is for eight choirs, each of
five voices; the voices only coming together occasionally.
Dr. Burney discredits Dr. Bull’s feat as “impossible,”
but I am assured by Dr. Rimbault and by Mr. Macfarren,
who have seen this Motet, that whether the story be true
or not, it was quite possible. In all cases the anecdote
may be taken as a proof of the very high reputation Dr.
Bull enjoyed.}
In 1613, Bull (to
whom many offers of preferment at foreign courts had been previously made)
quitted England, and went to reside in the Netherlands, where he entered the
service of the Archduke.

The emigration of musicians was not confined to a few of the most eminent, for
we hear, indirectly, of many in the employ of foreign courts, whose movements
would not otherwise be recorded. Thus Taylor, the water-poet, who had just
described the Lutes, Viols, Bandoras, Recorders, Sackbuts, and Organs, in the
Chapel of the Graf (or Count) of Schomburg, says, “I was conducted an English
mile on my way by certain of my countrymen, my Lord’s musicians.”

We are indebted to foreign countries for the preservation of many of the works
of our best musicians of this age, as well as of our popular tunes. Dr. Bull’s
music is chiefly to be found in foreign manuscripts.\footnote{\textit{}
One foreign manuscript volume of Dr. Bull’s works
is now in my possession, and another in that of Mr.
Richard Clarke, who asserts that it contains “God save
the King,” of which more hereafter. The contents of
both are described in Ward’s \textit{Lives of the Gresham Professors}.
}
Dowland tells us that “some
part of his poor labours” had been printed in eight cities beyond the seas, viz.,
Paris, Antwerp, Cologne, Nuremburg, Frankfort, Leipzig, Amsterdam, and Hamburg. 
Much of the music printed in Holland in the seventeenth century was also
by English Composers. The right of printing music in England was a monopoly,
generally in the hands of one or two musicians,\footnote{\textit{}
It was held by Tallis and Byrd from 1575 to 1596, then
by Morley and his assignee. See Introduction to Rimbault’s \textit{Bibliothica Madrigaliana}, 8vo., 1847.}
and therefore very little, and
only such as they chose, could be printed. Hence the scarcity, as well as the
frequent imperfection, of these early works.

In London, each ward of the city had its musicians; there was also the Finsbury
Music, the Southwark and the Blackfriars Music, as well as the Waits of
London and Westminster. Morley thus alludes to the Waits, in the dedication
of his \textit{Consort Lessons} to the Lord Mayor and Aldermen: “As the ancient
custom of this most honourable and renowned city hath been ever to retain and
maintain excellent and expert musicians to adorn your honours’ favours, feasts,
and solemn meetings: to those, your Lordships’ Wayts, I recommend the same.”
A “Wayte,” in the time of Edward IV., had to \textit{pipe} watch four times in the
night, from Michaelmas to Shrovetide, \pagebreak and three in the summer, as well as to
%250
“make \textit{bon gayte}” at every chamber door; but Morley’s \textit{Consort Lessons}, as
before mentioned, required six instruments to play them,\footnote{\textit{}
A few specimens of the tunes of the waits of different
towns will be given under the reign of Charles~II.}
and the city bands are
commonly quoted as playing in six parts.\footnote{\textit{}
So in Heywood’s \textit{The English Traveller}, last scene of act i., 1633--
\settowidth{\versewidth}{\textit{Riot}. The best consort in the city for six parts.}
\begin{fnverse}
\vleftofline{“}\textit{Riot}. Fear not you shall have a full table.\\
\textit{Young L}. What, and music?\\
\textit{Riot}. The best consort in the city for six parts.\\
\textit{Young L}. We shall have songs, then?”\\
\end{fnverse}
}

After the act of the 39th year of Elizabeth, which rendered all “minstrels
wandering abroad” liable to punishment as “rogues, vagabonds, and sturdy
beggars,” all itinerant musicians were obliged to wear cloaks and badges, with the
arms of some nobleman, gentleman, or corporate body, to denote in whose service
they were engaged, being thereby excepted from the operation of the act. So in
\textit{Ram Alley}, 1611, Sir Oliver says—
\settowidth{\versewidth}{Lightly, lightly, and by my knighthood’s spurs}
\begin{scverse}
\vin\vin\vin\vin “Musicians, on!\\
Lightly, lightly, and by my knighthood’s spurs\\
This year you shall have \textit{my protection},\\
And yet not buy your livery coats yourselves.”
\end{scverse}

And as late as 1699, we find in \textit{Historia Histrionic}a, “It is not unlikely that the
lords in those days, and persons of eminent quality, had their several gangs of
players, as some \textit{have now} of fiddlers, to whom they give cloaks and badges.”

Musicians in the service of noblemen and gentlemen seem to have held a
prescriptive right to go and perform to the friends and acquaintances of their
masters, whenever they wanted money: such visits were received as compliments,
and the musicians were rewarded in proportion to the rank of their masters.
Innumerable instances of this will be found in early books of household expenditure; 
but, in James’ reign, musicians not actually in employ presumed so far
upon the license, that their intrusion into all companies, and at all times, became
a constant subject of rebuke. Ben Jonson’s Club, the Apollo, which met at the
Devil tavern, chiefly for conversation, was obliged to make a law that no fiddler
should enter, unless requested.\footnote{\textit{}
The rules of this club, in Latin, will be found in Ben
Jonson’s Works. the following translation is by one of
his adopted poetical sons:—
\settowidth{\versewidth}{Let none but guests, or clubbers, hither come;}
\begin{fnverse}
\vleftofline{“}Let none but guests, or clubbers, hither come;\\
Let dunces, fools, sad sordid men, keep home,\\
Let learned, civil, merry men b’invited,\\
And modest, too; nor be choice ladies slighted.\\
Let nothing in the treat offend the guests;\\
More for delight than cost, prepare the feasts.\\
The cook and purvey’r must our palates know,\\
And none contend who shall sit high or low.\\
Our waiters must quick-sighted be, and dumb,\\
And let the drawers quickly hear and come.\\
Let not our wine be mix'd, but brisk and neat,\\
Or else the drinkers may the vintners beat.\\
And let our only emulation be,\\
Not drinking much, but talking wittily.\\
Let it be voted lawful to stir up\\
Each other with a moderate chirping cup;\\
Let not our company be, or talk too much;\\
On serious things, or sacred, let's not touch\\
With sated heads and bellies. Neither may\\
Fiddlers unask’d obtrude themselves to play.\\
With laughing, leaping, dancing, jests and songs,\\
And whate’er else to grateful mirth belongs,\\
Let’s celebrate our feasts: and let us see\\
That all our jests without reflection be.\\
Insipid poems let no man rehearse,\\
Nor any be compelled to write a verse.\\
All noise of vain disputes must be forborn,\\
And let no lover in a corner mourn.\\
To fight and brawl, like Hectors, let none dare,\\
Glasses or windows break, or hangings tear.\\
Whoe’er shall publish what’s here done or said,\\
From our society must be banished.\\
Let none by drinking do or suffer harm,\\
And, while we stay, let us be always warm.”\\
\vin \textit{Poems and Songs by Alexander Brome}, 8vo., 1661
\end{fnverse}}
Nevertheless, they were generally welcome, and
generally well paid; more especially, at merry-makings where their services were
ever required. In those days a wedding was of a much gayer character than
now. There was first the hunt’s-up, \pagebreak or morning song, to awake the bride; then
%251
the music to conduct her to church (young maids and bachelors following, with
garlands in their hands); the same from church; the music at dinner; and
singing, dancing, and merry-making throughout the evening. For those who had
no talent to write a hunt’s-up, there were songs ready printed (like “The Bride’s
Good-morrow,” in the Roxburghe Collection), but the hunt’s-up was not confined
to weddings, it was a usual compliment to young ladies, especially upon their
birthdays. The custom seems now to be continued only with princesses, and on
the last birthday of the Princess Royal, the court newsman, at a loss how to
describe this old English custom, gave it the name of a “Matinale.”

As to music at weddings, see the following allusions:—

“Then was there a fair bride-cup of silver and gilt carried before her [the
bride], wherein was a goodly braunch of rosemarie gilded very faire, hung about
with silken ribbonds of all colours; next there was \textit{a noyse\footnote{\textit{}
A noise of musicians means a company of musicians.
It is an expression frequently occurring: “those terrible
\textit{noyses}, with threadbare cloakes, that live by red lattices
and ivy-bushes” [that is by ale-houses and taverns],
“having authority to thrust into any man’s room, only
speaking but this—‘Will you have any musicke?’”---
Dekker's \textit{Belman of London}, 1608.}
of musitians, that
played all the way before her}; after her came all the chiefest maydens of the
countrie, some bearing great bride-cakes, and some garlands of wheat finely
gilded, and so she past unto the church.”—Deloney’s \textit{Pleasant History of John
Winchcomb, in his younger years called Jacke of Newberie}.

“Come, come, we’ll to church presently. Prythee, Jarvis, \textit{whilst the musick
plays just upon the delicious close}, usher in the brides.”—Rowley’s \textit{A Match at
Midnight},~1633.

In Ben Jonson’s \textit{Tale of a Tub}, Turfe, the constable, “will let no music go afore
his child to church,” and says to his wife—
\settowidth{\versewidth}{Because you have entertained [musicians] all from Highgate,}
\begin{scverse}
\vleftofline{“}Because you have entertained [musicians] all from Highgate,\\
To shew your pomp, you’d have your daughters and maids\\
Dance o’er the fields like faies to church this frost.\\
I’ll have no \textit{rondels}, I, in the queen’s paths!\\
Let them scrape the gut at home, where they have fill’d it.”
\end{scverse}
And again, where Dame Turfe insists on having them to play at dinner, Clench
adds—
\begin{scverse}
\vleftofline{“}She is in the right, sir, vor your wedding dinner\\
Is starv’d without the music.”
\end{scverse}

Even at funerals musicians were in request: dirges were sung, and recorders the
instruments usually employed. It appears that the Blue-coat boys sang at City
Funerals;\footnote{\textit{}
See Brome’s \textit{City Wit}, act iii. sc. 1.}
being then taught music, as they \textit{should} be now. Music was not less
esteemed as a solace for grief, than as an excitement to merriment. Peacham says,
“the physicians will tell you that the exercise of music is a great lengthener of life,
by stirring and reviving the spirits, holding a secret sympathy with them; besides
it is an enemy to melancholy and dejection of mind; yea, a curer of some diseases.” 
(\textit{Compleat Gentleman}, 1622.) And Burton, “But I leave all declamatory
speeches in praise of divine music, I will confine myself to my proper subject:
besides that excellent power it hath to expel many other diseases, it is a sovereign
remedy against despair and melancholy, and will drive away the devil himself.” 
(\textit{Anatomy of Melancholy}.) So, in \textit{Henry IV}., Shakespeare says—
\pagebreak
%252
\changefontsize{1.05\defaultfontsize}

\settowidth{\versewidth}{Let there be no noise made, my gentle friends.}
\begin{scverse}
\vleftofline{“}Let there be no noise made, my gentle friends.\\
Unless some slow and favourable hand\\
Will whisper music to my weary spirit.”\\
\attribution \textit{Part II}, act iv., sc. 9.
\end{scverse}

Shakespeare purchased his house in Blackfriars, in 1612, from Henry Walker,
who is described in the deed as “Citizen and Minstrel, of London.” The price
paid was £140,\footnote{\textit{}
Shakespeare’s autograph, attached to the counterpart
of this deed, was sold by auction by Evans, on 24th May,
1841, for £155.}
which, considering the difference in the value of money, is equal
to, at least, £700 now. Of what class of “minstrel” Walker was, we know not,
but there were very few of any talent who had not the opportunity of saving money,
if so disposed. Even the itinerant fiddler who gave “a fytte of mirth for a groat,”
was well paid. The long ballads were usually divided into two or three “fyttes,”
and if he received a shilling per ballad, it would purchase as many of the necessaries
of life as five or six times that amount now. The groat was so generally his
remuneration, whether it were for singing or for playing dances, as to be
commonly called “fiddlers’ money,” and when the groat was no longer current,
the term was transferred to the sixpence.

It appears that in the reign of James, ballads were first collected into little
miscellanies, called Garlands, for we have none extant of earlier date. Thomas
Deloney and Richard Johnson (author of the still popular boys’ book, called \textit{The
Seven Champions of Christendom}) were the first who collected their scattered productions, 
and printed them in that form.

Deloney’s \textit{Garland of Good-will}, and Johnson’s \textit{Crown Garland of Golden Roses},
were two of the most popular of the class. They have been reprinted, with some
others, by the Percy Society, and the reader will find some account of the authors
prefixed to those works.

During the reign of Henry VIII., “the most pregnant wits” were employed
in compiling ballads.\footnote{\textit{}
See \textit{The Nature of the Four Elements}, written about
1517.}
Those in the possession of Captain Cox, described in
Laneham’s \textit{Letter from Kenilworth} (1575), as “all ancient,”\footnote{\textit{}
The list of Captain Cox’s ballads has been so often reprinted,
that I do not think it necessary to repeat it. The
reader will find it, with many others, in the introduction
to Ritson’s \textit{Ancient Songs}, as well as in more recently printed
books.}
could not well be
of later date than Henry’s reign; and at Henry’s death we find, with the list of
musical instruments left in the charge of Philip van Wilder, “sondrie bookes and
\textit{skrolles of songes and ballattes}.” In the reign of James, however, poets rarely
wrote in ballad metre; ballad writing had become quite a separate employment,
and (from the evidently great demand for ballads) I should suppose it to have
been a profitable one. In Shakespeare’s \textit{Henry IV}., when Falstaff threatens
Prince Henry and his companions, he says, “An I have not ballads made on you
all, and sung to filthy tunes, let a cup of sack be my poison;” and after Sir
John Colvile had surrendered, he thus addresses Prince John: “I beseech your
grace, let it he booked with the rest of this day’s deeds; or by the Lord, I will
have it in a particular ballad else, with mine own picture at the top of it, Colvile
kissing my foot.”

To conclude this introduction, \pagebreak I have subjoined a few quotations to shew the
%253
universality of ballads, as well as their influence upon the public mind; but limiting
myself to dramatists, to Shakespeare’s contemporaries, and to one passage
from each, author.

\medskip

\changefontsize{0.98\defaultfontsize}

In Ben Jonson’s \textit{Bartholomew Fair}, when Trash, the gingerbread-woman,
quarrels with Leatherhead, the hobby-horse seller, she threatens him—
\settowidth{\versewidth}{“I'll find a friend shall right me, and make a ballad of thee, and thy cattle all over.”}
\begin{scverse}“I'll find a friend shall right me, and make a ballad of thee, and thy cattle all over.”
\end{scverse}

In Heywood’s \textit{A Challenge for Beauty}, Valladaura says—
\settowidth{\versewidth}{But you must have some scurvy pamphlets and lewd}
\begin{scverse}“She has told all; I shall be balladed—\\
Sung up and down by minstrels.’’
\end{scverse}

In Fletcher’s \textit{Queen of Corinth}, Euphanes says—
\begin{scverse}\dots “and whate’er he be\\
Can with unthankfulness assoil me, let him\\
Dig out mine eyes, and sing my name in verse,\\
In ballad verse, at every drinking-house.’’
\end{scverse}

In Massinger’s \textit{Parliament of Love}, Chamont threatens Lamira—
\begin{scverse}\dots “I will have thee\\
Pictured as thou art now, and thy whole story\\
Sung to some villainous tune in a lewd ballad,\\
And make thee so notorious in the world,\\
That boys in the streets shall hoot at thee.”
\end{scverse}

In Chapman’s \textit{Monsieur d’Olive}, he says—
\begin{scverse}\vleftofline{“}I am afraid of nothing but I shall be balladed.”
\end{scverse}

In a play of Dekker’s (Dodsley, iii. 224) Matheo says—

“Sfoot, do you long to have base rogues, that maintain a Saint Anthony’s fire in
their noses by nothing but two-penny ale, make ballads of you?”

In Webster’s \textit{Devil's Law Case}, the officers are cautioned not to allow any to
take notes, because—
\begin{scverse}\vleftofline{“}\vleftofline{“}We cannot have a cause of any fame,\\
But you must have some scurvy pamphlets and lewd\\
Ballads engendered of it presently.”
\end{scverse}

In Ford’s \textit{Love's Sacrifice}, Fiormonda says—
\begin{scverse}\dots “Better, Duke, thou hadst been born a peasant;\\
Now boys will sing thy scandal in the streets,—\\
Tune ballads to thy infamy.”
\end{scverse}

In Marlow’s \textit{Edward II}., Mortimer says to the King—
\begin{scverse}\vleftofline{“}Libels are cast against thee in the street;\\
Ballads and rhymes made of thy overthrow.”
\end{scverse}

In Machin’s \textit{The Dumb Knigh}t—
\begin{scverse}\vleftofline{“}The slave will make base songs on my disgrace.”
\end{scverse}

In Middleton’s \textit{The Roaring Gir}l—
\begin{scverse}“O, if men’s secret youthful faults should judge ’em,\\
’Twould be the general’st execution\\
That e’er was seen in England!\\
There would be few left to sing the ballads,\\
There would be so much work.”
\end{scverse}

This is in allusion to the ballads on last dying speeches.
\pagebreak
%254

In the academic play of \textit{Lingua}, Phantastes says—

“O heavens! how am I troubled these latter times with poets—ballad-makers. Were
it not that I pity the printers, these sonnet-mongers should starve for conceits for all
Phantastes.”
\centerrule
The popular music of the time of Charles I. was so much like that of James,
as not to require separate notice. I have therefore included many ballads
of Charles’ reign in this division; but reserved those which relate to the troubles
and to the civil war, for the period of the Protectorate.
\centerrule
\musictitle{Upon A Summer’s-Day.}

In \textit{The Dancing Master}, from 1650 to 1665, and in \textit{Musick’s Delight on the
Cithren}, 1666, this is entitled “Upon a Summer’s-day;” and in later editions of
\textit{The Dancing Master}, viz., from 1670 to 1690, it is called “The Garland, or a
Summer’s-day.”

The song, “Upon a Summer’s-day” is in \textit{Merry Drollery Complete}, 1661,
p. 148. “The Garland” refers, in all probability, to a ballad in the Roxburghe
Collection, i. 22, or Pepysian, i. 300; which is reprinted in Evans’ \textit{Old Ballads},
iv. 345 (1810), beginning, “Upon a Summer’s \textit{time}.” It is more frequently
quoted by the last name in ballads. In the Pepys Collection, vol. i., is a
“Discourse between a Soldier and his Love;”--
\settowidth{\versewidth}{For land nor sea could make her stay behind.}
\begin{scverse}\vleftofline{“}Shewing that she did bear a faithful mind,\\
For land nor sea could make her stay behind.\\
\attribution To the tune of \textit{Upon a Summer time}”
\end{scverse}
It begins, “My dearest love, adieu.” And at p. 182 of the same volume,
“I smell a rat: to the tune of \textit{Upon a Summer tide}, or \textit{The Seminary Priest}.”
It begins, “I travell’d far to find.”

In the Roxburghe Collection, vol. i. 526, “The good fellow’s advice,” \&c., to
the tune of \textit{Upon a Summer time};” the burden of which is—
\settowidth{\versewidth}{Good fellows, great and small,}
\begin{scverse}\vleftofline{“}Good fellows, great and small,\\
Pray let me you advise\\
To have a care withall;\\
’Tis good to be merry and wise.”
\end{scverse}
And at p. 384 of the same volume, another by L.P., called “Seldom cleanely, or—
\settowidth{\versewidth}{Then lend your attention, while I do unfold}
\begin{scverse}A merry new ditty, wherein you may see\\
The trick of a huswife in every degree;\\
Then lend your attention, while I do unfold\\
As pleasant a story as ever was told.\\
\attribution To the tune of \textit{Upon a Summer's time}.”
\end{scverse}
It begins—
\settowidth{\versewidth}{I’ll tell you here a new conceit,}
\begin{scverse}
\vleftofline{“}Draw near, you country girls,\\
And listen unto me;\\
I’ll tell you here a new conceit,\\
Concerning huswifry.”
\end{scverse}

I have chosen a song which illustrates an old custom, instead of the original
words to this tune, because it is not \pagebreak desirable to reprint them. In \textit{Wit and}
%255
\textit{Mirth}, 1707, the following song, entitled \textit{The Queen of May}, is joined to an
indifferent composition:—

\musicinfo{Slowly and smoothly.}{}

\includemusic{chappellV1130.pdf}

\changefontsize{1.02\defaultfontsize}

\settowidth{\versewidth}{And homewards straight they went.}
\begin{dcverse}\begin{altverse}
From morning till the evening\\
Their controversy held,\\
And I, as judge, stood gazing on,\\
To crown her that excell’d.\\
At last when Phœbus’ steeds\\
Had drawn their wain away,\\
We found and crown’d a damsel\\
To be the Queen of May.
\end{altverse}

\begin{altverse}
Full well her nature from her\\
Face I did admire;\\
Her habit well became her,\\
Although in poor attire.\\
Her carriage was so good,\\
As did appear that day,\\
That she was justly chosen\\
To be the Queen of May.
\end{altverse}

\begin{altverse}
Then all the rest in sorrow,\\
And she in sweet content,\\
Gave over till the morrow,\\
And homewards straight they went.\\
But she, of all the rest,\\
Was hinder’d by the way,\\
For ev’ry youth that met her,\\
Must kiss the Queen of May.
\end{altverse}
\end{dcverse}


\musictitle{The Hunter In His Career.}

This is one of the songs alluded to in Walton’s \textit{Angler}. \textit{Piscator}. “I’ll
promise you I’ll sing a song that was lately made at my request by Mr. William
Basse, one that made the choice songs of ‘The Hunter in his career,’ and ‘Tom
of Bedlam,’ and many others of note.” The tune was translated from lute
tablature by Mr. G. F. Graham, of Edinburgh. It is taken from the “Straloch
Manuscript,” formerly in the possession of Mr. Chalmers, the date of which is
given in the original MS. from 1627 to 1629. It is also in the Skene MS., \&c.
A copy of the song is in the Pepys Collection, i. 452, entitled “Maister Basse
his careere, or The Hunting of the Hare. \pagebreak To a new court tune.” Printed for
%256
E[liz.] A[llde]. On the same sheet is “The Faulconer’s Hunting; to the tune
of \textit{Basse his careere}.” The words are also in \textit{Wit and Drollery, Jovial Poems},
1682, p. 64, and in \textit{Old Ballads}, second edition, 1738, iii. 196.

\musicinfo{With spirit.}{}

\smallskip

\includemusic{chappellV1131.pdf}

\settowidth{\versewidth}{Dapple-grey waxeth bay in his blood;}
\indentpattern{110110110110}
\begin{dcverse}\begin{patverse}
\vin Now bonny bay\\
In his foine waxeth gray;\\
Dapple-grey waxeth bay in his blood;\\
White-Lily stops\\
With the scent in her chaps,\\
And Black-Lady makes it good.\\
Poor silly Wat,\\
In this wretched state,\\
Forgets these delights for to hear\\
Nimbly she bounds\\
From the cry of the hounds,\\
And the music of their career.
\end{patverse}

\begin{patverse}
\vin Hills, with the heat\\
Of the gallopers’ sweat\\
Reviving their frozen tops,\\
{[And]} the dale’s purple flowers,\\
That droop from the showers\\
That down from the rowels drops.\\
Swains their repast,\\
And strangers their haste\\
Neglect, when the horns they do hear;\\
To see a fleet\\
Pack of hounds in a sheet,\\
And the hunter in his career.
\end{patverse}

\begin{patverse}
\vin Thus he careers,\\
Over heaths, over meres,\\
Over deeps, over downs, over clay;\\
Till he hath won\\
The noon from the morn,\\
And the evening from the day.\\
His sport then he ends,\\
And joyfully wends\\
Home again to his cottage, where\\
Frankly he feasts\\
Himself and his guests,\\
And carouses in his career.
\end{patverse}
\end{dcverse}
