\frontmatter

%===============================================================================

\thispagestyle{empty}
%\changefontsize{0.85\defaultfontsize}
\begin{center}

%\vspace*{\baselineskip}

\Huge POPULAR MUSIC
\titlespace

\small OF THE

\titlespace

\Huge OLDEN TIME:

\titlespace

\small A COLLECTION OF

\titlespace

\huge ANCIENT SONGS, BALLADS,

\titlespace

\small AND

\titlespace

\huge DANCE TUNES,

\titlespace

\small ILLUSTRATIVE OF THE

\titlespace

\huge NATIONAL MUSIC OF ENGLAND.

\titlespace

\normalsize WITH SHORT INTRODUCTIONS TO THE DIFFERENT REIGNS,

AND NOTICES OF THE AIRS FROM WRITERS OF THE

SIXTEENTH AND SEVENTEENTH CENTURIES.

\titlespace

\small ALSO

\titlespace

\large A SHORT ACCOUNT OF THE MINSTRELS.

\titlespace

\small BY

\titlespace

\Large W. CHAPPELL, F.S.A.

\titlespace

\normalsize THE WHOLE OF THE AIRS HARMONIZED BY G. A. MACFARREN.

\titlespace

\large VOL. I.

\titlespace

\footnotesize “Prout sunt illi Anglicani concentus suavissimi quidem, ac elegantes.”

\textit{Thesaurus Harmonicus} \textsc{Laurencini}, \textit{Romani}, 1603.

\titlespace

\small 1855 Edition LONDON:

\textsc{Cramer, Beale \& Chappell, 201 Regent Street}

\titlespace

2021 Edition MILTON KEYNES

\textsc{Catrah Press}

\normalsize
\vfill
\end{center}

\pagebreak

%===============================================================================
\thispagestyle{empty}
\noindent\includegraphics*[width=\textwidth]{images/Plate1X.pdf}
\pagebreak


%===============================================================================

\setmainfont{Baskerville10Pro}
\thispagestyle{empty}

\vspace*{12\baselineskip}

\begin{center}
{\large{CONTENTS OF VOL. I.}}
\end{center}

\vspace*{2\baselineskip}

\centerrule

\noindent\textsc{General Introduction.\hfill	page}
\bigskip

\tocstyle{Minstrelsy from the Saxon period to the reign of Edward I.}{1}

\tocstyle{Music of the middle ages, and Music in England to the end of the thirteenth century}{11}

\tocstyle{English Minstrelsy from 1270 to 1480, and the gradual extinction of the old Minstrels}{28}

\tocstyle{Introduction to the reigns of Henry VII., Henry VIII., Edward VI., and Queen Mary}{48}

\tocstyle{Songs and ballads of ditto	}{56 to 97}

\tocstyle{Introduction to the reign of Queen Elizabeth}{98}

\tocstyle{Songs and ballads of ditto}{110 to 243}

\tocstyle{Introduction to the reign of James I.}{244}

\tocstyle{Songs and ballads of the reigns of James I. and Charles I.}{254 to 384}

\vspace{2\baselineskip}
\centerrule
\vspace*{4\baselineskip}
\vfill
%===============================================================================

\intentionalemptypage

%%===============================================================================
%v
%\changefontsize{0.92\defaultfontsize}

%\vspace*{\baselineskip}
\chapter{INTRODUCTION.}

\hspace*{\parindent}\textsc{It} is now nearly twenty years since the publication of my collection of 
\textit{National English Airs} (the first of the kind), and about fourteen since the edition
was exhausted. In the interval, I found such numerous notices of music and
ballads in old English books, that nearly every volume supplied some fresh
illustration of my subject. If “Sternhold and Hopkins” was at hand--the
title-page told that the psalms were penned for the “laying apart of all
ungodly songs and ballads,” and the translation furnished a list of musical
instruments in use at the time it was made: if Myles Coverdale’s \textit{Ghostly
Psalms}--in the preface he alludes to the ballads of our courtiers, to the
whistling of our carters and ploughmen, and recommends young women at the
distaff and spinning-wheel to forsake their “\textit{hey, nonny, nonny---hey, trolly, lolly},
and such like fantasies;” thus shewing what were the usual burdens of their
songs. Even in the twelfth century, Abbot Ailred’s, or Ethelred’s, reprehension
of the singers gives so lively a picture of their airs and graces, as to resemble an
exaggerated description of opera-singing at the present day; and, if still receding
in point of date, in the life of St. Aldhelm, or Oldham, we find that, in order to
ingratiate himself with the lower orders, and induce them to listen to serious
subjects, he adopted the expedient of dressing himself like a minstrel, and first
sang to them their popular songs.

If something was to be gleaned from works of this order, how much more from
the comedies and other pictures of English life in the sixteenth and seventeenth
centuries! I resolved, therefore, to defer the re-publication for a few years, and
then found the increase of materials so great, that it became easier to re-write than
to make additions. Hence the change of title to the work.

Since my former publication, also, I have been favoured with access to the
ballads collected by Pepys, the well-known diarist; and the nearly equally celebrated
“Roxburghe Collection” (formed by Robert, Earl of Oxford, and increased
by subsequent possessors) has been added to the library of the British Museum.
These and other advantages, such as the permission to examine and make extracts
from the registers of the Stationers’ Company (through the liberality of the
governing body), have induced me to attempt a chronological arrangement of the
airs. Such an arrangement is necessarily imperfect, on account of the impossibility
of tracing the exact dates of tunes by unknown authors; but in every case
the reader has before him the evidence upon which the classification has been
founded.  

\pagebreak

%%===============================================================================
%vi
\renewcommand\versoheader{introduction.}
\renewcommand\rectoheader{introduction.}
%\changefontsize{0.83\defaultfontsize}

It might be supposed that the registers of the Company of Stationers would 
furnish a complete list of ballads and ballad-printers, but, having seen all the
entries from 1577 to 1799, I should say that not more than one out of every
hundred ballads was registered. The names of some of the printers are not to
be found in the registers.

It appears from an entry referring to the “white book” of the Company
(which is not now existing), that seven hundred and ninety-six ballads were left
in the council-chamber of the Company at the end of the year 1560, to be handed
over to the new Wardens, and at the same time but forty-four books.

Webbe, in a \textit{Discourse of English Poetrie}, printed in 1586, speaks of “the
\textit{un-countable} rabble of ryming ballet-makers and compylers of senseless sonnets,”
and adds, “there is not anie tune or stroke which may be sung or plaide on
instruments, which hath not some poetical ditties framed according to the numbers
thereof: some to \textit{Rogero}, some to \textit{Trenchmore}, to \textit{Downright Squire}, to galliardes,
to pavines, to jygges, to brawles, to all manner of tunes; which every fidler knows
better than myself, and therefore I will let them passe.” Here the class of music
is named with which old English ditties were usually coupled—dance and ballad
tunes. The great musicians of Elizabeth’s reign did not often compose airs of
the short and rhythmical character required for ballads. These were chiefly the
productions of older musicians, or of those of lower grade, and some of ordinary
fiddlers and pipers. The \textit{Frog Galliard} is the only instance I know of a popular
ballad-tune to be traced to a celebrated composer of the latter half of the sixteenth
century. The scholastic music then in vogue was of a wholly different character.
Point and counterpoint, fugue and the ingenious working of parts, were the great
objects of study, and rhythmical melody was but lightly esteemed.

In the reigns of James I. and Charles I., we find a few “new court tunes”
employed for ballads, but it was not until Charles II. ascended the throne that
composers of \textit{high} repute commenced, or re-commenced, the writing of simple
airs, and then but sparingly. Matthew Locke’s “\textit{The delights of the bottle}” is
perhaps the first song composed for the stage, that supplied a tune to ballads.

My former publication contained two hundred and forty-five airs; the present
number exceeds four hundred. Of these, two hundred are contained in the first
volume, which extends no further than the reign of Charles I. This portion of
the work may be considered as a \textit{collection}; but the number of airs extant of later
date is so much larger than of the earlier period, that the second volume can be
viewed only in the light of a \textit{selection}. To have made it upon the same scale as the
first would have occupied at least three volumes instead of one. My endeavour
has therefore been, to give as much variety of character as possible, but especially
to include those airs which were popular as ballad-tunes. Some of those contained
in the old collection have now given place to others of more general interest but
more than two hundred are retained. Every air has been re-harmonized, upon a
simple and consistent plan,—the introductions to the various reigns have been
added,—and nearly every line in the book has been rewritten.

I have been at some trouble to trace \pagebreak to its origin the assertion that the English 
%%===============================================================================
%vii
have no national music. It is extraordinary that such a report 
should have 
obtained credence, for England may safely challenge any nation not only to produce
as much, but also to give the same satisfactory proofs of antiquity. The
report seems to have gained ground from the unsatisfactory selection of English
airs in Dr.~Crotch’s \textit{Specimens of various Styles of Music}; but the national music
in that work was supplied by Malchair, a Spanish violin-player at Oxford, whose
authority Crotch therein quotes. It is perhaps not generally known that at the
time of the publication Dr.~Crotch was but nineteen years of age. No collection
of English airs had at that time been made to guide Malchair, and he followed
the dictum of Dr.~Burney in such passages as the following:—

“It is related by Giovanni Battista Donado that the Turks have a limited
number of tunes, to which the poets of their country have continued to write for
ages; and the vocal music of our own country seems long to have been equally
circumscribed: for, till the last century, it seems as if the number of our secular
and popular melodies did not greatly exceed that of the Turks.” In a note, he
adds, that the tunes of the Turks were in all twenty-four, which were to depict
melancholy, joy, or fury,—to be mellifluous or amorous. (\textit{History},~ii.~553.)

Again, in Shakespeare’s \textit{Midsummer Night’s Dream}, when Bottom has been
turned into an ass, and says “I have a reasonable good ear in music; let me have
tongs and bones,” the stage direction is “Musick tongs, Rural Music.” Burney
inverts the stage direction, and adds “Poker and tongs, marrowbones and cleavers,
salt-box, hurdy-gurdy, \&c., are the old national instruments of our island.”
(iii.~335.)

Jean Jacques Rousseau published a letter on French music, which he summed
up by telling his countrymen that “their harmony was abominable; their airs
were not airs; their recitative was not recitative; that they had no music, and
could not have any.” (Rousseau, \textit{Ecrits sur la Musique}, Paris, edit.~1823,
p.~312.) Dr.~Burney seems to have improved upon this model, for Rousseau did
not resort to misquotation to prove his case, but Dr.~Burney’s History is one
continuous misrepresentation of English music and musicians, only rendered
plausible by misquotation of every kind.

The effect of the misquotation is that he has been believed; and passages as
absurd as the following have been copied by writers who have relied upon his~authority:—

\begin{quotation}“The low state of our regal music in the time of Henry VIII., 1530, may be
gathered from the accounts given in Hall’s and Hollinshed’s Chronicles, of a masque
at Cardinal Wolsey’s palace, Whitehall, where the King was entertained with
‘a concert of drums and fifes.’ But this was soft music compared with that of
his heroic daughter Elizabeth, who, according to Hentzner, used to be regaled
during dinner ‘with twelve trumpets and two kettle-drums; which, together with
fifes, cornets, and side-drums, made the hall ring for half an hour together.’”
(\textit{History}, iii. 143.)
\end{quotation}

There is nothing of the kind in the 
books Dr.~Burney pretends to quote. The 
account of the chroniclers is of \pagebreak
Henry the Eighth’s landing at Wolsey’s palace, 
%%===============================================================================
%%viii 
where by a preconcerted arrangement, “divers chambers” (short cannon that 
made a loud report) were let off, and he was conducted into the hall with “such
a noise of drums and flutes as seldom had been heard the like,” for the purpose
of \textit{surprising} the Cardinal and the masquers. Not a word of the music of the
masque.

As to Queen Elizabeth, Hentzner describes only the military music to give notice
in the palace that dinner was being carried in. Music then answered the purpose
of the dinner-bell. He says “the queen dines and sups alone.”

Burney carries his depreciation of English authors systematically throughout
his work. It might be supposed that he would have allowed an author of so early
a date as John Cotton, who flourished soon after Guido, to pass unchallenged, but
he first misrepresents, and then contradicts him. Burney tells us that Cotton
ascribes the invention of neumæ erroneously to Guido (ii.~144). Now Cotton
speaks of various modes of writing music by the musical signs called neumæ, and
attributes the last only to Guido. It is certain that Burney read no more of the
treatise than the heading of a chapter (\textit{Quid utilitatis afferant neumæ a Guidone
inventæ}), for he proves by a note upon neumæ, that he only half understood what
they were. To any reader of Cotton’s treatise, such misapprehension would have
been impossible. (See Gerbert’s \textit{Scriptores Ecclesiastici de Musicâ}, ii. 257.)

It is not always easy to prove that a writer reviewed works without reading
them, but I doubt if any musician can now be found who believes that Burney
had examined “all the works he could find” of Henry Lawes, with the “care
and candour” that he professes; while in the case of Morley’s \textit{Concert-Lessons},
it is certain that he passed his facetious judgment upon them after scoring only
a portion of two parts, the work being in six. This is proved by his own manuscript
(Addit. MSS. 11,587, Brit. Mus.), and there was no perfect copy of the
work extant at the time.

When Burney tells us that the Catch Club sang old compositions “better than
the authors \textit{intended}” (iii. 123),—that “our secular vocal music, during the first,
years of Elizabeth’s reign, seems to have been much inferior to that of the Church,”
and has no better proof of it than a book of songs \textit{composed by an amateur musician},
“Thomas Wythorne, Gent.,” in 1571 (iii.~119),—when he says that, in
the same reign, “the violin was hardly known to the English in shape or in
name!” (iii. 143),—and that Playford was the \textit{first} who published music in the
seventeenth century, yet commenced in 1653! (iii. 417 and~418),—he shews not
only a desire to underrate, but also a deficiency of knowledge, that must weaken
all confidence in him as an historian.

In his review of the music in Elizabeth’s reign, he tells us that “the art of
singing, further than was necessary to keep a performer in tune and time, must
have been unknown\ldots\  \textit{solo songs, anthem}s, and cantatas, being productions of
later times” (iii. 114). A more strange misconception could scarcely have been
penned. No songs to the lute? No ballads? If so, Miles Coverdale might have
spared himself the trouble of telling the courtier “not to rejoice in his ballads,” 
and Chaucer should have represented \pagebreak
at least three persons as serenading the
%%===============================================================================%
%%ix
carpenter's wife, and not  one. As to the art of singing, Dr.~Burney has himself 
quoted the description of John of Salisbury, written four hundred years before
Queen Elizabeth’s reign, and that is quite enough to refute the opinion above
expressed; but, if more be required, the reader will find it here in the long note
at p.~404

There was a proverb, of French origin, current both in Latin and English in
the fourteenth and fifteenth centuries, respecting the manner of singing by different
nations. The Latin version was “Galli cantant, Angli jubilant, Hispani
plangunt, Germani ulutant, Itali caprizant:” the English was “The French
sing,” or “The French pipe, the English carol [rejoice, or sing merrily], the
Spaniards wail, the Germans howl, the Italians caper.” (The allusion to the
Italians is rather as to their unsteady holding of notes than to their facility in
florid singing; \textit{caper} signifying “a goat.”) Burney, without any authority,
renders it “the English \textit{shout}” (iii. 182). Now, although we have no modern
English verb that is an exact translation of “jubilare,” the Italian “giubilare”
has precisely the same signification; and Pasqualigo, the Venetian ambassador
to Henry VIII., describing the singing of the English choristers in the King’s
chapel, says “their voices are really rather divine than human—non cantavano
ma jubilavano,” which can be understood only in a highly complimentary sense.

It is sufficient for my present purpose to say that Dr.~Burney’s \textit{History} is
written throughout in this strain. What with mistake, and what with misrepresentation, it can but mislead the reader as to English music or musicians; and
from the slight search I have made into his early Italian authorities, I doubt
whether even that portion is very reliable. The public has now forgotten
the contention between the rival histories of music of Hawkins and Burney, and
a few words should be placed upon record. Hawkins’s entire work was published
in 1776, and Burney’s first volume in the same year, his second in 1782, and his
third and fourth in 1789. Burney obtained a great reputation by his first volume,
which is upon the music of the ancients. In that he was assisted by the researches
of the Rev. Thomas Twining, the translator of Aristotle’s Poetics, who relinquished
his own projected, and partly-written history, in Burney’s favour.
Hawkins’s work is of great original research, and he is a far more reliable
authority for fact than Burney: still the history is by no means so well digested.
It is an analysis of book after book and life after life, fitted rather for supplying
materials to those who will dig them out, than to be read as a whole. Burney’s
is a very agreeably written book, but he made history pleasant by such lively
sallies as those I have quoted: he took his authorities at second hand, when the
originals were accessible; and copied especially from Hawkins, without acknowledgment,
and disguised the plagiarism by altering the language. Many of his appropriations
are to be traced by errors which it is impossible that two men reading
independently could commit. Burney had but one love,—the Italian school,—and
he thought the most minute particulars of the Italian opera of his day worthy
of being chronicled. The madrigal with him was a “many-headed monster” (iii. 385): 
French music was \pagebreak “displeasing to all ears but those of France,” and 
%%===============================================================================
%%x
Rousseau’s letter upon it “an excellent piece of musical criticism,” combining 
“good sense, taste, and reason” (iv. 615): he dismisses Sebastian Bach in half
a dozen lines; and, although he devotes much space to Handel’s operas, his
oratorios are often dismissed with the barest record of their existence by a line in
a note. \textit{Israel in Egypt}, \textit{Acis and Galatea}, \&c., are unnoticed.

The present collection will sufficiently prove that “the number of our secular
and popular melodies” was not quite as “circumscribed” as Dr.~Burney has
represented; but, indeed, he had a book in his library which alone gave a complete
refutation to his limited estimate. I have now before me one of the editions
of \textit{The Dancing Master}, a collection of Country Dances, published by Playford,
which was formerly in Burney’s possession. It contains more than two hundred
tunes, the names of which must convince an ordinary reader that at least a considerable
number among them are song and ballad tunes, while a musician will as
readily perceive many others to be of the same class, from the construction of
the melody. If a doubt should remain as to the character of the airs in collections
of this kind, further evidence is by no means wanting. Sir Thomas Elyot, writing
in 1531, and describing many ancient modes of dancing, says (in The \textit{Governour}),
“As for the special names [of the dances], they were taken as \textit{they be now}, either
of the names of the first inventour, or of the measure and number they do conteine,
or of the \textit{first words of the ditties} which the song comprehendeth, \textit{whereoff}
the daunce was made;” and, to approach nearer to the time of the publication in
question, Charles Butler, in 1636, speaks of “the infinite multitude of ballads
set to sundry pleasant and delightful tunes by cunning and witty composers, with
country dances \textit{fitted unto them}” See his \textit{Principles of~Musick}.

The eighteen editions of \textit{The Dancing Master} are of great assistance in the
chronological arrangement of our popular tunes from 1650 to 1728;\dcfootnote{ %a
The first edition of this collection is entitled “The
English Dancing Master: or Plaine and easie rules for
the dancing of Country Dances, with the tune to each
dance (104 pages of music). Printed by Thomas Harper,
and are to be sold by John Playford, at his shop in the
Inner Temple, neere the Church doore.” The date is 1651,
but it was entered at Stationers’ Hall on 7th Nov., 1650.
This edition is on larger paper than any of the subsequent.
The next is \textit{The Dancing Master},\ldots\  with the tune to
each dance, to he play’d on the treble Violin: the \textit{second
edition}, enlarged and corrected from many grosse errors
which were in the former edition.” This was “Printed
for John Playford,” in 1652 (112 pages of music). The
two next editions, those of 1657 and 1665, each contain
132 country dances, and are counted by Playford as one
edition. To both were added “the tunes of the most
usual French dances, and also other new and pleasant
English tunes for the treble Violin.” That of 1665 was
“Printed by W. G., and sold by J. Playford and Z. Watkins,
at their shop in the Temple.” It has 88 tunes for
the violin at the end. (The tunes for the violin
were afterwards printed separately as Apollo's Banquet,
and are not included in any other edition of The
Dancing Master.) The date of the fourth edition is
1670 (155 pages of music). Fifth edition, 1675, and 160
pages of music. (The contents of the sixth edition are
ascertained to be almost identical with the fifth, by the
new tunes added to the seventh being marked with *, but
I have not seen a copy. From advertisements in Playford’s
other publications, it appears to have been printed
in 1680.) The seventh edition bears date 1686 (208 pages),
but to this “an additional sheet,” containing 32 tunes,
was first added, then “a new additional sheet” of 12
pages,” and lastly “a new addition” of 6 more. The
eighth edition was “Printed by E. Jones for H. Playford,”
and great changes made in the airs. It has 220 pages,—date,
1690. The ninth edition, 196 pages,—date, 1695.
“The second part of the Dancing Master,“24 pages,—date,
1696. The tenth edition, 215 pages,—date, 1698;
also the second edition of the second part, ending on p.~48
(irregularly paged), 1698. The eleventh is the first edition
in the new tied note, 312 pages,—date, 1701. The twelfth
edition goes back to the old note, 354 pages,—date, 1703.
The later editions are well known, but the above are
scarce.
} %end footnote
for, although
some airs run through every edition, we may tell by the omission of others, when
they fell into desuetude, as well as the airs by which their places were supplied. 

\pagebreak

%===============================================================================


%%===============================================================================
%%xi
%\changefontsize{0.89\defaultfontsize}
Many of our ballad-tunes were not fitted for dancing, and therefore were not 
included in \textit{The Dancing Master}; but a considerable number of these is supplied
by the ballad-operas which were printed after the extraordinary success of \textit{The
Beggars' Opera} in 1728.

I might name many other books which have contributed their quota, especially
\textit{Wit and Mirth, or Pills to purge Melancholy}, with its numerous editions from 1699
to 1720,—but all are indicated in the work. I cannot, however, refrain from
some notice of the numerous foreign publications in which our national airs are
included. Sometimes they are in the form of country dances,—at others, as
songs, or as tunes for the lute. I have before me three sets of country dances
printed in Paris during the last century, and as one of these is the “5\textsuperscript{ême} Recueil
d’Anglaises telle qu’elles se dansent ché la Reine,” there must have been at least
four more of that series. Many of my readers may not know that the “Quadrille
de Contredanses” in which they join under the name of “a set of Quadrilles,”
is but our old “Square Country Dance” come back to us again. The
new designation commenced no longer ago than 1815,—just after the war.

Horace Walpole tells us in his letters, that our country dances were all the rage
in Italy at the time he wrote, and, as collections were printed at Manheim, Munich,
in various towns of the Netherlands, and even as far North as Denmark, it is
clear that they travelled over the greater part of Europe. The Danish collection
now before me consists of 296 pages, with a volume of nearly equal thickness to
describe the figures.

Some of the works printed in Holland during the seventeenth century, which
contain English airs, have materially assisted in the chronological arrangement.
Of these, Vallet’s \textit{Tablature de Luth, entitulé Le Secret des Muses}, was published
at Amsterdam iu 1615. \textit{Bellerophon, of Lust tot Wysheit}, in 1620, and other
editions at later dates. Valerius’s \textit{Nederlandtsche Gedenck-Clanck}, at Haerlem,
in 1626. Starter’s \textit{Friesche Lust-Hof}, and his \textit{Boertigheden}, in 1634, and other
editions without date. Camphuysen’s \textit{Stichtelycke Rymen}, 1647, 1652, and
without date. Pers’s \textit{Gesangh der Zeeden}, 1662, and without date. \textit{Urania},
1648, and without date.

It is only necessary to remark upon the chronological arrangement, that, in
order to ascertain what airs or ballads were popular in any particular reign, the
reader will have occasion to refer also to those which precede it. Without endless
repetition, it could not have been otherwise.

Facsimiles of a few of the manuscripts will be found in the following~pages.

I have now the pleasing duty of returning thanks to those who have assisted
me in this collection; and first to Edward F. Rimbault, LL.D., and Mr. G. A.
Macfarren. Dr.~Rimbault has been the largest contributor to my work, and a
contributor in every form. To him I am indebted for pointing out many airs
which would have escaped me, and for adding largely to my collection of notices
of others; for the loan of rare books; and for assisting throughout with his extensive
musical and bibliographical knowledge. To Mr. G. A. Macfarren for 
having volunteered to re-arrange the \pagebreak
airs which were to be taken from my former 
%%===============================================================================
%%xii
collection, as well as to harmonize the new upon a simple and consistent plan 
throughout. In my former work, some had too much harmony, and others even
too little, or such as was not in accordance with the spirit of the words. The
musician will best understand the amount of thought required to find characteristic
harmonies to melodies of irregular construction, and how much a simple air
will sometimes gain by being well fitted.

To the Right Hon. the Earl of Abergavenny I am indebted for the loan of
“Lady Nevell’s Virginal Book,” a manuscript collection of music for the virginals,
transcribed in 1591. To the late Lord Braybrooke I owe the means of
access to Pepys’s collection of ballads, which was indispensable for the due
prosecution of the work.

To Mr. J. Payne Collier, F.S.A., I am indebted for the loan of a valuable
manuscript of poetry, transcribed in the reign of James I., containing much of
still earlier date; and for free access to his collection of ballads and of rare books:
to Mr. George Daniel, of Canonbury, for copies of several Elizabethan ballads,
which are to be found only in his unique collection; and to Mr. David Laing,
F.S.A. Scot., for the loan of several rare books.

To Sir Frederick Madden, K.H., Keeper of the Manuscripts in the British
Museum, I am indebted for much information about manuscripts, readily given,
and with such uniform courtesy, that it becomes an especial pleasure to
acknowledge it.

\medskip
\hfill W.C.\hspace*{4em}
\medskip

3, \textit{Harley Place (N. W.),}

\qquad \textit{or} 201, \textit{Regent Street. (W.)} 

\pagebreak
%


%%===============================================================================
\renewcommand\versoheader{explanation of the facsimiles.}
\renewcommand\rectoheader{explanation of the facsimiles.}
%\changefontsize{0.88\defaultfontsize}
%xiii

\chapter{EXPLANATION OF THE FACSIMILES.}

\vfill

\quad Plate 1 (facing the title-page).—“\textsc{Sumer is icumen in},” from one of the Harleian 
Manuscripts in the British Museum, No. 978. It is literally a “six men’s song,”
such as is alluded to in the burlesque romance of The \textit{Turnament of Tottenham},
and, being of the middle of the thirteenth century, is perhaps the greatest musical
curiosity extant. The directions for singing it are in Latin: “Hanc rotam cantare
possunt quatuor socii. A paucioribus autem quam a tribus aut saltem duobus non
debet dici, preter eos qui dicunt pedem. Canitur autem sic. Tacentibus ceteris, unus
inchoat cum hiis qui tenent pedem. Et cum venerit ad primam notam post crucem,
inchoat alius, et sic de ceteris. Singuli vero repausent ad pausaciones scriptas, et non
alibi, spacio unius longæ notæ.” [Four companions can sing this Round. It should
not, however, be sung by less than three, or at least two, besides those who sing the
burden. It is to be sung thus:—One begins with those who sing the burden, the
others remaining silent; but when he arrives at the first note after the cross, another
begins. The rest follow' in the same order. Each singer must pause at the written
pauses for the time of one long note, but not elsewhere.] The directions for singing
the “Pes,” or Burden, are, to the first voice, “Hoc repetit unus quociens opus est,
faciens pausacionem in fine” [One voice repeats this as often as necessary, pausing at
the end]; and, to the second, “Hoc dicit alius, pausans in medio, et non in fine, sed
immediate repetens principium.” [Another sings this, pausing in the middle, and
not at the end, but immediately re-commencing.]

The melody of this Round is printed in modern notation at p.~24, and in the pages
which precede it (21 to 24) the reader will find some account of the manuscript from
which it is taken. It only remains to add that the composition is in what was called
“perfect time,” and therefore every long note must be treated as dotted, unless it is
immediately followed by a short note (here of diamond shape) to fill the time of the
dot. The music is on six lines, and if the lowest line were taken away, the remaining
would be the five now employed in part-music where the C clef is used on the third
line for a counter-tenor voice.

The composition will be seen in score in Hawkins’s and Burney’s Histories of
Music. The Round has been recently sung in public, and gave so much satisfaction,
even to modern hearers, that a repetition was demanded. It is published in a detached
form for four voices.

\bigskip
Plate 2.—“\textsc{Ah, the syghes that come fro’ my heart},” from a manu\-script of
the time of Henry VIII., in the British Museum (MSS. Reg., Append., 58). For
the melody in modern notation, see p.~57. 
%xiii
\pagebreak
%%===============================================================================
%
%%xiv
 %\changefontsize{0.85\defaultfontsize}


In transcribing old music without bars, it is necessary to know that the ends of 
phrases and of lines of poetry are commonly expressed by notes of longer duration
than their relative value. Much of the music in Stafford Smith's \textit{Musica Antiqua} is
wrongly barred, and the rhythm destroyed by the non-observance of this rule. As
one of many instances, see “Tell me, dearest, what is love,” taken from a manuscript
of James the First’s time (\textit{Mus. Antiq}., i.~55). By carrying half the semibreve at
the end of the second bar into the third, he begins the second line of poetry (“\,'Tis
a lightning from above”) on the half-bar instead of at the commencement, and thus
falsifies the accent of that line and of all that follows. The antiquarian way would have
been, either to print the semibreve within the bar, or, which is far better, a minim with
a pause over it. In modernizing the notation, even the pause is unnecessary. Webbe
also bars incorrectly in the \textit{Convito Armonico}. For instance, in “We be three poor
mariners,” the tune is right the first time, but at the recurrence (on “Shall we go
dance the Round, the Round, the Round?”) he commences on the half-bar, because
he has given too much time to the word “ease” in the bar immediately preceding.

\bigskip
Plate 3.—“\textsc{Green Sleeves},” a tune mentioned by Shakespeare, from “Will\-iam
Ballet’s Lute Book,” described in note \textsuperscript{b} at p.~86. This is the version I have printed
at p.~230, but an exact translation of the copy will be found in my “National English
Airs,” i. 118. It is only necessary to remark that, in lute-music of the sixteenth
century, bars are placed rather to guide the eye than to divide the tune equally. The
time marked over the lines is the only sure guide for modern barring.

\bigskip
Plate 4.—“\textsc{Sellenger’s Round},” from a manuscript in the Fitzwilliam Museum,
at Cambridge, commonly known as “Queen Elizabeth’s Virginal Book.” See also
p.~71.

Dr.~Burney speaks of this manuscript first as “going under the name of Queen
Elizabeth’s Virginal Book,” and afterwards quotes it as if it had really been so.
I am surprised that he should not have discovered the error, considering that he had it
long enough in his possession to extract one of the pieces, and to give a full description
of the contents, (iii. 86, et seq.) It is now so generally known by that name,
that, for brevity’s sake, I have employed it throughout the work. Nevertheless, it
can never have been the property of Queen Elizabeth. It is written throughout in
one handwriting, and in that writing are dates of 1603, 1605, and 1612.

It is a small-sized folio volume, in red morocco binding of the time of James~I.,
elaborately tooled and ornamented with fleurs de lis, \&c., gilt edges, and the pages
are numbered to 419, of which 418 are written.

The manuscript was purchased at the sale of Dr.~Pepusch’s collection, in 1762, by
R. Bremner, the music-publisher, at the price of ten guineas, and by him given to
Lord Fitzwilliam.

Ward gives an account of Dr.~Bull’s pieces included in this virginal book, in his
\textit{Lives of the Gresham Professors}, fol., 1740, p.~203, but does not say a word of the
volume having belonged to Queen Elizabeth. We first hear of it in Dr.~Pepusch’s
possession, and, as he purchased many of his manuscripts in Holland (especially those
including Dr.~Bull’s compositions), it is by no means improbable that this English
manuscript may also have been obtained there. I am led to the conjecture by finding 
the only composer’s name invariably abbreviated is that of “Tregian.” At the commencement \pagebreak 
of Verstegan’s \textit{Restitution of decayed Intelligence}, Antwerp,~1605, is a  
%xiv
%%===============================================================================
%
%%xv
“sonnet concerning this work,” signed “Fr. Tregian,” shewing the connection of
the family with Holland, and in the virginal book one piece (No.~105, p.~196) has
only three letters of the author’s name, “Fre.” No. 60, p.~111, is “Treg. Ground;”
No. 80, p.~152, is “Pavana dolorosa, Treg.;” but No.~213, p.~315, is “Pavana
Chromatica, Mrs. Katherin Tregian’s Paven, by William Tisdall.” In the margin of
p.~312, is written, in a later hand, “R. Rysd silas.”

English music was so much in request in Holland in the early part of the seventeenth
century, that this collection of two hundred and ninety-six pieces of virginal
music may, not improbably, have been made for, or by, an English resident there,
and possibly designed as a present.

\medskip
Plate 5.—“\textsc{The Hunt’s up},” from \textit{Musick's Delight on the Cithren}, 1666, and
“\textsc{Parth\-enia},” from a flageolet book, printed in 1682.

These are only given as specimens of musical notation. The curious will find exact
translations in \textit{National English Airs}, i. 118.
\normalsize

\vfill
{\hspace*{\fill}\rule{4em}{0.4pt}\raisebox{-3.15pt}{\rotatebox{45}{\rule{5pt}{5pt}}}\rule{4em}{0.4pt}\hspace*{\fill}}
\vfill
%xv
\pagebreak
%===============================================================================

 
\thispagestyle{empty}
\noindent\includegraphics*[width=\textwidth]{images/Plate2X.pdf}
\pagebreak
%===============================================================================
 
\thispagestyle{empty}
\noindent\includegraphics*[width=\textwidth]{images/Plate3X.pdf}
\pagebreak
%%===============================================================================
 
\thispagestyle{empty}
\noindent\includegraphics*[width=\textwidth]{images/Plate4X.pdf}
\pagebreak
%%===============================================================================
 
\thispagestyle{empty}
\noindent\includegraphics*[width=\textwidth]{images/Plate5X.pdf}
\pagebreak

\intentionalemptypage 

%===============================================================================
\renewcommand\versoheader{list of subscribers.}
\renewcommand\rectoheader{list of subscribers.}
%\changefontsize{0.85\defaultfontsize}

\begin{Spacing}{1.15}
%xvii
\chapter{LIST OF SUBSCRIBERS.}
\thispagestyle{empty}
\setcounter{page}{17}

\setlength{\columnseprule}{0.4pt}
\begin{multicols}{2}\small
\raggedright
\subscriber{Ashe}, J. W. L., Esq., Exeter.

\subscriber{Allen}, G. F., Esq., Organist of St.~George’s, Wolverhampton.

\subscriber{Allen}, G. B., Esq., Armagh.

\subscriber{Alston}, The Rev. E. C., Dennington
Rectory.

\subscriber{Amott}, J., Esq., Organist of the Cathedral,
Gloucester.

\subscriber{Atkins}, R. A., Esq., Organist of the
Cathedral, St. Asaph’s.

\subscriber{Arcedeckne, Andrew}, Esq.

\subscriber{Ashburnham}, The Right Hon. the
Countess of.

\subscriber{Ashpitel, Arthur}, F.S.A.
\bigskip

\subscriber{Bartholomew}, Mrs. \textsc{Mounsey}.

\subscriber{Bacon}, G. P., Esq., Lewes.

\subscriber{Bagnold}, Mrs.

\subscriber{Barnett}, Mr. \textsc{John Francis}.

\subscriber{Barlow}, Mr. W., Manchester.

\subscriber{Beevor}, C., Esq.

\subscriber{Benedict, Jules}, Esq.

\subscriber{Best}, T. W., Esq., Organist of St.~George’s
Hall, Liverpool.

\subscriber{Binfield}, Miss, Reading. 2 \textit{copies}.

\subscriber{Blake, Thomas}, Esq.

\subscriber{Blockley, John}, Esq.

\subscriber{Bloxsome, Charles}, Esq., Sheffield.

\subscriber{Bishop, John}, Esq., Cheltenham.

\subscriber{Barnett, John}, Esq., Cheltenham.

\subscriber{Blumenthal}, J., Esq.

\subscriber{Black}, Dr.

\subscriber{Bowers}, Miss, Clifton.

\subscriber{Bosanquet}, Mrs. \textsc{Godfrey}

\subscriber{Bourne}, Dr., Ashted House, Birmingham.

\subscriber{Braine}, W. R., Esq., Kensington.
%\columnbreak

\subscriber{Broadwood, Henry} F., Esq.

\subscriber{Broadwood, Walter}, Esq.

\subscriber{Bros, Thomas}, Esq.

\subscriber{Brown, Graham}, Esq.

\subscriber{Brown}, T. F., Esq., Jersey.

\subscriber{Bruce}, Lord \textsc{Charles}.

\subscriber{Bucher}, M., Edinburgh.

\subscriber{Bussell}, H., Esq., Dublin.

\subscriber{Buck}, Dr., Norwich.

\subscriber{Bunnett}, E., Esq., Norwich.

\subscriber{Burchett}, J. R., Esq.

\subscriber{Benson}, G., Esq.

\subscriber{Birkenhead}, Miss.

\subscriber{Begrez}, J., Esq.

\subscriber{Butterworth}, Mr., Sheffield.
\bigskip

\subscriber{Cawdor}, The Right Hon. the Earl of.

\subscriber{Calthorp, Thomas}, D., Esq.

\subscriber{Cawood, Martin}, Esq., Leeds.

\subscriber{Callcott}, W. H., Esq., Kensington.

\subscriber{Chaplin}, Captain, Hastings.

\subscriber{Chater, George}, Esq.

\subscriber{Chisholm}, H. W., Esq.

\subscriber{Close, Thomas}, Esq., Nottingham.

\subscriber{Clarke}, Mr., Preston.

\subscriber{Cocks, Robert Lincoln}, Esq.

\subscriber{Cohen, Albert}, Esq.

\subscriber{Cooper}, G. \textsc{Armytage}, Esq., Lecturer on
Music.

\subscriber{Compton, Henry}, Esq., Melbourne, Australia.

\subscriber{Copeland}, W. R., Esq., Liverpool.

\subscriber{Cox}, Captain C. J., Fordwich House,
Kent.

\subscriber{Crotch}, Rev. R. W., Uphill House, near
Weston-super-Mare.
\end{multicols}
%xvii
\pagebreak
%%===============================================================================
%%xviii
 

%xviii

\begin{multicols}{2}\small
\setlength{\parindent}{0em}
\setlength{\parskip}{0em}
\raggedright

\subscriber{Crossley}, G. J., Esq., Bowden, near Manchester.

\subscriber{Cummings}, — Esq.

\subscriber{Curtis}, W., Esq., Nottingham.
\bigskip

\subscriber{Davis}, Miss.

\subscriber{Davidson}, W., Esq., Aberdeen.

\subscriber{Davenport, John}, Esq.

\subscriber{De Capell Broke}, Lady.

\subscriber{De la Rue, William}, Esq. 2 \textit{copies}.

\subscriber{Delavante} and Co., Manchester.

\subscriber{De Monti}, Mr. J. H., Glasgow. 2 \textit{copies}.

\subscriber{De Vos}, Polydore, Esq., Jersey.

\subscriber{Dixon, Robert} W., Esq., Seaton Carew.

\subscriber{Done}, W., Esq., Organist of the Cathedral,
Worcester.

\subscriber{Doneraile}, The Right Hon. Viscountess.

\subscriber{Duck}, Mr. W., Bath.

\subscriber{Duerdin, John}, Esq. 2 \textit{copies}.

\subscriber{Dunraven}, The Right Hon. the Earl of.

\subscriber{Duggan, Joseph}, Esq.

\subscriber{Dyce}, The Rev. \textit{Alexander}.
\bigskip

\subscriber{Eborall}, Miss, Lichfield.

\subscriber{Edgar}, Mrs. \textsc{George}.

\subscriber{Ella, John}, Esq. 2 \textit{copies}.

\subscriber{Engel, Carl}, Esq.

\subscriber{Euing, William}, F.S.A., Scot., Glasgow.
\bigskip

\subscriber{Favarger, Réné}, Esq.

\subscriber{Fagg}, Mr., Hull.

\subscriber{Farren, William}, Esq.

\subscriber{Farrer}, T. H., Esq.

\subscriber{Fellowes}, Lady.

\subscriber{Ferrari}, Signor \textsc{Adolfo}.

\subscriber{Fitzwilliam}, Mrs. \textsc{Edward}.

\subscriber{Foster}, John, Esq., Gent, of Her Majesty’s
Chapels Royal.

\subscriber{Fowler, Charles}, Esq., Torquay.
\bigskip

\subscriber{Grattan}, H. W., Esq.

\subscriber{Giraud, Frank}, Esq., Faversham.

\subscriber{Gibbs, Walter} S., Esq., Bath.

\subscriber{Green, John}, Esq.

\subscriber{Green}, Miss, Ashby de la Zouche.

\subscriber{Goss}, The Rev. \textsc{John}, Hereford.

\subscriber{Goss, John}, Esq., Organist and Composer
to Her Majesty’s Chapels Royal.

\subscriber{Godfrey, Charles}, Esq.
\bigskip

\subscriber{Hutton, Maxwell}, Esq., Dublin.

\subscriber{Harrison}, J., Esq., Deal.

\subscriber{Hacking, Richard}, Esq., Mus. Bac., Bury,
Lancashire,

\subscriber{Harrison}, W., Esq., F.G.S., F.Z.S., Galligreaves
House.

\subscriber{Hale}, Mr. C., Cheltenham.

\subscriber{Hawkins}, The Rev. W. B.

\subscriber{Hayes}, Mrs., St. Catharine’s.

\subscriber{Harris}, Mr., Northampton.

\subscriber{Harvey}, The Rev. G. H., Wimbledon.

\subscriber{Hammond}, Mr.

\subscriber{Hayward, Henry}, Esq., Wolverhampton.

\subscriber{Hallé, Charles}, Esq.

\subscriber{Hatton}, J. L., Esq.

\subscriber{Hendrie}, Robert, Esq.

\subscriber{Hennell}, G. R., Esq.

\subscriber{Hemery}, Mr.

\subscriber{Hills, William}, Esq.

\subscriber{Higham}, F., Esq., Wolverhampton.

\subscriber{Hime}, Mr. B., Manchester.

\subscriber{Holl, William}, Esq.

\subscriber{Holl, Henry}, Esq.

\subscriber{Holl, Francis}, Esq.

\subscriber{Hopkins}, E. J., Esq., Organist of the
Temple Church.\looseness=-1

\subscriber{Holmes}, W. H., Esq.

\subscriber{Hopkins}, J. L., Mus. Doc., Cambridge.

\subscriber{Howe}, H., Esq., Isleworth.

\subscriber{Hopkinson}, Messrs., Leeds.

\subscriber{Horne, F. Lennox}, Esq.

\subscriber{Holland}, W., Esq.

\subscriber{Hullah}, J. P., Esq.

\subscriber{Hutton}, The Rev. \textsc{Hugh}.
\smallskip

\subscriber{Isaac}, B. R., Esq., Liverpool.
\smallskip

\subscriber{Jackson}, Mr., Bradford.

\subscriber{Jewson}, J. P., Esq., Stockton-on-Tees.
\bigskip

\subscriber{Knapton}, Miss, Ripon.

\subscriber{Kerslake}, Mr., Bristol. 2 \textit{copies}.

\subscriber{Kreeft}, S. C., Esq., Consul-General.
\end{multicols}
%xviii
\pagebreak
%%===============================================================================
%%xix

%xix

\begin{multicols}{2}\small
\setlength{\parindent}{0em}
\setlength{\parskip}{0em}
\raggedright
\raggedcolumns

\subscriber{Kendall}, J. F., Esq.

\subscriber{Kerry}, John, Esq.

\subscriber{Keene, Charles}, Esq.

\subscriber{Ketelle}, S. W., Esq., Newcastle.

\subscriber{Kirk}, J. M., Esq., Halifax.

\subscriber{King, H. Stavely}, Esq., M.D.

\subscriber{Knight}, Mr., Chichester.

\subscriber{King, Donald}, Esq.
\bigskip

\subscriber{Lablache}, Signor F.

\subscriber{Land, Edward}, Esq.

\subscriber{Lambert}, D., Esq., York.

\subscriber{Lambeth, Henry} A., Esq., Glasgow.

\subscriber{Lewis}, Miss, Putney.

\subscriber{Linwood}, Miss.

\subscriber{Lohr}, G. A., Esq., Leicester.

\subscriber{Lover, Samuel}, Esq.

\subscriber{Lovett}, Mr.

\subscriber{Lockey, Charles}, Esq.
\bigskip

\subscriber{Mason, Joseph Edward}, Esq.

\subscriber{Mason, Thomas}, jun., Esq., Newcastle-under-Lyme.

\subscriber{Marshall}, J. W., Esq., Richmond, Yorkshire.

\subscriber{Masson}, Miss.

\subscriber{Martineau, Philip,} Esq.

\subscriber{Mackenzie, J. Whitefoord}, Esq., Edinburgh.

\subscriber{Martin}, G. W., Esq.

\subscriber{Mackway}, Mr.

\subscriber{Mercer}, The Rev. W., M.A., Sheffield.

\subscriber{Mills}, R., jun., Esq.

\subscriber{Morant}, The Lady \textsc{Henrietta}.

\subscriber{Morant, John}, Esq., Brockenhurst House,
Lymington.

\subscriber{Monk, Edwin George}, Mus. Doc., Organist
of the Cathedral, York.

\subscriber{Monk}, W. H., Esq., Organist of King’s
College, London.

\subscriber{Morris, Val}., Esq.

\subscriber{Murby, Thomas}, Esq.

\subscriber{Mario}, Signor.

\subscriber{Maclure, Andrew}, Esq.

\subscriber{Morier}, G. J., Esq.

\subscriber{Moore}, The Hon. Mrs. \textsc{Montgomery},
Dublin.
\columnbreak

\subscriber{Newman}, The Rev. W. A., D.D., Wolverhampton.

\subscriber{Nichol}, — Esq.

\subscriber{Nicholls}, W. H., Esq.

\subscriber{Nicholson, Alfred}, Esq.

\subscriber{Norbury}, The Right Hon. the Earl of.

\subscriber{Norman}, Rev. R. W., St. Peter’s College,
near Abingdon.

\subscriber{Norwood, Jasper}, Esq., Preston.

\subscriber{Norman}, Mrs., The Rookery, Bromley.
\bigskip

\subscriber{Ollivier}, Mr. \textsc{Robert}.

\subscriber{Osborne}, G. A., Esq.

\subscriber{Ouseley}, The Rev. Sir \textsc{Frederick Arthur}
\subscriber{Gore}, Bart., Professor of Music in the
University of Oxford.

\subscriber{Ould, Edwin}, Esq.
\bigskip

\subscriber{Paget}, R., Esq.

\subscriber{Parker, John William}, Esq.

\subscriber{Paterson}, Mr. R. \textsc{Roy}, Edinburgh.

\subscriber{Pole, S. Chandos}, Esq.

\subscriber{Purdie}, Mr. \textsc{John}, Edinburgh.

\subscriber{Pye, Kellow}, Esq., Wimbledon.
\bigskip

\subscriber{Rawlings}, — Esq., Shrewsbury.

\subscriber{Ravey}, Mr.

\subscriber{Reed, T. German}, Esq.

\subscriber{Richards}, Mr. T.

\subscriber{Richardson}, — Esq., Swindon.

\subscriber{Romer, Frank}, Esq.

\subscriber{Robinson, Joseph}, Esq., Dublin.

\subscriber{Rowe}, Mrs., Plymouth.

\subscriber{Rhodes, Jeremiah}, Esq., Pontefract.

\subscriber{Robb}, Mrs.

\subscriber{Robeck}, The Baron de, Swordlestown.

\subscriber{Rudall, Rose, and Carte}, Messrs.
\bigskip

\subscriber{Stanford, John}, Esq., Dublin.

\subscriber{Salaman, Charles}, Esq.

\subscriber{Sandys}, W., Esq., F.S.A.

\subscriber{Salt, George}, M., Esq., Shrewsbury.

Sacred Harmonic Society (The)

\subscriber{Spark}, W., Esq., Organist, Leeds.

\subscriber{Smith, G. Townshend}, Esq., Organist of the
Cathedral, Hereford.
\end{multicols}
%xix
\pagebreak
%%===============================================================================
%%20
 
\begin{multicols}{2}\small
\setlength{\parindent}{0em}
\setlength{\parskip}{0em}
\raggedright

\subscriber{Spencer}, The Rev. \textsc{Peter}, Temple Ewell,
near Dover.

\subscriber{Simms}, E., Esq., Coventry.

\subscriber{Simms}, Mr. H., Bath.

\subscriber{Smart}, Sir \textsc{George}.

\subscriber{Smith, Montem}, Esq.

\subscriber{Smith, Albert}, Esq.

\subscriber{Smith, Samuel}, Esq., Bradford.

\subscriber{Smith, W. H}., Esq., Sheffield.

\subscriber{Smith}, Mr. \textsc{John}, Leeds.

\subscriber{Smith, George}, Esq.

\subscriber{Singleton}, Rev. R. C., Kingstown, Dublin.

Signet Library, (The) Edinburgh.

\subscriber{Silver, William}, Esq.

\subscriber{Swift, Joseph}, Esq.

\subscriber{Skelton}, G. J., Esq., Hull.

\subscriber{Shea, Alexander}, Esq.

\subscriber{Shelmerdine}, W., Esq., Nottingham.

Shoreham College.

\subscriber{Stanyon, John}, Esq., Leicester.

\subscriber{Stanley}, E. H., Esq.

\subscriber{Stephenson}, W. F., Esq., Organist, Bishopton.

\subscriber{Sloper, Lindsay}, Esq.

\subscriber{Sutton} and \textsc{Potter}, Messrs., Dover.

\subscriber{Stocqueler}, J. H., Esq.

\subscriber{Sykes, Luke}, R., Esq.

\subscriber{Spalding}, S., Esq.

\subscriber{Squire}, F., Esq.

\subscriber{Stokes}, Dr., Dublin.
\bigskip

\subscriber{Travers}, Miss M.T., Hartsbourne, Bushey,
Herts.

\subscriber{Thacker}, A. C., Esq., Peterborough.

\subscriber{Taylor, Hussey}, Esq.

\subscriber{Taylor, Bianchi}, Esq., Bath.

\subscriber{Thackeray}, W., Esq.

\subscriber{Tait}, W., Esq., Melrose.
%\columnbreak

\subscriber{Thomson, James}, Esq., Glasgow.

\subscriber{Thrupp, John}, Esq.

\subscriber{Turner}, Mr., Stockport. 3 \textit{copies}.

\subscriber{Turle}, Miss, Lyme Regis.

\subscriber{Travers}, Miss, Hillingdon.
\bigskip

\subscriber{Vernon}, The Right. Hon. Lord.

\subscriber{Vantini, Townsend}, Esq.
\bigskip

\subscriber{Ward}, The Right Hon. Lord. 2 \textit{copies}.

\subscriber{Walker, George}, Esq., Aberdeen.

\subscriber{Wallerstein}, F. G., Newcastle-on-Tyne.

\subscriber{Walsh}, E., Esq.

\subscriber{Warren}, E., Esq., Organist of Christ
Church, Dover.

\subscriber{Wallace, W. Vincent}, Esq.

\subscriber{Ward}, The Rev. \textsc{Arthur} R., St. John’s
College, Cambridge.

\subscriber{Warren}. Mr. \textsc{John}, Royston.

\subscriber{Weiss}, W. H., Esq.

\subscriber{Werner, Louis}, Esq.

\subscriber{Webster, Alfred}, Esq., Bath.

\subscriber{Williams}, T., Esq., Fairlight, Tunbridge
Wells.

\subscriber{Wighton}, Mr. A. J., Dundee.

\subscriber{Wilson}, M. C., Esq.

\subscriber{Winslow, Forbes}, Esq., M.D., D.C.L.

\subscriber{Winn}, W., Esq.

\subscriber{Winstanley}, J. B., Esq., Bramston House,
near Leicester.

\subscriber{Woolcombe}, The Rev. W. \textsc{Walker}, Manchester.

\subscriber{Wood}, Messrs. J. M., and Co., Glasgow.
2~\textit{copies}.

\subscriber{Wood}, Messrs., Edinburgh. 2~\textit{copies}.

\subscriber{Woodman}, W., Esq., Hobhill, Morpeth.

\subscriber{Wylde, Henry}, Esq., Mus. Doc.
\end{multicols}
\end{Spacing}
%xx
\pagebreak
\setlength{\columnseprule}{0pt}
%===============================================================================



\mainmatter



%===============================================================================
%
%001

\renewcommand\versoheader{english minstrelsy.}
\renewcommand\rectoheader{gleemen, scalds, bards.}


%\changefontsize{0.87\defaultfontsize}


%001
\chapter{ON ENGLISH MINSTRELSY,\par
SONGS AND BALLADS.}


\section*{CHAPTER I.}


\subsection*{Minstrelsy from the Saxon Period to the Reign of Edward I.}

\textsc{Music and Poetry} are, in every country, so closely connected, during the 
infancy of their cultivation, that it is scarcely possible to speak of the one without
the other. The industry and learning that have been devoted to the subject of
English Minstrelsy, and more especially in relation to its Poetry, by Percy,
Warton, and Ritson, have left an almost exhausted field to their successors.
But, while endeavouring to combine in a compressed form the various curious
and interesting notices that have been collected by their researches, or which
the labours of more recent writers have placed within my reach, I hope I may
not prove altogether unsuccessful in my endeavour to throw a few additional rays
of light upon the subject, when contemplated, chiefly, in a musical point of view.

“The Minstrels,” says Percy, “were the successors of the ancient Bards, who
under different names were admired and revered, from the earliest ages, among
the people of Gaul, Britain, Ireland, and the North; and indeed by almost all
the first inhabitants of Europe, whether of Celtic or Gothic race; but by none
more than by our own Teutonic ancestors, particularly by all the Danish tribes.
Among these, they were distinguished by the name of \textit{Scalds}, a word which
denotes ‘smoothers and polishers of language.’ The origin of their art was
attributed to Odin or Wodin, the father of their Gods; and the professors of it
were held in the highest estimation. Their skill was considered as something
divine; their persons were deemed sacred; their attendance was solicited by kings;
and they were everywhere loaded with honours and rewards\ldots\  As these
honours were paid to Poetry and Song, from the earliest times, in those countries
which our Anglo-Saxon ancestors inhabited before their removal into Britain, we
may reasonably conclude that they would not lay aside all their regard for men
of this sort, immediately on quitting their German forests. At least, so long as
they retained their ancient manners and opinions, they would still hold them in
high estimation. But as the Saxons, soon after their establishment in this
island, were converted to Christianity, in proportion as literature prevailed among 
them, this rude admiration would begin to \pagebreak
abate, and poetry would no longer be a 
%001
%%===============================================================================
%002 
peculiar profession. Thus the poet and the minstrel early with us became two 
persons. Poetry was cultivated by men of letters indiscriminately; and many of
the most popular rhymes were composed amidst the leisure and retirement of
monasteries. But the Minstrels continued a distinct order of men for many ages
after the Norman conquest; and got their livelihood by singing verses to the
harp, principally at the houses of the great. There they were still hospitably
and respectfully received, and retained many of the honours shown to their predecessors, 
the bards and scalds. And though, as their art declined, many of
them only recited the compositions of others, some of them still composed songs
themselves, and all of them could probably invent a few stanzas on occasion.
I have no doubt but most of the old heroic ballads\ldots\  were composed by this
order of men.”



The term Minstrel, however, comprehended eventually not mere\-ly those who
sang to the harp or other instrument, romances and ballads, but also such as
were distinguished by their skill in instrumental music only. Of this abundant
proof will be given in the following pages. Warton says, “As literature, the
certain attendant, as it is the parent, of true religion and civility, gained ground
among the Saxons, poetry no longer remained a separate science, and the profession
of bard seems gradually to have declined among them: I mean the bard
under those appropriated characteristics, and that peculiar appointment, which he
sustained among the Scandinavian pagans. Yet their natural love of verse and
music still so strongly predominated, that in the place of their old Scalders, a new
rank of poets arose, called \textsc{Gleemen}, or Harpers.\dcfootnote{ %a
\textsc{Gleemen}, or Harpers. Fabyan, speaking of Blagebride,
an ancient British king, famous for his skill in
poetry and music, calls him “a conynge musicyan, called
of the Britons god of \textit{Gleemen}.” The learned Percy says:
“This word \textit{glee} is derived from the Anglo-Saxon {\saxon ʒliʒʒ}
(gligg), \textit{musica}, music, minstrelsy (Somner). This is,
the common radix, whence arises such a variety of terms
and phrases relating to the minstrel art, as affords the
strongest internal proof that this profession was extremely
common and popular here before the Norman conquest.
\ldots\ The Anglo-Saxon harpers and gleemen were the
immediate successors and imitators of the Scandinavian
Scalds.” We have also the authority of Bede for the
practice of social and domestic singing to the harp, in
the Saxon language, upon this island, at the beginning of
the eighth century.
} 
These probably gave rise to
the order of English Minstrels, who flourished till the sixteenth century.”

Ritson, in his Dissertation on Romance and Minstrelsy (prefixed to his Collection
of Ancient English Metrical Romances), denies the resemblance between
the Scalds and the Minstrels, and attacks Percy with great acrimony for ascribing
with too great liberality, the composition of our ancient heroic songs
and metrical legends, to those by whom they were generally recited. Percy,
in the earlier editions of his Reliques of Ancient Poetry, said: “The Minstrels
seem to have been the genuine successors of the ancient Bards, who united the
arts of poetry and music, and sung verses to the harp, \textit{of their own composing},”
which he afterwards modified into “\textit{composed by themselves or others}.” With this
qualification there appears to be no essential difference between their systems, as
the following quotation from Ritson will show: “That the different professors of
minstrelsy were, in ancient times, distinguished by names appropriated to their
respective pursuits, cannot reasonably be disputed, though it may be difficult to
prove. The \textit{Trouveur}, \textit{Trouverre}, or \textit{Rymour},
 was he who composed \textit{romans}, \pagebreak
%002
%===============================================================================
%003 
\textit{contes}, \textit{fabliaux}, \textit{chansons} and \textit{lais}; and those who confined themselves to the 
composition of \textit{contes} and \textit{fabliaux} obtained the appellation of \textit{contours}, \textit{conteours}, or
\textit{fabliers}. The \textit{Menetrier}, \textit{Menestrel}, or \textit{Minstrel}, was he who accompanied his song
by a musical instrument, both the words and the melody being occasionally furnished
by himself, and occasionally by others.”


Le Grand says: “This profession which misery, libertinism, and the vaga\-bond
life of this sort of people, have much decried, required, however, a multiplicity of
attainments, and of talents, which one would, at this day, have some difficulty to
find reunited, and we have more reason to be astonished at them in those days of
ignorance; for besides all the songs, old and new,—besides the current anecdotes, 
the tales and fabliaux, which they piqued themselves on knowing,—besides
the romances of the time which it behoved them to know and to possess in part, they
could declaim, sing, compose music, play on several instruments, and accompany
them. Frequently even were they authors, and made themselves the pieces
they uttered.”—\textit{Ritson’s Dissertation}, p. clxiii.

The spirit of chivalry which pervades the early metrical romances could not
have been imparted to this country by the Romans. As Warton observes,
“There is no peculiarity which more strongly discriminates the manners of the
Greeks and Romans from those of modern times, than that small degree of attention
and respect with which those nations treated the fair sex, and the inconsiderable
share which they were permitted to take in conversation, and the general
commerce of life. For the truth of this observation, we need only appeal to the
classic writings: from which it appears that their women were devoted to a state
of seclusion and obscurity. One is surprised that barbarians should be greater
masters of complaisance than the most polished people that ever existed. No
sooner was the Roman empire overthrown, and the Goths had overpowered
Europe, than we find the female character assuming an unusual importance and
authority, and distinguished with new privileges, in all the European governments
established by the northern conquerors. Even amidst the confusions of
savage war, and among the almost incredible enormities committed by the Goths
at their invasion of the empire, they forbore to offer any violence to the women.”

That the people of England have in all ages delighted in secular or social
music, can be proved by numerous testimonies. The Scalds and Minstrels were
held in great repute for many ages, and it is but fair to infer that the reverence
shown to them arose from the love and esteem in which their art was held. The
Romans, on their first invasion of this island, found three orders of priesthood
established here from a period long anterior. The first and most influential were
the Druids; the second the Bards, whose business it was to celebrate the praises
of their heroes in \textit{verses and songs}, which they sang to their harps; and the third
were the Eubates, or those who applied themselves to the study of philosophy.



The Northern annals abound with pompous accounts of the honors conferred
on music by princes who were themselves proficients in the art; for music had
become a regal accomplishment, as we find by all the ancient metrical romances 
and heroic narrations,—and to sing to the \pagebreak
harp was the necessary accomplishment 
%===============================================================================
%004
of a perfect prince, or a complete hero. The harp seems to have been, for many 
ages, the favorite instrument of the inhabitants of this island, whether under
British, Saxon, Danish, or Norman kings. Even so early as the first invasion of
Britain by the Saxons, we have an incident which records the use of it, and which
shows that the Minstrel or Bard was well-known among this people; and that their
princes themselves could, upon occasion, assume that character. Colgrin, son of that
Ella who was elected king or leader of the Saxons, in the room of Hengist, was
shut up in York, and closely besieged by Arthur and his Britons. Baldulph,
brother of Colgrin, wanted to gain access to him, and to apprize him of a reinforcement
which was coming from Germany. He had no other way to accomplish
his design, but by assuming the character of a Minstrel. He therefore
shaved his head and beard, and dressing himself in the habit of that profession,
took his harp in his hand. In this disguise he walked up and down the trenches
without suspicion, playing all the while upon his instrument as a harper. By
little and little he advanced near to the walls of the city, and making himself
known to the sentinels, was in the night drawn up by a rope. Rapin places the
incident here related under the year 495. The story of King Alfred entering
and exploring the Danish camp under the disguise of a Minstrel, is related by
Ingulph, Henry of Huntingdon, Speed, William of Malmesbury, and almost all
the best modern historians; but we are also told that before he was twelve years
old, he could repeat a variety of Saxon songs, which he had learned from hearing
them sung by others, who had themselves, perhaps, only acquired them by tradition,
and that his genius was first roused by this species of erudition.

Bale asserts that Alfred’s knowledge of music was perfect; and it is evident
that he was an enthusiast in the art, from his paraphrase of Bede’s description of
the sacred poet Cædmon’s embarrassment when the harp was presented to him in
turn, that he might sing to it, “be hearpan singan;” Bede’s words are simply
“Surgebat a mediâ cænâ, et egressus, ad suum domum repedabat:” but Alfred
adds, that he arose for \textit{shame} (aras he for sceome); implying that it was a disgrace
to be found ignorant of the art.

We may also judge of the Anglo-Saxon love for song, from the course pursued
by St. Aldhelme, Abbot of Malmesbury, who died in 709. Being desirous of
instructing his then semi-barbarous countrymen, he was in the daily habit of
taking his station on the bridges and high roads, as if a Gleeman or Minstrel
by profession, and of enticing them to listen to him, by intermixing more serious
subjects with minstrel ballads.—\textit{Gul. Malms. de Pontificalibus. Lib}. 5. And
in the ancient life of St. Dunstan (whose feat of taking the evil one by the nose
with a pair-of red-hot pincers, was so favorite a sign for inns and taverns) he is
said, not only to have learnt “the vain songs of his nation,” but also “to have
constructed an organ with brass pipes, and filled with air from bellows.”
The Saint was a monk of Glastonbury, and born about 925.



That the harp was the common musical instrument of the Anglo-Saxons, may
also be inferred from the word itself, which is not derived from the British, or 
any other Celtic language, but of genuine \pagebreak
Gothic original, and current among 
%004
%===============================================================================
%005
every branch of that people, viz.:Ang. Sax. \textit{hearpe} and \textit{hearpe}; Iceland, \textit{harpa} 
and \textit{haurpa}; Dan. and Belg. \textit{harpe}; German, \textit{harpffe} and \textit{harpffa}; Gal. \textit{harpe};
Span, \textit{harpa}; Ital. \textit{arpa}. The Welsh, or Cambro-Britons, call their harp \textit{teylin},
a word for which no etymon is to be found in their language. In the Erse its
name is \textit{crwth}. That it was also the favorite musical instrument of the Britons
and other Northern nations in the middle ages, is evident from their laws,
and various passages in their history. By the laws of Wales (Leges Wallicæ), a
harp was one of the three things that were necessary to constitute a gentleman,
or a freeman; and none could pretend to that character who had not one of these
favorite instruments, or could not play upon it. To prevent slaves from pretending
to be gentlemen, it was expressly forbidden to teach, or to permit, them
to play upon the harp; and none but the king, the king’s musicians, and
gentlemen, were allowed to have harps in their possession. A gentleman’s harp
was not liable to be seized for debt; because the want of it would have degraded
him from his rank, and reduced him to that of a slave.

Alfred entered the Danish camp \ad 878; and about sixty years after, a
Danish king made use of the same disguise to explore the camp of our king
Athelstan. With his harp in his hand, and dressed like a minstrel, Aulaff, king
of the Danes, went among the Saxon tents; and taking his stand by the king’s
pavilion, began to play, and was immediately admitted. There he entertained
Athelstan and his lords with his singing and his music, and was at length dismissed
with an honorable reward, though his songs might have disclosed the fact
that he was a Dane. Athelstan was saved from the consequences of this stratagem
by a soldier, who had observed Aulaff bury the money which had been given him,
either from some scruple of honor or superstitious feeling. This occasioned
a discovery.

Now if the Saxons had not been accustomed to have Minstrels of their own,
Alfred’s assuming so new and unusual a character would have excited suspicions
among the Danes. On the other hand, if it had not been customary with the
Saxons to show favor and respect to the Danish Scalds, Aulaff would not have
ventured himself among them, especially on the eve of a battle. From the
uniform procedure of both these kings, we may fairly conclude that the same
mode of entertainment prevailed among both people, and that the Minstrel was
a privileged character with~each.

May it not be further said,—what a devotion to the art of music must have
existed in those rude times, when the vigilance of war was lulled into sleep and
false security, and the enmities of two detesting nations were forgotten for
awhile, in the enjoyment of sweet sounds!

That the Gleeman or Minstrel held a stated and continued office in the court
of our Anglo-Saxon kings, can be proved satisfactorily. We have but to turn to
the Doomsday Book, and find under the head: Glowecesterscire, fol.~162, col.~1.—“Berdic, 
Joculator Regis, habet iii villas,” \&c. That the word Joculator (at
this early period) meant Harper or Minstrel, is sufficiently evident from Geoffrey 
of Monmouth, of whom Dr.~Percy observes \pagebreak
very justly, “that whatever credit is 
%005
%===============================================================================
%006
due to him as a relator of
\textit{facts}, he is certainly as good authority as any for the 
signification of words.”



The musical instruments \textit{principally} in use among the Anglo-Saxons, were the
Harp, the Psaltry, the Fiðele, and a sort of Horn called in Saxon “Pip” or
Pipe. The Harp, however, was the national instrument. In the Anglo-Saxon
Poem of Beowulf it is repeatedly mentioned.

“There was the noise of the harp, the clear song of the poet.”\ldots\ “There
was song and sound altogether, before Healfdene’s Chieftains; the wood of joy
(harp) was touched, the song was often sung.”\ldots\  “The beast of war (warrior)
touched the joy of the harp, the wood of pleasure,” \&c.

The Fiðele (from which our words fiddler and fiddle are derived) was a sort of
viol, played on by a bow. The Psaltry, or Sawtrie, was strung with~wire.\dcfootnote{ %a
Representations of Anglo-Saxon harps and pipes will
be found in Harl. MSS. 603, which also contains a
psaltry, in shape like the lyre of Apollo, but with more
strings, and having a concave hack. It agrees with that
which Augustine describes as carried in the hand of the
player, which had a shell or concave piece of wood on it,
that caused the strings to resound, and is much more
elegant in shape than those in Sir John Hawkins’s History,
copied from Kircher’s Musurgia. A representation
of the Fithele will he found in the Cotton Collection,
Tiberius, c. vi., and in Strutt’s Sports and Pastimes.
Both the manuscripts cited are of the tenth century.
} %end footnote

The Normans were a colony from Norway and Denmark, where the Scalds
had arrived at high renown before Rollo’s expedition into France. Many
of those men no doubt accompanied him to the duchy of Normandy, and left
behind them successors in their art; so that when his descendant William
invaded this kingdom, \ad 1066, he and his followers were sure to favor the
establishment of the minstrel profession here, rather than suppress it; indeed,
we read that at the battle of Hastings, there was in William’s army a valiant
warrior, named Taillefer, distinguished no less for the minstrel arts, than for his
courage and intrepidity. This man, who performed the office of Herald-minstrel
(Menestrier huchier), advanced at the head of the army, and with a loud voice
animated his countrymen, singing a war-song of Roland, \ie, “Hrolfr or Rollo,”
says our Anglo-Saxon historian, Sharon Turner;—then rushing among the
thickest of the English, and valiantly fighting, lost his~life.

The success of his ancestor Rollo, was one of the topics of the speech in which
William addressed his army before the battle, to excite in them the emulation of
establishing themselves in England as he had done in Normandy. A “Chanson
de Roland” continued in favor with the French soldiers as late as the battle of
Poictiers, in the time of their king John, for, upon his reproaching one of them with
singing it at a time when there were no Rolands left, he was answered that
Rolands would still be found if they had a Charlemagne at their head. This was
in 1356.

Dr.~Burney conjectured that the song, “L’homme armée,” which was so popular
in the fifteenth century, was \textit{the} Chanson de Roland; but M. Bottée de Toulmon
has quoted the first four lines of “L’homme armée” from the Proportionales
Musices of John Tinctor, and proved it to be only a love-song. He has also
printed the tune, which he extracted from one of the many Masses in which it
was used as a subject to make Descant on.\dcfootnote{ %b
Annuaire Historique pour l’année, 1837. Publié par
la Sociéte de l’Histoire de France.
} %end footnote
%006
\pagebreak



%===============================================================================
\renewcommand\rectoheader{normans.---battle of hastings.}
%\changefontsize{0.87\defaultfontsize}



%007
Robert Wace, in the Roman de Rou, says that Taillefer sang with a loud voice 
(chanta à haute voix) the songs of \textit{Charlemagne}, Roland, \&c., and M. de
Toulmon considers the song of Roland to have been a Chanson de Geste, or
metrical romance; and that Taillefer merely \textit{declaimed} parts of such poems, holding
up those heroes as models to the assembled soldiers. The Chanson de Roland,
that was printed in Paris in 1837-8 (edited by M. Michel) from a copy in the
Bodleian Library, is a metrical romance in praise of the French hero, the \textit{Orlando
Innamorato}, and \textit{Furioso} of Boiardo, Berni, and Ariosto, but apparently of no such
antiquity,\dcfootnote{ %a
It contains, also, about 4,000 verses; and it seems still
more improbable that so lengthy a composition should
have been generally and popularly known. It is more
likely to have originated in the favor with which an earlier
song was received.
} %end footnote
and it seems improbable that \textit{he} should have been the subject of the
Norman minstrel's song. All metrical romances, however, were originally recited
or chanted with an accompaniment; and Dr.~Crotch has printed a tune in the
third edition of his Specimens of Various Styles of Music, vol. 1, p.~133, as the
“\textsc{Chanson Roland} sung by the Normans as they advanced to the battle of
Hastings, 1066,” which I give as a curiosity, but without vouching for its
authenticity.

\musictitle{Chanson Roland.}

\medskip\lilypondfile{lilypond/007-chanson-roland}


Dr.~Crotch does not name the source from which he obtained this air, nor
have I been successful in tracing it.\dcfootnote{ %b
The Chanson de Roland that has been printed recently, 
edited by Sir Henry Bishop, is a Composition by
the Marquis de Paulmy, taken from Burney's History of
Music, vol ii. p.~276, but Dr.~Burney does not give it as
an ancient song or tune. The tune, indeed, is not even
in imitation of antiquity.
} %end footnote
The story of Taillefer may, however, be
altogether apochryphal, as it is not mentioned by any \textit{contemporary} historian.



The English, according to Fordun, \pagebreak 
spent the night preceding the battle in 
%007
%%===============================================================================
%008
singing and drinking. “Illam noctem Angli totam in cantibus et potibus 
insomnem duxerunt.”—c.~13.

Ingulphus, a contemporary of William the Conqueror, speaks of the popular
ballads of the English in praise of their heroes; and William of Malmesbury,
in the twelfth century, mentions them also. Three parishes in Gloucestershire
were appropriated by William to the support of his minstrel; and although his
Norman followers would incline only to such of their own countrymen as excelled
in the art, and would listen to no other songs but those composed in their own
Norman-French, yet as the great mass of the original inhabitants were not extirpated,
these could only understand their own native Gleemen or Minstrels; and
accordingly, they fostered their compatriot Minstrels with a spirit of emulation
that served to maintain and encourage them and their productions for a considerable
period after the invasion. That they continued devoted to their AngloSaxon
tongue,\dcfootnote{ %a
“The dialect of our Alfred, of the ninth century, in his
Saxon translation of Boethius and Bede, is more clear
and intelligible than the vulgar language, \textit{equally ancient},
of any other country in Europe. For I am acquainted
with no other language, which, like our own, can mount
in a regular and intelligible series, from the dialect now in
use to the ninth century: that is, from pure English to
pure Saxon, such as was spoken and written by King
Alfred, unmixed with Latin, Welch, or Norman.”—
\textit{Burney's History of Music}, vol. ii. p.~209.
} %end footnote
notwithstanding the opposition of their tyrannical conquerors, is
sufficiently plain.

“Of this,” says Percy, “we have proof positive in the old metrical romance
of Horn-Child, which, although from the mention of Sarazens, \&c., must have
been written at least after the first crusade in 1096, yet, from its Anglo-Saxon
language, or idiom, can scarcely be dated later than within a century after the
Conquest. This, as appears from its very exordium, was intended to be sung to a
popular audience, whether it was composed by or for a Gleeman, or Minstrel. But
it carries all the internal marks of being the work of such a composer. It appears
of \textit{genuine English growth}; for, after a careful examination, I cannot discover any
allusion to French or Norman customs, manners, composition, or phraseology: no
quotation, ‘as the romance sayeth:’ not a name or local reference, which was
likely to occur to a French rimeur. The proper names are all of northern
extraction. Child Horn is the son of Allof (\ie, Olaf or Olave), king of Sudenne
(I suppose Sweden), by his queen Godylde, or Godylt. Athulf and Fykenyld are
the names of subjects. Eylmer, or Aylmere, is king of Westnesse (a part of
Ireland); Rymenyld is his daughter; as Erminyld is of another king, Thurstan;
whose sons are Athyld and Beryld. Athelbrus is steward of king Aylmer, \&c. \&c.
All these savour only of a northern origin, and the whole piece is exactly such a
performance as one would expect from a Gleeman or Minstrel of the north of
England, who had derived his art and his ideas from his Scaldic predecessors
there.”

Although Ritson disputed the English origin of this romance, Sir Frederick
Madden, in a note to the last edition of Warton’s English Poetry, has proved
Percy to be right, and that the French Romance, Dan Horn (on the same subject
as Child Horn), \textit{is} a translation from the English. In the Prologue to another
Romance, King Atla, it is expressly stated that the stories of Aelof (Allof),
Tristan, and others, had been translated into French from the English. 

%008
\pagebreak

%===============================================================================

%009
\renewcommand\rectoheader{william i. to richard i.}
%\changefontsize{0.84\defaultfontsize}

After the Conquest, the first notice we have relating to the Minstrels is the 
founding of the Priory and Hospital of St. Bartholo\-mew,\dcfootnote{ %a
Vide the \textit{Monasticon}, tom. ii. pp.~166-67, for a curious
history of this priory and its founder. Also \textit{Stowe's Survey}. 
In the \textit{Pleasaunt History of Thomas of Reading}, 4to.
1662, he is likewise mentioned. His monument, in good
preservation, may yet be seen in the parish church of
St. Bartholomew, in Smithfield, London.
} %end footnote 
in Smithfield, by
Royer, or Raherus, the King’s Minstrel, in the the third year of King Henry I.,
\ad 1102. Henry’s conduct to a luckless Norman minstrel who fell into his power,
tells how keenly the minstrel’s sarcasms were felt, as well as the ferocity of Henry’s
revenge. “Luke de Barre,” said the king, “has never done me homage, but he has
fought against me. He has composed facetiously indecent songs upon me; he has
sung them openly to my prejudice, and often raised the horse-laughs of my malignant
enemies against me.” Henry then ordered his eyes to be pulled out. The
wretched minstrel rushed from his tormentors, and dashed his brains against
the wall.\dcfootnote{ %b
Quoted from Ordericus Vitalis. Hist. Eccles. in Sharon
Turner's Hist. England.
} %end footnote


In the reign of King Henry II., Galfrid or Jeffrey, a harper, received in 1180
an annuity from the Abbey of Hide, near Winchester; and as every harper was
expected to sing,\dcfootnote{ %
So in Horn-Child, K. Allof orders his steward,
Althebrus to “teche him of harpe and song.” And
Chaucer, in his description of the Limitour or Mendicant
Friar, speaks of harping as inseparable from singing—“in
his harping, when that he had sung.” Also in 1481, see
Lord Howard's agreement with William Wastell, Harper
of London, to teach a boy named Colet “to harp and to sing.”
} %end footnote
we cannot doubt that this reward was bestowed for his music
and his songs, which, as Percy says, if they were for the solace of the monks there,
we may conclude would be in the English language. The more rigid monks,
however, both here and abroad, were greatly offended at the honours and rewards
lavished on Minstrels. John of Salisbury, who lived in this reign, thus declaims
against the extravagant favour shown to them: “For \textit{you} do not, like the fools of
this age, pour out rewards to Minstrels (Histriones et Mimos\dcfootnote{ %
Histrio, Mimus, Joculator, and Ministrallus, are all
nearly equivalent terms for Minstrels in Mediaeval Latin.
“Incepit \textit{more Histrionico}, fabulas dicere, et plerumque
cantare.” “Super quo \textit{Histriones cantabant}, sicut modo
cantatur de Rolando et Oliverio.” “Dat sex \textit{Mimis}
Domini Clynton, \textit{cantantibus}, \textit{citharisantibus}, ludentibus,” 
\&c. 4 s. Geoffrey of Monmouth uses \textit{Joculator} as
equivalent to \textit{Citharista}, in one place, and to \textit{Cantor} in
another. See Notes to Percy’s Essay.
} %end footnote
)and monsters of
that sort, for the ransom of your fame, and the enlargement of your name.”
—(\textit{Epist}. 247.)

“Minstrels and Poets abounded under Henry’s patronage: they spread the love
of poetry and literature among his barons and people, and the influence of the
royal taste soon became visible in the improved education of the great, in the
increasing number of the studious, and in the multiplicity of authors, who wrote
during his reign and the next.”—\textit{Sharon Turner’s Hist. Eng}.

In the reign of Richard I. (1189.) minstrelsy flourished with peculiar splendour.
His romantic temper, and moreover his own proficiency in the art, led him to be
not only the patron of chivalry, but also of those who celebrated its exploits.
Some of his poems are still extant. The romantic release of this king from the
castle of Durrenstein, on the Danube, by the stratagem and fidelity of his Minstrel
Blondel, is a story so well known, that it is needless to repeat it here.\dcfootnote{ %
The best authority for this story, which has frequently
been doubted, is the Chronique de  Rains, written in the 13th Century.—See 
\textit{Wright's Biograph,Brit., Anglo Norman  p.~325.}
%TODO lost word -- Period
} %end footnote



Another circumstance which proves how easily Minstrels could always gain
admittance even into enemies’ camps and prisons, occurred in this reign. The
young heiress of D’Evreux, Earl of Salisbury, \pagebreak
“was carried abroad, and secreted
%===============================================================================
%010 
by her French relations in Normandy. To discover the place of her concealment, 
a knight of the Talbot family spent two years in exploring that province, at first
under the disguise of a pilgrim; till having found where she was confined, in
order to gain admittance he assumed the dress and character of a harper, and
being a \textit{jocose} person, exceedingly skilled in ‘the Gests of the Ancients,’—so they
called the romances and stories which were the delight of that age,—he was gladly
received into the family, whence he took an opportunity to carry off the young
lady, whom he presented to the king; and he bestowed her on his natural brother,
William Longespee (son of fair Rosamond), who became, in \textit{her} right, Earl of~Salisbury.

In the reign of king John (\ad 1212) the English Minstrels did good service
to Ranulph, or Randal, Earl of Chester. He, being beseiged in his Castle of
Rothelan (or Rhuydland), sent for help to De Lacy, Constable of Chester, who,
“making use of the Minstrels of all sorts, then met at Chester fair, by the allurements
of their music, assembled such a vast number of people, who went forth
under the conduct of a gallant youth, named Dutton (his steward and son-in-law)
that he intimidated the Welsh, who supposed them to be a regular body of armed
and disciplined soldiers, so that they instantly raised the siege and retired.”

For this deed of service to Ranulph, both De Lacy and Dutton had, by
respective charters, patronage and authority over the Minstrels and others, who,
under the descendants of the latter, enjoyed certain privileges and protection for
many~ages.

Even so late as the reign of Elizabeth, when this profession had fallen into such
discredit that it was considered in law a nuisance, the Minstrels under the jurisdiction
of the family of Dutton are expressly excepted out of all acts of Parliament
made for their suppression; and have continued to be so excepted ever since.\dcfootnote{ %
See the statute of Eliz. anno. 39. cap.~iv. entitled an
Act for punishment of rogues, vagabonds, \&c.; also a renewal
of the same clauses in the last act on this subject,
passed in the reign of George~III. The ceremonies
attending the exercise of this jurisdiction are described
by Dugdale (Bar~i..~p.~101), and from him, by Percy.
} %end footnote

“We have innumerable particulars of the good cheer and great rewards given to
the Minstrels in many of the convents, which are collected by Warton and others.
But one instance, quoted from Wood’s Hist. Antiq. Ox., vol. i. p.~67, during the reign
of king Henry III. (sub. an. 1224), deserves particular mention. Two itinerant
priests, on the supposition of their being Minstrels, gained admittance. But the
cellarer, sacrist, and others of the brethren, who had hoped to have been entertained
by their diverting arts, \&c., when they found them to be only two indigent ecclesiastics,
and were consequently disappointed of their mirth, beat them, and turned
them out of the monastery.”

In the same reign (\ad 1252) we have mention of Master Richard, the king’s
Harper, to whom that monarch gave not only forty shillings and a pipe of wine,
but also a pipe of wine to Beatrice, his wife. Percy remarks, that the title of
Magister, or Master, given to this Minstrel, deserves notice, and shows his
respectable~situation.



“The learned and pious Grosteste, bishop \pagebreak 
of Lincoln who died in 1253, is said, 
%===============================================================================
%011
\renewcommand\rectoheader{king john to edward i.}
in some verses of Robert de Brunne, who flourished about the beginning of the 
next century, to have been very fond of the metre and music of the Minstrels.
The good prelate had written a poem in the Romanse language, called \textit{Manuel
Peche}, the translation of which into English, Robert de Brunne commenced in~1302, 
with a design, as he tells us himself, that it should be sung to the harp at
public entertainments.”

\DFNsingle

\settowidth{\versewidth}{Hys Harper’s chaumbre was fast ther}
\begin{dcverse} For lewde [unlearned] men I undertoke\\
In Englysshe tunge to make thys boke,\\
For many ben of swyche manere\\

That talys and rymys wyl blithly here,\\
Yn gamys and festys, and at the ale\\
Love men to listene trotevale. [triviality]
\end{dcverse}


The following anecdote concerning the love which his author, bishop
Gros\-teste, had for music, seems to merit a place here, though related in rude~rhymes.


\settowidth{\versewidth}{Hys Harper’s chaumbre was fast ther}

\begin{dcverse} I shall yow telle as I have herde\\
Of the bysshope Seynt Roberde,\\
Hys toname [surname] is Grostest\\
Of Lynkolne, so seyth the gest,\\
He loved moche to here the Harpe,\\
For mannes wytte it makyth sharpe.\\
Next hys chaumbre, besyde his study,\\
Hys Harper’s chaumbre was fast therby.\\
Many tymes, by nightes and dayes,\\
He had solace of notes and layes,\\
One askede hym the resun why\\
He hadde delyte in Mynstralsy?\\
He answerde hym on thys manere\\
Why he helde the Harpe so dere:\\
\columnbreak

\vleftofline{“}The vertu of the Harpe, thurgh\\
\vin\vin\vin [through] skylle and ryght,\\
\vleftofline{“}Wyll destrye the fendys [fiends] myght;\\
\vleftofline{“}And to the Cros by gode skylle\\
\vleftofline{“}Is the Harpe ylykened weyl.\\
\vleftofline{“}Tharefore, gode men, ye shall lere, [learn]\\
\vleftofline{“}Whan ye any Gleman here,\\
\vleftofline{“}To wurschep God at your powere,\\
\vleftofline{“}As Davyd seyth in the Santere. [Psalter]\\
\vleftofline{“}In harpe and tabour and symphan\dcfootnote{ %a
\centering Either part-singing, or the instrument called the symphony.
} %end footnote
gle\\
\vleftofline{“}Wurschep God: in trumpes and sautre,\\
\vleftofline{“}In cordes, in organes, and bells ringyng:\\
\vleftofline{“}In all these wurschepe the hevene Kyng, \&c.”
\end{dcverse}\normalsize


Before entering on the reign of Edward I., I quit the Minstrels for awhile, to
endeavour to trace the progress of music up to that period. It will be necessary
to begin with the old Church Scales, it having been asserted that all national
music is constructed upon them—an assertion that I shall presently endeavour
to confute; and by avoiding, as far as possible, all obsolete technical, as well
as Greek terms, which render the old treatises on Music so troublesome a study,
I hope to convey such a knowledge of those scales as will answer the purpose of
such general readers as possess only a slight knowledge of music.

\section*{CHAPTER II.}

\subsection{Music of the Middle Ages.—Music in England to the end of
the Thirteenth Century.}

During the middle ages Music was always ranked, as now, among the seven
liberal arts, these forming the \textit{Trivium} and \textit{Quadrivium}, and studied by all
those in Europe who aspired at reputation for learning. The Trivium comprised
Grammar, Rhetoric, and Logic; the Quadrivium comprehended Music, 
\pagebreak
%===============================================================================
%012
Arithmetic, Geometry, and Astronomy. Sharon Turner remarks, that these 
comprised not only all that the Romans knew, cultivated, or taught, but
embodied “the whole encyclopaedia of ancient knowledge.” If we may trust
the following jargon hexameters, which he quotes as “defining the subjects
they comprised,” Music was treated as an art rather than as a science, and
a practical knowledge of it was all that was required:—

\renewcommand\versoheader{music of the middle ages.}
\renewcommand\rectoheader{gregorian tones.}
\DFNdouble
\settowidth{\versewidth}{\textit{Mus. canit}; Ar. numerat; Geo. ponderat; Ast. colit astra.}
\begin{quotation}\small
\noindent Gramm, loquitur; Dia. vera docet; Rhet. verba colorat\\
\textit{Mus. canit}; Ar. numerat; Geo. ponderat; Ast. colit astra.
\end{quotation}

\noindent But the methods of teaching both the theory and the practice of music were so
dark, difficult, and tedious, before its notation, measure, and harmonial laws were
settled, that we cannot wonder when we hear of youth having spent nine or ten
years in the study of scholastic music, and apparently to very little purpose.

In the latter part of the fourth century (\ad 374 to 397), Ambrose, bishop of
Milan, introduced a model of Church melody, in which he chose four series
or successions of notes, and called them simply the first, second, third, and fourth
tones, laying aside, as inapplicable, the Greek names of Doric, Phrygian, Lydian,
Æolic, Ionic, \&c. These successions distinguished themselves only by the position
of the semitones in the degrees of the scale, and are said to be as~follows:

\begin{center}
\begin{tabular}{llllllllllllll}
1st tone, & d & e & f & g & a & b & c & d \\
2nd tone, &\raisebox{1.5pt}{\rule{1em}{1pt}} & e & f & g & a & b & c & d & e \\
3rd tone, &\multicolumn{2}{c}{\raisebox{2pt}{\rule{3em}{1pt}}}& f & g & a & b & c & d & e & f \\
4th tone, &\multicolumn{3}{c}{\raisebox{2pt}{\rule{5em}{1pt}}} & g & a & b & c & d & e & f & g \\
\end{tabular}
\end{center}

These, Pope Gregory the Great (whose pontificate extended from 590 to 604)
increased to eight. He retained the four above-mentioned of Ambrose, adding to
them four others, which were produced by transposing those of Ambrose a fourth
lower; so that the principal note (or key-note, as it may be called) which formerly
appeared as the first in that scale, now appeared in the middle, or strictly
speaking, as the fourth note of the succession, the four additional scales being
called the \textit{plagal}, to distinguish them from the four more ancient, which received
the name of \textit{authentic}.

In this manner their order would of course be disarranged, and, instead of being
the first, second, third, and fourth tones, they became the first, third, fifth, and~seventh.

The following are the eight ecclesiastical tones (or scales) which still exist as such
in the music of the Romish church, and are called Gregorian, after their~founder:

\medskip
\small
\noindent \begin{tabular}{lcllrlrrclllll}
1st&tone&Authentic,&\multicolumn{2}{l}{\raisebox{2pt}{\rule{5em}{1pt}}}&D&e\tie f&g&A&b\tie c&D\\
2d&do.&Plagal,&A&b\tie c&D&e\tie f&g&A\\
3rd&do.&Authentic,&\multicolumn{3}{l}{\raisebox{2pt}{\rule{7em}{1pt}}}&E\tie f&g&a&B\tie c&d&E\\
4th&do.&Plagal,&\raisebox{1.5pt}{\rule{1em}{1pt}}&B\tie c&d&E\tie f&g&a&B&\\
5th&do.&Authentic,&\multicolumn{3}{l}{\raisebox{2pt}{\rule{7em}{1pt}}}&F&g&a&b\tie C&d&e\tie F\\
6th&do.&Plagal,&\raisebox{1.5pt}{\rule{1em}{1pt}}&C&d&e\tie F&g&a&b\tie C\\
7th&do.&Authentic,&\multicolumn{4}{l}{\raisebox{2pt}{\rule{10em}{1pt}}}&G&a&b\tie c&D&e\tie f&G\\
8th&do.&Plagal,&\multicolumn{2}{l}{\raisebox{2pt}{\rule{5em}{1pt}}}&D&e\tie f&G&a&b\tie c&D
\end{tabular}
\medskip
\normalsize



\noindent It will be perceived at the first glance,
that these Gregorian tones have only  
\pagebreak
%===============================================================================
%013
the intervals of the diatonic scale of C, such as are the white keys of the pianoforte,
without any sharps or flats. The only allowable accidental note in the Canto
fermo or plain song of the Romish church is B flat, the date of the introduction
of which has not been correctly ascertained.\dcfootnote{ %a
It was probably derived from the tetrachords of the
Greek scale, which admitted both \textit{b} flat and \textit{b} natural, but
which it is not necessary to discuss here.
} %end footnote 
No sharp occurs in genuine chants
of high antiquity. In some modern books the flat is placed at the clef upon \textit{b}, for
the fifth and sixth modes, but the strict adherents to antiquity do not admit this
innovation. These tones only differ from one another in the position of the half
notes or semitones, as from \textit{b} to \textit{c}, and from \textit{e} to \textit{f}. In the four plagal modes, the
final or key note remains the same as in the relative authentic; thus, although in the
sixth mode we have the \textit{notes} of the scale of C, we have not in reality the key of
C, for the fundamental or key note is \textit{f}; and although the first and eighth tones
contain exactly the same notes and in the same position, the fundamental note of
the first is \textit{d}, and of the eighth \textit{g}. There is no other difference than that the
melodies in the four authentic or principal modes are generally (and should
strictly speaking be) confined within the compass of the eight notes above the key
note, while the four plagal go down to a fourth below the key note, and only
extend to a fifth above it.

No scale or key of the eight ecclesiastical modes is to us complete. The first
and second of these modes being regarded, according to the modern rules of
modulation, as in the key of D minor, want a flat upon \textit{b}; the third and fourth
modes having their termination in E, want a sharp upon \textit{f}; the fifth and sixth
modes being in F, want a flat upon \textit{b}; and the seventh and eighth, generally
beginning and ending in G major, want an \textit{f} sharp.

The names of Dorian, Phrygian, Lydian, Mixolydian, \&c., have been applied to
them with equal impropriety (more particularly since Glareanus, who flourished
in the sixteenth century); they bear no more resemblance to the Greek scales than
to the modern keys above cited.

Pope Gregory made an important improvement by discarding the thoroughly
groundless system of the tetrachord, adopted by the ancient Greeks,\dcfootnote{ %b
In the old Greek notation there were 1620 tone characters, 
with which Musicians were compelled to burthen
their memories, and 990 marks actually different from
each other.
} %end footnote
and by
founding in its place that of the octave, the only one which nature indicates. And
another improvement no less important, in connexion with his system of the
octave, was the introduction of a most simple nomenclature of the seven sounds of
the scale, by means of the first seven letters of the alphabet. Burney says that the
Roman letters were first used as musical characters between the time of Boethius,\dcfootnote{ %c
It appears from Burney, that Boethius used the first
fifteen letters of the alphabet, but only as marks of
reference in the divisions of the monochord, not as
musical notes or characters.
} %end footnote
who died in 526, and St. Gregory; but Kiesewetter\dcfootnote{ %d
“History of the Modern Music of Western Europe,
from the first century of the Christian era, to the present
day,” \&c., by R. G. Kiesewetter, translated by Robert
Müller, 8vo., 1848. It is a very clearly and concisely
written history, and contains in an appendix within the
compass of a few pages, as much of the Greek music as
any modern can require to know.
} %end footnote 
attributes this improvement
in notation entirely to Gregory, in whose time the scale consisted only of two \pagebreak
octaves, the notes of the lower octave being expressed by capital letters, and the 
%===============================================================================
%014
higher by small letters. Eventually a third octave was added to the scale, four 
notes of which are attributed to Guido, and one to his pupils; the two remaining 
notes still later. The highest octave was then expressed by double letters; as, \textit{aa},
\textit{bb}, \&c. These three octaves in modern notes would constitute the following scale:

\bigskip
\medskip\lilypondfile[indent]{lilypond/three-octaves}




This is the alphabetical system of names for the notes which we, in England,
still retain for every purpose but that of exercising the voice, for which solfaing
on vowels is preferred.

Gregory’s alphabetical system of notation was, however, only partially adoptaed.
Some wrote on lines varying from seven to fifteen in number, placing dots, like
modern crotchet-heads, upon them, but making no use of the spaces. Others used
spaces only, and instead of the dots wrote the words themselves in the spaces, disjointing
each syllable to place it in the position the note should occupy. A third
system was by points, accents, hooks, and strokes, written over the words, and they
were intended to represent to the singer, by their position, the height of the note,
and by their upward or downward tendency, the rising or falling of the voice. It
was, however, scarcely possible for the writer to put down a mark so correctly,
that the singer could tell exactly which note to take. It might be one or two
higher or lower. To remedy this, a red line was drawn over, and parallel to the
words of the text, and the marks were written above and below it. A further
improvement was the use of two lines, one red and the other yellow, the red for F,
the yellow for C, as it only left three notes (G, A, and B) to be inserted between~them.\dcfootnote{ %a
Specimens of this notation, with red and yellow lines,
will be found in Martini’s Storia della Musica, vol. i.
p.~184; in Burney’s History, vol. ii. p.~37; in Hawkins’s
History, p.~947 (8vo. edition); and in Kiesewetter’s
p.~280. Also of other systems mentioned~above.
} %end footnote

Such was the notation before the time of Guido, a monk of Arezzo, in Tuscany,
who flourished about 1020. He extended the number of lines by drawing one
line under F, and another between F and C, and thus obtained four lines and
spaces, a number, which in the Rituals of the Romish Church has never been
exceeded.

The clefs were originally the letters F and C, used as substitutes for those red
and yellow lines. The Base clef still marks the position of F, and the Tenor
clef of C, although the forms have been changed.



Guido, in his Antiphonarium, gives the hymn
\textit{Ut queant laxis}\dcfootnote{ %b
Hymn for St. John the Baptist’s day, written by
Paul the Deacon, about 774.

\settowidth{\versewidth}{UT queant laxis}
\begin{fnverse}\scriptsize
UT queant laxis\\
REsonare fibris,\\
MIra gestorum\\
FAmuli tuorum:\\
SOLve polluti,\\
LAbii reatum,\\
\vin\vin Sancte Johannes.
\end{fnverse}

SI was not the settled name for B until nearly the end
of the seventeenth century; and, although it was proposed
in 1547, Butler in his Principles of Musick, 1636, gives 
the names of the notes as Ut, re, mi, fa, sol, la, \textit{pha}. In
1673, Gio. Maria Bononcini, father of Handel's pseudorival,
used Do in place of Ut, but the French still retain Ut.
} %end footnote
\pagebreak
(from the 
%014
%===============================================================================
%015
initial lines of which the names of the notes, Ut, re, mi, fa, sol, la, were taken), in 
old ecclesiastical notation, and in the Chronicle of Tours, under the year 1033, he
is mentioned as the first who applied those names to the notes. He did not add
the Greek gamma (our G) at the bottom of the scale,\dcfootnote{ %a
To distinguish G on the lowest line of the Base from
the G in the fifth space, the former was marked with the
Greek \Gamma, and hence the word gammut, applied to the
whole scale.
} %end footnote
as was long supposed,
for Odo, Abbot of Cluny, in Burgundy, had used it as the lowest note, in his
Enchiridion, a century before.

\renewcommand\rectoheader{scales, notation, clefs, and descant.}
To Franco, of Cologne (who, by the testimony of Sigebert, his cotemporary,
had acquired great reputation for his learning in 1047, and lived at least till 1083,
when he filled the office of Preceptor of the Cathedral of Liege), is to be ascribed
the invention of characters for \textit{time}.\dcfootnote{ %b
John de Muris, who flourished in 1330, in giving a list
of anterior musicians, who had merited the title of
inventors, names Guido, who constructed the gammut, or
scale, for the monochord, and placed notes upon lines and
spaces; after whom came Magister Franco, who invented
the figures, or notes, of the Cantus mensurabilis (qui
invenit in cantu mensuram figurarum). Marchetto da
Padova, who wrote in 1274, calls Franco the inventor of the
four first musical characters; and Franchinus Gaffurius
twice quotes him as the author of the time-table.
} %end footnote
By this he conferred the most important
benefit on music, for, till then, \textit{written} melody was entirely subservient to syllabic
laws, and music in parts must have consisted of simple counterpoint, such, says
Burney, as is still practised in our parochial psalmody, consisting of note against
note, or sounds of equal length.

The first ecclesiastical harmony was called Descant, and by the Italians, Mental
Counterpoint (Contrappunto alia mente). It consisted of extemporaneous singing
in fourths, fifths, and octaves, above and below the plain song of the Church; and
although in its original sense, it implied only singing in two parts, it had made
considerable advances in the ninth century, towards the end of which we find
specimens, still existing, of harmony in three and four parts. When Descant was
reduced to writing, it was called Counterpoint, from \textit{punctum contra punctum},
point against point, or written notes placed one against the other.

Hubald, Hucbald, or Hugbald, as he is variously named, and who died in~930,
at nearly ninety years of age, has left us a treatise, called Musica Enchiriadis,
which has been printed by the Abbé Gerbert, in his Scriptores Ecclesiastici. In
chapters X. to XIV., De Symphoniis, he says: “There are three kinds of
symphony (harmony), in the fourth, fifth, and octave, and as the combination of
some letters and syllables is more pleasing to the ear than others, so is it with
sounds in music. All mixtures are not equally sweet.” In the fifteenth chapter
he uses a transient second and third, both major and minor; and in the eighteenth
he employs four thirds in succession. Burney says: “Hubald’s idea that one
voice might wander at pleasure through the scale, while the other remains fixed,
shows him to have been a man of genius and enlarged views, who, disregarding
rules, could penetrate heyond the miserable practice of his time, into our Points
d’Orgue, Pedale, and multifarious harmony upon a holding note, or single base,
and suggests the principal, at least, of the boldest modern harmony.” It is in
this last sense of amplifying a point, that we still retain the verb to \textit{descant} in
common use. Guido describes the Descant \pagebreak
existing in his time, as consisting of 
%===============================================================================
%016
fourths, fifths, and octaves under the plain-song or chant, and of octaves (either 
to the plain song or to this base) above it. He suggests what he terms a smoother
and more pleasing method of under-singing a plain-song, in admitting, besides the
fourth and the tone, the major and minor thirds; rejecting the semitone and the
fifth. “No advances or attempts at variety seem to have been made in counterpoint,
from the time of Hubald, to that of Guido, a period of more than a hundred
years; for with all its faults and crudities, the counterpoint of Hubald is at least
equal to the best combinations of Guido;” but the monk, Engelbert, who wrote in
the latter end of the thirteenth century, tells us that all “regular descant” consists
of the union of fourths, fifths, and octaves, so that these uncouth and barbarous
harmonies, in that regular succession which has been since prohibited,
continued in the Church for four centuries.



Before the use of lines, there were no characters or signs for more than two
kinds of notes in the Church; nor since ecclesiastical chants have been written
upon four lines and four spaces, have any but the square and lozenge characters,
commonly called Gregorian notes, been used in Canto fermo: and, although the
invention of the time-table extended the limits of ingenuity and contrivance to
the utmost verge of imagination, and became all-important to secular music,
the Church made no use whatever of this discovery.

That melody received no great improvement from the monks, need excite
no wonder, as change and addition were alike forbidden; but not to have
improved harmony more than they did for many centuries after its use was
allowed, is a matter of just surprise, especially since the cultivation of music
was a necessary part of their profession.

We have occasional glimpses of secular music through their writings; for
instance, Guido, who gives a fair definition of harmony in the sense it is now
understood (Armonia est diversarum vocum apta coadunatio), says that he
merely writes for the Church, where the pure Diatonic genus was first used, but
he was aware of the deficiency as regards other music. “Sunt prœterea et alia
musicorum genera aliis mensuris aptata.” Franco (about 1050) just mentions
Discantum in Cantilenis Rondellis—“Descant to Rounds or Roundelays,”—but
no more.

When Franco writes in four parts, he sometimes gives five lines to each part,
the five lowest for the Tenor or plain song, the next five for the Medius, five for
the Triplum Discantus, and the highest for the Quadruplum. Each has a clef
allotted to it. Although many changes in the form of musical notes have been
made since his time, the lines and spaces have remained without augmentation or
diminution, four for the plain song of the Romish Church, and five for secular
music.

He devotes one chapter to characters for measuring silence, and therein gives
examples of rests for Longs, Breves, Semibreves, and final pauses. He also
suggests dots, or points of augmentation. Bars are placed in the musical examples,
as pauses for the singers to take breath at the end of a sentence, verse, or phrase
of melody. And this is the only use made of bars in Canto~fermo.
\pagebreak
%===============================================================================



%017
\renewcommand\rectoheader{anglo-saxon music.}
%\changefontsize{0.83\defaultfontsize}



Turning to England, Milton tells us, from the Saxon annals, that in 668, 
Pope Vitalian sent singers into Kent, and in 680, according to the Venerable
Bede,\dcfootnote{ %a
As a proof of the veneration in which Bede was held,
and the absurd legends relating to him, I quote from
a song of the fifteenth century:—

\settowidth{\versewidth}{\vin\textit{Songs and Carols. Percy Soc}. No. 73, p.~31.}

\begin{fnverse}\scriptsize
\vleftofline{“}When Bede had prechd to the stonys dry\\
The my[gh]t of God made [t]hem to cry\\
Amen:--certys this no ly[e]!”\\
\vin\textit{Songs and Carols. Percy Soc}. No. 73, p.~31.
\end{fnverse}
} %end footnote
Pope Agatho sent John, the Præcentor of St. Peter’s at Rome, to
instruct the monks of Weremouth in the manner of performing the ritual, and he
opened schools for teaching music in other parts of the kingdom of Northumberland.
Bede was also an able musician, and is the reputed author of a short
musical tract in two parts, \textit{de Musica theorica}, and \textit{de Musica practica, seu mensurata};
but Burney says, although the first may have been written by him, the
second is manifestly the work of a much more modern author, and he considers it
to have been produced about the twelfth century, \ie, between the time of Guido
and the English John de Muris. There must always be a difficulty in identifying
the works of an author who lived at so remote a period, without the aid of
contemporary authority, or of allusions to them of an approximate date; and when
he has written largely, such difficulties must be proportionably increased. But,
rejecting both the treatises on music, if he be the author of the Commentary on
the Psalms, which is included in the collected editions of his works of 1563
and 1688, sufficient evidence will remain to prove, not only his knowledge of
music, but of all that constituted the “regular” descant of the church from the
ninth to the thirteenth century. I select one passage from his Commentary on
the 52nd Psalm. “As a skilful harper in drawing up the cords of his instrument,
tunes them to such pitches, that the higher may agree in harmony with the lower,
some differing by a semitone, a tone, or two tones, others yielding the consonance
of the fourth, fifth, or octave; so the omnipotent God, holding all men predestined
to the harmony of heavenly life in His hand like a well-strung harp, raises some
to the high pitch of a contemplative life, and lowers others to the gravity of active
life.” And he thus continues:—“Giving the consonance of the octave, which
consists of eight strings;”\ldots\  “the consonance of the fifth, consisting of five
strings; of the fourth, consisting of four strings, and then of the smaller vocal
intervals, consisting of two tones, one tone, or a semitone, and of there being
semitones in the high as well as the low strings.”\dcfootnote{ %b
“Sient peritus citharæda chordas plures tendens in
cithara, temperat eas acumine et gravitate tali, ut
superiores inferioribus conveniant in melodia, quædam
semitonii, quædam unius toni, quædam duorum tonorum
differentiam gerentes, aliæ vero diatesseron, aliæ autem
diapente, vel etiam diapason consonantiam reddentes: ita
et Deus omnipotens omnes homines ad cœlestis vitæ
harmoniam prædestinatos in manu sua, quasi eitharam
quandam, chordis convenientibus ordinatam, habens,
quosdam quidem ad acutum contemplativæ vitæ sonum
intendit, alios verò ad activæ vitæ gravitatem temperando
remittit.”—“ut ad alios comparati quasi diapason consonantiam,
quæ oeto chordis constat, reddant..Quos
autem ad diapente consonantiam, quinque chordis constantem,
eligit, illi possunt intelligere qui tantæ jam perfectionis
sunt\ldots\  Diatesseron quatuor chordis constans.
\ldots\  Per minora vero vocum intervalla quæ duos tonos
aut unum, vel semitonium sonant\ldots\  Sed quia tam in
altisonis quam in grandisonis chordis habetur semitonium.”
\&c.—\textit{Bedæ Presbyteri Opera}, vol. 8, p.~1070, fol.
\textit{Basilæ}, 1563, or \textit{Coloniæ Agrippinæ}, vol. 8, p.~908,
fol. 1688.
} %end footnote
Our great king, Alfred,
according to Sir John Spelman, “provided himself of musitians, not common, or
such as knew but the practick part, but men skilful in the art itself;” and in 866, 
according to the annals of the Church of \pagebreak
Winchester, and the testimony of many 
%===============================================================================
%018
ancient writers, he founded a Professorship at Oxford,\dcfootnote{ %a
The earliest express mention of the University of
Oxford, after the foundation of the schools there by
Alfred, is from the historian Ingulphus, whose youth
coincided with the early part of the reign of Edward the
Confessor. He tells us that, having been born in the City
of London, he was first sent to school at Westminster,
and that from Westminster he proceeded to Oxford,
where he studied the Aristotelian Philosophy, and the
rhetoritical writings of Cicero.
} %end footnote
for the cultivation of music 
as a science. The first who filled the chair was Friar John, of St. David’s, who
read not only lectures on Music, but also on Logic and Arithmetic. Academical
honors in the faculty of music have only been traced back to the year 1463, when
Henry Habington was admitted to the degree of Bachelor of Music, at Cambridge,
and Thomas Saintwix, Doctor of Music, was made Master of King’s College, in
the same university; but it is remarkable that music was the only one of the
seven sciences that conferred degrees upon its students, and England the only
country in which those degrees were, and are still conferred.

\renewcommand\versoheader{music in england, time of henry ii.}
\renewcommand\rectoheader{giraldus cambrensis’ account.}
%%\changefontsize{1.0\defaultfontsize}

About 1159, when Thomas à Becket conducted the negociations for the
marriage of Henry the Second’s eldest son with the daughter of Louis VII., and
went to Paris, as chancellor of the English Monarch, he entered the French towns,
his retinue being displayed with the most solicitous ostentation, “preceded by two
hundred and fifty boys on foot, in groups of six, ten, or more together, singing
English songs, according to the custom of their country.”\dcfootnote{ %b
“In ingressu Gallicanarum villarum et castrorum,
primi veuiebant garciones pedites quasi ducenti quinquaginta, 
gregatim euntes sex vel deni, vel plures simul,
aliquid lingua sua pro more patriæ suæ cantantes.”—
\textit{Stephanides, Vita S. Thomæ Cantuar}, pp.~20,~21.
} %end footnote 
This singing in groups
resembled the “turba canentium,” of which Giraldus afterwards speaks; and the
following passage from John of Salisbury, about 1170, shows at least the delight
the people had in listening to part-singing, or descant. “The rites of religion
are now profaned by music; and it seems as if no other use were made of it than
to corrupt the mind by wanton modulations, effeminate inflexions, and frittered
notes and periods, even in the \textit{Penetralia}, or sanctuary, itself. The senseless
crowd, delighted with all these vagaries, imagine they hear a concert of sirens,
in which the performers strive to imitate the notes of nightingales and parrots,
not those of men, sometimes descending to the bottom of the scale, sometimes
mounting to the summit; now softening, and now enforcing the tones, repeating
passages, mixing in such a manner the grave sounds with the more grave, and
the acute with the most acute, that the astonished and bewildered ear is unable
to distinguish one voice from another.”\dcfootnote{ %
Musica cultum religionis incestat, quod ante conspectum
Domini, in ipsis penetralibus sanctuarii, lascivientis
vocis luxu, quadam ostentatione sui, muliebribus
modis notularum articulorumque cæsuris, stupentes
animulas emollire nituntur. Cum præcinentium, et succinentium, 
canentium, et decinentium, intercinentium,
et occinentium, præmolles modulationes audieris, Sirenarum
concentus credas esse, non hominum et de vocum
facilitate miraberis, quibus philomela vel psittacus, aut
si quid sonorius est, modos suos nequeunt coæquare. Ea
siquidem est, ascendeudi descendendique facilitas; ea
sectio vel geminatio notularum, ea replicatio articulorum,
singulorumque consolidatio; sic acuta vel acutissima,
gravibus et subgravibus temperantur, ut auribus sui
indicii fere subtrahetur auturitas.—\textit{Policraticus, sive de
Nugis Curialium}, lib. i., c.~6.
} %end footnote
It was probably this abuse of descant
that excited John’s opposition to music, and his censures on the minstrels, as
shown in the passage before quoted. It proves also, that descant in England did
not then consist merely of singing in two parts, but included the licenses and
ornaments of florid song. Even singing in canon seems to be comprised in the
words, “præcinentium et succinentium, canentium et decinentium.”

%%\changefontsize{1.0\defaultfontsize}
About 1185, Gerald Barry, or Giraldus Cambrensis, 
archdeacon, and afterwards \pagebreak
%\end{fixedpage}%018
%===============================================================================
%019
bishop, of St. David’s, gave the following description of the peculiar manner 
of singing of the Welsh, and the inhabitants of the North of England: “The
Britons do not sing their tunes in unison, like the inhabitants of other countries,
but in different parts. So that when a company of singers meets to sing, as is
usual in this country, as many different parts are heard as there are singers, who
all finally unite in consonance and organic melody, under the softness of B flat.\dcfootnote{ %a
“Uniting under the softness of B flat,” is not very
intelligible, but one thing may be inferred from it, that
they sang in the natural scale, such as the fifth mode
became by the use of B flat in the scale of F, and not in
the modes that were peculiar to the church. B flat was
only used in the fifth mode and its plagal.
} %end footnote
In the Northern parts of Britain, beyond the Humber, and on the borders of
Yorkshire, the inhabitants make use of a similar kind of symphonious harmony
in singing, but with only two differences or varieties of tone and voice, the one
murmuring the under part, the other singing the upper in a manner equally soft
and pleasing. This they do, not so much by art, as by a habit peculiar to themselves,
which long practice has rendered almost natural, and this method of singing
has taken such deep root among this people, that hardly any melody is accustomed
to be uttered simply, or otherwise than in many parts by the former, and in two
parts by the latter. And what is more astonishing, their children, as soon as they
begin to sing, adopt the same manner. But as not all the English, but only those
of the North sing in this manner, I believe they had this art at first, like their
language, from the Danes and Norwegians, who were more frequently accustomed
to occupy, as well as longer to retain, possession of those parts of the island.”\dcfootnote{ %b
In musico modulamine non uniformiter ut alibi,
sed multipliciter multisque modis et modulis cantilenas
emittunt, adeò ut in turba canentium, sicut huic genti
mos est, quot videas capita tot audias carmina discriminàque
vocum varia, in unam denique sub B mollis
dulcedine blanda consonantiam etorganicam convenientia
melodiam. In borealibus quoque majoris Britanniæ partibus
trans Humbrum, Eboracique finibus Anglorum
populi qui partes illas inhabitant simili canendo symphonica
utuntur harmonia: binis tamen solummodo
tonorum differentiis et vocum modulando varietatibus,
una inferius sub murmurante altera verò supernè demulcente
pariter et delectante. Nec arte tantum sed usu
longævo et quasi in naturam mora diutina jam converso,
hæc vel illa sibi gens hanc specialitatem comparavit.
Qui adeò apud utramque invaluit et altas jam radices
posuit, ut nihil hic simpliciter, ubi multipliciter ut apud
priores, vel saltem dupliciter ut apud sequentes, mellitè
proferri consuaverit. Pueris etiam (quòd magis admirandum) 
et ferè infantibus, (cum prinum à fletibus in
cantus erumpunt) eandem modulationem observantibus,
Angli verò quoniam non generaliter omnes sed boreales
solùm hujusmodi vocum utuntur modulationibus, credo
quòd a Dacis ct Norwagiensibus qui partes illas insulæ
frequentiùs occupare ac diutiùs obtinere solebant, sicut
loqueudi affinitatem, sic canendi proprietatem contraxerunt.—\textit{Cambriæ
Descriptio}, cap. xiii.
} %end footnote
Now, allowing a little for the hyperbolic style so common with writers of that age,
this may fairly be taken as evidence that part-singing was common in Wales, or
that at least they made descant to their tunes, in the same way that singers did
to the plain-song or Canto fermo of the Church at the same period; also that
singing in two parts was common in the North of England, and that children tried
to imitate it. Burney and Hawkins think that what Giraldus says of the singing
of the people of Northumberland, in two parts, is reconcileable to probability,
because of the schools established there in the time of Bede, but Burney doubts
his account of the Welsh singing in many parts, and makes this “turba
canentium” to be \textit{of the common people}, adding, “we can have no exalted idea
of the harmony of an \textit{untaught} crowd.” These, however, are his own inferences;
Giraldus does not say that the singers were untaught, or that they were of
the common people. As he is describing \pagebreak
what was the custom in his own time, 
%===============================================================================
%020
not what had taken place a century before, there seems no sufficient ground 
for disbelieving his statement,\dcfootnote{ %a
Dr.~Percy says, “The credit of Giraldus, which hath
been attacked by some partial and bigoted antiquaries,
the reader will find defended in that learned and curious
work, ‘Antiquities of Ireland,’ by Edward Ledwich,
LL.D. Dublin, 1790, 4to., p.~207. et seq.”
} %end footnote
and least of all, should they who are of the opinion
that all musical knowledge was derived from the monasteries, call it in question,
since, as already shown, part music had then existed in the Church, in the form
of descant, for three centuries.

\renewcommand\versoheader{harpers not taught by monks.}
\renewcommand\rectoheader{character of tunes often derived from instruments.}
%%\changefontsize{1.06\defaultfontsize}

“If, however,” says Burney, “incredulity could be vanquished with respect to
the account which Giraldus Cambrensis gives of the state of music in Wales
during the twelfth century, it would be a Welsh MS. in the possession of Richard
Morris, Esq., of the Tower, which contains pieces for the harp, that are in \textit{full
harmony} or \textit{counterpoint}; they are written in a peculiar notation, and supposed
to be as old as the year 1100; at least, such is the known antiquity of many of
the songs mentioned in the collection,” \&c. It is not necessary here to enter
into the defence of Welsh music, but the specimens Dr.~Burney has printed from
that manuscript, which he describes as in full harmony and counterpoint, are
really nothing more than the few simple chords which must fall naturally under
the hand of any one holding the instrument, and such as would form a child’s
first lessons. First the chord, G C E, and then that of B D F, form the entire
bass of the only two lessons he has translated; and though from B to F is
a “false fifth,” it must be shown that the harper derived his knowledge of
the instrument from the Church, before the assertion that it is more modern
harmony than then in use can have any weight. In England, at least, not
only the evidence of Giraldus, but all other that I can find, is against such a
supposition. I have before alluded to the Romance of Horn-Child, (note \textit{c}, to
page 9), and here give the passage, to prove that such knowledge was not
derived from the Church, as well as to show what formed a necessary part of
education for a knight or warrior. It is from that part of the story where
Prince Horn appears at the court of the King of Westnesse.

\settowidth{\versewidth}{His steward, and [to] him said thus}
\begin{dcverse}
\vin\vin\textsc{Original Words.}

“The kyng com in to halle,\\
Among his knyhtes alle,\\
Forth he clepeth Athelbrus,\\
His stiward, and him seide thus:\\
‘Stiward, tac thou here\\
My fundling, for to lere\\
Of thine, mestere\\
Of wode and of ryuere,\\
\textit{Ant toggen o the harpe}\\
\textit{With is nayles sharpe}.\\
Ant tech him alle the listes\\
That thou euer wystest,\\
Byfore me to keruen.\\
And of my coupe to seruen:

\vin\vin\textsc{Words Modernized.}

The king came into [the] hall\\
Among his knights all,\\
Forth he calleth Athelbrus,\\
His steward, and [to] him said thus\\
“Steward, take thou here\\
My foundling, for to teach\\
Of thy mystery\\
Of wood and of river,\\
And to play on the harp\\
With his nails sharp.\\
And teach him all thou listest,\\
That thou ever knewest,\\
Before me to carve\\
And my cup to serve: 
\end{dcverse}

\pagebreak
%===============================================================================

%021

%\changefontsize{0.85\defaultfontsize}

\settowidth{\versewidth}{Horn-Child, thou vnderstond}
\begin{dcverse}
Ant his feren deuyse\\
With ous other seruise;\\
Horn-Child, thou vnderstond\\
\textit{Tech him of harpe and of song}.’”

And devise for his fellows\\
With us other service;\\
Horn-Child, thou understand\\
Teach him of harp and of song.”
\end{dcverse}

\noindent In another part of the poem he is introduced playing on his harp. 

\begin{dcverse}
Horn sette him abenche,\\
Is harpe he gan clenche\\
He made Rymenild a lay\\
Ant hue seide weylaway, \&c.\dcfootnote{ %a
Warton’s History of English Poetry, vol. i., p.~38, 8vo., 1840.
}

Horn seated himself on a bench,\\
His harp he began to clench;\\
He made Rymenild a lay\\
And he said wellaway! \&c.
\end{dcverse}

In searching into the early history of the music of any country, the first
subject of inquiry should be the nature and character, as well as the peculiarities
of scale, of the musical instruments they possessed. If the musical instruments
in general use had an imperfect scale, the national music would generally, if not
universally, have retained the peculiarities of that scale. Hence the characteristics
of Scottish music, and of some of the tunes of the North of England, which resemble
it. In the following collection many can he pointed out as bagpipe tunes,
such as “Who liveth so merry in all this land, as doth the poor widow that selleth
the sand,” and “By the border’s side as I did pass,” both of which seem to
require the accompaniment of the drone, while others, like “Mall (or Moll)
Sims,” strictly retain the character of harp music. Where, however, the harp
was in general use, the scale would be more perfect than if some other instruments
were employed, and hence the melodics would exhibit fewer peculiarities,
unless, indeed, the harp was tuned to some particular scale, which, judging by
the passage above quoted from Bede, does not seem to have been the case in
England.

About 1250 we have the song, \textit{Sumer is icumen in}, the earliest secular composition, 
in parts, known to exist in any country. Sir John Hawkins supposed that
it could not he earlier than the fifteenth century, because John of Dunstable, to
whom the invention of figurative music has been attributed, died in 1455. But
Dr.~Burney remarks that Dunstable could not have been the inventor of that art,
concerning which several treatises were written before John was born, and shows
that mistake to have originated in a passage from Proportionales Musices, by John
Tinctor, a native of Flanders, and the “most ancient composer and theorist of
that country, whose name is upon record.” It is as follows: “Of which new art,
as I may call it (counterpoint), the fountain and source is said to have been among
the English, of whom Dunstable was the \textit{chief}.”\dcfootnote{ %b
“Cujus, ut ita dicam, novæ artis (Contrapunctis), fons
et origo apud Anglos, quorum caput Dunstaple extitit,
fuisse perhibetur.” From Proportionale Musices, dedicated
to Ferdinand, king of Sicily, Jerusalem, and
Hungary (who reigned from 1458 to 1494), by John
Tinctor, Chaplain and Maestro di Capella to that Prince.
} %end footnote 
“Caput,” literally meaning
“head,” had been understood in its secondary sense of “originator or beginner.”

%%\changefontsize{1.05\defaultfontsize}
Dr.~Burney’s opinion with respect to the age of this canon seems to have been
very unsettled (if indeed he can be said to have formed one at all). He first
presents it as a specimen of the harmony \pagebreak
in our country, “about the fourteenth 
%\end{fixedpage}%021
%===============================================================================
%022
and fifteenth century.” On the same page he tells us that the notes of the 
MS. resemble those of Walter Odington’s Treatise\dcfootnote{ %a
The Best summary of the state of music in England,
about 1230, is contained in Walter Odington’s Treatise,
which is fully described in Burney’s History of Music,
vol. ii., p.~155, et seq. Burney considers it the most
complete of all the early treatises, whether written here
or abroad.
} %end footnote
(1230), and seem to be of the
thirteenth or fourteenth century, and he can hardly imagine the canon much
more modern. Then he is “sometimes inclined to imagine” it to have been
the production of the Northumbrians, (who, according to Giraldus Cambrensis,
used a kind of natural symphonious harmony,) \textit{but with additional parts}, and a
second drone-base of later times. By “additional parts” I suppose Burney
to mean \textit{adding to the length} of the tune, and so continuing the canon. Next
in reviewing “the most ancient musical tract that has been preserved in our
vernacular tongue” (by Lyonel Power), he says, this rule (a prohibition of
taking fifths and octaves in succession) seems to have been so much unknown
or disregarded by the composer of the canon, “Sumer is icumen in,” as to
excite a suspicion that it is “much more ancient than has been imagined.”
And finally, “It has been already shown that counterpoint, in the Church,
began by adding parts to plain chant; and in secular music, by harmonizing
old tunes, as florid melody did by variations on these tunes. It was long
before men had the courage to invent new melodies. It is a matter of surprise
that so little plain counterpoint is to be found, and of this little, none
correct, previous to attempts at imitation, fugue, and canon; contrivances to which
there was a very early tendency, in all probability, during times of extemporary
descant, before there was any such thing as written harmony: for we find in the
most ancient music in parts that has come down to us, that fugue and canon had
made considerable progress at the time it was composed. The song, or round,
‘Sumer is icumen in,’ is a very early proof of the cultivation of this art.” He
then proceeds to show how, according to Martini, from the constant habit of
descanting in successive intervals, new melodies would be formed in harmony with
the original, and whence imitations would naturally arise.

\renewcommand\versoheader{manuscripts---thirteenth century.}
\renewcommand\rectoheader{sumer is icumen in.}

Ritson, who knew more of the age of manuscripts than of musical history, is
of opinion that Burney and Hawkins were restrained by fear from giving their
opinion of its date, and says it may be referred to as early a period (\textit{at least}) as
the year 1250. Sir Frederick Madden,\dcfootnote{ %b
Keeper of the Manuscripts in the British Museum.
} %end footnote
in a note to the last edition of Warton’s
English Poetry, says: “Ritson justly exclaims against the ignorance of those who
refer the song to the fifteenth century, when the MS. itself is certainly of the
middle of the thirteenth.” Mr. T. Wright, who has devoted his attention
almost exclusively to editing Anglo-Norman, Anglo-Saxon, and early English
manuscripts, says: “The latter part of this manuscript, containing, among others,
the long political song printed in my Pol. Songs, p.~72, was certainly written
during the interval between the battle of Lewes, in May, 1264, and that of
Evesham, in the year following, and most probably immediately after the firstmentioned
event. The earlier part of the MS., 
which contains the music, was evidently written at an earlier \pagebreak period—perhaps by twenty or 
thirty years—and 
%\end{fixedpage}%022
%===============================================================================
%023
the song with its music must therefore be given to the first half of the thirteenth 
century, at latest.” I have thus entered into detail concerning this song
(though all the judges of manuscripts, whom I have been enabled to consult, are
of the same opinion as to its antiquity), because it is not only one of the first
English songs with or without music, but the first example of counterpoint in six
parts, as well as of fugue, catch, and canon; and at least a century, if not two
hundred years, earlier than any composition of the kind produced out of
England.\dcfootnote{ %a
The earliest specimen of secular part-music that has
yet been discovered on the Continent, is an old French
song, for three voices, the supposed production of a singer
and poet, by name Adam de la Hale, called Le Boiteux
d'Arras, who was in the service of the Comte de Provence.
The discovery has been recently made and communicated
by M. Fétis, in his Revue Musicale. “It may be placed
about the year~1280, if a dilettante of the discantus of \textit{the
following age} has not experimentalised on the melody left
by De la Hale, as on a tenor or Canto fermo; since the
other songs, in similar notation, are not in counterpoint;
and the manuscript may be assigned to the \textit{fourteenth}
century.” It is given in Kiesewetter’s History of Music.
} %end footnote

%%\changefontsize{0.98\defaultfontsize}
The antiquity of the words has not been denied, the progress of our language
having been much more studied than our music, but the manuscript deserves much
more attention from musicians than it has yet received.\dcfootnote{ %b
The Musical Notation in this MS, (Harl. 978) is
throughout the same. Only two forms of note are used
with occasional ligatures. “Sumer is icumen in” is on the
back of page 9, and just after it is an Antiphon in praise
of Thomas à Becket. At page 12 we have the musical
scale in letters, exactly corresponding with the scale of
Guido, with the ut, re, mi, fa, \&c., but only extending to
two octaves and four notes, without even the “\textit{e e},” said
to have been added by his pupils. At the back of that
page is an explanation of the intervals set to music, to
impress them on the memory by singing, and examples of
the ligatures used in the notation of the manuscript. At
page 8 is a hymn, “Ave gloriosa mater Salvatoris,” with
Latin and Norman French words, in score in three parts,
on fifteen red lines undivided, and with three clefs for the
voices. The remainder of the musical portion of the
manuscript consists of hymns, \&c., in one or two parts.
} %end footnote
It is not in Gregorian
notation, which might have been a bar to all improvement, but very much resembles
that of Walter Odington, in 1230. All the notes are black. It has neither
marks for time, the red note, nor the white open note, all of which were in use in
the following century.

The chief merit of this song is the airy and pastoral correspondence between
the words and music, and I believe its superiority to be owing to its having been
a national song and tune, selected, according to the custom of the time, as a basis
for harmony, and that it is not entirely a scholastic composition. The fact of its
having a natural drone bass would tend rather to confirm this view than otherwise.
The bagpipe, the true parent of the organ, was then in use as a rustic instrument
throughout Europe. The rote, too, which was in somewhat better estimation, had
a drone, like the modern hurdy-gurdy, from the turning of its wheel. When the
canon is sung, the key note may be sustained \textit{throughout}, and it will be in accordance
with the rules of modern harmony. But the foot, or burden, as it stands
in the ancient copy, will produce a very indifferent effect on a modern ear,\dcfootnote{ %
We ought, perhaps, to except the lover of Scotch
Reels.
} %end footnote
from its
constantly making fifths and octaves with the voices, although such progressions
were not forbidden by the laws of music in that age. No subject would be more
natural for a pastoral song than the approach of Summer; and, curiously enough,
the late Mr. Bunting noted down an Irish song from tradition, the title of
which he translated “Summer is coming,” and the tune begins in the same way.
That is the air to which Moore adapted the words, “Rich and rare were the gems 
she wore.” Having given a fac-simile of \pagebreak
“Sumer is icumen in,” taken from the 
%\end{fixedpage}%023
%===============================================================================
%024
manuscript, and as it may be seen in score in Burney and Hawkins’ Histories, 
the tune is here printed, harmonized by Mr. Macfarren, as the first of National
English Airs. A few obsolete words have been changed, but the original are
given below.

\renewcommand\versoheader{sumer is icumen in.}
\renewcommand\rectoheader{songs with music---thirteenth century.}
%%\changefontsize{1.1\defaultfontsize}


\musictitle{Sumer is icumen in.}
\musicinfo{Rather slow and smoothly.}{About 1250.}
\medskip\lilypondfile[staffsize=14]{lilypond/024-sumer-is-icumen-in}


\settowidth{\versewidth}{Groweth sed, and bloweth med}
\indentpattern{01013}
\begin{dcverse}
\vin\vin\textsc{Original Words.}

\begin{patverse}
Sumer is icumen\dcfootnote{ %a
“\textit{icumen}” come (from the Saxon verb, \textit{cuman}, to
come); so in Robert of Gloucester, \textit{i\/}paied for paid.
} in,\\
Lhude\dcfootnote{ %
Lhude, wde, awe, and calve, are all to be pronounced as
of two syllables.
} sing Cuccu,\\
Groweth sed, and bloweth med\\
And springth the wde nu\\
Sing Cuccu!
\end{patverse}

\vin\vin\textsc{Words Modernized.}

\begin{patverse}
Summer is come in,\\
Loud sing Cuckoo!\\
Groweth seed, and bloweth mead\\
And spring’th the wood now\\
Sing Cuckoo. 
\end{patverse}
\end{dcverse}


%\end{fixedpage}%024
\pagebreak
%===============================================================================



%025

%\changefontsize{0.90\defaultfontsize}


\settowidth{\versewidth}{Bulluc sterteth, bucke verteth}
\indentpattern{0101300}
\begin{dcverse}
\begin{patverse}
Awe bleteth after lomb\\
Lhouth after calve cu;\\
Bulluc sterteth, bucke verteth\\
Murie sing Cuccu,\\
Cuccu, Cuccu.\\
Wel singes thu Cuccu\\
Ne swik thu naver nu.
\end{patverse}

\begin{patverse}
Ewe bleateth after lamb,\\
Loweth after calf [the] cow.\\
Bullock starteth,\dcfootnote{\centering Jumps.} buck verteth\dcfootnote{\centering Frequents the green fern} \\
Murie sing Cuccu,\\
Cuckoo, Cuckoo!\\
Well sing’st thou Cuckoo\\
Nor cease thou never now.
\end{patverse}
\end{dcverse}

In the original, the “Foot,” or Burden, is sung, as an under part by two 
voices, to the words, “Sing Cuccu, nu, sing Cuccu,” making a rude base to it.

Two other songs of the thirteenth century on the approach of Summer are
printed in Reliquiæ Antiquæ (8vo. Bond. 1841), but without music. The first
is taken from MSS. Egerton, No. 613, Brit. Mus., and begins thus:—

\settowidth{\versewidth}{“Somer is comen, and winter is gon, this day beginniz to longe [lengthen],}

\begin{scverse}
\vleftofline{“}Somer is comen, and winter is gon, this day beginniz to longe [lengthen],\\
And this foules everichon [birds every one] joy [t]hem wit[h] songe.”
\end{scverse}

The other from MSS. Digby, No. 86, Oxford, of the Thrush and the Nightingale:

\settowidth{\versewidth}{With blostme [blossom], and with brides roune [birds’ songs]}

\begin{scverse}
\indentpattern{001}
\begin{patverse}
\vleftofline{“}Somer is comen with love to toune\\
With blostme [blossom], and with brides roune [birds’ songs]\\
The note [nut] of hasel springeth,’' \&c.
\end{patverse}
\end{scverse}

In the Douce Collection (Bod. Lib., Ox., MS. No. 139), there is an English
song with music, beginning—
\settowidth{\versewidth}{Foweles in the frith, the fisses in the flod.}
\begin{scverse}
“Foweles in the frith, the fisses in the flod.”
\end{scverse}
and the MS., which contains it, is of the thirteenth century, but it is only in
two parts; and in Harl. MSS. No. 1717, is a French or Anglo-Norman song,
“Parti de Mal,” which seems to have been cut from an older manuscript to form
the cover of a Chronicle of the Dukes of Normandy, written by order of Henry II.
It is only for one voice, and a sort of hymn, but a tolerable melody. Both these
may be seen in Stafford Smith’s Musica Antiqua, Vol. 1.

Another very early English song, with music, is contained in a manu\-script,
“Liber de Antiquis Legibus,” now in the Record Room, Town Clerk’s Office,
Guildhall. It contains a Chronicle of Mayors and Sheriffs of London, and of the
events that occurred in their times, from the year 1188 to the month of August,
1274, at which time the manuscript seems to have been completed. It is the
Song of a Prisoner. The first four lines are more Saxon than modern English:—
%\medskip

\settowidth{\versewidth}{Geltles ihc sholye muchele schame}
\begin{dcverse}
\vin\vin\textsc{Original Words.}

Ar ne kuthe ich sorghe non\\
Nu ich mot manen min mon\\
Karful wel sore ich syche\\
Geltles ihc sholye muchele schame\\
Help, God, for thin swete name\\
\vin Kyng of Hevene riche.

\vin\vin\textsc{Words Modernized.}

Ere [this] knew I sorrow none\\
Now I must utter my moan\\
Full of care well sore I sigh\\
Guiltless I suffer much shame\\
Help, God, for thy sweet name,\\
\vin King of Heaven-Kingdom. \\
\end{dcverse}

%\end{fixedpage}%025
\pagebreak



%26
\renewcommand\versoheader{national songs not on church scales.}
\renewcommand\rectoheader{church music always in arrear.}
%\changefontsize{0.88\defaultfontsize}



In the Arundel Collection (No. 292), there is a song in “a handwriting of
the time of Edward II.,” beginning—
\settowidth{\versewidth}{“Uncomly in cloystre I coure [cower] ful of care,”}
\begin{scverse}
“Uncomly in cloystre I coure [cower] ful of care,”
\end{scverse}
which is on the comparative difficulties of learning secular and church music,
but, except in the line, “Thou bitest asunder bequarre for bemol” (B natural
for B flat), there is no reference to the practice of music.

Secular music must have made considerable progress before the end of the
thirteenth century, for even Franco had spoken of a sort of composition called
“Conductus,” in which, instead of merely adding parts to a plain song, the student
was first to compose as pretty a tune as he could, and then to make descant
upon it;\dcfootnote{ %a
“In Couductis aliter est operandum, quia qui vult
facere Conductum, primum cantum invenire debet pulchriorem
quam potest, deinde uti debet illo, ut de tenore,
faciendo discantura.”
} %end footnote
and he further says, that in every other case, some melody already made
is chosen, which is called the tenor, and governs the descant originating from it:
but it is different in the Conductus, where the cantus (or melody) and the descant
(or harmony) are both to be produced. This was evidently applied to secular
composition, since, about 1250, Odo, Archbishop of Rheims, speaks of Conducti et
Motuli as “jocose and scurrilous songs.”

Accidental sharps, discords and their resolutions, and even chromatic counterpoint,
are treated on by Marchetto of Padua (in his Romerium Artis Musicæ
Mensurabilis) in 1274, and the Dominican Monk, Peter Herp, mentions in
Chronicle of Frankfort, under the year 1300, that new singers, composers, and
harmonists had arisen, who used other scales or modes than those of the Church.\dcfootnote{ %b
“Novi cantores surrexere, et componistæ, et figuristæ.
qui inceperunt alios modos assuere.” When music deviated
from the Church scales, it was called by the old
writers generally, \textit{Musica falsa}, and by Franchinus,
\textit{Musica ficta, seu colorata}, from the chromatic semitones
used in it.
} %end footnote
Pope John XXII. (in his decree given at Avignon in 1322) reproves those who,
“attending to the \textit{new notes and new measures of the disciples of the new school},
would rather have their ears tickled with semibreves and minims, and \textit{such frivolous}
inventions, than hear the ancient ecclesiastical chant.” White minims, with tails,
to distinguish them from semibreves, seem first to have been used by John de
Muris, about 1330, retaining the lozenge-shaped head to the note. He also used
signs to distinguish triple from common time. These points should be borne in
mind in judging of the age of manuscripts.

It will be observed that “Sumer is icumen in” is not within the compass of
any Church scale. It extends over the octave of F, and ends by descending to the
seventh below the key note for the close, which, indeed, is one of the most common
and characteristic terminations of English airs. The dance tune which follows
next in order has the same termination, and extends over a still greater compass
of notes. I shall therefore quit the subject of Church scales, relying on the
practical refutation which a further examination of the tunes will afford. Burney
has remarked that at any given period secular music has always been at least a
century in advance of Church music. And notwithstanding the improvements
in musical \textit{notation} made by monks, the Church still adhered to her imperfect 
system, as well as to bad harmony, for \pagebreak
 centuries after better had become general.  
%\end{fixedpage}%026
 %===============================================================================
%027
Even in the sixteenth century, modulation being still confined to the ecclesiastical
modes, precluded the use of the most agreeable keys in music. Zarlino, who
approved of the four modes added by Glareanus, speaks of himself, and a few
others, having composed in the eleventh mode, or key of C natural (which was not
one of the original eight), to which they were led \textit{by the vulgar musicians of the
streets and villages}, who generally accompanied rustic dances with tunes in this
key, and which was then called, \textit{Il modo lascivo}—The wanton key. I suppose it
acquired this name, because, like the “sweet Lydian measure” of old, the interval
from the seventh to the octave is only a semitone.

\musictitle{Dance Tune.}

\musicinfo{}{About 1300.}

\medskip\lilypondfile{lilypond/027-dance-tune}


The above dance tune is taken from the Musica Antiqua by John Stafford
Smith. He transcribed it from a manuscript then in the possession of Francis
Douce, Esq. (who bequeathed the whole of his manuscripts to the Bodleian
Library), and calls it, “a dance tune of the reign of Edward II., or earlier.”
The notation of the MS. is the same \pagebreak
as in that which contains \textit{Sumer is icumen in}, 
%\end{fixedpage}%027
%===============================================================================
%028
and I do not think it can be dated later than 1300. Dr: Crotch remarks:—
“The abundance of appoggiaturas in so ancient, a melody, and the number of bars
in the phrases, four in one and five in another—nine in each part, are its most
striking peculiarities. It is formed on an excellent design, similar to that of
several fine airs of different nations. It consists of three parts, resembling each
other excepting in the commencement of their phrases, in which they tower above
each other with increasing energy, and is altogether a curious and very favorable
specimen of the state of music at this very early period.”

\renewcommand\versoheader{english minstrelsy resumed.}
\renewcommand\rectoheader{edward i.}
\DFNsingle

The omission of the eighth bar in each phrase would make it strictly in modern~rhythm.


\section*{CHAPTER III.}

\subsection{English Minstrelsy from 1270 to 1480, and the gradual extinction
of the old Minstrel.}

%\changefontsize{0.86\defaultfontsize}
Edward the First, according to the Chronicle of Walter Hemmingford, about the
year 1271, a short time before he ascended the throne, took his harper with him
to the Holy Land, who must have been a close and constant attendant on his
master, for when Edward was wounded at Ptolemais, the harper (Citharæda
suus), hearing the struggle, rushed into the royal apartment, and, striking the
assassin on the head with a tripod or trestle, beat out his brains.

“That Edward ordered a massacre of the Welsh bards,” says Sharon Turner,
“seems rather a vindictive tradition of an irritated nation than an historical fact.
The destruction of the independent sovereignties of Wales abolished the patronage
of the bards, and in the cessation of internal warfare, and of external ravages,
they lost their favorite subjects, and most familiar imagery. They declined
because they were no longer encouraged.” The Hon. Daines Barrington could
find no instances of severity against the Welsh in the laws, \&c. of this monarch,\dcfootnote{ %a
\centering See his observations on the statutes, 4to. 4th Ed.
} %end footnote
and that they were not extirpated is proved by the severe law which we find in
the Statute Book, 4 Henry IV. (1402), c. 27, passed against them during the
resentment occasioned by the outrages committed under Owen Glendour. In that
act they are described as Rymours and Ministralx, proving that our ancestors
could not distinguish between them and our own minstrels.

\DFNdouble

In May, 1290, was celebrated the marriage of Queen Eleanor’s daughter Joan,
surnamed of Acre, to the Earl of Gloucester, and in the following July, that of
Margaret, her fifth daughter, to John, son of the Duke of Brabant. Both ceremonies
were conducted with much splendour, and a multitude of minstrels flocked
from all parts to Westminster: to the first came King Grey of England, King
Caupenny from Scotland, and Poveret, the minstrel of the Mareschal of Champagne.
The nuptials of Margaret, however, seem to have eclipsed those of her sister. 
Walter de Storton, the king’s harper, \pagebreak
distributed a hundred pounds, the gift of 
%\end{fixedpage}%028
%===============================================================================
%029
the bridegroom, among 426 minstrels, as well English as others.\dcfootnote{ %
 Pages lxix. and lxx. Introduction to Manners and
Household Expenses of England in the 13th and 15th
centuries, illustrated by original records. 4to. London.
Printed for the Roxburghe Club, 1841, and quoted from
Wardrobe Book, 18 Edward I. Rot. Miscell. in Turr.
Lond. No. 56.
} %end footnote
 In 1291, in the
accounts of the executors of Queen Eleanor, there is an entry of a payment of
39\textit{s}., for a cup purchased to be given to one of the king’s minstrels.



The highly valuable roll, preserved among the records in the custody of the
Queen’s Remembrancer, which has been printed for the Roxburghe Club, marks
the gradations of rank among the minstrels, and the corresponding rewards
bestowed upon them. It contains the names of those who attended the \textit{cour
plenière} held by King Edward at the Feast of Whitsuntide, 1306, at Westminster,
and also at the New Temple, London; because “the royal palace,
although large, was nevertheless small for the crowd of comers.” Edward then
conferred the honor of knighthood upon his son, Prince Edward, and a great
number of the young nobility and military tenants of the crown, who were summoned
to receive it, preparatory to the King’s expedition to Scotland to avenge
the murder of John Comyn, and the revolt of the Scotch.

On this occasion there were six kings of the minstrels, five of whom, viz.,
Le Roy de Champaigne, Le Roy Capenny, Le Roy Boisescue, Le Roy Marchis,
and Le Roy Robert, received each five marks, or 3\textit{l}. 6\textit{s}. 8\textit{d}., the mark being
13\textit{s}. 4\textit{d}. It is calculated that a shilling in those days was equivalent to fifteen
shillings of the present time; according to which computation, they received 50\textit{l}.
each. The sixth, Le Roy Druet, received only 2\textit{l}. The list of money \textit{given} to
minstrels is principally in Latin; but that of \textit{payments} made to them being in
Norman French, it is difficult to distinguish English minstrels from others. Le
Roy de Champaigne was probably “Poveret, the minstrel of the Mareschal of
Champagne,” of 1290, Le Roy Capenny, “King Caupenny from Scotland,” and
Le Roy Robert, whom we know to have been the English king of the minstrels
by other payments made to him by the crown (see Anstis’ Register of the Order
of the Garter, vol. ii. p.~303), was probably the “King Grey of England” of
the former date. Among the names we find, Northfolke, Carletone, Ricard de
Haleford, Adam de Werintone (Warrington?), Adam de Grimmeshawe, Merlin,
Lambyn Clay, Fairfax, Hanecocke de Blithe, Richard Wheatacre, \&c. The
harpers are generally mentioned only by their Christian names, as Laurence,
Mathew, Richard, John, Robert, and Geoffrey, but there are also Richard de
Quitacre, Richard de Leylonde, William de Grimesar, William de Duffelde, John
de Trenham, \&c., as well as Adekyn, harper to the Prince, who was probably
a Welsh bard. In these lists only the principal minstrels are named, the remaining
sum being divided, by the kings and few others, among the \textit{menestraus de la
commune}. Harpers are in the majority where the particular branch of minstrelsy
is specified. Some minstrels are locally described, as Robert “de Colecestria,”
John “de Salopia,” and Robert “de Scardeburghe;” others are distinguished
as the harpers of the Bishop of Durham, Abbot of Abyngdon, Earls of Warrenne,
Gloucester, \&c.; one is Guillaume sans manière; another, Reginald le menteur;
a third is called Makejoye; and a fourth, Perle in the eghe. 
%\end{fixedpage}%029
\pagebreak



%===============================================================================
%030
\renewcommand\versoheader{english minstrelsy.}
\renewcommand\rectoheader{edward ii.}




The total sum expended was about 200\textit{l}., which according to the preceding
estimate would be equal to about 3,000\textit{l}. of our money.

The minstrels seem to have been in many respects upon the same footing as the
heralds; and the King of the Heralds, like the King at Arms, was, both here and
on the Continent, an usual officer in the courts of princes. Heralds seem even to
have been included with minstrels in the preceding account, for Carletone, who
occupies a fair position among them, receiving 1\textit{l}. as a payment, and 5\textit{s}. as a
gratuity, is in the latter case described as Carleton “Haralde.”

In the reign of Edward II., besides other grants to “King Robert,” before
mentioned, there is one in the sixteenth year of his reign to William de Morlee,
“The king’s minstrel, styled \textit{Roy de North}” of houses that had belonged to
John le Boteler, called Roy Brunhaud. So, among heralds, \textit{Norroy} was usually
styled \textit{Roy d'Armes de North} (Anstis. ii. 300), and the Kings at Arms in general
were originally called Reges Heraldorum, as these were Reges Minstrallorum.\dcfootnote{\scriptsize %a
Heralds and minstrels seem to have been on nearly
the same footing abroad. For instance, Froissart tells us
“The same day th’ Erle of Foix gave to \textit{Heraudes} and
\textit{Minstrelles} the somme of fyve hundred frankes: and
gave to the Duke of Tourayn’s Minstrelles gowns of
Cloth of Gold, furred with Ermyns, valued at two hundred
franks.”—\textit{Chronicle Ed}. 1525, book 3, ch. 31.
} %end footnote
—\textit{Percy's Essay}.

The proverbially lengthy pedigrees of the Welsh were registered by their bards,
who were also heralds.\dcfootnote{\scriptsize %b
“The Welshman’s pedigree was his title-deed, by
which he claimed his birthright in the country. Every
one was obliged to shew his descent through nine generations,
in order to be acknowledged a free native, and by
which right he claimed his portion of land in the community. 
Among a people, where surnames were not in
use, and where the right of property depended on descent,
an attention to pedigree was indispensable. Hence arose
the second order of Bards, who were the \textit{Arwyddvierdd}, or
Bard-Heralds, whose duty it was to register arms and
pedigrees, as well as undertake the embassies of state.
The \textit{Arwyddvardd}, in early Cambrian history, was an
officer of national appointment, who, at a later period,
was succeeded by the \textit{Prydydd}, or Poet. One of these was
to attend at the birth, marriage, and death of any man of
high descent, and to enter the facts in his genealogy.
The \textit{Marwnad}, or Elegy, composed at the decease of such
a person, was required to contain truly and at length his
genealogy and descent; and to commemorate the survivor,
wife or husband, with his or her descent and progeny.
The particulars were registered in the books of the
\textit{Arwyddvardd}, and a true copy therefrom delivered to the
heir, to be placed among the authentic documents of the
family. The bard’s fee, or recompense, was a stipend
out of every plough land in the district; and he made a
triennial Bardic circuit to correct and arrange genealogical
entries.”—\textit{Extruded from Meyrick's Introduction to his
edition of Lewis Durm's Heraldic Visitations of Wales
2 vols. 4to. Llandovery}. 1846.
} %end footnote

In the reign of Edward II., \ad 1309, at the feast of the installation of Ralph,
Abbot of St. Augustin’s, at Canterbury, seventy shillings was expended on
minstrels, who accompanied their songs with the harp.— \textit{Warton}, vol. i., p.~89.

In this reign such extensive privileges were claimed by these men, and by dissolute
persons assuming their character, that it became a matter of public grievance, 
and a royal decree was issued in 1315 to put an end to it, of which the
following is an extract:—

“Edward by the grace of God, \&c. to sheriffes, \&c. greetyng, Forasmuch as\ldots\   many
idle persons, under colour of Mynstrelsie, and going in messages, and other faigned
business, have ben and yet be receaved in other mens houses to meate and drynke, and
be not therwith contented yf they be not largely consydered with gyftes of the lordes
of the houses: \&c\ldots\  We wyllyng to restrayne suche outrageous enterprises and idleness,
\&c. have ordeyned\ldots\  that to the houses of prelates, earles, and barons, none
resort to meate and drynke, unlesse he be a Mynstrel, and of these Minstrels that there
come none except it be three or four \textsc{Minstrels of honour} at the most in one day,
unlesse he be desired of the lorde of the house. \pagebreak
And to the houses of meaner men 
%\end{fixedpage}%030
%===============================================================================
%031
that none come unlesse he be desired, and that such as shall come so, holde themselves
contented with meate and drynke, and with such curtesie as the maister of the house
wyl shewe unto them of his owne good wyll, without their askyng of any thyng.
And yf any one do agaynst this Ordinaunce, at the firste tyme he to lose his \textit{Minstrelsie},
and at the second tyme to forsweare his craft, and never to be receaved for
a Minstrel in any house\ldots\  .Geven at Langley the vi. day of August, in the ix yere of
our reigne.”—\textit{Hearne’s Append. ad Leland Collect}., vol. vi., p.~36.

Stow, in his Survey of London, in an estimate of the annual expenses of
the Earl of Lancaster about this time, mentions a large disbursement for the
liveries of the minstrels. That they received vast quantities of money and costly
habiliments from the nobles, we learn from many authorities; and in a poem on
the times of Edward II., knights are recommended to adhere to their proper
costume lest they be mistaken for minstrels.

\settowidth{\versewidth}{“Kny[gh]tes schuld weare clothes}
\indentpattern{010101012003}
\begin{dcverse}
\begin{patverse}
\vleftofline{“}Kny[gh]tes schuld weare clothes\\
I-schape in dewe manere,\\
As his order wo[u]ld aske,\\
As wel as schuld a frere [friar]:\\
Now thei beth [are] disgysed,\\
So diverselych i-digt [bedight],\\
That no man may knowe\\
A mynstrel from a knyg[h]t\\
Well ny:\\
So is mekenes[s] falt adown\\
And pride aryse an hye."\\
\textit{Percy Soc., No}. 82, \textit{p}. 23.
\end{patverse}
\end{dcverse}

That minstrels were usually known by their dress, is shown by the following
anecdote, which is related by Stowe:—“When Edward II. this year (1316)
solemnized the feast of Pentecost, and sat at table in the great hall of Westminster,
attended by the peers of the realm, a certain woman, \textit{dressed in the habit
of a Minstrel, riding on a great horse, trapped in the Minstrel fashion}, entered the
hall, and going round the several tables, acting the part of a Minstrel, at length
mounted the steps to the royal table, on which she deposited a letter. Having
done this, she turned her horse, and, saluting all the company, she departed.”
The subject of this letter was a remonstrance to the king on the favors heaped
by him on his minions to the neglect of his faithful servants. The door-keepers
being called, and threatened for admitting such a woman, readily replied, “that
it never was the custom of the king’s palace to deny admission to Minstrels,
especially on such high solemnities and feast days.”

On the capital of a column in Beverley Minster, is the inscription, “Thys
pillor made the meynstyrls.” Five men are thereon represented, four in short
coats, reaching to the knee, and one with an overcoat, all having chains round
their necks and tolerably large purses. The building is assigned to the reign of
Henry VI., 1422 to 1460, when minstrelsy had greatly declined, and it cannot
therefore be considered as representing minstrels in the height of their prosperity.
They are probably only instrumental performers (with the exception, perhaps, of
the lute player); but as one holds a pipe and tabor, used only for rustic dances,
another a crowd or treble viol, a third what appears to be a bass flute, and a
fourth either a treble flute or perhaps that kind of hautboy called a wayght, or
wait, and there is no harper among them—I do not suppose any to have been of
that class called minstrels of honour, \pagebreak
who rode on horseback, with their servants 
%\end{fixedpage}%031
%===============================================================================
%32
 to attend them, and who could enter freely into a king’s palace. Such distinctions
among minstrels are frequently drawn in the old romances. For instance, in the
romance of Launfel we are told, “They had menstralles of moche honours,” and
also that they had “Fydelers, sytolyrs (citolers), and trompoteres.” It is not,
however, surprising that they should be rich enough to build a column of a
Minster, considering the excessive devotion to, and encouragement of, music which
characterised the English in that and the two following centuries.

No poets of any country make such frequent and enthusiastic mention of minstrelsy
as the English. There is scarcely an old poem but abounds with the
praises of music. Adam Davy, or Davie, of Stratford-le-Bow, near London,
flourished about 1312. In his Life of Alexander, we have several passages like~this:—
\settowidth{\versewidth}{The mynstrall synge, the jogelour carpe” (recite).}
\begin{scverse}
\vleftofline{“}Mer[r]y it is in halle to he[a]re the harpe,\\
The mynstrall synge, the jogelour carpe” (recite).
\end{scverse}
And again,—
\settowidth{\versewidth}{“Mery is the twynkelyng of the harpour.”}
\begin{scverse}
“Mery is the twynkelyng of the harpour.”
\end{scverse}
The fondness of even the most illiterate, to hear tales and rhymes; is much
dwelt on by Robert de Brunne, or Robert Mannyng, “the first of our vernacular
poets who is at all readable now.” All rhymes were then sung with accompaniment, 
and generally to the harp. So in 1338, when Adam de Orleton, bishop of
Winchester, visited his Cathedral Priory of St. Swithin, in that city, a minstrel
named Herbert was introduced, who sang the \textit{Song of Colbrond}, a Danish Giant,
and the tale of \textit{Queen Emma delivered from the plough-shares}, or trial by fire, in
the hall of the Prior. A similar festival was held in this Priory in 1374, when
similar gestes or tales were sung. Chaucer’s Troilus and Cresseide, though almost
as long as the Æneid, was to be “redde, or else songe,” and Warton has printed
a portion of the Life of St. Swithin from a manuscript, with points and accents
inserted, both over the words and dividing the line, evidently for the purposes of
singing or recitation (\textit{History of English Poetry}, vol. i., p.~15. 1840). We have
probably by far more tunes that are fitted for the recitation of such lengthy stories
than exist in any other country.

In the year 1362, an Act of Parliament passed, that “all pleas in the court
of the king, or of any other lord, shall be pleaded and adjudged in the English
tongue” (stat. 36 Edw. III., cap.~15); and the reason, which is recited in the
preamble, was, that the French tongue was so unknown in England that the
parties to the law-suits had no knowledge or understanding of what was said for
or against them, because the counsel spoke French. This was the era of Chaucer,
and of the author of Pierce Plowman—two poets whose language is as different as
if they had been born a century apart. Longland, instead of availing himself of the
rising and rapid improvements of the English language, prefers and adopts the style
of the Anglo-Saxon poets, even prefering their perpetual alliteration to rhyme.
His subject—a satire on the vices of the age, but particularly on the corruptions
of the clergy and the absurdities of superstition—does not lead him to say much
of music, but he speaks of ignorance of the art as a just subject of reproach.
\settowidth{\versewidth}{“They kennen [know] no more mynstralcy, ne musik, men to gladde,}
\begin{scverse}
\vleftofline{“}They kennen [know] no more mynstralcy, ne musik, men to gladde,\\
Than Mundy the muller [miller], of \textit{multa fecit Deus}!’’ 
\end{scverse}

\pagebreak


%===============================================================================
%033

 

\renewcommand\rectoheader{pierce plowman.--chaucer.}
\noindent He says, however, of himself, in allusion to the minstrels:—

\settowidth{\versewidth}{Ich can nat tabre, ne trompe, ne telle faire gestes,”}
\begin{scverse}\vleftofline{“}Ich can nat tabre, ne trompe, ne telle faire gestes,\\
Ne fithelyn, at fe[a]stes, ne harpen:\\
Japen ne jagelyn, ne gentilliche pipe;\\
Nother sailen [leap or dance], ne sautrien, ne singe with the giterne.”
\end{scverse}
He also describes his Friar as much better acquainted with the “\textit{Rimes of
Robinhode} and of \textit{Randal, erle of Chester},” than with his Paternoster.

Chaucer, throughout his works, never loses an opportunity of describing or
alluding to the general use of music, and of bestowing it as an accomplishment
upon the pilgrims, heroes, and heroines of his several tales or poems, whenever
propriety admits. We may learn as much from Chaucer of the music of his day,
and of the estimation in which the art was then held in England, as if a treatise
had been written on the subject.

Firstly, from the Canterbury Tales, in his description of the Squire (line 91
to~96), he says:—

\begin{scverse}\vleftofline{“}\textit{Syngynge he mas, or flowtynge} [fluting] \textit{al the day};\\
He was as fresh as is the moneth of May:\\
Short was his gonne, with sleevès long and wyde;\\
Well cowde he sitte on hors, and faire ryde.\\
\textit{He cowde songes wel make and endite,}\\
Juste (fence) and eke daunce, and wel p[o]urtray and write.”
\end{scverse}

Of the Nun, a Prioress (line 122 to 126), he says:—
\begin{scverse}\vleftofline{“}\textit{Ful wel sche sang the servise devyne},\\
\textit{Entuned in hire nose ful seemyly};\\
And Frensch sche spak ful faire and fetysly [neatly],\\
Aftur the schole of Stratford attè Bowe,\\
For Frensch of Parys was to hire unknowe” [unknown].
\end{scverse}

The Monk, a jolly fellow, and great sportsman, seems to have had a passion for
no music but that of hounds, and the bells on his horse’s bridle (line 169 to~171):
\begin{scverse}\vleftofline{“}And whan he rood [rode], men might his bridel heere\\
Gyngle in a whistlyng wynd so cleere,\\
And eke as lowde as doth the chapel belle.”
\end{scverse}

Of his Mendicant Friar, whose study was only to please (lines 235-270),
he says:
\begin{scverse}\vleftofline{“}And certayn he hadde a mery note;\\
\textit{Wel couthe he synge and playe on a rote }[hurdy-gurdy]\ldots\  \\
Somewhat he lipsede [lisped] for wantounesse,\\
To make his Englissch swete upon his tunge;\\
And \textit{in his harpyng, whan that he had sunge},\\
His eyghen [eyes] twynkeled in his he[a]d aright,\\
As don the sterrès [do the stars] in the frosty night.”
\end{scverse}

Of the Miller (line 564 to 568), he says:—
%\settowidth{\versewidth}{Was never trompe [trumpet] of half so gre[a]t a soun” (sound).}
\begin{scverse}\vleftofline{“}Wel cowde he ste[a]le corn, and tollen thries [take toll thrice];\\
And yet he had a thombe of gold,\footnotemark\ pardé,\\
A whight cote and blewe hood we[a]red he; \\
\end{scverse}

\footnotetext{ %
Tyrwhitt says there is an old proverb—“Every \textit{honest}
miller has a thumb of gold.” Perhaps it means that
nevertheless he was as honest as his brethren. There are
many early songs on thievish millers and bakers.
} %end footnotetext


\pagebreak
%===============================================================================
%34
\settowidth{\versewidth}{\textit{A baggepipe cowde he blowe and sowne} [sound],}
\begin{scverse}
\textit{A baggepipe cowde he blowe and sowne} [sound],\\
And therewithal he brought us out of towne.”\dcfootnote{ %a
A curious reason for the use of the Bagpipe in Pilgrimages
will he found in State Trials—Trial of William
Thorpe. Henry IV., an. 8, shortly after Chaucer’s death.
“I say to thee that it is right well done, that Pilgremys
have with them both Syngers, and also Pipers, that whan
one of them, that goeth bar[e]fo[o]te, striketh his too upon
a stone, and hurteth hym sore, and maketh hym to blede;
it is well done that he or his fel[l]ow begyn than a Songe,
or else take out of his bosome a \textit{Baggepype} for to drive
away with soche myrthe the hurte of his felow.”
} %end footnote
\end{scverse}
Of the Pardoner (line 674 to 676):—
\begin{scverse}\vleftofline{“}\textit{Ful lowde he sang, ‘Come hider, love, to me}.’\\
This Sompnour bar[e] to him a \textit{stif burdoun},\footnotemark \\
Was never trompe [trumpet] of half so gre[a]t a soun” (sound).
\end{scverse}
\footnotetext{ %b
This Sompnour (Sumner or Summoner to the Ecclesiastical
Courts, now called Apparitor) supported him by
singing the \textit{burden}, or \textit{bass}, to his song in a deep loud
voice. \textit{Bourdon} is the French for \textit{Drone}; and \textit{Foot},
\textit{Under-song}, and \textit{Burden} mean the same thing, although
Burden was afterwards used in the sense of Ditty, or
any line often recurring in a song, as will be seen here-
after.
} %end footnotetext
Of the poor scholar, Nicholas (line 3213 to 3219):—
\begin{scverse}\vleftofline{“}And al above ther lay a gay \textit{sawtrye} [psaltry],\\
On which he made, a-nightes, melodye\\
So swetely, that al the chambur rang:\\
And \textit{Angelus ad Virginem} he sang.\\
And after that he sang \textit{The Kynge’s note};\\
Ful often blessed was his mery throte.”
\end{scverse}
Of the Carpenter’s Wife (lines 3257 and 8):--
\begin{scverse}\vleftofline{“}But of her song, it was as lowde and yerne [brisk]\\
As eny swalwe [swallow] chiteryng on a berne” [barn].
\end{scverse}
Of the Parish Clerk, Absolon (lines 3328 to 3335):—
\begin{scverse}\vleftofline{“}In twenty manners he coude skip and daunce,\\
After the schole of Oxenfordè tho,\\
And with his leggès casten to and fro;\\
\textit{And pleyen songes on a small Rubible}\dcfootnote{Ribible (the diminutive of Ribibe or Rebec) is a small
fiddle with three strings.} [Rebec],\\
\textit{Ther-to he sang som tyme a lowde quynyble};\footnotemark\\
\textit{And as wel coude he pleye on a giterne}:\\
In al the toun nas [nor was] brewhous ne taverne\\
That he ne visited with his solas” [solace].
\end{scverse}
\footnotetext{ %d
To sing a “quinible” means to descant by singing
fifths on a plain-song, and to sing a “quatrible” to descant
by fourths. The latter term is used by Cornish in
his Treatise between Trowthe and Enformacion. 1528.
} %end footnotetext
He serenades the Carpenter’s Wife, and we have part of his song (lines 3352—64):%
\begin{scverse}\vleftofline{“}The moone at night ful cleer and brightè schoon,\\
And Absolon his giterne hath i-take,\\
For paramours he seyde he wold awake\ldots\  \\
He syngeth in hys voys gentil and smal—\\
Now, deere lady, if thi wille be,\\
I pray you that ye wol rewe [have compassion] on me.’\\
Full wel acordyng to his gyternyng,\\
This carpenter awook, and herde him syng.”
\end{scverse}
Of the Apprentice in the Cook’s Tale, who plays both on the ribible and gitterne:
\begin{scverse}\vleftofline{“}At every brideale wold he synge and hoppe;\\
He loved bet [better] the taverne than the schoppe.” 
\end{scverse}


\pagebreak
%===============================================================================
%35
\renewcommand\rectoheader{notices of music by chaucer.}
\settowidth{\versewidth}{\ldots\  . “some, for they can synge and daunce,}

The Wife of Bath says (lines 5481 and 2, and 6039 and 40), that wives were
chosen—
%\vspace{-\baselineskip}
\begin{scverse}\ldots\  . “	some, for they \textit{can synge and daunce},\\
And some for gentilesse or daliaunce\ldots\  \\
How couthe I dannce to an harpe smale,\\
And synge y-wys as eny nightyngale.”
\end{scverse}
I shall conclude Chaucer’s inimitable descriptions of character with that of his
Oxford Clerk, who was so fond of books and study, that he loved Aristotle better
%\vspace{-\baselineskip}
\begin{scverse}\vleftofline{“}Than robès riche, or fidel or sautrie\ldots\  \\
Souning in moral virtue was his speech,\\
And gladly would he lerne and gladly teche.”
\end{scverse}
We learn from the preceding quotations, that country squires in the fourteenth
century could pass the day in singing, or playing the flute, and that some could
“Songès well make and indite:” that the most attractive accomplishment in
a young lady was to be able to sing well, and that it afforded the best chance of
her obtaining an eligible husband; also that the cultivation of music extended
to every class. The Miller, of whose education Pierce Plowman speaks so slightingly,
could play upon the bagpipe; and the apprentice both on the ribible and
gittern. The musical instruments that have been named are the harp, psaltry,
fiddle, bagpipe, flute, trumpet, rote, rebec, and gittern. There remain the lute,
organ, shalm (or shawm), and citole, the hautboy (or wayte), the horn, and
shepherd’s pipe, and the catalogue will be nearly complete, for the cittern or
cithren differed chiefly from the gittern, in being strung with wire instead of gut,
or other material. The sackbut was a bass trumpet with a slide,\dcfootnote{\scriptsize %a
“As he that plaies upon a Sagbut, by pulling it up
and down alters his tones and tunes.”—\textit{Burton’s Anatomy
of Melancholy}, 8vo. Edit, of 1800, p.~379.
} %end footnote 
like the modern
trombone; and the dulcimer differed chiefly from the psaltry in the wires being
struck, instead of being twitted by a plectrum, or quill, and therefore requiring
both hands to perform on it.

In the commencement of the Pardoner’s Tale he mentions lutes, harps, and
gitterns for dancing, as well as singers with harps; in the Knight’s Tale he represents
Venus with a citole in her right hand, and the organ is alluded to both in
the History of St. Cecilia, and in the tale of the Cock and the Fox.

In the House of Fame (Urry’s Edit., line 127 to 136), he says:
\settowidth{\versewidth}{How that he dorstè not his sorwe [sorrow] telle,}
\begin{scverse}
\vleftofline{“}That madin loude Minstralsies\\
In Cornmuse [bagpipe] and eke in \textit{Shalmies},\dcfootnote{\scriptsize %b
A very early 1414 drawing of the Shalm, or Shawm, is in
one of the illustrations to a copy of Froissart, in the Brit.
Mus.—\textit{Royal} MSS. 18, E. Another in Commenius’
Visible World, translated by Hoole, 1650, (he translates
the Latin word \textit{gingras}, shawm,) from which it is copied
into Cavendish’s Life of Wolsey, edited by Singer, vol. i.
p.~114., Ed. 1825. The modern clarionet is an improvement
upon the shawm, which was played with a reed,
like the wayte, or hautboy, but being a bass instrument,
with about the compass of an octave, had probably more
the tone of a bassoon. It was used on occasions of state.
“What \textit{stately} music have you? You have shawms?
Ralph plays a stately part, and he must needs have
shawms.”—\textit{Knight of the Burning Pestle}. Drayton speaks
of it as shrill-toned: “E’en from the \textit{shrillest shawm}, unto
the cornamute."—\textit{Polyolbion}, vol. iv., p.~376. I conceive
the shrillness to have arisen from over-blowing, or else
the following quotation will appear contradictory:—
\settowidth{\versewidth}{It mountithe not to hye, but \textit{kepithe rule and space}.}
\begin{fnverse}
“A \textit{Shawme} maketh a \textit{swete} sounde, \\
\hspace{\vgap} for he \textit{tunythe the basse},\\
It mountithe not to hye, but \textit{kepithe rule and space}.\\
Yet \textit{yf it be blowne withe to vehement a wynde,}\\
It \textit{makithe it to mysgoverne out of his kynde}.”
\end{fnverse}
This is one of the “proverbis” that were written about
the time of Henry VII., on the walls of the Manor House
at Leckingfield, near Beverley, Yorkshire, anciently belonging
to the Percys, Earls of Northumberland, but uow
destroyed. There were many others relating to music,
and musical instruments (harp, lute, recorder, claricorde,
clarysymballis, virgynalls, clarion, organ, singing, and
musical notation,) and the inscribing them on the walls
adds another to the numberless proofs of the estimation
in which the art was held. A manuscript copy of them
is preserved in Bib, Reg. 18. D. 11. Brit. Mus.
}  \\
\end{scverse}


\pagebreak
%===============================================================================
%36
\settowidth{\versewidth}{That bin at feastes with the brede [bread]:}
\begin{scverse}
And in many an othir pipe,\\
That craftely began to pipe\\
Bothe in \textit{Douced} and eke in \textit{Rede},\dcfootnote{ %a
Tyrwhitt thinks \textit{Doucete} an Instrument, and quotes
Lydgate—
\settowidth{\versewidth}{“Ther were trumpes and trumpetes,}
\begin{fnverse}
\vleftofline{“}Ther were trumpes and trumpetes,\\
Lowde shall[m]ys and doucetes.”
\end{fnverse}
but it seems to me only to mean soft pipes in opposition
to loud shalms. By the distinction Chaucer draws, “both
in douced and in reed” (the shalm being played on by
a reed), I infer by “douced” that flutes are intended; the
tone of which, especially the large flute, is extremely soft.
I had a collection of English flutes, of which one was
nearly a yard and a half long. All had mouth-pieces like
the flageolet, and were blown in the same manner; the
tone very pleasing, but less powerful and brilliant than
the modern or “German” flute.
} \\
That bin at feastes with the brede [bread]:\\
And many a \textit{Floite} and litlyng \textit{Horne}\\
And \textit{Pipes made of grenè corne}.\\
As have these little Herdègroomes\\
That kepin Beastes [keep oxen] in the broomes.”
\end{scverse}

As to the songs of his time, see the Frankeleyne’s Tale (line 11,254 to 60):—
\settowidth{\versewidth}{Sauf [save] in his songès somewhat wolde he wreye [betray]}
\begin{scverse}
\vleftofline{“}He was dispeired, nothing dorst he seye\\
Sauf [save] in his songès somewhat wolde he wreye [betray]\\
His woo, as in a general compleyning;\\
He said he loved, and was beloved nothing.\\
Of suche matier made he many \textit{Layes},\\
\textit{Songes, Compleyntes, Roundelets, Virelayes}:\\
How that he dorstè not his sorwe [sorrow] telle,\\
But languisheth as doth a fuyr in helle.”
\end{scverse}
and he speaks elsewhere of \textit{Ditees, Rondils, Balades}, \&c.

The following passages relate to minstrelsy, and to the manner of playing the
harp, pointing and performing with the nails, as the Spaniards do now with the
guitar. The first is from the House of Fame (Urry, line 105 to 112):—
\settowidth{\versewidth}{\ldots\  “Stoden\ldots\  the castell all aboutin}
\begin{scverse}
\ldots\  “Stoden\ldots\  the castell all aboutin\\
Of all manir of Minstralis\\
And gestours that tellen tales\\
Both of wepyng and of game,\\
And all that ’longeth unto fame;\\
There herde I playin on an \textit{Harpe}\\
\textit{That ysounid bothe well and sharpe}”
\end{scverse}
and from Troylus, lib. 2, 1030:—
\settowidth{\versewidth}{Touch aie o (one) string, or aie o warble harpe,}
\begin{scverse}
“For though that the best barper upon live\\
Would on the bestè sounid jolly harpe\\
That evir was, with all his fingers five\\
Touch aie o (one) string, or aie o warble harpe,\\
\textit{Were his nailes poincted nevir so sharpe}\\
It shoulde makin every wight to[o] dull\\
To heare [h]is Glee, and of his strokes ful.”
\end{scverse}

Even the musical gamut is mentioned by Chaucer. In the supplementary tale
he makes the host give “an hid[e]ouse cry in ge-sol-re-ut the haut,” and there is
scarcely a subject connected with the art as practised in his day, that may not be
illustrated by quotation from his works;
\begin{scverse}
“For, gif he have nought sayd hem, leeve [dear] brother,\\
In o bo[o]k, he hath seyd hem in another.” 
\end{scverse}


\pagebreak

%===============================================================================
%037
\renewcommand\rectoheader{gower.--richard ii.}
I shall conclude these numerous extracts with one of the song of nature, from
the Knighte’s Tale, (line 1493 to 98):—

\settowidth{\versewidth}{And with his stremès dryeth in the greves [groves]}
\begin{scverse}
\vleftofline{“}The busy larkè, messager of daye,\\
Salueth in hire song the morwe [morning] gray;\\
And fyry Phebus ryseth up so bright,\\
That al the orient laugheth of the light,\\
And with his stremès dryeth in the greves [groves]\\
The silver dropès, hongyng on the leeves.”
\end{scverse}
Having quoted so largely from Chaucer, whose portraiture of character and
persons has never been excelled, it will be unnecessary to refer to his contemporary, Gower, further than to say that in his \textit{Confessio Amantis}, Venus greets
Chaucer as her disciple and poet, who had filled the land in his youth with
dittees and “songès glade,” which he had made for her sake; and Gower says of
himself:—
\settowidth{\versewidth}{For her on whom myn hert laie.”}
\begin{scverse}
\vleftofline{“}And also I have ofte assaide\\
Roundel, Balades, and Virelaie\\
For her on whom myn hert laie.”
\end{scverse}
But about the same time, in the Burlesque Romance, The T[o]urnament of
Tottenham (written in ridicule of chivalry), we find a notice of songs in six parts
which demands attention. In the last verse:—
\settowidth{\versewidth}{Mekyl mirth was them among;}
\begin{scverse}
\vleftofline{“}Mekyl mirth was them among;\\
In every corner of the hous\\
Was melody delycyous\\
For to he[a]re precyus\\
\vin\vin Of six menys song.”
\end{scverse}

It has been supposed that this is an allusion to \textit{Sumer is icumen in}, which
requires six performers, but in all probability there were many such songs,
although but one of so early a date has descended to us. We find in the Statutes
of New College, Oxford (which was founded about 1380), that William of
Wykeham ordered his scholars to recreate themselves on festival days with songs
in the hall, both after dinner and supper; and as part-music was then in common
use, it is reasonable to suppose that the founder intended the students thereby to
combine improvement and recreation, instead of each singing a different song.

In the fourth year of king Richard II. (1381), John of Gaunt erected at
Tutbury, in Staffordshire, a \textit{Court of Minstrels} similar to that annually kept at
Chester; and which, like a court-leet, or court-baron, had a legal jurisdiction,
with full power to receive suit and service from the men of this profession within
five neighbouring counties, to determine their controversies and enact laws; also
to apprehend and arrest such of them as should refuse to appear at the said court,
annually held on the 16th of August. For this they had a charter, by which
they were empowered to appoint a King of the Minstrels, with four officers to
preside over them. They were every year elected with great ceremony; the
whole form of which, as observed in 1680, is described by Dr.~Plot in his History
of Staffordshire. That the barbarous diversion of bull-running was no part of the
original institution, is fully proved by the Rev. Dr.~Pegge, in Archæologia, vol. ii.,
No. xiii., p.~86. The bull-running tune, however, is still popular in Staffordshire. 

\pagebreak

%===============================================================================
%038
Du Fresne in his Glossary (art. Ministrelli), speaking of the King of the
Minstrels, says, “His office and power are defined in a French charter of
Henry~IV., king of England, in the Monasticon Anglicanum, vol. i., p.~355;”
but though I have searched through Dugdale’s Monasticon, I find no such
charter.


In 1402, we find the before-mentioned statute against the Welsh bards,%\linebreak
(4 Henry~IV., c.~27).\dcfootnote{ %a
It runs in these terms: “Item, pour eschuir plusieurs
diseases et mischiefs qont advenuz devaunt ces heures en
la terre de Gales par plusieurs Westours Rymours,
Minstralx et autres Vacabondes, ordeignez est, et
establiz, que nul Westour, Rymour Minstral, ne Vacabond
soit aucunemeut sustenuz en la terre de Gales pur
faire kymorthas ou coillage sur la commune poeple
iliocques.”
} %end footnote
As they had excited their countrymen to rebellion
against the English government, it is not to he wondered (says Percy) that the
Act is conceived in terms of the utmost indignation and contempt against this
class of men, who are described as Rymours, Ministralx, which are apparently
here used as only synonymous terms to express the Welsh bards, with the usual
exuberance of our Acts of Parliament; for if their Ministralx had been mere
musicians, they would not have required the vigilance of the English legislature
to suppress them. It was their songs, exciting their countrymen to insurrection,
which produced “les diseases et mischiefs en la terre de Gales.”

At the coronation of Henry V., which took place in Westminster Hall (1413),
we are told by Thomas de Elmham, that “the number of harpers was exceedingly
great; and that the sweet strings of their harps soothed the souls of the guests
by their soft melody.” He also speaks of the dulcet sounds of the united
music of other instruments, in which no discord interrupted the harmony,
as “inviting the royal banqueters to the full enjoyment of the festival.”—
(Vit. et. Gest. Henr. V., c. 12, p.~23.) Minstrelsy seems still to have
flourished in England, although it had declined so greatly abroad; the Provençals
had ceased writing during the preceding century. When Henry was preparing
for his great voyage to France in 1415, an express order was given for his
minstrels to attend him.—(Rymer, ix.,~255.) Monstrelet speaks of the English
camp resounding with the national music (170) the day preceding the battle of
Agincourt, but this must have been before the king “gave the order for silence,
which was afterwards strictly observed.”

When he entered the City of London in triumph after the battle, the gates and
streets were hung-with tapestry representing the histories of ancient heroes; and
boys with pleasing voices were placed in artificial turrets, singing verses in his
praise. But Henry ordered this part of the pageantry to cease, and commanded
that for the future no “ditties should be made and sung by Minstrels\dcfootnote{ %b
Hollinshed, quoting from Thomas de Elmham, whose
words are, “Quod cantus de suo triumpho fieri, sen per
\textit{Citharistas} vel alios quoscunque cantari penitus prohibebat.”
It will be observed that Hollinshed translates
Citharistas (literally harpers) minstrels.
} or others,”
in praise of the recent victory; “for that he would whollie have the praise
and thankes altogether given to God.”

Nevertheless, among many others, a minstrel-piece soon appeared on the
\textit{Seyge of Harflett} (Harfleur), and the \textit{Battayle of Agynkourte}, “evidently,” says
Warton, “adapted to the harp,” and of 
which he has printed some portions. 
\pagebreak
%===============================================================================
%39
(Hist. Eng. Poet., vol. ii. p.~257.) Also the following song, which Percy has
printed in his Reliques of Ancient Poetry, from a M.S. in the Pepysian Library,
and Stafford Smith, in his Collection of English Songs, 1779 fol., in fac-simile of
the old notation, as well as in modern score, and with a chorus in three parts to
the words, “Deo gratias, Anglia, redde pro victoria.” The tune is here given
with the first verse of the words,\dcfootnote{ %a
I do not intend to reprint songs or Ballads that are
contained in Percy's Reliques of Ancient Poetry, without
some particular motive, for that delightful book can be
purchased in many shapes and at a small cost. As a
general rule, the versions given by Percy are best suited
to music, because more metrical than others, although
they maybe less exactly and minutely in accordance
with old copies, which are often very carelessly printed
or transcribed.
} %end footnote
for although the original is a regular composition
in three parts, it serves to shew the state of melody at an early period, and
the subject is certainly a national one.

\renewcommand\rectoheader{henry v.}

\musictitle{Song on the Victory of Agincourt.}

\musicinfo{Slowly and Majestically.}{1415.}

\lilypondfile[staffsize=14]{lilypond/039-song-on-the-victory-of-agincourt}

There are also two well-known ballads on the Battle of Agincourt; the one
commencing “A council grave our king did hold;” the other “As our king lay
musing in his bed,” which will be noticed under later dates; and a three-men’s
song, which was sung by the tanner and his fellows, to amuse the guests, in
Heywood’s play, \textit{King Edward IV}., beginning—

\settowidth{\versewidth}{Agincourt! Agincourt! know ye not Agincourt?}
\begin{scverse}\vleftofline{“}Agincourt! Agincourt! know ye not Agincourt?\\
Where the English slew or hurt\\
\vin All the French foemen?” \&c.
\end{scverse}

Although Henry had forbidden the minstrels to celebrate his victory, the order
evidently did not proceed from any disregard for the professors of music or of
song, for at the Feast of Pentecost, which he celebrated in 1416, having the
Emperor and the Duke of Holland as his guests, he ordered rich gowns for sixteen 
of his minstrels. And having before his \pagebreak 
 death orally granted an annuity of an 
%===============================================================================
%40
hundred shillings to each of his minstrels, the grant was confirmed in the first
year of his son, Henry VI. (\ad 1423), and payment ordered out of the exchequer.
Both the biographers of Henry declare his love for music.\dcfootnote{ %a
“Musicis delectabatur.”—Tit. Liv., p. 5. “Instr\-umentis
organicis pluri\-mum deditus”—Elmham.
} %end footnote
Lydgate
and Occleve, the poets whom he patronized, attest also his love of literature, and
the encouragement he gave to it.

John Lydgate, Monk of Bury St. Edmunds, describes the minstrelsy of his
time less completely, but in nearly the same terms as Chaucer.

Lydgate was a very voluminous writer. Ritson enumerates 251 of his pieces,
and the list is far from being complete. Among his minor pieces are many songs
and ballads, chiefly satirical, such as “On the forked head-dresses of the ladies,”
on “Thievish Millers and Bakers,” \&c. A selection from these has been recently
printed by the Percy Society.

Among the devices at the coronation banquet of Henry VI. (1429), were, in
the first course, a “sotiltie” (subtlety) of St. Edward and St. Lewis, in coat
armour, holding between them a figure like King Henry, similarly armed, and
standing with a \textit{ballad under his feet}. “In the second, a device of the Emperor
Sigismund and King Henry V., arrayed in mantles of garter, and a figure like
Henry VI. kneeling before them with \textit{a ballad against the Lollards};\dcfootnote{ %b
Ritson has printed one of these ballads against the
Lollards, in his Ancient Songs, p.~63, 1790, taken, from
\textit{MS. Cotton, Vespasian, B. 16. Brit. Mus}.
} %end footnote
and in the
third, one of our Lady, sitting with her child in her lap, and holding a crown in
her hand, St. George and St. Denis kneeling on either side, presenting to her
King Henry \textit{with a ballad in his hand}.\dcfootnote{ %c
Quoted by Sharon Turner, from Fab. 419.
} %end footnote
These subtleties were probably devised
by the clergy, who strove to smother the odium which, as a body, their vices had
excited, by turning public attention to the further persecution of the Lollards.\dcfootnote{ %
Sir John Oldcastle, Lord Cohham, bad been put to
death in the preceding reign.
} %end footnote
In a discourse which was prepared to be delivered at the Convocation of the
Clergy, ten days after the death of Edward IV., and which still exists in MS.
(MS. Cotton Cleopatra, E. 3), exhorting the clergy to amendment, the writer
complains that “The people laugh at us, and make us their songs all the day
long.” Vicious persons of every description had been induced to enter the church
on account of the protection it afforded against the secular power, and the facilities
it provided for continued indulgence in their vices.

In that age, as in more enlightened times, the people loved better to be pleased
than instructed, and the minstrels were often more amply paid than the clergy.
During many of the years of Henry VI., particularly in the year 1430, at the
annual feast of the fraternity of the \textsc{Holie Crosse}, at Abingdon, a town in
Berkshire, twelve priests each received four pence for singing a dirge: and the
same number of minstrels were rewarded each with two shillings and four pence,
besides diet and horse-meat. Some of these minstrels came only from Maydenhithe, 
or Maidenhead, a town at no great distance, in the same county. (Liber
Niger, p.~598.) In the year 1441, eight priests were hired from Coventry,
to assist in celebrating a yearly obit in the church \pagebreak 
of the neighbouring priory of 
%===============================================================================
%41
Maxtoke; as were six minstrels (\textsc{Mimi}) belonging to the family of Lord Clinton,
who lived in the adjoining Castle of Maxtoke, to sing, harp, and play in the hall
of the monastery, during the extraordinary refection allowed to the monks on that
anniversary. Two shillings were given to the priests, and four to the minstrels:
and the latter are said to have supped in \textit{camera picta}, or the painted chamber of
the convent, with the sub-prior, on which occasion the chamberlain furnished
eight massive tapers of wax. (Warton, vol. ii., p.~309.) However, on this occasion,
the priests seem to have been better paid than usual, for in the same year
(1441) the prior gave no more than sixpence to a preaching friar.

\renewcommand\rectoheader{henry vi.}

As late as in the early part of the reign of Elizabeth, we find an entry in the
books of the Stationers’ Company (1560) of a similar character: Item, payd to
the preacher, 6\textit{s}. 2\textit{d}. Item, payd to the minstrell, 12\textit{s}.; so that even in the
decline of minstrelsy, the scale of remuneration was relatively the same.

A curious collection of the songs and Christmas carols of this reign (Henry~VI.) 
have been printed recently by the Percy Society. (Songs and Carols, No. 73.)

The manuscript book from which they are taken, had, in all probability, belonged
to a country minstrel who sang at festivals and merry makings, and it has been,
most judiciously, printed entire, as giving a general view of the classes of poetry
then popular. A proportion of its contents consists of carols and religious songs,
such as were sung at Christmas, and perhaps at other festivals of the Church.
Another class, in which the MS. is, for its date, peculiarly rich, consists of
drinking songs. It also contains a number of those satirical songs against the
fair sex, and especially against shrews, which were so common in the middle ages,
and have a certain degree of importance as showing the condition of private
society among our forefathers. The larger number of the songs, including some
of the most interesting and curious, appear to be unique, and the others
are in general much better and more complete copies than those previously
known (viz. in MS. Sloane, No. 2593, Brit. Mus). The editor of the MS.
(Mr. T. Wright) observes that “The great variations in the different copies of
the same song, show that they were taken down from oral recitation, and had
often been preserved by memory among minstrels, who were not unskilful at
composing, and who were not only in the habit of, voluntarily or involuntarily,
modifying the songs as they passed through their hands, and adding or omitting
stanzas, but of making up new songs by stringing together phrases and lines, and
even whole stanzas from the different compositions which were imprinted on their
memories.” But what renders the manuscript peculiarly interesting, is, that it
contains the melodies of some of the songs as well as the words. From this it
appears that the same tune was used for different words. At page 62 is a note,
which in modern spelling is as follows: “This is the tune for the song following;
if so be that ye will have another tune, it may be at your pleasure, for I have set
all the song.” The words of the carol, “Nowell, Nowell,” (Noel) are written
under the notes, but the wassail song that follows, and for which the tune was also
intended, is of a very opposite character, “Bryng us in good ale.” I have
printed the first verse of each under the tune, but it requires to be sung more
quickly for the wassail song than for the carol. 

\pagebreak
%===============================================================================
%042
% \noindent\begin{minipage}{\textwidth}
%\musictitle{Christmas Carol.\textsuperscript{a}}
%\musicinfo{The Burden or Chorus}{About 1460.}
%\vspace{\baselineskip}
%

\musictitle{christmas carol.}
\musicinfo{The Burden or Chorus}{About 1460.}
\bigskip

%\fbox{
\begin{picture}(400pt,120pt)(0,0)
\drawline(205,0)(205,120)(380,120,)(380,0)
%\drawline(0,100)(65,100)(65,10)
\end{picture}
%} %end fbox
\vspace{-120pt}


\lilypondfile[staffsize=16]{lilypond/042-christmas-carol}

\footnotetext[1]{The two bars marked off by a line are added, because
there would not otherwise be music enough for the \textit{Wassail
Song}. They are a mere repetition of the preceding,
and can be omitted at pleasure. The only way in
which the latter could have been sung to the music as
written in the manuscript, would be by omitting the line
“And bring us in good ale;” but, as it is \textit{merely} a repetition,
it \textit{could} be omitted.}


\pagebreak
%===============================================================================
%043
\renewcommand\rectoheader{christmas carol and wassail song.}
The notation of the original is in semibreves, minims, and crotchets, which
are diminished to crotchets, quavers, and semiquavers, as became necessary in
modernizing the notation; for the quickest note then in use was the crotchet.\dcfootnote{ %a
After the Percy Society had printed the Songs, I was
to have had the opportunity of transcribing \textit{all} the Music;
but, in the mean time, the bookbinder to whom this rare
MS. was entrusted, disappeared, and with him the manuscript, 
which is, perhaps, already in some library in the
United States.
} %end footnote
The Christmas carol partakes so much of the character of sacred music, that it is
not surprising it should be in an old scale. If there were not the flat at the signature,
which takes off a little of the barbarity, it would be exactly in the eighth
Gregorian tone.

There are seven verses to the carol, but as they are not particularly interesting,
perhaps the words of the wassail song will be preferred, although we should not
now sing of “our blessed lady,” as was common in those days.
\settowidth{\versewidth}{Bring us in no brown bread, for that is made of bran,}
\begin{scverse}
Bring us in no brown bread, for that is made of bran,\\
Nor bring us in no white bread, for therein is no gain,\\
\hspace{\vgap}But bring us in good ale, and bring us in good ale;\\
\hspace{\vgap}For our blessed Lady’s sake, bring us in good ale.

Bring us in no beef, for there is many bones,\\
But bring us in good ale, for that go’th down at once. And bring, \&c.

Bring us in no bacon, for that is passing fat,\\
But bring us in good ale, and give us enough of that. And bring, \&c.

Bring us in no mutton, for that is passing lean,\\
Nor bring us in no tripes, for they be seldom clean. But bring, \&c.

Bring us in no eggs, for there are many shells,\\
But bring us in good ale, and give us nothing else. But bring, \&c.

Bring us in no butter, for therein are many hairs,\\
Nor bring us in no pig’s flesh, for that will make us bears. But bring, \&c.

Bring us in no puddings, for therein is all God’s good,\\
Nor bring us in no venison, that is not for our blood. But bring, \&c.

Bring us in no capon’s flesh, for that is often dear,\\
Nor bring us in no duck’s flesh, for they slobber in the mere, [mire]\\
\hspace{\vgap}But bring us in good ale, and bring us in good ale,\\
\hspace{\vgap}For our blessed lady’s sake, bring us in good ale.
\end{scverse}

An inferior copy of this song, without music, is in Harl. M.S., No. 541, from
which it has been printed in Ritson’s Ancient Songs, p. xxxiv. and xxxv.

With the reign of Edward IV. we may conclude the history of the \textit{old wandering}
minstrel. In 1469, on a complaint that persons had collected money in different
parts of the kingdom by assuming the title and livery of the king’s minstrels, he
granted to Walter Halliday, \textit{Marshal}, and to seven others whom he names,
a charter of incorporation. They were to be governed by a marshal appointed for
life, and two wardens to be chosen annually, who were authorized to admit members; 
also to examine the pretensions of all who exercised the minstrel profession, \pagebreak
and to regulate, govern, and punish them 
throughout the realm (those of Chester 
%===============================================================================
%44
excepted). “This,” says Percy, “seems to have some resemblance to the Earl
Marshal’s court among the heralds, and is another proof of the great affinity and
resemblance which the minstrels bore to the College of Arms.” Walter Halliday,
above mentioned, had been retained in the service of the two preceding monarchs,
and Edward had granted him an annuity of ten marks for life, in~1464.

%\changefontsize{1.06\defaultfontsize}
In this reign we find also mention of a \textit{Serjeant} of the minstrels, who upon
one occasion did his royal master a singular service, and by which his ready access
to the king at all hours is very apparent: for “as he [K. Edward IV.] was in
the north contray, in the Monneth of Septembre, \textit{as he lay in his bedde}, one
named Alexander Carlile, that was Sarjaunt of the Mynstrellis, cam to him
in grete hast, and badde hym aryse, for he hadde enemyes cumming for to take
him, the which were within six or seven miles,” \&c.

Edward seems to have been very liberal to his minstrels. He gave to several
annuities of ten marks a year (6 Parl. Rolls, p.~89), and, besides their
regular pay, with clothing and lodging for themselves and \textit{their horses}, they had
two servants to carry their instruments, four gallons of ale per night, wax candles,
and other indulgences. The charter is printed in Rymer, xi. 642, by Sir
J. Hawkins, vol. iv., p.~366, and Burney, vol. ii., p.~429. All the minstrels
have English names.

When Elizabeth, his queen, went to Westminster Abbey to be church\-ed (1466),
she was preceded by troops of choristers, chanting hymns, and to these succeeded
long lines of the noblest and fairest women of London and its vicinity, attended by
bands of musicians and trumpeters, and forty-two royal singers. After the banquet
and state ball, a state concert commenced, at which the Bohemian ambassadors
were present, and in their opinion as well as that of Tetzel, the German who accompanied
them, and who has also recounted their visit to England, no better
singers could be found in the whole world,\dcfootnote{ %a
Tetzel says, “Nach dem Tantz do muosten des
Kunigs Cantores kumen und muosten singen\ldots\  ich
mein das, in der Welt, nit besser Cantores sein.” “\textit{Des
böhmischen Herrn Leo’s von Rozmital Ritter,—Hof und
Pilger—Reise, 1465-1467,” \&c., 8vo., Stuttgart}, 1844, p.~157.

Again Tetzel says, “Do hörten wir das aller kostlichst
Korgesang, das alls gesatzt was, das lieblich zu hören
was.”—\textit{Ib}. p.~158.

Leo Von Rozmital, brother of the Queen of Bohemia
says "Musicos nullo uspiam in loco jucundiores et
suaviores audivimus, quam ibi: eorum chorus sexaginta,
circiter cantoribus constat,” —\textit{Ib}. p.~42.
} %end footnote
than those of the English king.
These ambassadors travelled through France, Belgium, Spain, Portugal, Italy,
and parts of Germany, as well as England, affording them, therefore, the widest
field for comparison with the singers of other countries.

At this time every great family had its establishment of musicians, and among
them the harper held a prominent position. Some who were less wealthy retained
a harper only, as did many bishops and abbots. In Sir John Howard’s expenses
(1464) there is an entry of a payment as a new year’s gift to Lady Howard’s
grandmother’s harper, “that dwellyth in Chestre.” When he became Lord
Howard he retained in his service, Nicholas Stapylton, William Lyndsey, and
“little Richard,” as singers, besides “Thomas, the harperd,” (whom he provided
with a “lyard,” or grey “gown”), and children of the chapel, who were successively
four, five, and six in number at different dates. Mr. Payne Collier, who
edited his Household Book from 1481 to \pagebreak
 1485 for the Roxburghe Club, remarks 
%===============================================================================
%45
on “the great variety of entries in connection with music and musical performers,”
as forming “a prominent feature” of the hook. “Not only were the musicians
attached to noblemen, or to private individuals, liberally rewarded, but also those
who were attached to particular towns, and who seem to have been generally
required to perform before Lord Howard on his various journies. On the 14th of
October, 1841, he entered into an agreement with William Wastell, harper of
London, that he should teach the son of John Colet, of Colchester, harper, for
a year, in order, probably, to render him competent afterwards to fill the post of
one of the family musicians.”

\renewcommand\rectoheader{edward iv.} %045

Here also a part of the stipulation was that, at the end of the year, Lord
Howard should give Wastell a \textit{gown}, which seems to have been the distinguishing
feature of a harper’s dress. In Laneham’s letter from Kenilworth (1575),
describing the “device of an \textit{ancient} minstrel and his song,” which was to have
been proffered for the amusement of queen Elizabeth, this “Squire minstrel, of
Middlesex, who travelled the country this summer season, unto worshipful men’s
houses,” is represented as a harper with a long gown of Kendal green, gathered
at the neck with a narrow gorget, and fastened before with a white clasp; his
gown having long sleeves down to mid-leg, but slit from the shoulders to the
hand, and lined with white. His harp was to be “in good grace dependent before
him,” and his “wrest,” or tuning-key, “tied to a green lace, and hanging by.”
He wore a red Cadiz girdle, and the corner of his handkerchief, edged with blue
lace hung from his bosom. Under the gorget of his gown hung a chain, “resplendent
upon his breast, of the ancient arms of Islington.” The acts of king
Arthur were the subject of his song.

The Romances which still remained popular [1480] are mentioned by William
of Nassyngton [in a MS. which Warton saw in the library of Lincoln Cathedral],
who gives his readers fair notice that \textit{he} does not intend to amuse them.


\settowidth{\versewidth}{“I warne you first at the begynnynge}

\begin{dcverse}
“I warne you first at the begynnynge\\
That I will make no vayne carpynge,\\
Of dedes of armes, ne of amours,\\
As does Mynstrellis and Gestours,\\
That maketh carpynge in many a place\\
Of \textsc{Octaviane} and \textsc{Isenbrace},

And of many other \textit{Gestes},\\
As namely, when they come to festes;\\
Ne of the lyf of \textsc{Bevys of Hamptoune},\\
That was a Knyght of grete renowne;\\
Ne of \textsc{Syr Gye of Warwyke}, \&c.\\
\hfill\textit{Warton}, vol. iv., p.~368.
\end{dcverse}


The invention of printing, coupled with the increased cultivation of poetry and
music by men of genius and learning, accelerated the downfall of the Minstrels.
They could not long withstand the superior standard of excellence in the sister
arts, on the one hand, and the competition of the ballad-singer (who sang without
asking remuneration, and sold his songs for a penny) on the other. In little more
than fifty years from this time they seem to have fallen into utter contempt. We
have a melancholy picture of their condition, in the person of Richard Sheale,
which it is impossible to read without sympathy, if we consider that to him we
are indebted for the preservation of the celebrated heroic ballad of \textit{Chevy Chace},\pagebreak
at which Sir Philip Sidney’s heart was 
 wont to beat, “as at the sound of a  
%===============================================================================
%46
trumpet;”\dcfootnote{ %
“I never heard the old song of Percy and Douglas, that 
I found not my heart moved more than with a trumpet: and
yet it is sung but by some blind crowder, with no rougher
voice than rude style; which being so evil aparelled
in the dust and cobweb of that uncivil age, what would it
work, trimmed in the gorgeous eloquence of Pindare!”—
\textit{Sir Philip Sidney’s Defence of Poetry}.
} %end footnote
and of which Ben Jonson declared he would rather have been the 
author, than of all he had ever written. This luckless Minstrel had been robbed
on Dunsmore Heath, and, shame to tell, he was unable to persuade the public
that a son of the Muses had ever been possessed of sixty pounds, which he
averred he had lost on the occasion. The account he gives of the effect upon his
spirits is melancholy, and yet ridiculous enough. [As the preservation of the
old spelling is no longer essential to the rhyme or metre, I venture to give it in
modern orthography.]

\settowidth{\versewidth}{\textit{From the “Chant of Richard Sheale,”—British Bibliographer},}
\begin{scverse}
\vleftofline{“}After my robbery my memory was so decay’d\\
That I could neither sing, nor talk, my wits were so dismay’d.\\
My audacity was gone, and all my merry talk,\\
There are some here have seen me as merry as a hawk;\\
But now I am so troubled with fancies in my mind,\\
I cannot play the merry knave, according to my kind.\\
Yet to take thought, I perceive, is not the next way\\
To bring me out of debt,—my creditors to pay.\\
I may well say that I had but evil hap\\
For to lose about threescore pounds at a clap.\\
The loss of my money did not grieve me so sore,\\
But the talk of the people did grieve me much more.\\
Some said I was not robb’d, I was but a lying knave,\\
\textit{It was not possible for a Minstrel so much money to have}.\\
Indeed, to say the truth, it is right well known\\
That I never had so much money of my own,\\
But I had friends in London, whose names I can declare,\\
That at all times would lend me two hundred pounds of ware,\\
And with some again such friendship I found,\\
That they would lend me in money nine or ten pound.\\
The occasion why I came in debt I shall make relation—\\
My wife, indeed, is a silk-woman, by her occupation;\\
In linen cloths, most chiefly, was her greatest trade,\\
And at fairs and markets she sold sale-ware that she made,\\
As shirts, smocks, and partlets, head-clothes, and other things,\\
As silk thread and edgings, skirts, bands, and strings.\\
At Lichfield market, and Atherston, good customers she found,\\
Also at Tamworth, where I dwell, she took many a pound.\\
When I had got my money together, my debts to have paid,\\
This sad mischance on me did fall, that cannot be denay’d; [denied]\\
I thought to have paid all my debts and to have set me clear,\\
And then what evil did ensue, ye shall hereafter hear:\\
Because my carriage should be light I put my money into gold,\\
And without company I rode alone—thus was I foolish bold;\\
\textit{I thought by reason of my harp no man would me suspect,\\
For Minstrels oft with money, they be not much infect."\\
\hfill From the “Chant of Richard Sheale,”—British Bibliographer}, vol. iv., p.~100. 
\end{scverse}
\pagebreak
%===============================================================================
%047

\renewcommand\versoheader{richard sheale.---extinction of minstrelsy.}


Sheale was a Minstrel in the service of Edward, Earl of Derby, who died in
1574, celebrated for his bounty and hospitality, of whom Sheale speaks most
gratefully, as well as of his eldest son, Lord Strange. The same MS. contains an
“Epilogue” on the Countess of Derby, who died in January, 1558, and his
version of Chevy Chace must have been written at \textit{least} ten years before the
latter date, if it be the one mentioned in the Complaynte of Scotland, which was
written in 1548.

In the thirty-ninth year of Elizabeth, an act was passed by which “Minstrels,
wandering abroad” were held to be “rogues, vagabonds, and sturdy beggars,”
and were to be punished as such. This act seems to have extinguished the profession
of the Minstrels, who so long had basked in the sunshine of prosperity.
The name, however, remained, and was applied to itinerant harpers, fiddlers,
and other strolling musicians, who are thus described by Puttenham, in his \textit{Arte
of English Poesie}, printed in 1589. Speaking of ballad music, he says, “The
over busy and too speedy return of one manner of tune, doth too much annoy,
and, as it were, glut the ear, unless it be in small and popular musicks sung by
these \textit{Cantabanqui} upon benches and barrels’ heads, where they have none other
audience than boys or country fellows that pass by them in the street; or else by
blind harpers, or such like tavern minstrels, that give a fit of mirth for a groat;
and their matter being for the most part stories of old time, as the Tale of Sir
Topas, Bevis of Southampton, Guy of Warwick, Adam Bell and Clym of the
Clough, and such other old romances or historical rhimes, \textit{made purposely for} the
recreation of the common people at Christmas dinners and bride-ales, and in
taverns and alehouses, and such other places of base resort. Also they” [these
short times] “be used in Carols and Rounds, and such like light and lascivious
poems, which are commonly more commodiously uttered by these buffons, or vices
in plays than by another~person.”

Ritson, whose animosity to Percy and Warton seems to have extended itself
to the whole minstrel race, quotes, with great glee, the following lines on their
downfall, which were written by Dr.~Bull, a rival musician:—

\settowidth{\versewidth}{He turned the Minstrels out of doors,}
\begin{scverse}
“When Jesus went to Jairus’ house,\\
(Whose daughter was about to die)\\
He turned the Minstrels out of doors,\\
Among the rascal company:\\
Beggars they are with one consent,—\\
And rogues, \textit{by act of Parliament.}” 
\end{scverse}

%\vspace{2\baselineskip}
\centerrule
\pagebreak
%048
%===============================================================================
\chapter{POPULAR MUSIC,\par SONGS AND BALLADS,}

\section*{REIGNS OF HENRY VII., HENRY VIII., EDWARD VI.,
AND MARY.}

\centerrule

Little occurs about music and ballads during the short reigns of Edward~V. and
Richard III.

Richard was very liberal to his musicians, giving annuities to some, and
gratuities to others. (See Harl. MS., No. 433.) But his chief anxiety seems to
have been to increase the already splendid choral establishment of the Chapel
Royal. For that purpose he empowered John Melynek, one of the gentlemen of
the chapel, “to take and seize for the king” not only children, but also “all
such singing men expert in the science of music, as he could find and think able
to do the king’s service, within all places of the realm, as well cathedral churches,
colleges, chapels, houses of religion, and all other \textit{franchised or exempt places}, or
elsewhere.” (Harl. MS., 433, p.~189.) But it is not my object to pursue the
subject of royal establishments~further.

In the privy purse expenses of Henry VII., from the seventh to the twentieth
year of his reign, there are many payments relating to music and to popular
sports, from which the following are selected:—
\bigskip

\noindent\footnotesize
\begin{tabular}{lllrrr}
1492.&Feb. 4th,& To the childe that playeth on the records\\
&&\indent [recorder]\dotfill &£1&0&0\\
&April 6th,&To Gwyllim for flotes [flutes] with a case\dotfill&3&10&0\\
&May 8th,&For making a case for the kinges suerde, and a\\
&&\indent case for James Hide’s harp\dotfill&1&0&8\\
&July 8th,&To the maydens of Lambeth for a May\dotfill&0&10&0\\
&August 1st,&At Canterbury, To the children, for singing in	 the \\
&&\indent gardyn\dotfill&0&3&4\\
1493.&Jan 1st,&To the Queresters [choristers], at Paule’s and	\\
&&\indent St. Steven\dotfill&0&13&4\\
&Jan. 6th,&To Newark [William Newark, the composer] for\\
&&\indent making a song\dotfill&1&0&0\\
&Nov. 12th,&To one Cornysshe for a prophecy, in rewards\dotfill&0&13&4
\end{tabular}
\normalsize
\medskip

Probably William Cornish, jun., composer, who belonged to the king’s chapel,
and was the author of a poem, called “A Treatise between Trouth and Informacion.”
He was paid 13\textit{s}. 4\textit{d}. on Christmas day, 1502, for setting a carol. 

\pagebreak


%49
%===============================================================================
\renewcommand\rectoheader{henry vii.}
\noindent\footnotesize
\begin{tabular}{lllrrr}
          &Nov. 30th,&Delivered to a merchaunt,  for a pair\footnotemark of\\
          &&\indent Organnes\dotfill&30&0&0\\
&Dec. 1st,&To Basset, riding for th’ organ pleyer of\\
&&\indent Lichefelde\dotfill&0&13&4\\
1494.&Jan. 2,&For playing of the Mourice [Morris] Daunce\dotfill&2&0&0\\
&Nov. 29th,&To Burton, for making a Masse\dotfill&1&0&0\\
&\indent ”& To my Lorde Prince’s Luter, in rewarde\dotfill&1&0&0\\
1495.&Aug 2nd,&To the women that songe before the king and\\
&&\indent the quene, in rewarde\dotfill&0&6&8\\
&Nov. 2nd,&	To a woman that singeth with a fidell\dotfill&0&2&0\\
&Nov. 27th,&To Hampton of Wourcestre, for making of\\
&&\indent	Balades, in rewarde\dotfill&1&0&0\\
1496.&April 25th,&To Hugh Denes, for a lute\dotfill&0&13&4\\
&June 25th,&To Frensheman, player of the organes\dotfill&0&6&8\\
&Aug. 5th,&To a Preste that wrestelled at Ceceter\dotfill&0&6&8\\
&Aug. 17th,&To the quene’s fideler, in rewarde\dotfill&1&6&8\\
1499.&June 6th,&To the May-game at Greenwich\dotfill&0&4&0\\
1501.&May 21st,&For a lute for my lady Margaret [the king’s\\
&&\indent eldest daughter, then about twelve years\\
&&\indent  old, afterwards Queen of Scots]\dotfill&0&13&4\\
&Sept. 30th,&To theym that daunced the mer’[morris] daunce\dotfill&1&6&8\\
&Dec. 4th,&To the Princesse stryng mynstrels at \\
&&\indent Westminster\dotfill&2&0&0\\
1502.&Jan. 7th,&To one that sett the king’s cleyvecordes\dotfill&0&10&4\\
&Feb. 4th,&To one Lewes, for a morris daunee\dotfill&1&13&4\\
1504.&March 6th,&For a pair of Clavycordes\dotfill&0&13&0\\
&\indent”&To John Sudborough, for a songe\dotfill&1&0&0\\
1505.&July 25th,&To the gentylmen of the kinges chapell, for to\\
&&\indent drinke with a bucke\dotfill&2&0&0\\
&Aug. 1st,&For a lute for my Lady Mary\dotfill&0&13&4\\
\end{tabular}
\bigskip
\normalsize

There is also a great variety of payments to the musicians of different towns,
as the “"Waytes” of Dover, Canterbury, Dartford, Coventry, and Northampton;
the minstrels of Sandwich, the shawms of Maidstone; to bagpipers, the king’s
piper (repeatedly), the piper at Huntingdon, \&c.; to harpers, some of whom were
Welsh. And there are also several entries “To a Walsheman for a ryme;”
liberal presents to the poets, of his mother (the Countess of Richmond), of the
prince, and of the king; to “the rymer of Scotland,” who was in all probability the
Scotch poet, William Dunbar, who celebrated the nuptials of James IV. and the
princess Margaret, in his “Thistle and the Rose,” and to an Italian poet. All
these may be seen in Excerpta Historica (8vo., 1833), and, as the editor
remarks:—“To judge from the long catalogue of musicians and musical instruments, 
flutes, recorders, trumpets, sackbuts, harps, shalmes, bagpipes, organs, and
round organs, clavicords, lutes, horns, pipers, fiddlers, singers, and dancers, Henry’s
love of music must have been great, which is further established by the fact, that
in every town he entered, as well as on board the ship which conveyed him to
Calais, he was attended by minstrels and waits.” 

\footnotetext{ A \textit{pair} of organs, means a \textit{set} of organs, \ie, an organ.
A pack of cards was formerly called a pair of cards, and
we still say, “a \textit{pair} of steps”—“up two \textit{pair} of stairs.”
}
\pagebreak

%50
%===============================================================================
\renewcommand\versoheader{henry viii.}

A manuscript, containing a large number of songs and carols, bas been recently
found in the library of Balliol Coll., Oxford, where it bad been accidently concealed,
behind a book-case, during a great number of years. It is in the handwriting
of Richard Hill, merchant of London, and contains entries from the year
1483 to 1535. Six or eight of the songs and carols are the same as in the book
printed by the Percy Society, to which I have referred at page 41, and especially
the carol, “Nowell Nowell,” but the volume does not contain music. The song
of the contention between Holly and Ivy, beginning “Holly beareth berries, berries
red enough,” which is printed in Ritson’s Ancient Songs, from a manuscript
of Henry the Sixth’s time, is there also, proving that some of the songs are
of a much earlier date than the manuscript, and that they were still in favor. At
fol. 210, v. is a copy of the “Nut-browne Mayde,” and at the end of it “Explicit
quod, Rich. Hill,” which was the usual mode of claiming authorship of a work.

In the Pepysian Library, Magdalene College, Cambridge, there is a manuscript
book of vocal music (No. 87), containing the compositions of the most eminent
masters, English and foreign, of the time of Henry VII., written for the then
Prince of Wales. It was the Prince’s book, is beautifully written on vellum, and
illuminated with his figure in miniature.

Henry VIII. was not only a great patron of music, but also a composer; and,
according to Lord Herbert of Cherbury, who wrote his life, he composed two
complete services, which were often sung in his chapel. Hollinshed, in speaking
of the removal of the court to Windsor, when Henry was beginning his progress,
tells us that he “exercised himselfe dailie in shooting, singing, dansing, wressling,
casting of the barre, plaieing at the recorders, flute, virginals, in setting of songs,
and making of ballades.” All accounts agree in describing him as an amiable and
accomplished prince in the early part of his reign; and the character given of him
to the Doge of Venice, by his three ambassadors at the English court, could
scarcely be expressed in more favorable terms. In their joint despatch of
May 3rd, 1515, they say: “He is so gifted and adorned with mental accomplishments
of every sort, that we believe him to have few equals in the world. He
speaks English, French, and Latin; understands Italian well; plays almost
on every instrument, and composes fairly (delegnamente); is prudent and sage,
and free from every vice.”\dcfootnote{ %a
Despatch written by Pasqualigo, Badoer, and Giustinian
conjointly. See four years at the Court of Henry
VIII., Selection of Despatches addressed to the Signory
of Venice, from January, 1515, to July 26, 1519. Translated
by Rawdon Brown. 8vo., 1854. vol. i., p.~76.
} %end footnote

In the letter of Sagudino (Secretary to the Embassy), writen to Alvise Foscari,
at this same date, he says: “He is courageous, an excellent musician, plays the
virginals well, is learned for his age and station, and has many other endowments
and good parts.” On the 1st of May, 1515, after the celebration of May-day at
Greenwich, the ambassadors dined at the palace, and after dinner were taken into
certain chambers containing a number of organs, virginals (clavicimbani), flutes,
and other instruments; and having heard from the ambassadors that Sagudino 
was a proficient on some of them, he \pagebreak
was asked by the nobles to play, which 
%51
%===============================================================================
he did for a long while, both on the virginals and organ, and says that he bore
himself bravely, and was listened to with great attention. The prelates told him
that the king would certainly wish to hear him, for he practised on these instruments
day and~night.

\renewcommand\rectoheader{venetian ambassadors---erasmus.} 

Pasqualigo, the ambassador-extraordinary, gives a similar account at the same
time. Of Henry he says: “He speaks French, English, and Latin, and a little
Italian, plays well on the lute and virginals, sings from book \textit{at sight}, draws the
bow with greater strength than any man in England, and jousts marvellously.
Believe me he is in every respect a most accomplished prince; and I, who have
now seen all the sovereigns in Christendom, and last of all these two of France
and England, might well rest content,” \&c. Of the chapel service, Pasqualigo
says: “We attended High Mass, which was chaunted by the bishop of Durham,
with a superb and noble descant choir”\dcfootnote{ %a
Descant choir is not a proper term, because the Music
of the King’s Chapel was not extempore descant, but in
written counterpoint of four parts. Several of the manuscripts
in use about this period, are preserved in the
King’s Library, British Museum, and some were Henry’s
own books. They are beautifully written manuscripts
on parchment, bearing the King’s arms. In one a Canon
in eight parts is inserted on the words “Honi soit qui
mal y pense.” The references to these manuscripts
will be found in Mr. Oliphant’s Catalogue of Musical
MSS., British Museum, towards the commencement.
See Nos. 12, 13, 21, \&c.
} %end footnote
(Capella di Discanto); and Sagudino
says: “High Mass was chaunted, and it was sung by his majesty’s choristers,
whose voices are really rather divine than human; they did not chaunt, but sung
like angels (non cantavano, ma jubilavano); and as for the deep bass voices,
I don’t think they have their equals in the world.”\dcfootnote{ %b
The florid character of the counterpoint in use in
churches in those days is slyly reproved in a dialogue between
Humanity and Ignorance, in the Interlude of \textit{The
Four Elements}, printed about 1510. (Prick-song meant
harmony written or pricked down, in opposition to plain-song, 
where the descant rested with the will of the singer.)
\settowidth{\versewidth}{Peace, man, prick-song may not be desp}
\begin{fnverse}
\indentpattern{0222}
\begin{patverse}
\textit{Hu}. -- “Peace, man, prick-song may not be despised,\\
For therewith God is well pleased,\\
Honoured, praised, and served\\
In the Church oft-times among.”
\end{patverse}

\indentpattern{0222222}
\begin{patverse}
\textit{Ig}. —“Is God well pleased, trow’st thou, thereby?\\
Nay, nay! for there is no reason why:\\
For is it not as good to say plainly\\
‘Give me a spade’\\
As ‘giveme a spa-ve-va, ve-va-ve-vade?’\\
But if thou wilt have a song that is good,\\
I have one of Robin Hood,” \&c.
\end{patverse}
\end{fnverse}
} %end footnote
(Vol. i., p.~77.)

Upon these despatches the editor remarks: “As Pasqualigo had been ambassador
at the courts of Spain, Portugal, Hungary, France, and of the Emperor, he was
enabled to form comparisons between the state of the science in those kingdoms
and our own; and, indeed, it is the universal \textit{experience} of the Venetian ambassadors, 
and their peculiar freedom from prejudice or partiality (no jealousy or
rivalry existing between them and England), that makes their comments on our
country so valuable.” (Vol. 1, p.~89.)

Erasmus, speaking of the English, said that they challenge the prerogative of
having the most handsome women, of keeping the best tables, and of being most
accomplished in the skill of music of any people;\dcfootnote{ %c
“Britanni, præter alia, formam, musicam, et lautas
mensas propriè sibi vindicent.” — \textit{Erasmus Enconium
Moriæ}.
} %end footnote
and it is certain that the beginning
of the sixteenth century produced in England a race of musicians equal to
the best in foreign countries, and in point of \textit{secular} music decidedly in advance
of them. When Thomas Cromwell, afterwards Earl of Essex, went from Antwerp
to Rome, in 1510, to obtain from Pope Julius II. the renewal of the “greater and
lesser pardon”\dcfootnote{ %d
These pardons, says Foxe, gave them the power to\textsuperscript{}
receive full remission, “apæna et culpa;” also pardon
for souls in purgatory, on payment of 6\textit{s}. 8\textit{d}. for the first
year, and 12\textit{d}. for every year after, to the Church of St.
Botolph’s, Boston.
} %end footnote
for the town of Boston, for the \pagebreak
maintenance of their decayed port, 
%52
%===============================================================================
“being loth,” says Foxe, “to spend much time, and more loth to spend his money,
among the greedy cormorants of the Pope’s court,” he devised to meet him on his
return from hunting; and “having knowledge how the Pope’s holy tooth greatly
delighted in new-fangled strange delicates and dainty dishes, it came into his
mind to prepare certain fine dishes of jelly, made after our country manner here
in England; which to them of Rome was not known nor seen before. This done,
Cromwell observing his time accordingly, as the Pope was newly come from
hunting into his pavilion, he, with his companions, approached with his English
presents, brought in with a \textit{three-man's song} (as we call it) in the English tongue,
and all after the English fashion. The Pope suddenly marvelling at the strangeness
of the song, and understanding that they were Englishmen, and that they
came not empty-handed, willed them to be called in; and seeing the strangeness of
the dishes, commanded by and by his Cardinal to make the assay; who in tasting
thereof, liked it so well, and so likewise the Pope after him, that knowing of them
what their suits were, and requiring them to make known the making of that meat,
he, incontinent, without any more ado, stamped both their pardons, as well the
greater as the lesser.” (Acts and Monuments.) The introduction of these songs
into Italy is also mentioned by Michael Drayton in his Legend of Thomas
Cromwell, Earl of Essex, which was first printed in quarto in 1609.

\settowidth{\versewidth}{Which won much licence for my countrymen.}
\begin{scverse}
“Not long it was ere Rome of me did ring.\\
Hardly shall Rome such full days see again;\\
Of \textit{Freemen's Catches} to the Pope I sing,\\
Which won much licence for my countrymen.\\
Thither the which I was the first did bring,\\
That were unknown in Italy till then,” \&c.
\end{scverse}

In the Life of Sir Peter Carew, by John Vowell, alias Hoker, of Exeter
(Archæo\-logia, vol. 28), Freemen’s Songs are again mentioned. “From this time
he (Sir Peter) continued for the most part in the court, spending his time in
all courtly exercises, to his great praise and commendation, and especially to the
good liking of the king (Henry VIII.), who had a great pleasure in him, as
well for his sundry noble qualities, as also for his singing. For the king himself
being much delighted to sing, and Sir Peter Carew having a pleasant voice, the
king would often use him to sing with him certain songs they call \textit{Freemen Songs},
as namely, ‘By the bancke as I lay,’ and ‘As I walked the wode so wylde,’” \&c.

To sing at sight was so usual an accomplishment of gentlemen in those days,
that to be deficient in that respect was considered a serious drawback to success in
life. Skelton, in his \textit{Bowge at Court}, introduces Harvy Hafter as one who cannot
sing “on the booke,” but he thus expresses his desire to learn:—
\settowidth{\versewidth}{“Wolde to God it wolde please you some day,}
\begin{scverse}
“Wolde to God it wolde please you some day,\\
A balade boke before me for to laye,\\
And lerne me for to synge \textit{re, mi, fa, sol},\\
And when I fayle, bobbe me on the noll.”\\
\hfill\textit{Skelton’s Works, Ed. Dyce}, vol. i., p.~40. \hspace*{4em}
\end{scverse}

\pagebreak


%53
%===============================================================================
\renewcommand\rectoheader{the english love of songs and ballads.}

Barklay, in his fourth Eclogue, (about 1514) says—
\settowidth{\versewidth}{“When your fat dishes smoke hot upon your table,}
\begin{scverse}
“When your fat dishes smoke hot upon your table,\\
Then laude ye songs, and ballades magnifie;\\
If they be merry, or written craftely,\\
Ye clap your handes and to the making harke,\\
And one say to another, Lo, here a proper warke!”
\end{scverse}
The interlude of “The Four Elements” was printed by Rastall about 1510;
and, in that, Sensual Appetite, one of the characters, recommends Humanity “to
comfort his lyf naturall” with “daunsing, laughyng, or plesaunt songe,” and
says—
\settowidth{\versewidth}{For I see it is but folly for to have a sad mind.”}
\begin{scverse}
“Make room, sirs, and let us be merry,\\
With huff a galand, syng Tyrll on the berry,\\
\vin And let the wide world wynde.\\
Sing Frisk a jolly, with Hey trolly lolly,\\
For I see it is but folly for to have a sad mind.”\\
\hfill \textit{Percy Soc}., No. 74.\hspace*{4em}
\end{scverse}
“Hey, ho, frisca jolly, under the greenwood tree,” is the burden of one of the
songs in the musical volume of the reign of Henry VIII. (MS. Reg. Append. 58.)
from which I have extracted several specimens. It contains, also, some instrumental
pieces, such as “My Lady Carey’s Dompe,” and “My Lady Wynkfield’s
Rownde,” which when well played on the virginals, as recently, by an able lecturer,
are very effective and musical.

Some of Henry the Eighth’s own compositions are still extant. In a collection
of anthems, motets, and other church offices, in the handwriting of John Baldwin,
of Windsor, (who also transcribed that beautiful manuscript, Lady Neville’s
Virginal Book, in 1591), is a composition for three voices, “Quam pulchra es, et
quam decora.” It bears the name Henricus Octavus at the beginning, and “quod
Henricus Octavus” at the end of the cantus part. The anthem “O Lord, the
maker of all things,” which is attributed to him in Boyce’s Cathedral Music, is
the composition of William Mundy; the words only are taken from Henry the
Eighth’s primer. Some music for a mask, which Stafford Smith attributes to
him, will be found in the Arundel Collection of MS. (Brit. Mus.) or in Musica
Antiqua, vol. i.; and one of his ballads, “Pastime with good company,” is given
as a specimen in the following pages.

In 1533 a proclamation was issued to suppress “fond [foolish] books, ballads,
rhimes, and other lewd treatises in the English tongue;” and in 1537 a man of
the name of John Hogon was arrested for singing a political ballad to the tune of
“The hunt is up.” It was not only among the upper classes that songs and
ballads were then so general, although the allusions to the music of the lower
classes are less frequently to be met with at this period than a little later, when
plays, which give the best insight to the manners and customs of private life, had
become general. One passage, however, from Miles Coverdale’s “Address unto
the Christian reader” prefixed to his “Goastly Psalmes and Spirituall Songes,”
[1538] will suffice to prove it. “Wolde God that our Mynstrels had none other
thynge to play upon, neither our \textit{carters} \pagebreak
and \textit{plowmen} other thynge to whistle 
%54
%===============================================================================
upon, save psalmes, hymns, and such like godly songes... And if women at
the rockes,\dcfootnote{ %a
Rock, a distaff: that is, the staff on which flax was
held, when spinning was performed without a wheel; or
the corresponding part of the spinning wheel.—\textit{Nares’ Glossary}.
} %end footnote 
and spinnynge at the wheles, had none other songes to pass their tyme
withall, than such as Moses’ sister,... songe before them, they should be better
occupied than with \textit{Hey, nonny, nonny—Hey, trolly, lolly}, and such like fantasies.”
Despite the excellent intent with which this advice was given, it did not evidently
make much impression, either then or after. The traditional tunes of every
country seem as natural to the common people as warbling is to birds in a
state of nature; the carters and ploughmen continued to be celebrated for their
whistling, to the end of the eighteenth century, and the women thought rather with
Ophelia: “You must sing \textit{down, a-down}, an you call him a-down-a, Oh, \textit{how the
wheel becomes it}!”

Anthony à Wood says that Sternhold, who was Groom of the Chamber to
Henry VIII, versified fifty-one of the Psalms, and “caused musical notes to be
set to them, thinking thereby that the courtiers would sing them instead of their
sonnets, but did not, only some few excepted.” They were not, however, printed
till 1549. On the title page it is expressed that they were to be sung “in private
houses, for godly solace and comfort, and for the laying apart all ungodly songes
and ballads.”

Although Henry VIII. had given all possible encouragement to ballads and
songs in the early part of his reign, both in public and private,—and in proof
of their having been used on public occasions, I may mention the coronation of
Anne Boleyn, when a choir of men and boys stood on the leads of St. Martin’s
Church, and sang new ballads in praise of her majesty,—yet, when they were resorted
to as a weapon against the Reformation, or in opposition to any of his own
opinions and varying commands, he adopted the summary process of suppressing
them altogether. It is in some measure owing to that act, but principally to their
perishable nature, that we have no \textit{printed} ballads now remaining of an earlier
date than that on the downfall of his former favorite, Thomas, Lord Cromwell,
which is in the library of the Society of Antiquaries, at Somerset House. The
act, which was passed in 1543, is entitled “An act for the advancement of true
religion, and for the abolishment of the contrary” (Anno 34-35, c. i.), and recites
that “froward and malicious minds, intending to subvert the true exposition of
scripture, have taken upon them, by printed ballads, rhymes, etc., subtilly and
craftily to instruct his highness’ people, and specially the youth of this his realm,
untruly. For reformation whereof, his majesty considereth it most requisite to
purge his realm of all such books, ballads, rhymes, and songs, as be pestiferous
and noisome. Therefore, if any printer shall print, give, or deliver, any such, he
shall suffer for the first time imprisonment for three months, and forfeit for every
cop.~10\textit{l}., and for the second time, forfeit all his goods and his body be committed
to perpetual prison.” Although the act only expresses “all such books, ballads,
rhymes, and songs as be pestiferous and noisome,” there is a list of exceptions 
to it, and no ballads of any description \pagebreak
are excepted. “Provided, also, that 
%055
%===============================================================================
all books printed before the year 1540, entituled Statutes, Chronicles, Canterbury
Tales, Chaucer’s books, Gower’s books, and stories of men’s lives, shall not be
comprehended in the prohibition of this act.” It was not, however, the first time
that ballads had been employed for controversy on religious subjects. The ballads
against the Lollards, and those against the old clergy, have been mentioned at
page 40; and there is a large number extant against monks and friars, many of
which were, and some still are, popular.

\renewcommand\rectoheader{acts of parliament and proclamations against ballads.}

The first collection of songs in parts that was \textit{printed} in England, was in 1530;
but of that only a base part now remains.\dcfootnote{ %a
It contained compositions by Cornish, Pygot, Ashwell,
Taverner, Gwynneth, Jones, Dr.~Cowper, and Dr.
Fairfax. See the Index in Ritson’s Ancient Songs,
p. xxiii., last edition, Stafford Smith’s are principally by
Fairfax, Newark, Heath, Turges, Sheringham, and Sir
Thomas Philipps; but this list of composers might be
increased greatly by including those in other manuscripts.
} %end footnote
There are, however, many such collections
in manuscript in public and private libraries. Stafford Smith’s printed
collection of songs in score, composed about the year 1500, is almost entirely
taken from one~manuscript.

Henry VIII left a large number of musical instruments at his death, the inventory
of which may be seen in Harl. MSS. No. 1419, fol. 200; and, as might
be expected, all his children were well taught in music.

“Ballads,” says Mr. Collier, “seem to have multiplied after Edward VI. came
to the throne; no new proclamation was issued, nor statute passed on the subject,
while Edward continued to reign; but in less than a month after Mary became
queen, she published an edict against ‘books, ballads, rhymes, and treatises,’
which she complained had been ‘set out by printers and stationers, of an evil
zeal for lucre, and covetous of vile gain.’ There is little doubt, from the few
pieces remaining, that it was, in a considerable degree, effectual for the end
in view.”

\vfill
\centerrule
\vfill

The following tunes are occasionally classed rather under the dates to which
I consider them to belong, than by those of the copies from which they are derived;
but as the authorities are given in every case, the reader has the means before him
of forming his own opinion. Some, however, are classed rather for convenience of
subject, as songs of Robin Hood, songs or tunes mentioned by Shakespeare,~\&c.

After a few from manuscripts of the time of Henry VIII., there are specimens
of “King Henry’s Mirth, or Freemen’s Songs,” from a collection printed in 1609,
which contains many “fine vocal compositions of very great antiquity.”\dcfootnote{ %
In 1609, Thomas Ravenscroft, Mus. Bac., collected
and printed 100 old Catches, Rounds, and Canons, under
the title of “Pammelia: Musick’s Miscellanie, or mixed
varietie of pleasant Roundelayes and delightful Catches.”
It met with so much success, that in the same year he
published a second, called “Deuteromelia: or the second
part of Musick's Melodie, or melodious musicke of pleasant
Roundelayes, K. H. [\textit{King Henry's] Mirth, or Freemen’s
Songs},”~\&c.; and in 1611, a third collection, called
“Melismata: Musical Phansies, fitting the court, city,
and countrey humours.” Some of the Songs and Catches
in these collections are undoubtedly of the reign of Henry
VII., and it is to be presumed that the authors of all
were unknown to Ravenscroft, as, contrary to custom,
he does not mention them in any instance.
} %end footnote
But
of those, I have only selected such as were also used as song or ballad tunes,
sung by a single~voice. 

\pagebreak
%056
%===============================================================================
\renewcommand\versoheader{english song and ballad music.}

\musictitle{Pastime with good Company.}

The words and music of this song are preserved in a manuscript of the time of
Henry VIII., formerly in Ritson’s possession, and now in the British Museum
(Add. MSS., 5665); in which it is entitled \textsc{The King’s Ballad}. Ritson
mentions it in a note to his Historical Essay on Scotish Song, and Stafford Smith
printed it in his \textit{Musica Antiqua} in score for three men’s voices. It is the first of
those mentioned in Wedderburn’s \textit{Complaint of Scotland}, which was published in
1549: “Now I will rehearse some of the sweet songs that I heard among them
(the shepherds) as after follows: in the first \textit{Pastance with good Company},” \&c.
The tune is also to be found arranged for the lute (without words) in the volume
among the king’s MSS. before cited (Append. 58), of which “Dominus Johannes
Bray” was at one time the possessor. This may be considered as another proof
of its former popularity.

\musicinfo{In moderate time.}{Song by Henry VIII.}

\lilypondfile[staffsize=15]{lilypond/056-pastime-with-good-company}\normalsize

 \settowidth{\versewidth}{All thoughts and fantasies to digest,}
 \indentpattern{0000111221}

\begin{dcverse}\begin{patverse}
Youth will needs have dalliance,\\
Of good or ill some pastance;\\
Company me thinketh the best\\
All thoughts and fantasies to digest,\\
For idleness\\
Is chief mistress\\
Of vices all:\\
Then who can say\\
But pass the day\\
Is best of all?
\end{patverse}

\begin{patverse}
Company with honesty\\
Is virtue,—and vice to flee:\\
Company is good or ill,\\
But ev’ry man hath his free will,\\
The best I sue,\\
The worst eschew:\\
My mind shall be\\
Virtue to use\\
Vice to refuse\\
I shall use me. 
\end{patverse}
\end{dcverse}

\pagebreak


%057
%===============================================================================
\renewcommand\rectoheader{from henry vii. to mary.}

\musictitle{Ah! The sighs that come fro’ my heart.}

This little love-song is the first in MSS. Reg. Append. 58., of the time of
Henry VIII., and the air is both elegant and expressive. The cadence, or flourish
at the end, is characteristic of the period, and there is a pretty attempt at
musical expression on the words, “fro? my \textit{love} depart.”

\musicinfo{Smoothly and with expression.}{}

\lilypondfile{lilypond/057-ah-the-sighs-that-come-fro-my-heart}

\settowidth{\versewidth}{Farewell my joy! and welcome pain!}
\begin{dcverse}\begin{altverse}
Ah! the sighs that come from my heart,\\
They grieve me passing sore,\\
Sith I must fro’ my love depart,\\
Farewell, my joye, for evermore.
\end{altverse}

\begin{altverse}
Oft to me, with her goodly face,\\
She was wont to cast an eye:\\
And now absence to me in place?\\
Alas! for woe I die, I die!
\end{altverse}

\begin{altverse}
I was wont her to behold,\\
And take in armès twain;\\
And now, with sighès manifold,\\
Farewell my joy! and welcome pain!
\end{altverse}

\begin{altverse}
Ah! me think that should I yet,\\
As would to God that I might!\\
There would no joys compare with it\\
Unto my heart, to make it light.
\end{altverse}
\end{dcverse}

\musictitle{Western wind, when wilt thou blow?}

This is also taken from MSS. Reg. Append. 58, time of Henry VIII. As the
tune appears to be in the ancient Dorian mode, it has been harmonized in that
mode, to preserve its peculiarity of character.

The writer of a quarto volume on ancient Scotish melodies has asserted that
\textit{all} the ancient English music in Ritson’s, or other collections, is of a heavy
drawling character. An assertion so at variance with fact must either have
proceeded from narrow-minded prejudice, or from his not having understood
ancient musical notation. That he could not discriminate between Scotch and
English music is evinced by the fact of his having appropriated some of the best
known English compositions as ancient Scotish melodies.\dcfootnote{ %a
This writer also cites the authority of Giraldus Cambrensis,
 \textit{who says nothing of the kind}; and in the same
sentence, appropriates what Giraldus says in favour of
\textit{Irish} music to Scotch.
} %end footnote
\pagebreak
%58
%===============================================================================

The following song is one of those adduced by him in proof of the drawling of
English music; but I have restored the words to their proper places, and it is by
no means a drawling song. It should be borne in mind that these specimens of
English music are long anterior to any Scotish music that has been produced.

\musicinfo{Moderate time.}{}

\lilypondfile{lilypond/058-western-wind-when-wilt-thou-blow}

\musictitle{Blow thy horn, Hunter!}

This is also copied from MSS. Reg. Append. 58, time of Henry VIII. It is a
spirited tune, and should be sung more quickly in proportion than the others,
because in modernizing the notation, I have only made a crotchet into a quaver,
instead of into a semiquaver, as would have been more correct, considering the
date of the manuscript.

\musicinfo{Boldly and well marked.}{}

\lilypondfile{lilypond/058-blow-thy-horn-hunter}\normalsize

\pagebreak

%59
%===============================================================================

\musictitle{The Three Ravens.}

This song is one of those included under the head of “Country Pastimes” in
Melismata, 1611. Ritson in his \textit{Ancient Songs}, remarks: “It will be obvious
that this ballad is much older, not only than the date of that book, but than most
of the other pieces contained in it.” It is nevertheless still so popular in some
parts of the country, that I have been favored with a variety of copies of it,
written down from memory; and all differing in some respects, both as to words
and tune, but with sufficient resemblance to prove a similar origin.

\musicinfo{Slowly, smoothly, and with great expression.}{}

\lilypondfile[staffsize=14]{lilypond/059-the-three-ravens}

\settowidth{\versewidth}{Down in yonder green field, Down a down, hey down, hey down,}
\begin{scverse}Down in yonder green field, Down a down, hey down, hey down,\\
There lies a knight slain, under his shield. With a down.\\
His hounds they lie down at his feet,\\
So well ‘do’ they their master keep. With a down, derry, \&c.

His hawks they fly so eagerly, Down a down, \&c.\\
There’s no fowl ‘that’ dare him come nigh. With a down.\\
Down there comes a fallow doe,\\
As great with young as she might go. With a down, derry, \&c.

She lifted up his bloody head, Down a down, \&c.\\
And kiss’d his wounds that were so red; With a down.\\
She got him up upon her back,\\
And carried him to earthen lake. With a down, \&c.

She buried him before the prime: With a down, \&c.\\
She was dead herself ere even-song time. With a down.\\
God send every gentleman\\
Such hawks, such hounds, and such a leman [lov’d one]. With a down, \&c. 
\end{scverse}
\pagebreak

%60
%===============================================================================

\musictitle{The Hunt Is Up}

Among the favorites of Henry the Eighth, Puttenham notices “one Gray,
what good estimation did he grow unto with the same King Henry, and afterwards
with the Duke of Somerset, Protectour, for making certaine merry ballades,
whereof one chiefly was, \textit{The hunte is up, the hunte is up}.” Perhaps it was the
same William Gray who wrote a ballad on the downfall of Thomas Lord Cromwell
in 1540, to which there are several rejoinders in the library of the Society of
Antiquaries. The tune \textit{The Hunt is up} was known as early as 1537, when
information was sent to the Council against one John Hogon, who had offended
against the proclamation of 1533, which was issued to suppress “fond books,
ballads, rhimes, and other lewd treatises in the English tongue,” by singing,
“with a crowd or a fyddyll,” a political song to that tune. Some of the words
are inserted in the information, but they were taken down from recitation, and are
not given as verse (see Collier’s Shakespeare, i., p. cclxxxviii.) In the Complaint
of Scotland, 1549, \textit{The Hunt is up} is mentioned as a tune for dancing, for which,
from its lively character, it seems peculiarly suited; and Mr. Collier has a MS.
which contains a song called “The Kinges Hunt is upp,” which may be the very
one written by Gray, since “Harry our King” is twice mentioned in it, and a
religious parody as old as the reign of Henry VIII. is in precisely the same
measure. The following is the song:—

\musicinfo{Merrily.}{The Kinges Hunt is upp.}

\lilypondfile{lilypond/060-the-kinges-hunt-is-upp}

\settowidth{\versewidth}{The east is bright with morning light,}
 \begin{dcverse}    \begin{altverse}
   The east is bright with morning light,\\
And darkness it is fled,\\
And the merie home wakes up the morne\\
To leave his idle bed.
\end{altverse}

\begin{altverse}{Beholde the skyes with golden dyes\\
Are glowing all around,\\
The grasse is greene, and so are the treene,\\
All laughing at the sound.}\end{altverse}

\begin{altverse}{The horses snort to be at the sport,\\
The dogges are running free,\\
The woddes rejoyce at the mery noise\\
Of hey tantara tee ree!}\end{altverse}

\begin{altverse}{The sunne is glad to see us clad\\
All in our lustie greene,\\
And smiles in the skye as he riseth hye,\\
To see and to he seene.}\end{altverse}

\end{dcverse}


\settowidth{\versewidth}{Awake, all men, I say agen,}
\begin{scverse}    \begin{altverse}
Awake, all men, I say agen,\\
Be mery as you maye,\\
For Harry our Kinge is gone hunting,\\
To bring his deere to baye. 
\end{altverse}
\end{scverse}
\pagebreak

%%61
%%===============================================================================

The tune is taken from \textit{Musick’s delight on the Cithren}, edition of 1666, which
contains many very old and popular tunes, such as “Trip, and go,” and “Light
o’ Love” (both mentioned by Shakespeare), which I have not found in any other
printed collection. Ritson, in his Ancient Songs, quotes the following song of
one verse, which is in the same measure, and was therefore probably sung to the
same tune. It may be found in \textit{Merry Drollery Complete}, 1661, and the \textit{New
Academy of Complements}, 1694 and 1713.

\settowidth{\versewidth}{“The hunt is up, the hunt is up,}
\begin{scverse}\begin{altverse}
\vleftofline{“}The hunt is up, the hunt is up,\\
And now it is almost day;\\
And he that’s ‘at home, in bed with his wife,’\\
’Tis time to get him away.”
\end{altverse}
\end{scverse}

Any song intended to arouse in the morning—even a love-song—was formerly
called a \textit{hunt’s-up}. Shakespeare so employs it in \textit{Romeo and Juliet}, Act 3, Sc. 5;
and the name was of course derived from a tune or song employed by early
hunters. Butler, In his \textit{Principles of Musik}, 1636, defines a \textit{hunt’s-up} as
“morning music;” and Cotgrave defines “Resveil” as a hunt’s-up, or \textit{Morning
Song} for a new-married wife. In Barnfield’s \textit{Affectionate Shepherd}, 1594,—

\settowidth{\versewidth}{“And every morn by dawning of the day,}
\indentpattern{010100}
\begin{scverse}\begin{patverse}
\vleftofline{“}And every morn by dawning of the day,\\
When Phoebus riseth with a blushing face,\\
Silvanus’ chapel clerks shall chaunt a lay,\\
And play thee \textit{hunt’s-up} in thy resting place.\\
My cot thy chamber, my bosòm thy bed,\\
Shall be appointed for thy sleepy head.”
\end{patverse}
\end{scverse}

Again, in \textit{Wit's Bedlam}, 1617,—

%\settowidth{\versewidth}{“Maurus, last morne, at’s mistress’ window plaid}
\begin{scverse}\vleftofline{“}Maurus, last morne, at’s mistress’ window plaid\\
An hunt’s-up on his lute,” \&c.
\end{scverse}

The following song, which is also taken from Mr. Collier’s manuscript, is of
the character of a love-song:—
\musictitle{The New Hunt’s-up}
\vspace{-\baselineskip}

\settowidth{\versewidth}{The hunt is up, the hunt is up,}
\begin{dcverse}\begin{altverse}
The hunt is up, the hunt is up,\\
Awake, my lady free,\\
The sun hath risen, from out his prison,\\
Beneath the glistering sea.
\end{altverse}

\begin{altverse}
The hunt is up, the hunt is up,\\
Awake, my lady bright,\\
The morning lark is high, to mark\\
The coming of day-light.
\end{altverse}

\begin{altverse}
The hunt is up, the hunt is up,\\
Awake, my lady fair,\\
The kine and sheep, but now asleep,\\
Browse in the morning air.
\end{altverse}

\begin{altverse}
The hunt is up, the hunt is up,\\
Awake, my lady gay,\\
The stars are fled to the ocean bed,\\
And it is now broad day.
\end{altverse}

\begin{altverse}
The hunt is up, the hunt is up,\\
Awake, my lady sheen,\\
The hills look out, and the woods about,\\
Are drest in lovely green.
\end{altverse}

\begin{altverse}
The hunt is up, the hunt is up,\\
Awake, my lady dear,\\
A morn in spring is the sweetest thing\\
Cometh in all the year.
\end{altverse}
\end{dcverse}

\begin{scverse}\begin{altverse}
The hunt is up, the hunt is up,\\
Awake, my lady sweet,\\
I come to thy bower, at this lov'd hour,\\
My own true love to greet. 
\end{altverse}
\end{scverse}
\normalsize
\pagebreak

%%62
%%========================i

The religious parody of \textit{The Hunt is up}, which was written by John Thorne,
has been printed by Mr. Halliwell, at the end of the moral play of \textit{Wit and
Science}, together with other curious songs from the same manuscript (Addit. MS.,
No.~15,233, Brit. Mus.) There are seventeen verses; the first is as follows:—

\settowidth{\versewidth}{“The hunt ys up, the hunt ys up,}
\begin{scverse}
\begin{altverse}
\vleftofline{“}\vleftofline{“}The hunt ys up, the hunt ys up,\\
Loe! it is allmost daye;\\
For Christ our Kyng is cum a huntyng,\\
And browght his deare to staye,” \&c.
\end{altverse}
\end{scverse}
but a more lively performance is contained in “Ane compendious booke of Godly
and Spirituall Songs... with sundrie... ballates changed out of prophaine
Sanges,” \&c., printed by Andro Hart in Edinburgh in 1621. The writer is very
bitter against the Pope, who, he says, never ceased, “under dispence, to get our
pence,” and who sold “remission of sins in auld sheep skins;” and compares
him to the fox of the hunt. The original edition of that book was printed in 1590.

In Queen Elizabeth’s and Lady Neville’s Virginal Books, is a piece, with twelve
variations, by Byrde, called “The Hunt is up,” which is also called “Pescod
Time,” in another part of the former book. It bears no appearance of ever having
been intended for words; certainly the songs in question could not be sung
to it.

A tune called \textit{The Queene’s Majesties new Hunt is up}, is mentioned in Anthony
Munday’s \textit{Banquet of daintye conceits}, 1588; and the ditty he gives, to be sung
to it, called “Women are strongest, but truth overcometh all things,” is in the
same measure as the above, but I have not found any copy of the tune under that
name. In 1565, William Pickering paid 4\textit{d}. for a license to print “a ballett
intituled The Hunte ys up,” \&c. (see \textit{Registers of Stationers’ Company}, p.~129).

\musictitle{Yonder comes a Courteous Knight.}

This is one of King Henry’s Mirth or Freemen’s Songs, in Deutero\-melia, 1609,
and is to be found as a ballad in Wit and Mirth, or Pills to purge Melancholy,
vol.~i.~1698 and 1707, or in vol. iii. of the edition of 1719. The story seems to
have been particularly popular, as there are three ballads of later date upon the
same subject. It is of a young lady who, being alone and unprotected, finds the
too urgent addresses of a knight likely to prove troublesome; and, to escape
from that position, pretends to yield to him, and persuades him to escort her
home; but—

\settowidth{\versewidth}{“When she came to her father’s hall,}
\indentpattern{01014}
\begin{scverse}
\begin{patverse}
\vleftofline{“}When she came to her father’s hall,\\
It was well walled round about,\\
She yode in at the wicket gate,\\
And shut the four-ear’d fool without.\\
Then she sung down, a-down,” \&c.
\end{patverse}
\end{scverse}
The knight, regretting the lost opportunity, expresses himself in very uncourteous
terms on the deceit of women. The ballad is printed in Ritson’s Ancient Songs. 
\pagebreak

%\origpage{63}%63
%%===============================================================================

\musicinfo{Gracefully.}{}

\lilypondfile{lilypond/063-yonder-comes-a-courteous-knight}

\musictitle{Oft have I ridden upon my Grey Nag}

This is evidently a version of the tune called \textit{Dargason}. (See p.~65.) The latter
part differs, but that may be because this copy is taken from Pammelia, 1609,
where three old tunes, “Shall I go walk the woods so wild,” “Robin Hood, Robin
Hood, said Little John,” and this, are arranged to be sung together by three
persons at the same time. Perhaps, the two lines from the Isle of Gulls, which
are quoted at page 64, formed a portion of this song. Only one verse is given in
Pammelia, and I have not succeeded in finding any other copy.

\lilypondfile{lilypond/063-oft-have-i-ridden-upon-my-grey-nag}\normalsize
\pagebreak

%064
%===============================================================================

\musictitle{Dargason.\dcfootnote{\scriptsize %
This tune is inserted in Jones’ \textit{Musical and Poetical
Relics of the Welsh Bards}, p.~129, under the name of “The
melody of Cynwyd;” and some other curious coincidences
occur in the same work. At page 172, the tune called
“The Welcome of the Hostess” is evidently our “Mitter
Rant.” At page 176, the tune called “Flaunting two,”
is, the country dance of “The Hemp Dresser, or the London
Gentlewoman.” At page 129, “The Delight of the
men of Dovey,” appears to he an inferior copy of
“Green Sleeves.” At page 174, is “Hunting the Hare,”
which we also claim. At page 162, “The Monks’ March”
(of which Jones says, “Probably the tune of the Monks
of Bangor, when they marched to Chester, about the year
603,”) is “\textit{General} Monk’s March,” published by Playford, 
and the quick part, “The Rummer;” and at page
142, the air called “White Locks” is evidently Lord
Commissioner \textit{Whitelocke’s} coranto, an account of which,
with the tune, is contained in Sir J. Hawkins’ \textit{History of
Music}, vol. iv. page 51, and in Burney’s \textit{History of Music},
vol. iii. page 378. In several of these, particularly in the
last, which is identified by the second part of the tune
(and especially by a very different version, under the same
name, in Parry’s \textit{Cambrian Harmony}, published about
fifty years ago), there is considerable variation, as may he
expected in tunes traditionally preserved for so long a
time, but their identity admits of little question. In
vol. ii., at p.~25, “The Willow Hymn” is, “By the osiers
so dank.” At p.~44, “The first of August” is, “Come,
jolly Bacchus,” with a little admixture of “In my cottage
near a wood. “At page 33, a tune called “The Britons,”
which is in \textit{The Dancing Master} of 1696, is claimed. At
p.~45, “Mopsy’s Tune, the old way,” is “The Barking
Barber,” and “Prestwich Bells” is “Talk no more of
Whig or Tory,” contained in many collections. At vol. iii.,
p.~15, “The Heiress of Montgomery” is another version
of “As down in the meadows.” At p.~16, “Captain
Corbett” is “Of all comforts I miscarried;” and at p.~49,
“If love’s a sweet passion,” is claimed.” In addition
to these, Mr. Jones has himself noticed a coincidence
between the tune called “The King’s Note,” (vol. iii.)
and “Pastyme with good Company.” Such mistakes will
always occur when an editor relies solely on tradition.
} %end footnote
}
In Ritson’s Ancient Songs, class 4 (from the reign of Edward VI. to Elizabeth)
is “A merry ballad of the Hawthorn tree,” to be sung to the tune of \textit{Donkin
Dargeson}. This curiosity is copied from a miscellaneous collection in the Cotton
Library (Vespasian A. 25), and Ritson remarks, “This tune, whatever it was,
appears to have been in use till after the Restoration.” I have found several
copies of the tune; one is in the Public Library, Cambridge, among Dowland’s
manuscripts. The copy here given is from the Dancing Master, 1650-51, where
it is called Dargason, or the Sedany. The Sedany was a country dance, the figure
of which is described in the \textit{The Triumph of Wit, or Ingenuity displayed}, p.~206.
In Ben Jonson’s \textit{Tale of a Tub}, we find, “But if you get the lass from \textit{Dargison},
what will you do with her?” Gifford, in a note upon this passage, says, “In
some childish book of knight-errantry, which I formerly read, but which I cannot
now recall to mind, there is a dwarf of this name (Dargison), who accompanies a
lady, of great beauty and virtue, through many perilous adventures, as her guard
and guide.” In the \textit{Isle of Gulls}, played by the children of the Revels, in the
Black Fryars, 1606, may be found the following scrap, possibly of the original
ballad:
\settowidth{\versewidth}{An ambling nag, and a-down, a-down,}
\begin{scverse}\vleftofline{“}An ambling nag, and a-down, a-down,\\
We have borne her away to \textit{Dargison}”
\end{scverse}
See also “Oft have I ridden upon my grey nag,” page 63. In the Douce collection
of Ballads (fol. 207), Bodleian Library, as well as in the Pepysian, is a song
called “The Shropshire Wakes, or hey for Christmas, being the delightful sports
of most countries, to the tune of \textit{Dargason}.” It begins thus:
\settowidth{\versewidth}{Young men and maids ‘may dance away,’” }
\begin{scverse}
\begin{altverse}
“Come Robin, Ralph, and little Harry,\\
And merry Thomas to our green;\\
Where we shall meet with Bridget and Sary,\\
And the finest girls that e’er were seen.\\
Then hey for Christmas a once year,\\
When we have cakes, with ale and beer,\\
For at Christmas ‘every day’\\
Young men and maids ‘may dance away,’” \&c. 
\end{altverse}
\end{scverse}
\pagebreak
%065
%===============================================================================

\noindent There are sixteen verses in the song. The tune is one of those which only end
when the singer is exhausted; for although, strictly speaking, it consists of but
eight bars (and in the seventh edition of \textit{The Dancing Master }only eight bars are
printed), yet, from never finishing on the key-note, it seems never to end. Many
of these short eight-bar tunes terminate on the fifth of the key, but when longer
melodies were used, such as sixteen bars, they generally closed with the key-note.
There were, however, exceptions to the rule, especially among dance tunes, which
required frequent repetition.

\musicinfo{Pastoral character.}{A mery Ballet of the hathorne tree.}

\lilypondfile{lilypond/065-Dargason}

\settowidth{\versewidth}{But how, an they chance to cut thee do,}

\begin{dcverse}
The tree made answer by and by,\\
I have cause to grow triumphantly,\\
The sweetest dew that ever be seen,\\
Doth fall on me to keep me green.

Yea, quoth the maid, but where you grow\\
You stand at hand for every blow,\\
Of every man for to be seen,\\
I marvel that you grow so green.

Though many one take flowers from me,\\
And many a branch out of my tree;\\
I have such store they will not be seen,\\
For more and more my twigs grow green.

But how, an they chance to cut thee down,\\
And carry thy branches into the town?\\
Then they will never more be seen\\
To grow again so fresh and green. 
\end{dcverse}

\pagebreak

%066
%===============================================================================

\settowidth{\versewidth}{Though that you do it is no hoot,}
\begin{dcverse}
Though that you do it is no boot,\\
Although they cut me to the root,\\
Next year again I will be seen\\
To hud my branches fresh and green.

And you, fair maid, can not do so,\\
For ‘when your beauty once does go,’\\
Then will it never more be seen,\\
As I with my branches can grow green.

The Maid with that began to blush,\\
And turn’d her from the hawthorn bush;\\
She thought herself so fair and clean,\\
Her beauty still would ever grow green.\\
\hspace{8em}* * * * *\\
But after this never I could hear\\
Of this fair maiden any where,\\
That ever she was in forest seen\\
To talk again with the hawthorn green.
\end{dcverse}


The above will be found in Ritson’s Ancient Songs, in Evans’ Collection of Old
Ballads (vol. i., p.~342, 1810), and in Peele’s Works, vol. ii., p.~256, edited by
Dyce. It is included in the last named work, because in the MS. the name of
“G. Peele” is appended to the song, but by a comparatively modern hand. The
Rev. Alexander Dyce does not believe Peele to have been the author, and Ritson,
who copied from the same manuscript, does not mention his name.

\musictitle{Shall I Go Walk the Woods So Wild?}

This is mentioned in the Life of Sir Peter Carew as one of the Freemen’s Songs,
which he used to sing with Henry VIII.—(See page 52). It must have enjoyed
an extensive and long-continued popularity, for there are three different arrangements
of it in Queen Elizabeth’s Virginal Book, all by Byrde; it is in Lady
Neville’s Virginal Book; in \textit{Pammelia} (1609) it is one of the three tunes that
could be sung together; and it is in \textit{The Dancing Master}, from the first edition,
in 1650, to that of 1690. In the edition of 1650, it is called \textit{Greenwood}, and in
some of the later copies, \textit{Greenwood, or The Huntsman}.

There were probably different words to the tune, because in the Life of Sir
Peter Carew it is called “\textit{As I walked} the woods so wild;” in Lady Neville’s
Virginal Book, “\textit{Will you walk} the woods so wild?” and in \textit{Pammelia}, “\textit{Shall
I go walk},” \&c.

\lilypondfile{lilypond/066-shall-i-go-walk-the-woods-so-wild}

\pagebreak

%067
%===============================================================================

\musictitle{JOHN DORY.}

This celebrated old song is inserted among the \textit{Freemen’s Songs} of three voices
in \textit{Deuteromelia}, 1609. It is also to be found in Playford’s \textit{Musical Companion},
1687, and for one voice in \textit{Wit and Mirth, or Pills to Purge Melancholy}, vol. i.,
1698 and 1707. It is, however, much older than any of these books. Carew,
in his Survey of Cornwall, 1602, p.~135, says, “The prowess of one Nicholas,
son to a widow near Foy, is descanted upon in an \textit{old three-man's song}, namely,
how he fought bravely at sea, with one John Dory (a Genowey, as I conjecture),
set forth by John, the French King, and after much blood shed on both sides, took
and slew him,” \&c. Carew was born in 1555. The only King John of France
died a prisoner in England, in 1364. In the play of \textit{Gammer Gurton's Needle}
there is a song, “I cannot eat but little meat,” which was sung \textit{to the tune of
John Dory}. The play was printed in 1575, but the song appears to be older.
(See page 72). Bishop Corbet thus mentions \textit{John Dory}, with others, in his
“Journey to Fraunce:”

\settowidth{\versewidth}{“But woe is me! the guard, those men of warre,}
\begin{scverse}
\vleftofline{“}But woe is me! the guard, those men of warre,\\
Who but two weapons use, beef and the barre,\\
Begun to gripe me, knowing not the truth,\\
That I had sung \textit{John Dory} in my youth;\\
Or that I knew the day when I could chaunt,\\
\textit{Chevy}, and \textit{Arthur}, or \textit{The Siege of Gaunt}.”
\end{scverse}

Bishop Earle, in his “Character of a Poor Fiddler,” says, “Hunger is the greatest
pains he takes, except a broken head sometimes, and labouring \textit{John Dory}.” In
Fletcher’s comedy \textit{The Chances}, Antonio, a humourous old man, receives a wound,
which he will only suffer to be dressed on condition that the song of \textit{John Dory} be
sung the while, and he gives 10\textit{s}. to the singers. It is again mentioned by
Fletcher in \textit{The Knight of the Burning Pestle}; by Brathwayte in \textit{Drunken
Barnaby's Journal}; in \textit{Vox Borealis, or the Northern Discoverie}, 1641; in some
verses on the Duke of Buckingham, 1628:

\begin{scverse}
\vleftofline{“}Then Viscount Slego telleth a long storie\\
Of the supplies, as if he sung \textit{John Dorie};”
\end{scverse}

\noindent and twice by Gayton, in his \textit{Pleasant Notes upon Don Quixote}, 1654.

A parody was made upon it by Sir John Mennis, on the occasion of Sir John
Suckling’s troop of horse, which he raised for Charles I., running away in the
civil war, and it was much sung by the Parliamentarians at the time. In will be
found in \textit{Wit Restored}, 1658, entitled “Upon Sir John Suckling’s most warlike
preparation for the Scottish War,” and begins—

\begin{scverse}
“Sir John got him an ambling nag.”
\end{scverse}

In the epilogue to a farce called the \textit{Empress of Morocco}, 1674, intended to
ridicule a tragedy of the same name by Elk. Settle, and Sir W. Davenant’s
alteration of \textit{Macbeth} (which had been lately revived with the addition of music
by Mathew Locke), “the most renowned and melodious song of John Dory was
to be heard in the air, sung in parts by spirits, to raise the expectation and charm 
the audience with thoughts sublime and worthy of the heroic scene which follows.” %\origpage{}
It is quoted in \textit{Folly in print}, 1667;
in \textit{Merry Drollery complete}, 1670; and in 
\pagebreak
%%068
%%===============================================================================
many songs. Dryden refers to it, as one of the most hackneyed in his time,
in one of his lampoons:
\settowidth{\versewidth}{“But Sunderland, Godolphin, Lory,}
\begin{scverse}%\vskip -12pt plus 6pt minus 6pt
\vleftofline{“}But Sunderland, Godolphin, Lory,\\
These will appear such chits in story,\\
’Twill turn all politics to jest,\\
To be repeated, like \textit{John Dory},\\
When fiddlers sing at feasts.”
\end{scverse}
The above lines were also printed under the name of the “Earl of Rochester.”

The name of the fish called John Dory, corrupted from dorée or dorn, is
another proof of the great popularity of this song.

\musicinfo{Cheerfully.}{}

\lilypondfile{lilypond/068-john-dory}

\settowidth{\versewidth}{And when John Dory to Paris was come}

\begin{dcverse}\footnotesizerrr\begin{altverse}
And when John Dory to Paris was come,\\
A little before the gate-a;\\
John Dory was fitted, the porter was witted,\\
To let him in thereat-a.
\end{altverse}

\begin{altverse}
The first man that John Dory did meet,\\
Was good King John of France-a:\\
John Dory could well of his courtesie,\\
But fell down in a trance-a.
\end{altverse}

\begin{altverse}
A pardon, a pardon, my liege and king,\\
For my merry men and me-a:\\
And all the churls in merry England\\
I’ll bring them bound to thee-a.
\end{altverse}

\begin{altverse}
And Nichol was then a Cornish man,\\
A little beside Bohyde-a;\\
And he manned forth a good black bark,\\
With fifty good oars on a side-a.
\end{altverse}

\begin{altverse}
Run up, my boy, into the main top,\\
And look what thou canst spy-a;\\
Who, ho! who, ho! a good ship I do see,\\
I trow it be John Dory-a.
\end{altverse}

\begin{altverse}
They hoist their sails, both top and top,\\
The mizen and all was tried-a;\\
And every man stood to his lot,\\
Whatever should betide-a.
\end{altverse}

\begin{altverse}
The roaring cannons then were plied,\\
And dub-a-dub went the drum-a;\\
The braying trumpets loud they cried,\\
To courage both all and some-a.
\end{altverse}

\begin{altverse}
The grappling hooks were brought at length,\\
The brown bill and the sword-a:\\
John Dory at length, for all his strength,\\
Was clapt fast under board-a. 
\end{altverse}
\end{dcverse}\normalsize
\pagebreak

%%069
%%===============================================================================

\musictitle{SELLENGER’S ROUND, or THE BEGINNING OF THE WORLD.}

\musicinfo{Smoothly and in moderate time.}{}

\lilypondfile{lilypond/069-sellengers-round-or-the-beginning-of-the-world}

\pagebreak


%%070
%%===============================================================================

This tune, which Sir John Hawkins thought to be “the oldest country-dance
tune now extant” (an opinion to which I do not subscribe), is to be found in
Queen Elizabeth’s and Lady Neville’s Virginal Books, in \textit{Music’s Handmaid},
1678, \&c. It is difficult to say from whom it derived its name. It might be from
“Sir Thomas Sellynger,” who was buried in St. George’s Chapel, Windsor,
before the year 1475, as appears by a brass plate there; or from Sir Antony
St. Leger, whom Henry VIII. appointed Lord Deputy of Ireland in 1540.

In \textit{Bacchus' Bountie} (4to., 1593), we find this passage: “While thus they
tippled, the fiddler he fiddled, and the pots danced for joy the old hop-about
commonly called \textit{Sellengar’s Round}.” In Middleton’s \textit{Father Hubburd’s Tales}
(1604):—“Do but imagine now what a sad Christmas we all kept in the country,
without either carols, wassail bowls, \textit{dancing of Sellenger's Round in moonshine
nights about Maypoles}, shoeing the mare, hoodman-blind, hot cockles, or any of our
Christmas gambols,—no, not not so much as choosing king and queen on Twelfth
Night!” In Heywood’s \textit{Fair Maid of the West}, part ii.:—“They have so tired
me with their moriscoes [morris dances], and I have so tickled them with our
country dances, \textit{Sellenger’s Round} and \textit{Tom Tiler}. We have so fiddled it!”

A curious reason for the second name to this tune is given in the comedy
of \textit{Lingua}, 1607. \textit{Anamnestes}: “By the same token the first tune the planets
played; I remember Venus, the treble, ran sweet division upon Saturn, the base.
The first tune they played was \textit{Sellenger’s Round}, in memory whereof, ever since,
it hath been called \textit{The Beginning of the World}.” On this, \textit{Common Sense} asks:
“How comes it we hear it not now? and \textit{Memory}, another of the characters,
says: “Our ears are \textit{so well acquainted with the sound}, that we never mark it.”
In Shirley’s \textit{Lady of Pleasure}, Lady Bornwell says that, “to hear a fellow make
himself merry and his horse with whistling \textit{Sellenger’s Round}, and to observe with
what solemnity they keep their wakes, moriscoes, and Whitsun-ales, are the \textit{only}
amusements of the country.”

It is mentioned as \textit{The Beginning of the World} by Deloney in his history of
Jack of Newbury, and .the times to which he refers are those of Henry VIII.;
but, so great was its popularity, that it is mentioned three or four times by
Heywood; also by Ben Jonson, by Taylor the water-poet, by Fletcher, Shirley,
Brome, Farquhar, Wycherley, Morley (1597), Clieveland (1677), Marmion (1641);
by the author of \textit{The Return from Parnassus}, and by many other writers.

There is a wood-cut of a number of young men and women dancing \textit{Sellenger’s
Round}, with hands joined, round a Maypole, on the title page of a black letter
garland, called “The new Crown Garland of princely pastime and mirth,” printed
by J. Back, on London Bridge. In the centre are two musicians, the one playing
the fiddle, the other the pipe, with the inscription, “Hey for Sellenger’s Round!”
above them.

As the dance was so extremely popular, I shall, in this instance, give the figure
from the \textit{The Dancing Master} of 1670, where it is described as a round dance
“for as many as will.”

“Take hands, and go round twice: \pagebreak
back again. All set and turn sides: that 
%%071
%%===============================================================================
again. Lead all in a double forward and back: that again. Two singles and a
double back, set and turn single: that again. Sides all: that again. Arms all:
that again. As before, as before.” Country dances were formerly danced as
often in circles as in parallel lines.

\DFNsingle

The following songs were sung to the tune:—“The merry wooing of Robin and
Joan, the West-country Lovers, to the tune of the Beginning of the World, or
Sellenger’s Round.”—\textit{Roxburgh Collection}. “The Fair Maid of Islington, or the
London Vintner over-reached,” in the \textit{Bagford Collection}. “Robin’s Courtship,”
in \textit{Wit Restored}, 1658.

As a specimen of old harmony, I have added the arrangement of Sellenger’s
Round by Byrd, from Queen Elizabeth’s Virginal Book. Having an instrument
that would not sustain the tone (for the virginals, like the harpsichord, only
twitted the wires with a quill) it is curious to see how he has filled up the harmony
by an inner part, that seems intended to imitate the prancing of the hobby-horse.
The hobby-horse was the usual attendant on May-day and May Games.

\musicinfo{In moderate time.}{with the old harmony by byrd.}

\lilypondfile{lilypond/071-sellengers-round-byrd-harmony}

\footnotetext[1]{\centering Hobby horse.}
\pagebreak

%%072
%%===============================================================================

\musictitle{I CANNOT EAT BUT LITTLE MEAT.}

This song was sung in “a right pithy, pleasant, and merry comedy,” called
\textit{Gammer Gurton’s Needle}, which was printed in 1575, but the Rev. Alex. Dyce
has given a copy of double length from a manuscript in his possession, and
“certainly of an earlier date than the play.” It may be seen in his account of
Skelton and his writings, vol. i., p.~7. I have selected four from the eight
verses, as sufficiently long for singing. Warton calls it “the first drinking song of
any merit in our language.” In early dramas it was the custom to sing old songs,
or to play old tunes, both at the commencement and at the end of the acts. For
instance, in \textit{Summer’s Last Will and Testament}, which was performed in 1593,
the direction to the actors in the Prologue is to begin the play with “a fit
of mirth and an old song:” and at the end of the comedy, \textit{Ram Alley}, “strike up
music; let’s have an old song.” In \textit{Peele’s Arraignment of Paris}, Venus “singeth
an old song, called \textit{The wooing of Colman}.” In Marston’s \textit{Antonio and Mellida},
Feliche sings “the old ballad, \textit{And was not good king Solomon}.” To these instances
many others might be added; indeed, in the very play (\textit{Gammer Gurto}n),
at the end of the second act, Diccon says:—

\settowidth{\versewidth}{“In the mean time, fellows, pipe up your fiddles, I say take them}
\begin{scverse}
“In the mean time, fellows, pipe up your fiddles, I say take them\\
And let your friends have such mirth as ye can make them.”
\end{scverse}

The tune is printed in Stafford Smith’s Musica Antiqua, and in Ritson’s English
Songs. Ritson says: “Set, four parts in one, by Mr. Walker, before the year
1600.” And Smith, not knowing, I suppose, who Mr. Walker was, seems to have
guessed Weelkes; but it is the old tune of John Dory in common time.

\musicinfo{In moderate time and well marked.}{}

\lilypondfile{lilypond/072-i-cannot-eat-but-little-meat}


\settowidth{\versewidth}{But belly, God send thee good ale enough,}
\begin{dcverse}\begin{altverse}
Though I go bare, take ye no care,\\
‘For I am never’ cold:\\
I stuff my skin so full within\\
Of jolly good ale and old.\\
Back and side, go bare, go bare,\\
Both foot and hand go cold:\\
But belly, God send thee good ale enough,\\
Whether it be new or old.
\end{altverse}

\begin{altverse}
I love no roast, but a nut-brown toast,\\
And a crab laid in the fire,\\
A little bread shall do me stead,\\
Much bread I never desire.\\
No frost nor snow, nor wind, I trow,\\
Can hurt me, if it would;\\
I am so wrapp’d, so thoroughly lapp’d\\
With jolly good ale and old.\\
\hspace{6em}Back and side, \&c. 
\end{altverse}
\end{dcverse}

\pagebreak

%%073
%%===============================================================================

\settowidth{\versewidth}{I care right nought, I take no thought}
\begin{dcverse}
\begin{altverse}
I care right nought, I take no thought\\
For clothes to keep me warm,\\
Have I good drink I surely think\\
‘That none’ can do me harm.\\
For truly then I fear no man,\\
‘Though never he’ so bold,\\
When I am arm’d and thoroughly warm’d\\
With jolly good ale and old.\\
\hspace{6em}Back and side, \&c.
\end{altverse}

\begin{altverse}
Now let them drink till they nod and wink,\\
Even as good fellows should do,\\
They shall not miss to have the bliss\\
Good ale doth bring men to;\\
And all poor souls that scour black bowls,\\
Or have them lustily troled,\\
God save the lives of them and their wives,\\
Whether they be young or old.\\
\hspace{6em}Back and side, \&c.
\end{altverse}
\end{dcverse}
\normalsize


\musictitle{Hanskin, or Half Hannikin.}

In Queen Elizabeth’s Virginal Book there is a tune called \textit{Hanskin}, and in all
the early editions of \textit{The Dancing Master}, viz., from 1650 to 1690, one called
\textit{Half Hannikin}. Hankin or Hannikin was the common name of a clown:

\settowidth{\versewidth}{“Thus for her love and loss poor \textit{Hankin} dies;}
\begin{scverse}
“Thus for her love and loss poor \textit{Hankin} dies;\\
His amorous soul down flies\\
To th’ bottom of the cellar, there to dwell:\\
Susan, farewell, farewell!”—\textit{Musarum Delicicæ}, 1655.
\end{scverse}

And \textit{Hankin Booby} was used as term of contempt. Nash, meaning to call his
opponent a Welsh clown, calls him a “Gobin a Grace ap Hannikin,” and says,
“No vulgar respects have I, what Hoppenny Hoe and his fellow \textit{Hankin Booby}
think of me.” (\textit{Have with you to Saffron-Waldon}, 1596.)

We find \textit{Hankin Booby} mentioned as a tune in the interlude of Thersytes,
which was written in 1537:

\settowidth{\versewidth}{“And we wyll have minstrelsy}
\begin{scverse}
“And we wyll have minstrelsy\\
That shall pype \textit{Hankin boby}.”
\end{scverse}

\noindent Skelton, in his \textit{Ware the Hauke}, says:
\settowidth{\versewidth}{“With troll, cytrace, and trovy}
\begin{dcverse}“With troll, cytrace, and trovy,\\
They ranged, \textit{hankin bovy},\\
My churche all aboute.\\
This fawconer then gan showte,\\
These be my gospellers,

These be my pystillers, [epistlers]\\
These be my querysters [choristers]\\
To help me to synge,\\
My hawkes to mattens rynge.\\
\hspace{-1em}\textit{Skelton’s Works, Ed. Dyce}, vol. i., p.~159.
\end{dcverse}

\noindent By an extract from Sir H. Herbert’s office-book of revels and plays performed at
Whitehall at Christmas, 1622-3, quoted by Mr. Collier, in his Annals of the
Stage, we find that on Sunday, 19th Jan., 1623, after the performance of Ben
Jonson’s masque, \textit{Time Vindicated}, “The Prince did lead the measures with the
French Ambassador’s wife,” and “the measures, braules, corrantos, and galliards,
being ended, the masquers, with the ladies, did daunce two countrey dances,
namely, \textit{The Soldier’s Marche} and \textit{Huff Hamukin}.” I believe that by \textit{Huff
Hamukin}, \textit{Half Hannikin} is intended, the letters are so nearly alike in form, and
might be so easily mistaken. In Brome’s \textit{Jovial Crew}, 1652,—“Our father is so
pensive that he makes us even sick of his sadness, that were wont to ‘See my	
gossip’s cock to-day,’ mould cocklebread, daunce \textit{Clutterdepouch} and \textit{Hannykin
booby}, bind barrels, or do anything before him, and he would laugh at us.”

The tune called \textit{Hanskin} in Queen
Elizabeth’s Virginal Book is the same as 

\pagebreak
%%074
%%===============================================================================
“Jog on, the foot-path way,” and will be found in this collection among the airs
that are mentioned by Shakspeare. The following is \textit{Half Hannikin}, from \textit{The
Dancing Master}.

\lilypondfile{lilypond/074-half-hannikin}

\musictitle{Malt’s Come Down.}

This is one of the tunes in Queen Elizabeth’s Virginal Book, where it is
arranged by Byrd. The words are from \textit{Deuteromelia}, 1609, but it appears that
Ravenscroft, in arranging it as a round, has taken only half the tune.

\lilypondfile{lilypond/074-malts-come-down}

\settowidth{\versewidth}{The greatest drunkards in the town}
\indentpattern{006}
\begin{scverse}
\begin{patverse}
The greatest drunkards in the town\\
Are very glad that malt’s come down.\\
Malt’s come down, \&c.  
\end{patverse}
\end{scverse}
\pagebreak

%%075
%%===============================================================================

\musictitle{of all the birds.}

In Beaumont and Fletcher’s play, \textit{The Knight of the Burning Pestle}, Old
Merrythought sings many snatches of old songs, and among others—

\settowidth{\versewidth}{And who gave thee this jolly red nose? }
\begin{scverse}
“Nose, nose, jolly red nose\\
And who gave thee this jolly red nose? \\
Cinnamon, ginger, nutmegs and cloves\\
And they gave me this jolly red nose;”
\end{scverse}

\noindent which are the four last lines of this song. It is one of the King Henry’s Mirth
or Freemen’s Songs in \textit{Deuteromelia}, 1609.

\lilypondfile{lilypond/075-of-all-the-birds}

\pagebreak
%%076
%%===============================================================================

\musictitle{Who's the fool now?}

This tune is in Queen Elizabeth’s Virginal Book, and it is one of the Freemen’s
Songs in \textit{Deuteromelia}, 1609. It was entered on the books of the Stationer’s
Company as a ballad in 1588, when Thomas Orwyn had a license to print it; and
it is alluded to in Dekker’s comedy, \textit{Old Fortunatus}, where \textit{Shadow} says: “Only
to make other idiots laugh, and wise men to cry ‘\textit{Who's the fool now}?’” which
is the burden of every verse. It is thought to be a satire upon those who tell
wonderful stories.

\lilypondfile{lilypond/076-whos-the-fool-now}

\settowidth{\versewidth}{Thou hast well drunken, man—}
\indentpattern{04040014}
\begin{dcverse}\begin{patverse}
I saw the man in the moon;\\
Fie! man, fie!\\
I saw the man in the moon;\\
Who’s the fool now?\\
I saw the man in the moon\\
Clouting of St. Peter’s shoon;\\
Thou hast well drunken, man—\\
Who’s the fool now?
\end{patverse}

\begin{patverse}
I saw a hare chase a hound;\\
Fie! man, fie!\\
I saw a hare chase a hound;\\
Who’s the fool now?\\
I saw a hare chase a hound,\\
Twenty miles above the ground;\\
Thou hast well drunken, man—\\
Who’s the fool now?
\end{patverse}

\begin{patverse}
I saw a goose ring a hog;\\
Fie! man, fie!\\
I saw a goose ring a hog;\\
Who’s the fool now?\\
I saw a goose ring a hog,\\
And a snail bite a dog;\\
Tho hast well drunken, man—\\
Who’s the fool now?
\end{patverse}

\begin{patverse}
I saw a mouse catch a cat;\\
Fie! man, fie!\\
I saw a mouse catch a cat;\\
Who’s the fool now?\\
I saw a mouse catch a cat,\\
And the cheese eat the rat;\\
Thou hast well drunken, man—\\
Who’s the fool now? 
\end{patverse}
\end{dcverse}
\pagebreak

%%077
%%===============================================================================
\DFNdouble

\musictitle{We be soldiers three.}

This is also one of the King Henry’s Mirth or Freemen’s Songs in \textit{Deuteromelia},
1609, and will be found as a song in \textit{Wit and Mirth, or Pills to Purge
Melancholy}, vol. i., 1698 and 1707.

\lilypondfile{lilypond/077-we-be-soldiers-three}

\settowidth{\versewidth}{To all good fellows, wherever they be,}
\begin{dcverse}
\begin{altverse}
Here, good fellow, I drink to thee,\\
\textit{Pardona moy, je vous an pree;}\dcfootnote{ %
“These \textit{pardonnez-moy's} who stand so much on the
new form.”—\textit{Romeo and Juliet}, act ii., sc. 4. Dr.~Johnson
in a note says; “\textit{Pardonnez moi} became the language
of doubt or hesitation among men of the sword, when the
point of honour was grown so delicate that no other mode
of contradiction would be endured.”
} \\ %end footnote
To all good fellows, wherever they be,\\
With never a penny of money.
\end{altverse}

\begin{altverse}
And he that will not pledge me this,\\
\textit{Pardona moy, je vous an pree,}\\
Pays for the shot whatever it is,\\
With never a penny of money.
\end{altverse}
\end{dcverse}


\begin{scverse}
\begin{altverse}
Charge it again, boy, charge it again,\\
\textit{Pardona moy, je vous an pree;}\\
As long as there is any ink in thy pen,\\
With never a penny of money.
\end{altverse}
\end{scverse}



\musictitle{We be Three Poor Mariners}

This is one of the King Henry’s Mirth or Freemen’s Songs in \textit{Deuteromelia},
1609, and is to be found as a dance tune in the Skene MS. (about 1630), called
\textit{Brangill of Poictu},—\ie, Branle, or Braule of Poictu.

Braules\dcfootnote{ %
Braules, which, Mr. M. Mason observes, seem to be
what we now call cotillons, are described by Philips as
“a kind of dance in which several persons danced together
in a ring, holding one another by the hand.” In Marston’s
play of \textit{The Malcontent} there is a minute, but perhaps
not now very intelligible description of the figures. See
Dodsley's Collection of old Plays, vol.~iv. Braules are
alluded to by Shakespeare, Ben Jonson, Massinger, and
others.
} %end footnote
were dances much in vogue with the upper classes during the sixteenth
and seventeenth centuries. Their being danced at Whitehall in 1623, has
been mentioned at page 73; and Pepys speaks of them at the Court of Charles II.
Branle de Poictu is explained by Morley (1597) as meaning the Double Branle,
in contradistinction to the French Branle, or Branle-Simple.

Another Branle de Poictu (quite a different tune) will be found in the Straloch
Manuscript, for the name was given to \pagebreak any air used for the dance. It was so 
%%078
%%===============================================================================
usual in England, formerly, to make dances out of such song and ballad tunes as
were of a sufficiently cheerful character, that nearly \textit{every} air in the \textit{first} edition
of \textit{The Dancing Master}, 1650-51, can be proved to be that of a song or “ballet”
of earlier date than the book. It has for that reason been so valuable an aid in
the present collection. About 1690, tunes composed \textit{expressly} for dancing were
becoming more general, and in the editions of \textit{The Dancing Master} from 1715
to 1728, the song and the dance tunes are nearly equally divided.

\lilypondfile{lilypond/078-we-be-three-poor-mariners}

\settowidth{\versewidth}{\vin Come pledge me on the ground, aground, aground.}
\begin{scverse}\begin{altverse}
We care not for those martial men\\
That do our states disdain;\\
But we care for the merchantmen\\
Who do our states maintain.\\
To them we dance this round, around, around,\\
To them we dance this round;\\
And he that is a bully [jolly] boy,\\
Come pledge me on the ground, aground, aground. 
\end{altverse}
\end{scverse}
\pagebreak

%%079
%%===============================================================================

\musictitle{My Little Pretty One}

This ancient melody is also transcribed from a MSS. of the time of Henry the
Eighth (No. 4900, Additional MS., Brit. Mus.). The original is, as usual, without
bars, but with an accompaniment in tablature for the lute. In the same
volume are songs by John Taverner, Shepherde, Heywood, \&c. It has the same
peculiarity as the dance tune at page 27, each part consisting of nine bars. A
song called “My little pretty one” is in the Koxburgh Collection of Ballads,
“to a pleasant new tune,” but the measure is different.

\lilypondfile{lilypond/079-my-little-pretty-one}

\musictitle{Robin, Lend to Me thy Bow}

This song is still known in some parts of the country, and was written down for
me by a friend, in Leicestershire, some years ago. In the “very mery and pithie
commedie” called \textit{The longer thou livest the more fool thou art}, there is a
stage direction—“Here entreth Moros, counterfaiting a vaine gesture and foolish
countenance, synging the foote [burden] of many songes, as fooles were wont.”
Among the burdens is the following:—

\settowidth{\versewidth}{Robin, the bow, \textit{Robin, lend to me thy bow-a}.”}
\begin{scverse}
\vleftofline{“}Robin, lende me thy bowe, thy bowe,\\
Robin, the bow, \textit{Robin, lend to me thy bow-a}.” 
\end{scverse}
\pagebreak

%%080
%%===============================================================================

The play was entered at Stationers’ Hall in 1568-9. “That it was a popular
song in the beginning of Queen Elizabeth’s reign appears also from its being
mentioned, amongst others, in a curious old musical piece (MS. Harl. 7578),
containing the description and praises of the city of Durham, written about that
time.” It is to be found as one of the “pleasant roundelayes” in \textit{Pammelia},
1609, and has likewise been printed by Ritson, in his Ancient Songs. The tune
differs slightly from the copy in \textit{Pammelia}, but I think for the better.

\DFNsingle

\musicinfo{Smoothly and slow.}{}

\lilypondfile{lilypond/080-robin-lend-to-me-thy-bow}

\settowidth{\versewidth}{And thou shalt have my hawke, my hound, and eke my bow,}
\begin{scverse}
\begin{altverse}
And whither will thy Lady go?\\
Sweet Wilkin, tell it unto me;\\
And thou shalt have my hawke, my hound, and eke my bow,\\
To wait on thy Lady.
\end{altverse}

\begin{altverse}
My Lady will to Uppingham,\dcfootnote{ %a
\centering A market-town in Rutlandshire.
} \\%end footnote
To Uppingham forsooth will shee;\\
And I myself appointed for to be the man\\
To wait on my Lady.
\end{altverse}

\begin{altverse}
Adieu, good Wilkin, all beshrewde,\\
Thy hunting nothing pleaseth mee;\\
But yet beware thy babling hounds stray not abroad\\
For ang’ring of thy Lady.
\end{altverse}

\begin{altverse}
My hounds shall be led in the line,\\
So well I can assure it thee;\\
Unless by straine of view some pursue I may finde,\\
To please my sweet Ladye.
\end{altverse}

\begin{altverse}
With that the Lady shee came in,\\
And will’d them all for to agree;\\
For honest hunting never was accounted sinne.\\
Nor never shall for mee.
\end{altverse}
\end{scverse}

\pagebreak

%%081
%%===============================================================================

\musictitle{Who Liveth So Merry in All This Land?}

This is also one of the \textit{King Henry’s Mirth or Freemen’s Songs}, in \textit{Deuteromelia}.
In the first year of the Registers of the Stationers’ Company (1557-58)
there is an entry of a license to Mr. John Wallye and Mrs. Toye to print a
“Ballette” called

\settowidth{\versewidth}{Who lyve so mery and make such sporte,}
\begin{scverse}
\vleftofline{“}Who lyve so mery and make such sporte,\\
As thay that be of the poorest sorte?”
\end{scverse}

These lines will be found in the last verse of the song, and were probably printed
at the head of it as the title. Ballets were songs of a cheerful character, which
being “sung to a ditty may likewise be danced.” So the “Merry Ballet of the
Hawthorn Tree” (see page 64), was to be sung to the tune of \textit{Dargason}, which is
also mentioned as a dance tune.

The following song will also be found in \textit{Wit and Drollery, Jovial Poems}, p.~252,
and in \textit{Wit and Mirth, or\textit{ Pills to purge Melancholy}}, vol. i., 1698 and 1707. In
\textit{Wit and Drollery}, as well as in \textit{Deuteromelia}, the third and fourth lines of each
verse are marked to be sung in chorus.

\musicinfo{Moderate time.}{}

\lilypondfile[staffsize=16]{lilypond/081-who-liveth-so-merry-in-all-this-land}

\settowidth{\versewidth}{With his pretty, sweet wife he maketh full merry.}
\begin{scverse}
The broom-man maketh his living most sweet.\\
With carrying of brooms from street to street.\\
\vleftofline{\textit{Chorus}.—}Who would desire a pleasanter thing\\
Than all the day long to do nothing but sing?

The chimney-sweeper all the long day,\\
He singeth and sweepeth the soot away;\\
\vleftofline{\textit{Ch}.—}Yet when he comes home, although he be weary,\\
With his pretty, sweet wife he maketh full merry.

The cobbler he sits cobbling till noon,\\
And cobbles his shoes till they be done;\\
\vleftofline{\textit{Ch}.—}Yet doth he not fear, and so doth say,\\
For he knows that his work will soon decay.

The merchantman he doth sail on the seas,\\
And lie on the ship-board with little ease;\\
\vleftofline{\textit{Ch}.—}For always he doubts that the rocks are near,—\\
How can he he merry and make good cheer?
\end{scverse}
\pagebreak

%%82
%%===============================================================================
\DFNsingle

\settowidth{\versewidth}{And when he comes home he serveth his sow;    }
\begin{scverse}
The husbandman all day goeth to plough,\\
And when he comes home he serveth his sow;\\
\vleftofline{\textit{Ch}.—}He moileth and toileth all the long year,—\\
How can he he merry and make good cheer?

The serving-man waiteth from street to street,\\
Either blowing his nails or beating his feet;\\
\vleftofline{\textit{Ch}.—}Yet all that serves for, four angels\dcfootnote{\tiny %a
The angel was a gold coin worth about ten shillings, so named 
from having the representation of an angel upon it.
} %end footnote
a year,\\
Impossible ’tis that he make good cheer.


Who liveth so merry and maketh such sport\\
As those that be of the poorest sort?\\
\vleftofline{\textit{Ch}.—}The poorest sort, wheresoever they be,\\
They gather together by one, two, and three.
\end{scverse}

\begin{center}\setlength{\fboxrule}{0pt}
\framebox{
	\small\begin{minipage}{0.52\linewidth}
	And every man will spend his penny,
	
What makes such a shot among a great many.
	\end{minipage}}
	\huge \} \small\textit{bis}. \normalsize
\end{center}

\musictitle{To-Morrow the Fox Will Come to Town, or Trenchmore.}

In \textit{The Dancing Master} this tune is called \textit{Trenchmore}. In \textit{Deuteromelia} it is
one of the \textit{King Henry’s Mirth or Freemen’s Songs}, under the name of “Tomorrow
the fox will come to town.”

In a Morality, by William Bulleyn, called \textit{A Dialogue both pleasant and pietyfull,
wherein is a goodly regimen against the fever pestilence}, \&c., 1564, a minstrel
is thus described: “There is one lately come into the hall, in a green Kendal coat,
with yellow hose; a beard of the same colour, only upon the upper lip; a russet
hat, with a great plume of strange feathers; and a brave scarf about his neck;
in cut buskins. He is playing at the \textit{trea trippe} with our host’s son; he playeth
trick upon the gittern, daunces \textit{Trenchmore} and \textit{Heie de Gie}, and telleth news
from Terra Florida.”

Taylor, the water-poet, in \textit{A Merry Wherry-ferry Voyage}, says:
\settowidth{\versewidth}{Heigh, \textit{to the tune of Trenchmore} I could write}
\begin{scverse}
\vleftofline{“}Heigh, \textit{to the tune of Trenchmore} I could write\\
The valiant men of Cromer’s sad affright;”
\end{scverse}
\noindent and in \textit{A Navy of Land Ships}, 1627, “Nimble-heel’d mariners, like so many
dancers, capering a morisco [morris dance], or \textit{Trenchmore} of forty miles long,
to the tune of ‘Dusty, my dear,’ ‘Dirty, come thou to me,’ ‘Dun out of the mire,’
or ‘I wail in woe and plunge in pain:’ all these dances have no other music.”
Deloney, in his \textit{History of the gentle craft}, 1598, says: “like one dancing the
\textit{Trenchmore}, he stamp’d up and down the yard, holding his hips in his hands.”

Burton, in his \textit{Anatomy of Melancholy}, 1621, says that mankind are at no
period of their lives insensible to dancing. “Who can withstand it? be we young
or old, though our teeth shake in our heads like Virginal Jacks, or stand parallel
asunder like the arches of a bridge,—there is no remedy: we must dance \textit{Trenchmore}
over tables, chairs, and stools.” The following amusing description is from
Selden’s \textit{Table Talk}:

“The court of England is much alter’d. At a solemn dancing, first you had the
grave measures, then the corantoes and the galliards, and this kept up with ceremony; \pagebreak
and \textit{at length to Trenchmore} and \textit{the Cushion Dance}: then all the company dances, 
%%083
%%===============================================================================
lord and groom, lady and kitchen maid, no distinction. So in our court in Queen
Elizabeth’s time, \textit{gravity and state were kept up}. In King James’s time things were
pretty well, but in King Charles’s time, there has been \textit{nothing but} Trenchmore and
the Cushion Dance, omnium gatherum, tolly polly, hoite come toite.”



\textit{Trenchmore} is mentioned also in Stephen Gosson’s Schoole of Abuse, 1579; in
Heywood’s \textit{A Woman Killed with Kindness}, 1600; in Chapman’s \textit{Wit of a
Woman}, 1604; in Barry’s \textit{Ram Alley}, 1611; in Beaumont and Fletcher’s \textit{Island
Princess}; in Weelkes’ \textit{Ayres or Phantasticke Sprites}, 1608; and in 1728 was
still to he found in \textit{The Dancing Master}. In the comedy of \textit{The Rehearsal},
1672, the earth, sun, and moon, are made to dance \textit{the Hey} to the tune of
\textit{Trenchmore}.

Several political songs were sung to it, one of which is in the collection of
“Poems on Affairs of State, from 1640 to 1704.” In the Roxburghe Collection
of Ballads is one called “The West-country Jigg, or a Trenchmore Galliard,”
“Four-and-twenty lasses went over Trenchmore Lee.”

The following is the song in \textit{Deuteromelia}.

\musicinfo{Moderate time.}{}

\lilypondfile{lilypond/083-tomorrow-the-fox-will-come-to-town}
\pagebreak


%%084
%%===============================================================================
\DFNdouble

\settowidth{\versewidth}{He’ll steal the cock out from his flock,}
\indentpattern{01014}
\begin{dcverse}\begin{patverse}
He’ll steal the cock out from his flock,\\
Keep, keep, keep, keep, keep;\\
He'll steal the cock e’en from his flock,\\
O keep you all well there.\\
I must desire you, \&c.
\end{patverse}

\begin{patverse}
He’ll steal the hen out of the pen,\\
Keep, keep, \&c.;\\
He’ll steal the hen out of the pen,\\
O keep you all well there.\\
I must desire you, \&c.
\end{patverse}

\begin{patverse}
He’ll steal the duck out of the brook,\\
Keep, keep, \&c.;\\
He’ll steal the duck out of the brook,\\
O keep you all well there.\\
I must desire you, \&c.
\end{patverse}

\begin{patverse}
He’ll steal the lamb e’en from his dam,\\
Keep, keep, \&c.;\\
He’ll steal the lamb e’en from his dam,\\
O keep you all well there.\\
I must desire you, \&c.
\end{patverse}
\end{dcverse}

\musictitle{The Shaking of the Sheet, or the Dance of Death.}

This is frequently mentioned by writers in the sixteenth and seventeenth centuries, 
both as a country dance and as a ballad tune. In the recently-discovered
play of \textit{Misogonus}, produced about 1560,\dcfootnote{ %a
See Collier’s \textit{History of Early Dramatic Poetry}, v. 2,
p.~474.
} %end footnote 
\textit{The Shaking of the Sheets}, \textit{The Vicar
of St. Fools}, and \textit{The Catching of Quails}, are mentioned as country dances.\dcfootnote{ %b
Sometimes it is called \textit{The Night Piece}, or \textit{The Shaking
of the Sheets}.
} %end footnote
There is a manuscript copy of the ballad in the British Museum (Add. MSS.
No. 15,225), in which it is ascribed to Thomas Hill; and printed copies, in black
letter, are to be found in the Roxburghe Collection (i., 499), and in that of
Anthony à Wood, in the Ashmolean Museum, Oxford (vol. 401., f. 60). In
1568-9, it was entered at Stationers’ Hall to John Awdelay (see \textit{Collier’s
Extracts}, vol. i., p.~195).

\textit{Dance after my pipe}, which is the second title , of the ballad, seems to have
been a proverbial expression. In Ben Jonson’s \textit{Every man out of his humour},
Saviolina says: “Nay, I cannot stay to \textit{dance after your pipe}.” In \textit{Vox Borealis},
1641,—“I would teach them to sing another song, and make them \textit{dance after
my pipe}, ere I had done with them.” And in Middleton’s \textit{The World Lost at
Tennis},—“If I should \textit{dance after your pipe} I should soon dance to the devil;”
and so in many other instances.

In \textit{The Meeting of Gallants at an Ordinary}, the host, describing a young man
who died of the plague, in London, in 1603, says: “But this youngster \textit{daunced
the shaking of one sheete} within a few daies after” (Percy Soc, Reprint, p.~20);
and in \textit{A West-country Jigg}, or a \textit{Trenchmore Galliard}, verse 5:

\settowidth{\versewidth}{“The piper he struck up,}
\begin{scverse}
\begin{altverse}
“The piper he struck up,\\
And merrily he did play\\
\textit{The Shaking of the Sheets},\\
And eke \textit{The Irish Hay}.”
\end{altverse}
\end{scverse}

The tune is also mentioned in Lilly’s \textit{Pappe with a Hatchet}, 1589; in Gosson’s
\textit{Schoole of Abuse}, 1579; by Rowley, Middleton, Taylor the water-poet, Marston,
Massinger, Heywood, Dekker, Shirley, \&c., \&c.

There are two tunes under this name, the one in William Ballet’s Lute Book,
which is the same as printed by Sir John Hawkins in his \textit{History of Music}
(vol. 2, p.~934, 8vo. edit.); the other, and in all probability the more popular one,
is contained in numerous publications,\dcfootnote{ %
The tune of \textit{The Catching of Quails} is also in
\textit{The Dancing Master}.
} %end footnote 
from \textit{The Dancing Master} of 1650-51, to
\textit{The Vocal Enchantress} of 1783. 

\pagebreak




%%085
%%===============================================================================

Many ballads were sung to it, and among them, \textit{King Olfrey and the old Abbot},
which is on the same story as \textit{King John and the Abbot of Canterbury}; and \textit{The
Song of the Caps}, in the Roxburgh Collection, which is also, in an altered form,
in \textit{Wit and Mirth, or Pills to Purge Melancholy}.

The following ballad is from a black-letter copy, in the Ashmolean Museum.

\begin{center}\uppercase{The doleful dance and song of death.}

\textsc{Intituled \uppercase{dance after my pipe}. -- to a pleasant new tune.}
\end{center}

\musicinfo{Moderate time.}{}

\lilypondfile{lilypond/085-the-doleful-dance-and-song-of-death}


\settowidth{\versewidth}{Merchants, have you made your mart in France,}
\indentpattern{0101000}
\begin{dcverse}\begin{patverse}
Bring away the beggar and the king,\\
And every man in his degree;\\
Bring away the old and youngest thing,\\
Come all to death, and follow me;\\
The courtier with his lofty looks,\\
The lawyer with his learned books,\\
The banker with his baiting hooks.
\end{patverse}

\begin{patverse}
Merchants, have you made your mart in France,\\
In Italy, and all about,\\
Know you not that you and I must dance,\\
Both our heels wrapt in a clout;\\
What mean you to make your houses gay,\\
And I must take the tenant away,\\
And dig for your sake the clods of clay?
\end{patverse}
\end{dcverse}

\pagebreak


%%086
%%===============================================================================

\settowidth{\versewidth}{Must dance with Death wheresoe’er you dwell.}
\indentpattern{0101000}
\begin{dcverse}\footnotesize
\begin{patverse}
Think you on the solemn ’sizes past,\\
How suddenly in Oxfordshire\\
I came, and made the judges all aghast,\\
And justices that did appear,\\
And took both Bell and Barham away,\dcfootnote{ %a
Anthony à Wood observes: “This solemn Assize,
mentioned in the foregoing page, was kept in the Courthouse
in the Castle-yard at Oxon, 4 Jul., 1577. The Judges
who were infected and dyed with the dampe, were Sir
Rob. Bell, Baron of the Exchequer, and Sir Nich. Barham, 
Serjeant at Lawe.” See Hist, et Antiq. Univ. Oxon.
lib. i. sub an. 1577. This verse, therefore, cannot have
been in the ballad entered to Awdelay, in 1568-9.
} %end footnote
\\
And many a worthy man that day,\\
And all their bodies brought to clay.
\end{patverse}

\begin{patverse}
Think you that I dare not come to schools,\\
Where all the cunning clerks be most;\\
Take I not away both wise and fools,\\
And am I not in every coast?\\
Assure yourselves no creature can\\
Make Death afraid of any man,\\
Or know my coming where or whan.
\end{patverse}

\begin{patverse}
Where be they that make their leases strong,\\
And join about them land to land,\\
Do you make account to live so long,\\
To have the world come to your hand?\\
No, foolish nowle, for all thy pence,\\
Full soon thy soul must needs go hence;\\
Then who shall toyl for thy defence?
\end{patverse}

\begin{patverse}
And you that lean on your ladies’ laps,\\
And lay your heads upon their knee,\\
‘May think that you’ll escape, perhaps,\\
And need not come to dance with me.’\\
But no! fair lords and ladies all,\\
I will make you come when I do call,\\
And find you a pipe to dance withall.
\end{patverse}

\begin{patverse}
And you that are busy-headed fools,\\
To brabble for a pelting straw,\\
Know you not that I have ready tools\\
To cut you from your crafty law?\\
And you that falsely buy and sell,\\
And think you make your markets well,\\
Must dance with Death wheresoe’er you dwell.
\end{patverse}

\begin{patverse}
Pride must have a pretty sheet, I see,\\
For properly she loves to dance;\\
Come away my wanton wench to me,\\
As gallantly as your eye doth glance;\\
And all good fellows that flash and swash\\
In reds and yellows of revell dash,\\
I warrant you need not be so rash.
\end{patverse}

\begin{patverse}
For I can quickly cool you all,\\
How hot or stout soever you be,\\
Both high and low, both great and small,\\
I nought do fear your high degree;\\
The ladies fair, the beldames old,\\
The champion stout, the souldier bold,\\
Must all with me to earthly mould.
\end{patverse}

\begin{patverse}
Therefore take time while it is lent,\\
Prepare with me yourselves to dance;\\
Forget me not, your lives lament,\\
I come oft-times by sudden chance.\\
Be ready, therefore,—watch and pray,\\
That when my minstrel pipe doth play,\\
You may to heaven dance the way.
\end{patverse}
\end{dcverse}


\musictitle{Wolsey’s Wild.}



This tune is called \textit{Wolsey's Wild} in Queen Elizabeth’s Virginal Book, but in
William Ballet’s Lute Book\dcfootnote{ %b
This highly interesting manuscript, which is in the
library of Trinity College, Dublin (D. I. 21), contains a
large number of the popular tunes of the sixteenth century. 
“Fortune my foe,” “Peg a Ramsey,” “Bonny
sweet Robin,” “Calleno,” “Lightie love Ladies,” “Green
Sleeves,” “Weladay” (all mentioned by Shakspeare),
besides “The Witches Dawnce,” “Thehunt is up,” “The
Shaking of the Shetes,” “The Quadran Pavan,” “a Hornpipe,” 
“Robin Reddocke,” “Barrow Foster’s Dreame,”
“Dowland’s Lachrimæ,” “Lusty Gallant,” The Blacksmith,” 
“Rogero,” “Turkeyloney,” “Staynes Morris,”
“Sellenger’s Rownde,” “All flowers in brome,” “Baloo,”
“Wigmore’s Galliard,” “Robin Hood is to the greenwood
gone,” \&c., \&c., are to be found in it. “Queen Mariees
Dump” (in whose reign it was probably commenced)
stands first in the book. The tunes are in lute tablature,
a style of notation now obsolete, in which the letters of
the alphabet up to K are used to designate the strings and
frets of the instrument.
} %end footnote
it is called \textit{Wilson's Wile}, and in \textit{Musick's Delight
on the Cithren}, 1666, \textit{Wilson's Wild}. In the Bagford Collection of Ballads,
Brit. Mus., there is one called “A proper newe sonet, declaring the Lamentation
of Beccles, a town in Suffolk,” \&c., by T. D. (Thomas Deloney), to \textit{Wilson's Tune},
and dated 1586, but it does not appear, from the metre, to have been intended \pagebreak
for this air. Another “proper new ballad” to \textit{Wilson's New Tune} is in the 
%%087
%%==============================================================================
Library of the Society of Antiquaries. It is on Ballard and Babington’s conspiracy,
and was written just after their execution, in 1586. \textit{Wilson’s Delight},
\textit{Arthur a Bradley}, and \textit{Mall Dixon’s Round}, are mentioned as popular tunes in
Braithwaite’s \textit{Strappado for the Devil},~1615.

The song, “Quoth John to Joan,” or “I cannot come every day to woo,” is
certainly as old as the time of Henry VIII., because the first verse is to be found
elaborately set to music in a manuscript of that date, formerly in the possession
of Stafford Smith (who'printed the song in Musica Antiqua, vol. i., p.~32), and now
in that of Dr.~Rimbault. There are two copies of the words in vol. ii. of the
Roxburghe Collection of Ballads, and it is in all the editions of \textit{Wit and Mirth, or\textit{
Pills to purge Melancholy}}, from 1698 to 1719. In \textit{Wit’s Cabinet}, 1731, it is
called “The Clown’s Courtship, sung to the King at Windsor.”

\musicinfo{Moderate time.}{}

\lilypondfile[staffsize=16]{lilypond/087-wolseys-wild}

\settowidth{\versewidth}{In the nook of the chimney, instead of a bag.}
\begin{dcverse}I’ve corn and hay in the barn hard by,\\
And three fat hogs pent up in the sty;\\
I have a mare, and she is coal-black,\\
I ride on her tail to save her back.\\
\hspace{6em}Then say, my Joan, \&c.

I have a cheese upon the shelf,\\
And I cannot eat it all myself;\\
I’ve three good marks that lie in a rag,\\
In the nook of the chimney, instead of a bag.\\
\hspace{6em}Then say, my Joan, \&c.
\end{dcverse}

\begin{scverse}To marry I would have thy consent,\\
But, faith, I never could compliment;\\
I can say nought but “hoy, gee ho,”\\
Words that belong to the cart and the plough:\\
\hspace{6em}Then say, my Joan, say, my Joan, will that not do,\\
\hspace{6em}I cannot come every day to woo.
\end{scverse}
\pagebreak
%%088
%%===============================================================================

\musictitle{The Marriage of the Frog and the Mouse.}

In Wedderburn’s \textit{Complaint of Scotland}, 1549, one of the songs sung by the
shepherds is \textit{The frog cam to the myl dur} [mill-door]. In 1580, a ballad of
“A most strange wedding of the frog and the mouse” was licensed to Edward
White, at Stationers’ Hall: and in 1611, this song was printed with music, among
the “Country Pastimes,” in \textit{Melismata}. It is the progenitor of several others;
one beginning—
\settowidth{\versewidth}{“There was a frog lived in a well,}
\begin{scverse}
“There was a frog lived in a well,\\
And a farce mouse in a mill;”
\end{scverse}
another, “A frog he would a-wooing go;” a third in \textit{Pills to purge Melancholy}, \&c., \&c.

\musicinfo{Moderate time.}{}

\lilypondfile{lilypond/088-the-marriage-of-the-frog-and-the-mouse}

\settowidth{\versewidth}{The frogge would a-wooing ride,}

\begin{dcverse}The frogge would a-wooing ride,\\
\hspace{4em}Humble-dum, humble-dum;\\
Sword and buckler by his side,\\
\hspace{4em}Tweedle, tweedle, twino.\\
When upon his high horse set,\\
\hspace{4em}Humble-dum, \&c.,\\
His boots they shone as black as jet,\\
\hspace{4em}Tweedle, \&c.

When he came to the merry mill pin,\\
Lady Mouse beene you within?\\
Then came out the dusty mouse:\\
I am lady of this house;\columnbreak

Hast thou any mind of me?\\
I have e’en great mind of thee.\\
Who shall this marriage make?\\
Our lord, which is the rat.

What shall we have to our supper?\\
Three beans in a pound of butter.\\
But, when supper they were at,\\
The frog, the mouse, and e’en the rat,

Then came in Gib, our cat,\\
And caught the mouse e’en by the back.\\
Then did they separate:\\
The frog leapt on the floor so flat;
\end{dcverse}

\vspace{-\baselineskip}
\begin{scverse}
Then came in Dick, our drake,\\
And drew the frog e’en to the lake;\\
The rat he ran up the wall,\\
‘And so the company parted all.’
\end{scverse}

\musictitle{The Cramp.}

This is one of the three country dance tunes arranged to be sung together in
\textit{Pammelia}, and is frequently referred to as a ballad tune.

In the Ashmolean library, in the same manuscript volume with \textit{Chevy Chace}
(No. 48), is a ballad by Elderton, describing the articles sold in the market in
time of Lent. The observance of Lent was compulsory in those days, and it was
by no means palatable to all. In 1570, William Pickering had a license to print
\pagebreak
%%089
%%===============================================================================
a ballad, entitled \textit{Lenton Stuff}, which was, in all probability,.the same. Elderton’s
ballad is called—

\settowidth{\versewidth}{You know well enough you must}
\begin{scverse}
“A new ballad, entitled \textit{Lenton Stuff},\\
For a little money ye may have enough;”\\
\end{scverse}

\begin{center} to the tune of \textit{The Cramp}.\end{center}

\begin{scverse}
“Lenton stuff is come to the town,\\
\vin The cleansing week comes quickly;\\
You know well enough you must kneel down,\\
\vin Come on, take ashes trickly;\\
That neither are good flesh nor fish,\\
But dip with Judas in the dish,\\
And keep a rout not worth a ryshe” [rush].\\
\vin\vin\vin\vin\vin\vin{[Heigh ho! the cramp-a.]}
\end{scverse}

It is not noticed by Ritson in his list of Elderton’s ballads, Bibl. Poet. p.~195-8;
but Mr. Halliwell has printed it in the volume containing \textit{The Marriage of Wit
and Wisdom}, for the Shakespeare Society. The following is from \textit{Pammelia}.

\musicinfo{Moderate time.}{}

\lilypondfile{lilypond/089-the-cramp}

\pagebreak

%%090
%%===============================================================================

\musictitle{I Have House and Land in Kent.}

This song, which is one of the “Country Pastimes,” in \textit{Melismata}, 1611, is on
the same subject as \textit{Quoth John to Joan}, page 87. The tune begins like \textit{The
Three Ravens}, but is in quicker time. In \textit{Melismata} it is called \textit{A Wooing Song
of a Yeoman of Kent's son}, and the words are given in the Kentish dialect.

\musicinfo{Moderate time.}{}

\lilypondfile{lilypond/090-i-have-house-and-land-in-kent}


\indentpattern{121200}
\settowidth{\versewidth}{\textit{Chorus}.—For he can bravely clout his shoone,}
\begin{dcverse}\footnotesizer\begin{patverse}
\vin Ich am my vather's eldest zonne,\\
My mother eke doth love me well;\\
For ich can bravely clout my shoone,\\
And ich full well can ring a bell.\dcfootnote{ %a
Bell-ringing was formerly a great amusement of the
English, and the allusions to it are of frequent occurrence.
Numerous payments to bell-ringers are generally to be
found in Churchwardens’ accounts of the 16th and 17th
centuries.
} %end footnote
\\
\textit{Chorus}.—For he can bravely clout his shoone,\\
\vinphantom{\textit{Chorus}.—}And he full well can ring a bell.
\end{patverse}

\begin{patverse}
\vin My vather he gave me a bogge,\\
My mouther she gave me a zow;\\
I have a godvather dwells there by,\\
And he on me bestowed a plow.\\
\textit{Chorus}.—He has a godvather dwells there by,\\
\vinphantom{\textit{Chorus}.—}And he on him bestowed a plow.
\end{patverse}

\begin{patverse}
\vin One time I gave thee a paper of pins,\\
Anoder time a taudry lace;\\
And if thou wilt not grant me love,\\
In truth ich die bevore thy vace.\\
\textit{Chorus}.—And if thou wilt not grant his love,\\
\vinphantom{\textit{Chorus}.—}In truth he’ll die bevore thy face.
\end{patverse}

\begin{patverse}
\vin Ich have beene twise our Whitson lord,\\
Ich have had ladies many vare;\\
And eke thou hast my heart in hold,\\
And in my mind zeemes passing rare.\\
\textit{Chorus}.—And eke thou hast his heart in hold,\\
\vinphantom{\textit{Chorus}.—}And in his mind zeemes passing rare.
\end{patverse}

\begin{patverse}
\vin Ich will put on my best white slopp,\\
And ich will wear my jellow hose,\\
And on my head a good gray hat,\\
And in’t ich stick a lovely rose.\\
\textit{Chorus}.—And on his head a good gray hat,\\
\vinphantom{\textit{Chorus}.—}And i'nt he'll stick a lovely rose.
\end{patverse}

\begin{patverse}
\vin Wherefore cease off, make no delay,\\
And if you’ll love me, love me now;\\
Or else ich zeek zome oder where,\\
For I cannot come every day to woo.\\
\textit{Chorus}.—Or else he'll zeek zome oder where,\\
\vinphantom{\textit{Chorus}.—}For he cannot come every day to woo.
\end{patverse}
\end{dcverse}

\pagebreak



%%091
%%===============================================================================

\musictitle{Lusty Gallant.}

This tune, which was extremely popular in former times, is to be found in
William Ballet’s Lute Book. It resembles “Now foot it as I do, Tom, boy, Tom,”
which is one of three country dances, arranged to be sung together as a round, in
\textit{Pammelia}.

Nicholas Breton mentions \textit{Old Lusty Gallant} as a dance tune in his \textit{Works of
a Young Wit}, 1577:\qquad \qquad \qquad\dots “by chance,

\settowidth{\versewidth}{And then, you know, the youth must needs go dance,}
\begin{scverse}
Our banquet done, we had our music by,\\
And then, you know, the youth must needs go dance,\\
First galliards—then larousse, and heidegy—\\
\textit{Old Lusty Gallant—All flowers of the broom};\\
And then a hall, for dancers must have room;”
\end{scverse}
and Elderton, wrote, “a proper new balad in praise of my Ladie Marques, whose
death is bewailed,” to the tune of \textit{New Lusty Gallant}. A copy of that ballad is
in the possession of Mr. George Daniel, of Canonbury; but I assume it to have
been intended for another air, because there are seven lines in each stanza. The
following is the first:—

\settowidth{\versewidth}{And what is the cause I court it not}
\begin{scverse}
\vleftofline{“}Ladies, I thinke you marvell that\\
I writ no mery report to you:\\
And what is the cause I court it not\\
So merye as I was wont to dooe?\\
Alas! I let you understand\\
It is no newes for me to me to show\\
The fairest flower of my garland.”
\end{scverse}
If sung to this tune, the last line of each stanza would require repetition.

Nashe, in his \textit{Terrors of the Night}, 1594, says, “After all they danced \textit{Lusty
Gallant}, and a drunken Danish levalto or two.”

There is a song beginning, “Fain would I have a pretie thing to give unto my
ladie” (to the tune of \textit{Lusty Gallant}), in \textit{A Handefull of Pleasant Delites}, and
although that volume is not known to have been printed before 1584, it seems to
have been entered at Stationers’ Hall as early as 1565-6. \textit{Fain would I}, \&c.,
must have been written, and have attained popularity, either in or before the
year 1566, because, in 1566-7, a moralization, called \textit{Fain would I have a godly
thing to shew unto my lady}, was entered, and in MSS. Ashmole\dcfootnote{ %
Mr. W. H. Black, in his Catalogue of the Ashmolean
MSS., describes this volume as ‘‘written in the middle of
the sixteenth century”-—(it is the manuscript which contains
\textit{Chevy Chace}). Mr. Halliwell has printed the ballad
of \textit{Troilus and Creseida}, in the volume containing The
\textit{Marriage of Wit and Wisdom}, for the Shakespeare Society.
} %end footnote 
48, fol. 120, is a
ballad of \textit{Troilus and Creseida}, beginning—
\settowidth{\versewidth}{When Troilus dwelt in Troy town,}
\begin{scverse}
\vleftofline{“}When Troilus dwelt in Troy town,\\
A man of noble fame-a”—
\end{scverse}
to the tune of \textit{Fain would I find some pretty thing}, \&c., so that, from the popularity
of the ballad, the tune had become known by its name also.

I have not found any song called \textit{Lusty Gallant}: perhaps it is referred to in
Massinger’s play, \textit{The Picture}, where Ferdinand says:

\pagebreak



%%092
%%===============================================================================
\DFNsingle

\settowidth{\versewidth}{With which you chanted \textit{Room for a lusty Gallant},}
\begin{scverse}
\vin\vin\vin\vin\vin\vin ---“is your Theorbo\\
Turn’d to a distaff, Signior, and your voice,\\
With which you chanted \textit{Room for a lusty Gallant},\\
Tuned to the note of \textit{Lachrymæ}?”\dcfootnote{\centering %a
\textit{Lachrymæ}, a tune often referred to, composed by Dowland.
} %end footnote
\end{scverse}


The ballad of “A famous sea-fight between Captain Ward and the Rainbow”
(in the Roxburghe Collection) “to the tune of Captain Ward,” \&c., begins, “Strike
up, you lusty Gallants.”

In the \textit{Gorgeous Gallery of gallant Inventions}, 1578, there is a “proper dittie,”
to the tune of \textit{Lusty Gallant}; and Pepys mentions a song with the burden of
“St. George for England,” to the tune of \textit{List, lusty Gallants}.

\musicinfo{Moderate time.}{}

\lilypondfile[noindent, current-font-as-main, staffsize=17, 
noragged-right, language=english, nofragment]
{lilypond/092-lusty-gallant}\normalsize


\settowidth{\versewidth}{Twenty journeys would I make,}
\begin{dcverse}
\begin{altverse}
Twenty journeys would I make,\\
And twenty days would hie me,\\
To make adventure for her sake,\\
To set some matter by me.
\end{altverse}

\begin{altverse}
Some do long for pretty knacks.\\
And some for strange devices;\\
God send me what my lady lacks,\\
I care not what the price is.
\end{altverse}
\end{dcverse}

There are eight more stanzas, which will be found in Evans’ \textit{Old Ballads}, vol. 1,
p.~123, edit. 1810, or in the reprint of \textit{A Handefull of Pleasant Delites}.

\musictitle{By a Bank as I Lay.}

In the Life of Sir Peter Carew, before quoted (page 52), “By the bank as
I lay” is mentioned as one of the \textit{Freemen’s Songs} which Sir Peter used to sing
with Henry VIII.; and this is one of the \textit{King Henry’s Mirth or Freemen’s Songs}
in \textit{Deuteromelia}. In Laneham’s letter from Kenilworth, 1565, “By a bank as
I lay” is included in the “bunch of ballads and songs, all ancient,” which were
then in the possession of Captain Cox, the Mason of Coventry. In Wager’s interlude, 
\textit{The longer thou livest the more fool thou art}, 1568, Moros sings the two
following lines:—
\begin{scverse}
“By a bank as I lay, I \textit{lay}.\\
\textit{Musing on things past}, heigh ho!”
\end{scverse}

In Royal MSS. Append. 58, there is another song, \pagebreak of which the first line is the
%%093
%%===============================================================================
same, but the second differs; and the music to it is not of the light and popular
class called \textit{Freemen’s Songs}, but a studied composition. The words of the latter
have been printed by Mr. Payne Collier, in his Extracts from the Registers of
the Stationers’ Company, vol. i., page 193. They are in the same metre, and
therefore might also be sung to this tune.

The last line of the song, as printed in \textit{Deuteromelia}, is “And save noble \textit{James}
our king,” because the book was printed in his reign.
\DFNdouble

\musicinfo{Moderate time.}{}

\lilypondfile[noindent, current-font-as-main, staffsize=17, 
noragged-right, language=english, nofragment]
{lilypond/093-by-a-bank-as-i-lay}\normalsize


\indentpattern{00110}
\settowidth{\versewidth}{To hear the bird how merrily she could sing,}
\begin{dcverse}\begin{patverse}
\vin O the gentle nightingale,\\
The lady and the mistress of all musick.\\
She sits down ever in the dale;\\
Singing with her notès smale [small],\\
And quavering them wonderfully thick.
\end{patverse}

\begin{patverse}
\vin Oh, for joy, my spirits were quick,\\
To hear the bird how merrily she could sing,\\
And I said, good Lord, defend\\
England, with thy most holy hand,\\
And save noble ‘Henry’ our king.
\end{patverse}
\end{dcverse}

\musictitle{Rogero.}

This tune is to be found among Dowland’s Manuscripts,\dcfootnote{ %a
The references to these Manuscripts are, D. d. 2. 11.
--D. d. 3. 18.--D. d. 4. 23.—D. d. 9. 33.—D. d. 14. 24.,
\&c. Some appear to be in the handwriting of Dowland,
the celebrated lutenist of Elizabeth's reign. The tune of
\textit{Rogero} is in three or four of them.
} %end footnote 
in the public library,
Cambridge; in William Ballet’s Lute Book, and in Dallis’ Lute Book, both in
the library of Trinity College, Dublin.

The first entry in Mr. Payne Collier’s Extracts from the Registers of the
Stationers’ Company, is to William Pickering,  “Ballett called \textit{Arise and wake}”
(1557). In the Roxburghe Collection of Ballads, there is one commencing,
“Arise and \textit{a}wake,” entitled—

\settowidth{\versewidth}{Shewing the duty of every degree,”}
\begin{scverse}
\vleftofline{“}A godly and Christian A.B.C.,\\
Shewing the duty of every degree,”
\end{scverse}
to the tune of \textit{Rogero}. It may be the ballad referred to, although the copy in the
Roxburghe Collection was printed at a later date. In the same year, 1557, there
is an entry of “A Ballett of the A.B.C. of a Priest, called Hugh Stourmy,”
and another of “The aged man’s A.B.C.”

\pagebreak



%%094
%%===============================================================================

\textit{Rogero} is mentioned as a dance tune in Stephen Gosson’s \textit{School of Abuse},
1579; in Heywood’s \textit{A woman killed with kindness} (acted before 1604); and in
Nashe’s \textit{Have with you to Saffron-Walden}, 1596; also by Dekker, in \textit{The Shoemaker’s
Holiday}, \&c.

Many ballads were sung to the tune of \textit{Rogero}. In the first volume of the
Roxburghe Collection, for instance, there are at least four.\dcfootnote{ %a
See folios 130, 258, 482, and 492.
} %end footnote
Others in the
Pepysian Collection; in \textit{The Grown Garland of Golden Roses}, 1612; in Deloney’s
\textit{Strange Histories},\dcfootnote{ %b
\textit{The Crown Garland} and \textit{Strange Histories} have been
reprinted by the Percy Society.
} %end footnote 
1607; in Percy’s \textit{Reliques of Ancient Poetry}; and in Evans’
\textit{Old Ballads}. \textit{Arise and awake} is also referred to as a ballad tune.

The following, which is entitled “The valiant courage and policy of the
Kentishmen with long tails, whereby they kept their ancient laws and customs,
which William the Conqueror sought to take from them \dcfootnote{ %c
Evans, who prints this ballad from another copy (\textit{The
Garland of Delight}) extracts the following account of the
event which gave rise to it, from \textit{The Lives of the three
Norman Kings of England}, by Sir John Heyward, 4to, 1613,
p.~97: “Further, by the counsel of Stigand, Archbishop
of Canterbury, and of Eglesine, Abbot of St. Augustine’s
(who at that time were the chief governors of Kent), as the
King was riding towards Dover, at Swanscombe, two
miles from Gravesend, the Kentishmen came towards him
armed and bearing boughs in their hands, as if it had been
a moving wood; they enclosed him upon the sudden, and
with a firm countenance, but words well tempered with
modesty and respect, they demanded of him the use of
their ancient liberties and laws: that in other matters
they would yield obedience unto him: that without this
they desired not to live. The king was content to strike
sail to the storm, and to give them a vain satisfaction for
the present; knowing right well that the general customs
and laws of the residue of the realm would in short time
overflow these particular places. So pledges being given
on both sides, they conducted him to Rochester, and
yielded up the county of Kent, and the castle of Dover
into his power.”
} %end footnote 
—to the tune of \textit{Rogero}
is from \textit{Strange Histories}, \&c., 1607. It was written by Deloney, “the ballading
silk-weaver,” who died in or before 1600.

\musicinfo{Boldly and marked.}{}

\lilypondfile[noindent, current-font-as-main, staffsize=15, 
noragged-right, language=english, nofragment]
{lilypond/094-rogero}\normalsize


\settowidth{\versewidth}{Which being done, he changed quite}
\begin{dcverse}\begin{altverse}
On Christmas-day in solemn sort\\
Then was he crowned here,\\
By Albert archbishop of York,\\
With many a noble peer.
\end{altverse}

\begin{altverse}
Which being done, he changed quite\\
The customs of this land,\\
And punisht such as daily sought\\
His statutes to withstand:
\end{altverse}

\begin{altverse}
And many cities he subdued,\\
Fair London with the rest;\\
But Kent did still withstand his force,\\
And did his laws detest.
\end{altverse}

\begin{altverse}
To Dover then he took his way,\\
The castle down to fling,\\
Which Arviragus builded there,\\
The noble British king.
\end{altverse}
\end{dcverse}

\pagebreak

%%095
%%===============================================================================

\settowidth{\versewidth}{With sword and spear, with bill and bow,}
\begin{dcverse}\scriptsizerrr
\begin{altverse}
Which when the brave archbishop bold\\
Of Canterbury knew,\\
The abbot of Saint Augustines eke,\\
With all their gallant crew,
\end{altverse}

\begin{altverse}
They set themselves in armour bright,\\
These mischiefs to prevent,\\
With all the yeomen brave and bold\\
That were in fruitful Kent.
\end{altverse}

\begin{altverse}
At Canterbury did they meet\\
Upon a certain day,\\
With sword and spear, with bill and bow,\\
And stopt the conqueror’s way.
\end{altverse}

\begin{altverse}
Let us not live like bond-men poor\\
To Frenchmen in their pride,\\
But keep our ancient liberty,\\
What chance so e’er betide,
\end{altverse}

\begin{altverse}
And rather die in bloody field,\\
In manlike courage prest (ready),\\
Than to endure the servile yoke,\\
Which we so much detest.
\end{altverse}

\begin{altverse}
Thus did the Kentish commons cry\\
Unto their leaders still,\\
And so march’d forth in warlike sort,\\
And stand at Swanscomb hill:
\end{altverse}

\begin{altverse}
Where in the woods they hid themselves,\\
Under the shady green,\\
Thereby to get them vantage good,\\
Of all their foes unseen.
\end{altverse}

\begin{altverse}
And for the conqueror’s coming there,\\
They privily laid wait,\\
And thereby suddenly appal'd\\
His lofty high conceit;
\end{altverse}

\begin{altverse}
For when they spied his approach,\\
In place as they did stand,\\
Then marched they, to hem him in,\\
Each one a bough in hand,
\end{altverse}

\begin{altverse}
So that unto the conqueror’s sight,\\
Amazed as he stood,\\
They seem’d to be a walking grove,\\
Or else a moving wood.
\end{altverse}

\begin{altverse}
The shape of men he could not see,\\
The boughs did hide them so:\\
And now his heart for fear did quake,\\
To see a forest go;
\end{altverse}

\begin{altverse}
Before, behind, and on each side,\\
As he did cast his eye,\\
He spied those woods with sober pace\\
Approach to him full nigh:
\end{altverse}

\begin{altverse}
But when the Kentish-men had thus\\
Enclos’d the conqueror round,\\
Most suddenly they drew their swords,\\
And threw the boughs to ground;
\end{altverse}

\begin{altverse}
Their banners they display’d in sight,\\
Their trumpets sound a charge,\\
Their rattling drums strike up alarms,\\
Their troops stretch out at large.
\end{altverse}

\begin{altverse}
The conqueror, and all his train,\\
Were hereat sore aghast,\\
And most in peril, when they thought\\
All peril had been past.
\end{altverse}

\begin{altverse}
Unto the Kentish men he sent,\\
The cause to understand,\\
For what intent, and for what cause,\\
They took this war in hand;
\end{altverse}

\begin{altverse}
To whom they made this short reply,\\
For liberty we fight,\\
And to enjoy king Edward’s laws,\\
The which we hold our right.
\end{altverse}

\begin{altverse}
Then said the dreadful conqueror,\\
You shall have what you will,\\
Your ancient customs and your laws,\\
So that you will be still:
\end{altverse}

\begin{altverse}
And each thing else that you will crave\\
With reason, at my hand,\\
So you will but acknowledge me\\
Chief king of fair England.
\end{altverse}

\begin{altverse}
The Kentish men agreed thereon.\\
And laid their arms aside,\\
And by this means king Edward’s laws\\
In Kent do still abide;
\end{altverse}
\end{dcverse}

\settowidth{\versewidth}{And in no place in England else}
\begin{scverse}\scriptsizerrr
\begin{altverse}
And in no place in England else\\
These customs do remain,\\
Which they by manly policy\\
Did of duke William gain.
\end{altverse}
\end{scverse}

\musictitle{Turkeyloney.}

The figure of the dance called \textit{Turkeyloney} is described with others in a manu\-script
in the Bodleian Library (MS. Rawl. Poet. 108), which was written about
1570. Stephen Gosson, in his \textit{Schoole of Abuse, containing a pleasant Invective
against Poets, Pipers, Players, Jesters}, \&c., \pagebreak 1579, alludes to the tune as one of
\pagebreak
%%096
%%===============================================================================
the most popular in his day. He says, “Homer, with his music, cured the sick
soldiers in the Grecians’ camp, and purged every man’s tent of the plague.
Think you that those miracles could be wrought with playing dances, dumps,
pavans, galliards, fancies, or new strains? They never came where this grew,
nor knew what it meant\dots Terpander neither piped \textit{Rogero}, nor \textit{Turkeloney},
when he ended the brabbles at Lacedemon, but, putting them in mind of Lycurgus’
laws, taught them to tread a better measure:” but, “if you enquire how many
such poets and pipers we have in our age, I am persuaded that every one of them
may creep through a ring, or dance the wild morris in a needle’s eye. We have
infinite poets and pipers, and such peevish cattle among us in England, that live
by merry begging, maintained by alms, and privily encroach upon every man’s
purse, but if they in authority should call an account to see how many Chirons,
Terpandri, and Homers are here, they might cast the sum without pen or
counters, and sit down with Rachel to weep for her children, because they are not.”

Turkeylony is also mentioned, as a dance tune, in Nashe’s \textit{Have with you to
Saffron-Walden}, 1596; and the music will be found in William Ballet’s Lute
Book, described in a note at page 86.

The words here coupled with the tune are taken from a manuscript in the
possession of Mr. Payne Collier. Although the manuscript is of the reign of
James I., the “ballett” \textit{Yf ever I marry, I will marry a mayde}, was entered
at Stationers’ Hall as early as 1557-8. The name of the air to which it should
be sung is neither given in the MS., nor in the entry at Stationers’ Hall; but the
words and music agree so well together, that it is very probable the ballet was
written to this tune.

\musicinfo{In moderate time, and smoothly.}{}

\lilypondfile[noindent, current-font-as-main, staffsize=17, 
noragged-right, language=english, nofragment]
{lilypond/096-turkeyloney}\normalsize

\pagebreak

%%097
%%===============================================================================

\settowidth{\versewidth}{And cost ne’er so much, she will ever go brave, [gaily dress’d]}
\begin{scverse}
A maid is so sweet, and so gentle of kind,\\
That a maid is the wife I will choose to my mind;\\
A widow is froward, and never will yield;\\
Or if such there be, you will meet them but seeld. [seldom]

A maid ne’er complaineth, do what so you will;\\
But what you mean well, a widow takes ill:\\
A widow will make yon a drudge and a slave,\\
And cost ne’er so much, she will ever go brave, [gaily dress’d]

A maid is so modest, she seemeth a rose,\\
When first it beginneth the bud to unclose;\\
But a widow full blowen, full often deceives,\\
And the next wind that bloweth shakes down all her leaves.

That widows be lovely I never gainsay,\\
But too well all their beauty they know to display;\\
But a maid hath so great hidden beauty in store,\\
She can spare to a widow, yet never be poor.

Then, if ever I marry, give me a fresh maid,\\
If to marry with any I be not afraid;\\
But to marry with any it asketh much care,\\
And some bachelors hold they are best as they are.
\end{scverse}

\vfill

\centerrule

\vfill
\vfill


%%098
%%===============================================================================


\section*{REIGN OF ELIZABETH.}

\centerrule

During the long reign of Elizabeth, music seems to have been in universal
cultivation, as well as in universal esteem. Not only was it a necessary qualification
for ladies and gentlemen, but even the city of London advertised the musical
abilities of boys educated in Bridewell and Christ’s Hospital, as a mode of
recommending them as servants, apprentices, or husbandmen.\dcfootnote{ %a
“That the preachers be moved at the sermons at the
Crosse” [St. Paul’s Cross] “and other convenient times,
and that all other good notorious meanes be used, to require
both citizens, artificers, and other, and also all
farmers and other for husbandry, and gentlemen and other
for their kitchens and other services, to take servants and
children both out of Bridewell and Christ’s Hospital at
their pleasures,\ldots\  with further declaration that many
of them be of toward qualities in readyng, wryting, grammer, 
and \textit{musike}.” This is the 66th and last of the
“Orders appointed to be executed in the cittie of London,
for setting rog[u]es and idle persons to worke, and for
releefe of the poore.” “At London, printed by Hugh
Singleton, dwelling in Smith Fielde, at the signe of the
Golden Tunne;” reprinted in \textit{The British Bibliographer}.
Edward VI. granted the charters of incorporation for
Bridewell and Christ’s Hospital, a few days before his
death. Bridewell is a foundation of a mixed and singular
nature, partaking of the hospital, prison, and work-house. 
Youths were sent to the Hospital as apprentices
to manufacturers, who resided there; and on leaving, received
a donation of 10\textit{l}., and their freedom of the city.
Pepys, in his Diary, 5th October, 1664, says, “To new
Bridewell, and there I did with great pleasure see the
many pretty works, and the little children employed,
every one to do something, which was a very fine sight,
and worthy encouragement.”
} %end footnote
In Deloney’s
\textit{History of the gentle Craft}, 1598, one who tried to pass for a shoemaker was
detected as an imposter, because he could neither “sing, sound the trumpet, play
upon the flute, nor reckon up his tools in rhyme.” Tinkers sang catches; milkmaids
sang ballads; carters whistled; each trade, and even the beggars, had
their special songs; the base-viol hung in the drawing room for the amusement of
waiting visitors; and the lute, cittern, and virginals, for the amusement of waiting
customers, were the necessary furniture of the barber’s shop. They had
music at dinner; music at supper; music at weddings; music at funerals; music
at night; music at dawn; music at work; and music at play.

He who felt not, in some degree, its soothing influences, was viewed as a
morose, unsocial being, whose converse ought to be shunned, and regarded with
suspicion and distrust.

\settowidth{\versewidth}{Nor is not mov’d with concord of sweet sounds,}
\begin{scverse}
“The man that hath no music in himself,\\
Nor is not mov’d with concord of sweet sounds,\\
Is fit for treasons, stratagems, and spoils;\\
The motions of his spirit are as dull as night,\\
And his affections dark as Erebus:\\
Let no such man be trusted.”\\
\vin\vin\vin\vin\textit{Merchant of Venice}, act v., sc. 1.

“Preposterous ass! that never read so far\\
To know the cause why music was ordain’d!\\
Was it not to refresh the mind of man\\
After his studies, or his usual pain?”\\
\vin\vin\vin\vin\textit{The Taming of the Shrew}, act ii., sc. 3.
\end{scverse}

\pagebreak

%%99
%%===============================================================================
\renewcommand\rectoheader{reign of elizabeth} 

Steevens, in a note upon the above passage in \textit{The Merchant of Venice}, quotes the
authority of Lord Chesterfield against what he terms this “capricious sentiment”
of Shakespeare, and adds that Peacham requires of his gentleman \textit{only} to be able
“to sing his part \textit{sure, and at first sight}, and withall to play the same on a viol,
or lute.” But this sentiment, so far from being peculiar to Shakespeare, may be
said to have been the prevailing one of Europe. Nor was Peacham an exception,
for, although he says, “I dare not pass so rash a censure of these” (who love not
music) “as Pindar doth; or the Italian, having fitted a proverb to the same effect,
\textit{Whom God loves not, that man loves not music};” he adds, “but I am verily persuaded
that they are by nature very ill disposed, and of such a brutish stupidity
that scarce any thing else that is good and savoureth of virtue is to he found
in them.”\dcfootnote{ %a
The Compleat Gentleman: fashioning him absolute in
the most necessary and commendable qualities, concerning
mind or bodie, that may be required in a noble gentleman,
By Henry Peacham, Master of Arts, \&c, 1622.
} %end footnote
Tusser, in his “Points of Huswifry united to the comfort of
Husbandry,”~1570, recommends the country huswife to select servants that sing
at their work, as being usually the most pains-taking, and the best. He says:

\settowidth{\versewidth}{That sing in their labour, as birds in the wood;”}
\begin{scverse}
“Such servants are oftenest painfull and good,\\
That sing in their labour, as birds in the wood;”
\end{scverse}
and old Merrythought says, “Never trust a \textit{tailor} that does not sing at
his work, for his mind is of nothing but filching.”—(\textit{Dyce’s Beaumont and
Fletcher}, vol. ii., p.~171.)


Byrd, in his \textit{Psalmes, Sonnets, and Songs}, \&c., 1588, gives the following eight
reasons why every one should learn to sing:—

1st.—“It is a knowledge easily taught, and quickly learned, where there is a good
master and an apt scholar.”

2nd.—“The exercise of singing is delightful to nature, and good to preserve the
health of man.”

3rd.—“It doth strengthen all parts of the breast, and doth open the pipes.”

4th.—“It is a singular good remedy for a stutting and stammering in the speech.”

5th.—“It is the best means to procure a perfect pronunciation, and to make a good
orator.”

6th.—“It is the only way to know where nature hath bestowed a good voice; \ldots\ 
and in many that excellent gift is lost, because they want art to express nature.”

7th.—“There is not any music of instruments whatsoever, comparable to that which
is made of the voices of men; where the voices are good, and the same well sorted
and ordered.”

8th.—“The better the voice is, the meeter it is to honour and serve God therewith;
and the voice of man is chiefly to be employed to that end.”

\begin{scverse}
“Since singing is so good a thing,\\
I wish all men would learn to sing.”
\end{scverse}

Morley, in his \textit{Introduction to Pratical Musick}, 1597, written in dialogue,
introduces the pupil thus: “But supper being ended, and music books,
according to custom, being brought to the table, the mistress of the house presented
me with a part, earnestly requesting me to sing; but when, after many
excuses, I protested unfeignedly that \textit{I could not}, every one began to wonder; yea,
\pagebreak
%%100
%%===============================================================================
some whispered to others, demanding how I was brought up, so that upon shame
of mine ignorance, I go now to seek out mine old friend, Master Gnorimus, to
make myself his scholar.”

Laneham, to whom we are indebted for the description of the pageants at Kenilworth
in 1575, thus describes his own evening amusements. “Sometimes I foot
it with dancing; now with my gittern, and else with my cittern, then at the
virginals (ye know nothing comes amiss to me): then carol I up a song withal;
that by and by they come flocking about me like bees to honey; and ever they
cry, ‘Another, good Laneham, another.’” He who thus speaks of his playing
upon three instruments and singing, had been promoted from a situation in the
royal stables, through the. favour of the Earl of Leicester, to the duty of keeping
eaves-droppers from the council-chamber door.

Dekker, in \textit{The Gull’s Horn-book}, tells us that the usual routine of a young
gentlewoman’s education was “to read and write; to play upon the virginals,
lute, and cittern; and to read prick-song (\ie, music written or pricked down) \textit{at
first sight}.” Whenever a lady was highly commended by a writer of that age,
her skill in music was sure to be included; as—

\settowidth{\versewidth}{Her own tongue speaks all tongues, and her own hand}
\begin{scverse}
\vleftofline{“}Her own tongue speaks all tongues, and her own hand\\
Can teach all strings to speak in their best grace.”\\
\vin\vin\vin\vin\vin\vin \textit{Heywood’s A Woman kill’d with kindness}.
\end{scverse}

“Observe,” says Lazarillo, who is instructing the ladies how to render themselves
most attractive, “it shall be your first and finest praise to sing the note of
every new fashion at first sight.—(\textit{Middleton’s Blurt, Master Constable}, 1602.)
Gosson, in his \textit{Schoole of Abuse}, 1579, alluding to the custom of serenading,
recommends young ladies to be careful not to “flee to inchaunting,” and says, “if
assaulted with music in the night, close up your eyes, stop your ears, tie up your
tongues; when they speak, answer them not; when they halloo, stoop not; when
they sigh, laugh at them; when they sue, scorn them.” He admits that “these are
hard lessons,” but advises them “nevertheless to drink up the potion, though it
like not [please not] your taste.” In those days, however, the “serenate, which
the starv’d lover sings to his proud fair,” was not quite so customary in England
as the Morning song or \textit{Hunt’s-up}; such as—

\settowidth{\versewidth}{Fain would I wake you, sweet, but fear}
\indentpattern{000031331}
\begin{scverse}
\begin{patverse}
\vleftofline{“}Fain would I wake you, sweet, but fear\\
I should invite you to worse cheer;\ldots\  \\
I’d wish my life no better play,\\
Your dream by night, your thought, by day:\\
Wake, gently wake,\\
Part softly from your dreams!\\
\textit{The Morning flies}\\
To your fair eyes,\\
To guide her special beams.”
\end{patverse}
\end{scverse}


As to the custom of having a base-viol (or viol da gamba) hanging up in drawing
rooms for visitors to play on, one quotation from Ben Jonson may suffice:
“In making love to her, never fear to be out, for\ldots\   a base viol shall hang o’ the
wall, of purpose, shall put you in \pagebreak presently.—(\textit{Gifford’s Edit}. vol. ii., p.~162.)
%%101
%%===============================================================================
If more to the same purport be required, many similar allusions will be found in
the same volume. (See pages 125,126, 127, and 472, and Gifford’s Notes.)

The base-viol was also played upon by ladies (at least during the following
reign), although thought by some “an unmannerly instrument for a woman.”
The mode in which some ladies passed their time is described in the following
lines, and perhaps, even in the present day, instances not wholly unlike might be
found.

\settowidth{\versewidth}{Sit and answer them that woo;}
\begin{scverse}\vleftofline{“}This is all that women do,\\
Sit and answer them that woo;\\
Deck themselves in new attire,\\
To entangle fresh desire;\\
After dinner sing and play,\\
Or dancing, pass the time away.”
\end{scverse}

“England,” says a French writer of the seventeenth century, “is the paradise of
women, as Spain and Italy are their purgatory.”\dcfootnote{ %a
Description of England by Jorevin de Rocheford.
Paris, 1672.
} %end footnote

The musical instruments principally in use in barbers’ shops, during the
sixteenth and seventeenth centuries, were the cittern, the gittern, the lute, and
the virginals. Of these the cittern was the most common, perhaps because most
easily played. It was in shape somewhat like the English guitar of the last
century, but had only four double strings of wire, \ie, two to each note.\dcfootnote{ %b
Sir John Hawkins, in his \textit{History of Music}, vol.ii.,
p.~602, 8vo., copies the \textit{Cistrum} from Mersenne, as the
\textit{Cittern}, but it has six strings, and therefore more closely
resembles the English guitar.
} %end footnote
These
were tuned to the notes \textit{g}, \textit{b}, \textit{d}, and \textit{e} of the present treble staff, or to corresponding
intervals; for no rules are given concerning the pitch of these instruments,
unless they were to be used in concert. The instructions for tuning are generally
to draw up the treble string as high as possible, without breaking it, and to tune
the others from that. A particular feature of the cittern was the carved head,
which is frequently alluded to by the old writers.\dcfootnote{ %c
In \textit{Love’s Labour Lost}, act v., sc. 2, Boyet compares
Holofernes’ countenance to that of a cittern head. In
Forde's \textit{Lovers’ Melancholy}, act ii., sc. 1, “Barbers shall
wear thee on their citterns;” and in Fletcher’s \textit{Love’s
Cure}, “Yon cittern head! you ill-countenanced cur!”
\&c., \&c.
} %end footnote 
Playford in his “\textit{Musick’s
Delight on the Cithren} restored and refined to a more easie and pleasant manner of
playing than formerly,” 1666, speaks of having revived the instrument, and restored
it to what it was in the reign of Queen Mary, and his tuning agrees with
that in Anthony Holborne’s \textit{Cittharn Schoole}, 1597, and in Thomas Robinson’s \textit{New
Citharen Lessons}, 1609. The peculiarity of the cittern, or cithren, was that the
third string was tuned lower than the fourth, so that if the first or highest string
were tuned to \textit{e}, the third would be the \textit{g} below, and the fourth the intermediate \textit{b}.
The cittern appears to have been an instrument of English invention.\dcfootnote{ %
The word \textit{Cetera}, as employed by Galilei (father of
the great astronomer, Galileo Galilei), I assume to mean
Cittern, because the word \textit{Liuto}, for Lute, was in common
use. He says, “Fu la \textit{Cetera} usata prima tra gli Inglesi
che da altre nazioni, nella quale Isola si lavoravano già
in eccellenza; quantunque hoggi le più riputate da loro
siano quelle che si lavorano in Brescia; con tutto questo
è adoperata ed apprezzata da nobili, e fu così detta dagli
autori di essa, per forse resuscitare l’antica Cithara; ma
la differenza che sia tra la nostra e quella, si è possuto
benissimo conoscere da quello che se n’è di sopra detto.”—
\textit{Dialogo di Vincenzo Galilei, nobile Fiorentino}, fol.~1581,
p.~147.
} %end footnote

Of the gittern or ghitterne, I can say but little, not having seen any instruction-book
for the instrument. Ritson says it differed chiefly from the cittern
\pagebreak
%%102
%%===============================================================================
in being strung with gut instead of wire; and, from the various allusions to it,
I have no doubt of his correctness. Perhaps, also, it was somewhat less in size.
In the catalogue of musical instruments left in the charge of Philip van Wilder,
at the death of Henry VIII, we find “four Gitterons, which are called Spanish
vialles.” As Galilei says, in 1581, that “Viols are little used in Spain, and that
they do not make them,”\dcfootnote{ %a
“La viola da gamba, e da braccio, nella Spagna non
se ne fanno, e poco vi si usano.”—\textit{Dialogo della Musica},
fol. 1581., p.~147.
} %end footnote
I assume Spanish viol to mean the guitarra, or guitar.
The gittern is ranked with string instruments in the following extract from the
old play of \textit{Lingua}, written in this reign:—

\settowidth{\versewidth}{’Tis true the finding of a dead horse-head}
\begin{scverse}
\vleftofline{“}’Tis true the finding of a dead horse-head\\
Was the first invention of \textit{string} instruments,\\
Whence rose the \textit{Gitterne, Viol, and the Lute};\\
Though others think the Lute was first devis’d\\
In imitation of a tortoise back,\\
Whose sinews, parched by Apollo’s beams,\\
Echo’d about the concave of the shell:\\
And seeing the shortest and smallest gave shrillest sound,\\
They found out \textit{Frets}, whose sweet diversity\\
(Well touched by the skilful learned fingers)\\
Raiseth so strange a multitude of \textit{Chords};\\
Which, their opinion, many do confirm,\\
Because \textit{Testudo} signifies a Lute.’’\\
\vin\vin\vin\vin\vin\vin\textit{Dodsley's Old Plays}, vol. v., p.~198.
\end{scverse}

Coles, in his Dictionary, describes gittern as a \textit{small} sort of cittern, and Playford
printed \textit{Cithren and Gittern Lessons, with plain and easie Instructions for Beginners
thereon}, together in one book, in 1659. Ritson may have gained his information
from this book, as he mentions it in the second edition of his \textit{Ancient Songs}, but
I have not succeeded in finding a copy.

The lute (derived from the Anglo-Saxon \textit{Hlud}, or \textit{Lud}, \ie, \textit{sounded}), was
once the most popular instrument in Europe, although now rarely to be seen,
except represented in old pictures. It has been superseded by the guitar, but
for what reason it is difficult to say, unless from the greater convenience of the
bent sides of the guitar for holding the instrument, when touching the higher notes
of the finger-board. The tone of the lute is decidedly superior to the guitar, being
larger, and having a convex back, somewhat like the vertical section of a gourd, or
more nearly resembling that of a pear. As it was used chiefly for accompanying
the voice, there were only eight frets, or divisions of the finger-board, and these
frets (so called from fretting, or stopping the strings) were made by tying
pieces of cord, dipped in glue, tightly round the neck of the lute, at intervals
of a semitone. It had virtually six strings, because, although the number
was eleven or twelve, five, at least, were doubled, the first, or treble, being
sometimes a single string.\dcfootnote{ %b
I speak only of the usual English lute. There were
lutes of various sizes, from the mandura, or mandore,
to the theorbo and arch-lute; some with less, and others
with more strings.
} %end footnote 
The head, in which the pegs to turn the strings were
\pagebreak
%%103
%%===============================================================================
inserted, receded almost at a right angle. The most usual mode of tuning it was
as follows: assuming \textit{c} in the third space of the treble clef to be the pitch of the
first string (\ie, \textit{cc} in the scale given at page 14), the base, or sixth string would
be \textit{C}; the tenor, or fifth, \textit{F}; the counter-tenor, or fourth, \textit{b} flat; the great
mean, or third, \textit{d}; the small mean, or second, \textit{g}; and the minikin, or treble, \textit{cc}.\dcfootnote{ %a
The notes which these letters represent will be seen
by referring to the scale at p.~14.
} %end footnote

Lute strings\dcfootnote{ %
Mace, in his \textit{Musick’s Monument}, 1678, speaking of
lute-strings, says, “Chuse your trebles, seconds, and
thirds, and some of your small octaves, especially the
sixth, out of your \textit{Minikins}; the fourth and fifth, and
most of your octaves, of \textit{Venice Catlins}; your \textit{Pistoys} or
\textit{Lyons} only for the great bases.” In the list of CustomHouse
duties printed in 1545, the import duty on “lutestrings
called \textit{Mynikins}” was 22\textit{d}. the gross, but as no
other lute-strings are named, I assume that only the
smallest were then occasionally imported. Minikin is
one of the many words, derived from music or musical
instruments, which have puzzled the commentators on
the old dramatists. The first string of a violin was also
called a minikin.
} %end footnote
were a usual present to ladies as new-year’s gifts. From
Nichols’ \textit{Progresses} we learn that queen Elizabeth received a box of lute-strings,
as a new-year’s gift, from Innocent Corry, and at the same time, a box of lute-strings
and a glass of sweet water from Ambrose Lupo. When young men
in want of money went to usurers, it was their common practice to lend it
in the shape of goods which could only be re-sold at a great loss; and lute-strings
were then as commonly the medium employed as bad wine is now. In Lodge’s
\textit{Looking Glasse for London and Englande}, 1594, the usurer being very urgent
for the repayment of his loan, is thus answered, “I pray you, Sir, consider that
my loss was great by the commodity I took up; you know, Sir, I borrowed of you
forty pounds, whereof I had ten pounds in money, and thirty pounds in lute-strings, 
which, when I came to sell again, I could get but five pounds for them, so
had I, Sir, but fifteen pounds for my forty.” So in Dekker’s \textit{A Night’s Conjuring}, 
the spendthrift, speaking of his father, says, “He cozen’d young gentlemen
of their land, only for me, had acres mortgaged to him by wiseacres for three
hundred pounds, paid in hobby-horses, dogs, bells, and lute-strings, which, if they
had been sold by the drum, or at an out-rop (auction), with the ciy of ‘No man
better?’ would never have yielded \textit{£}50.” Nash alludes twice to the custom. In
\textit{Will Summer’s Last Will and Testament}, he says, “I know one that ran in debt,
in the space of four or five years, above fourteen thousand pounds in lute-strings
and grey paper;” and in \textit{Christ’s Tears over Jerusalem}, 1593; “In the first instance, 
spendthrifts and prodigals obtain what they desire, but at the second time
of their coming, it is doubtful to say whether they shall have money or no: the world
grows hard, and we are all mortal: let them make him any assurance before a
judge, and they shall have some hundred pounds (per consequence) in silks and
velvets. The third time, if they come, they have baser commodities. The fourth
time, lute-strings and grey paper; and then, I pray you pardon me, I am not for
you: pay me what you owe me, and you shall have anything.” (Dodsley, v.~9,
p.~22.)

The virginals (probably so called because chiefly played upon by young girls),
resembled in shape the “square” pianoforte of the present day, as the harpsichord
did the “grand.” The sound of the pianoforte is produced by a hammer \textit{striking}
the strings, but when the keys of the virginals or harpsichord were pressed, the
“jacks,” (slender pieces of wood, armed \pagebreak at the upper ends with quills) were
%%104
%%===============================================================================
raised to the strings, and acted as \textit{plectra}, by impinging, or twitching them.
These \textit{jacks} were the constant subject of simile and pun; for instance, in a play
of Dekker’s, where Matheo complains that his wife is never at home, Orlando says,
“No, for she’s like a pair of virginals, always with \textit{jacks} at her tail.”—(Dodsley’s
Old Plays, vol. iii., p.~398). And in Middleton’s \textit{Father Hubburd’s Tales}, describing
Charity as frozen, he says, “Her teeth chattered in her head, and leaped
up and down like virginal jacks.”

One branch of the barber’s occupation in former days was to draw teeth, to bind
up wounds, and to let blood. The parti-coloured pole, which was exhibited at the
doorway, painted after the fashion of a bandage, was his sign, and the teeth
he had drawn were suspended at the windows, tied upon lute strings. The lute,
the cittern, and the gittern hung from the walls, and the virginals stood in the
corner of his shop. “If idle,” says the author of \textit{The Trimming of Thomas
Nashe}, “barbers pass their time in life-delighting musique,” (1597). The
barber in Lyly’s \textit{Midas}, (1592), says to his apprentice, “Thou knowest I have
taught thee the knacking of the hands,\dcfootnote{ %
The knacking of the hands was a peculiar crack with
the fingers, by knocking them together, which every
barber was expected to make while shaving a customer.
} %end footnote
like the tuning of a cittern,” and
Truewit, in Ben Jonson’s \textit{Silent Woman}, wishes the barber “may draw his own
teeth, and add them to the lute-string.” In the same play, Morose, who had
married the barber’s daughter, thinking her faithless, exclaims “That cursed
barber! I have married his \textit{cittern}, that is common to all men.” One of the
commentators not understanding this, altered it to “I have married his \textit{cistern},”~\&c. 
Dekker also speaks of “a barber’s cittern for every serving-man to play
upon.”

One of the \textit{Merrie-conceited jests of George Peele} is the stealing of a barber’s
lute, and in \textit{Lord Falkland’s Wedding Night}, we read “He has travelled
and speaks languages, as a barber’s boy plays o’th’ gittern.” Ben Jonson says,\dcfootnote{ %b
\textit{Every man in his humour.} Act iii., sc, 2.
} %end footnote
“I can compare him to nothing more happily than a barber’s virginals; for every
man may \textit{play} upon him,” and in \textit{The Staple of News}, “My barber Tom, one
Christmas, got into a Masque at court, by his wit and the good means of his
cittern, holding up thus for one of the music.” To the latter passage Gifford adds
another in a note. “For you know, says Tom Brown, that a cittern is as natural
to a barber, as milk to a calf, or dancing bears to a bagpiper.”

As to the music they played, we may assume it to have been, generally,
the common tunes of the day, and such as would be familiar to all. Morley, in
his \textit{Introduction to Music}, tells us that the tune called the \textit{Quadrant Pavan}, was
called \textit{Gregory Walker}, “because it walketh ’mongst barbers and fiddlers more
common than any other,” and says in derision, “Nay, you sing you know not
what; it should seem you came lately from a barber’s shop, where you had
\textit{Gregory Walker}, or a Coranto, played in the new proportions by them lately found
out.” Notwithstanding this, we find the \textit{Quadran Pavan} (so called, I suppose,
because it was a pavan for four to dance) was one of the tunes arranged for
queen Elizabeth in her Virginal Book; \pagebreak and Morley, himself, arranged it for
%%105
%%===============================================================================
several instruments in his \textit{Consort Lessons}. I have alluded to the custom of
introducing \textit{old} songs into plays, and playing \textit{old} tunes at the beginning and end
of the acts, at p.~72. Queen Elizabeth’s Virginal Book, and Lady Neville’s,
contain little else than old tunes, arranged with variations, or as then more
usually termed, with “division.” It is often difficult to extract the air accurately
from these arrangements, if there be no other copy as a guide. Occasionally
a mere skeleton of the tune is given, sometimes it is “in prolation,” \ie, with
every note drawn out to two, four, or eight times its proper duration, sometimes
the melody is in the base, at others it is to be found in an inner part.

The rage for popular tunes abroad had shewn itself in the Masses set to
music by the greatest composers. Baini, in his Life of Palestrina, gives, what
he terms, a \textit{short} list (“breve elenco”) of some of them. It contains the
names of eighty secular tunes upon which Masses had been composed, and sung
even in the Pope’s chapel. The tunes have principally French names, some
are of lascivious songs, others of dance tunes. He names fifty different authors
who composed them, and intimates that there is a much larger number than he
has cited in the library of the Vatican.\dcfootnote{\scriptsize %a
“Memorie storico-critiche della Vita, e delle Opere di
Giovanni Pierluigi da Palestrina.”—\textit{Roma}, 2 vols, 4to.,
1828. Vol. i., p.~136, et seq. This evil was checked by a
decree of the Council of Trent.
} %end footnote
Even our island was not quite irreproachable
on this point. Shakespeare speaks of Puritans singing psalms to
hornpipes, and the Presbyterians sang their \textit{Divine Hymns} to the tunes of
popular songs, the titles of some of which the editor of \textit{Sacred Minstrel}sy (vol. i.,
p.~7) “would not allow to sully his pages.” Generally, however, the passion
for melody expended itself in singing old tunes about the country, in the streets,
and at the ends of plays, in playing them in barbers’ shops, or at home, when
arranged for chamber use with all the art and embellishment our musicians could
devise. The scholastic music of that age, great as it was, was so entirely devoted
to harmony, and that harmony so constructed upon old scales, that scarcely anything
like tune could be found in it—I mean such tune as the uncultivated ear
could carry away. Many would then, no doubt, say with Imperia, “I cannot abide
these dull and lumpish tunes; the musician stands longer a pricking them than
I would do to hear them: no, no, give me your light ones.”—(Middleton’s \textit{Blurt,
Master Constable.}) No line of demarcation could be more complete than that
between the music of the great composers of the time, and, what may be termed,
the music of the people. Perhaps the only instance of a tune by a well-known
musician of that age having been afterwards used as a ballad tune, is that of \textit{The
Frog Galliard}, composed by Dowland. Musicians held ballads in contempt, and
the great poets rarely wrote in ballad metre.

Dr.~Drake, in his \textit{Shakespeare and his Times}, gives a list of two hundred and
thirty-three British poets\dcfootnote{\scriptsize %b
The word “Poet” is here too generally applied. “It
is already said (and, as I think, truly said) it is not
rhyming and versing that maketh poesy: one may be a
poet without versing, and a versifier without poetry.”—
\textit{Sir Philip Sidney’s Defence of Poesy}.
} %end footnote 
(forty major, and one hundred and ninety-three
minor), who were contemporaneous with Shakespeare, and even that list, large as
it is, might be greatly extended from miscellanies, and from ballads. Some idea
of the number of ballads that were printed \pagebreak in the early part of the reign of
%%106
%%===============================================================================
Elizabeth may be formed from the fact that seven hundred and ninety-six ballads,
left for entry at Stationers’ Hall, remained in the cupboard of the council chamber
of the company at the end of the year 1560, to be transferred to the new
Wardens, and only forty-four books.\dcfootnote{ %a
See \textit{Collier's Extracts from the Registers of the Stationers’
Company}, vol. i., p.~28.
} %end footnote
As to the latter part of her reign, see
Bishop Hall, 1597.

\settowidth{\versewidth}{Some drunken rhymer thinks his time well spent}
\begin{scverse}
\vleftofline{“}Some drunken rhymer thinks his time well spent\\
If he can live to see his name in print;\\
Who, when he once is fleshed to the press,\\
And sees his handsell have such fair success,\\
\textit{Sung to the wheel, and sung unto the pail},\dcfootnote{ %b
“Sung to the wheel,” \ie, to the spinning wheel; and
“sung to the pail,” sung by milk-maids, of whose love of
ballads further proofs will be adduced.
} %end footnote
\\
He sends forth \textit{thraves\dcfootnote{ %c
“Thrave” signifies a number of sheaves of corn set
up together; metaphorically, an indefinite number of anything.—
\textit{Nares' Glossary}.
} %end footnote 
of ballads to the sale}.”
\end{scverse}

And to the same purport, in \textit{Martin Mar-sixtus}, 1592: “I lothe to speak it,
every red-nosed rhymester is an author; every drunken man’s dream is a book;
and he, whose talent of little wit is hardly worth a farthing, yet layeth about him
so outrageously as if all Helicon had run through his pen: in a word, scarce a cat
can look out of a gutter, but out starts a halfpenny chronicler, and presently a
proper new ballet of a strange sight is indited.”

Henry Chettle, in his pamphlet entitled \textit{Kind Hart’s Dream}, 1592, speaks of
idle youths singing and selling ballads in every corner of cities and market towns,
and especially at fairs, markets, and such like public meetings. Contrasting that
time with the simplicity of former days, he says, “What hath there not, contrary
to order, been printed? Now ballads are abusively chanted in every street; and
from London this evil has overspread Essex and the adjoining counties. There is
many a tradesman of a worshipful trade, yet no stationer, who after a little bringing
up apprentices to singing brokery, takes into his shop some fresh men, and
trusts his servants of two months’ standing with a dozen groatsworth of ballads.
In which, if they prove thrifty, he makes them pretty chapmen, able to spread
more pamphlets by the state forbidden, than all the booksellers in London.”
He particularly mentions the sons of one Barnes, most frequenting Bishop’s
Stortford, the one with a squeaking treble, the other with an ale-blown base, as
bragging that they earned twenty shillings a day; whilst others, horse and man,
the man with many a hard meal, and the horse pinched for want of provender,
have together hardly taken ten shillings in a week.

In a pamphlet intended to ridicule the follies of the times, printed in 1591, the
writer says, that if men that are studious would “read that which is good, a poor
man may be able”—not to obtain bread the cheaper, but as the most desirable of
all results, he would be able “to buy three ballets for a halfpenny.”\dcfootnote{ %d
\textit{Fearefull and lamentable effects of two dangerous Comets
that shall appeare}, \&c., 4to, 1591.
} %end footnote
\settowidth{\versewidth}{And tell prose writers, stories are so stale}
\begin{scverse}
\vleftofline{“}And tell prose writers, stories are so stale,\\
That penny ballads make a better sale.”\\
\vin\vin\vin\vin\vin\vin\textit{Pasquill’s Madness}, 1600.
\end{scverse}

The words of the ballads were written by such men as Elderton, “with his ale-crammed 
nose,” and Thomas Deloney, “the balleting silk-weaver of Norwich.”

\pagebreak
%%107
%%===============================================================================

\noindent The former is thus described in a MS. of the time of James I., in the possession
of Mr. Payne Collier:—

\settowidth{\versewidth}{Will. Elderton’a red nose is famous everywhere,}
\begin{scverse}
\vleftofline{“}Will. Elderton’a red nose is famous everywhere,\\
And many a ballet shows it cost him very dear;\\
In ale, and toast, and spice, he spent good store of coin,\\
You need not ask \textit{him} twice to take a cup of wine.\\
But though his nose was red, his hand was very white,\\
In work it never sped, nor took in it delight;\\
No marvel therefore ’tis, that white should be his hand,\\
That ballets writ a score, as you well understand.”
\end{scverse}

Nashe, in \textit{Have with you to Saffron Walden}, says of Deloney, “He hath rhyme
enough for all miracles, and wit to make a \textit{Garland of Good Will}, \&c., but
whereas his muse, from the first peeping forth, hath stood at livery at an ale-house
wisp, never exceeding a penny a quart, day or night—and this dear year,
together with the silencing of his looms, scarce that—he is constrained to betake
himself to carded ale” (i. e., ale mixed with small beer), “whence it proceedeth
that since Candlemas, or his jigg of \textit{John for the King}, not one \textit{merry} ditty will
come from him; nothing but \textit{The Thunderbolt against swearers, Repent, England,
repent}, and the \textit{Strange Judgments of God}.”

In 1581, Thomas Lovell, a zealous puritan, (one who objected to the word
Christmas, as savouring too much of popery, and calls it Chris\textit{tide}), published
“A Dialogue between Custom and Verity, concerning the use and abuse of
dauncinge and minstralsye.” From this, now rare book, Mr. Payne Collier has
printed various extracts. The object was to put down dancing and minstrelsy;
Custom defends and excuses them, and Verity, who is always allowed to have the
best of the argument, attacks and abuses them. It shows, however, that the old
race of minstrels was not quite extinct. Verity says:—

\settowidth{\versewidth}{But this do minstrels clean forget:}
\begin{scverse}
\begin{altverse}
\vleftofline{“}But this do minstrels clean forget:\\
Some godly songs they have,\\
Some wicked ballads and unmeet,\\
As companies do crave.\\
For filthies they have filthy songs;\\
For ‘some’ lascivious rhymes;\\
For honest, good; for sober, grave\\
Songs; so they watch their times.\\
Among the lovers of the truth,\\
Ditties of truth they sing;\\
Among the papists, such as of\\
Their godless legends spring\ldots\ \\
\textit{The minstrels do, with instruments,\\
With songs, or else with jest,\\
Maintain themselves}: but, as they use, [act]\\
Of these naught is the best.”\\
\vin\textit{Collier’s Extracts Reg. Stat. Comp}., vol. ii., pp.~144,~145.
\end{altverse}
\end{scverse}

Carew, in his \textit{Survey of Cornwall}, 1602, speaking of Tregarrick, then the
\pagebreak
%%108
%%===============================================================================
residence of Mr. Buller, the sheriff, says, “It was sometime the Wideslade’s
inheritance, until the father’s rebellion forfeited it,” and the “son then led
a walking life with his harp, to gentlemen’s houses, where-through, and by his
other active qualities, he was entitled Sir Tristram; neither wanted he (as some
say) a ‘\textit{belle Isound},’ the more aptly to resemble his pattern.”

So in the “Pleasant, plain, and pithy pathway, leading to a virtuous and honest
life” (about 1550),

\settowidth{\versewidth}{By which my minstrelsy and my fair speech and sport,}
\begin{scverse}\vleftofline{“}Very lusty I was, and pleasant withall,\\
To sing, dance, and play at the ball\ldots\  \\
And besides all this, I could then finely play\\
On the harp much better than now far away,\\
By which my minstrelsy and my fair speech and sport,\\
All the maids in the parish to me did resort.”
\end{scverse}

As minstrelsy declined, the harp became the common resource of the blind,
and towards the end of the reign of Elizabeth, harpers were proverbially blind:—
%\settowidth{\versewidth}{If thou’lt not have her look’d on by thy guests,}
\begin{scverse}\vleftofline{“}If thou’lt not have her look’d on by thy guests,\\
Bid none but harpers henceforth to thy feasts.”\\
\vin\vin\vin\vin\vin\vin\vin\vin\textit{Guilpins Skialetheia}, 1598.
\end{scverse}

There are many ballads about blind harpers, and many tricks were played upon
them, such as a rogue engaging a harper to perform at a tavern, and stealing the
plate “while the unseeing harper plays on.” As to the other street and tavern
musicians, Gosson tells us, in his \textit{Short Apologie of the Schoole of Abuse}, 1586,
that “London is so full of unprofitable pipers and fiddlers, that a man can no
sooner enter a tavern, than two or three cast (\ie, companies) of them, hang at
his heels, to give him a dance before he departs,” but they sang ballads and
catches as well as played dances. They also played at dinner,
%\settowidth{\versewidth}{But to the music, nor a drop of wine}
\begin{scverse}\vin\vin\vin “Not a dish removed\\
But to the music, nor a drop of wine\\
Mixt with the water, without harmony.”
\end{scverse}

“Thou need no more send for a fidler to a feast (says Lyly), than a beggar to~a~fair.”

Part-Singing, and especially the singing Rounds, or Roundelays, and Catches,
was general throughout England during the sixteenth and seventeenth centuries.
In the Moralities and the earliest plays, when part-music was sung instead of old
ballads, it was generally in Canon, for although neither Round, Catch, nor Canon
be specified, we find some direction from the one to the other to sing \textit{after} him.\dcfootnote{ %a
Catch, Round or Roundelay, and Canon in unison, are,
in music, nearly the same thing. In all, the harmony is to
be sung by several persons; and is so contrived, that,
though each sings precisely the same notes as his fellows,
yet, by beginning at stated periods of time from each
other, there results a harmony of as many parts as there
are singers. The Catch differs only in that the words of
one part are made to answer, or catch the other; as, “Ah!
how, Sophia,” sung like “a house o' fire,” “Burney’s
History,” like “burn his history,” \&c.
} %end footnote
Thus, in the old Morality called \textit{New Custome} (Dodsley, vol. i.), Avarice says:—
%\settowidth{\versewidth}{If you he good fellows, let us depart with a song.”}

\begin{scverse}\vleftofline{“}But, Sirs, because we have tarried so long,\\
If you be good fellows, let us depart with a song.”
\end{scverse}
To which Cruelty answers:—
\begin{scverse}\vleftofline{“}I am pleased, and therefore let every man\\
\textit{Follow after in order} as well as he can.”
\end{scverse}

\pagebreak

%%109
%%===============================================================================
\noindent And in John Heywood’s \textit{The Four P’s}, one of our earliest plays, the Apothecary,
haying first asked the Pedler whether he can sing at sight, says, “Who that lyste
\textit{sing after me}.” In neither case are the words of the Round given.

Tinkers, tailors, blacksmiths, servants, clowns, and others, are so constantly
mentioned as singing music in parts, and by so many writers, as to leave no doubt
of the ability of at least many among them to do so.

Perhaps the form of Catch, or Round, was more generally in favour, because,
as each would sing the same notes, there would he but one part to remember, and
the tune would guide those who learnt by ear.

We find Roundelays generally termed “merry,” and cheerfulness was the
common attribute of country songs.

In Peele’s \textit{Arraignment of Paris}, 1584:—

\settowidth{\versewidth}{“Some Rounds, or merry Roundelays,—we sing no other songs;}
\begin{scverse}
“Some Rounds, or merry Roundelays,—we sing no other songs;\\
Your \textit{melancholic notes not to our country mirth belongs}.”
\end{scverse}

And in his King Edward I., the Friar says:—

\begin{scverse}
“And let our lips and voices meet in a merry country song.”
\end{scverse}

In Shakespeare’s \textit{A Winter’s Tale}, when Autolycus says that the song is a
merry one, and that “there’s scarce a maid westward but she sings it,” Mopsa
answers, “We can both sing it: if thou wilt hear a part, thou shalt hear—’tis
in three parts.”

Tradesmen and artificers had evidently not retrograded in their love of music
since the time of Chaucer, whose admirable descriptions have been before quoted,
(p.~33, et seq.) Occleve, a somewhat later poet, has also remarked the different
effect produced by the labour of the hand and of the head. He says:—

\settowidth{\versewidth}{And forth their labour passeth with gladness;}
\begin{scverse}
\vleftofline{“}These artificers see I, day by day,\\
In the hottest of all their business,\\
Talken and sing, and make game and play,\\
And forth their labour passeth with gladness;\\
But we labour in travailous stillness;\\
We stoop and stare upon the sheep-skin,\\
And keep most our song and our words in.”
\end{scverse}

From the numerous allusions to their singing in parts, I have selected the
following. Peele, in his \textit{Old Wive’s Tale}, 1595, says, “This \textit{smith} leads a life as
merry as a king. Sirrah Frolic, I am sure you are not without some Round or
other; no doubt but Clunch (the smith) can bear his part;” which he accordingly
does. In \textit{Damon and Pithias}, 1571, Grimme the \textit{collier} sings “a bussing base,”
and Jack and Will, two of his fellows, “quiddell upon it,” that is, they sing the
tune and words of the song whilst he buzzes the burden or under-song. In Ben
Jonson’s \textit{Silent Woman}, we find, “We got this cold sitting up late and singing
Catches with \textit{cloth-workers}.” In Shakespeare’s \textit{Twelfth Night}, Sir Toby says,
“Shall we rouse the night-owl in a Catch that will draw three souls out of one
\textit{weaver}?” and, in the same play, Malvolio says, “Do you make an ale-house of
my lady’s house that ye squeak out your \textit{cozier’s Catches}, without any mitigation 
or remorse of voice?” Dr.~Johnson says cozier \pagebreak means a \textit{tailor}, from “coudre,”
%%110
%%===============================================================================
to sew; but Nares quotes four authorities to prove it to mean a \textit{cobbler}. In
Beaumont and Fletcher’s \textit{Coxcomb} we find—

\settowidth{\versewidth}{“Where were the \textit{Watch} the while? Good sober gentlemen,}
\begin{scverse}
“Where were \textit{the Watch} the while? Good sober gentlemen,\\
They were, like careful members of the city,\\
Drawing in diligent ale, and singing Catches.”
\end{scverse}

In \textit{A Declaration of egregious Impostures}, 1604, by Samuel Harsnet (afterwards
Archbishop of York), he speaks of “the master setter of Catches, or Rounds,
used to be sung by \textit{tinkers} as they sit by the fire, with a pot of good ale between
their legs.”

Sometimes the names of these Catches are given, as, for instance, “Three blue
beans in a blue bladder, rattle, bladder, rattle,” mentioned in Peele’s \textit{Old Wive’s
Tale}, in Ben Jonson’s \textit{Bartholomew Fair}, and in Dekker’s \textit{Old Fortunatus}; or
“Whoop, Barnaby,” which is also frequently named. But whoever will read the
words of those in \textit{Pammelia}, \textit{Deuteromelia}, Hilton’s \textit{Catch that catch can}, or Playford’s
\textit{Musical Companion}, will not doubt that many of the \textit{Catches} were intended for
the ale-house and its frequenters; but not so generally, the Rounds or Roundelays. 
Singing in parts was, by no means, confined to the meridian of London;
Carew, in his \textit{Survey of Cornwall}, 1602, says the same of Cornishmen: “Pastimes
to delight the mind, the Cornishmen have guary miracles [miracle plays] and
\textit{three-men’s songs}, cunningly contrived for the ditty, and pleasantly for the note.”

Catches seem to have increased in use towards the latter part of the seventeenth
century, for, although I cannot cite an instance of one composed by a
celebrated musician of Elizabeth’s reign, in that of Charles II. such cases were
abundant.

Some of the dances in favour in the reign of Elizabeth will be mentioned as
the tunes occur; the Queen herself danced galliards in her sixty-ninth year, and,
when given up by her physicians in her last illness, refusing to take medicine, she
sent for her band to play to her; upon which Beaumont, the French Ambassador,
remarks, in the despatch to his court, that he believed “she meant to die as
cheerfully as she had lived.” Her singing and playing upon the lute and
virginals have been so often mentioned, that I will not further allude to them
here.

\centerrule

\musictitle{All in a Garden Green.}

By the Registers of the Stationers’ Company we find that in 1565 William
Pickering had a license to print “A Ballett intituled \textit{All in a garden grene},
between two lovers;” and in 1568-9, William Griffith had a similar license. In
1584, “an excellent song of an outcast lover,” beginning “My fancie did I fire
in faithful form and frame,” to the tune of \textit{All in a garden grene}, appeared in
\textit{A Handeful of Pleasant Delites}.

In the rare tract called “Westward for smelts, or the Waterman’s fare of mad
merry Western Wenches,” quarto, 1603, the boatman, finding his fare sleeping,
sprinkles a little cool water on them with his oar, and, to “keep them from melancholy 
sleep,” promises “to strain the best voice he has, \pagebreak and not to cloy their ears
%%111
%%===============================================================================
with an \textit{old fiddler’s song}, as \textit{Riding to Rumford}, or \textit{All in a garden green}, but to
give them a new one of a serving man and his mistress, which neither fiddler nor
ballad-singer had ever polluted with their unsavoury breath.”

In the British Museum is a copy of “Psalmes, or Songs of Sion, turned into
the language, and set to the tunes of a strange land, by W[illiam] S[latyer],
intended for Christmas Carols, and fitted to divers of the most noted and common,
but solemne tunes, every where in this land familiarly used and knowne.” 1642.
Upon this copy, a former possessor has written the names of some of the tunes to
which the author designed them to be sung. \textit{One of these is All in a garden grene}.

The tune is in William Ballet’s Lute Book, from which this copy is taken, and
in \textit{The Dancing Masters} of 1651, 1670, 1686, 1690, \&c. The first part of the
air is the same as another in \textit{The Dancing Master}, called \textit{Gathering of Peascods}.
(See Index.)

The words are contained in a manuscript volume, in the possession of Mr.
Payne Collier.

\musicinfo{Moderate time.}{}

\lilypondfile[noindent, current-font-as-main, staffsize=17, 
noragged-right, language=english, nofragment]
{lilypond/111-all-in-a-garden-green}\normalsize
\pagebreak


%%112
%%===============================================================================

\settowidth{\versewidth}{And shines as bright and hot}
\begin{dcverse}\begin{altverse}
Quoth he, “Most lovely maid,\\
My troth shall aye endure;\\
And be not thou afraid,\\
But rest thee still secure,
\end{altverse}

\begin{altverse}
That I will love thee long\\
As life in me shall last;\\
Now I am strong and young,\\
And when my youth is past.
\end{altverse}

\begin{altverse}
When I am gray and old,\\
And then must stoop to age,\\
I’ll love thee twenty-fold,\\
My troth I here engage.”
\end{altverse}

\begin{altverse}
She heard with joy the youth,\\
When he thus far had gone;\\
She trusted in his truth,\\
And, loving, he went on:
\end{altverse}

\begin{altverse}
\vleftofline{“}Yonder thou seest the sun\\
Shine in the sky so bright,\\
And when this day is done,\\
And cometh the dark night,
\end{altverse}

\begin{altverse}
No sooner night is not,\\
But he returns alway,\\
And shines as bright and hot\\
As on this gladsome day.
\end{altverse}

\begin{altverse}
He is no older now\\
Than when he first was born;\\
Age cannot make him bow,\\
He laughs old Time to scorn.
\end{altverse}

\begin{altverse}
My love shall be the same,\\
It never shall decay,\\
But shine without all blame,\\
Though body turn to clay.”
\end{altverse}

\begin{altverse}
She listed to his song,\\
And heard it with a smile,\\
And, innocent as young,\\
She dreamèd not of guile.
\end{altverse}

\begin{altverse}
No guile he meant, I ween,\\
For he was true as steel,\\
As was thereafter seen\\
When she made him her weal.
\end{altverse}
\end{dcverse}

\vspace{-\baselineskip}
\settowidth{\versewidth}{Full soon both two were wed,}
\begin{scverse}
\begin{altverse}
Full soon both two were wed,\\
And these most faithful lovers\\
May serve at board at bed,\\
Example to all others.
\end{altverse}
\end{scverse}
\normalsize
\musictitle{Row Well, Ye Mariners.}

From the Registers of the Stationers’ Company, we find that in 1565-6,
William Pickering had a license to print a ballet entitled, \textit{Row well, ye mariners},
and in the following year, “Row well, ye mariners, moralized.” In 1566-7,
John Allde had a license to print “Stand fast, ye mariners,” which was, in all
probability, another moralization; and in the following year, two others; the one,
“Row well, ye mariners, moralized, with the story of Jonas,” the other, “Row
well, Christ’s mariners.” In 1567-8, Alexander Lacy took a license to print
“Row well, God’s mariners,” and in 1569-70, John Sampson to print “Row
well, ye mariners, for those that look big.” These numerous entries sufficiently
prove the popularity of the original, and I regret the not having succeeded in
finding a copy of any of these ballads.

Three others, to the tune of \textit{Row well, ye mariners}, have been reprinted by
Mr. Payne Collier, in his \textit{Old Ballads}, for the Percy Society. The first (dated
1570)—
\settowidth{\versewidth}{A lamentation from Rome, how the Pope doth bewail}
\begin{scverse}
\vleftofline{“}A lamentation from Rome, how the Pope doth bewail\\
That the rebels in England cannot prevail.”
\end{scverse}
The second, “The end and confession of John Felton, who suffred in Paules
Churcheyarde, in London, the 8th August [1570], for high treason.” Felton
placed the Bull of Pope Pius V., excommunicating Elizabeth, on the gate of the
palace of the Bishop of London, and was hung on a gallows set up expressly
before that spot. The third, “A warning to London by the fall of Antwerp.”
\pagebreak
%113
%%===============================================================================
\DFNsingle

In \textit{A Handefull of Pleasant Delites}, 1584, there is “A proper sonet, wherein
the lover dolefully sheweth his grief to his love and requireth pity,” which is
also, to the tune of \textit{Row well, ye mariners}.

The tune is printed in Thomas Robinson’s \textit{Schoole of Musick}, fol., 1603, and
in every edition of \textit{The Dancing Master} that I have seen, from the first, dated
1651, to the eighteenth, 1725.

Not having the original words, a few verses from the “Lamentation from
Rome,” above mentioned, are given as a specimen of the merry political ballad of
those days. It is the Song of a fly buzzing about the Pope’s nose. The Pope and
his court are supposed to be greatly disconcerted at the news of the defeat of the
rebels in Northumberland.

\musicinfo{Moderate time and smoothly.}{}

\lilypondfile[noindent, current-font-as-main, staffsize=17, 
noragged-right, language=english, nofragment]
{lilypond/113-row-well-ye-mariners}\normalsize

\footnotetext[1]{\centering\tiny I have added the old burden over the verse, feeling no doubt of its having
been sung to this part of the tune.}
\pagebreak

%%114
%%===============================================================================

\indentpattern{010100001111}
\settowidth{\versewidth}{New news to him was brought that night,}
\begin{dcverse}\begin{patverse}
But as he was asleep,\\
Into the same again I got;\\
I crept therein so deep,\\
That I had almost burnt my coat.\\
New news to him was brought that night,\\
The rebels they were put to flight;\\
But, Lord, how then the Pope took on,\\
And called for a Mary-bone.\\
Up-ho!-make-haste,\\
My lovers all he like to waste;\\
Rise-Cardinal,-up-Priest,\\
Saint Peter he doth what he list.
\end{patverse}

\begin{patverse}
So then they fell to mess;\\
The friars on their heads did pray;\\
The Pope began to bless,\\
At last he wist not what to say.\\
It chanced so the next day morn,\\
A post came blowing of his horn,\\
Saying, Northumberland is take;\\
But then the Pope began to quake.\\
He-then-rubb’d-his-nose,\\
With pilgrim-salve he ’noint his hose;\\
Run-here,-run-there,\\
His nails, for anger, ’gan to pare.
\end{patverse}

\begin{patverse}
When he perceived well\\
The news was true to him was brought.\\
Upon his knees he fell,\\
And then Saint Peter he besought\\
That he would stand his friend in this,\\
To help to aid those servants his,\\
And he would do as much for him—\\
But Peter sent him to Saint Sim.\\
So-then-he-snuff’d,\\
The friars all about he cuff’d,\\
He-roar’d,-he-cried;\\
The priests they durst not once abide.
\end{patverse}

\begin{patverse}
The Cardinals then begin\\
To stay, and take him in their arms,\\
He spurn’d them on the shin.\\
Away they trudg’d, for fear of harms.\\
So then the Pope was left alone;\\
Good Lord! how he did make his moan!\\
The stools against the walls he threw,\\
And me, out of his nose he blew.\\
I-hopp’d,-I-skipp’d,\\
From place to place, about I whipp’d;\\
He-sware,-he-tare,\\
Till from his crown he pull’d the hair.
\end{patverse}
\end{dcverse}

\musictitle{Lord Willoughby.}

This tune is referred to under the names of Lord Willoughby; Lord Wil\-loughby’s
\textit{March}, and Lord Willoughby’s \textit{Welcome Home}. In Queen’s Elizabeth’s
Virginal Book, it is called Rowland.

In Lady Neville’s Virginal Book (MS., 1591), and in Robinson’s \textit{School of
Music}, 1603, it is called “Lord Willobie’s Welcome Home:” the ballad of The
Carman’s Whistle was to be sung to the tune of The Carman’s Whistle, or to
Lord Willoughby’s \textit{March}; and that of “Lord Willoughby—being a true relation
of a famous and bloody battel fought in Flanders, \&c., against the Spaniards;
where the English obtained a notable victory, to the glory and renown of our
nation”—was to the tune of “Lord Willoughby, \textit{\&c}.” A copy of the last will
be found in the Bagford Collection of Ballads, British Museum.

Peregrine Bertie, Lord Willoughby of Eresby, one of the bravest and most
skilful soldiers of this reign, had distinguished himself in the Low Countries in
1586, and in the following year, on the recall of the Earl of Leicester, was
made commander of the English forces. The tune, with which his name was
associated, was as popular in the Netherlands as in England, and continued so, in
both countries, long after his death, which occurred in 1601. It was printed at
Haerlem, with other English tunes, in 1626, in Neder-landtsche Gedenck-clanck,
under the name of \textit{Soet Robbert}, and \textit{Soet, soet Robbertchen} [Sweet Robert, and
Sweet, sweet little Robert], which it probably derived from some other ballad
sung to the time.

As the ballad of “Brave Lord Willoughby” is printed in Percy’s \textit{Reliques of
Ancient Poetry}, a few verses, only, are subjoined.
\pagebreak
%%115
%%===============================================================================

\musicinfo{In Marching time.}{}

\lilypondfile[noindent, current-font-as-main, staffsize=17, 
noragged-right, language=english, nofragment]
{lilypond/115-lord-willoughby}\normalsize

\settowidth{\versewidth}{For yonder comes Lord Willoughbey}
\begin{dcverse}
\begin{altverse}
Stand to it, noble pikemen,\\
And look you round about:\\
And shoot you right, you bowmen,\\
And we will keep them out:\\
You musquet and calìver men,\\
Do you prove true to me,\\
I’le be the foremost man in fight,\\
Says brave Lord Willoughbèy.
\end{altverse}

\begin{altverse}
The sharp steel-pointed arrows,\\
And bullets thick did fly,\\
Then did our valiant soldiers\\
Charge on most furiously;\\
Which made the Spaniards waver,\\
They thought it best to flee,\\
They fear’d the stout behaviour\\
Of brave Lord Willoughbèy.
\end{altverse}

\begin{altverse}
Then quoth the Spanish general,\\
Come let us march away,\\
I fear we shall be spoiled all\\
If here we longer stay;\\
For yonder comes Lord Willoughbey\\
With courage fierce and fell,\\
He will not give one inch of way\\
For all the devils in hell.
\end{altverse}

\begin{altverse}
And then the fearful enemy\\
Was quickly put to flight,\\
Our men pursued couragiously,\\
And caught their forces quite;\\
But at last they gave a shout,\\
Which ecchoed through the sky,\\
God, and St. George for England!\\
The conquerors did cry.
\end{altverse}
\end{dcverse}
\pagebreak

%%116
%%===============================================================================

\settowidth{\versewidth}{To the souldiers that were maimed,}
\begin{dcverse}
\begin{altverse}
To the souldiers that were maimed,\\
And wounded in the fray,\\
The queen allowed a pension\\
Of fifteen pence a day;\\
And from all costs and charges\\
She quit and set them free:\\
And this she did all for the sake\\
Of brave Lord Willoughbèy.
\end{altverse}

\begin{altverse}
Then courage, noble Englishmen,\\
And never be dismaid;\\
If that we be but one to ten\\
We will not be afraid\\
To fight with foraign enemies,\\
And set our nation free.\\
And thus I end the bloody bout\\
Of brave Lord Willoughbèy.
\end{altverse}
\end{dcverse}

\musictitle{All Flowers of the Broom.}

This is mentioned as a dance tune by Nicholas Breton, in a passage already
quoted from his \textit{Works of a young Wit}, 1577 (ante p.~91); and by Nashe, in the
following, from his \textit{Have with you to Saffron-Walden}, 1596:—

“Or doo as Dick Harvey did, that having preacht and beat downe three pulpits in
inveighing against dauncing, one Sunday evening, when his wench or friskin was footing
it aloft on the greene, with foote out and foote in, and as busie as might be at
\textit{Rogero}, \textit{Basilino}, \textit{Turkelony}, \textit{All the flowers of the broom}, \textit{Pepper is black}, \textit{Greene
Sleeves}, \textit{Peggie Ramsey},\dcfootnote{\centering %
All the tunes here mentioned will be found in this Collection, except \textit{Basilino}.
} %end footnote
he came sneaking behind a tree, and lookt on; and though
hee was loth to be seene to countenance the sport, having laid God’s word against it so
dreadfully; yet to shew his good will to it in heart, hee sent her eighteen pence in
hugger-mugger (\ie, in secret), to pay the fiddlers.”

The tune is contained in William Ballet’s Lute Book, under the name of
\textit{All floures in broome}.

\lilypondfile[noindent, current-font-as-main, staffsize=17, 
noragged-right, language=english, nofragment]
{lilypond/117-all-flowers-of-the-broom}\normalsize

\pagebreak
%%117
%%===============================================================================
\DFNdouble
\musictitle{I am the Duke of Norfolk, or Paul’s Steeple.}

This tune is frequently mentioned under both names. In Playford’s \textit{Dancing
Master}, from 1650 to 1695, it is called Paul’s Steeple. In his \textit{Division Violin},
1685, at page 2, it is called \textit{The Duke of Norfolk, or Paul’s Steeple}; and at
page 18, \textit{Paul’s Steeple, or the Duke of Norfolk}.

The steeple of the \textit{old} Cathedral of St. Paul was proverbial for height. In the
\textit{Vulgaria}, printed by Wynkin de Worde, in 1530, we read: “Poule’s Steple is a
mighty great thing, and so hye that unneth [hardly] a man may discerne
the wether cocke,—the top is unneth perceived.” So in Lodge’s \textit{Wounds of
Civil War}, a clown talks of the \textit{Paul’s Steeple of honour}, as the highest point
that can be attained. The steeple was set on fire by lightning, and burnt
down on the 4th June, 1561; and within seven days, a ballad of “The true
report of the burning of the steeple and church of Paul’s, in London,” was
entered, and afterwards printed by William Seres, “at the west-ende of Pawles
church, at the sygne of the Hedghogge.” In 1564, a ballad was entered for
“the encouraging all kind of men to the re-edifying and building Paul’s steeple
again;” but the spire was never re-constructed. Mr. Payne Collier has printed
a ballad, written on the occasion of the fire, in his \textit{Extracts from the Registers of
the Stationers’ Company}, vol. i., p.~40; and it seems to have been intended for the
time. The first verse is as follows:—

\settowidth{\versewidth}{Lament each one the blazing fire,}
\indentpattern{0101221}
\begin{scverse}
\begin{patverse}
\vleftofline{“}Lament each one the blazing fire,\\
That down from heaven came,\\
And burnt S. Pawles his lofty spire\\
With lightning’s furious flame.\\
Lament, I say,\\
Both night and day,\\
Sith London’s sins did cause the same.”
\end{patverse}
\end{scverse}

In 1562-3, John Cherlewood had a license for printing another, called “When
young Paul’s steeple, old Paul’s steeple’s child.”

In Fletcher’s comedy, \textit{Monsieur Thomas}, act iii., sc. 3, a fiddler, being questioned
as to what ballads he is best versed in, replies:

\begin{scverse}
\vleftofline{“}Under your mastership’s correction, I can sing\\
\textit{The Duke of Norfolk}; or the merry ballad\\
\textit{Of Diverus and Lazarus; The Rose of England;\\
In Crete, when Dedimus first began;\\
Jonas, his crying out against Coventry;\\
Maudlin, the merchant's daughter;\\
The Devil and ye dainty dames;\\
The landing of the Spaniards at Bow;\\
With the bloody battle at Mile-End}.”\dcfootnote{ %a
Of the ballads mentioned aboye, \textit{Diverus and Lazarus}
seems to be an intentional corruption of \textit{Dives and Lazarus}.
\textit{The Rose of England} may be—
\settowidth{\versewidth}{The rose, the rose, the English rose,}
\begin{fnverse}
\vleftofline{“}The rose, the rose, the English rose,\\
It is the fairest flower that blows;”
\end{fnverse}
a copy of which is in Mr. Payne Collier's Manuscript; or,
perhaps, Deloney’s ballad of \textit{Fair Rosamond}, reprinted in
\textit{Percy's Reliques of Ancient Poetry}. \textit{In Crete} is often referred
to as a ballad tune; for instance, \textit{My mind to me a
kingdom is}, was to be sung to the tune of \textit{In Crete}, according
to a black-letter copy in the Pepysian Collection,
\textit{Maudlin, the merchant's daughter}, is \textit{The merchant's daughter
of Bristow} [Bristol], to the tune of The maiden's joy. (See
Roxburghe Collection, vol~i., 232, or Collier’s Roxburghe
Ballads, p.~104). \textit{Ye dainty dames}, are the first words of
\textit{A warning for maidens}, to the tune of \textit{The ladies’ fall}. (Sec
Roxburghe Collection, vol. i., 501). \textit{The landing of the
Spaniards}, \&c. (probably on some mock-fight of the train
hands, who exercised at Mile-end) seems to be referred to
in\textit{ The Knight of the Burning Pestle}, act ii., sc. 2.
} %end footnote
\end{scverse}

\pagebreak

%%118
%%===============================================================================
\noindent In the Pepysian Collection, vol. i., 146, and Roxburghe Collection, vol. i., 180,
is a black-letter ballad, called “A Lanthorne for Landlords” to the tune of
\textit{The Duke of Norfolk}, the initial lines of which are—

\settowidth{\versewidth}{With sobbing grief my heart will break}
\begin{scverse}
\vleftofline{“}With sobbing grief my heart will break\\
Asunder in my breast, \&c.”
\end{scverse}

In \textit{The Loyal Garland}, 1686, and in the Roxburghe Collection, vol. ii., 188 (or
Collier’s Roxburghe Ballads, p.~312), \textit{God speed the plough, and bless the corn-mow},
\&c., to the tune of \textit{I am the Duke of Norfolk}, beginning—

\settowidth{\versewidth}{My noble friends, give ear,}
\indentpattern{001001}
\begin{scverse}
\begin{patverse}
\vleftofline{“}My noble friends, give ear,\\
If mirth you love to hear,\\
I’ll tell you as fast as I can,\\
A story very true:\\
Then mark what doth ensue,\\
Concerning a husbandman.”
\end{patverse}
\end{scverse}
This ballad-dialogue, between a husbandman and a serving-man, has been orally
preserved in various parts of the country. One version will be found in Mr. Davies
Gilbert’s \textit{Christmas Carols}; a second in Mr. J. H. Dixon’s \textit{Ancient Poems and
Songs of the Peasantry} (printed for the Percy Society); and a third in “Old
English Songs, as now sung by the Peasantry of the Weald of Surrey and Sussex,”
\&c.,; “harmonized for the Collector” [the Rev. Mr. Broadwood] “in 1843, by
G. A. Dusart.”

In the \textit{Collection of Poems on Affairs of State}, vol. iii., 70, is “A new ballad
to an old tune, called \textit{I am the Duke of Norfolk}.” It is a satire on Charles II.,
and begins thus:—

\settowidth{\versewidth}{I am a senseless thing, with a hey, with a hey;}
\begin{scverse}
\indentpattern{00110}
\begin{patverse}
\vleftofline{“}I am a senseless thing, with a hey, with a hey;\\
Men call me a king, with a ho;\\
To my luxury and ease,\\
They brought me o’er the seas,\\
With a hey nonny, nonny, nonny no.”
\end{patverse}
\end{scverse}

In Shadwell’s \textit{Epsom Wells}, 1673, act iii., sc. 1, we find, “Could I not play
\textit{I am the Duke of Norfolk}, \textit{Green Sleeves}, and the fourth Psalm, upon the
virginals?” and in Wycherley’s \textit{Gentleman Dancing Master}, Ger. says, “Sing
him \textit{Arthur of Bradley}, or \textit{I am the Duke of Norfolk}.”

A curious custom still remains in parts of Suffolk, at the harvest suppers, to
sing the song “I am the Duke of Norfolk” (here printed with the music); one
of the company being crowned with an inverted pillow or cushion, and another
presenting to him a jug of ale, kneeling, as represented in the vignette of the
Horkey. [See \textit{Suffolk Garland}, 1818, p.~402.] The editor of the \textit{Suffolk
Garland} says, that “this custom has most probably some allusion to the homage
formerly paid to the Lords of Norfolk, the possessors of immense domains in the
county.” To “serve the Duke of Norfolk,” seems to have been equivalent to
making merry, as in the following speech of \textit{Mine host}, at the end of the play of
\textit{The merry Devil of Edmonton}, 1617:—

% The following footnote text continues onto this page in the original but now fits on the previous page as would be normal
%\footnotetext[0]{\hspace{-2em}\textit{of Bristow} [Bristol], to the tune of The maiden's joy. (See
%Roxburghe Collection, vol~i., 232, or Collier’s Roxburghe
%Ballads, p.~104). \textit{Ye dainty dames}, are the first words of
%\textit{A warning for maidens}, to the tune of \textit{The ladies’ fall}. (Sec
%Roxburghe Collection, vol. i., 501). \textit{The landing of the
%Spaniards}, \&c. (probably on some mock-fight of the train
%hands, who exercised at Mile-end) seems to be referred to
%in\textit{ The Knight of the Burning Pestle}, act ii., sc. 2.}
\pagebreak


%%119
%%===============================================================================
\DFNsingle

\settowidth{\versewidth}{Ha! ere’t be night, \textit{I’ll serve the good Duke of Norfolk}.}
\begin{scverse}
\vleftofline{“}Why, Sir George, send for Spendle’s noise\dcfootnote{\centering %
Spindle’s noise, \ie, Spindle's band, or company of musicians.
} %end footnote
presently;\\
Ha! ere’t be night, \textit{I’ll serve the good Duke of Norfolk}.”
\end{scverse}
To which Sir John rejoins:—
\begin{scverse}
\vleftofline{“}Grass and hay! mine host, let’s live till we die,\\
And be merry; and there’s an end.”\\
\vin\vin\vin\vin\vin\textit{Dodsley’s Old Plays}, vol. v., 271.
\end{scverse}

Dr.~Letherland, in a note which Steevens has printed on King Henry IV.,
Part I., act ii., sc. 4 (where Falstaff says, “This chair shall be my state, this
dagger my sceptre, and this \textit{cushion my crown}”), observes that the country people
in Warwickshire also use a \textit{cushion for a crown}, at their harvest home diversions;
and in the play of King Edward IV., Part II., 1619, is the following passage:—
\settowidth{\versewidth}{Then comes a slave, one of those drunken sots,}
\begin{scverse}
\vleftofline{“}Then comes a slave, one of those drunken sots,\\
In with a tavern reck’ning for a supplication,\\
Disguised with a cushion on his head.”
\end{scverse}

In the Suffolk custom, he who is crowned with the pillow, is to take the ale, to
raise it to his lips, and to drink it off without spilling it, or allowing the cushion
to fall; but there was, also, another drinking custom connected with this tune.
In the first volume of \textit{Wit and Mirth, or\textit{ Pills to purge Melancholy}}, 1698 and
1707, and the third volume, 1719, is a song called \textit{Bacchus’ Health}, “to be sung
by all the company together, with directions to be observed.” They are as
follows: “First man stands up, with a glass in his hand, and sings—
\indentpattern{0202012002002002}
\begin{scverse}
\begin{patverse}
Here’s a health to jolly Bacchus, (\textit{sung three times})\\
I-ho, I-ho, I-ho;\\
For he doth make us merry, (\textit{three times})\\
I-ho, I-ho, I-ho.\\
\vleftofline{* }Come sit ye down together, (\textit{three times})\\
{\footnotesize (At this star all bow to each other and sit down.)}\\
I-ho, I-ho, I-ho;\\
And bring† more liquor hither (\textit{three times})\\
{\footnotesize (At this dagger all the company beckon to the drawer.)}\\
I-ho, I-ho, I-ho.\\
It goes into the * cranium, (\textit{three times})\\
{\footnotesize \vleftofline{(At this star the} first man drinks his glass, while the others sing and point at him.)}\\
I-ho, I-ho, I-ho;\\
And † thou’rt a boon companion (\textit{three times})\\
{\footnotesize \vleftofline{(At this dag}ger all sit down, each clapping the next man on the shoulder.)}\\
I-ho, I-ho, I-ho.
\end{patverse}
\end{scverse}
Every line of the above is to be sung three times, except I-ho, I-ho, I-ho. Then
the second man takes his glass and sings; and so round.

About 1728, after the success of \textit{The Beggars’ Opera}, a great number of other
ballad operas were printed. In the \textit{Cobblers’ Opera}, and some others, this tune is
called \textit{I am the Duke of Norfolk}; but in \textit{The Jovial Crew}, \textit{The Livery Rake}, and
\textit{The Lover his own Rival}, it is called \textit{There was a bonny blade}. It acquired that
name from the following song, which is still occasionally to be heard, and which
is also in \textit{Pills to purge Melancholy}, from 1698 to 1719:—

\pagebreak

%%120
%%===============================================================================

\indentpattern{010110}
\settowidth{\versewidth}{But ah! and alas! she was dumb, dumb, dumb.}
\begin{dcverse}\footnotesize
\begin{patverse}
\vin There was a bonny blade,\\
Had married a country maid,\\
And safely conducted her home, home, home;\\
She was neat in every part,\\
And she pleas’d him to the heart,\\
But ah! and alas! she was dumb, dumb, dumb.
\end{patverse}

\begin{patverse}
\vin She was bright as the day,\\
And brisk as the May,\\
And as round and as plump as a plum,\\
But still the silly swain\\
Could do nothing but complain\\
Because that his wife she was dumb.
\end{patverse}

\begin{patverse}
\vin She could brew, she could bake,\\
She could sew, and she could make,\\
She could sweep the house with a broom;\\
She could wash, and she could wring,\\
And do any kind of thing,\\
But ah! and alas! she was dumb.
\end{patverse}

\begin{patverse}
\vin To the doctor then he went,\\
For to give himself content,\\
And to cure his wife of the mum:\\
\vleftofline{“}Oh! it is the easiest part\\
That belongs unto my art\\
For to make a woman speak that is dumb.”
\end{patverse}

\begin{patverse}
\vin To the doctor he did her bring,\\
And he cut her chattering string,\\
And at liberty he set her tongue;\\
Her tongue began to walk,\\
And she began to talk\\
As though she never had been dumb.
\end{patverse}

\begin{patverse}
\vin Her faculty she tries,\\
And she fills the house with noise,\\
And she rattled in his ears like a drum;\\
She bred a deal of strife,\\
Made him weary of his life—\\
He’d give any thing again she was dumb.
\end{patverse}

\begin{patverse}
\vin To the doctor then he goes,\\
And thus he vents his woes:\\
\vleftofline{“}Oh! doctor, you’ve me undone;\\
For my wife she’s turn’d a scold,\\
And her tongue can never hold,\\
I’d give any kind of thing she was dumb.”
\end{patverse}

\begin{patverse}
\vin \vleftofline{“}When I did undertake\\
To make thy wife to speak,\\
It was a thing easily done,\\
But ’tis past the art of man,\\
Let him do whate’er he can,\\
For to make a scolding wife hold her tongue.”
\end{patverse}
\end{dcverse}

From the last line of the verses of this song, the tune also became known as
“Alack! and alas! she was dumb,” or “Dumb, dumb, dumb.”

\musicinfo{Rather slow.}{}

\lilypondfile[noindent, current-font-as-main, staffsize=17, 
noragged-right, language=english, nofragment]
{lilypond/120-i-am-the-duke-of-norfolk}\normalsize
\pagebreak


%%121
%%===============================================================================

\musictitle{Pepper is Black}

This tune is to be found in \textit{The Dancing Master}, from 1650 to 1690. It is
mentioned as a dance tune by Nashe in\textit{ Have with you to Saffron-Walden}, 1596.
(See ante p.~116.) A copy of the following ballad by Elderton is in the collection
of Mr. George Daniel, of Canonbury: “Prepare ye to the plough, to the tune
of \textit{Pepper is black}.”

\settowidth{\versewidth}{“The Queen holds the plough, to continue good seed,}
\begin{scverse}
\vleftofline{“}The Queen holds the plough, to continue good seed,\\
Trusty subjects, be ready to help if she need.”
\end{scverse}

\musicinfo{Moderate time.}{}

\lilypondfile[noindent, current-font-as-main, staffsize=14, 
noragged-right, language=english, nofragment]
{lilypond/121-pepper-is-black}\normalsize

\settowidth{\versewidth}{Can bring about that I found out,}
\begin{dcverse}
Parnaso hill, not all the skill\\
Of nymphs, or muses feigned,

Can bring about that I found out,\\
By Christ himself ordained, \&c.
\end{dcverse}

There are twelve stanzas, each of eight lines, subscribed W. Elderton. Printed
by Wm. How, for Richard Johnes.

\musictitle{Walsingham.}

This tune is in Queen Elizabeth’s, and Lady Neville’s, Virginal Books (with
thirty variations by Dr.~John Bull); in Anthony Holborne’s \textit{Cittharn Schoole},
1597; in Barley’s \textit{New Booke of Tablature}, 1596, \&c. It is called “\textit{Walsingham},”
“\textit{Have with you to Walsingham}” and “\textit{As I went to Walsingham}.”

It belongs, in all probability, to an earlier reign, as the Priory of Walsingham,
in Norfolk, which was founded during the Episcopate of William, Bishop of
Norwich (1146 to 1174), was dissolved in 1538.

Pilgrimages to this once famous shrine commenced in or before the reign of
Henry III., who was there in 1241. Edward I. was at Walsingham in 1280, and
again in 1296; and Edward II. in 1315. The author of \textit{The Vision of Piers
Ploughman}, says—

\settowidth{\versewidth}{Wenten to Walsyngham, and her [their] wenches after.}
\begin{scverse}
\vleftofline{“}Heremytes on a hepe, with hooked staves,\\
Wenten to Walsyngham, and her [their] wenches after.”
\end{scverse}

A curious reason why pilgrims should have both singers and pipers to accompany
them, will be found in note \textit{a}, at page 34.

Henry VII., having kept his Christmas of 1486-7, at Norwich, “from thence
went in manner of pilgrimage to Walsingham, where he visited Our Lady’s Church, 
famous for miracles;  and made his prayers \pagebreak and vows for help and deliverance.”
%%122
%%===============================================================================
And in the following summer, after the battle of Stoke, “he sent his banner to
be offered to Our Lady of Walsingham, where before he made his vows.”

“Erasmus has given a very exact and humorous description of the superstitions
practised there in his time. See his account of the \textit{Virgo Parathalassia}, in his
colloquy, intitled \textit{Peregrinatio Religionis ergo}. He tells us, the rich offerings in
silver, gold, and precious stones, that were shewn him, were incredible; there being
scarce a person of any note in England, but what some time or other paid a visit,
or sent a present, to Our Lady of Walsingham. At the dissolution of the monasteries
in 1538, this splendid image, with another from Ipswich, was carried to
Chelsea, and there burnt in the presence of commissioners; who, we trust, did not
burn the jewels and the finery.”—\textit{Percy’s Reliques}.

The tune is frequently mentioned by writers of the sixteenth and seventeenth
centuries. In act v. of Fletcher’s \textit{The Honest Man’s Fortune,} one of the servants
says, “I’ll renounce my five mark a year, and all the hidden art I have in carving,
to teach young birds to whistle \textit{Walsingham}.” A verse of “As you came from
Walsingham,” is quoted in \textit{The Knight of the Burning Pestle}, and in \textit{Hans Beerpot,
his invisible Comedy}, 4to., 1618.

In \textit{The weakest goes to the wall}, 1600, the scene being laid in Burgundy, the
following lines are given:—
\settowidth{\versewidth}{Christ his cross be his good speed, Christ his foes to quell,} 
\begin{scverse}
\vleftofline{“}King Richard’s gone to Walsingham, to the Holy Land,\\
To kill Turk and Saracen, that the truth do withstand;\\
Christ his cross be his good speed, Christ his foes to quell,\\
Send him help in time of need, and to come home well.”
\end{scverse}
In the Bodleian Library is a small quarto volume, apparently in the hand-writing
of Philip, Earl of Arundel (eldest son of the Duke of Norfolk, who suffered in
Elizabeth’s time), containing \textit{A lament for Walsingham}. It is in the ballad style,
and the two last stanzas are as follows:—
\settowidth{\versewidth}{Weep, weep.~0 Walsingham!}
\begin{dcverse}
\vleftofline{“}Weep, weep, O Walsingham!\\
Whose days are nights;\\
Blessings turn’d to blasphemies—\\
Holy deeds to despites.

Sin is where Our Lady sat,\\
Heaven turned is to hell;\\
Satan sits where Our Lord did sway:\\
Walsingham, Oh, farewell!”
\end{dcverse}
In Nashe’s \textit{Have with you to Saffron-Walden}, 1596, sign. L, “As I went to
Walsingham” is quoted, which is the first line of the ballad in the Pepysian
Collection, vol. i., p.~226, and a verse of which is here printed to the music.

One of the \textit{Psalmes and Songs of Sion, turned into the language, and set to the
tunes of a strange land}, 1642, is to the tune of \textit{Walsingham}; and Osborne, in his
\textit{Traditional Memoirs on the Reigns of Elizabeth and James}, 1653, speaking of the
Earl of Salisbury, says:—
\settowidth{\versewidth}{Many a hornpipe he tuned to his Phillis,}
\begin{scverse}
\vleftofline{“}Many a hornpipe he tuned to his Phillis,\\
And sweetly sung \textit{Walsingham} to’s Amaryllis.”
\end{scverse}
In \textit{Don Quixote}, translated by J. Phillips, 1687, p.~278, he says, “An infinite
number of little birds, with painted wings of various colours, hopping from branch
to branch, all naturally singing Walsingham, and whistling \textit{John, come kiss
me now}.”

Two of the ballads are reprinted in Percy’s \textit{Reliques of Ancient Poetry}; the
one beginning, “Gentle herdsman, tell to me;” the other, “As ye came from the
\pagebreak
%%123
%%===============================================================================
Holy Land.” The last will also be found in Deloney’s \textit{Garland of Goodwill},
reprinted by the Percy Society.

\musicinfo{Slow and plaintive.}{}

\lilypondfile[noindent, current-font-as-main, staffsize=17, 
noragged-right, language=english, nofragment]
{lilypond/123-walsingham}\normalsize

This ballad is on one of the affairs of gallantry that so frequently arose out of
pilgrimages.

\vspace{-1\baselineskip}

\musictitle{Packington’s, or Paggington's Pound.}

This tune is to be found in Queen Elizabeth’s Virginal Book; in \textit{A New Book
of Tablature}, 1596; in the \textit{Collection of English Songs}, printed at Amsterdam, in
1634; in \textit{Select Ayres}, 1659; in \textit{A Choice Collection of 180 Loyal Songs}, 1685;
in Playford’s \textit{Pleasant Musical Companion}, Part II., 1687; in \textit{The Beggars’
Opera}, 1728; in \textit{The Musical Miscellany}, vol. v.; and in many other collections.

It probably took its name from Sir John Packington, commonly called “lusty
Packington,” the same who wagered that he would swim from the Bridge at
Westminster, \ie, Whitehall Stairs, to that at Greenwich, for the sum of 3,000\textit{l}.
“But the good Queen, who had particular tenderness for handsome fellows, would
not permit Sir John to run the hazard of the trial.” His portrait is still preserved
at Westwood, the ancient seat of the family.

In Queen Elizabeth’s Virginal Book it is called Packington’s Pound; by Ben
Jonson, \textit{Paggington’s Pound}; and, in a MS. now in Dr.~Rimbault’s possession,
\textit{A Fancy of Sir John Pagington}.

Some copies, viz., that in the Virginal Book, and in the Amsterdam Collection,
have the following difference in the melody of the first four bars:

\noindent\lilypondfile[noindent, current-font-as-main, staffsize=17, 
noragged-right, language=english, nofragment]
{lilypond/123-packington-or-paggintons-pound}\normalsize

\noindent and it is probably the more correct reading, as the other closely resembles the
commencement of “Robin Hood, Robin Hood, said Little John.”

The song in Ben Jonson’s comedy of \textit{Bartholomew Fair}, commencing, “My
masters and friends, and good people, draw near,” was written to this air, and is
thus introduced:—
\settowidth{\versewidth}{\textit{Cokes.} (Sings) Fa, la la la, la la la, fa la la, la! Nay, I’ll put thee in tune and all!}
\begin{scverse}
\textit{Night}. To the tune of Paggington's Pound, Sir?\\
\textit{Cokes.} (Sings) Fa, la la la, la la la, fa la la, la! Nay, I’ll put thee in tune and all!\\
\vin\vin\vin Mine own country dance! Pray thee begin."—\textit{Act} 3.
\end{scverse}

\pagebreak
%%124
%%===============================================================================

The songs written to the tune are too many for enumeration. Besides those
in the various Collections of Ballads in the British Museum, in D’Urfey’s \textit{Pills},
and in the \textit{Pill to purge State Melancholy}, 1716,—in one Collection alone, viz.,
\textit{The Choice Collection of 180 Loyal Songs}, there are no fewer than thirteen. The
following are curious:—

No. 1. A popular Beggars’ Song, by which the tune is often named, commencing:—
\settowidth{\versewidth}{Or who is so richly cloathed as we.”—\textit{Select Ayres}, 1659.}
\begin{scverse}\vleftofline{“}From hunger and cold who liveth more free?\\
Or who is so richly cloathed as we.”—\textit{Select Ayres}, 1659.
\end{scverse}

No. 2. “Blanket Fair, or the History of Temple Street. Being a relation of
the merry pranks plaid on the river Thames during the great Frost.”
\begin{scverse}“Come, listen awhile, though the weather be cold.”
\end{scverse}

No. 3. “The North Country Mayor,” dated 1697, from a manuscript volume
of Songs by Wilmot, Earl of Rochester, and others, in the Harleian Library:—
\indentpattern{000022000}
\begin{scverse}\begin{patverse}
\vleftofline{“}I sing of no heretic Turk, or of Tartar,\\
But of a suffering Mayor who may pass for a Martyr;\\
For a story so tragick was never yet told\\
By Fox or by Stowe, those authors so old;\\
How a vile Lansprasado\\
Did a Mayor bastinado,\\
And played him a trick worse than any Strappado:\\
O Mayor, Mayor, better ne’er have transub’d, [turn’d Papist]\\
Than thus to be toss’d in a blanket and drubb’d,” \&c.
\end{patverse}
\end{scverse}

The following song, in praise of milk, is from Playford’s \textit{Musical Companion},
Part II., 1687:—

\musicinfo{Moderate time and smoothly.}{}

\lilypondfile[noindent, current-font-as-main, staffsize=16, 
noragged-right, language=english, nofragment]
{lilypond/124-in-praise-of-dairy}\normalsize
\pagebreak

%%125
%%===============================================================================
\DFNdouble

\settowidth{\versewidth}{For 'tis sack makes the man, though 'tis milk makes the nurse.}
\indentpattern{0011100}
\begin{scverse}\footnotesize
\begin{patverse}
The first of fair dairy-maids, if you’ll believe,\\
Was Adam’s own wife, our great-grandmother Eve,\\
Who oft milk’d a cow,\\
As well she knew how;\\
Though butter was not then so cheap as ’tis now,\\
She hoarded no butter nor cheese on her shelves,\\
For butter and cheese in those days made themselves.
\end{patverse}

\begin{patverse}
In that age or time there was no horrid money,\\
Yet the children of Israel had both milk and honey:\\
No queen you could see,\\
Of the highest degree,\\
But would milk the brown cow with the meanest she;\\
Their lambs gave them clothing, their cows gave them meat,\\
And in plenty and peace all their joys were compleat.
\end{patverse}

\begin{patverse}
Amongst the rare virtues that milk does produce,\\
For a thousand of dainties it’s daily in use;\\
Now a pudding I’ll tell thee,\\
Ere it goes in the belly,\\
Must have from good milk both the cream and the jelly:\\
For a dainty fine pudding, without cream or milk,\\
Is a citizen's wife without satin or silk.
\end{patverse}

\begin{patverse}
In the virtues of milk there is more to be muster’d,\\
The charming delights both of cheese-cake and custard,\\
For at Tottenham Court,\\
You can have no sport,\\
Unless you give custards and cheese-cake too for’t;\\
And what's the jack-pudding that makes us to laugh,\\
Unless he hath got a great custard to quaff?
\end{patverse}

\begin{patverse}
Both pancake and fritter of milk have good store,\\
But a Devonshire whitepot\dcfootnote{\scriptsize %a
Devonshire white-pot, or hasty-pudding, consisting of
flour and milk boiled together.
} %end footnote
must needs have much more;\\
No state you can think,\\
Though you study and wink,\\
From the lusty sack-posset\dcfootnote{\scriptsize %b
The following is a receipt for sack-posset:—
\settowidth{\versewidth}{Fetch sugar, half a pound; fetch sack, from Spain,}
\indentpattern{01113}
\begin{fnverse}\scriptsize
\begin{patverse}
\vin\vin\vleftofline{“}From fair Barbadoes, on the western main,\\
Fetch sugar, half a pound; fetch sack, from Spain,\\
A pint; then fetch, from India’s fertile coast,\\
Nutmeg, the glory of the British toast."\\
\textit{Dryden's Miscellany Poems}, vol. v., p,~138.
\end{patverse}
\end{fnverse}
} %end footnote  
to poor posset drink,\\
But milk’s the ingredient, though sack’s ne’er the worse,\\
For 'tis sack makes the man, though 'tis milk makes the nurse.
\end{patverse}
\end{scverse}

Elderton’s ballad, called “News from Northumberland,” a copy of which is in
the Library of the Society of Antiquaries, was probably written to this tune.

\musictitle{The Staines Morris Tune.}

This tune is taken from the first edition of \textit{The Dancing Master}.	It is also in
William Ballet’s Lute Book (time of Elizabeth); and was printed as late as about
1760, in a Collection of Country Dances, by Wright.

The Maypole Song, in \textit{Actæon and Diana}, seems so exactly fitted to the air,
that, haying no guide as to the one intended, I have, on conjecture, printed it
with this tune.

\pagebreak

%%126
%%===============================================================================
\DFNsingle

\musicinfo{Boldly and rather quick.}{}
\lilypondfile[noindent, current-font-as-main, staffsize=17, 
noragged-right, language=english, nofragment]
{lilypond/126-staines-morris}\normalsize

\indentpattern{000011}
\settowidth{\versewidth}{What kisses you your sweethearts gave,}
\begin{dcverse}\begin{patverse}
It is the choice time of the year,\\
For the violets now appear;\\
Now the rose receives its birth,\\
And pretty primrose decks the earth.\\
Then to the May-pole come away,\\
For it is now a holiday.
\end{patverse}

\begin{patverse}
Here each batchelor may chuse\\
One that will not faith abuse;\\
Nor repay with coy disdain\\
Love that should be loved again.\\
Then to the May-pole come away,\\
For it is now a holiday.
\end{patverse}

\begin{patverse}
And when you well reckoned have\\
What kisses you your sweethearts gave,\\
Take them all again, and more,\\
It will never make them poor.\\
Then to the May-pole come away,\\
For it is now a holiday.
\end{patverse}

\begin{patverse}
When you thus have spent the time\\
Till the day be past its prime,\\
To your beds repair at night,\\
And dream there of your day’s delight.\\
Then to the May-pole come away,\\
For it is now a holiday.
\end{patverse}
\end{dcverse}

\musictitle{The Shepherd’s Daughter.}

This is in every edition of \textit{The Dancing Master}, except the first, either under
the name of \textit{The Shepherd’s Daughter}, or \textit{Parson and Dorothy}. It is also under
the latter title in several of the ballad operas. Percy says the ballad of \textit{The
Knight and Shepherd’s Daughter}, “was popular in the time of Queen Elizabeth,
being usually printed with her picture before it, as Hearne informs us in his preface
to \textit{Gul. Neubrig. Hist. Oxon}., vol. i., 70.

Four lines are quoted in Fletcher’s comedy \textit{The Pilgrim}, act iv., sc. 2: “He
called down his merry men all,” \&c.; and in \textit{The Knight of the Burning Pestle}:
“He set her on a milk-white steed,”~\&c.


\footnotetext[1]{\scriptsize\centering In William Ballet’s Lute Book, the third note of the melody is E; in the 2nd edition of \textit{The Dancing Master}, B.}

\pagebreak
%
%%127
%%===============================================================================

Copies of the ballad will be found in the Roxburghe Collection, vol. ii., 30;
and in the Douce Collection, with the burden or chorus, “Sing, trang, dildo dee,”
at the end of each verse, which is not given by Percy. The two last bars are
here added for the burden. In some copies the four first bars are repeated.

\musicinfo{Rather slow.}{}
\lilypondfile[noindent, current-font-as-main, staffsize=17, 
noragged-right, language=english, nofragment]
{lilypond/127-the-shepherds-daughter}\normalsize

The ballad will be found in Percy’s \textit{Reliques of Ancient Poetry}, series 3, book i.

\musictitle{The Frog Galliard, or Now, O Now!}

This is the only tune, composed by a well-known musician of the age, that
I have found employed as a ballad tune.

In Dowland’s \textit{First Book of Songes}, 1597, it is adapted to the words, “Now,
O now, I needs must part” (to be sung by one voice with the lute, or by four
without accompaniment); but in his Lute Manuscripts it is called \textit{The Frog
Galliard}, and seems to have been commonly known by that name.

In Morley’s \textit{Consort Lessons}, 1599 and 1611, it is called \textit{The Frog Galliard};
in Thomas Robinson’s \textit{New Citharen Lessons}, 1609, The \textit{Frog}; and in the Skene
Manu\-script, \textit{Froggis Galziard}.

In \textit{Nederlandtsche Gedenck-Clanck}, printed at Haerlem in 1626, it is called
\textit{Nou, nou} [for Now, O now]; but all the ballads I have seen, that were written
to it, give the name as \textit{The Frog Galliard}.

In Anthony Munday’s \textit{Banquet of daintie Conceits}, 1588, there is a song to the
tune of \textit{Dowland’s Galliard}, but it could not be sung to this air.

It seems probable that \textit{Now, O now}, was originally a dance tune, and the
composer finding that others wrote songs to his galliards, afterwards so adapted
it likewise.

The latest Dutch copy that I have observed is in Dr.~Camphuysen’s \textit{Stichtelycke
Rymen}, printed at Amsterdam in 1647.

Dowland is celebrated in the following sonnet, which, from having appeared in
\textit{The Passionate Pilgri}m, has been attributed to Shakespeare, but was published
previously in a Collection of Poems by Richard Barnfield.

\pagebreak

%%128
%%===============================================================================

\settowidth{\versewidth}{One knight loves both, and both in thee remain!}
\begin{scverse}\vleftofline{“\textit{To his}}\textit{ friend,, Master R. L., in praise of Music and Poetry}!”\\
\indentpattern{01010101010100}
\begin{patverse}
\vleftofline{“}If music and sweet poetry agree,\\
As they must needs, (the sister and the brother,)\\
Then must the love be great ’twixt thee and me,\\
Because thou lov’st the one, and I the other.\\
\textit{Dowland} to thee is dear, whose heavenly touch\\
Upon the lute doth ravish human sense;\\
\textit{Spenser} to me, whose deep conceit is such,\\
As, passing all conceit, needs no defence;\\
Thou lov’st to hear the sweet melodious sound\\
That Phœbus’ lute, the queen of music, makes,\\
And I, in deep delight am chiefly drown’d,\\
When as himself to singing he betakes;\\
One God is good to both, as poets feign,\\
One knight loves both, and both in thee remain!”
\end{patverse}
\end{scverse}

Anthony Wood says of Dowland, that “he was the rarest musician that the
age did behold.” In \textit{No Wit, no Help, like a Woman’s}, a comedy by Thomas
Middleton (1657), the servant tells his master bad news; and is thus answered:
“Thou plaiest Dowland’s \textit{Lachrimæ} to thy master.”

In Peacham’s \textit{Garden of Heroical Devices}, are the following verses, portraying
Dowland’s forlorn condition in the latter part of his life:—
\settowidth{\versewidth}{So since (old friend) thy years have made thee white,}
\indentpattern{010100}
\begin{scverse}\begin{patverse}
\vleftofline{“}Here Philomel in silence sits alone\\
In depth of winter, on the bared briar,\\
Whereon the rose had once her beauty shown,\\
Which lords and ladies did so much desire!\\
But fruitless now, in winter’s frost and snow,\\
It doth despis’d and unregarded grow.
\end{patverse}

\begin{patverse}
So since (old friend) thy years have made thee white,\\
And thou for others hast consum’d thy spring,\\
How few regard thee, whom thou didst delight,\\
And far and near came once to hear thee sing!\\
Ungrateful times, and worthless age of ours,\\
That lets us pine when it hath cropt our flowers.”
\end{patverse}
\end{scverse}

The device which precedes these stanzas, is a nightingale sitting on a bare
brier, in the midst of a wintry storm.

The following ballads were sung to the tune under the title of \textit{The Frog
Galliard}:—“The true love’s-knot untyed: being the right path to advise princely
virgins how to behave themselves, by the example of the renouned Princess, the
Lady Arabella, and the second son to the Lord Seymore, late Earl of Hertford;”
commencing—
\begin{scverse}\begin{altverse}
“As I to Ireland did pass,\\
I saw a ship at anchor lay,\\
Another ship likewise there was,\\
Which from fair England took her way.\\
This ship that sail’d from fair England,\\
Unknown unto our gracious King,\\
The Lord Chief Justice did command,\\
That they to London should her bring,” \&c.
\end{altverse}
\end{scverse}
\pagebreak

%%129
%%===============================================================================

\noindent A copy in the British Museum Collection, and printed by Evans in \textit{Old Ballads},
1810, vol. iii., 184.

Also, “The Shepherd’s Delight,” commencing—
\settowidth{\versewidth}{And by that flower there stands a bower,}
\begin{scverse}
\begin{altverse}
“On yonder hill there stands a flower,\\
Fair befall those dainty sweets;\\
And by that flower there stands a bower,\\
Where all the heavenly muses meet,” \&c.
\end{altverse}
\end{scverse}

A copy in the Roxburghe Collection, vol, i., 388, and Evans, vol. i., 388.

\musicinfo{Slowly and smoothly.}{}
\lilypondfile[noindent, current-font-as-main, staffsize=17, 
noragged-right, language=english, nofragment]
{lilypond/129-the-frog-galliard-or-now-o-now}\normalsize

\settowidth{\versewidth}{Dear, when I from thee am gone,}
\begin{dcverse}
\begin{altverse}
Dear, when I from thee am gone,\\
Gone are all my joys at once!\\
I loved thee, and thee alone,\\
In whose love I joyed once.\\
While I live I needs must love,\\
Love lives not when life is gone:
\end{altverse}

\begin{altverse}
Now, at last, despair doth prove\\
Love divided loveth none.\\
And although your sight I leave,\\
Sight wherein my joys do lie,\\
Till that death do sense bereave,\\
Never shall affection die.
\end{altverse}
\end{dcverse}
\pagebreak



%%130
%%===============================================================================
\DFNdouble

\musictitle{Paul’s Wharf}

This tune is in Queen Elizabeth’s Virginal Book, and in \textit{The Dancing Master},
from 1650 to 1665.

Paul’s Wharf was, and still is, one of the public places for taking water, near
to St. Paul’s Cathedral. In “The Prices of Fares and Passages to be paide to
Watermen,” printed by John Cawood, (n.d.,) is the following: “Item, that no
Whyry manne, with a pare of ores, take for his fare from Pawles Wharfe, Queen
hithe, Parishe Garden, or the blacke Fryers to Westminster, or White hall, or
lyke distance to and fro, above iij\textit{d}.

\musicinfo{Gracefully.}{}

\lilypondfile[noindent, current-font-as-main, staffsize=17, 
noragged-right, language=english, nofragment]
{lilypond/130-pauls-wharf}\normalsize

\musictitle{Trip and Go}
This was one of the favorite Morris-dances of the sixteenth and seventeenth
centuries, and frequently alluded to by writers of those times.

Nashe, in his Introductory Epistle to the surreptitious edition of Sidney’s
\textit{Astrophel and Stella}, 4to., 1591, says, “Indeede, to say the truth, my stile is
somewhat heavie gated, and cannot daunce \textit{Trip and goe} so lively, with ‘Oh my
love, ah my love, all my love gone,’ as other shepheards that have beene \textit{Fooles in
the morris}, time out of minde.” He introduces it more at length, and with a
description of the Morris-dance, in the play of \textit{Summer’s last Will and Testament},
1600:

\settowidth{\versewidth}{“\textsc{Ver}.-- \textit{goes in and fetcheth out the Hobby-horse and the Morris-dance, who}.}
\begin{scverse}
“\textsc{Ver}.-- \textit{goes in and fetcheth out the Hobby-horse and the Morris-dance, who
dance about}.
\end{scverse}

\textit{Ver}.—“About, about! lively, put your horse to it; rein him harder; jerk him with
your wand. Sit fast, sit fast, man! Fool, hold up your ladle\dcfootnote{ %a
The ladle is still used by the sweeps on May-day.
} %end footnote
there.”

\textit{Will Summer}.—“O brave Hall!\dcfootnote{ %b
The tract of “Old Meg of Herefordshire for a Mayd
Marian, and Hereford towne for a Morris-dance,” 4to,
1609, is dedicated to \textit{old Hall}, a celebrated Taborer of
Herefordshire; and the author says,—“The People of
Herefordshire are beholding to thee; thou givest the men
light hearts by thy pipe, and the women light heeles by
thy tabor, O wonderful piper! O admirable tabor-man!”
\dots “The wood of this olde Hall’s tabor should
have beene made a paile to carie water in at the beginning
of King Edward the Sixt’s reigne; but Hall (being wise,
because hee was even then reasonably well strucken in
years) saved it from going to the water, and converted it
in these days to a tabor.” For more about old Hall and
his pipe and tabor, see page~134.
} %end footnote
O well said, butcher! Now for the credit of
Worcestershire. The finest set of Morris-dancers that is between this and Streatham. \pagebreak
Marry, methinks there is one of them danceth like a clothier’s horse, with a wool-pack
%%131
%%===============================================================================
upon his back. You, friend, with the hobby-horse, go not too fast, for fear of wearing
out my lord’s tile-stones with your hob-nails.”

\textit{Ver}.—“So, so, so; trot the ring twice over, and away.”

After this, three clowns and three maids enter, dancing, and singing the song
which is here printed with the music.

\textit{Trip and go} seems to have become a proverbial expression. In Gosson’s \textit{Schoole
of Abuse}, 1579: “\textit{Trip and go}, for I dare not tarry.” In \textit{The two angrie Women
of Abington}, 1599: “Nay, then, \textit{trip and go}.” In Ben Jonson’s \textit{Case is altered}:
“O delicate \textit{trip and go}.” And in Shakespeare’s \textit{Love's Labour Lost}: “\textit{Trip
and go}, my sweet.”

The tune is taken from \textit{Musick's Delight on the Cithren}, 1666. It resembles
another tune, called \textit{The Boatman}. (See Index.)

\musicinfo{Moderate time and trippingly.}{}
\lilypondfile[noindent, current-font-as-main, staffsize=17, 
noragged-right, language=english, nofragment]
{lilypond/131-trip-and-go}\normalsize

The Morris-dance was sometimes performed by itself, but was much more
frequently joined to processions and pageants, especially to those appointed for
the celebration of May-day, and the games of Robin Hood. The festival, instituted
in honour of Robin Hood, was usually solemnized on the first and
succeeding days of May, and owes its original establishment to the cultivation
and improvement of the manly exercise of archery, which was not, in former
times, practised merely for the sake of amusement.

“I find,” says Stow, “that in the month of May, the citizens of London, of all
estates, lightly in every parish, or sometimes two or three parishes joining
together, had their several \textit{Mayings}, and did fetch in May-poles, with divers
\textit{warlike shews}, with good \textit{archers, Morris-dancers}, and other devices for pastime all
\pagebreak
%%132
%%===============================================================================
the day long: and towards the evening they had stage-plays and bonfires in the
streets\ldots\  These great Mayings and May-games, made by the governors and
masters of this city, with the triumphant setting up of the great shaft (a principal
Maypole in Cornhill, before the parish church of St. Andrew, which, from the pole
being higher than the steeple itself, was, and still is, called St. Andrew Undershaft), 
by means of an insurrection of youths against aliens on May-day, 1517,\dcfootnote{ %
The “story of Ill May-day, in the time of Henry the
Eight, and why it is so called; and how Queen Catherine
begged the lives of two thousand London apprentices,” is
the subject of an old ballad in Johnson’s \textit{Crown Garland
of Golden Roses}, and has been reprinted in Evans’ \textit{Old
Ballads}, vol. iii. p.~76, edition of 1810.
} %end footnote
the
ninth of Henry the Eighth, have not been so freely used as afore.”—\textit{Survey of
London}, 1598, p.~72.

\renewcommand\versoheader{morris dance and may-day.} 

The celebration of May-day may be traced as far back as Chaucer, “who, in
the conclusion of his \textit{Court of Love}, has described the Feast of May, when—”

\settowidth{\versewidth}{And namely hawthorn brought, both page and groom;}
\begin{scverse}
\vleftofline{“}Forth go’th all the court, both most and least.\\
To fetch the floures fresh, and braunch and bloom—\\
And namely hawthorn brought, both page and groom;\\
And they rejoicen in their great delight;\\
Eke each at other throw the floures bright,\\
The primerose, the violete, and the gold,\\
With freshe garlants party blue and white.’’
\end{scverse}

Henry the Eighth appears to have been particularly attached to the exercise of
archery, and the observance of May. “Some short time after his coronation,”
says Hall, “he came to Westminster, with the queen, and all their train: and on
a time being there, his grace, the Earls of Essex, Wiltshire, and other noblemen,
to the number of twelve, came suddenly in a morning into the queen’s chamber,
all appareled in short coats of Kentish Kendal, with hoods on their heads, and
hosen of the same, every one of them his bow and arrows, and a sword and
buckler, like outlaws or Robin Hood’s men; whereof the queen, the ladies, and
all other there, were abashed, as well for the strange sight, as also for their
sudden coming: and, after certain dances and pastime made, they departed.”—
\textit{Hen. VIII}., fo. 6, b. The same author gives a curious account of Henry and
Queen Catherine going a Maying.

Bourne, in his \textit{Antiquitates Vulgares}, says, “On the Calends, or first-day of
May, commonly called May-day, the juvenile part of both sexes were wont to rise
a little before midnight and walk to some neighbouring wood, accompanied with
music, and the blowing of horns, where they brake down branches from the trees,
and adorn them with nosegays and crowns of flowers. When this is done, they
return with their booty homewards, about the rising of the sun, and make their
doors and windows to triumph in the flowery spoil. The after part of the day is
chiefly spent in dancing round a tall pole, they call a May-pole; which being
placed in a convenient part of the village, stands there, as it were consecrated
to the goddess of flowers, without the least violence offered it in the whole circle
of the year.” Borlase, in his \textit{Natural History of Cornwall}, tells us, “An ancient
custom, still retained by the Cornish, is that of decking their doors and porches,
on the first of May, with green sycamore and hawthorn boughs, and of planting
trees, or rather stumps of trees, before their houses: and on May-eve, they from
\pagebreak
%%133
%%===============================================================================
towns make excursions into the country, and having cut down a tall elm, brought
it into town, fitted a straight and taper pole to the end of it, and painted the
same, erect it in the most public places, and on holidays and festivals adorn it
with flower garlands, or insigns and streamers.”

Philip Stubbes, the puritan, who declaims as vehemently against May-games as
against dancing, minstrelsy, and other sports and amusements, thus describes
“the order of their May-games” in this reign. “Against May, Whitsuntide, or
some other time of the year, every parish, town, and village, assemble themselves
together, both men, women, and children; and either all together, or dividing
themselves into companies, they go, some to the woods and groves, some to the
hills and mountains, some to one place, some to another, and in the morning they
return, bringing with them birch, boughs, and branches of trees, to deck their
assemblies withal\ldots\   But their chiefest jewel they bring from thence is their
May-pole, which they bring home with great veneration, as thus: they have
twenty or forty yoke of oxen, every ox having a sweet nosegay of flowers tied to
the tip of his horns; and these oxen draw home this May-pole, (this stinking
idol rather), which is covered all over with flowers and herbs, bound round
about with strings, from the top to the bottom, and sometime painted with
variable colours, with two or three hundred men, women, and children, following
it with great devotion. And thus, being reared up, with handkerchiefs
and flags streaming on the top, they strew the ground about, bind green boughs
about it, set up summer halls, bowers, and arbours, hard by it; and then fall
they to banquet and feast, to leap and dance about it, as the heathen people
did at the dedication of their idols, whereof this is a perfect pattern, or rather
the thing itself.”—(\textit{Anatomie of Abuses}, reprint of 1585 edit., p.~171.)

Browne, also, has given a similar description of the May-day rites, in his
\textit{Britannia's Pastorals}, book ii., song 4:—
\settowidth{\versewidth}{When envious night commands them to be gone,}
\begin{scverse}
“As I have seen the Lady of the May\\
Sit in an arbour,\ldots\ \\
Built by a May-pole, where the jocund swains\\
Dance with the maidens to the bagpipe’s strains,\\
When envious night commands them to be gone,\\
Call for the merry youngsters one by one,\\
And, for their well performance, ‘she’ disposes\\
To this a garland interwove with roses;\\
To that a carved hook, or well-wrought scrip;\\
Gracing another with her cherry lip:\\
To one her garter; to another, then,\\
A handkerchief, cast o’er and o’er again;\\
And none returneth empty, that hath spent\\
His pains to fill their rural merriment.”
\end{scverse}

The Morris-dance, when performed on May-day, and \textit{not} connected with the
Games of Robin Hood, usually consisted of the Lady of the May, the fool or jester,
a piper, and two, four, or more, morris-dancers. But, on other occasions, the hobby-horse, 
and sometimes a dragon, with Robin Hood, Maid Marian, Friar Tuck, Little \pagebreak
John, and other characters supposed to have been the companions of that famous
%%134
%%===============================================================================
outlaw, were added to the dance. Maid Marian was sometimes represented by a
smooth-faced youth, dressed in a female garb; Friar Tuck, Robin Hood’s chaplain,
by a man of portly form, in the habit of a Franciscan friar; the hobby-horse was a
paste-board resemblance of the head and tail of a horse, on a wicker frame, and
attached to the body of a man, whose feet being concealed by a foot-cloth hanging
to the ground, he was to imitate the ambling, the prancing, and the curveting of
the horse; the dragon (constructed of the same materials) was made to hiss, yell,
and shake his wings, and was frequently attacked by the man on the hobby-horse,
who then personated St. George.

The garments of the Morris-dancers were adorned with bells, which, were not
placed there merely for the sake of ornament, but were sounded as they danced.
These, which were worn round the elbows and knees, were of unequal sizes,
and differently denominated; as the fore bell, the second bell, the treble, the mean
or countertenor, the tenor, the great bell or base, and sometimes double bells were
worn.\dcfootnote{ %a
For the bells of the Morris, see Ford’s play,\textit{ The Witch
of Edmonton}, act 2, sc. 1. Weber is mistaken as to
“mean” meaning tenor.
} %end footnote 
The principal dancer in the Morris was more superbly habited than his
companions; as appears from a passage in \textit{The blind Beggar of Bethnall Green}
(dramatised from the ballad of the same name), by John Day, 1659: “He wants
no clothes, for he hath a cloak laid on with gold lace, and an embroidered jerkin;
and thus he is marching hither \textit{like the foreman of a morris}.”

In \textit{The Vow-breaker}, or \textit{Fair Maid of Clifton}, by William Sampson, 1636,
we find, “Have I not practised my reins, my careers, my prankers, my ambles,
my false trots, my smooth ambles, and Canterbury paces—and shall the mayor
put me, besides the hobby-horse? I have borrowed the fore-horse bells, his
plumes, and braveries; nay, I have had the mane new shorn and frizzled. Am
I not going to buy ribbons and toys of sweet Ursula for the Marian—and shall
I not play the hobby-horse? Provide thou the dragon, and let me alone for the
hobby-horse.” And afterwards: “Alas, sir! I come only to borrow a few
ribbands, bracelets, ear-rings, wire-tiers, and silk girdles, and handkerchers, for a
Morris and a show before the queen; I come to furnish the hobby-horse.”

There is a curious account of twelve persons of the average age of a hundred
years, dancing the Morris, in an old book, called “Old Meg of Herefordshire for
a Mayd Marian, and Hereford towne for a Morris-dance; or twelve Morris-dancers
in Herefordshire of 1200 years old,”\dcfootnote{ %b
Brand, in his Popular Antiquities, vol.2, p.208, 1813,
gives an account of a May-game, or Morris-dance, by
\textit{eight} persons in Herefordshire, whose ages, computed
together, amounted to 800 years; probably the same as
mentioned by Lord Bacon, as happening “a few years
since in the county of Hereford.” See \textit{History, Natural
and Experimental, of Life and Death}, 1638.
} %end footnote
quarto, 1609. It is dedicated to the renowned
old Hall, taborer of Herefordshire, and to “his most invincible weatherbeaten
nut-brown tabor, which hath made bachelors and lasses dance round
about the May-pole, three-score summers, one after another in order, and is not
yet worm-eaten.” Hall, who had then “stood, like an oak, in all storms, for
ninety-seven winters,” is recommended to “imitate that Bohemian Zisca, who at
his death gave his soldiers a strict command to flay his skin off, and cover a drum
with it, that alive and dead he might sound like a terror in the ears of his enemies: \pagebreak
so thou, sweet Hereford Hall, bequeath in thy last will, thy vellum-spotted skin
%%135
%%===============================================================================
to cover tabors; at the sound of which to set all the shires a dancing\dots The
court of kings is for stately \textit{measures}; the city for light heels and nimble footing;
western men for, gambols; Middlesex men for tricks above ground; Essex men
for the \textit{Hey}; Lancashire for \textit{Hornpipes}; Worcestershire for bagpipes; but Herefordshire
for a Morris-dance, puts down not only all Kent, but very near (if one
had line enough to measure it) three quarters of Christendom. Never had Saint
Sepulchre’s a truer ring of bells; never did any silk-weaver keep braver time;
never could Beverley Fair give money to a more sound taborer; nor ever had
Robin Hood a more deft Maid Marian.”

Full particulars of the Morris-dance and May-games may be found by referring
to Strutt’s \textit{Sports and Pastimes}; to Ritson’s \textit{Robin Hood}; to an account of a
painted window, appended to part of Henry IV., in Steevens’ \textit{Shakespeare}, the
xv. vol. edition; to Gifford’s \textit{Ben Jonson}, vol. i., pages 50, 51, 52, vol. iv., p.~405,
and vol.~vii., p.~397; to \textit{The British Bibliographer}, vol. iv., p.~326; Brand’s
\textit{Popular Antiquities}; Douce’s \textit{Illustrations of Shakespeare}; and Dr.~Drake’s
\textit{Shakespeare and his Times}, vol. i., \&c., \&c.

\musictitle{Barley Break}

From Lady Neville's Virginal Book, which was transcribed in 1591.

\musicinfo{Stately.}{}

\lilypondfile[noindent, current-font-as-main, staffsize=17, 
noragged-right, language=english, nofragment]
{lilypond/135-barley-break}\normalsize
\pagebreak

%%136
%%===============================================================================

\renewcommand\versoheader{english song and ballad music.}

Gifford has given the following description of the sport called Barley-break, in
a note upon Massinger’s \textit{Virgin Martyr}, act v., sc. 1:—“Barley-break was
played by six people\dcfootnote{ %a
Rather, perhaps, by \textit{not less} than six people.
“Heyday! there are a \textit{legion} of young cupids at Barlibreak."
—\textit{The Guardian}, act i., sc. 1.
} %end footnote 
(three of each sex), who were coupled by lot. A piece of
ground was then chosen and divided into three compartments, of which the middle
one was called Hell. It was the object of the couple condemned to this division,
to catch the others, who advanced from the two extremities; in which case a
change of situation took place, and hell was filled by the couple who were excluded
by pre-occupation, from the other places: in this ‘catching,’ however, there was
some difficulty, as, by the regulations of the game, the middle couple were not
to separate before they had succeeded, while the others might break hands whenever
they found themselves hard pressed. When all had been taken in turn, the
last couple was said \textit{to be in hell}, and the game ended.” In this description,
Gifford does not in any way allude to it as a dance, but Littleton explains \textit{Chorus
circularis}, barley-break, when they dance, taking their hands round. See Payne
Collier’s note on Dodsley’s \textit{Old Plays}, vol. iii., p.~316. Strutt, in his \textit{Sports and
Pastimes}, quotes only two lines from Sidney, which he takes from Johnson’s
Dictionary:—
\settowidth{\versewidth}{“By neighbours prais’d, she went abroad thereby,}
\begin{scverse}
“By neighbours prais’d, she went abroad thereby,\\
At barley-brake her sweet swift feet to try.”
\end{scverse}
In the Roxburghe Collection, vol. i., 344, is a ballad called “The Praise of our
Country Barley-brake, or—
\settowidth{\versewidth}{Up this loving old sport, called Barley-brake."}
\begin{scverse}
Cupid’s advisement for young men to take\\
Up this loving old sport, called Barley-brake."
\end{scverse}
“To the tune of \textit{When this old cap was new}.” It commences thus:—
\settowidth{\versewidth}{Both young men, maids, and lads,}
\begin{scverse}
\begin{altverse}
\vleftofline{“}Both young men, maids, and lads,\\
Of what state or degree,\\
Whether south, east, or west,\\
Or of the north country;\\
I wish you all good health,\\
That in this summer weather\\
Your sweet-hearts and yourselves\\
Play at barley-break together.” \&c.
\end{altverse}
\end{scverse}

Allusions to \textit{Barley-break} occur repeatedly in our old writers. Mr. M. Mason
quotes a description of the pastime with allegorical personages, from Sir John
Suckling:—
%\vspace{-1\baselineskip}
\settowidth{\versewidth}{And Hate consorts with Pride; \textit{so dance they},” \&c.}
\begin{scverse}
\vleftofline{“}Love, Reason, Hate, did once bespeak\\
Three mates to play at Barley-break;\\
Love Folly took, and Reason Fancy;\\
And Hate consorts with Pride; \textit{so dance they},” \&c.
\end{scverse}

\musictitle{Watkin’s Ale}


The tune from Queen Elizabeth’s Virginal Book, where it is arranged by Byrd.
Ward, in his \textit{Lives of the Gresham Professors}, states that it is also contained in
one of the MSS. formerly belonging to Dr.~John Bull. A copy of the original
ballad is in the collection of Mr. George Daniel, of Canonbury. \textit{Watkin’s Ale} is
referred to in a letter prefixed to Anthony Munday’s translation of \textit{Gerileon in}
\pagebreak
%%137
%%===============================================================================
\textit{England}, part ii., 1592, and in Henry Chettle’s pamphlet, \textit{Kind-harts Dreame},
printed in the same year. The ballad is entitled:

\settowidth{\versewidth}{A warning well weighed, though counted a tale.”}
\begin{scverse}“A ditty delightful of Mother Watkin’s ale\\
A warning well weighed, though counted a tale.”
\end{scverse}

\musicinfo{Moderate time.}{}
\lilypondfile[noindent, current-font-as-main, staffsize=16, 
noragged-right, language=english, nofragment]
{lilypond/137-watkins-ale}\normalsize

Each part of the tune is to be repeated for words. The following stanzas
is the seventh:--
\settowidth{\versewidth}{He made thereof a country dance.}
\begin{dcverse}\indentpattern{00002224}
\begin{patverse}
Thrice scarcely changed hath the moon\\
Since first this pretty trick was done;\\
Which being heard of one by chance,\\
He made thereof a country dance.\\
And as I heard the tale,\\
He called it Watkin’s Ale,\\
Which never will be stale\\
I do believe;
\end{patverse}

\indentpattern{22240000}
\begin{patverse}
\vin\vin This dance is now in prime,\\
And chiefly us’d this time,\\
And lately put in rhime:\\
Let no man grieve,\\
To hear this merry jesting tale,\\
The which is called Watkin’s Ale:\\
It is not long since it was made,\\
The finest flower will soonest fade.
\end{patverse}
\end{dcverse}

\musictitle{The Carman’s Whistle}

This tune is in Queen Elizabeth’s and Lady Neville’s Virginal Books (arranged
by Byrd), as well as in several others of later date. The ballad is mentioned in a
letter, bearing the signature of T. N., addressed to his good friend A[nthony]
M[unday], prefixed to the latter’s translation of \textit{Gerileon of England}, part ii.,
quarto, 1592; and by Henry Chettle in his \textit{Kind-harts Dreame}, printed in the
same year.

\pagebreak

%%138
%%===============================================================================

The Carmen of the sixteenth and seventeenth centuries appear to have been
singularly famous for their musical abilities; but especially for whistling their
tunes. Falstaff’s description of Justice Shallow is, that “he came ever in the
rear-ward of the fashion,” and “sang the tunes he heard the carmen whistle,
and sware they were his Fancies, or his Good-nights.”\dcfootnote{ %a
Good-nights are “Last dying speeches” made into
ballads. See Essex’s last Good-night.
} %end footnote 
—(\textit{Henry IV}., Part ii.,
act 3.) In Ben Jonson’s \textit{Bartholomew Fair}, Waspe says, “I dare not let him
walk alone, for fear of learning vile tunes, which he will sing at supper, and in
the sermon times! If he meet but a carman in the street, and I find him not
talk to keep him off on him, he will whistle him all his tunes over at night, in his
sleep.”—(Act i., sc. 1.) In the tract called “The World runnes on Wheeles,”\dcfootnote{ %b
Taylor's tract was written against coaches, which injured
his trade as a waterman. He says, “In the year
1564, one William Boonen, a Dutchman, brought first the
use of coaches hither, and the said Boonen was Queen
Elizabeth’s coachman, for indeed a coach was a strange
monster in those days, and the sight of them put both
horse and man into amazement. Some said it was a great
crab-shell, brought out of China, and some imagined it
to be one of the Pagan temples, in which the cannibals
adored the devil.” He argues that the cart-horse is a
more learned beast than a coach-horse, “for scarce any
coach-horse in the world doth know any letter in the book;
when as every cart-horse doth know the letter \textit{G} most
understandingly.”
} %end footnote 
by Taylor, the Water-poet, he says, “If the carman’s horse be melancholy or
dull with hard and heavy labour, then will he, like a kind piper, whistle him a
fit of mirth to any tune, from above. Eela to below Gammoth;\dcfootnote{ %c
Gamut, then the lowest note of the scale, as Eela was
the highest.
} %end footnote 
of which generosity
and courtesy your \textit{coachman} is altogether ignorant, for he never whistles,
but all his music is to rap out an oath.” And again he says, “The word \textit{carmen},
as I find it in the [Latin] dictionary, doth signify a verse, or a song; and betwixt
car\textit{men} and car\textit{man}, there is some good correspondence, for versing, singing, and
whistling, are all three musical.” Burton, in his \textit{Anatomy of Melancholy}, says,
“A carman’s whistle, or a boy singing some ballad early in the street, many
times alters, revives, recreates a restless patient that cannot sleep;” and again,
“As carmen, boys, and prentices, when a new song is published with us, go singing
that new tune still in the streets.” Henry Chettle, in his \textit{Kind-hart’s
Dreame}, says, “It would be thought the carman, that was wont to whistle to his
beasts a comfortable note, might as well continue his old course, whereby his
sound served for a musical harmony in God’s ear, as now to follow profane
jigging vanity.” In\textit{ The Pleasant Historie of the two angrie Women of Abington},
quarto, 1599, Mall Barnes asks, “But are ye cunning in the carman’s lash, and
can ye whistle well?” In \textit{The Hog hath lost its Pearl}, Haddit, the poet, tells the
player shortly to expect “a notable piece of matter; such a jig, whose tune, with
the natural whistle of a carman, shall be more ravishing to the ears of shopkeepers
than a whole concert of barbers at midnight.”--(\textit{Dodsley’s Old Plays},
vol. vi.) So in Lyly’s \textit{Midas}, “A carter with his whistle and his whip, in \textit{true}
ears, moves as much as Phœbus with his fiery chariot and winged horses.” In
Heywood’s \textit{A Woman kill’d with Kindness}, although all others are sad, the stage
direction is, “Exeunt, except Wendall and Jenkin; \textit{the carters whistling}.” And
Playford, in his \textit{Introduction to the skill of Music}, 1679, says, “Nay, the poor
labouring beasts at plough and cart are cheered by the sound of music, though it
be but their master’s whistle.”

\pagebreak

%%139
%%===============================================================================

The following ballads were sung to the tune:—“The Comber’s Whistle, or The
Sport of the Spring,” commencing—
\settowidth{\versewidth}{“All in a pleasant morning;”}
\begin{scverse}
“All in a pleasant morning;”
\end{scverse}
a copy in Pepys’ Collection, vol. iii., 291, and Roxburghe Collection, vol. ii., 67.
“All is ours and our husbands’, or the Country Hostesses’ Vindication;” a copy
in the Roxburghe Collection, vol. ii., 8.

“The Courteous Carman and the Amorous Maid: or the Carman’s Whistle,”\dcfootnote{ %a
There are twelve stanzas in the ballad, of which five
are here omitted. A black-letter copy in the Douce
Collection, fol. 33, and one in Mr. Payne Collier’s Collection.
} %end footnote 
\&c., “To the tune of \textit{The Carman's Whistle}; or \textit{Lord Willoughby's March}.”

\musicinfo{Gracefully.}{}
\lilypondfile[noindent, current-font-as-main, staffsize=17, 
noragged-right, language=english, nofragment]
{lilypond/139-the-carmans-whistle}\normalsize

\settowidth{\versewidth}{So comely was her countenance,}
\begin{dcverse}\begin{altverse}
So comely was her countenance,\\
And ‘winning was her air,’\\
As though the goddess Venus\\
Herself she had been there;\\
And many a smirking smile she gave\\
Amongst the leaves so green,\\
Although she was perceived,\\
She thought she was not seen.
\end{altverse}

\begin{altverse}
At length she chang’d her countenance,\\
And sung a mournful song,\\
Lamenting her misfortune\\
She staid a maid so long;\\
Sure young men are hard-hearted,\\
And know not what they do,\\
Or else they want for compliments\\
Fair maidens for to woo.
\end{altverse}
\end{dcverse}

\pagebreak
%%140
%%===============================================================================
\settowidth{\versewidth}{Why should young virgins pine away}
\begin{dcverse}\begin{altverse}
Why should young virgins pine away\\
And lose their chiefest prime;\\
And all for want of sweet-hearts,\\
To cheer us up in time?\\
The young man heard her ditty,\\
And could no longer stay,\\
But straight unto the damosel\\
With speed he did away.
\end{altverse}

\begin{altverse}
When he had played unto her\\
One merry note or two,\\
Then was she so rejoiced,\\
She knew not what to do:\\
O God-a-mercy, carman,\\
Thou art a lively lad;\\
Thou hast as rare a whistle\\
As ever carman had.
\end{altverse}

\begin{altverse}
Now, if my mother chide me\\
For staying here so long;\\
What if she doth, I care not.\\
For this shall be my song:\\
‘Pray, mother, be contented,\\
Break not my heart in twain;\\
Although I have been ill a-while,\\
I now am well again.’
\end{altverse}

\begin{altverse}
Now fare thee well, brave carman,\\
I wish thee well to fare,\\
For thou didst use me kindly,\\
As I can well declare:\\
Let other maids say what they will,\\
The truth of all is so,\\
The bonny Carman’s whistle\\
Shall for my money go.
\end{altverse}
\end{dcverse}

The following is the old arrangement of the tune of \textit{The Carman’s Whistle},
by Byrd, taken from Queen Elizabeth’s Virginal Book.

\lilypondfile[noindent, current-font-as-main, staffsize=17, 
noragged-right, language=english, nofragment]
{lilypond/140-the-carmans-whistle-byrd}\normalsize

\musictitle{Go From My Window.}

This tune is arranged both by Morley and by John Munday, in Queen Elizabeth’s 
Virginal Book; it is in \textit{A new Book of Tablature}, 1596; in Morley’s \textit{First
Booke of Consort Lessons}, 1599 and 1611; and in Robinson’s \textit{Schoole of Musick},
1603. In \textit{The Dancing Master}, from 1650 to 1686, it appears under the title of
“The new Exchange, or Durham Stable;” but the tune is there altered into
\timesig{6}{4} time, to fit it for dancing.

On the 4th March, 1587-8, John Wolfe had a license to print a ballad called
“Goe from \textit{the} windowe.” Nash, in his controversial tracts with Harvey, 1599,
mentions a song, “Go from my \textit{garden}, go.” In Beaumont and Fletcher’s
\textit{Knight of the Burning Pestle}, Old Merrythought sings—
\pagebreak



%%141
%%===============================================================================
\musicinfo{Slowly and smoothly.}{}
\lilypondfile[noindent, current-font-as-main, staffsize=17, 
noragged-right, language=english, nofragment]
{lilypond/141-go-from-my-window}\normalsize

\settowidth{\versewidth}{Come up to my window, love, come, come, come,}
\begin{scverse}
Begone, begone, my juggy, my puggy,\\
Begone, my love, my dear;\\
The weather is warm,\\
’Twill do thee no harm:\\
Thou canst not be lodged here.”
\end{scverse}
In Fletcher’s \textit{Monsieur Thomas}, we find—
%\settowidth{\versewidth}{Come up to my window, love, come, come, come,}
\begin{scverse}
\indentpattern{00110}
\begin{patverse}
\vleftofline{“}Come up to my window, love, come, come, come,\\
Come to my window, my dear;\\
The wind nor the rain\\
Shall trouble thee again:\\
But thou shalt be lodged here.”
\end{patverse}
\end{scverse}
It is again quoted by Fletcher in \textit{The Woman’s Prize, or the Tamer tamed}, act i.,
sc. 3; by Middleton in \textit{Blurt, Master Constable}; and by Otway in \textit{The Soldier’s
Fortune}.

It is one of the ballads that were parodied in “Ane compendious booke of
Godly and Spiritual! Songs... with sundrie of other ballates, chainged out of
prophaine Songes, for avoiding of Sinne and Harlotrie;” printed in Edinburgh
in 1590 and 1621. There are twenty-two stanzas in the Godly Song, the following
are the two first:—
%\settowidth{\versewidth}{Quho [who] is at my windo, who, who?}
\begin{scverse}
\vleftofline{“}Quho [who] is at my windo, who, who?\\
Goe from my windo; goe, goe.\\
Quha calles there, so like ane strangere?\\
Go from my windo, goe.

Lord, I am here, ane wratched mortall,\\
That for thy mercie dois crie and call\\
Unto Thee, my Lord celestiall;\\
See who is at my windo, who?”
\end{scverse}

At the end of Heywood’s \textit{The Rape of Lucrece}, a song is printed beginning—
%\settowidth{\versewidth}{The weather is warme, ’twill doe thee no harme,}
\begin{scverse}
\begin{altverse}
\vleftofline{“}Begone, begone, my Willie, my Billie,\\
Begone, begone, my deere;\\
The weather is warme, ’twill doe thee no harme,\\
Thou canst not be lodged here.”
\end{altverse}
\end{scverse}
which is also in \textit{Wit and Drollery, Jovial Poems}, 1661, p.~25.

\pagebreak


%142
%%===============================================================================


In \textit{Pills to purge Melancholy}, 1707, vol. ii., 44, or 1719, vol. iv., 44, is another
version of that song, beginnings “Arise, arise, my juggy, my puggy;” but in
both editions it is printed to the tune of “Good morrow, ’tis St. Valentine’s day,”
and not to the original music.

I received the following \textit{traditional} version of “Go from my window” from a
very kind friend of former days, the late R. M. Bacon, of Norwich.\dcfootnote{ %a
Mr. Bacon was for many years the well-known editor,
as well as principal proprietor, of \textit{The Norwich Mercury},
and editor of \textit{The Quarterly Musical Review}. His memory
was so stored with traditional songs, learnt in boyhood,
that, having accepted a challenge at the tea-table to sing
a song upon any subject a lady would mention, I have
heard him sing verse after verse upon tea-spoons, and
other such themes, proposed as the most unlikely for
songs to have been written upon. He had learnt a number
of sea songs, principally from one old sailor, and some
were so descriptive, that, it was almost thrilling to hear
them sung by him. Seventeen years ago, these appeared
to me too irregular and declamatory to be reduced
to rhythm, but I have since greatly regretted the loss of
an opportunity that can never recur.
} %end footnote
The tune is
very like that of Ophelia’s Song, “And how should I your true love know;” the
first and last strains being the same in both. The words promise an improvement
of the original, and it is to be regretted that my informant had only heard
the first stanza, which is here printed to the music.

\musicinfo{Rather slow.}{}
\lilypondfile[noindent, current-font-as-main, staffsize=17, 
noragged-right, language=english, nofragment]
{lilypond/142-go-from-my-window-bacon}\normalsize

\musictitle{Dulcina.}

This tune is referred to under the names of “Dulcina;” “As at noon Dulcina
rested;” “From Oberon in fairy-land;” and “Robin Goodfellow.”

The ballad of “The merry pranks of Robin Goodfellow” (attributed to Ben
Jonson) commences with the line, “From Oberon in fairy-land;” and in the old
black-letter copies, is directed to be sung to the tune of \textit{Dulcina}. The ballad of
“As at noon Dulcina rested,” is said, upon the authority of Cayley and Ellis, to
have been written by Sir Walter Raleigh. Both are printed in Percy’s \textit{Reliques
of Ancient Poetry}, series iii., book 2.

The Milk-woman in Walton’s \textit{Angler}, says, “What song was it, I pray
you? Was it, “Come, shepherds, deck your heads,” or “As at noon Dulcina
rested,” \&c.

\pagebreak
%%143
%%===============================================================================

The following ballads were also sung to the tune:--

“The downfall of dancing; or the overthrow of three fiddlers and three bagpipers,”
\&c., “to the tune of \textit{Robin Goodfellow}. Copies in the Douce and Pepys
Collections.

“A delicate new ditty, composed upon the posie of a ring, being, ‘I fancy none
but thee alone:’ sent as a new year’s gift by a lover to his sweet-heart. To the
tune of \textit{Dulcina}.” Roxburghe Collection, vol. i., 80.

“The desperate damsel’s tragedy, or the faithless young man;” beginning,
“In the gallant month of June.”

“A pleasant new song, betwixt a sailor and his love. To the tune of \textit{Dulcina};”
beginning, “What doth ail my love so sadly.” In the Bagford and Roxburghe
Collections, where several more will be found.

A Cavalier’s drinking-song, by Matt. Arundel, to the tune of \textit{Robin Goodfellow},
commencing, “Some say drinking does disguise men,” is printed in \textit{Tixall
Poetry}, quarto, 1813. The last verse dates this after the Restoration.

\textit{Dulcina} was also one of the tunes to the “Psalms and Songs of Sion; turned
into the language and set to the tunes of a strange land,” 1642.

\musicinfo{Cheerfully.}{\textit{Tune of} Dulcina}
\lilypondfile[noindent, current-font-as-main, staffsize=17, 
noragged-right, language=english, nofragment]
{lilypond/143-dulcina}\normalsize
\pagebreak

%%144
%%===============================================================================

\musictitle{Who List To Lead A Soldier’s Life.}

This tune is in \textit{The Dancing Master}, from 1650 to 1725, called “A soldier’s
life, or “Who list to lead a soldier’s life.” There were, evidently, two tunes
under the same name (one of which I have not discovered), because some of the
ballads could not be sung to this air. In Peele’s \textit{Edward I}., 1593, we find,
“Enter a harper and sing, to the tune of \textit{Who list to lead a soldier’s life}, the
following:—
\settowidth{\versewidth}{And play the men both great and small,” \&c.}
\begin{scverse}\vleftofline{“}Go to, go to, you Britons all,\\
And play the men both great and small,” \&c.;
\end{scverse}
and in Deloney’s \textit{Strange Histories}, 1607—
\settowidth{\versewidth}{In woeful wars had victorious been,” \&c;}
\begin{scverse}\vleftofline{“}When Isabell, fair England’s queen,\\
In woeful wars had victorious been,” \&c;
\end{scverse}
neither of which could be sung to \textit{this} air, but “A Song of an English Knight,
that married the Royal Princess, Lady Mary, sister to Henry VHL, which Knight
was afterwards made Duke of Suffolk;” beginning—
\settowidth{\versewidth}{“Eighth Henry ruling in this land,}
\begin{scverse}\begin{altverse}
\vleftofline{“}Eighth Henry ruling in this land,\\
He had a sister fair;”
\end{altverse}
\end{scverse}
and “A Song of the Life and Death of King Richard III., who, after many
murders by him committed, \&c., was slain at the battle of Bosworth, by
Henry VII., King of England;” beginning—
\settowidth{\versewidth}{“In England once there reigned a king,}
\begin{scverse}\begin{altverse}
\vleftofline{“}In England once there reigned a king,\\
A tyrant fierce and fell,”\dcfootnote{ %a
These two ballads have been reprinted by Evans in
\textit{Old Ballads} vol. iii., 30 and 84 (1810); but he has omitted
the names of the tunes to which they were to be sung, not
only in these, but in numberless other instances.
} %end footnote

\end{altverse}
\end{scverse}
as well as several others, are exactly fitted to the tune.

Ophelia’s Song, “Good morrow, ’tis St. Valentine’s day,” and the traditional
air to “Lord Thomas and Fair Ellinor,” are only different versions of this.

In the Pepys Collection, vol i., is a black-letter ballad of “The joyful peace
concluded between the King of Denmark and the King of Sweden, by the means of
our most worthy sovereign James,” \&c., to the tune of “Who list to lead a
soldier’s life;” dated 1613.

In \textit{The Miseries of inforced Marriage} (Dodsley’s Old Plays, vol. v.), the song,
“Who list to have a lubberly load,” was, perhaps, a parody on “Who list to lead
a soldier’s life,” the words of which I have not been successful in finding.

\musicinfo{Gracefully.}{}
\lilypondfile[noindent, current-font-as-main, staffsize=14, 
noragged-right, language=english, nofragment]
{lilypond/144-who-list-to-lead-a-soldiers-life}\normalsize

\pagebreak

%%145
%%===============================================================================

\musictitle{Lord Thomas and Fair Ellinor.}

This traditional version of the tune of \textit{Lord Thomas and Fair Ellinor} is taken
from Sandys’ Collection of Christmas Carols. It is, evidently, the air of \textit{Who
list to lead a soldier's life}? adapted for words of a somewhat different measure.
(See the opposite page.)

At p.~17 of Ritson’s \textit{Observations on the Minstrels}, in enumerating the probable
“causes of the rapid decline of the minstrel profession, since the time of Elizabeth,”
he says, “It is conceived that a few individuals, resembling the character,
might have been lately, and may possibly be still found, in some of the least
polished or less frequented parts of the kingdom. It is not long since the public
papers announced the death of a person of this description, somewhere in Derbyshire;
and another was within these two years to be seen in the streets of London;
he played on an instrument of the rudest construction, which he, properly enough,
called a \textit{hum-strum}, and chanted (amongst others) the old ballad of \textit{Lord Thomas
and Fair Ellinor}, which, by the way, has every appearance of being originally a
minstrel song.”

The ballad will be found in book i., series 3, of Percy’s \textit{Reliques of Ancient
Poetry}, and it is one of those still kept in print in Seven Dials. The black-letter
copies direct it to be sung “to a pleasant new tune.” See Douce Collection,~i.~121.

\musicinfo{Gracefully.}{}
\lilypondfile[noindent, current-font-as-main, staffsize=17, 
noragged-right, language=english, nofragment]
{lilypond/145-lord-thomas-and-fair-ellinor}\normalsize

\musictitle{The Friar and The Nun.}

In Henry Chettle’s \textit{Kind-hart's Dreame}, 1592, two lines are quoted from the
ballad of “The Friar and the Nun.” The tune is in \textit{The Dancing Master,} from
1650 to 1725; in \textit{Musick's Delight on the Cithren}, 1666; in \textit{Pills to purge
Melancholy}; and in many of the ballad-operas, such as \textit{The Beggars' Opera}, \textit{The
Devil to pay}, \textit{The Jovial Crew}, \&c. Henry Carey wrote a song to the tune in his
\textit{Honest Yorkshireman}, 1735, and there are three, or more, in \textit{Pills to purge Melancholy}.
In vol. ii. of some editions, and vol. iv. of others, the title and tune of
“The Friar and the Nun” are printed by mistake with the song of “Fly, merry 
news,” which has no reference to them. 
\pagebreak
The ballad of \textit{The London Prentice} was
%%146
%%===============================================================================
occasionally sung to it, and in some of the ballad-operas the tune bears that name.
In \textit{The Plot}, 1735, it is called “The merry songster.” The composer of the
modern song, “Jump, Jim Crow,” is under some obligations to this air.

Henry Carey’s song is called “The old one outwitted,” and begins—
\begin{scverse}
\begin{altverse}
\vleftofline{“}There was a certain usurer,\\
He had a pretty niece,” \&c.
\end{altverse}
\end{scverse}

In \textit{The Beggars’ Opera}, the name of “All in a misty morning” is given to
the tune, from the first line of a song called \textit{The Wiltshire Wedding}, which will
be found in \textit{Pills to purge Melancholy}, iv. 148, or ii. 148. There are fifteen
verses, of which the following nine suffice to tell the story.

\musicinfo{Quick.}{}
\lilypondfile[noindent, current-font-as-main, staffsize=17, 
noragged-right, language=english, nofragment]
{lilypond/146-the-friar-and-the-nun}\normalsize

\settowidth{\versewidth}{With how d’ye do? and how d’ye do?}
\begin{dcverse}\begin{altverse}
The rustic was a thresher,\\
And on his way he hied,\\
And with a leather bottle\\
Fast buckled by his side;\\
And with a cap of woollen,\\
Which covered cheek and chin;\\
With how d’ye do? and how d’ye do?\\
And how d’ye do? again.
\end{altverse}

\begin{altverse}
I went a little further,\\
And there a met a maid\\
Was going then a milking,\\
A milking, sir, she said;\\
Then I began to compliment,\\
And she began to sing:\\
With how d’ye do? \&c.
\end{altverse}

\begin{altverse}
This maid, her name was Dolly,\\
Cloth’d in a gown of gray,\\
I, being somewhat jolly,\\
Persuaded her to stay:\\
Then straight I fell to courting her,\\
In hopes her love to win,\\
With how d’ye do? \&c.
\end{altverse}

\begin{altverse}
I told her I would married be,\\
And she should be my bride,\\
And long we should not tarry,\\
With twenty things beside:\\
“I’ll plough and sow, and reap and mow,\\
Whilst thou shalt sit and spin,”\\
With how d’ye do? \&c.
\end{altverse}
\end{dcverse}
\pagebreak

%%147
%%===============================================================================

\settowidth{\versewidth}{With how d’ye do? and how d’ye}
\begin{dcverse}\begin{altverse}
\vleftofline{“}Kind sir, I have a mother,\\
Besides, a father, still,\\
And so, before all other.\\
You must ask their good will;\\
For if I be undutiful\\
To them, it is a sin;”\\
With how d’ye do? \&c.
\end{altverse}

\begin{altverse}
Now, there we left the milking-pail,\\
And to her mother went,\\
And when we were come thither,\\
I asked her consent;\\
I doff’d my hat, and made a leg,\\
When I found her within;\\
With how d’ye do? \&c.
\end{altverse}

\begin{altverse}
Her dad came home full weary,\\
(Alas! he could not choose;)\\
Her mother being merry,\\
She told him all the news.\\
Then he was mighty jovial too,\\
His son did soon begin\\
With how d’ye do? \&c.
\end{altverse}

\begin{altverse}
The parents being willing,\\
All parties were agreed,\\
Her portion, thirty shilling;\\
We married were with speed.\\
Then Will, the piper, he did play,\\
Whilst others dance and sing;\\
With how d’ye do? and how d’ye do?\\
And how d’ye do? again.
\end{altverse}
\end{dcverse}

\musictitle{John, Come Kiss Me Now.}

This favorite old tune will be found in Queen Elizabeth’s Virginal Book; in
Playford’s \textit{Introduction}; in \textit{Apollo’s Banquet for the Treble Violin}; and in the
\textit{First part of the Division Violin, containing a collection of Divisions upon several
excellent grounds}, printed by Walsh; as well as Playford’s \textit{Division Violin} (1685.)
In \textit{Pills to purge Melancholy}, vol. iii., 1707., and vol. v., 1719, it is adapted to a
song called \textit{Stow, the Friar}. It is mentioned in Heywood’s \textit{A Woman kill’d with
Kindness}, 1600:
\settowidth{\versewidth}{“I come to dance, not to quarrel: come, what shall it be? \textit{Rogero}?}
\begin{scverse}
\textit{Jack Slime}.—“I come to dance, not to quarrel: come, what shall it be? \textit{Rogero}?\\
\textit{Jenkin}.—“\textit{Rogero}, no; we will dance \textit{The Beginning of the World}.\\
\textit{Sisly}.—“I love no dance so well as \textit{John, come kiss me now}.”
\end{scverse}
In ’\textit{Tis merry when Gossips meet}, 1609:
\settowidth{\versewidth}{Such dauncing, coussen, you would hardly thinke it;}
\begin{scverse}
\vleftofline{\textit{Widow}.—“}No musique in the evening did we lacke;\\
Such dauncing, coussen, you would hardly thinke it;\\
Whole pottles of the daintiest burned sack,\\
’Twould do a wench good at the heart to drinke it.\\
Such store of tickling galliards, I do vow;\\
Not an old dance, but \textit{John, come kisse me now}.
\end{scverse}
In a song in \textit{Westminster Drollery}, 1671 and 1674, beginning, “My name is
honest Harry:”
\settowidth{\versewidth}{And when that we have danc’d a round,}
\begin{scverse}
\vleftofline{“}The fidlers shall attend us,\\
And first play, \textit{John, come kisse me};\\
And when that we have danc’d a round,\\
They shall play, \textit{Hit or misse me}.”\dcfootnote{ %a
\textit{Hit or miss} is a tune in \textit{The Dancing Master} of 1650,
and later editions. It is referred to by Whitlock, in his
\textit{Zootamia, or present Manners of the English}, 12mo., 1654,
where he speaks of one whose practice in physic is
“nothing more than the country dance called \textit{Hit or
misse}.”
} %end footnote
\end{scverse}

In Burton’s \textit{Anatomy of Melancholy}, 1621: “Yea, many times this \textit{love} will
make old men and women, that have more toes than teeth, dance \textit{John, come kiss
me now}.” It is also mentioned in \textit{The Scourge of Folly}, 8vo. (n.d.); in Brathwayte’s
\textit{Shepherd’s Tales}, 1623; in \textit{Tom Tiler and his Wife}, 1661; and in Henry
Bold’s \textit{Songs and Poems}, 1685.

\pagebreak

%%148
%%===============================================================================

\settowidth{\versewidth}{Green Sleeves and Pudding Pyes,’ ‘The P-------’s Delight,’}
\begin{scverse}\vleftofline{“}In \textit{former} times’t hath been upbraided thus,\\
That barber’s music was most barbarous;\\
For that the cittern was confin’d unto\\
\vleftofline{‘}The Ladies’ Fall,’ or ‘\textit{John, come kiss me now},’\\
\vleftofline{‘}Green Sleeves and Pudding Pyes,’ ‘The P-------’s Delight,’\\
\vleftofline{‘}Winning of Bulloigne,’ ‘Essex’s last Good-night,’ \&c.”
\end{scverse}

From lines “On a Barber who became a great Master of Musick.” The ground of
\textit{John, come kiss me now}, was a popular theme for fancies and divisions (now
called fantasias' and variations) for the virginals, lute, and viols. In the
Virginal Book, only the first part of the tune is taken, and it is doubtful if it
then had any second part; the copy we have given is from Playford’s and Walsh’s
\textit{Division Violin}. It is one of the songs parodied in Andro Hart’s \textit{Compendium
of Godly Songs}, before mentioned, on the strength of which the tune has been
claimed as Scotch, although it has no Scotch character, nor has hitherto been
found in any old Scotch copy. Not only are all the other tunes to the songs in the
\textit{Compendium}, of which any traces are left, English, but what little secular music
was printed in Scotland until the eighteenth century, was entirely English or
foreign. The following are the first, second, and twenty-first stanzas of the
“Godly Song”:—
\settowidth{\versewidth}{My prophites call, my preacher's cry,}
\begin{dcverse}\begin{altverse}
John, come kisse me now;\\
John, come kisse me now,\\
Johne, come kisse me by and by,\\
And make no more adow.
\end{altverse}

\begin{altverse}
The Lord thy God I am,\\
That John dois thee call;\\
\end{altverse}
\begin{altverse}
John represents man,\\
By grace celestiall.
\end{altverse}

\begin{altverse}
My prophites call, my preacher's cry,\\
John, come kisse me now;\\
John, come kisse me by and by,\\
And make no more adow.
\end{altverse}
\end{dcverse}

\musicinfo{Rather slow and stately.}{}
\lilypondfile[noindent, current-font-as-main, staffsize=17, 
noragged-right, language=english, nofragment]
{lilypond/148-john-come-kiss-me-now}\normalsize
\pagebreak

%%149
%%===============================================================================

\musictitle{All You That Love Good Fellows, or The London Prentice.}

The tunes called \textit{Nancie} in Queen Elizabeth’s Virginal Book; \textit{Eduward
Nouwels}, in Bellerophon (Amsterdam, 1622, p.~115); \textit{Sir Eduward Nouwel’s
Delight}, in \textit{Friesche Lust-hof}, 1634; and \textit{The London Prentice}, in \textit{Pills to purge
Melancholy} (vi., 342), and in \textit{The Devil to pay}, 1731, are the same: but the two
last contain only sixteen bars, while all the former consist of twenty-four.

The following is the version called \textit{Sir Edward Noels Delight}.

\musicinfo{In marching time.}{}
\lilypondfile{lilypond/149-all-you-that-love-good-fellows}\normalsize

The ballad of “The honour of a London Prentice: being an account of his
matchless manhood, and brave adventures done in Turkey, and by what means he
married the king’s daughter,” is evidently a production of the reign of Elizabeth.
The apprentice maintains her to be “the phœnix of the world,” “the pearl of
princely majesty,” \&c., against “a score of Turkish Knights,” whom he overthrows
at tilt.

The ballad is printed in Ritson’s \textit{English Songs} (among the Ancient Ballads),
and in Evans’ \textit{Old Ballads}, vol. iii., 178. Copies will also be found in the Bagford,
Roxburghe (iii. 747), and other Collections. It was “to be sung to the tune
of \textit{All you that love good fellows} under which name the air is most frequently
mentioned.

\pagebreak

%%150
%%===============================================================================
\DFNsingle

Bishop Earle, in his \textit{Micosmography}, 1628, in giving the character of a Potpoet,
says, “He is a man now much employed in commendations of our navy, and
a bitter inveigher against the Spaniard. His frequentest works go out in single
sheets, and are chanted from market to market to a vile tune, and a worse throat;
whilst the poor country wench melts, like her butter, to hear them. And these
are the stories of some men of Tyburn, or A strange monster out of Germany.”
One of these ballads of “strange monsters out of Germany” will be found in the
Bagford and in the Pepys Collection (ii. 66), “to the tune of \textit{All you that love
good fellows}.” It is entitled “Pride’s fall: or a warning for all English women
by the example of a strange monster born late in Germany, by a merchant’s
proud wife of Geneva.” The ballad, evidently a production of the reign of
James I.,\dcfootnote{\centering %a
See Fairholt's \textit{Satirical Song}s and \textit{Poems on Costume}, p.~107; printed for the Percy Society.
} %end footnote
 is perhaps the one alluded to by Bishop Earle.

There are other ballads about London apprentices; one of “The honors achiev\-ed
in Fraunce and Spayne by four prentises of London,” was entered to John
Danter, in 1592. “Well, my dear countrymen, \textit{What-d’ye-lacks}” (as apprentices
were frequently called, from their usual mode of inviting custom), “I’ll have you
chronicled, and all to be praised, and sung in sonnets, and bawled in new brave
ballads, that all tongues shall troul you \textit{in sœecula seculorum}.”—\textit{Beaumont and
Fletcher's Philaster}.

One of the ballads to the tune of “the worthy London prentice” relates
to a very old superstition, and will recall to us the “Out, damned spot!” in
\textit{Macbeth}. It is entitled the “True relation of Susan Higges, dwelling in Risborow,
a towne in Buckinghamshire, and how she lived twenty years by robbing
on the high wayes, yet unsuspected of all that knew her; till at last coming to
Messeldon, and there robbing and murdering a woman, which woman knew her,
and standing by her while she gave three groanes, \textit{she spat three drops of blood in
her face, which never could be washt out}, by which she was knowne, and executed
for the aforesaid murder, at the assises in Lent at Brickhill.” A copy is in the
Roxburghe Collection, i. 424; also in Evans’ \textit{Old Ballads}, i. 203 (1810).

I have not found any song or ballad commencing “All you that \textit{love} good
fellows,” although so frequently quoted as a tune; but there are several “All you
that \textit{are},” and “All you that \textit{be} good fellows,” which, from similarity of metre,
I assume to be intended for the \textit{same air}.

In a chap-book called “The arraigning and indicting of Sir John Barleycorn,
knight; newly composed by a well-wisher to Sir John, and all that love him,” are
two songs, “All you that are good fellows,” and “All you that \textit{be} good fellows,”
“to the tune of \textit{Sir John Barleycorn}, or \textit{Jack of all trades}.” Lowndes speaks of
this tract as printed for T. Passenger in 1675, and of the author as Thomas
Robins; but there are Aldermary and Bow Church-yard editions of later date.

Another “All you that \textit{are} good fellows” is here printed to the shorter copy of
the tune. It is from a little black-letter volume (in Wood’s library, Ashmolean
Museum) entitled “Good and true, fresh and new Christmas Carols,” \&c.,
printed by E. P. for Francis Coles, dwelling in the Old Bailey, 1642. It is one
\pagebreak
%%151
%%===============================================================================
of the \textit{merry} Christmas carols, and to be sung to the tune of “All you that
are good fellows.”

\musicinfo{In marching time.}{}
\lilypondfile{lilypond/151-all-you-that-are-good-fellows}\normalsize

\settowidth{\versewidth}{Plum-porridge, roast beef, and minc’d pies,}
\begin{dcverse}\begin{altverse}
This is a time of joyfulness,\\
And merry time of year,\\
When as the rich with plenty stor’d\\
Do make the poor good cheer.\\
Plum-porridge, roast beef, and minc’d pies,\\
Stand smoking on the board;\\
With other brave varieties,\\
Our master doth afford.
\end{altverse}

\begin{altverse}
Our mistress and her cleanly maids\\
Have neatly play’d the cooks;\\
Methinks these dishes eagerly\\
At my sharp stomach looks,\\
As though they were afraid\\
To see me draw my blade;\\
But I revenged on them will be,\\
Until my stomach’s stay’d.
\end{altverse}

\begin{altverse}
Come fill us of the strongest,\\
Small drink is out of date;\\
Methinks I shall fare like a prince,\\
And sit in gallant state:\\
This is no miser’s feast,\\
Although that things be dear;\\
God grant the founder of this feast\\
Each Christmas keep good cheer.
\end{altverse}

\begin{altverse}
This day for Christ we celebrate,\\
Who was born at this time;\\
For which all Christians should rejoice,\\
And I do sing in rhyme.\\
When you have given thanks,\\
Unto your dainties fall,\\
Heav’n bless my master and my dame,\\
Lord bless me, and you all.
\end{altverse}
\end{dcverse}

\pagebreak

%%152
%%===============================================================================

\musictitle{The British Grenadiers.}

The correct date of this fine old melody appears altogether uncertain, as it
is to be found in different forms at different periods; but it is here placed in juxtaposition
to \textit{Sir Edward Noel’s Delight}, and \textit{All you that love good fellows}, or
\textit{The London Prentice}, because evidently derived from the same source. The
commencement of the air is also rather like \textit{Prince Rupert’s March}, and the end
resembles \textit{Old King Cole}, with the difference of being major instead of minor.
Next to the National Anthems, there is not any tune of a more spirit-stirring
character, nor is any one more truly characteristic of English national music.
This version of the tune is as played by the band of the Grenadier Guards. The
words are from a copy about a hundred years old, with the music.

\musicinfo{March.}{}
\lilypondfile{lilypond/152-the-british-grenadiers}\normalsize

\settowidth{\versewidth}{Sing tow, row, row, row, row, row, for the British Grenadiers.}
\begin{scverse}
Those heroes of antiquity ne’er saw a cannon ball,\\
Or knew the force of powder to slay their foes withal;\\
But our brave boys do know it, and banish all their fears,\\
Sing tow, row, row, row, row, row, for the British Grenadiers.\\
\vin\vin\vin\vin\vin\vin\textit{Chorus}.—But our brave boys, \&c.
\end{scverse}

\pagebreak


%%153
%%===============================================================================

\settowidth{\versewidth}{Here come the Grenadiers, my boys, who know no doubts or fears.}
\begin{scverse}
Whene’er we are commanded to storm the palisades,\\
Our leaders march with fusees, and we with hand grenades,\\
We throw them from the glacis, about the enemies’ ears,\\
Sing tow, row, row, row, row, row, the British Grenadiers.\\
\vin\vin\vin\vin\vin\vin\textit{Chorus}.—We throw them, \&c.

And when the siege is over, we to the town repair,\\
The townsmen cry Hurra, boys, here comes a Grenadier,\\
Here come the Grenadiers, my boys, who know no doubts or fears.\\
Then sing tow, row, row, row, row, row, the British Grenadiers.\\
\vin\vin\vin\vin\vin\vin\textit{Chorus}.—Here come the, \&c.

Then let us fill a bumper, and drink a health to those\\
Who carry caps and pouches, and wear the louped clothes,\\
May they and their commanders live happy all their years,\\
With a tow, row, row, row, row, row, for the British Grenadiers.\\
\vin\vin\vin\vin\vin\vin\textit{Chorus}.—May they, \&c.
\end{scverse}

\musictitle{The Cushion Dance.}

The Cushion Dance was in favour both in court and country in the reign of
Elizabeth, and is occasionally danced even at the present day. In Lilly’s \textit{Euphues},
1580, Lucilla, says, “Trulie, Euphues, you have \textit{mist the cushion}, for I was neither
angrie with your long absence, neither am I well pleased at your presence.” This
is, perhaps, in allusion to the dance, in which each woman selected her partner by
placing the cushion before him. Taylor, the water-poet, calls it “a pretty little
provocatory dance,” for he before whom the cushion was placed, was to kneel and
salute the lady. In Heywood’s \textit{A Woman kill’d with Kindness}, (which Henslow
mentions in his diary, in 1602), the dances which the country people call for are,
\textit{Rogero}; \textit{The Beginning of the World}, or \textit{Sellenger’s Round}; \textit{John, come kiss me
now}; \textit{Tom Tyler}; \textit{The hunting of the Fox}; \textit{The Hay}; \textit{Put on your smock a
Monday}; and \textit{The Cushion Dance}; and Sir Francis thus describes their style of
dancing:—
\settowidth{\versewidth}{Made with their high shoes: though their skill be small,}
\begin{scverse}
\vleftofline{“}Now, gallants, while the town-musicians\\
Finger their frets within; and the mad lads\\
And country lasses, every mother’s child.\\
With nosegays and bride-laces in their hats,\\
Dance all their country measures, rounds, and jigs,\\
What shall we do? Hark! they’re all on the hoigh;\\
They toil like mill-horses, and turn as round;\\
Marry, not on the toe: aye, and they caper,\\
But not without cutting; you shall see, to-morrow,\\
The hall floor peck’d and dinted like a mill-stone,\\
Made with their high shoes: though their skill be small,\\
Yet they tread heavy where their hob-nails fall.”
\end{scverse}

When a partner was selected in the dance, he, or she, sang “Prinkum-prankum
is a fine dance,” \&c.; which line is quoted by Burton, in his \textit{Anatomy
of Melancholy}; and, “No dance is lawful but Prinkum-prankum,” in \textit{The Muses’
Looking-glass}, 1638.

In the \textit{Apothegms of King James, \pagebreak the Earl of Worcester}, \&c., 1658, a wedding %\pagebreak
%%154
%%===============================================================================
entertainment is spoken of: and, “when the masque was ended, and time had
brought in the supper, \textit{the cushion led the dance} out of the parlour into the hall.”
Selden, speaking of \textit{Trenchmore} and \textit{The Cushion Dance} in Queen Elizabeth’s
time, says, “Then all the company dances, lord and groom, lady and kitchen-maid, 
no distinction.”—(See ante p.~82.) In \textit{The Dancing Master} of 1686, and
later editions, the figure is thus described:—

“This dance is begun by a single person (either man or woman), who, taking a
cushion in hand, dances about the room, and at the end of the tune, stops and sings,
‘This dance it will no further go.” The musician answers, ‘I pray you, good Sir,
why say you so?’—\textit{Man}. ‘Because Joan Sanderson will not come too.’—\textit{Musician}.
‘She must come too, and she shall come too, and she must come whether she will
or no.’ Then he lays down the cushion before the woman, on which she kneels, and
he kisses her, singing ‘Welcome, Joan Sanderson, welcome, welcome.’ Then she
rises, takes up the cushion, and both dance, singing, ‘Prinkum-prankum is a fine dance,
and shall we go dance it once again, once again, and once again, and shall we go dance
it once again.’ Then making a stop, the woman sings as before, ‘This dance it will
no further go.’—\textit{Musician}. ‘I pray you, madam, why say you so?'— \textit{Woman}. ‘Because
John Sanderson will not come too.’—\textit{Musician}. ‘He must come too, and he shall
come too, and he must come whether he will or no.’ And so she lays down the
cushion before a man, who kneeling upon it, salutes her; she singing, ‘Welcome,
John Sanderson, welcome, welcome.’ Then he taking up the cushion, they take
hands, and dance round, singing as before. And thus they do, till the whole company
are taken into the ring; and if there is company enough, make a little ring in its
middle, and within that ring, set a chair, and lay the cushion in it, and the first man
set in it. Then the cushion is laid before the first man, the woman singing, ‘This
dance it will no further go;’ and as before, only instead of ‘Come too,’ they sing, ‘Go
fro;’ and instead of ‘Welcome, John Sanderson,’ they sing, ‘Farewell, John Sanderson,
farewell, farewell;’ and so they go out one by one as they came in. \textsc{Note}.—\textit{The
women are kissed by all the men in the ring at their coming and going out, and likewise
the men by all the women}.”

This agreeable pastime tended, without doubt, to popularize the dance.

One of the engravings in Johannis de Brunes \textit{Emblemata} (4to., Amsterdam,
1624, and 1661) seems to represent the Cushion Dance. The company being
seated round the room, one of the gentlemen, hat in hand, and with a cushion held
over the left shoulder, bows to a lady, and seems about to lay the cushion at her
feet.

In 1737, the Rev. Mr. Henley, or “Orator Henley,” as he called himself,
advertised in the \textit{London Daily Post} that he would deliver an oration on the
subject of the Cushion Dance.

A political parody is to be found in \textit{Poems on Affairs of State, from 1640 to
1704}, called, “The Cushion Dance at Whitehall, by way of Masquerade. To the
tune of Joan Sanderson.”

\settowidth{\versewidth}{The trick of trimming is a fine trick,}
\begin{scverse}
\textit{\vleftofline{Enter God}frey Aldworth, followed by the King and Duke.}\\
\vleftofline{\textit{King}. “}The trick of trimming is a fine trick,\\
And shall we go try it once again?\\
\vleftofline{\textit{Duke}. “}The plot it will no further go.\\
\vleftofline{\textit{King}. “}I pray thee, wise brother, why say you so,” \&c.
\end{scverse}
\pagebreak

%%155
%%===============================================================================
\DFNdouble

The tunes of Cushion-Dances (like Barley-Breaks) have the first part in
\timesig{3}{4}, and the last in \timesig{6}{8} time. The earliest printed copy I have found is in \textit{Tablature
de Luth, intitulé Le Secret des Muses}, 4to., Amsterdam, 1615, where it is
called \textit{Gaillarde Anglaise}. In \textit{Nederlandtsche Gedenck-Clanck}, Haerlem, 1626,
the same air is entitled \textit{Gallarde Suit Margrie}t, which being intended as English,
may he guessed as “Galliard, Sweet Margaret.” It is the following:--

\musicinfo{Slow.}{}

\lilypondfile{lilypond/155-the-cushion-dance}\normalsize

The Galliard (a word meaning brisk, gay; and used in that sense by Chaucer)
is described by Sir John Davis as a swift and wandering dance, with lofty turns
and capriols in the air. Thoinot Arbeau, in his \textit{Orchesographie}, 1589, says that,
\textit{formerly}, when the dancer had taken his partner for the galliard, they first placed 
themselves at the end of the room, and, \pagebreak after a bow and curtsey, they walked once
%\pagebreak
%156
%===============================================================================
or twice round it. Then the lady danced to the other end, and remained there
dancing, while the gentleman followed; and presenting himself before her, made
some steps, and then turned to the right or left. After that she danced to the
other end, and he followed, doing other steps; and so again, and again. “But
now,” says he, “in towns they dance it tumultuously, and content themselves
with making the five steps and some movements without any design, caring only
to be in position on the sixth of the bar” (pourvu qu’ils tombent en cadence).
In the four first steps, the left and right foot of the dancer were raised alternately,
and on the fifth of the bar he sprang into the air, twisting round, or capering, as
best he could. The repose on the sixth note gave more time for a lofty spring.\dcfootnote{ %a
Narea, in his Glossary, refers to \textit{Cinque pace}, but that
was a dance in common time; four steps to the four beats
of the bar, and the fifth on a long note at the commencement
of the second bar.
} %end footnote
“Let them take their pleasures,” says Burton, in his \textit{Anatomy of Melancholy};
“young men and maids flourishing in their age, fair and lovely to behold, well
attired, and of comely carriage, dancing a Greek \textit{Galliarde}, and, \textit{as their dance
requireth}, keep their time, now turning, now tracing, now apart, now altogether,
now a curtesie, then a caper, \&c., it is a pleasant sight.”

The following tune is from \textit{The Dancing Master} of 1686, called “\textit{Joan Sanderson,
or The Cushion Dance}, an old Round Dance.”

\lilypondfile{lilypond/156-joan-sanderson-or-the-cushion-dance}\normalsize


\pagebreak

%%157
%%===============================================================================

Reverting to the pavan and galliard, Morley says, “The pavan” (derived from
pavo, a peacock) “for grave dancing; galliards, which usually follow pavans, they
are for a lighter and more stirring kind of dancing.” The pavan was sometimes
danced by princes and judges in their robes, and by ladies with long trains held up
behind them; but usually the galliard followed the pavan, much in the same manner
as the gavotte follows the minuet. Butler, in his \textit{Principles of Musick}, 1636, says,
“Of this sort (the Ionic mood) are pavans, invented for a slow and soft kind of
dancing, altogether in duple proportion [common time]. Unto which are framed
galliards for more quick and nimble motion, always in triple proportion: and,
therefore, the triple is oft called galliard time, and the duple pavan time. In this
kind is also comprehended the \textit{infinite multitude of Ballads}, set to sundry pleasant
and delightful tunes by cunning and witty composers, with \textit{country dances} fitted
unto them,... and which surely might and would be more freely permitted by
our sages, were they used as they ought, only for health and recreation.”—(p.~8.)
At this time Puritanism was nearly at its height.

\musictitle{With My Flock As Walked I.}

Stafford Smith found this song, with the tune, in a manuscript of about the year
1600, and printed it in his \textit{Musica Antiqua}, p.~57. I discovered a second copy of
the tune in Elizabeth Rogers’ MS. Virginal book, in the British Museum, under
the name of \textit{The faithful Brothers}.

The song is evidently in allusion to Queen Elizabeth, and in the usual complimentary
style to her beauty, to her vow of virginity, \&c.

\musicinfo{Gracefully.}{}
\lilypondfile[noindent, current-font-as-main, staffsize=17, 
noragged-right, language=english, nofragment]
{lilypond/157-with-my-flock-as-walked-i}\normalsize

\pagebreak

%%158
%%===============================================================================
\DFNsingle

\settowidth{\versewidth}{That they themselves might move}
\begin{dcverse}Such a face she had for to\\
Invite any man to love her;\\
But her coy behaviour taught\\
That it was but in vain to move her;\\
For divers so this dame had wrought\\
That they themselves might move her.\dcfootnote{\centering %a
This line is evidently incorrect, but I have no other copy to refer to.
} %end footnote

Phœbus for her favour spent\\
His hair, her fair brows to cover;\\
Venus’ cheek and lips were sent,\\
That Cupid and Mars might move her;\\
But Juno, alone, her nothing lent,\\
Lest Jove himself should love her.

Though she be so pure and chaste,\\
That nobody can disprove her;\\
So demure and straightly cast,\\
That nobody dares to move her;\\
Yet is she so fresh and sweetly fair\\
That I shall always love her.

Let her know, though fair she be,\\
That there is a power above her;\\
Thousands more enamoured shall be,\\
Though little it will move her;\\
She still doth vow virginity,\\
When all the world doth love her.
\end{dcverse}
\musictitle{Go No More A Rushing.}

This tune is called \textit{Go no more a rushing}, in a MS. Virginal Book of Byrd’s
arrangements and compositions, in the possession of Dr.~Rimbault; and \textit{Tell me,
Daphne}, in Queen Elizabeth’s Virginal Book.

\musicinfo{Moderate time.}{}
\lilypondfile{lilypond/158-go-no-more-a-rushing}\normalsize

\musictitle{The Blind Beggar’s Daughter of Bethnal Green.}

This tune was found by Dr.~Rimbault in a MS. volume of Lute Music, written
by Rogers, a celebrated lutenist of the reign of Charles II., in the library at
Etwall Hall, Derbyshire. It is there called \textit{The Cripple}, and the ballad of
\textit{The stout Cripple of Cornwall} is directed to be sung to the tune of \textit{The blind
Beggar}. See Roxburghe Collection, i. 389, and Bagford, i. 32. It is also in
Evans’ \textit{Old Ballads}, i. 97 (1810); but, as too frequently the case, the name of
the tune to which it was to be sung, is there omitted.

\pagebreak

%%159
%%===============================================================================

Pepys, in his diary, 25th June, 1663, speaks of going with Sir William and
Lady Batten, and Sir J. Minnes, to Sir W. Rider’s, at Bednall Green, to dinner,
“a fine place;” and adds, “This very house was built by the blind Beggar of
Bednall Green, so much talked of and sang in ballads; but they say it was only
some outhouses of it.” The house was called Kirby Castle, then the property of
Sir William Ryder, Knight, who died there in 1669.

“This popular old ballad,” says Percy, “was written in the reign of Elizabeth,
as appears not only from verse 23, where the arms of England are called the
‘Queenes armes;’ but from its tune being quoted in other old pieces written in
her time. See the ballad on \textit{Mary Ambree},” \&c.

In a black-letter book called \textit{The World’s Folly}, we read that “a dapper fellow,
that in his youth had spent more than he got, on his person, fell to singing
\textit{The blind Beggar}, to the tune of \textit{Heigh ho}!”—(\textit{Brit. Bibliographer}, ii. 560.)

In the “Collection of Loyal Songs written against the Rump Parliament,” and
in “Rats rhimed to death, or the Rump Parliament hang’d up in the shambles”
(1660), are many songs to the tune of \textit{The blind Beggar}, as well as in the King’s
Pamphlets, Brit. Museum.

Among them, “A Hymn to the gentle craft, or Hewson’s lamentation”
(a satire on Lord Hewson, one of Cromwell’s lords, who had been a cobbler,
and had but one eye), and “The second Martyrdom of the Rump.”

The tune was sometimes called \textit{Pretty Bessy}, and a ballad to be sung to it,
under that name, is in the Roxburghe Collection, i. 142.

\musicinfo{Moderate time and with expression.}{}
\lilypondfile{lilypond/159-the-blind-beggars-daughter-of-bethnal-green}\normalsize
\pagebreak

%%160
%%===============================================================================

The ballad of The blind Beggar will be found in Percy’s \textit{Reliques}, book ii.,
series 2; in the Roxburghe Collection, i. 10; and in Dixon’s \textit{Songs of the Peasantry
of England}. It is still kept in print in Seven Dials, and sung about the country,
but to the following tune.

\bigskip
\lilypondfile[noindent, current-font-as-main, staffsize=17, 
noragged-right, language=english, nofragment]
{lilypond/160-the-blind-beggars-daughter-of-bethnal-green}\normalsize

\musictitle{Cock Lorrel, Or Cook Lawrel.}

This tune is in the \textit{Choice Collection of 180 Loyal Songs}, \&c. (3rd edit. 1685),
and in \textit{Pills to purge Melancholy}, as well as in every edition of \textit{The Dancing
Master}, from 1650 to 1725. In \textit{The Dancing Master} it is called \textit{An old man is
a bed full of bones}, from a song, of which four lines are quoted in Rowley’s
\textit{A Match at Midnight}, act i., sc. 1., and one in Shirley’s \textit{The Constant Maid},
act ii., sc. 2., where the usurer’s niece sings it.

The song of \textit{Cook Lorrel} is in Ben Jonson’s masque, \textit{The Gipsies metamorphosed}. 
Copies are also in the Pepys Collection of Ballads; in Dr.~Percy’s folio
MS., p.~182;\dcfootnote{\centering %a
See Dr.~Dibdin’s Decameron, vol. 3.
} %end footnote
 and, with music, in \textit{Pills to purge Melancholy}. It is a satire upon
rogues and knaves of all classes supposed to be doomed to perdition. Cook
Lorrel, a notorious rogue, invites his Satanic Majesty into the Peak in Derbyshire
to dinner; and he, somewhat inconvenienced by the roughness of the
road, commences by feasting on the most delicate sinner:

\settowidth{\versewidth}{His stomach was queasie (for, riding there coach’d,}
\begin{scverse}
\begin{altverse}
\vleftofline{“}His stomach was queasie (for, riding there coach’d,\\
The jogging had caused some crudities rise);\\
To help it he called for a Puritan \textit{poach’d},\\
That used to turn up the \textit{eggs} of his eyes, \&c.”
\end{altverse}
\end{scverse}

\pagebreak


%%161
%%===============================================================================

Wynken de Worde printed a tract called \textit{Cocke Lorrel’s Bote}; in which persons
of all classes, and, among them the \textit{Mynstrelles}, are summoned to go on board
his Ship of Fools. \textit{Cock Lorels’s Boat} is mentioned in a MS. poem in the
Bodleian Library, called \textit{Doctour Double Ale}, and in John Heywood’s \textit{Epigrams
upon 300 Proverbs}, 1566 (in the Epigram upon a Busy-body, No. 189).

In S. Rowland’s \textit{Martin Markhall, his defence and answer to the Bellman of
London}, 1610, is a list of rogues by profession, in which \textit{Cock Lorrel} stands
second. He is thus described: “After him succeeded, by the general council,
one Cock Lorrell, the most notorious knave that ever lived. By trade he was a
tinker, often carrying a pan and hammer for shew; but when he came to a good
booty, he would cast his profession in a ditch, and play the padder.” In 1565,
a book was printed called \textit{The Fraternitye of Vacabondes; whereunto also is
adjoyned the twventy-five orders of knaves: confirmed for ever by Cocke Lorell}.

In \textit{Satirical Poems} by Lord Rochester (Harl. MSS., 6913) there is a ballad to
the tune of \textit{An old man is a bed full of bones}, but the air is far more generally
referred to by the name of \textit{Cock Lorrel}.

In the “Collection of Loyal Songs written against the Rump Parliament”
there are many to this air, such as “The Rump roughly but righteously
handled;” “The City’s Feast to the Lord Protector;” “St. George for England”
(commencing, “The Westminster Rump hath been little at ease”); \&c., \&c.
Others in the King’s Pamphlets, Brit. Mus.; in the \textit{Collection of 180 Loyal
Songs}, 1685; in \textit{Poems on Affairs of State}, vol. i., 1703; and in the Roxburghe
Collection of Ballads.

A tune called \textit{The Painter} is sometimes mentioned, and it appears to be
another name for this air, because the ballad of “The Painter’s Pastime: or a
woman defined after a new fashion,” \&c., was to be sung to the tune of \textit{Cook
Laurel}. A black-letter copy is in the Douce Collection (printed by P. Brooksby,
at the Golden Ball, \&c.).

Some copies of the tune are in a major, others in a minor key. The four lines
here printed to it are from an \textit{Antidote to Melancholy}, 1651, for, although some
of the ballads above quoted are witty, they would not be admissible in the
present day.

\lilypondfile{lilypond/161-cock-lorrel-or-cook-lawrel}\normalsize

\pagebreak
%%162
%%===============================================================================

\musictitle{Fortune My Foe.}

The tune of \textit{Fortune} is in Queen Elizabeth’s Virginal Book; in William
Ballet’s MS. Lute Book; in Vallet’s \textit{Tablature de Luth}, book i., 1615, and
book ii., 1616; in \textit{Bellerophon}, 1622; in \textit{Nederlandtsche Gedenck-Clanck}, 1626; in
Dr.~Camphuysen’s \textit{Stichtelycke Rymen}, 1652; and in other more recent publications. 
In the Dutch books above quoted, it is always given as an English air.

A ballad “Of one complaining of the mutability of Fortune” was licensed to
John Charlewood to print in 1565-6 (See Collier’s \textit{Ex. Reg. Stat. Comp}., p.~139).
A black-letter copy of “A sweet sonnet, wherein the lover exclaimeth against
Fortune for the loss of his lady’s favour, almost past hope to get it again, and in
the end receives a comfortable answer, and attains his desire, as may here appear:
to the tune of \textit{Fortune my foe},” is in the Bagford Collection of Ballads (643 m.,
British Museum). It begins as follows:—

\musicinfo{Slow.}{}

\lilypondfile{lilypond/162-fortune-my-foe}\normalsize

There are twenty-two stanzas, of four lines each, in the above.

\textit{Fortune my foe} is alluded to by Shakespeare in \textit{The Merry Wives of Windsor},
act ii., sc. 3; and the old ballad of \textit{Titus Andronicus}, upon which Shakespeare
founded his play of the same name, was sung to the tune. A copy of that ballad
is in the Roxburghe Collection, i. 392, and reprinted in Percy’s \textit{Reliques}.

Ben Jonson alludes to \textit{Fortune my foe}, in \textit{The case is altered}, and in his masque
\textit{The Gipsies Metamorphosed}; Beaumont and Fletcher, in \textit{The Custom of the
Country}, \textit{The Knight of the Burning Pestle}, and \textit{The Wild Goose Chase}; Lilly
gives the first verse in his \textit{Maydes Metamorphosis}, 1600; Chettle mentions the
tune in \textit{Kind-hart's Dreame},~1592; Burton, in his \textit{Anatomy of Melancholy}, 1621; 
Shirley, in \textit{The Grateful Servant},~1630; \pagebreak Brome, in his \textit{Antipodes}, 1638. See
%%163
%%===============================================================================
also Lodge’s \textit{Rosalind}, 1590; \textit{Lingua}, 1607; \textit{Every Woman in her humour}, 1609;
\textit{The Widow’s Tears}, 1612; Henry Hutton’s \textit{Follie’s Anatomie}, 1619; \textit{The two
merry Milkmaids}, 1620; \textit{Vox Borealis}, 1641; \textit{The Rump, or Mirror of the
Times}, 1660; \textit{Tom’s Essence}, 1677, \&c. In Forbes’ \textit{Cantus}, 1682, is a parody
on \textit{Fortune my foe}, beginning, \textit{Satan my foe, full of iniquity}, with which the tune
is there printed.
\DFNsingle

One reason for the great popularity of this air is that “the metrical lamentations
of extraordinary criminals have been usually chanted to it for upwards of
these two hundred years.” Rowley alludes to this in his \textit{Noble Soldier}, 1634:
\settowidth{\versewidth}{The King! shall I be bitter ’gainst the King?}
\begin{scverse}
\vleftofline{“}The King! shall I be bitter ’gainst the King?\\
I shall have scurvy ballads made of me,\\
Sung to \textit{the hanging tune}!”
\end{scverse}
And in “The penitent Traytor: the humble petition of a Devonshire gentleman,
who was condemned for treason, and executed for the same, anno 1641,” the
last verse but two runs thus:
\begin{scverse}
\vleftofline{“}How could I bless thee, couldst thou take away\\
My life and infamy both in one day?\\
But this in ballads will survive I know,\\
Sung to that \textit{preaching tune, Fortune my foe}."
\end{scverse}
The last is from “Loyal Songs written against the Rump Parliament.”

Deloney’s ballad, “The Death of King John,” in his \textit{Strange Histories}, 1607;
and “The most cruel murder of Edward V., and his brother the Duke of York,
in the Tower, by their uncle, the Duke of Gloucester” (reprinted in Evans’ \textit{Old
Ballads}, iii. 13, ed. 1810), are to this tune; but ballads of this description which
were sung to it are too many for enumeration. In the first volume of the Roxburghe
Collection, at pages 136, 182, 376, 392, 486, 487, 488, and 490, are
ballads to the tune of \textit{Fortune}, and all about murders, last dying speeches, or some
heavy misfortunes.

In the Pepys’ Collection, i. 68, is a ballad of “The lamentable burning of the
city of Cork, by the lightning which happened the last day of May, 1622, after
the prodigious battle of the stares” (\ie, starlings), “which fought most strangely
over and near the city the 12th and 14th May, 1621.”

Two other ballads require notice, because the tune is often referred to under
their names, \textit{Dr.~Faustus}, and \textit{Aim not too high}. The first, according to the title
of the ballad, is “The Judgment of God shewed upon Dr.~John Faustus: tune,
\textit{Fortune my foe}.” A copy is in the Bagford Collection.\dcfootnote{\centering %a
It is also printed in my National English Airs, quarto, part i., 1838.
} %end footnote
It is illustrated by two
woodcuts at the top: one representing Dr.~Faustus signing the contract with the
devil; and the other shewing him standing in a magic circle, with a wand in his
left hand, and a sword with flame running up it, in his right: a little devil
seated on his right arm. Richard Jones had a licence to print the ballad “of the
life and deathe of Dr.~Faustus, the great cungerer,” on the 28th Feb., 1588-9.

In the Roxburghe Collection, i. 434, is “Youth’s warning piece,” \&c., “to the
tune of \textit{Dr.~Faustus};” printed for A. K., 1636. And in Dr.~Wild’s\textit{ Iter
Boreal}e, 1671, “The recantation of a penitent Proteus,” \&c., to the tune of
\textit{Dr.~Faustus}.

\pagebreak

%%164
%%===============================================================================

The other name is derived from—
\settowidth{\versewidth}{So give Him thanks that shall encrease it still,”}
\begin{scverse}
\vleftofline{“}An excellent song, wherein you shall finde\\
Great consolation for a troubled mind.
\end{scverse}
To the tune of \textit{Fortune my foe}.” Commencing thus:
\begin{scverse}
\vleftofline{“}\textit{Ayme not too hie} in things above thy reach;\\
Be not too foolish in thine owne conceit;\\
As thou hast wit and worldly wealth at will,\\
So give Him thanks that shall encrease it still,”\&c.
\end{scverse}

This ballad is also in the Roxburghe Collection, i. 106., printed by the “Assignes
of Thomas Symcocke:” and, in the same, others to the tune of \textit{Aim not too high}
will be found, viz., in vol. i., at pages 70, 78, 82, 106, 132, and 482; in vol. ii.,
at pages 128, 130, 189, 202, 283, 482, and 562, \&c.

In the Douce Collection there is a ballad of “The manner of the King’s”
[Charles the First’s] “Trial at Westminster Hall,” \&c.; “the tune is \textit{Aim not
too high}.”

\musictitle{Death And The Lady.}

\textit{Death and the Lady} is one of a series of popular ballads which had their rise
from the celebrated \textit{Dance of Death}. A \textit{Dance of Death} seems to be alluded to
in \textit{The Vision of Pierce Plowman}, written about 1350:
\begin{scverse}
\vleftofline{“}Death came driving after, and al[l] to dust pashed\\
Kyngs and Kaisars, Knights and Popes;”
\end{scverse}
but the subject was rendered especially popular in England by Lydgate’s free
translation from a French version of the celebrated German one by Machaber.

Representations of \textit{The Dance of Death} were frequently depicted upon the
walls of cloisters and cathedrals. Sir Thomas More speaks of one “pictured in
Paules,” of which Stow, in his \textit{Survey of London}, gives the following account:—
“John Carpenter, town clerk of London in the reign of Henry VI., caused, with
great expense, to be curiously painted upon board, about the north cloister of
Paul’s, a monument of Death leading all estates, with the speeches of Death, and
answer of every state. This cloister was pulled down in 1549.”

On the walls of the Hungerford Chapel in Salisbury Cathedral was a painting
executed about 1460, representing Death holding conversation with a young
gallant, attired in the fullest fashion, who thus addresses him:—
\settowidth{\versewidth}{Alasse, Dethe, alasse! a blessful thing thou were}
\begin{scverse}
\vleftofline{“}Alasse, Dethe, alasse! a blessful thing thou were\\
If thou woldyst spare us in our lustynesse,\\
And cum to wretches that bethe of he[a]vy chere,\\
When they thee clepe [call] to slake their dystresse.\\
But, owte alasse! thyne owne sely self-willdnesse\\
Crewelly we[a]rieth them that sighe, wayle, and weepe,\\
To close their eyen that after thee doth clepe.’’
\end{scverse}
To which Death gloomily replies:
\begin{scverse}
\vleftofline{“}\vleftofline{“}Graceles Gallante, in all thy luste and pryde\\
Remembyr that thou ones schalte dye;\\
De[a]th shold fro’ thy body thy soule devyde,\\
Thou mayst him not escape, certaynlỳ.
\end{scverse}

\pagebreak
%%165
%%===============================================================================

\settowidth{\versewidth}{To the de[a]de bodys cast downe thyne eye,}
\begin{scverse}
To the de[a]de bodys cast downe thyne eye,\\
Behold them well, consyder and see,\\
For such as they are, such shalt thou be.”
\end{scverse}

Among the Roxburghe Ballads is one entitled “Death’s uncontrollable summons,
or the mortality of mankind; being a dialogue between Death and a young
man,” which very much resembles the verses in the Hungerford Chapel, above
quoted. We have also “The dead man’s song,” reprinted in Evans’ Collection,
“Death and the Cobbler,” and “Death’s Dance,” proving the popularity of these
moralizations on death. Another “Dance and Song of Death,” which was
licensed in 1568, has been printed at page 85.

In the Douce Collection is a black-letter copy of “The midnight messenger, or
a sudden call from an earthly glory to the cold grave, in a dialogue between Death
and a rich man,” \&c., beginning—
\begin{scverse}\vleftofline{“}Thou wealthy man, of large possessions here,\\
Amounting to some thousand pounds a year,\\
Extorted by oppression from the poor,\\
The time is come that thou shalt be no more,” \&c.;
\end{scverse}
which is reprinted in Dixon’s \textit{Songs of the Peasantry}, \&c.

In Mr. Payne Collier’s MS. volume, written in the reign of James I., is a
dialogue of twenty-four stanzas, between “Life and Death,” commencing—
\settowidth{\versewidth}{Nay, what art thou, that I should give}
\begin{scverse}\vleftofline{\textit{Life}.—}“Nay, what art thou, that I should give\\
\vin To thee my parting breath?\\
Why may not I much longer live?”\\
\vleftofline{\textit{Death}.—}\vin “Behold! my name is Death.”\\
\vleftofline{\textit{Life}.—}“I never have seen thy face before;\\
\vin Now tell me why thou came:\\
I never wish to see it more—\\
\vleftofline{\textit{Death}.—}\vin “Behold! Death is my name,” \&c.
\end{scverse}

The following “Dialogue betwixt an Exciseman and Death” is from a copy in
the Bagford Collection, dated 1659.

\begin{dcverse}\footnotesize
Upon a time when Titan’s steeds were driven\\
To drench themselves against the western heaven;\\
And sable Morpheus had his curtains spread,\\
And silent night had laid the world to bed,\\
’Mongst other night-birds which did seek for prey,\\
A blunt exciseman, which abhorr’d the day,\\
Was rambling forth to seeke himself a booty\\
’Mongst merchants’ goods which had not paid the duty:\\
But walking all alone, Death chanc’d to meet him,\\
And in this manner did begin to greet him.\\
\vin\vin\vin\vin \textsc{death}.\\
Stand, who comes here? what means this knave to peepe\\
And sculke abroad, when honest men should sleepe?\\
Speake, what’s thy name? and quickly tell me this,\\
Whither thou goest, and what thy bus’ness is?\\
\vin\vin\vin\vin \textsc{exciseman}.\\
Whate’er my bus’ness is, thou foule-monthed scould,\\
I’de have you know I scorn to be coutroul’d\\
By any man that lives; much less by thou,\\
Who blurtest out thou knowst not what, nor how;\\
I goe about my lawful bus’ness; and\\
I’le make you smarte for bidding of mee stand.\\
\vin\vin\vin\vin \textsc{death}.\\
Imperious cox-combe! is your stomach vext?\\
Pray slack your rage, and harken what comes next:\\
I have a writt to take you up; therefore,\\
To chafe your blood, I bid yon stand, once more.
\end{dcverse}
\pagebreak

%%166
%%===============================================================================

\settowidth{\versewidth}{Their falsehood; therefore hold your hand,— give over.}
\begin{dcverse}\footnotesize
\vin\vin\vin\vin \textsc{exciseman}.

A writt to take mee up! excuse mee, sir,\\
You doe mistake, I am an officer\\
In publick service, for my private wealth;\\
My bus’ness is, if any seeke by stealth\\
To undermine the states, I doe discover\\
Their falsehood; therefore hold your hand,— give over.

\vin\vin\vin\vin \textsc{death}.

Nay, fair and soft! ’tis not so quickly done\\
As you conceive it is: I am not gone\\
A jott the sooner, for your hastie chat\\
Nor bragging language; for I tell you flat\\
’Tis more than so, though fortune seeme to thwart us,\\
Such easie terms I don’t intend shall part us.\\
With this impartial arme I’ll make you feele\\
My fingers first, and with this shaft of steele\\
I’le peck thy bones! as thou alive wert hated,\\
So dead, to doggs thou shalt be segregated.

\vin\vin\vin\vin \textsc{exciseman}.

I’de laugh at that; I would thou didst but dare\\
To lay thy fingers on me; I’de not spare\\
To hack thy carkass till my sword was broken,\\
I’de make thee eat the wordes which thou hast spoken;\\
All men should warning take by thy transgression,\\
How they molested men of my profession.\\
My service to the states is so welle known,\\
That I should but complaine, they’d quickly owne\\
My publicke grievances; and give mee right\\
To cut your eares, before to-morrow night.

\vin\vin\vin\vin \textsc{death}.

Well said, indeed! but bootless all, for I\\
Am well acquainted with thy villanie;\\
I know thy office, and thy trade is such,\\
Thy service little, and thy gaines are much:\\
Thy braggs are many; but ’tis vaine to swagger,\\
And thinke to fighte mee with thy guilded dagger:\\
As I abhor thy person, place, and threate,\\
So now I’le bring thee to the judgement seate.

\columnbreak
\vin\vin\vin\vin \textsc{exciseman}.

The judgement seate! I must confess that word\\
Doth cut my heart, like any sharpnèd sword:\\
What! come t’ account! methinks the dreadful sound\\
Of every word doth make a mortal wound,\\
Which sticks not only in my outward skin,\\
But penetrates my very soule within.\\
’Twas least of all my thoughts that ever Death\\
Would once attempt to stop excisemen’s breath.\\
But since ’tis so, that now I doe perceive\\
You are in earnest, then I must relieve\\
Myself another way: come, wee’l be friends,\\
If I have wrongèd thee, I’le make th’ amendes.\\
Let’s joyne together; I’le pass my word this night\\
Shall yield us grub, before the morning light.\\
Or otherwise (to mitigate my sorrow),\\
Stay here, I’le bring you gold enough tomorrow.

\vin\vin\vin\vin \textsc{death}.

To-morrow’s gold I will not have; and thou\\
Shalt have no gold upon to-morrow: now\\
My final writt shall to th’ execution have thee,\\
All earthly treasure cannot help or save thee.

\vin\vin\vin\vin \textsc{exciseman}.

Then woe is mee! ah! how was I befool’d!\\
I thought that gold (which answereth all things) could\\
Have stood my friend at any time to baile mee!\\
But griefe growes great, and now my trust doth faile me.\\
Oh! that my conscience were but clear within,\\
Which now is rackèd with my former sin;\\
With horror I behold my secret stealing,\\
My bribes, oppression, and my graceless dealing;\\
My office-sins, which I had clean forgotten,\\
Will gnaw my soul when all my bones are rotten:\\
I must confess it, very griefe doth force mee,\\
Dead or alive, both God and man doth curse mee,\\
Let all excisemen hereby warning take,\\
To shun their practice for their conscience sake.
\end{dcverse}

Of all the ballads on the subject of Death, the most popular, however, was
\textit{Death and the Lady}. In Mr. George Daniel’s Collection there is a ballad
“imprinted at London by Alexander Lacy” (about 1572), at the end of which
is a still older woodcut, representing \textit{Death and the Lady}. It has been used as
an ornament to fill up a blank in one to which it bears no reference; but was, in
all probability, engraved for this, or one on the same subject. The tune is in
\pagebreak
%%167
%%===============================================================================
Henry Carey’s \textit{Musical Century}, 1738. He calls it “\textit{the old tune} of Death and
the Lady.” Also in \textit{The Cobbler's Opera}, 1729; \textit{The Fashionable Lady}; and
many others about the same date.

The oldest copies of \textit{Aim not too high} direct it to be sung to the tune of \textit{Fortune},
but there is one class of ballads, said to be to the tune of \textit{Aim not too high}, that
could not well be sung to that air. The accent of \textit{Fortune my foe} is on the first
syllable of each line; exactly agreeing with the tune. But these ballads on
Death have the accent on the second, and agree with the tune of \textit{Death and the
Lady}. See, for instance, the four lines above quoted from \textit{The Dialogue between
Death and the rich man}, which the black-letter copies direct to be sung to the
tune of \textit{Aim not too high}. I believe, therefore, that \textit{Aim not too high} had either
a separate tune, which is the same I find under the name of \textit{Death and the Lady},
or else, \textit{Fortune}, being altered by the singer for the accent of those ballads, and
sung in a major key, gradually acquired a different shape. (Many of these airs
are found both in major and minor keys.) This would account for \textit{Fortune} and
\textit{Aim not too high} being so frequently cited as different tunes in ballads printed
about the same period.

I suppose, then, that ballads to the tune of \textit{Aim not too high} may be either
to \textit{Fortune}, or \textit{Death and the Lady}; a point to be determined generally by the
accent of the words.

The ballad of \textit{Death and the, Lady} is printed in a small volume entitled \textit{A Guide
to Heaven}, 12mo., 1736; and it is twice mentioned in Goldsmith’s popular tale,
\textit{The Vicar of Wakefield}, first printed in 1776.

\musicinfo{Slow.}{}
\lilypondfile{lilypond/167-death-and-the-lady}\normalsize
\pagebreak

%%168
%%===============================================================================

\settowidth{\versewidth}{From whence you come, and whither I must go!}
\begin{dcverse}\scriptsize
\vin\vin\vin\vin\vin LADY.

What bold attempt is this? pray let me know\\
From whence you come, and whither I must go!\\
Shall I, who am a lady, stoop or bow\\
To such a pale-fac’d visage? Who art thou?

\vin\vin\vin\vin\vin DEATH

Do you not know me? Well, I’ll tell you, then:\\
’Tis I who conquer all the sons of men!\\
No pitch of honour from my dart is free;\\
My name is Death! have you not heard of me?

\vin\vin\vin\vin\vin LADY.

Yes, I have heard of thee time after time;\\
But, being in the glory of my prime,\\
I did not think you would have called so soon.\\
Why must my sun go down before its noon?

\vin\vin\vin\vin\vin DEATH

Talk not of noon! you may as well be mute;\\
This is no more the time for to dispute:\\
Your riches, jewels, gold, and garments brave—\\
Houses and lands, must all new masters have.\\
Though thy vain heart to riches was inclin’d,\\
Yet thou must die, and leave them all behind.

\vin\vin\vin\vin\vin LADY.

My heart is cold; I tremble at the news!\\
Here’s bags of gold if thou wilt me excuse,\\
And seize on them: and finish thou the strife\\
Of those that are most weary of their life.\\
Are there not many bound in prison strong,\\
In bitter grief of soul have languish’d long?\\
All such would find the grave a place of rest\\
From all the griefs by which they are opprest.\\
Besides, there’s many both with hoary head,\\
And palsied joints, from which all strength is fled.\\
Release thou those, whose sorrows are so great,\\
But spare my life to have a longer date.

\vin\vin\vin\vin\vin DEATH

Though they, by age, are full of grief and pain,\\
Yet their appointed time they must remain.\\
I come to none before their warrant’s seal’d,\\
And when it is, all must submit and yield;\\
I take no bribe, believe me this is true;\\
Prepare yourself, for now I come for you.

\vin\vin\vin\vin\vin LADY.

Be not severe! O Death! let me obtain\\
A little longer time to live and reign!\\
Fain would I stay, if thou my life wilt spare,\\
I have a daughter, beautiful and fair;\\
I’d live to see her wed, whom I adore;\\
Grant me but this, and I will ask no more.

\vin\vin\vin\vin\vin DEATH

This is a slender, frivolous excuse,\\
I have you fast, and will not let you loose;\\
Leave her to Providence, for you must go\\
Along with me, whether you will or no.\\
I, Death, command e’en kings to leave their crown,\\
And at my feet they lay their sceptres down.\\
If unto kings this favour I don’t give,\\
But cut them off, can you expect to live\\
Beyond the limits of your time and space?\\
No I I must send you to another place.

\vin\vin\vin\vin\vin LADY.

You learned doctors, now express your skill,\\
And let not Death of me obtain his will;\\
Prepare your cordials, let me comfort find,\\
And gold shall fly like chaff before the wind!

\vin\vin\vin\vin\vin DEATH

Forbear to call, their skill will never do,\\
They are but mortals here, as well as you;\\
I gave the fatal wound, my dart is sure;\\
’Tis far beyond the doctor’s skill to cure.\\
How freely can you let your riches fly\\
To purchase life, rather than yield to die!\\
But while you flourish’d here in all your store,\\
You would not give one penny to the poor,\\
Who in God’s name their suit to you did make;\\
You would not spare one penny for His sake.\\
The Lord beheld wherein you did amiss,\\
And calls you hence to give account for this.

\vin\vin\vin\vin\vin LADY.

Oh, heavy news! must I no longer stay?\\
How shall I stand at the great judgment day.”\\
Down from her eyes the crystal tears did flow:\\
She said, “None knows what now I undergo.\\
Upon a bed of sorrow here I lie.\\
My carnal life makes me afraid to die;\\
My sins, alas! are many, gross, and foul,\\
Lord Jesus Christ have mercy on my soul!\\
And though I much deserve thy righteous frown.\\
Yet pardon, Lord, and send a blessing down!”

Then, with a dying sigh, her heart did break,\\
And she the pleasures of this world forsake.\\
Thus do we see the high and mighty fall,\\
For cruel death shows not respect at all\\
To any one of high or low degree:\\
Great men submit to death, as well as we.\\
If old or young, our life is but a span—\\
A lump of clay—so vile a creature’s man.\\
Then happy they whom Christ has made his care—\\
Die in the Lord, and ever blessed are!
\end{dcverse}

\pagebreak
%%169
%%===============================================================================
\musictitle{The King And The Miller Of Mansfield.}

This tune was found by Dr.~Rimbault in a MS. volume of virginal music in the
possession of T. Birch, Esq., of Repton, Derbyshire. The black-letter copies of
the ballad of \textit{King Henry II. and the Miller of Mansfield}, direct it to be sung to
the tune of \textit{The French Levalto}, and, as the air was found under that name, it
\textit{may} be a French tune, although neither Dr.~Rimbault nor I think it so. The
progression of the last four notes in each part is very English in character.

There are copies of the ballad in the Roxburghe Collection (v. i. 178 and 228);
in the Bagford (p.~25); and in the Pepys. It is also in \textit{Old Ballads}, 1727,
v. i., p.~53; and in Percy’s \textit{Reliques}, series 3, book ii. \textit{The French Levalto} is
frequently referred to as a ballad tune.

\musicinfo{Rather slow and gracefully.}{}
\lilypondfile[staffsize=16]{lilypond/169-the-king-and-the-miller-of-mansfield}\normalsize

\pagebreak

%%170
%%===============================================================================
\settowidth{\versewidth}{All a long summer’s day rode the king pleasantlye,}
\indentpattern{010100}
\begin{scverse}
\begin{patverse}
All a long summer’s day rode the king pleasantlye,\\
With all his princes and nobles eche one;\\
Chasing the hart and hind, and the bucke gallantlye,\\
Till the darke evening forc’d all to turne home.\\
Then at last, riding fast, he had lost quite\\
All his lords in the wood, late in the night.
\end{patverse}

\begin{patverse}
Wandering thus wearilye, all alone, up and downe,\\
With a rude miller he mett at the last:\\
Asking the ready way unto faire Nottingham;\\
Sir, quoth the miller, I meane not to jest,\\
Yet I thinke, what I thinke, sooth for to say,\\
You doe not lightlye ride out of your way.
\end{patverse}

\begin{patverse}
Why, what dost thou thinke of me, quoth our king merrily\\
Passing thy judgment upon me so briefe?\\
Good faith, sayd the miller, I meane not to flatter thee;\\
I guess thee to be but some gentleman thiefe;\\
Stand thee backe, in the darke; light not adowne,\\
Lest that I presentlye crack thy knaves crowne. \&c.
\end{patverse}
\end{scverse}

\musictitle{Little Musgrave and Lady Barnard.}

This ballad is quoted in Fletcher’s \textit{Knight of the Burning Pestle}, and \textit{Monsieur
Thomas}; in \textit{The Varietie}, 1649; and in Davenant’s \textit{The Wits}, where Twack, an
antiquated beau, boasting of his qualifications, says—

“Besides, I sing \textit{Little Musgrove}; and then
For \textit{Chevy Chase} no lark comes near me.”

A copy of the ballad is in the Bagford Collection, entitled “A lamentable
ballad of Little Musgrove and the Lady Barnet, to an excellent new tune.” It is
also in \textit{Wit restored}, 1658; in Dryden’s \textit{Miscellany Poems}, iii. 312 (1716); and
in Percy’s \textit{Reliques}, series 3, book i.

The tune is the usual traditional version.

\musicinfo{Gracefully.}{}
\lilypondfile{lilypond/170-little-musgrave-and-lady-barnard}\normalsize

\pagebreak


%%171
%%===============================================================================
\musictitle{The Gipsies’ Round.}

The tune from Queen Elizabeth’s Virginal Book.

Whenever gipsies are introduced in old plays, we find some allusions to their
singing, dancing, or music, and generally a variety of songs to be sung by them.
In Middleton’s \textit{Spanish Gipsy}, Roderigo, being invited to turn gipsy, says—
\settowidth{\versewidth}{I can neither dance, nor sing; but if my pen}
\begin{scverse}
\vleftofline{“}I can neither dance, nor sing; but if my pen\\
From my invention can strike music tunes,\\
My head and brains are yours.”
\end{scverse}
In other words, “I think I can invent tunes, and therefore have one qualification
for a gipsy, although I cannot dance, nor sing.”

By \textit{Round} is here meant a country dance. Country dances were formerly danced
quite as much in rounds as in parallel lines; and in the reign of Elizabeth were
in favour at court, as well as at the May-pole. In the Talbot papers, Herald’s
College, is a letter from the Earl of Worcester to the Earl of Shrewsbury, dated
Sep.~19th, 1602, in which he says, “We are frolic here in court; much dancing
in the privy chamber of country dances before the Queen’s Majesty, who is
much pleased therewith.”—(Lodge, iii. 577.)

\musicinfo{Boldly.}{}
\lilypondfile{lilypond/171-the-gipsies-round}\normalsize

\musictitle{The Legend Of Sir Guy.}

This ballad was entered to Richard Jones on Jan. 5th, 1591-2, as “A plesante
songe of the valiant actes of Guy of Warwicke, to the tune of \textit{Was ever man so
lost in love}.” The copy in the Bagford Collection (p.~19) is entitled “A pleasant
song of the valiant deeds of chivalry achieved by that noble knight, Sir Guy of
Warwick, who, for the love of fair Phillis, became a hermit, and died in a cave of
\pagebreak
%%172
%%===============================================================================
a craggy rock, a mile distant from Warwick. Tune, \textit{Was ever man}, \&c.” Other
copies are in the Pepys Collection; Roxburghe, iii. 50; and in Percy’s \textit{Reliques},
series 3, book~ii.

It is quoted in Fletcher’s \textit{Knight of the Burning Pestle}, act ii., sc. 8; and in
\textit{The little French Lawyer}, act ii., sc. 3.

William of Nassyngton (about 1480) mentions stories of Sir Guy as usually
sung by minstrels at feasts. (See ante page 45.) Puttenham, in his \textit{Art of
Poetry}, 1589, says they were then commonly sung to the harp at Christmas
dinners and bride-ales, for the recreation of the lower classes. And in Dr.~King’s
\textit{Dialogues of the Dead}, “It is the negligence of our ballad singers that makes us to
be talked of less than others: for who, almost, besides \textit{St. George, King Arthur,
Bevis, Guy}, and \textit{Hickathrift}, are,in the chronicles.”—(Vol. i., p.~153.)

This tune is from the ballad-opera of \textit{Robin Hood}, 1730, called \textit{Sir Guy}.

\musicinfo{Slow.}{}
\lilypondfile{lilypond/172-the-legend-of-sir-guy}\normalsize

\pagebreak

%%173
%%===============================================================================
\musictitle{Loth To Depart.}

Tune from Queen Elizabeth’s Virginal Book, where it is arranged by Giles
Farnaby.

In Beaumont and Fletcher’s \textit{Wit at several Weapons}, act ii., sc. 2, Pompey
makes his exit singing \textit{Loath to depart}. In Middleton’s \textit{The Old Law}, act iv.,
sc. 1, “The old woman is \textit{loath to depart}; she never sung other tune in her life.”
In the ballad of \textit{Arthur of Bradley}, which exists in black-letter, and in the \textit{Antidote
to Melancholy}, 1661, are the following lines:—
\settowidth{\versewidth}{Then Will and his sweetheart}
\begin{scverse}
\vleftofline{“}Then Will and his sweetheart\\
Did call for \textit{Loth to depart.}”
\end{scverse}
Also in Chapman’s \textit{Widow’s Tears}, 1612; \textit{Vox Borealis}, 1641; and many others.

A \textit{Loth to depart} was the common term for a song sung, or a tune played, on
taking leave of friends. So in a \textit{Discourse on Marine Affairs} (Harl. MSS.,
No.~1341) we find, “Being again returned into his barge, after that the trumpets
have sounded a \textit{Loathe to departe}, and the barge is fallen off a fit and fair birth
and distance from the ship-side, he is to be saluted with so many guns, for an
adieu, as the ship is able to give, provided that they be always of an odd
number.”—(Quoted in a note to Teonge’s Diary, p.~5.) In Tarlton’s \textit{News out of
Purgatory}, (about 1589), “And so, with a \textit{Loath to depart}, they took their
leaves;” and in the old play of \textit{Damon and Pithias}, when Damon takes leave,
saying, “Loth am I to depart,” he adds, “O Music, sound my doleful plaints
when I am gone away,” and the regals play “a mourning song.”

The following are the words of a round in \textit{Deuteromelia}, 1609:—
\settowidth{\versewidth}{Sing with thy mouth, sing with thy heart.}
\begin{scverse}
“Sing with thy mouth, sing with thy heart.\\
Like faithful friends, sing \textit{Loath to depart};\\
Though friends together may not always remain.\\
Yet \textit{Loath to depart} sing once again.”
\end{scverse}

The four lines here printed to the tune, are part of a song called “Loth to
depart,” in \textit{Wit’s Interpreter}, 1671. It is also in \textit{The Loyal Garland}; and, with
some alteration, in Dryden’s \textit{Miscellany Poems}, iv., 80. It is there attributed to
Mr. J. Donne.

\musicinfo{Slow.}{}
\lilypondfile{lilypond/173-loth-to-depart}\normalsize

\pagebreak

%%174
%%===============================================================================
\musictitle{Queen Eleanor’s Confession.}

This is the traditional tune to the ballad which is printed in Percy’s \textit{Reliques
of Ancient Poetry} (No. 8, series ii., book 2). A copy is in the Bagford Collection,
i.~26, to be sung to “a pleasant new tune.”

\musicinfo{Moderate time.}{}
\lilypondfile{lilypond/174-queen-eleanors-confession}\normalsize

\musictitle{Essex’s Last Good-Night, or Well-A-Day.}

This air is contained in Elizabeth Rogers’ MS. Virginal Book (Brit. Mus.);
and in a transcript of virginal music made by Sir John Hawkins, now in the possession
of Dr.~Rimbault. In the former it is entitled \textit{Essex’s last Good-night}, and
there are but eight bars in the tune; the latter is called \textit{Well-a-day}, and consists
of sixteen bars.

The ballad of \textit{Essex's last Good-night} is in the Pepys Collection, i. 106; and
Roxburghe, i. 101, and 185. In the Pepys Collection it is called “A lamentable
new ballad upon the Earl of Essex his death; to the tune of \textit{The King's last
Good-night}.” In the Roxburghe, i. 101, to the tune of \textit{Essex's last Good-night}.
It is printed in Evans’ \textit{Old Ballads}, iii. 167 (1810); but, as usual, without the
name of the tune. The first verse of the Pepys copy is as follows:—
\settowidth{\versewidth}{All you that cry O hone, O hone! [alas],}
\begin{scverse}
\begin{altverse}
\vleftofline{“}All you that cry O hone, O hone! [alas],\\
Come now and sing O Lord with me;\\
For why our jewel is from us gone,\\
The valiant knight of chivalry.\\
Of rich and poor belov’d was he;\\
In time an honorable knight;\\
When by our laws condemn’d was he.\\
And lately took his last \textit{Good-night}.”
\end{altverse}
\end{scverse}

This is on the death of Walter Devereux, Earl of Essex (father of Queen Elizabeth’s
favorite), who died in Dublin, in 1576. Another on the same subject, and
in the same metre, has been printed by Mr. Payne Collier, in his \textit{Extracts from
the Registers of the Stationers' Company}, ii. 35; beginning thus:—

\pagebreak

%%175
%%===============================================================================
%
\DFNdouble

\settowidth{\versewidth}{Lament, each subject, and the head}
\begin{scverse}
\begin{altverse}
\vleftofline{“}Lament, lament, for he is dead\\
Who serv’d his prince most faithfully;\\
Lament, each subject, and the head\\
Of this our realm of Brittany.\\
Our Queen has lost a soldier true;\\
Her subjects lost a noble friend:\\
Oft for his queen his sword he drew,\\
And for her subjects blood did spend,” \&c.
\end{altverse}
\end{scverse}

The ballad of \textit{Well-a-day} is entitled “A lamentable dittie composed upon the
death of \textit{Robert} Lord Devereux, late Earle of Essex, who was beheaded in the
Tower of London, upon Ash Wednesday, in the morning, 1601. To the tune of
\textit{Well-a-day}. Imprinted at London for Margret Allde, \&c., 1603. Reprinted in
Payne Collier’s \textit{Old Ballads}, 124, 8vo., 1840; and in Evans’, iii. 158. Copies
are also in the Bagford and Roxburghe Collections (i. 184); and Harl. MSS.,
293. The first verse is here given with the tune.

The ballads to the tune of \textit{Essex's last Good-night} are in quite a different metre
to those which were to be sung to \textit{Well-a-day}; and either the melody consisted
originally of but eight bars, and those bars were repeated for the last four lines
of each stanza, or else the second part differed from my copy.

\textit{Well-a-day} seems to be older than the date of the death of either Earl, because,
in 1566-7, Mr. Wally had a license to print “the second Well-a-day” (\textit{Ex. Reg.
Stat}., i. 151.); and, in 1569-70, Thomas Colwell, to print “A new Well-a-day,
As plain, Mr. Papist, as Dunstable way.”

To “sing well-away” was proverbial even in Chaucer’s time; for in the prologue
to the Wife of Bath’s Tale, speaking of her husbands, she says (lines
5597-600)
\settowidth{\versewidth}{I sette [t]hem so on werke, by my fay!}
\begin{scverse}
\vleftofline{“}I sette [t]hem so on werke, by my fay!\\
That many a night thay \textit{songen weylaway}.\\
The bacoun was nought fet for hem, I trowe,\\
That som men fecche in Essex at Dunmowe.” \dcfootnote{ %a
The claiming the Flitch of Bacon at Dunmow was
a custom to which frequent allusions are made in the
fourteenth and fifteenth centuries. See also a song in
\textit{Reliquiæ Antiquæ}, ii.~29.
} %end footnote
\end{scverse}
And in the Shipman’s Tale, “For I may synge allas and waylaway that I was
born.” So in the \textit{Owl and the Nightingale}, one of our earliest original poems, the
owl says to the nightingale—
\settowidth{\versewidth}{Thu singest a night, and noght a dai,}
\begin{scverse}
\vleftofline{“}Thu singest a night, and noght a dai,\\
And al thi song is wail awai.”
\end{scverse}
In the sixteenth century we find a similar passage in Nicholas Breton’s \textit{Farewell
to town}—
\begin{scverse}
\vleftofline{“}I must, ah me! wretch, as I may,\\
Go sing the song of \textit{Welaway}.”
\end{scverse}

The ballads sung to one or other of these tunes are very numerous. Among
them are—

“Sir Walter Rauleigh his Lamentation,” \&c., “to the tune of \textit{Well-a-day}.
Pepys Collection, i. 111, \textsc{b. l.}

“The arraignment of the Devil for stealing away President Bradshaw.” Tune,
\textit{Well-a-day, well-a-day}. (King’s Pamphlets, vol. 15, or Wright’s \textit{Political
Ballads}, 139.)

\pagebreak

%%176
%%===============================================================================
“The story of Ill May-day, \&c., and how Queen Catherine begged the lives of
2,000 London apprentices,” Tune \textit{Essex's Good-night}. (\textit{Crown Garland of
Golden Roses}, or Evans, iii. 76.)

“The doleful death of Queen Jane, wife of Henry VIII.,” \&c. “Tune,
\textit{The Lamentation of the Lord of Essex}.” (\textit{Crown Garland}, or Evans, iii. 92.)

A Carol, to the tune of Essex’s last Good-night, dated 1661. (Wright’s
Carols.)—
\settowidth{\versewidth}{All you that in this house be here,}
\begin{scverse}\begin{altverse}
\vleftofline{“}All you that in this house be here,\\
Remember Christ that for us died;\\
And spend away with modest cheer,\\
In loving sort this Christmas-tide,” \&c.
\end{altverse}
\end{scverse}

Several other tunes were named after the Earl of Essex. In Dr.~Camphuysen’s
\textit{Stichtelyche Rymen} (4to., Amsterdam, 1647) is one called \textit{Essex's Galliard}, and
another \textit{Essex's Lamentation}. The last is the same air as \textit{What if a day, or a
month, or a year}.

In \textit{The World's Folly} (\textsc{b.l.}) a widow “would sing \textit{The Lamentation of a Sinner},
to the tune of \textit{Well-a-daye}."


\musicinfo{Slow.}{}
\lilypondfile{lilypond/176-essexs-last-good-night-or-well-a-day}\normalsize

\musictitle{The Fit’s Come On Me Now.}

This song is quoted by Valentine in Beaumont and Fletcher’s \textit{Wit without
money}, act v., sc. 4., where a verse is printed.

One of my friends recollects his nurse singing a ballad with the burden—
\settowidth{\versewidth}{I must and will get married,}
\begin{scverse}
\vleftofline{“}I must and will get married,\\
The fit’s upon me now.”
\end{scverse}

\pagebreak

%%177
%%===============================================================================
The tune is from the seventh edition of \textit{The Dancing Master}. In some later,
editions it is called \textit{The Bishop of Chester’s Jig, or The fit’s come on me now}.

\musicinfo{Cheerfully.}{}
\lilypondfile{lilypond/177-the-fits-come-on-me-now}\normalsize

\musictitle{Mall Sims.}

This favorite old dance tune is in Queen Elizabeth’s Virginal Book; in Morley’s
\textit{Consort Lessons}, 1599 and 1611; in Rossiter’s \textit{Consort Lessons}, 1609; in Vallet’s
\textit{Tablature de Luth, intitulé Le Secret des Muses}, book i., 4to., Amsterdam, 1615,
entitled “Bal Anglois, Mal Simmes;” also in the second book of the same work,
1616; in \textit{Nederlandtsche Gedenck-Clank}, 1626; in Camphuysen’s \textit{Stichtelycke
Rymen}, 1647 (called “The English Echo, or Malsims”); in the Skene MS., \&c.

It is most likely one of the old harpers’ tunes, as it has quite the character of
harp music. In Rossiter’s \textit{Consort Lessons}, 1609, in which the names of the composers
are given to every other air, this is marked \textit{Incertus}: and if unknown
then, it is probably much older than the date of the book.

In \textit{Wit Restor’d}, 1658, is the ballad of “The Miller and the King’s Daughters,”
written by Dr.~James Smith, in which this tune is mentioned:
\settowidth{\versewidth}{What did he doe with her two shinnes?}
\begin{scverse}
\vleftofline{“}What did he doe with her two shinnes?\\
Unto the violl they danc’t \textit{Moll Syms}.”
\end{scverse}

\pagebreak

%%178
%%===============================================================================
%
\musicinfo{Pompously.}{}
\lilypondfile{lilypond/178-mall-sims}\normalsize

\musictitle{Crimson Velvet.}

This tune is found in one of the Dutch collections, \textit{Friesche Lust-Hof}, by Jan
Jansz Starter, the edition printed at Amsterdam in 1634. It is called “’Twas a
youthful Knight, which loved a galjant Lady,” which is the first line of the
ballad of “Constance of Cleveland: to the tune of \textit{Crimson Velvet}.” The
ballad is in the Roxburghe Collection, iii. 94, and in Collier’s \textit{Roxburghe
Ballads}, p.~163.

The tune of \textit{Crimson Velvet} was, as Mr. Collier remarks, “highly popular in
the reigns of Elizabeth and her successor.” Among the ballads that were sung
to it, are “The lamentable complaint of Queen Mary, for the unkind departure of
King Philip, in whose absence she fell sick and died;” beginning—
\settowidth{\versewidth}{Mary doth complain.}
\begin{scverse}\vleftofline{“}Mary doth complain.\\
Ladies, be you moved\\
With my lamentations\\
And my bitter groans,” \&c.
\end{scverse}
A copy in the \textit{Crown Garland of Golden Roses} (reprint of edit. of 1659, p.~69).

“An excellent ballad of a prince of England’s courtship to the King of
\pagebreak
%%179
%%===============================================================================
France’s daughter, and how the prince was disasterously slain; and how the
aforesaid princess was afterwards married to a forrester;” commencing—
\settowidth{\versewidth}{When fair France did flourish,”}
\begin{scverse}
\vleftofline{“}In the days of old,\\
When fair France did flourish,” \&c.
\end{scverse}
Copies in the Roxburghe Collection, i. 102, the Bagford, the Pepys, Deloney’s
\textit{Garland of good-will}, and Percy’s \textit{Reliques}, series iii., book 2, 16.

The following is the ballad of “Constance of Cleveland.”

\musicinfo{Slow.}{}
\lilypondfile{lilypond/179-crimson-velvet}\normalsize

\pagebreak


%%180
%%===============================================================================
\settowidth{\versewidth}{Thou shalt me in thy arms enclose;}
\begin{dcverse}
\indentpattern{01020102110110110110}
\begin{patverse}
His fair lady’s words\\
Nothing he regarded;\\
Wantonness affords,\\
To some, delightful sport;\\
While they dance and sing,\\
With great mirth prepared,\\
She her hands did wring\\
In most grievous sort.\\
Oh! what hap had I,\\
Thus to wail and cry,\\
Unrespected every day.\\
Living in disdain,\\
While that others gain\\
All the right I should enjoy!\\
I am left forsaken,\\
Others they are taken;\\
Ah! my love why dost thou so?\\
Her flatteries believe not,\\
Come to me and grieve not;\\
Wantons will thee overthrow.
\end{patverse}

\begin{patverse}
The knight, with his fair piece,\\
At length the lady spied,\\
Who did him daily fleece\\
Of his wealth and store;\\
Secretly she stood,\\
While she her fashions tryed\\
With a patient mind;\\
While deep the strumpet swore:\\
\vleftofline{“}O sir knight,” quoth she,\\
\vleftofline{“}So dearly I love thee,\\
My life doth rest at thy dispose.\\
By day, and eke by night,\\
For thy sweet delight\\
Thou shalt me in thy arms enclose;\\
I am thine for ever,\\
Still I will persever,\\
True to thee where’er I go.”\\
Her flatteries believe not,\\
Come to me and grieve not;\\
Wantons will thee overthrow.
\end{patverse}

\begin{patverse}
The virtuous lady mild\\
Enters then among them,\\
Being big with child\\
As ever she might be;\\
With distilling tears\\
She looked then upon them,\\
Filled full of fears,\\
Thus replyed she:\\
\vleftofline{“}Ah, my love and dear,\\
Wherefore stay you here,\\
Refusing me, your loving wife,\\
For an harlot’s sake,\\
Which each one will take;\\
Whose vile deeds provoke much strife.\\
Many can accuse her,\\
O, my love, refuse her,\\
With thy lady home return;\\
Her flatteries believe not,\\
Come to me and grieve not;\\
Wantons will thee overthrow.”
\end{patverse}

\begin{patverse}
All in a fury then\\
The angry knight upstarted,\\
Very furious when\\
He heard his lady’s speech;\\
With many bitter terms\\
His wife he ever thwarted,\\
Using hard extremes\\
While she did him beseech.\\
From her neck so white\\
He took away in spite\\
Her curious chain of purest gold:\\
Her jewels and her rings,\\
And all such costly things,\\
As he about her did behold;\\
The harlot, in her presence,\\
He did gently reverence,\\
And to her he gave them all.\\
He sent away his lady,\\
Full of woe as may be,\\
Who in a swound with grief did fall.
\end{patverse}

\begin{patverse}
At the lady’s wrong\\
The harlot fleer’d and laughed;\\
Enticements are so strong,\\
They overcome the wise:\\
The knight nothing regarded\\
To see the lady scoffed;\\
Thus she was rewarded\\
For her enterprise.\\
The harlot all this space\\
Did him oft embrace;\\
She flatters him, and thus doth say:\\
\vleftofline{“}For thee I’ll die and live,\\
For thee my faith I’ll give,\\
No woe shall work my love’s decay;\\
Thou shalt be my treasure,\\
Thou shalt be my pleasure,\\
Thou shalt be my heart’s delight;\\
I will be thy darling,\\
I will be thy worldling,\\
In despite of fortune’s spite.”
\end{patverse}
\end{dcverse}

\pagebreak

%%181
%%===============================================================================
%
\settowidth{\versewidth}{He bade from thence they should her take.}
\indentpattern{01010101220220220220}
\begin{dcverse}\begin{patverse}
Thus did he remain\\
In wasteful great expences,\\
Till it bred his pain,\\
And consum’d him quite.\\
When his lands were spent,\\
Troubled in his senses,\\
Then he did repent\\
Of his late lewd life;\\
For relief he hies,\\
For relief he flies\\
To them on whom he spent his gold;\\
They do him deny,\\
They do him defy,\\
They will not once his face behold.\\
Being thus distressed,\\
Being thus oppressed,\\
In the fields that night he lay;\\
Which the harlot knowing,\\
Through her malice growing,\\
Sought to take his life away.
\end{patverse}

\indentpattern{01020102110110110110}
\begin{patverse}
A young and proper lad\\
They had slain in secret\\
For the gold he had;\\
Whom they did convey,\\
By a ruffian lewd,\\
To that place directly,\\
Where the youthful knight\\
Fast a sleeping lay;\\
The bloody dagger, then,\\
Wherewith they kill’d the man,\\
Hard by the knight he likewise laid;\\
Sprinkling him with blood,\\
As he thought it good,\\
And then no longer there he stay’d.\\
The knight, being so abused,\\
Was forthwith accused\\
For this murder which was done;\\
And he was condemned\\
That had not offended,\\
Shameful death he might not shun.
\end{patverse}

\begin{patverse}
When the lady bright\\
Understood the matter,\\
That her wedded knight\\
Was condemned to die,\\
To the king she went\\
With all the speed that might be.\\
Where she did lament\\
Her hard destiny.\\
“Noble king,” quoth she,\\
“Pity take on me,\\
And pardon my poor husband’s life;\\
Else I am undone,\\
With my little son,\\
Let mercy mitigate this grief.”\\
\vleftofline{“}Lady fair, content thee,\\
Soon thou wouldst repent thee\\
If he should be saved so;\\
Sore he hath abus’d thee,\\
Sore he hath misus’d thee,\\
Therefore, lady, let him go.”
\end{patverse}

\begin{patverse}
\vleftofline{“}O, my liege,” quoth she,\\
\vleftofline{“}Grant your gracious favour;\\
Dear he is to me,\\
Though he did me wrong.”\\
The king replied again,\\
With a stern behaviour,\\
“A subject he hath slain,\\
Die, he shall, ere long:\\
Except thou canst find\\
Any one so kind\\
That will die and set him free.”\\
\vleftofline{“}Noble king,” she said,\\
\vleftofline{“}Glad am I apaid,\\
That same person will I be.\\
I will suffer duly,\\
I will suffer truly,\\
For my love and husband’s sake.”\\
The king thereat amazed,\\
Though he her beauty praised, \\
He bade from thence they should her take.
\end{patverse}

\begin{patverse}
It was the king’s command,\\
On the morrow after,\\
She should out of hand\\
To the scaffold go;\\
Her husband was\\
To bear the sword before her;\\
He must, eke alas!\\
Give the deadly blow.\\
He refus’d the deed,\\
She bade him to proceed\\
With a thousand kisses sweet.\\
In this woeful case\\
They did both embrace;\\
Which mov’d the ruffians in that place\\
Straight for to discover\\
This concealed murder;\\
Whereby the lady saved was.\\
The harlot then was hanged,\\
As she well deserved:\\
This did virtue bring to pass.
\end{patverse}

\end{dcverse}

\pagebreak

%%182
%%===============================================================================
\musictitle{Walking In A Country Town.}

The tune from Robinson’s \textit{Schoole of Musicke}, 1603, called \textit{Walking in a
country town}. In the Roxburghe Collection, i. 412, is a ballad beginning
“Walking in a meadow green,” and, from the similarity of the lines, and the
measure of the verse so exactly suiting the air, I infer this to be the tune of both.
The latter was printed by John Trundle, at the sign of the Nobody in Barbican,
rendered famous by Ben Jonson, who in his \textit{Every man in his Humour}, makes
Knowell say, “Well, if he read this with patience, I’ll ‘go,’ and troll ballads for
Master John Trundle yonder, the rest of my mortality.”

It is entitled “The two Leicestershire Lovers: to the tune of \textit{And yet methinks
I love thee}.” The first stanza is here printed to the music.

The last line of the verse is, “Upon the meadow brow,” and \textit{The meadow brow}
is often quoted as a tune. So in the Roxburghe Collection, i. 92, or Colliers’s
Roxburghe Ballads, p.~1, is “Death’s Dance” (beginning, “If Death would come
and shew his face”), “to be sung to a pleasant new tune called \textit{O no, no, no, not
yet}, or \textit{The meadow brow}.” And Bishop Corbet’s song, “Farewell, rewards and
fairies,” is “to be sung or whistled to the tune \textit{The meddow brow} by the learned:
by the unlearned, to the tune of \textit{Fortune}.”—(Percy, series iii., book 2.) All
might be sung to this tune.

\musicinfo{Slow.}{}
\lilypondfile{lilypond/182-walking-in-a-country-town}\normalsize

\musictitle{Phillida Flouts Me.}

In \textit{The Crown Garland of Golden Roses}, 1612, is “A short and sweet sonnet
made by one of the Maides of Honor upon the death of Queene Elizabeth, which
she sowed upon a sampler, in red silke: to a new tune, or \textit{Phillida flouts me};”
beginning—
\settowidth{\versewidth}{Whom we have lov’d so dear,” }
\begin{scverse}
\vleftofline{“}Gone is Elizabeth,\\
Whom we have lov’d so dear,” \&c.
\end{scverse}

\pagebreak

%%183
%%===============================================================================
Patrick Carey also wrote a ballad to the tune of \textit{Phillida flouts me}; beginning—
\settowidth{\versewidth}{Ned! she that likes thee now,}
\begin{scverse}
\vleftofline{“}Ned! she that likes thee now,\\
Next week will leave thee!”
\end{scverse}
It is contained in his “Trivial Poems and Triolets, written in obedience to
Mrs. Tomkin’s commands, 20th August, 1651.” In Walton’s \textit{Angler}, 1653, the
Milkwoman asks, “What song was it, I pray? Was it \textit{Come, shepherds, deck
your heads}, or \textit{As at noon Dulcina rested}, or \textit{Phillida flouts me}?”

The ballad of \textit{Phillida flouts me} is in the Roxburghe Collection, ii. 142, and in
the same volume, p.~24, “The Bashful Virgin, or The Secret Lover: tune of
\textit{I am so deep in love}, or \textit{Little boy}, \&c.” It begins—
\settowidth{\versewidth}{O what a plague it is}
\begin{scverse}
\begin{altverse}
\vleftofline{“}O what a plague it is\\
To be a lover;\\
Being denied the bliss\\
For to discover,” \&c.
\end{altverse}
\end{scverse}

This appears to be also to the air of \textit{Phillida flouts me}, although the first line of
that ballad is “Oh! what a plague is love,” not “I am so deep in love.”

The words and music are in Watts’ \textit{Musical Miscellany}, ii. 132 (1729), and an
answer, beginning, “O where’s the plague in love.” The tune is also in many of
the ballad-operas, such as \textit{The Quaker’s Opera}, 1728; \textit{Love in a Riddle}, 1729;
\textit{Damon and Phillida}, 1734, \&c.

Ritson printed the words in his \textit{Ancient Songs}, from a copy in \textit{The Theatre of
Compliments, or New Academy}, 1689, but did not discover the tune.

\musicinfo{Slowly and gracefully.}{}
\lilypondfile{lilypond/183-phillida-flouts-me}\normalsize

\pagebreak

%184
%%===============================================================================
\settowidth{\versewidth}{Thou shalt eat curds ond cream}
\indentpattern{010101010003}
\begin{dcverse}\footnotesize
\begin{patverse}
At the fair t’other day,\\
As she pass’d by me,\\
She look’d another way,\\
And would not spy me.\\
I woo’d her for to dine,\\
But could not get her;\\
Dick had her to the Vine,\\
He might intreat her.\\
With Daniel she did dance,\\
On me she would not glance;\\
Oh, thrice unhappy chance!\\
Phillida flouts me.
\end{patverse}

\begin{patverse}
Fair maid, be not so coy,\\
Do not disdain me;\\
I am my mother’s joy;\\
Sweet, entertain me.\\
I shall have, when she dies,\\
All things that’s fitting;\\
Her poultry and her bees,\\
And her goose sitting;\\
A pair of mattrass beds,\\
A barrel full of shreds:\\
And yet, for all these goods,\\
Phillida flouts me.
\end{patverse}

\begin{patverse}
I often heard her say,\\
That she lov’d posies;\\
In the last month of May\\
I gave her roses,\\
Cowslips and gilly-flowers,\\
And the sweet lily,\\
I got to deck the bow’rs\\
Of my dear Philly.\\
She did them all disdain,\\
And threw them back again;\\
Therefore ’tis flat and plain\\
Phillida flouts me.
\end{patverse}

\begin{patverse}
Thou shalt eat curds ond cream\\
All the year lasting,\\
And drink the crystal stream,\\
Pleasant in tasting:\\
Swig whey until you burst,\\
Eat bramble-berries,\\
Pye-lid, and pastry-crust,\\
Pears, plums, and cherries;\\
Thy garments shall he thin,\\
Made of a wether’s skin;\\
Yet all’s not worth a pin:\\
Phillida flouts me.
\end{patverse}

\begin{patverse}
Which way soe’er I go,\\
She still torments me;\\
And, whatsoe’er I do,\\
Nothing contents me:\\
I fade, and pine away\\
With grief and sorrow;\\
I fall quite to decay,\\
Like any shadow;\\
I shall be dead, I fear,\\
Within a thousand year,\\
And all because my dear\\
Phillida flouts me.
\end{patverse}

\begin{patverse}
Fair maiden, have a care,\\
And in time take me;\\
I can have those as fair,\\
If you forsake me;\\
There’s Doll, the dairy-maid,\\
Smil’d on me lately,\\
And wanton Winifred\\
Favours me greatly;\\
One throws milk on my clothes,\\
T’other plays with my nose;\\
What pretty toys are those!\\
Phillida flouts me.
\end{patverse}

\begin{patverse}
She has a cloth of mine,\\
Wrought with blue Coventry,\\
Which she keeps as a sign\\
Of my fidelity:\\
But if she frowns on me,\\
She shall ne’er wear it;\\
I’ll give it my maid Joan,\\
And she shall tear it.\\
Since ’twill no better be,\\
I’ll bear it patiently;\\
Yet, all the world may see,\\
Phillida flouts me.
\end{patverse}
\end{dcverse}

\musictitle{Lady, Lie Near Me.}

This ballad is entitled “The longing Shepherdess, or Lady” [Laddy] “lie
near me.” Copies are in the Pepys Collection, iii., 59, and Douce, p.~119, \&c.
It is also in the list of ballads that were printed by W. Thackeray, at the Angel,
in Duck Lane.

The tune (which bears a strong resemblance to \textit{Phillida flouts me}) is in \textit{The
Dancing Master}, from the first edition in 1650, to the eighth in 1690.


\pagebreak
%%185
%%===============================================================================
In Ritson’s \textit{North Country Chorister} there is another ballad, called “Laddy, lie
near me” (beginning, “As I walked! over hills, dales, and high mountains”); and
in 1793 Mr. George Thomson gave Burns a tune of that name, to write words to,
which is now included in Scotch Collections. It differs wholly from this.

\musicinfo{Slowly and gracefully.}{}
\lilypondfile{lilypond/185-lady-lie-near-me}\normalsize

\musictitle{Mill-field.}

In the collection of ballads and proclamations in the library of the Society of
Antiquaries is one by W. Elderton, entitled “A new ballad, declaring the great
treason conspired against the young King of Scots, and how one Andrew Browne,
an Englishman, which was the King’s Chamberlaine, prevented the same. To the
tune of \textit{Milfield}, or els to \textit{Greene sleeves}.” It was printed by “Yarathe James,”
to whom it was licensed on 30th May, 1581.

The tune is in \textit{The Dancing Master} from 1650 to 1658. The ballad in Percy’s
\textit{Reliques}, series ii., book 2, No. 16. The first stanza is here with the music.

\pagebreak

%%186
%%===============================================================================
\musicinfo{Gracefully.}{}
\lilypondfile{lilypond/186-mill-field}\normalsize

\musictitle{The Spanish Lady.}

Dr.~Percy says, “this beautiful old ballad most probably took its rise from one
of those descents made on the Spanish coasts in the time of Queen Elizabeth:
and, in all likelihood, from the taking of the city of Cadiz (called by our sailors,
corruptly, Cales), on June 21, 1596, under the command of the Lord Howard,
admiral, and of the Earl of Essex, general.”

The question as to who was the favored lover, has been fully discussed; it may
therefore be sufficient here to refer the reader to \textit{The Edinburgh Review} for April,
1846; \textit{The Times} newspapers of April 30, and May 1, 1846; and \textit{The Quarterly
Review} for October, 1846.

The ballad is quoted in \textit{Cupid’s Whirligig}, 1616, and parodied in Rowley’s
\textit{A Match at Midnight}, 1633. In the Douce Collection, ii. 210 and 212, there
are two copies, the one “to a pleasant new tune;” the other (which is of later
date) to the tune of \textit{Flying Fame}; but \textit{could no}t he sung to that air. In the
same volume, p.~254, is “The Westminster Wedding, or Carlton’s Epithalamium,”
(dated 1663): to the tune of \textit{The Spanish Lady}. It commences thus;

\settowidth{\versewidth}{Will you hear a German Princess,}
\begin{scverse}
\vleftofline{“}Will you hear a German Princess,\\
How she chous’d an English Lord,” \&c.
\end{scverse}

\pagebreak

%%187
%%===============================================================================
The tune is contained in the Skene MS., and in several of the ballad-operas,
such as \textit{The Quaker’s Opera}, 1728; \textit{The Jovial Crew}, 1731, \&c.

The words are found in \textit{The Garland of Good-will}, and in several of the celebrated
collections of ballads; also in Percy’s \textit{Reliques}, series ii., book 2.

\musicinfo{Slow.}{}
\lilypondfile{lilypond/187-the-spanish-lady}\normalsize

\musictitle{The Jovial Tinker, or Joan’s Ale Is New.}

On the 26th Oct., 1594, John Danter entered on the books of the Stationers’
Company, “for his copie, a ballet intituled Jone’s ale is newe;” and on the
15th Nov., of the same year, Edward White one called “The unthrifte’s adieu
to Jone’s ale is newe.”

In Ben Jonson’s \textit{Tale of a tub}, “old father Rosin, chief minstrel of Highgate,
and his two boys” play the dances called for by the company, which are “\textit{Tom
Tiler}; \textit{The jolly Joiner}; and \textit{The jovial Tinker}.” The burden of the song called
“The jovial Tinker” is “Joan’s ale is new.” ( “Tom Tiler” is one of the
country dances mentioned in Heywood’s \textit{A woman kill’d with kindness}.) In the
\textit{Mad Pranks and merry Jests of Robin Goodfellow}, 1628, there is a song to the
tune of \textit{The jovial Tinker}, which has a burden or chorus of four lines, unsuited to
this air, although the song itself could be sung to it. As tinkers were so famous
in song, there was probably another tune called \textit{The jovial Tinker}. “He that a
tinker, a tinker will be,” is one of the catches in the \textit{Antidote to Melancholy}, 1661;
“Tom Tinker lives a merry life,” is in Davenant’s play, \textit{The Benefice}; “Have
you any work for a tinker,” in \textit{Wit and Drollery}, 1661; and Ben Jonson says,
in \textit{Paris’ Anniversary}, “Here comes the tinker I told you of, with his kettledrum
before and after, a \textit{master of music}.”

\pagebreak

%%188
%%===============================================================================
The song of Joan’s ale is new is in the Douce Collection, p.~110. It is in the
list of those printed by W. Thackeray, at the Angel in Duck Lane, in the reign
of Charles II.; and is in both editions of \textit{Pills to purge Melancholy}, with the
tune.—(Ed. of 1707, iii. 133; or ed. of 1719, v. 61.)

The copy in the Douce Collection consists of thirteen stanzas, and has the
following lengthy title: “Joan’s ale is new; or a new merry medley, shewing
the power, the strength, the operation, and the virtue that remains in good ale,
which is accounted the mother-drink of England.”
\settowidth{\versewidth}{All you that do this merry ditty view}
\begin{scverse}
\vleftofline{“}All you that do this merry ditty view,\\
Taste of Joan’s ale, for it is strong and new, \&c.”
\end{scverse}

“To a pleasant new Northern tune.”

\musicinfo{Cheerfully.}{}
\lilypondfile{lilypond/188-the-jovial-tinker-or-joans-ale-is-new}\normalsize

\pagebreak

%%189
%%===============================================================================
\settowidth{\versewidth}{Where they drank soundly for a space}
\indentpattern{00010001}
\begin{dcverse}\begin{patverse}
The tinker he did settle\\
Most like a man of mettle,\\
And vow’d to pawn his kettle;\\
Now mark what did ensue:\\
His neighbours they flock in apace,\\
To see Tom Tinker’s comely face,\\
Where they drank soundly for a space,\\
Whilst Joan’s ale, \&c.
\end{patverse}

\settowidth{\versewidth}{And said they would drink for boon, man,}
\begin{patverse}
The cobbler and the broom-man\\
Came up into the room, man,\\
And said they would drink for boon, man,\\
Let each one take his due!\\
But when the liquor good they found,\\
They cast their caps upon the ground,\\
And so the tinker he drank round,\\
Whilst Joan’s ale, \&c.
\end{patverse}
\end{dcverse}

In another volume in the Douce Collection, p.~180, is an answer to the
above, to the same tune. It is the “The poet’s new year’s gift; or a pleasant
poem in praise of sack: setting forth its admirable virtues and qualities, and how
much it is to be preferred before all other sorts of liquors, \&c. To the tune of
\textit{The jovial Tinker}, or \textit{Tom a Bedlam};” commencing—
\settowidth{\versewidth}{\vin Doth far exceed your fountain,” \&c.}
\begin{scverse}\begin{altverse}
\vleftofline{“}Come hither, learned sisters,\\
And leave Parnassus mountain;\\
I will you tell where is a well\\
Doth far exceed your fountain,” \&c.
\end{altverse}
\end{scverse}

\musictitle{Under And Over.}

This is the same air as the preceding, but in a minor instead of a major key.
It is in every edition of \textit{The Dancing Master}, under the name of \textit{Under and over};
but in a MS. volume of virginal music, formerly in the possession of Mr. Windsor,
of Bath, it is entitled \textit{A man had three sons}.

The ballad of \textit{Under and over} is in the Pepys Collection, i. 264, \textsc{b.l.}, as “A new
little Northern Song, called—
\settowidth{\versewidth}{Under and over, over and under,}
\begin{scverse}\vleftofline{“}Under and over, over and under,\\
Or a pretty new jest and yet no wonder;\\
\vleftofline{“}Or a maiden mistaken, as many now be,\\
View well this glass, and you may plainly see.”
\end{scverse}
“To a pretty new Northern tune.”

It is very long, full of typographical errors, and devoid of merit; I have
therefore only printed the first verse with the music.

In the same volume are the following: “Rocke the babie, Joane: to the tune
of \textit{Under and over},” p.~396; beginning—
\settowidth{\versewidth}{A young man in our parish,}
\begin{scverse}\vleftofline{“}A young man in our parish,\\
His wife was somewhat currish,” \&c.
\end{scverse}
And at p.~404, another, commencing—
\begin{scverse}\vleftofline{“}There was a country gallant,\\
That wasted had his talent,” \&c.
\end{scverse}
In the Roxburghe, iii. 176, “Rock the cradle, John:
\settowidth{\versewidth}{Let no man at this strange story wonder,}
\begin{scverse}Let no man at this strange story wonder,\\
It goes to the tune of \textit{Over and under}.”
\end{scverse}

And in the same Collection, i. 411, “The Times’ Abuses; to the tune of \textit{Over and
under}; commencing—
\begin{scverse}\vleftofline{“}Attend, my masters, and give ear,” \&c.
\end{scverse}
The last is also printed in Collier’s \textit{Roxburghe Ballads}, p.~281.

\pagebreak

%%190
%%===============================================================================
\musicinfo{Cheerfully.}{}
\lilypondfile{lilypond/190-under-and-over}\normalsize

\musictitle{The Oxfordshire Tragedy.}

This is one of the old and simple chaunt-like ditties, which seem to have been
peculiarly suited to the lengthy narratives of the minstrels; and I am strongly
impressed with a belief that it was one of their tunes. It has very much the same
character as \textit{Sir Guy}, which I met with in another of the ballad operas, and
which—the entry at Stationers’ Hall proving to be earlier than 1592—may be
fairly supposed to he the air used, by the class of minstrel described by Puttenham,
in singing the adventures of Sir Guy, at feasts. See page 172.

I have seen no earlier copy of \textit{The Oxfordshire Tragedy}, than an edition
“printed and sold in Bow Church-Yard,” in which the name of the tune is not
mentioned. The ballad is in four parts, the third and fourth of which, being in
a different metre, must have been sung to another air.

“As I walk’d forth to take the air,” is the second line of the first part,
and a tune is often referred to under that title. As the measures agree, it may
be a second name for this air.

In the Douce Collection, \pagebreak 44, is a black-letter ballad of “Cupid’s Conquest, or
%\pagebreak
%%191
%%===============================================================================
Will the Shepherd and fair Kate of the Green, both united together in pure love:
to the tune, \textit{As I went forth to take the air};” commencing,—
\settowidth{\versewidth}{Now am I tost on waves of love;}
\begin{scverse}\vleftofline{“}Now am I tost on waves of love;\\
\vin Here like a ship that’s under sail,” \&c.
\end{scverse}
and in the Roxburgh ii. 149, “The faithful lovers of the West: tune, \textit{As I walkt
forth to take the air}.”

In Mr. Payne Collier’s Collection, is “The unfortunate Sailor’s Garland, with
an account how his parents murdered him for love of his gold.” It is in two
parts, and both to the tune of \textit{The Oxfordshire Tragedy}, After four lines of
exordium, it begins thus:—

\settowidth{\versewidth}{“Near Bristol liv’d a man of fame,}
\begin{scverse}\vleftofline{“}Near Bristol liv’d a man of fame,\\
But I’ll forbear to tell his name;\\
He had one son and daughter bright,\\
In whom he took a great delight,” \&c.
\end{scverse}

Another Garland, called “The cruel parents, or the two faithful lovers,” is to
the tune of \textit{The Oxfordshire Lady}, and in the same metre.

The tune of \textit{The Oxfordshire Tragedy} is in \textit{The Cobblers’ Opera}, 1729, \textit{The
Village Opera}, 1729, and \textit{Sylvia}, or \textit{The Country Burial}, 1731.

\musicinfo{Slow.}{}
\lilypondfile{lilypond/191-the-oxfordshire-tragedy}\normalsize

\settowidth{\versewidth}{Where a fair lady made great moan,}
\begin{dcverse}\footnotesize
Down by a crystal river side,\\
A gallant bower I espied,\\
Where a fair lady made great moan,\\
With many a bitter sigh and groan.

Alas! quoth she, my love's unkind,\\
My sighs and tears he will not mind;\\
But he is cruel unto me,\\
Which causes all my misery.

My father is a worthy knight,\\
My mother is a lady bright,\\
And I their only child and heir;\\
Yet love has brought me to despair.

A wealthy squire lived nigh,\\
Who on my beauty cast an eye;\\
He courted me, both day and night,\\
To be his jewel and delight.

To me these words he often said:\\
Fair, beauteous, handsome, comely maid,\\
Oh! pity me, I do implore,\\
For it is you I do adore.

He still did beg me to be kind,\\
And ease his love-tormented mind;\\
For if, said he, you should deny,\\
For love of you I soon shall die.

These words did pierce my tender heart,\\
I soon did yield, to ease his smart;\\
And unto him made this reply,—\\
For love of me you shall not die.

With that he flew into my arms,\\
And swore I had a thousand charms;\\
He call’d me angel, saint, and he\\
Did swear, for ever true to be.
\end{dcverse}

\pagebreak

%%192
%%===============================================================================
\settowidth{\versewidth}{With various thoughts, that broke his rest;}

\begin{dcverse}\scriptsizer
Soon after he had gain’d my heart,\\
He cruelly did from me part;\\
Another maid he does pursue,\\
And to his vows he bids adieu.

Tis he that makes my heart lament,\\
He causes all my discontent;\\
He hath caus’d my sad despair,\\
And now occasions this my care.

The lady round the meadow run,\\
And gather’d flowers as they sprung;\\
Of every sort she there did pull.\\
Until she got her apron full.

Now, there’s a flower, she did say,\\
Is named heart’s-ease; night and day,\\
I wish I could that flower find,\\
For to ease my love-sick mind.

But oh! alas! ’tis all in vain\\
For me to sigh, and to complain;\\
There’s nothing that can ease my smart,\\
For his disdain will break my heart.

The green ground served as a bed,\\
And flow’rs a pillow for her head;\\
She laid her down and nothing spoke,\\
Alas! for love her heart was broke.

But when I found her body cold,\\
I went to her false love, and told\\
What unto her had just befel;\\
I’m glad, said he, she is so well.

Did she think I so fond could be,\\
That I could fancy none but she?\\
Man was not made for one alone;\\
I take delight to hear her moan.

Oh! wicked man I find thou art,\\
Thus to break a lady's heart;\\
In Abraham’s bosom may she sleep,\\
While thy w'icked soul doth weep!

\vin\vin\vin\textsc{the answer.}

A second part, I bring you here,\\
Of the fair maid of Oxfordshire,\\
Who lately broke her heart for love\\
Of one, that did inconstant prove.

A youthful squire, most unjust,\\
When he beheld this lass at first,\\
A thousand solemn vows he made,\\
And so her yielding heart betray’d.

She mourning, broke her heart, and died,\\
Feeling the shades on every side;\\
\columnbreak
With dying groans and grievous cries,\\
As tears were flowing through her eyes.

The beauty which did once appear,\\
On her sweet cheeks, so fair and clear,\\
Was waxed pale,—her life was fled;\\
He heard, at length, that she was dead.

He was not sorry in the least,\\
But cheerfully resolv’d to feast;\\
And quite forgot her beauty bright,\\
Whom he so basely ruin’d quite.

Now, when, alas! this youthful maid,\\
Within her silent tomb was laid,\\
The squire thought that all was well,\\
He should in peace and quiet dwell.

Soon after this he was possest\\
With various thoughts, that broke his rest;\\
Sometimes he thought her groans he heard,\\
Sometimes her ghastly ghost appear’d

With a sad visage, pale and grim,'\\
And ghastly looks she cast on him;\\
He often started back and cried,\\
Where shall I go myself to hide?

Here I am haunted, night and day,\\
Sometimes methinks I hear her say,\\
Perfidous man! false and unkind,\\
Henceforth you shall no comfort find.

If through the fields I chance to go,\\
Where she receiv’d her overthrow,\\
Methinks I see her in despair;\\
And, if at home, I meet her there.

No place is free of torment now;\\
Alas! I broke a solemn vow\\
Which once I made; but now, at last,\\
It does my worldly glory blast.

Since my unkindness did destroy\\
My dearest love and only joy,\\
My wretched life must ended be,\\
Now must I die and come to thee.

His rapier from his side he drew,\\
And pierced his body thro’ and thro’;\\
So he dropt down in purple gore\\
Just where she did some time before.

He buried was within the grave\\
Of his true love. And thus you have\\
A sad account of his hard fate,\\
Who died in Oxfordshire of late.
\end{dcverse}

The third and fourth parts present a similar story, in different metre; but
it is the lady’s cruelty which causes the first suicide.

\pagebreak

%%193
%%===============================================================================
\musictitle{Put On Thy Smock On Monday.}

This is mentioned as a country dance tune in Heywood’s \textit{A Woman kill’d with
Kindness}, act i., sc. 2; and alluded to in Fletcher’s \textit{Love’s Cure}, act ii., sc. 2.
It is contained in the fourth, fifth, and later editions of \textit{The Dancing Master}.

\musicinfo{Moderate time.}{}
\lilypondfile{lilypond/193-put-on-thy-smock-on-monday}\normalsize

\musictitle{Drive The Cold Winter Away.}

This is the burden of a song in praise of Christmas, copies of which are in the
Pepys (i. 186) and Roxburghe (i. 24) Collections. It is entitled “A pleasant
countrey new ditty: merrily shewing how to drive the cold winter away. To
the tune of \textit{When Phoebus did rest},”\dcfootnote{ %a
A song beginning “When Phoebus \textit{addrest} his course
to the West,” will be found in \textit{Merry Drollery Complete},
Part ii., 1661; also in \textit{Wit and Drollery, Jovial Poems}.
The burden is, “O do not, do not kill me yet, for I am
not prepared to die.” By that name it is quoted in J.
Starter’s \textit{Boertigheden}, quarto, Amsterdam, 1634, where
the tune is also printed.
} %end footnote 
\&c.; black-letter, printed by H[enry]
G[osson]. It is one of those parodied in Andro Hart’s \textit{Compendium of Godly
Songs}.
\settowidth{\versewidth}{The wind blawis cald, furious and bald,}
\begin{scverse}
\begin{altverse}
\vleftofline{“}The wind blawis cald, furious and bald,\\
This lang and mony a day;\\
But, Christ’s mercy, we mon all die,\\
Or keep the cald wind away.\\
This wind sa keine, that I of meine,\\
It is the vyce of auld;\\
Our faith is inclusit, and plainely abusit,\\
This wind he’s blawin too cald,” \&c.\\
\vin\vin\vin\textit{Scottish Poems of 16th Century}, ii. 177, 8vo., 1801.
\end{altverse}
\end{scverse}

The tune is in every edition of \textit{The Dancing Master}; in \textit{Musick’s Delight on
the Cithren}, 1666; and in Walsh’s \textit{Dancing Master}: also in both editions of
\textit{Pills to purge Melancholy}, with an abbreviated copy of the words.

In the Roxburghe Collection, i. 518, is a ballad entitled “Hang pinching; or
The good fellow’s observation ’mongst a jovial crew, of them that hate flinching, 
but are always true blue. To the tune of \textit{Drive the cold winter away};”
commencing—
\settowidth{\versewidth}{All you that lay claim to a good fellow’s name,}
\begin{scverse}
\begin{altverse}
\vleftofline{“}All you that lay claim to a good fellow’s name,\\
And yet do not prove yourselves so,\\
Give ear to this thing, the which I will sing,\\
Wherein I most plainly will shew
\end{altverse}
\end{scverse}

\pagebreak

%%194
%%===============================================================================

\settowidth{\versewidth}{With proof and good ground, those fellows profound,}
\begin{scverse}
\begin{altverse}
With proof and good ground, those fellows profound,\\
That unto the alewives are true,\\
In drinking their drink, and paying their chink,\\
\textit{O such a good fellow's true blue}.”
\end{altverse}
\end{scverse}
Sometimes a tune named \textit{True blue} is quoted, and perhaps from this ballad. It is
subscribed W. B., and printed for Thomas Lambert, at the sign of the Horse
Shoe, in Smithfield. Lambert was a printer of the reigns of James and Charles I.

In the Pepys Collection, i. 362, is another black-letter ballad, entitled “The
father hath beguil’d the son: Or a wonderful tragedy which lately befell in Wiltshire,
as many men know full well; to the tune of \textit{Drive the cold winter away};”
beginning—
\settowidth{\versewidth}{“I often have known, and experience hath shown,}
\begin{scverse}
\begin{altverse}
\vleftofline{“}I often have known, and experience hath shown,\\
That a spokesman hath wooed for himself,\\
And that one rich neighbour will, underhand, labour\\
To overthrow another with pelf,” \&c.
\end{altverse}
\end{scverse}
Other ballads to the tune will be found in the Roxburghe Collection (i. 150 and
160, \&c.); in the King’s Pamphlets, and the Collection of Songs against the
Rump Parliament; in Wright’s \textit{Political Songs}; in \textit{Mock Songs}, 1675; in Evans’
Collection, i.~349,~\&c.

\musicinfo{Boldly and not too fast.}{Song in praise of Christmas.}
\lilypondfile{lilypond/194-drive-the-cold-winter-away}\normalsize

\pagebreak
%%195
%%===============================================================================

\settowidth{\versewidth}{If wrath be to seek, do not lend her thy cheek,zz}
\begin{dcverse}\footnotesize
\begin{altverse}
Let Misery pack, with a whip at his back,\\
To the deep Tantalian flood;\\
In Lethe profound, let envy be drown’d,\\
That pines at another man’s good;\\
Let Sorrow’s expanse be banded from hence,\\
All payments of grief delay,\\
And wholly consort with mirth and with sport\\
To drive the cold winter away.
\end{altverse}

\begin{altverse}
{’}Tis ill for a mind to anger inclin{’d}\\
To think of old injuries now;\\
If wrath be to seek, do not lend her thy cheek,\\
Nor let her inhabit thy brow.\\
Cross out of thy books malevolent looks,\\
Both beauty and youth’s decay.\\
And spend the long nights in honest delights,\\
To drive the cold winter away.
\end{altverse}

\begin{altverse}
The court in all state now opens her gate,\\
And bids a free welcome to most;\\
The city likewise, tho’ somewhat precise,\\
Doth willingly part with her cost:\\
And yet by report, from city and court,\\
The country will gain the day;\\
More liquor is spent, and with better content,\\
To drive the cold winter away.
\end{altverse}

\begin{altverse}
Our good gentry there, for cost do not spare,\\
The yeomanry fast not till Lent;\dcfootnote{\scriptsize %a
 For the support and encouragement of the fishing
towns, in the time of Elizabeth, Wednesdays and Fridays
were constantly observed as fast days, or days of abstinence
from flesh. This was by the advice of her minister,
Cecil; and by the vulgar it was generally called Cecil’s
Fast. See Warburton's and Blakeway’s notes in Boswell’s
edition of Shakespeare, x. 49 and~50.
} \\
The farmers, and such, think nothing too much,\\
If they keep but to pay for their rent.\\
The poorest of all do merrily call,\\
When at a fit place they can stay,\\
For a song or a tale, or a pot of good ale,\\
To drive the cold winter away.
\end{altverse}

\begin{altverse}
Thus none will allow of solitude now,\\
But merrily greets the time,\\
To make it appear, of all the whole year,\\
That this is accounted the prime:\\
December is seen apparel’d in green,\\
And January, fresh as May,\\
Comes dancing along, with a cup and a song,\\
To drive the cold winter away.
\end{altverse}

\begin{altverse}
\vin\vin\vin\textsc{the second part.}
\end{altverse}

\begin{altverse}
This time of the year is spent in good cheer,\\
And neighbours together do meet,\\
To sit by the fire, with friendly desire,\\
Each other in love to greet;\\
Old grudges forgot, are put in the pot,\\
All sorrows aside they lay,\\
The old and the young doth carol his song,\\
To drive the cold winter away.
\end{altverse}

\begin{altverse}
Sisley and Nanny, more jocund than any,\\
As blithe as the month of June,\\
Do carol and sing, like birds of the Spring,\\
(No nightingale sweeter in tune)\\
To bring in content, when summer is spent,\\
In pleasant delight and play,\\
With mirth and good cheer, to end the old year,\\
And drive the cold winter away.
\end{altverse}

\begin{altverse}
The shepherd and swain do highly disdain\\
To waste out their time in care,\\
And Clim of the Clough\dcfootnote{\scriptsize %b
 \textit{Clim of the Clough} means Clement of the Cleft. The
name is derived from a noted archer, once famous in the
north of England. See the old ballad, \textit{Adam Bell, Clim of
the Clough, and William of Cloudesly}, printed by Bp. Percy.
A \textit{Clough} is a sloping valley, breach, or \textit{Cleft}, from the
side of a hill, where trees or furze usually grow.
}\\
If he but a penny can spare,\\
To spend at the night in joy and delight,\\
Now after his labours all day,\\
For better than lands is the help of his hands,\\
To drive the cold winter away.
\end{altverse}

\begin{altverse}
To mask and to mum kind neighbours will come\\
With wassails of nut-brown ale,\\
To drink and carouse to all in the house,\\
As merry as bucks in the dale;\\
Where cake, bread and cheese, Is brought for your fees,\\
To make you the longer stay;\\
At the fire to warm will do you no harm,\\
To drive the cold winter away.
\end{altverse}

\begin{altverse}
When Christmas’s tide comes in like a bride,\\
With holly and ivy clad,\\
Twelve days in the year, much mirth and good cheer, \\
In every household is had;\\
The country guise is then to devise\\
Some gambols of Christmas play,\\
Whereat the young men do best that they can,\\
To drive the cold winter away.
\end{altverse}

\begin{altverse}
When white-bearded frost hath threatened his worst\\
And fallen from branch and brier,,\\
Then time away calls, from husbandry halls\\
And from the good countryman’s fire,\\
Together to go to plough and to sow,\\
To get us both food and array;\\
And thus with content the time we have spent\\
To drive the cold winter away. 
\end{altverse}
\end{dcverse}

\pagebreak

%%196
%%===============================================================================
%

\musictitle{Up, Tails All.}

This tune is in Queen Elizabeth’s Virginal Book, and in \textit{The Dancing Master}
from 1650 to 1690. It is alluded to in Sharpham’s \textit{Fleire}, 1610: “She every
day sings \textit{John for the King}, and at \textit{Up, tails all}, she’s perfect.” Also in Ben
Jonson’s \textit{Every man out of his humour}; in Beaumont and Fletcher’s \textit{Coxcomb};
Vanbrugh’s \textit{Provoked Wife}, \&c.

There are several political songs of the Cavaliers to this air, in the King’s
Pamphlets (Brit.. Mus.); in the Collection of Songs written against the Rump
Parliament; in \textit{Rats rhimed to Death}, 1660; and one in \textit{Merry Drollery complete},
1670: but party feeling was then so often expressed with more virulence than wit,
that few of them will bear republication. In both the editions of \textit{Pills to purge
Melancholy}, 1707 and 1719, the song of \textit{Up, tails all}, beginning “Fly, merry
news,” is printed by mistake with the title and tune of \textit{The Friar and the Nun}.

\musicinfo{Moderate time and lightly.}{}
\lilypondfile{lilypond/196-up-tails-all}\normalsize

\musictitle{Pescod Time.}

The tune of In \textit{Pescod Time} (\ie, peas-cod time, when the field peas are
gathered), was extremely popular towards the end of the sixteenth century. It is
contained in Queen Elizabeth’s and Lady Neville’s Virginal Books; in Anthony
Holborne’s \textit{Citharn Schoole} (1597); and in Sir John Hawkins’ transcripts; but
so disguised by point, augmentation, and other learned contrivances, that it was
only by scanning the whole arrangement (by Orlando Gibbons) that this simple
air could, be extracted. In Queen Elizabeth’s Virginal Book, the same air is
called \textit{The Hunt's up,} in another part of the book.

The words are in \textit{England's Helicon}, 1600 (or reprint in 1812, p.~206); in
Miss Cooper’s \textit{The Muses' Library}, 8vo, p.~281; and in Evans’ \textit{Old Ballads},
i. 332 (ed.~of~1810).

Two very important and popular ballads were sung to the tune: \textit{Chevy Chace},
and \textit{The Lady’s Fall}.

\textit{Chevy Chace} had also a separate air (see page 199); but the earlier printed
copies of the ballad direct it to be sung to “\textit{In Pescod Time}." 

\pagebreak

%%197
%%===============================================================================

The “\textit{Lamentable ballad of the Lady’s Fall}, to the tune of \textit{In Pescod Time},”
will be found in the Douce, Pepys, and Bagford Collections, and has been reprinted
by Percy and Ritson. It commences thus:—
\settowidth{\versewidth}{Mark well my heavy dolefull tale,}
\begin{scverse}
\begin{altverse}
\vleftofline{“}Mark well my heavy dolefull tale,\\
You loyal lovers all;\\
And heedfully bear in your breast\\
A gallant lady’s fall.”
\end{altverse}
\end{scverse}

Among the ballads to the tune of \textit{The Lady’s Fall} are \textit{The Bride’s Burial},
and \textit{The Lady Isabella’s Tragedy}; both in Percy’s Reliques. \textit{The life and death
of Queen Elizabeth}, in the \textit{Crown Garland of Golden Roses}, 1612 (page 39 of the
reprint), and in Evans’ \textit{Old Ballads}, iii. 171. \textit{The Wandering Jew, or the Shoemaker
of Jerusalem, who lived when our Saviour Christ was crucified, and appointed
to live until his coming again}; two copies in the British Museum, and one in
Mr. Halliwell’s Collection; also reprinted by Washbourne. It has the burden,
“Repent, therefore, O England,” and is, perhaps, the ballad by Deloney, to which
Nashe refers in \textit{Have with you to Saffron-Walde}n (ante page 107). \textit{The Cruel
Black}; see Evans’ \textit{Old Ballads}, iii. 232. \textit{A Warning for Maidens, or young
Bateman}; Roxburghe Collection, i. 501. It begins, “You dainty dames so finely
framed.” And \textit{You dainty dames} is sometimes quoted as a tune; also \textit{Bateman},
as in a ballad entitled “\textit{A Warning for Married Women}, to a West-country tune
called \textit{The Fair Maid of Bristol}, or Bateman, or \textit{John True}; Roxburghe, i. 502.

The following Carol is from a Collection, printed in 1642, a copy of which is in
Wood’s Library, Oxford. I have not seen it elsewhere.

\qquad\qquad “A Carol for Twelfth Day, to the tune of \textit{The Lady’s Fall}.”
\vspace{-0.5\baselineskip}
\settowidth{\versewidth}{Plum porridge, roast beef, and minc’d pies,}

\begin{dcverse}\begin{altverse}
Mark well my heavy doleful tale,\\
For Twelfth Day now is come,\\
And now I must no longer stay,\\
And say no word but mum.\\
For I perforce must take my leave\\
Of all my dainty cheer—\\
Plum porridge, roast beef, and minc’d pies,\\
My strong ale and my beer.
\end{altverse}

\begin{altverse}
Kind-hearted Christmas, now adieu,\\
For I with thee must part;\\
But oh! to take my leave of thee\\
Doth grieve me at the heart.\\
Thou wert an ancient housekeeper,\\
And mirth with meat didst keep;\\
But thou art going out of town,\\
Which causes me to weep.
\end{altverse}

\begin{altverse}
God knoweth whether I again\\
Thy merry face shall see;\\
Which to good fellows and the poor\\
Was always frank and free.\\
Thou lovest pastime with thy heart,\\
And eke good company;\\
Pray hold me up for fear I swound [swoon],\\
For I am like to die.
\end{altverse}

\begin{altverse}
Come, butler, fill a brimmer full,\\
To cheer my fainting heart,\\
That to old Christmas I may drink\\
Before he does depart.\\
And let each one that’s in the room\\
With me likewise condole,\\
And now, to cheer their spirits sad,\\
Let each one drink a bowl.
\end{altverse}

\begin{altverse}
And when the same it hath gone round,\\
Then fall unto your cheer;\\
For you well know that Christmas time\\
It comes but once a year.\\
But this good draught which I have drank\\
Hath comforted my heart;\\
For I was very fearful that\\
My stomach would depart.
\end{altverse}

\begin{altverse}
Thanks to my master and my dame,\\
That do such cheer afford;\\
God bless them, that, each Christmas, they\\
May furnish so their board.\\
My stomach being come to me,\\
I mean to have a bout;\\
And now to eat most heartily,—\\
Good friends, I do not flout. 
\end{altverse}
\end{dcverse}

\pagebreak

%%198
%%===============================================================================

\musicinfo{Rather slow and smoothly.}{}
\lilypondfile{lilypond/198-pescod-time}\normalsize

\musictitle{Chevy Chace.}

Although sometimes sung to the tunes of \textit{Pescod Time} and \textit{The Children in the
Wood}, this is the air usually entitled \textit{Chevy Chace}. It bears that name in all the
editions of \textit{Pills to purge Melancholy}, and in the ballad operas, such as \textit{The
Beggars' Opera}, 1728, \textit{Trick for Trick}, 1735, \&c. Another name, and probably
an older, is \textit{Flying Fame}, or \textit{When flying Fame}, to which a large number of
ballads have been written. In \textit{Pills to purge Melancholy}, “King Alfred and the
Shepherd’s Wife,” which the old copies direct to be sung to the tune of \textit{Flying
Fame}, is printed to this air.

Much has been written on the subject of \textit{Chevy Chace}; but as both the ballads
are printed in Percy’s \textit{Reliques of Ancient Poetry} (and in many other collections), 
it maybe sufficient here to refer the reader to that work, and to \textit{The
British Bibliographer} (iv. 97). The latter contains an account of Richard Sheale,
the minstrel to whom we are indebted for the preservation of the more ancient
ballad, and of his productions. The manuscript containing them is in the Ashmolean
Library, Oxford (No. 48, 4to). His verses on being robbed on Dunsmore
Heath have been already quoted (pages 45 to 47).

The ballad of \textit{Chevy Chace}, in Latin Rhymes, by Henry Bold, will be found in
Dryden’s \textit{Miscellany Poems}, ii. 288. The translation was made at the request of
Dr.~Compton, Bishop of London.

Bishop Corbet, in his \textit{Journey into Fraunce}, speaks of having sung \textit{Chevy
Chace} in his youth; the antiquated beau in Davenant’s play of \textit{The Wits}, also
prides himself on being able to sing it; and, in \textit{Wit’s Interpreter}, 1671, a man,
enumerating the good qualities of his wife, cites, after the beauties of her mind
and her patience, “her curious voice, wherewith she useth to sing \textit{Chevy Chace}.”
From these, and many similar allusions, it is evident that it was much sung in
the seventeenth century, despite its length.

Among the many ballads to the tune (either as \textit{Flying Fame} or \textit{Chevy Chace}),
the following require particular notice. 

\pagebreak

%%199
%%===============================================================================


“A lamentable song of the Death of King Lear and his three Daughters: to
the tune of \textit{When flying Fame}.” See Percy’s \textit{Reliques}, series i., book 2.

“A mournefull dittie on the death of Faire Rosamond; tune of \textit{Flying Fame}:”
beginning, “When as King Henry rul’d this land;” and quoted in Rowley’s
\textit{A~Match at Midnight}. See \textit{Strange Histories}, 1607; \textit{The Garland of Good-wil}l;
and Percy, series~ii., book 2.

“The noble acts of Arthur of the Round Table, and of Sir Launcelot du Lake:
tune of \textit{Flying Fame}.” See \textit{The Garland of Good-will}, 1678, and Percy, series~i.,
book 2. The first line of this ballad (“When Arthur first in court began”) is
sung by Falstaff in Part II. of Shakespeare’s \textit{King Henry IV}.; also in Marston’s
\textit{The Malcontent}, 1604, and in Beaumont and Fletcher’s \textit{The Little French Lawyer}.

“King Alfred and the Shepherd’s Wife: to the tune of \textit{Flying Fame}.” See
\textit{Old Ballads}, 1727, i. 43; \textit{Pills to purge Melancholy}, 1719, v. 289; and Evans’
\textit{Old Ballads}, 1810, ii. 11.

“The Union of the Red Rose and the White, by a marriage between King
Henry VII. and Elizabeth Plantagenet, daughter of Edward IV: to the tune of
When flying Fame.” See \textit{Crown Garland}, 1612, and Evans, iii. 35.

“The Battle of Agincourt, between the Englishmen and the Frenchmen: tune,
\textit{Flying Fame}.” (Commencing, “A council grave our King did hold.”) See
\textit{Crown Garland}, 1659, and Evans, ii. 351.

“The King and the Bishop: tune of \textit{Chevy Chace}.” Roxburghe, iii. 170.

“Strange and true newes of an Ocean of Flies dropping out a cloud, upon the
town of Bodnam [Bodmin?] in Cornwall: tune of \textit{Chevy Chace}” (dated 1647).
See King’s Pamphlets, Brit. Mus., vol. v., and Wright’s \textit{Political Ballads}.

“The Fire on London Bridge” (from which the nursery rhyme, “Three
children sliding on the ice,” has been extracted), “to the tune of \textit{Chevy Chace}.”
\textit{Merry Drollery complete}, 1670, \textit{Pills to purge Melancholy}, ii. 6, 1707, and
Rimbault’s \textit{Little Book of Songs and Ballads}, 12mo., 1851. Dr.~Rimbault quotes
other copies of the ballad, and especially one in the Pepys Collection (ii. 146),
to the tune of \textit{The Lady's Fall}; further proving the difficulty of distinguishing
between this tune and \textit{In Pescod Time}.

\musicinfo{Smoothly and rather slow.}{}
\lilypondfile{lilypond/199-chevy-chace}\normalsize

\pagebreak

%%200
%%===============================================================================

\musictitle{The Children In The Wood.}

In the Registers of the Stationers’ Company, under the date of 15th October,
1595, we find, “Thomas Millington entred for his copie under t’handes of bothe
the Wardens, a ballad intitutled ‘The Norfolk Gentleman, his Will and Testament,
and howe he commytted the keeping of his children to his owne brother, whoe delte
moste wickedly with them, and howe God plagued him for it.” This entry agrees,
almost verbatim, with the title of the ballad in the Pepys Collection (i.~518),
but which is of later date. Copies will also be found in the Roxburghe (i.~284),
and other Collections; in \textit{Old Ballads}, 1726, i. 222; and in Percy’s \textit{Reliques},
series~iii., book 2.

Sharon Turner says, “I have sometimes fancied that the popular ballad of
\textit{The Children in the Wood} may have been written at this time, on Richard [III.]
and his nephews, before it was quite safe to stigmatize him more openly.”—
(\textit{Hist.~Eng}., iii.~487, 4to). This theory has been ably advocated by Miss
Halsted, in the Appendix to her \textit{Richard III. as Duke of Gloucester and King of
England}. Her argument is based chiefly upon internal evidence, there being no
direct proof that the ballad is older than the date of the entry at Stationers’ Hall.

In Wager’s interlude, \textit{The longer thou livest the more fool thou art}, Moros says,
“I can sing a song of Robin Redbreast;” and in Webster’s \textit{The White Devil},
Cornelia says, “I’ll give you a saying which my grandmother was wont, when
she heard the bell toll, to sing, unto her lute:
\settowidth{\versewidth}{Call for the robin-redbreast and the wren.}
\begin{scverse}
Call for the robin-redbreast and the wren.\\
Since o’er the shady groves they hover,\\
And with leaves and flowers do cover\\
The friendless bodies of unburied men,” \&c.\\
\vin\vin\vin\vin \textit{Dodsley’s Old Plays}, vi. 312, 1825.
\end{scverse}
These \textit{may} be in allusion to the ballad.

In Anthony à Wood’s Collection, at Oxford, there is a ballad to the tune of
\textit{The two Children in the Wood}, entitled “The Devil’s Cruelty to Mankind,” \&c.

The history of the tune is somewhat perplexing. In the ballad-operas of
\textit{The Jovial Crew}, \textit{The Lottery}, \textit{An old man taught wisdom}, and \textit{The Beggars’
Opera}, it is printed under the title of \textit{Now ponder well}, which are the first words
of “\textit{The Children in the Wood}.”

The broadsides of \textit{Chevy Chace}, which were printed \textit{with music} about the commencement
of the last century, are also to this tune; and in the ballad-opera of
\textit{Penelope}, 1728, a parody on Chevy Chace to the same.

In \textit{Pills to purge Melancholy}, 1707 and 1719, the ballads of “Henry V. at the
battle of Agincourt,” “The Lady Isabella’s Tragedy,” and a song by Sir John
Birkenhead, are printed to it. The last seems to be a parody on “Some Christian
people all give ear,” or “The Fire on London Bridge.”

According to the old ballads, \textit{The Battle of Agincourt} should be to the tune of
\textit{Flying Fame}, \textit{The Lady Isabella's Tragedy} to \textit{In Pescod Time}, and \textit{The Fire on
London Bridge} to \textit{Chevy Chace}. I suppose the confusion to have arisen from.
\textit{Chevy Chace} being sung to all the three tunes.

The traditions of the stage also give this as the air of the Gravedigger’s Song
in \textit{Hamlet}, “A pick-axe and a spade.” 
\pagebreak


%%201
%%===============================================================================


\musicinfo{Slowly and smoothly.}{}
\lilypondfile{lilypond/201-the-children-in-the-wood}\normalsize

\musictitle{In Sad And Ashy Weeds.}

The four first stanzas of this song were found among the Howard papers in
the Heralds’ College, in the handwriting of Anne, Countess of Arundel, widow of
the Earl who died in confinement in the Tower of London in 1595. They were
written on the cover of a letter. Lodge, who printed them in his \textit{Illustrations of
British History} (iii. 241, 8vo., 1838), thought they “were probably composed”
by the Countess; and that “the melancholy exit of her lord was not unlikely to
have produced these pathetic effusions.” She could not, however, have been the
\textit{author} of verses, in her transcript of which the rhymes between the first and third
lines of every stanza have been overlooked.\dcfootnote{ %a
 In the Countess’s transcript, as printed by Lodge,
the first four lines stand thus—
\settowidth{\versewidth}{In sad and ashy weeds I sigh,}
\begin{fnverse}
\begin{altverse}
\vleftofline{“}In sad and ashy weeds I \textit{sigh},\\
I groan, I pine, I mourn;\\
My oaten yellow reeds\\
I all to jet and ebon turn;”
\end{altverse}
\end{fnverse}
instead of—
\begin{fnverse}
\begin{altverse}
“In sad and ashy weeds\\
I sigh, I groan, I pine, I mourn;”
\end{altverse}
\end{fnverse}
as “weeds” should rhyme with “reeds” in the third line,
and so in each verse.
} %end footnote
They were evidently written from
memory, and rendered more applicable to her case by a few trifling alterations,
such as “Not I, poor I, alone,” instead of “Now, a poor lad alone,” at the
commencement of the fourth stanza.

The tune is contained in a MS. volume of virginal music, transcribed by Sir
John Hawkins; the words in the \textit{Crown Garland of Golden Roses}, edition of
1659 (Percy Society reprint, p.~6.). It is there entitled “The good Shepherd’s
sorrow for the loss of his beloved son.”

Among the ballads to the tune of \textit{In sad and ashy weeds}, are “A servant’s
sorrow for the loss of his late royal mistress, Queen Anne” (wife to James I.),
“who died at Hampton Court” (May 2, 1618), beginning—
\settowidth{\versewidth}{In dole and deep distress,}
\begin{scverse}
\vleftofline{“}In dole and deep distress,\\
Poor soul, I, sighing, make my moan.”
\end{scverse}
It will be found in the same edition of the \textit{Crown Garland}; as well as an answer
to \textit{In sad and ashy weeds}, entitled “Coridon’s Comfort: the second part of the
good Shepherd commencing, “Peace, Shepherd, cease to moan.”

The tune is quoted under the title of “In sadness, or Who can blame my woe,”
as one for the \textit{Psalmes or Songs of Sion}, \&c., 1642. 

\pagebreak

%%202
%%===============================================================================

\musicinfo{Slowly and smoothly.}{}
\lilypondfile{lilypond/202-in-sad-and-ashy-weeds}\normalsize

\settowidth{\versewidth}{\vin But more than men make moan with me:}
\indentpattern{0101221221}
\begin{dcverse}\begin{patverse}
In sable robes of night\\
My days of joy consumed be,\\
My sorrow sees no light,\\
My lights through sorrow nothing see.\\
For now my sun\\
His course hath run,\\
And from my sphere doth go,\\
To endless bed\\
Of folded lead;\\
And who can blame my woe?
\end{patverse}

\begin{patverse}
My flocks I now forsake,\\
That so my sheep my grief may know,\\
The lilies loathe to take,\\
That since his death presum’d to grow.\\
I envy air,\\
Because it dare\\
Still breathe, and he not so;\\
Hate earth, that doth\\
Entomb his youth;\\
And who can blame my woe?
\end{patverse}

\begin{patverse}
Not I, poor I, alone,\\
(Alone, how can this sorrow be?)\\
Not only men make moan,\\
But more than men make moan with me:\\
The gods of greens, \\
The mountain queens,\\
The fairy-circled row,\\
The muses nine,\\
And powers divine,\\
Do all condole my woe.
\end{patverse}
\end{dcverse}

In the above lines I have chiefly followed the Countess of Arundel’s transcript.
There are three more verses in the \textit{Crown Garland of Golden Roses}, besides seven
in the second part. 
\pagebreak

%%203
%%===============================================================================
%

\musictitle{The Bailiff’S Daughter Of Islington.}

Copies of this ballad are in the Roxburghe, Pepys, and Douce Collections; it is
printed by Ritson among the ancient ballads in his \textit{English Songs}, and by Percy
(\textit{Reliques}, series iii., book 2, No. 8).

In the Roxburghe, ii. 457, and Douce, 230, it is entitled “True love requited,
or The Bailiff’s Daughter of Islington: to a \textit{North-country tune}, or \textit{I have a good
old mother at home}.'” In other copies it is to “I have a good \textit{old woman} at home,”
and “I have a good \textit{wife} at home.”

In the Douce, 32, is a ballad called “Crums of comfort for the youngest sister,
\&c., to a pleasant new \textit{West-country} tune;” beginning—
\settowidth{\versewidth}{I have a good old \textit{father} at home,}
\begin{scverse}\begin{altverse}
\vleftofline{“}I have a good old \textit{father} at home,\\
An ancient man is he:\\
But he has a mind that ere he dies\\
That I should married be.”
\end{altverse}
\end{scverse}

Dr.~Rimbault found the first tune in a lute MS., formerly in the possession of
the Rev. Mr. Gostling, of Canterbury, under the name of \textit{The jolly Pinder}. It is
in the ballad-opera of \textit{The Jovial Crew}, 1731, called “The Baily’s Daughter of
Islington.”

The second is the traditional tune to which it is commonly sung throughout the
country.

\musicinfo{Rather slow.}{First tune.}
\lilypondfile{lilypond/203-the-bailiffs-daughter-of-islington}\normalsize

\settowidth{\versewidth}{Yet she was coy, and would not believe}
\begin{dcverse}\begin{altverse}
Yet she was coy, and would not believe\\
That he did love her so,\\
No, nor at any time would she\\
Any countenance to him show.
\end{altverse}

\begin{altverse}
But when his friends did understand\\
His fond and foolish mind,\\
They sent him up to fair London,\\
An apprentice for to bind.
\end{altverse}

\begin{altverse}
And when he had been seven long years,\\
And never his love could see:\\
Many a tear have I shed for her sake,\\
When she little thought of me.
\end{altverse}

\begin{altverse}
Then all the maids of Islington\\
Went forth to sport and play,\\
All but the bailiff’s daughter dear;\\
She secretly stole away.
\end{altverse}

\begin{altverse}
She pulled off her gown of green,\\
And put on ragged attire,\\
And to fair London she would go,\\
Her true love to enquire.
\end{altverse}

\begin{altverse}
And as she went along the high road,\\
The weather being hot and dry,\\
She sat her down upon a green bank,\\
And her true love came riding by. 
\end{altverse}
\end{dcverse}
\pagebreak
%%\origpage{}%204
%%===============================================================================

\settowidth{\versewidth}{\vin Whom I thought I should never see more.}
\begin{dcverse}\begin{altverse}
She started up with a colour so red,\\
Catching hold of his bridle-rein;\\
One penny, one penny, kind sir, she said,\\
Will ease me of much pain.
\end{altverse}

\begin{altverse}
Before I give you one penny, sweet-heart,\\
Pray tell me where you were born:\\
At Islington, kind Sir, said she,\\
Where I have had many a scorn.
\end{altverse}

\begin{altverse}
I prythee, sweetheart, tell to me,\\
O tell me whether you know\\
The bailiff’s daughter of Islington?\\
She is dead, Sir, long ago.
\end{altverse}

\begin{altverse}
If she he dead, then take my horse,\\
My saddle and bridle also;\\
For I will into some far country,\\
Where no man shall me know.
\end{altverse}

\begin{altverse}
O stay, O stay, thou goodly youth,\\
She standeth by thy side;\\
She is here alive, she is not dead,\\
And ready to be thy bride.
\end{altverse}

\begin{altverse}
O farewell grief, and welcome joy,\\
Ten thousand times therefore;\\
For now I have found mine own true love,\\
Whom I thought I should never see more.
\end{altverse}
\end{dcverse}

\musicinfo{Rather slow and very smoothly.}{Second tune.}
\lilypondfile{lilypond/204-the-bailiffs-daughter-of-islington-second-tune}\normalsize

\musictitle{It Was A Lover And His Lass.}

From a quarto MS., which has successively passed through the hands of
Mr. Cranston, Dr.~John Leyden, and Mr. Heber; and is now in the Advocates’
Library, Edinburgh. It contains about thirty-four songs with words,\dcfootnote{ %a
Among these are Wither’s song, “Shall I, wasting
in despair,” and “Farewell, dear love,” quoted in \textit{Twelfth
Night}, the music of which, by Robert Jones (twelfth from
his first book, published in 1601) is reprinted in \textit{Musica
Antiqua: a Selection of Music from the commencement of
the twelfth to the beginning of the eighteenth century}, \&c.
edited by John Stafford Smith.
} %end footnote
and sixteen song and dance tunes without. The latter part of the manuscript, which bears
the name of a former proprietor, William Stirling, and the date of May, 1639,
consists of Psalm Tunes, evidently in the same handwriting, and written about
the same time as the earlier portion. This song is in the comedy of \textit{As you
like it}, the first edition of which was printed in 1623; and the inaccuracies in
that copy, which have given much trouble to commentators on Shakespeare, are
not to be found in this. In the printed copy, the last verse stands in the place of
the second: this was first observed and remedied by Dr.~Thirlby; and the words
“ring time,” there rendered “\textit{rang} time,” and by commentators altered to “\textit{rank}
time,” were first restored to the proper meaning by Steevens, who explains them \pagebreak
as signifying the aptest season for marriage. The words are here printed from the 
%%205
%%===============================================================================
\renewcommand\rectoheader{illustrating Shakespeare.} 
manuscript in the Advocates’ Library, (fol. 18), and other variations will be
found on comparing them with the published copies of the~play.

\musicinfo{Moderate time.}{}
\lilypondfile{lilypond/205-it-was-a-lover-and-his-lass}\normalsize

\settowidth{\versewidth}{With a hey, with a ho, with a hey, non ne no,}
\indentpattern{000022222}
\begin{dcverse}\begin{patverse}
Between the acres of the rye,\\
With a hey, with a ho, with a hey, non ne no,\\
And a hey non ne, no ni no.\\
These pretty country fools did lie,\\
In Spring time, in Spring time,\\
The only pretty ring time,\\
When birds do sing\\
Hey ding, a ding, a ding,\\
Sweet lovers love the Spring.
\end{patverse}

\settowidth{\versewidth}{This carol they began that hour,}
\indentpattern{0202}
\begin{patverse}
This carol they began that hour,\\
With a hey, \&c.\\
How that life was but a flow’r,\\
In Spring time, \&c.
\end{patverse}

\begin{patverse}
Then, pretty lovers, take the time,\\
With a hey, \&c.,\\
For love is crowned with the prime,\\
In Spring time, \&c. 
\end{patverse}
\end{dcverse}

\pagebreak

%%206
%%===============================================================================



\musictitle{Willow, Willow!}

The song of \textit{Oh! willow, willow}, which Desdemona sings in the fourth act of
\textit{Othello}, is contained in a MS. volume of songs, with accompaniment for the lute,
in the British Museum (Addit. MSS. 15,117). Mr. Halliwell considers the
transcript to have been made about the year 1633; Mr. Oliphant (who catalogued
the musical MS.) dates it about 1600; but the manuscript undoubtedly contains
songs of an earlier time, such as—
\settowidth{\versewidth}{O death! rock me asleep,}
\begin{scverse}
\vleftofline{“}O death! rock me asleep,\\
Bring me to quiet rest,” \&c.
\end{scverse}
attributed to Anne Boleyn, and which Sir John Hawkins found in a MS. of the
reign of Henry VIII.

The song of \textit{Willow, willow}, is also in the Roxburghe Ballads, i. 54; and was
printed by Percy from a copy in the Pepys Collection, entitled “A Lover’s
Complaint, being forsaken of his Love: to a pleasant tune.”

\textit{Willow, willow}, was a favorite burden for songs in the sixteenth century.
There is one by John Heywood, a favorite dramatist and court musician of the
reigns of Henry VIII. and Queen Mary, beginning—
\settowidth{\versewidth}{Alas! by what mean may I make ye to know}
\begin{scverse}
\vleftofline{“}Alas! by what mean may I make ye to know\\
The unkindness for kindness that to me doth grow?”
\end{scverse}
which has for the burden—
\settowidth{\versewidth}{All a green willow; willow, willow, willow;}
\begin{scverse}
\vleftofline{“}All a green willow; willow, willow, willow;\\
All a green willow, is my garland.”
\end{scverse}
It has been printed by Mr. Halliwell, with others by Heywood, Redford, \&c., for
the Shakespeare Society, in a volume containing the moral play of \textit{Wit and
Science}.

Another with the burden—
\begin{scverse}
\vleftofline{“}Willow, willow, willow; sing all of green willow;\\
Sing all of green willow, shall be my garland,”
\end{scverse}
will be found in \textit{A Gorgious Gallery of Gallant Inventions} (1578). It commences
thus:
\settowidth{\versewidth}{My love, what misliking in me do you find,}
\begin{scverse}
\indentpattern{02020202}
\begin{patverse}
\vleftofline{“}My love, what misliking in me do you find,\\
Sing all of green willow;\\
That on such a sudden you alter your mind?\\
Sing willow, willow, willow.\\
What cause doth compel you so fickle to be,\\
Willow, willow, willow, willow;\\
In heart which you plighted most,loyal to me?\\
Willow, willow, willow, willow.”--\textit{Heliconia}, i. 32.
\end{patverse}
\end{scverse}

In Fletcher’s \textit{The two Noble Kinsmen}, when the Jailer’s daughter went mad
for love, “She sung nothing but \textit{Willow, willow, willow}.”—Act iv., sc. 1.

In the tragedy of \textit{Othello}, Desdemona introduces the song “in this pathetic
and affecting manner:”

\pagebreak

%%207
%%===============================================================================
%


\settowidth{\versewidth}{And did forsake her: she had a song of Willow;}
\begin{scverse}
\vleftofline{“}My mother had a maid call’d Barbara;\\
She was in love; and he she lov’d prov’d mad,\\
And did forsake her: she had a song of \textit{Willow};\\
And \textit{old thing} ’twas, but it express’d her fortune,\\
And she died singing it. That song to-night\\
Will not go from my mind; I have much to do,\\
But to go hang my head all at one side,\\
And sing it like poor Barbara.”
\end{scverse}

\musicinfo{Rather slow and smoothly.}{}
\lilypondfile{lilypond/207-willow-willow}\normalsize


\pagebreak

%%208
%%===============================================================================


\settowidth{\versewidth}{The true tears fell from him would have melted the stones, Sing,}
\begin{scverse}
He sigh’d in his singing, and made a great moan, Sing, \&c.;\\
I am dead to all pleasure, my true love he-is gone, \&e.

The mute bird sat by him was made tame by his moans, \&c.;\\
The true tears fell from him would have melted the stones, Sing, \&c.

Come, all you forsaken, and mourn you with me, Sing, \&c.;\\
Who speaks of a false love, mine’s falser than she, \&c.

Let love no more boast her in palace nor bower, Sing, \&c.;\\
It buds, but it blasteth ere it be a flower, \&c.

Though fair, and more false, I die with thy wound, Sing, \&c.;\\
Thou hast lost the truest lover that goes upon the ground, \&c.'

Let nobody chide her, her scorns I approve [though I prove];\\
She was born to be false, and I to die for her love, \&c.

Take this for my farewell and latest adieu, Sing, \&c.;\\
Write this on my tomb, that in love I was true, \&c.
\end{scverse}

The above copy of the words is from the same manuscript as the music. It
differs from that in Percy’s \textit{Reliques of Ancient Poetry}; and Shakespeare has
somewhat varied it to apply to a female character.

\musictitle{Whoop! Do Me No Harm, Good Man.}

This is twice alluded to by Shakespeare, in act iv., sc. 3, of \textit{A Winter’s Tale};
and by Ford, in act iii., sc. 3, of \textit{The Fancies chaste and noble}, where Secco,
applying it to Morosa, sings “Whoop! do me no harm, good \textit{woman}.”

The tune was transcribed by Dr.~Rimbault, from a MS. volume of virginal
music, in the possession of the late John Holmes, Esq., of Retford. A song with
this burden will be found in Fry’s \textit{Ancient Poetry}, but it would not be desirable
for republication.

\musicinfo{Cheerfully.}{}
\lilypondfile{lilypond/208-whoop-do-me-no-harm-good-man}\normalsize

\pagebreak

%%209
%%===============================================================================

\musictitle{O Mistress Mine!}

This tune is contained in both the editions of Morley’s \textit{Consort Lessons}, 1599
and 1611. It is also in Queen Elizabeth’s Virginal Book, arranged by Byrd.

As it is to be found in print in 1599, it proves either that Shakespeare’s \textit{Twelfth
Night} was written in or before that year, or that, in accordance with the then prevailing
custom, \textit{O Mistress mine} was an old song, introduced into the play.

Mr. Payne Collier has proved \textit{Twelfth Night} to have been an established
favorite in February, 1602 (\textit{Annals of the Stage}, i. 327), but we have no evidence
of so early a date as 1599.

In act ii., sc. 3., the Clown asks, “Would you have a love-song, or a song of
good~life?”

\textit{Sir Toby}.—“A love-song, a love-song.”

\musicinfo{Moderate time and very smoothly.}{}
\lilypondfile[staffsize=16]{lilypond/209-o-mistress-mine}\normalsize

\settowidth{\versewidth}{Then come kiss me, sweet-and-twenty,}
\indentpattern{001001}
\begin{scverse}\begin{patverse}
\vleftofline{“}What is love?—’tis not hereafter;\\
Present mirth hath present laughter;\\
What’s to come is still unsure:\\
In delay there lies no plenty;\\
Then come kiss me, sweet-and-twenty,\\
Youth’s a stuff will not endure.”
\end{patverse}
\end{scverse}

\musictitle{Heart’s-Ease.}

The tune of \textit{Heart’s-ease} is contained in a MS. volume of lute music, of the
sixteenth century in the Public Library, Cambridge (D. d., ii. 11), as well as in
\textit{The Dancing Master}, from 1650 to 1698. It belongs, in all probability, to an
earlier reign than that of Elizabeth, as it was sufficiently popular about the year
1560 to have a song written to it in the interlude of \textit{Misogonus}. Shakespeare
thus alludes to it in \textit{Romeo and Juliet}, 1597 (act iv., sc. 5.)— 

\pagebreak

%%210
%%===============================================================================

\textit{Peter}.—“Musicians, O musicians, \textit{Heart’s-ease, heart’s-ease}: O an you will have
me live, play \textit{Heart’s-ease}.

\textit{1st Mus}.—Why \textit{Heart’s-ease}?

\textit{Peter}.—O musicians, because my heart itself plays \textit{My heart is full of woe}:\dcfootnote{ %a
This is the burden of “A pleasant new Ballad of two
Lovers: to a pleasant new tune;“beginning—

\settowidth{\versewidth}{Complain my lute, complain on him}
\begin{fnverse}
\begin{altverse}
\vleftofline{“}Complain my lute, complain on him\\
That stays so long away;\\
He promised to be here ere this,\\
But still unkind doth stay.\\
But now the proverb true I find,\\
Once out of sight then out of mind.\\
Hey, ho! \textit{my heart is full of woe},” \&c.
\end{altverse}
\end{fnverse}

It has been reprinted by Mr. Andrew Barton, in the first
volume of the Shakespeare Society's Papers, 1844.
} %end footnote
O~play me some merry dump,\dcfootnote{ %
A dump was a slow dance. \textit{Queen Mary's Dump} is
one of the tunes in William Ballet’s Lute Book, and My
\textit{Lady Carey's Dompe} is printed in Stafford Smith’s \textit{Musica
Antiqua}, ii. 470, from a manuscript in the British
Museum, temp. Henry VIII.
} %end footnote
to comfort me.”

The following song is from \textit{Misogonus}, by Thomas Rychardes; and, as Mr.
Payne Collier remarks, “recollecting that it was written about the year 1560,
may be pronounced quite as good in its kind as the drinking song\dcfootnote{ %c
“I cannot eat but little meat,” see page 72.
} %end footnote
in \textit{Gammer
Gurtons Needle}.” 

\musicinfo{Moderate time.}{}
\lilypondfile{lilypond/210-hearts-ease}\normalsize

\pagebreak


%%211
%%===============================================================================

\settowidth{\versewidth}{The miser’s wealth doth hurt his health;—}

\begin{dcverse}\begin{altverse}
“What doth’t avail far hence to sail,\\
And lead our life in toiling?\\
Or to what end should we here spend\\
Our days in irksome moiling? [labour]\\
It is the best to live at rest,\\
And take’t as God doth send it;\\
To haunt each wake, and mirth to make,\\
And with good fellows spend it.
\end{altverse}

\begin{altverse}
Nothing is worse than a full purse\\
To niggards and to pinchers;\\
They always spare, and live in care,\\
There’s no man loves such flinchers.\\
The merry man, with cup and can,\\
Lives longer than do twenty;\\
The miser’s wealth doth hurt his health;—\\
Examples we have plenty.
\columnbreak
\end{altverse}

\begin{altverse}
’Tis a beastly thing to lie musing\\
With pensiveness and sorrow;\\
For who can tell that he shall well\\
Live here until the morrow?\\
We will, therefore, for evermore,\\
While this our life is lasting,\\
Eat, drink, and sleep, and ‘merry’ keep,\\
’Tis Popery to use fasting.
\end{altverse}

\begin{altverse}
In cards and dice our comfort lies,\\
In sporting and in dancing,\\
Our minds to please and live at ease,\\
And sometimes to use prancing.\\
With Bess and Nell we love to dwell\\
In kissing and in ‘talking;’\\
But whoop! ho holly, with trolly lolly,\\
To them we’ll now be walking.”
\end{altverse}

\end{dcverse}
{\footnotesize\hfill Collier’s \textit{History of Early Dramatic Poetry}, ii. 470.}

\musictitle{Jog On, Jog On.}

This tune is in \textit{The Dancing Master}, from 1650 to 1698, called \textit{Jog on}; also in
Queen Elizabeth’s Virginal Book, under the name of \textit{Hanskin}. The words of
\textit{Jog~on}, of which the first verse is sung by Autolycus, in act iv., sc. 2, of
Shakespeare’s \textit{A Winter’s Tale}, are in \textit{The Antidote against Melancholy}, 1661.
Another name for the tune is \textit{Sir Francis Drake}, or \textit{Eighty-eight}.

The following is the song from \textit{The Antidote against Melancholy}:—
\settowidth{\versewidth}{Your merry heart goes all the day}
\begin{dcverse}
\begin{altverse}
\vleftofline{“}Jog on, jog on the footpath way,\\
And merrily hent\dcfootnote{ %a
To hent or hend is to hold or seize. At the head of
one of the chapters of Sir Walter Scott’s novels, this is
misquoted “bend.”

\settowidth{\versewidth}{And in his hand a battle-axe he \textit{hent}.”—\textit{Honor of the Garter}, by George Peele.}
\begin{fnverse}% vins needed due to us already being in a verse env (?)
\vin\vleftofline{“}And in his hand a battle-axe he \textit{hent}.”—\textit{Honor of the Garter}, by George Peele.\\
\vleftofline{“}Upon the sea, till Jhesu Crist him \textit{hente}.”—Chaucer, line~700.\\
\vin\vleftofline{“}Till they the reynes of his bridel \textit{henten}.”—Chaucer, line~906.\\
\vleftofline{“}Or reave it out of the hand that did it \textit{hend}.” —Spenser’s \textit{Faery Queen}.
\end{fnverse}
} %end footnote 
the stile-a;\\
Your merry heart goes all the day;\\
Your sad tires in a mile-a.

Your paltry money-bags of gold,\\
What need have we to stare for,\\
When little or nothing soon is told.\\
And we have the less to care for.
\end{altverse}
\end{dcverse}

\vspace{-\baselineskip}

\begin{scverse}
\begin{altverse}
Cast care away, let sorrow cease,\\
A fig for melancholy;\\
Let’s laugh and sing, or, if you please,\\
We’ll frolic with sweet Dolly.”
\end{altverse}
\end{scverse}

In the \textit{Westminster Drollery}, 3rd edit., 1672, is “An \textit{old} song on the Spanish
Armado,” beginning, “Some years of late, in eighty-eight;” and in MSS. Harl.,
791, fol. 59, and in \textit{Merry Drollery complete}, 1661, a different version of the same,
commencing, “In eighty-eight, ere I was born.” Both have been reprinted for
the Percy Society in Halliwell’s \textit{Naval Ballads of England}. The former is also
in \textit{Pills to purge Melancholy}, 1707, ii. 37, and 1719, iv. 37, or Ritson’s \textit{Ancient
Songs}, 1790, p.~271.

In the Collection of Ballads in the Cheetham Library, \pagebreak Manchester, fol. 30, is 
%%212
%%===============================================================================
“The Catholick Ballad, or an Invitation to Popery, upon considerable grounds and
reasons, to the tune of \textit{Eighty-eight}.” It is in black-letter, with a bad copy of the
tune, and another (No. 1103), dated 1674. It will also be found in \textit{Pills to purge
Melancholy}, 1707, ii. 32, or 1719, iv. 32. It commences thus:—
\settowidth{\versewidth}{Since Popery of late is so much in debate,}
\begin{scverse}
\begin{altverse}
\vleftofline{“}Since Popery of late is so much in debate,\\
And great strivings have been to restore it,\\
I cannot forbear openly to declare\\
That the ballad-makers are for it.”
\end{altverse}
\end{scverse}

This song attained some popularity, because others are found to the tune of
\textit{The Catholic Ballad}.

The following are the two ballads on the Spanish Armada; the first (with the
tune) as in the Harl. MS., and the second from \textit{Westminster Drollery}.

\musicinfo{Moderate time.}{}
\lilypondfile{lilypond/212-jog-on-jog-on}\normalsize

\settowidth{\versewidth}{Don Pedroa  hight, [called] as good a knight}
\begin{dcverse}

\begin{altverse}
Spain, with Biscay and Portugal,\\
Toledo and Grenada;\\
All these did meet, and made a fleet,\\
And call’d it the Armada.
\end{altverse}

\begin{altverse}
Where they had got provision,\\
As mustard, pease, and bacon;\\
Some say two ships were full of whips,\\
But I think they were mistaken.
\end{altverse}

\begin{altverse}
There was a little man of Spain\\
That shot well in a gun-a,\\
Don Pedro\dcfootnote{ %a
The person meant by Don Pedro was the Duke of
Medina Sidonia, commander of the Spanish fleet. His
name was not Pedro, but Alonzo \textit{Perez} di Guzman.
} %end footnote
hight, [called] as good a knight\\
As the Knight of the Sun-a.
\end{altverse}

\begin{altverse}
King Philip made him admiral,\\
And charg’d him not to stay-a,\\
But to destroy both man and boy,\\
And then to run away-a.
\end{altverse}

\begin{altverse}
The King of Spain did fret amain,\\
And to do yet more harm-a;\\
He sent along, to make him strong,\\
The famous Prince of Parma.
\end{altverse}

\begin{altverse}
When they had sail’d along the seas,\\
And anchor’d upon Dover,\\
Our Englishmen did board them then,\\
And cast the Spaniards over.
\end{altverse}

\begin{altverse}
Our Queen was then at Tilbury,\\
What could you more desire-a?\\
For whose sweet sake Sir Francis Drake\\
Did set them all on fire-a.
\end{altverse}

\begin{altverse}
But let them look about themselyes,\\
For if they come again-a,\\
They shall be serv’d with that same sauce\\
As they were, I know when-a. 
\end{altverse}

\end{dcverse}

\pagebreak
%%213
%%===============================================================================

“An old song of the Spanish Armado,” called, also, in \textit{Pills to purge Melancholy},
“Sir Francis Drake: or Eighty-Eight.” To the same tune. (The words
from \textit{Westminster Drollery}, 1672.)


\settowidth{\versewidth}{They brought two ships well fraught with whips,}
\begin{dcverse}\footnotesizerr\begin{altverse}
Some years of late, in eighty-eight,\\
As I do well remember;\\
It was, some say, the nineteenth of May,\\
And some say in September.
\end{altverse}

\begin{altverse}
The Spanish train, launch’d forth amain,\\
With many a fine bravado,\\
Their (as they thought, but it proved not)\\
Invincible Armado.
\end{altverse}

\begin{altverse}
There was a little man that dwelt in Spain,\\
Who shot well in a gun-a,\\
Don Pedro hight, as black a wight\\
As the Knight of the Sun-a.
\end{altverse}

\begin{altverse}
King Philip made him admiral,\\
And bid him not to stay-a,\\
But to destroy both man and boy,\\
And so to come away-a.
\end{altverse}

\begin{altverse}
Their navy was well victualled\\
With biscuit, pease, and bacon;\\
They brought two ships well fraught with whips,\\
But I think they were mistaken.
\end{altverse}

\begin{altverse}
Their men were young, munition strong,\\
And to do us more harm-a,\\
They thought it meet to join their fleet,\\
All with the Prince of Parma.
\end{altverse}

\begin{altverse}
They coasted round about our land,\\
And so came in to Dover;\\
But we had men, set on them then,\\
And threw the rascals over.
\end{altverse}

\begin{altverse}
The Queen was then at Tilbury,\\
What could we more desire-a,\\
And Sir Francis Drake, for her sweet sake,\\
Did set them all on fire-a.
\end{altverse}

\begin{altverse}
Then straight they fled by sea and land,\\
That one man kill’d three score-a;\\
And had not they all run away,\\
In truth he had kill’d more-a.
\end{altverse}

\begin{altverse}
Then let them neither brag nor boast.\\
But if they come again-a,\\
Let them take heed they do not speed,\\
As they did, you know when-a.
\end{altverse}
\end{dcverse}

\musictitle{Come, Live With Me, And Be My Love.}

This tune, which was discovered by Sir John Hawkins, “in a MS. as old as
Shakespeare’s time,” and printed in Steevens’ edition of Shakespeare, is also contained
in “The Second Booke of Ayres, some to sing and play to the Base-Violl
alone: others to be sung to the Lute and Base-Violl,” \&c., by W. Corkine,
fol.~1612.

In act iii., sc. 1, of \textit{The Merry Wives of Windsor}, 1602, Sir Hugh Evans sings
the following lines, which form part of the song:—
\settowidth{\versewidth}{Thou, in whose groves, by Dis above,}
\begin{scverse}
\vleftofline{“}To shallow rivers, to whose falls\\
Melodious birds sing madrigals;\\
There will we make our beds of roses,\\
And a thousand fragrant posies.”
\end{scverse}
In Marlow’s tragedy, \textit{The Jew of Malta}, written in or before 1591, he introduces
the first lines of the song in the following manner:—
\begin{scverse}
“Thou, in whose groves, by Dis above,\\
Shall live with me and be my love.”
\end{scverse}
In \textit{England's Helicon}, 1600, it is printed with the name “Chr. Marlow” as the
author. It is also attributed to Marlow in the following passage from Walton’s
\textit{Angler}, 1653:—“It was a handsome milkmaid, that had not attained so much
age and wisdom as to load her mind with any fears of many things that will never
be, as too many men often do; but she cast away all care and sung like a nightingale: 
her voice was good, and the ditty fitted for it: it was that smooth song
which was made by Kit. Marlow, now at least fifty years ago.”
\pagebreak
%%214
%%===============================================================================

On the other hand, it was first printed by W. Jaggard in “The passionate
Pilgrim and other sonnets by Mr. William Shakespeare,” in 1599; but Jaggard
is a very bad authority, for he included songs and sonnets by Griffin and Barnfield
in the same collection, and subsequently others by Heywood.

\textit{England’s Helicon} contains, also, “The Nimph’s reply to the Shepheard,”
beginning—
\settowidth{\versewidth}{And we will some new pleasures prove.}
\begin{scverse}
\vleftofline{“}If all the world and love were young,\\
And truth in every shepherd’s tongue;”
\end{scverse}
which is there subscribed “Ignoto,” but which Walton attributes to Sir Walter
Raleigh, “in his younger days;” and “Another of the same nature made since,”
commencing—
\begin{scverse}
\vleftofline{“}Come, live with me, and be my deere,\\
And we will revel all the yeere,”
\end{scverse}
with the same subscription.

Dr.~Donne’s song, entitled “The Bait,” beginning—
\begin{scverse}
\vleftofline{“}Come, live with me, and be my love,\\
And we will some new pleasures prove.\\
Of golden sands and crystal brooks,\\
With silken lines and silver hooks,” \&c.
\end{scverse}
which, as Walton observes, he “made to shew the world that he could make soft
and smooth verses, when he thought smoothness worth his labour,” is also in
\textit{The Complete Angler}; and the three above quoted from \textit{England’s Helicon}, are
reprinted in Ritson’s \textit{English Songs and Ancient Songs}; and two in Percy’s
\textit{Reliques of Ancient Poetry}, \&c., \&c.

In \textit{Choice, Chance, and Change, or Conceits in their colours}, 4to., 1606,
Tidero, being invited to live with his friend, replies, “Why, how now? do you
take me for a woman, that you come upon me with a ballad of \textit{Come, live with me,
and be my love}?”

Nicholas Breton, in his \textit{Poste with a packet of Mad Letters}, 4to., 1637, says,
“You shall hear the old song that you were wont to like well of, sung by the
black brows with the cherry cheek, under the side of the pied cow, \textit{Come, live with
me, and be my love}, you know the rest.”

Sir Harris Nicholas, in his edition of Walton’s \textit{Angler}, quotes a song in imitation
of \textit{Come, live with me}, by Herrick, commencing—
\begin{scverse}
\vleftofline{“}Live, live with me, and thou shalt see;”
\end{scverse}
and Steevens remarks that the ballad appears to have furnished Milton with the
hint for the last lines of \textit{L’Allegro and Penseroso}.

From the following passage in \textit{The World’s Folly}, 1609, it appears that there
may have been an older name for the tune:—“But there sat he, hanging his
head, lifting up the eyes, and with a deep sigh, singing the ballad of \textit{Come, live
with me, and be my love}, to the tune of \textit{Adew, my deere}.”\dcfootnote{ %a
A song in Harl. MSS. 2252, of the early part of Henry
the Eighth’s reign, “Upon the inconstancy of his mistress,” begins thus;—
\settowidth{\versewidth}{Mornyng, mornyng, thus may I sing,}
\begin{fnverse}
“Mornyng, mornyng, thus may I sing, \\
\vin\vin Adew, my dere, adew.”
\end{fnverse}
It is reprinted in Ritson's \textit{Ancient Songs} (p.~98), but the
metre differs from that of \textit{Come, live with me}, and with
out repeating words, could not have been sung to	the
same air.
}

In Deloney’s \textit{Strange Histories}, 1607, is the ballad of “The Imprisonment of 
Queen Eleanor,” \&c., to the tune of \textit{Come, live with me, and be my love}, but it has 
\pagebreak
%%215
%%===============================================================================
six lines in each stanza; and “The woefull lamentation of Jane Shore,” beginning,
“If Rosamond that was so fair” (copies of which are in the Pepys, Bagford, and
Roxburghe Collections), “to the tune of \textit{Live with me}” has four lines and a
burden of~two— 
\settowidth{\versewidth}{Then maids and wives in time amend,}
\begin{scverse}
\vleftofline{“}Then maids and wives in time amend,\\
For love and beauty will have end.”
\end{scverse}
From this it appears that either the half of the tune was repeated, or that there
were two airs to which it was sung. In \textit{Westminster Drollery}, 1671 and 1674, a
parody on \textit{Come, live with me}, is to the tune of \textit{My freedom is all my joy}. That
has also six lines, and the last is repeated.

Other ballads, like “A most sorrowful song, setting forth the miserable end of
Banister, who betrayed the Duke of Buckingham, his lord and master: to the tune
of \textit{Live with me};” and the Life and Death of the great Duke of Buckingham, who
came to an untimely end for consenting to the depositing of two gallant young
princes,” \&c.: to the tune of \textit{Shore’s Wife}, have, like \textit{Come, live with me}, only
four lines in each stanza. (See \textit{Crown Garland of Golden Roses}, 1612; and
Evans’ \textit{Old Ballads}, iii. 18 and~23.)

\musicinfo{Rather slow.}{}
\lilypondfile{lilypond/215-come-live-with-me-and-be-my-love}\normalsize

\settowidth{\versewidth}{And see the shepherds feed their flocks,}
\begin{dcverse}There will we sit upon the rocks,\\
And see the shepherds feed their flocks,\\
By shallow rivers, to whose falls\\
Melodious birds sing madrigals.

There will I make thee beds of roses,\\
And twine a thousand fragrant posies;\\
A cap of flowers, and a kirtle,\\
Embroider’d all with leaves of myrtle.

A gown made of the finest wool,\\
Which from our pretty lambs we pull;\\
Slippers lined choicely for the cold,\\
With buckles of the purest gold.

A belt of straw and ivy buds,\\
With coral clasps and amber studs:\\
And if these pleasures may thee move,\\
Come, live with me, and be my love.
\end{dcverse}


\begin{scverse}The shepherd swains shall dance and sing,\\
For thy delight, each May morning;\\
If these delights thy mind may move,\\
Then live with me, and be my love.
\end{scverse}

\footnotetext[1]
{In Sir John Hawkins’ copy, this note is written an
octave lower, probably because taken from a lute arrangement,
in which the note, being repeated, was to be played
on a lower string. In the second bar of the melody, his
copy, if transposed into this key, would be B \underline{A} D, instead
of B \underline{C} D; which latter seems right by the analogy of that
and the other phrases, although the difference is not very
material.}


\pagebreak
%%216
%%===============================================================================
\DFNsingle
\musictitle{Three Merry Men Be We.}

This is quoted in the same passage in\textit{ Twelfth Night} as \textit{Peg-a-Ramsey}. The tune
is contained in a MS. common-place book, in the handwriting of John Playford,
the publisher of \textit{The Dancing Master}, in the possession of the Hon. George
O’Callaghan.” The words are also in Peele’s \textit{The Old Wives’ Tale}, 1595 (Dyce,
i.~208), where it is sung instead of the' song proposed, \textit{O man in desperation}.

In the comedy of \textit{Laugh and lie down}, 1605, “He plaied such a song of the
\textit{Three Merry Men}.” In Fletcher’s \textit{The Bloody Brother}, the Cook, who is about to
be hung with two others, says:
\settowidth{\versewidth}{This hasty work was ne’er done well: give us so much time}
\begin{scverse}
\vleftofline{“}Good Master Sheriff, your leave too;\\
This hasty work was ne’er done well: \textit{give us so much time}\\
\textit{As but to sing our own ballads}, for we’ll trust no man,\\
Nor no tune but our own; ’twas done in ale too,\\
And therefore cannot be refus’d in justice:\\
\textit{Your penny-pot poets are such pelting thieves,\\
They ever hang men twice}.”
\end{scverse}
Each then sings a song, and they join in the chorus of—
\settowidth{\versewidth}{Three merry boys, and three merry boys,}
\begin{scverse}
\vleftofline{“}Three merry boys, and three merry boys,\\
And three merry boys are we,\\
As ever did sing in a hempen string\\
Under the gallow tree.”—\textit{Act iii., sc. 2, Dyce},--- x. 428.
\end{scverse}
“Three merry men be we” is also quoted in \textit{Westward Hoe}, by Dekker and
Webster, 1607; and in \textit{Ram Alley}, 1611.

\musicinfo{Moderate time and gaily.}{}
\lilypondfile{lilypond/216-three-merry-men-be-we}\normalsize

\musictitle{I Loathe That I Did Love.}

On the margin of a copy of the Earl of Surrey’s poems, in the possession of
Sir W. W. Wynne, some of the little airs to which his favorite songs were sung
are written in characters of the times. Dr.~Nott printed them from that copy in
his edition of Surrey’s \textit{Songs and Sonnets},\dcfootnote{ %a
\centering The music was added after a portion of the edition had been circulated.
} %end footnote
4to., 1814. From this the first tune
for “I loathe that I did love” is taken. The second is from a MS. containing
songs to the lute, in the British Museum (Addit. 4900), but it is more like the
regular composition of a musician than the former. 

\DFNdouble
\pagebreak
%%217
%%===============================================================================

Three stanzas from the poem are sung by the grave-digger in \textit{Hamlet}; but
they are much corrupted, and in all probability designedly, to suit the character
of an illiterate clown. On the stage the grave-digger now sings them to the tune
of \textit{The Children in the Wood}.

In the \textit{Gorgious Gallery of Gallant Inventions}, 1578, “the lover complaineth
of his lady’s inconstancy; to the tune of \textit{I lothe that I did love},” therefore a tune
was formerly known by that name, and probably one of the two here printed.

The song will be found among the ballads that illustrate Shakespeare, in Percy’s
\textit{Reliques of Ancient Poetry}.

\musicinfo{Slow.}{First tune.}
\lilypondfile{lilypond/217-i-loathe-that-i-did-love}\normalsize

\musicinfo{Slow.}{Second tune.}
\lilypondfile{lilypond/217-i-loathe-that-i-did-love-2}\normalsize

\pagebreak
%218
%%===============================================================================

\musictitle{Peg A Ramsey, Or Peggie Ramsey.}

In \textit{Twelfth Night}, act ii., sc. 3, Sir Toby says, “Malvolio’s a \textit{Peg-a-Ramsey},
and \textit{Three merry men be we}.” There are two tunes under the name of \textit{Peg-a-Ramsey},
and both as old as Shakespeare’s time. The first is called \textit{Peg-a-Ramsey}
in William Ballet’s Lute Book, and is given by Sir John Hawkins as the tune
quoted in \textit{Twelfth Night}. (See Steevens’ edition of Shakespeare.) He says,
“Peggy Ramsey is the name of some old song;” but, as usual, does not cite his
authority. It is mentioned as a dance tune by Nashe (see the passage quoted at
p.~116), and in \textit{The Shepheard’s Holiday}—


\settowidth{\versewidth}{And I am sure thou there shall find}
\begin{dcverse}
\vleftofline{“}Bounce it Mall, I hope thou will,\\
For I know that thou hast skill;\\
And I am sure thou there shall find\\
Measures store to please thy mind.\\
Roundelays—Irish hayes;\\
Cogs and Rongs, and \textit{Peggie Ramsy};\\
Spaniletto—The Venetto;\\
\textit{John come kiss me}—Wilson’s Fancy.\\
But of all there’s none so sprightly\\
To my ear, as \textit{Touch me lightly}.”\\
\vin\vin\vin \textit{Wit’s Recreations}, 1640.
\end{dcverse}

“Little Pegge of Ramsie” is one of the tunes in a manuscript by Dr.~Bull, which
formed a part of Dr.~Pepusch’s, and afterwards of Dr.~Kitchener’s library. Ramsey,
in Huntingdonshire, was formerly an important town, and called “Ramsey the
rich,” before the destruction of its abbey.

Burton, in his \textit{Anatomy of Melancholy}, says, “So long as we are wooers, we
may kiss at our pleasure, nothing is so sweet, we are in heaven as we think; but
when we are once tied, and have lost our liberty., marriage is an hell. ‘\textit{Give me
my yellow hose again}:’ a mouse in a trap lives as merrily.”

“Give me my yellow hose” is the burden of a ballad called—
\settowidth{\versewidth}{A merry jest of John Tomson, and Jackaman his wife,}
\begin{scverse}
\vleftofline{“}A merry jest of John Tomson, and Jackaman his wife,\\
Whose jealousy was justly the cause of all their strife;”
\end{scverse}
to the tune of \textit{Pegge of Ramsey}; beginning thus—

\settowidth{\versewidth}{But now I am a married man}
\begin{dcverse}
\begin{altverse}
\vleftofline{“}When I was a bachelor\\
I led a merry life,\\
But now I am a married man\\
And troubled with a wife,

I cannot do as I have done,\\
Because I live in fear;\\
If I go but to Islington,\\
My wife is watching there.
\end{altverse}
\end{dcverse}

\settowidth{\versewidth}{For now my wife she watcheth me,}
\begin{scverse}
\begin{altverse}
\textit{Give me my yellow again}.\\
Give me my yellow hose,\\
For now my wife she watcheth me,\\
See yonder where she goes.”
\end{altverse}
\end{scverse}
It has been reprinted in Evans’ \textit{Old Ballads}, i. 187 (1810.)

In \textit{Wit and Mirth, or\textit{ Pills to purge Melancholy}} (1707, iii. 219, or 1719,
v. 139), there is a song called “\textit{Bonny Peggy Ramsey},” to the second tune,
which in earlier copies is called \textit{O London is a fine town}, and \textit{Watton Town’s End}.

The original song, “Oh! London is a fine town,” is probably no longer extant.
A ballad to be sung to the tune was written on the occasion of James the First’s
visit to Cambridge, in March, 1614— 
\pagebreak
%%219
%%===============================================================================

\settowidth{\versewidth}{The Mayor and some few Aldermen}
\begin{scverse}\begin{altverse}
\vleftofline{“}Cambridge is a merry town,\\
And Oxford is another,\\
The King was welcome to the one,\\
And fared well at the other,” \&c.\\
\vin\vin\vin\vin See Hawkins’ \textit{Ignoramus}, xxxvi.
\end{altverse}
\end{scverse}
A second with the burden—
\begin{scverse}
\begin{altverse}
\vleftofline{“}London is a fine town,\\
Yet I their cases pity;\\
The Mayor and some few Aldermen\\
Have clean undone the city,”
\end{altverse}
\end{scverse}
will he found in the King’s Pamphlets, British Museum (fol. broadsides, vol. v.).
It begins, “Why kept your train-bands such a stir,” and is dated Aug. 13,1647.
(Reprinted in Wright’s \textit{Political Ballads}, for the Percy Society.)

In \textit{Le Prince d’Amour}, 12m., 1660, is a third, commencing thus:—
\settowidth{\versewidth}{Governed with scarlet gowns; give ear unto my ditty:}
\begin{scverse}\vleftofline{“}London is a fine town, and a brave city,\\
Governed with scarlet gowns; give ear unto my ditty:\\
And there is a Mayor, which Mayor he is a Lord,\\
That governeth the city by righteous record.\\
Upon Simon and Jude’s day their sails then up they hoist,\\
And then he goes to Westminster with all the galley foist.\\
\vin\vin\vin\vin London is a fine town,” \&c.
\end{scverse}
A fourth song beginning, “\textit{Oh!} London is a fine town,” will he found in \textit{Pills to
purge Melancholy}, 1707, ii. 40, or 1719, iv. 40; and in the same volume another
to the tune, beginning—
\settowidth{\versewidth}{As I came from Tottingham,}
\begin{dcverse}\begin{altverse}
\vleftofline{“}As I came from Tottingham,\\
Upon a market day,\\
There I met a bonny lass\\
Clothed all in gray.
\end{altverse}

\begin{altverse}
Her journey was to London\\
With buttermilk and whey,\\
\textit{To come down, a down,\\
To come down, down, a down-a}.”
\end{altverse}
\end{dcverse}
The burden, to this song suggests the possibility of its being the tune of a snatch
sung by Ophelia in \textit{Hamlet}—
\settowidth{\versewidth}{You must sing down, a down,}
\begin{scverse}\vleftofline{“}You must sing down, a down,\\
An you call him a down-a.”
\end{scverse}
One of D’Urfey’s “Scotch” Songs, called \textit{The Gowlin}, in his play of \textit{Trick for
Trick}, was also sung to this tune.

In \textit{The Dancing Master}, 1665 and after, it is called \textit{Watton Town’s End}; and
in the second part of \textit{Robin Goodfellow}, 1628, there is a song “to the tune of
\textit{Watton Town’s End},” beginning—
\begin{scverse}\vleftofline{“}It was a country lad,\\
That fashions strange would see,” \&c.
\end{scverse}
It is reprinted in Evans’\textit{Old Ballads} 1810, i. 200. Another entitled—
\begin{scverse}\vleftofline{“}\vleftofline{“}The common cries of London town,\\
Some go up street, some go down,”
\end{scverse}
is to the tune of \textit{Watton Townes End}, black-letter, 1662.

Many others will he found to these tunes, under their various names.

The following is a verse from the ballad quoted in Burton’s \textit{Anatomy of
Melancholy}. It consists of eighteen stanzas, each of eight lines, and a ditty of
four (“Give me my yellow hose again,” \&c.). See Evans’ \textit{Old Ballads}. 
\pagebreak
%%220
%%===============================================================================

\musicinfo{Moderate time.}{}
\lilypondfile{lilypond/220-peg-a-ramsey-or-peggie-ramsey}\normalsize

There are slight differences in the copies of the tune called \textit{Watton Town's End}
in \textit{The Dancing Master}, and \textit{Oh! London is a fine town} in \textit{Pills to purge Melancholy},
and in \textit{The Beggars’ Opera}. The following is \textit{The Beggars’ Opera} version:—

\musicinfo{Lively.}{}
\lilypondfile{lilypond/220-peg-a-ramsey-or-peggie-ramsey-2}\normalsize

\pagebreak


%%221
%%===============================================================================

\musictitle{Light O’love.}

\textit{Light of Love} is so frequently mentioned by writers of the sixteenth century,
that it is much to be regretted that the words of the original song are still
undiscovered. When played slowly and with expression the air is beautiful. In
the collection of Mr. George Daniel, of Canonbury, is “A \textit{very proper} dittie: to
the tune \textit{Lightie Love};” which was printed in 1570. The original may not have
been quite so “proper,” if “Light o’Love” was used in a sense in which it was
occasionally employed, instead of its more poetical meaning:—
\settowidth{\versewidth}{One of your \textit{London Light o'Loves}, a right one.}
\begin{scverse}
\vleftofline{“}One of your \textit{London Light o'Loves}, a right one.\\
Come over in thin pumps, and half a petticoat.”\\
\attribution Fletcher’s \textit{Wild Goose Chase}, act iv., sc. 2.
\end{scverse}

Or in the passage quoted by Douce: “There be wealthy housewives and good
housekeepers that use no starch, but fair water; their linen is as white, and they
look more Christian-like in small ruffs than \textit{Light of Love} looks in her great
starched ruffs, look she never so high, with her eye-lids awry.”—\textit{The Glasse of
Man's Follie},~1615.

Shakespeare alludes twice to the tune. Firstly in \textit{The Two Gentlemen of Verona},
act i., sc. 2—

\begin{scverse}
\vleftofline{“\textit{Julia.} }Some love of yours hath writ to you in rhime.\\
\vleftofline{\textit{Lucetta.} }That I might sing it, madam, to a tune:\\
\textit{Give me a note:—your ladyship can set}.\\
\vleftofline{\textit{Jul. }}As little by such toys as may be possible:\\
Best sing it to the tune of \textit{Light o'Love}.\\
\vleftofline{\textit{Luc. }}It is too heavy for so light a tune.\\
\vleftofline{\textit{Jul. }}Heavy? belike it hath some \textit{burden} then.\\
\vleftofline{\textit{Luc. }}Ay; and melodious were it would you sing it.\\
\vleftofline{\textit{Jul. }}And why not you?\\
\vleftofline{\textit{Luc. }}I cannot reach so high.\\
\vleftofline{\textit{Jul. }}Let’s see your song:—How now, minion?\\
\vleftofline{\textit{Luc. }}Keep tune there still, so you will sing it out:\\
And yet, methinks, I do not like this tune.\\
\vleftofline{\textit{Jul.}} You do not?\\
\vleftofline{\textit{Luc.}} No, madam; ’tis too sharp.\\
\vleftofline{\textit{Jul.}} You, minion, are too saucy.\\
\vleftofline{\textit{Luc.}} Nay, now you are too flat,\\
And mar the concord with too harsh a \textit{descant}:\\
There wanteth but a \textit{mean} to fill your song.\\
\vleftofline{\textit{Jul.}} The mean is drown’d with your unruly \textit{base}.’’
\end{scverse}

I have quoted this passage \textit{in extenso} as bearing upon the state of music at the
time, beyond the mere mention of the tune. Firstly, when Lucetta says, “Give
me a note [to sing it to]: your ladyship can set” [a song to music,] it adds one
more to the many proofs of the superior cultivation of the science in those days.
We should not now readily attribute to ladies, even to those who are generally  
considered to be well educated and accomplished, \pagebreak enough knowledge of 
%%222
%%===============================================================================
\DFNsingle
harmony to enable them to set a song correctly to music, however agile their
fingers may be. Secondly—
\settowidth{\versewidth}{Heavy? belike it hath some burden then!}
\begin{scverse}
\vleftofline{“}It is too heavy for so light a tune,\\
Heavy? belike it hath some burden then!”
\end{scverse}

The burden of a song, in the old acceptation of the word, was the base, foot, or
under-song. It was sung throughout, and not merely at the end of the verse.
Burden is derived from \textit{bourdoun}, a drone base (French, \textit{bourdon}.)

\begin{scverse}
“This Sompnour bare to him a stiff burdoun,\\
Was never trompe of half so gret a soun.”—\textit{Chaucer}.
\end{scverse}

We find as early as 1250, that \textit{Somer is icumen in} was sung with a foot, or burden,
in two parts throughout ( “Sing cuckoo, sing cuckoo” ); and in the preceding
century Giraldus had noticed the peculiarity of the English in singing under-parts
to their songs.


That burden still bore the sense of an under-part or base, and not merely of a
ditty,\dcfootnote{ %a
“Ditties, they are the ends of old ballads.”—Rowley’s \textit{A Match at Midnight}, act iii., sc. 1.
}
see \textit{A Quest of Inquirie}, \&c., 4to., 1595, where it is compared to the music
of a tabor:—“Good, people, beware of wooers’ promises, they are like the musique
of a tabor and pipe: the pipe says golde, giftes, and many gay things; but performance
is moralized in the tabor, which \textit{bears the burden} of ‘I doubt it, I doubt it.’—
(\textit{British Bibliographer}, vol. i.) In Fletcher’s \textit{Humorous Lieutenant}, act v., sc. 2,
“H’as made a thousand rhymes, sir, and plays the \textit{burden} to 'em on a Jew’s-
trump” (\textit{Jeugd-tromp}, the Dutch for a child’s horn). So in \textit{Much Ado about
Nothing}, in the scene between Hero, Beatrice, and Margaret, the last says, “Clap
us into \textit{Light o'Love}, that goes without a burden” [there being no man or men
on the stage to sing one]. “Do you sing it and I’ll dance it.” \textit{Light o'Love}
was therefore strictly a \textit{ballet}, to he sung and danced.

In the interlude of \textit{The Four Elements}, about 1510, Ignorance says—
\settowidth{\versewidth}{\textit{But there is a bordon, thou must bear it},}
\begin{scverse}
\vleftofline{“}But if thou wilt have a song that is good,\\
I have one of Robin Hood,\\
The best that ever was made.\\
\vleftofline{\textit{Humanity}. }Then i’ fellowship, let us hear it.\\
\vleftofline{\textit{Ign}. }\textit{But there is a bordon, thou must bear it},\\
\textit{Or else it will not be}.\\
\vleftofline{\textit{Hum}. }Then begin and care not to \dots\\
Downe, downe, downe, \&c.\\
\vleftofline{\textit{Ign}. }Robin Hood in Barnsdale stood,” \&c.
\end{scverse}

%\setlength{\DFNcolumnwidth}{0.5\textwidth}
%\addtolength{\DFNcolumnwidth}{-0.5\columnsep}

Here Humanity starts with the burden, giving the key for the other to sing in.
So in old manuscripts, the burden is generally found at the head of the song, and
not at the end of the first verse.

Many of these burdens were short proverbial expressions, such as— .
\begin{scverse}
“ ‘Tis merry in hall when beards wag all;”
\end{scverse}
which is mentioned as the “under-song or holding” of one in \textit{The Serving-man's
Comfort}, 1598, and the line quoted by Adam Davy, in his \textit{Life of Alexander}, as
early as about 1312. Peele, in his \textit{Edward I}., speaks of it as “the old 
\pagebreak
\DFNdouble 
%%223
%%===============================================================================
English proverb but he uses the word “proverb” also in the sense of song, for
in his \textit{Old Wives’ Tale}, 1595, Antick says, “Let us rehearse the old proverb—
\settowidth{\versewidth}{‘Three merry men and three merry men,}
\begin{scverse}
‘Three merry men and three merry men,\\
And three merry men be we,’” \&c.
\end{scverse}
Shakespeare puts the following four lines into the mouth of Justice Silence when
in his cups:—
\begin{scverse}
“Be merry, be merry, my wife has all,\\
For women are shrews, both short and tall;\\
\textit{’Tis merry in hall, when beards wag all}.\\
\vin And welcome merry Shrovetide.”
\end{scverse}
See also Ben Jonson, v. 235, and note; and vii. 273, Gifford’s edit.

Other burdens were mere nonsense words that went glibly off the tongue, giving
the accent of the music, such as \textit{hey nonny, nonny no; hey derry down}, \&c. The
“foot” of the first song in \textit{The pleasant Comedy of Patient Grissil} is—
\begin{scverse}
“Work apace, apace, apace, apace,\\
Honest labour bears a lovely face;\\
Then hey noney, noney; hey noney, noney.”
\end{scverse}
I am aware that “Hey down, down, derry down,” has been said to be “a modern
version of ‘Hai down, ir deri danno,’ the burden of an old song of the Druids,
signifying, ‘Come, let us hasten to the oaken grove’ (Jones’ \textit{Welsh Bards}, i. 128);
but I believe this to be mere conjecture, and that it would now be impossible to
prove that the Druids had such a song.

The last comment I have to make upon the passage from Shakespeare is on the
word \textit{mean}. The mean in music was the intermediate part between the tenor and
treble; not the tenor itself, as explained by Steevens. Descant has already been
explained at p.~15.

Reverting to \textit{Light o’Love}: it is also quoted as a tune by Fletcher in \textit{The Two
Noble Kinsmen}, The air was found by Sir J. Hawkins in an “ancient manuscript;” 
it is also contained in William Ballet’s MS. Lute Book, and in \textit{Musick’s
Delight on the Cithren},~1666.

In the volume of transcripts made by Sir John Hawkins there is a tune entitled
\textit{Fair Maid are you walking}, the first four bars of which are identical with \textit{Light
o'Love}; and in the Music School, Oxford, one of the manuscripts presented by
Bishop Fell, with a date 1620, has \textit{Light o'Love} under the name of \textit{Sicke and sicke
and very sicke}; but this must be a mistake, as that ballad could not be sung to it.
See \textit{Captain Car} in Ritson’s \textit{Ancient Songs}, 1790, p.~139.

In \textit{A Gorgious Gallery of Gallant Inventions}, 1578, the lover exhorteth his
lady to be constant: to the tune of \textit{Attend thee, go play thee};\dcfootnote{ %a
“Attend thee, go play thee,” is a song in A Hand\-efull
of Pleasant Delites, 1584, and is also the tune of one sung
by Wantonness in the interlude of \textit{The Marriage of Wit
and, Wisdom}. See Shakespeare Society’s Reprint, p.~20.
} %end footnote
“and begins with
the line, “Not \textit{Light o'Love}, lady.” The ballad, “The Banishment of Lord Maltravers
and Sir Thomas Gurney,” in Deloney’s \textit{Strange Histories}, \&c., 1607, and of
“A song of the wooing of Queen Catherine by Owen Tudor, a young gentleman
of Wales” are also to the tune of \textit{Light o'Love}. See \textit{Old Ballads}, 1727, iii. 32;
or Evans, ii. 356.

The following is the ballad by Leonard Gybson, a copy of which is in Mr.
George Daniel’s Collection. 

\pagebreak
%%224
%%===============================================================================

\musictitle{A Very Proper Dittie: To The Tune Of Lightie Love.}

\settowidth{\versewidth}{Leave lightie love, ladies, for fear ofyll name:}
\begin{scverse}\vleftofline{“}Leave lightie love, ladies, for fear of yll name:\\
And true love embrace ye, to purchase your fame.”
\end{scverse}

\musicinfo{Very slow and smoothly.}{}
\lilypondfile[staffsize=15]{lilypond/224-light-o-love}\normalsize

\begin{dcverse}\scriptsizerrr
Deceit is not dainty, it comes at each dish;\\
Fraud goes a fishing with friendly looks;\\
Through friendship is spoiled, the silly poor fish\\
That hover and shower upon your false hooks,\\
With bait you lay wait, to catch here and there,\\
Which causeth poor fishes their freedom to lose;\\
Then lout ye, and flout ye;—whereby doth appear,\\
Your lighty love, ladies, still cloaked with glose.

With Dian so chaste you seem’d to compare,\\
When Helens you be, and hang on her train;\\
Methinks faithful Thisbes be now very rare,\\
But one Cleopatra, I doubt, doth remain.\\
You wink, and you twink, until Cupid have caught,\\
And forceth through flames your lovers to sue:\\
Your lighty love, ladies, too dear they have bought, \\
When nothing will move you their causes to rue.

I speak not for spite, nor do I disdain\\
Your beauty, fair ladies, in any respect;\\
But one’s ingratitude doth me constrain,\\
As child hurt with fire, the flame to neglect.\\
For, proving in loving, I find by good trial,\\
When Beauty had brought me unto her beck,\\
She staying, not weighing, but making denial,\\
And shewing her lighty love, gave me the check.

Thus fraud for friendship did lodge in her breast;\\
Such are most women, that when they espy\\
Their lovers inflamed, with sorrows opprest,\\
They stand then with Cupid against their reply.\\
They taunt, and they vaunt, they smile when they view\\
How Cupid hath caught them under his train;\\
But warned, discerned, the proof is most true,\\
That lighty love, ladies, amongst you does reign.

Ye men that are subject to Cupid his stroke,\\
And therein seem now to have your delight,\\
Think, when you see bait, there is hidden a hook,\\
Which surely will have you, if that you do bite.\\
Such wiles, and such guiles by women are wraught,\\
That half of their mischiefs men cannot prevent;\\
When they are most pleasant, unto your thought,\\
Then nothing but lighty love is their intent.

Consider that poison doth lurk often time\\
In shape of sugar, to put some to pain;\\
And fair wordès painted, as dames can define,\\
The old proverb saith, doth make some fools fain.\\
Be wise and precise, take warning by me,\\
Trust not the crocodile, lest you do rue;\\
To women’s fair words do never agree,\\
For all is but lighty love;—this is most true. 
\end{dcverse}
\pagebreak
%%225
%%===============================================================================

\begin{dcverse}\footnotesize
I touch no such ladies as true love embrace,\\
But such as to lighty love daily apply;\\
And none will be grieved, in this kind of case,\\
Save such as are minded true love to deny.\\
Yet friendly and kindly I shew you my mind:\\
Fair ladies, I wish you to use it no more;\\
But say what you list, thus I have defin’d\\
That lighty love, ladies, you ought to abhor.

To trust women’s words, in any respect,\\
The danger by me right well it is seen;\\
And Love and his laws, who would not neglect,\\
The trial whereof hath most perilous been?\\
Pretending, the ending, if I have offended,\\
I crave of you, ladies, an answer again:\\
Amend, and what’s said shall soon be amended,\\
If case that your light love no longer do reign.
\end{dcverse}

%\backskip{2}

\musictitle{When That I Was A Little Tiny Boy.}

The Fool’s song which forms the Epilogue to \textit{Twelfth Night} is still sung on the
stage to this tune. It has no other authority than theatrical tradition. A song
of the same description, and with the same burden, is sung by the Fool in \textit{King
Lear}, act iii., sc.~2—

%\vspace{-0.5\baselineskip}

\begin{scverse}\begin{altverse}
“He that has a little tiny wit,\\
\textit{With a heigh ho! the wind and the rain},\\
Must make content with his fortunes fit,\\
\textit{For the rain it raineth every day}.”
\end{altverse}
\end{scverse}

%\vspace{-0.5\baselineskip}

The following is the song in \textit{Twelfth Night}:—

\musicinfo{In moderate time.}{}
\lilypondfile[staffsize=16]{lilypond/225-when-that-i-was-a-little-tiny-boy}\normalsize

\indentpattern{0404}
\begin{dcverse}\begin{patverse}
But when I came to man’s estate,\\
With a heigh ho! \&c.,\\
’Gainst knaves and thieves men shut their gate,\\
For the rain, \&c.
\end{patverse}

\begin{patverse}
But when I came, alas! to wive,\\
With a heigh ho! \&c.,\\
By swaggering I could never thrive,\\
For the rain, \&c.
\end{patverse}

\begin{patverse}
But when I came unto my bed,\\
With a heigh ho! \&c.,\\
With toss-pots still I’d drunken head,\\
For the rain, \&c.
\end{patverse}

\indentpattern{0101}
\begin{patverse}
A great while ago the world begun,\\
With a heigh ho! the wind and the rain;\\
But that is all one, our play is done,\\
And we’ll strive to please you every day. 
\end{patverse}
\end{dcverse}
\pagebreak
%%226
%%===============================================================================

\musictitle{Sick, Sick, And Very Sick.}

This tune is contained in Anthony Holborne’s \textit{Cittharn Schoole}, 4to., 1597, and
in one of the Lute MSS. in the Public Library, Cambridge. (D. d. iv. 23.) In
\textit{Much Ado about Nothing}, Hero says, “Why, how now! do you speak in the \textit{sick
tune}?” and Beatrice answers, “I am out of all other tune, methinks.” In
Nashe’s \textit{Summer’s last Will and Testament}, Harvest says, “My mates and fellows,
sing no more \textit{Merry, merry}, but weep out a lamentable \textit{Hooky, hooky}, and let your
sickles cry—
\settowidth{\versewidth}{For Harvest, your master, is}
\begin{scverse}
\begin{altverse}
Sick, sick, and very sick,\\
And sick and for the time;\\
For Harvest, your master, is\\
Abus’d without reason or rhyme.”
\end{altverse}
\end{scverse}

On 24th March, 1578, Richard Jones, had licensed to him “a ballad intituled
\textit{Sick, sick}, \&c., and on the following 19th June, “A new songe, intituled—
\settowidth{\versewidth}{For grief to see this wicked world, that will not mend, I fear.”}
\begin{scverse}
\textit{Sick, sick, in grave I would I were}.\\
For grief to see this wicked world, that will not mend, I fear.”
\end{scverse}
This was probably a moralization of the former.

In the \textit{Harleian Miscellany}, 4to, 10. 272, is “A new ballad, declaring the
dangerous shooting of the gun at the court (1578), to the tune of \textit{Sicke and sicke};
commencing—
\settowidth{\versewidth}{The seventeenth day of July last,}
\begin{dcverse}\settowidth{\versewidth}{And had the watermen to row,}
\begin{altverse}
\vleftofline{“}The seventeenth day of July last,\\
At evening toward night,\\
Our noble Queen Elizabeth\\
Took barge for her delight;\\
And had the watermen to row,\\
Her pleasure she might take,
\end{altverse}

\settowidth{\versewidth}{To think upon the gun was shot}
\indentpattern{012323}
\begin{patverse}
About the river to and fro,\\
As much as they could make.\\
Weep, weep, still I weep,\\
And shall do till I die,\\
To think upon the gun was shot\\
At court so dangerously.”
\end{patverse}
\end{dcverse}

The ballad from which the tune derives its name is probably that printed in
Ritson’s \textit{Ancient Songs}, (1793, p.~139) from a manuscript in the Cotton Library
(Vespasian, A 25), and entitled \textit{Captain Car}. The event which gave rise to it
occurred in the year 1571. The first stanza is here printed to the tune:—

\musicinfo{}{}
\lilypondfile{lilypond/226-sick-sick-and-very-sick}\normalsize

\pagebreak
%%227
%%===============================================================================

\musictitle{To-Morrow Is St. Valentine’s Day.}

This is one of Ophelia’s songs in \textit{Hamlet}. It is found in several of the ballad
operas, such as \textit{The Cobblers’ Opera} (1729), \textit{The Quakers’ Opera} (1728), \&c.,
under this name. In \textit{Pills to purge Melancholy} (1707, ii. 44) it is printed to a
song in \textit{Heywood’s Rape of Lucrece}, beginning, “Arise, arise, my juggy, my
puggy.” Other versions will be found under the names of “Who list to lead
a soldier’s life,” and “Lord Thomas and Fair Ellinor.” See pages 144 and 145.

\musicinfo{Cheerfully.}{}
\lilypondfile{lilypond/227-tomorrow-is-st-valentines-day}\normalsize

\musictitle{Green Sleeves.}

\textit{Green Sleeves}, or \textit{Which nobody can deny}, has been a favorite tune, from the
time of Elizabeth to the present day; and is still frequently to be heard in the
streets of London to songs with the old burden, “Which nobody ean deny.” It
will also be recognised as the air of \textit{Christmas comes but once a year}, and many
another merry~ditty.

“And set our credits to the tune of \textit{Greene Sleeves}.”—\textit{The Loyal Subject}, by
Beaumont and Fletcher.

\textit{Falstaff}.—“Let the sky rain potatoes! let it thunder to the tune of \textit{Green Sleeves},
hail kissing comfits, and snow eringoes, let there come a tempest of provocation, I will
shelter me here.” (\textit{Embracing her}.)—\textit{Merry Wives of Windsor}, act v., sc. 5.

“\textit{Mrs. Ford}.—“I shall think the worse of fat men, as long as I have an eye to
make difference of men’s liking. And yet he would not swear; praised women’s
modesty; and gave such orderly and well-behaved reproof to all uncomeliness, that
I would have sworn his disposition would have gone to the truth of his words: but
they do no more adhere and keep pace together, than the Hundredth Psalm to the
tune of \textit{Green Sleeves}.”—\textit{Merry Wives of Windsor}, act ii., sc. 1.

The earliest mention of the ballad of \textit{Green Sleeves} in the Registers of the
Stationers’ Company is in September, 1580, when Richard Jones had licensed to
him, “A new Northern Dittye of the \textit{Lady Greene Sleeves}.” The date of the
entry, however, is not always the date of the ballad; and this had evidently
attained some popularity before that time, because on the same day Edward 
 \pagebreak
%%228
%%===============================================================================
White had a license to print, “A ballad, being the Ladie Greene Sleeves \textit{Answere}
to Donkyn his frende.” Also Edward Guilpin in his \textit{Skialethia, or a Shadow of
Truth}, 1598, says:
\settowidth{\versewidth}{“Yet like th’ olde ballad of the \textit{Lord of Lorne},}
\begin{scverse}
\vleftofline{“}Yet like th’ olde ballad of the \textit{Lord of Lorne},\\
Whose last line\dcfootnote{ %a
The last lines of the Lord of Lorne are—
\settowidth{\versewidth}{Let Rebels therefore warned be,}
\begin{fnverse}
\begin{altverse}
\vleftofline{“}Let Rebels therefore warned be,\\
How mischief once they do pretend;

For God may suffer for a time,\\
But will disclose it at the end.”
\end{altverse}
\end{fnverse}

Perhaps Guilpin may mean that this formed part of an
older balled.
}
in King Harries days was borne.”
\end{scverse}

As the ballad of \textit{The Lord of Lorne and the False Steward}, which was entered on
the 6th October, 1580, was sung to the tune of \textit{Green Sleeves}, it would appear that
\textit{Green Sleeves} must be a tune of Henry’s reign. Copies of \textit{The Lord of Lorne} are in
the Pepys Collection (i. 494), and the Roxburghe (i. 222).

Within twelve days of the first entry of \textit{Green Sleeves} it was converted to a
pious use, and we have, “\textit{Greene Sleves} moralised to the Scripture, declaring the
manifold benefites and blessings of God bestowed on sinful man;” and on the
fifteenth day Edward White had “tollerated unto him by Mr. Watkins, a
ballad intituled Greene Sleeves and Countenance, in Countenance is Greene
Sleeves.” By the expression “tolerated” instead of “licensed,” we may infer
it to have been of questionable propriety.

Great, therefore, was the popularity of the ballad immediately after its publication, 
and this may he attributed rather to the merry swing of the tune, than to the
words, which are neither remarkable for novelty of subject, nor for its treatment.

An attempt was speedily made to improve upon them, or to supply others of
more attractive character, for in December of the same year, Jones, the original
publisher, had “tolerated to him A merry newe Northern Songe of \textit{Greene
Sleeves},” beginning, \textit{The bonniest lass in all the land}. This was probably the ballad
that excited William Elderton to write his “Reprehension against Greene Sleeves”
in the following February, for there appears nothing in the original song to have
caused it. The seventh entry within the year was on the 24th of August, 1581,
when Edward White had licensed “a ballad intituled—
\settowidth{\versewidth}{Greene Sleeves is worne awaie,}
\begin{scverse}
\vleftofline{“}Greene Sleeves is worne awaie,\\
Yellow Sleeves come to decaie.\\
Blaeke Sleeves I holde in despite,\\
But White Sleeves is my delight.”
\end{scverse}

Nashe, speaking of Barnes’ \textit{Divine Centurie of Sonets}, says they are “such
another device as the goodly ballet of John Careless, or the song of Green Sleeves
Moralized.” Fletcher says, “And, by my Lady Greensleeves, am I grown so
tame after all my triumphs?” and Dr.~Rainoldes, in his \textit{Overthrow of Stage
Plays}, 1599, says, “Now if this were lawfully done because he did it, then
William, Bishop of Ely, who, to save his honour and wealth, became a \textit{Green
Sleeves}, going in women’s raiment from Dover Castle to the sea-side, did therein
like a man;—although the women of Dover, when they found it out, by plucking
down his muffler and seeing his new shaven heard, called him a monster for it.”

In Mr. Payne Collier’s Collection, and in that of the Society of Antiquaries,
are copies of “A Warning to false Traitors, by example of fourteen; whereof six 
were executed in divers places neere about London, and two near Braintford, the  
\pagebreak
%%229
%%===============================================================================
28th day of August, 1588; also at Tyborne were executed the 30th day six;
viz., five men and one woman: to the tune of \textit{Green Sleeves},” beginning—
\settowidth{\versewidth}{To hurt our Queen in treacherous wise,}
\indentpattern{00010001}
\begin{scverse}
\begin{patverse}
\vleftofline{“}You traitors all that do devise\\
To hurt our Queen in treacherous wise,\\
And in your hearts do still surmise\\
Which way to hurt our England;\\
Consider what the end will be\\
Of traitors all in their degree:\\
Hanging is still their destiny\\
That trouble the peace of England.”
\end{patverse}
\end{scverse}

The conspirators were treated with very little consideration by the ballad-monger
in having their exit chaunted to a merry tune, instead of the usual
lamentation, to the hanging-tune of \textit{Fortune my foe}.

Elderton’s ballad, \textit{The King of Scots and Andrew Brown}, was to be sung to
the tune of \textit{Mill-field}, or else to \textit{Green Sleeves} (see p.~185), but the measure suits
the former and not the latter. However, his “New Yorkshire Song, intituled—
\settowidth{\versewidth}{For merry pastime and companie,}
\begin{scverse}
\vleftofline{“}Yorke, Yorke, for my monie,\\
Of all the cities that ever I see.\\
For merry pastime and companie,\\
\vin Except the cittie of London;”
\end{scverse}
which is dated “from Yorke, by W. E., and imprinted at London by Richard
Jones,” in 1584, goes so trippingly to \textit{Green Sleeves}, that, although no tune is
mentioned on the title, I feel but little doubt of its having been intended for that
air. It was written during the height of its popularity, and not long after his
own “Reprehension.”

The song of \textit{York for my money} is on a match at archery between the Yorkshire
and the Cumberland men, backed by the Earls of Essex and Cumberland,
which Elderton went to see, and was delighted with the city and with his
reception; especially by the hospitality, of Alderman Maltby of York.

Copies will be found in the Roxburghe Collection, i. 1, and Evans’ \textit{Old Ballads},
i. 20,. It begins, “As I come thorow the North countrey,” and is refered to in
Heywood’s \textit{King Edward IV}., 1600.

In Mr. Payne Collier’s \textit{Old Ballads}, printed for the Percy Society, there is one
of Queen Elizaheth at Tilbury Fort (written shortly anterior to the destruction of
the Spanish Armada) to the tune of \textit{Triumph and Joy}. The name of the air is
probably derived from a ballad which was entered on the Stationers’ books in
1581, of “The Triumpe shewed before the Queene and the French Embassadors,”
who preceded the arrival of the Duke of Anjou, and for whose entertainment
jousts and triumphs were held. The tune for this ballad is not named in the
entry at Stationers’ Hall, but if a copy should be found, I imagine it will prove
also to have been written to \textit{Green Sleeves}, from the metre, and the date
coinciding with the period of its great popularity.

Richard Jones, to whom \textit{Green Sleeves} was first licensed, was also the printer
of \textit{A Handefull of Pleasant Delites}, 1584, in which a copy of the ballad will be
found. Also in Ellis’ \textit{Specimens}, ii. 394, (1803). A few verses are subjoined, 
 \pagebreak
%%230
%%===============================================================================
as affording an insight into the dress and manners of an age with which we cannot
be too well acquainted.

The tune is contained in several of Dowland’s lute manuscripts; in William
Ballet’s Lute Book; in Sir John Hawkins’ transcripts of virginal music; in \textit{The
Dancing Master}; \textit{The Beggar's Opera}; and in many other books.

As the second part differs in the oldest copies, from others of later date, both
versions are subjoined.

The first is from William Ballet’s Lute Book compared with another in Sir
John Hawkins’ transcripts of virginal music; both having the older second part.

\musicinfo{Smoothly and in moderate time.}{Tune of Greensleeves. Oldest copy.}
\lilypondfile{lilypond/230-greensleeves-oldest-copy}\normalsize

\indentpattern{01012}
\settowidth{\versewidth}{I kept thee booth at board and bed,}
\begin{dcverse}\begin{patverse}
I have been ready at your hand\\
To grant whatever you would crave,\\
I have both waged life and land,\\
Your love and good-will for to have.\\
Greensleeves was all my joy, \&c.
\end{patverse}

\begin{patverse}
I bought thee kerchers to thy head,\\
That were wrought fine and gallantly,\\
I kept thee booth at board and bed,\\
Which cost my purse well favoredly.\\
Greensleeves was all my joy, \&c. 
\end{patverse}
\end{dcverse}
\pagebreak


%%231
%%===============================================================================

\indentpattern{01012}
\settowidth{\versewidth}{I bought thee petticoats of the best,}
\begin{dcverse}\begin{patverse}
I bought thee petticoats of the best,\\
The cloth so fine as might be;\\
I gave thee jewels for thy chest,\\
And all this cost I spent on thee.\\
Greensleeves was all my joy, \&c.
\end{patverse}

\begin{patverse}
Thy smock of silk, both fair and white,\\
With gold embroidered gargeously;\\
Thy petticoat of sendal right, [thin silk]\\
And these I bought thee gladly.\\
Greensleeves was all my joy, \&c.
\end{patverse}
\end{dcverse}
\noindent He then describes her girdle of gold, her purse, the crimson stockings all of silk,
the pumps as white as milk, the gown of grassy green, the satin sleeves, the
gold-fringed garters; all of which he gave her, together with his gayest gelding,
and his men decked all in green to wait upon her:
\settowidth{\versewidth}{Thy foot might not once touch the ground.}
\begin{scverse}\begin{patverse}
They set thee up, they took thee down,\\
They serv’d thee with humility;\\
Thy foot might not once touch the ground.\\
And yet thou wouldst not love me.\\
Greensleeves was all my joy, \&c.
\end{patverse}
\end{scverse}
She could desire no earthly thing without being gratified:

%\vspace{-0.25\baselineskip}

\begin{dcverse}\begin{patverse}
Well I will pray to God on high,\\
That thou my constancy mayst see,\\
And that yet once before I die\\
Thou wilt vouchsafe to love me.\\
Greensleeves was all my joy, \&c.
\end{patverse}

\begin{patverse}
Greensleeves, now farewell! adieu!\\
God I pray to prosper thee!\\
For I am still thy lover true,\\
Come once again and love me.\\
Greensleeves was all my joy, \&c.
\end{patverse}
\end{dcverse}

%\vspace{-0.25\baselineskip}


At the Revolution \textit{Green Sleeves} became one of the party tunes of the Cavaliers;
and in the “Collection of Loyal Songs written against the Rump Parliament,”
there are no less than fourteen to be sung to it. It is sometimes referred to under
the name of \textit{The Blacksmith}, from a song (in the Roxburghe Collection, i. 250)
to the tune of \textit{Green Sleeves}, beginning—

%\vspace{-0.25\baselineskip}

\begin{scverse}\vleftofline{“}Of all the trades that ever I see\\
There is none with the blacksmith’s compared may be,\\
For with so many several tools works he,\\
\vin \textit{Which nobody can deny”}
\end{scverse}

%\vspace{-0.25\baselineskip}

Pepys, in his diary, 22nd April, 1660, says that, after playing at nine-pins,
“my lord fell to singing a song upon the Rump, to the tune of \textit{The Blacksmith}.”

It was also called \textit{The Brewer}, or \textit{Old Noll, the Brewer of Huntingdon}, from a
satirical song about Oliver Cromwell, which is to be found in \textit{The Antidote to
Melancholy}, 1661, entitled “The Brewer, a ballad made in the year 1657, to the
tune of \textit{The Blacksmith};” also in \textit{Wit and Drollery, Jovial Poems}, 1661.

In \textit{The Dancing Master}, 1686, the tune first appears under the name of \textit{Green
Sleeves and Pudding Pies}; and in some of the latest editions it is called \textit{Green
Sleeves and Yellow Lace}. Percy says, “It is a received tradition in Scotland that
\textit{Green Sleeves and Pudding Pies} was designed to ridicule the Popish clergy,” but
the tradition most probably refers to a song of James the Second’s time called
\textit{At Rome there is a terrible rout},\dcfootnote{ %
This is entitled “Father Peters’ Policy discovered; or
the Prince of Wales proved a Popish Perkin.” London:
printed for R. M., ten stanzas, of which the following is
the first:—
\medskip
\settowidth{\versewidth}{Because the birth of the babe’s come out,}
\indentpattern{0001}
\begin{fnverse}
\begin{patverse}
\vleftofline{“}In Rome there is a most fearful rout;\\
And what do you think it is about?\\
Because the birth of the babe’s come out,\\
Sing Lullaby Baby, by, by, by.”
\end{patverse}
\end{fnverse}
} %end footnote
which was sung to the tune, and attained some
popularity, since in the ballad-opera of \textit{Silvia}, or \textit{The Country Burial}, 1731,
it appears under that name. Boswell, in his Journal, 8vo., 785, p.~319, prints
the following Jacobite song:— 

\pagebreak
%232
%%===============================================================================

\indentpattern{0001}
\begin{dcverse}\settowidth{\versewidth}{And I’ll be with her before she rise,}
\begin{patverse}
\vleftofline{“}Green Sleeves and Pudding Pies,\\
Tell me where my mistress lies,\\
And I’ll be with her before she rise,\\
Fiddle and aw together.
\end{patverse}

\settowidth{\versewidth}{And may our King come home with speed,}
\begin{patverse}
May our affairs abroad succeed,\\
And may our King come home with speed,\\
And all Pretenders shake for speed,\\
And let his health go round.
\end{patverse}
\end{dcverse}
\settowidth{\versewidth}{Let all Pretenders shake for dread,}
\begin{scverse}To all our injured friends in need;\\
This side and beyond the Tweed,\\
Let all Pretenders shake for dread,\\
\vin And let his health go round.”
\end{scverse}

There is no apparent connection between the subject of the first and that of the
remaining stanzas; and although the first may have been the burden of an older
song, it bears no indication of having refered to the clergy of any denomination.

There is scarcely a collection of old English songs in which at least one may
not be found to the tune of \textit{Green Sleeves}. In the West of England it is still
sung at harvest-homes to a song beginning, “A pie sat on a pear-tree top;” and
at the Maypole still remaining at Ansty, near Blandford, the villagers still dance
annually round it to this tune.

The following “Carol for New Year’s Day, to the tune of \textit{Green Sleeves},” is
from a black-letter collection printed in 1642, of which the only copy I have seen
is in the Ashmolean Library, Oxford.

\indentpattern{00010001}
\settowidth{\versewidth}{And now with new year’s gifts each friend}
\begin{dcverse}\begin{patverse}
The old year now away is fled,\\
The new year it is entered;\\
Then let us now our sins down tread,\\
And joyfully all appear.\\
Let’s merry be this holiday,\\
And let us run with sport and play,\\
Hang sorrow, let’s cast care away—\\
God send you a happy new year.
\end{patverse}

\begin{patverse}
And now with new year’s gifts each friend\\
Unto each other they do send;\\
God grant we may our lives amend,\\
And that the truth may appear.\\
Now like the snake cast off your skin\\
Of evil thoughts and wicked sin,\\
And to amend this new year begin—\\
God send us a merry new year.
\end{patverse}

\begin{patverse}
And now let all the company\\
In friendly manner all agree,\\
For we are here welcome all may see\\
Unto this jolly good cheer.\\
\columnbreak
I thank my master and my dame,\\
The which are founders of the same,\\
To eat to drink now is no shame—\\
God send us a merry new year.
\end{patverse}

\begin{patverse}
Come lads and lasses every one.\\
Jack, Tom, Dick, Bess, Mary, and Joan,\\
Let’s cut the meat unto the bone,\\
For welcome you need not fear.\\
And here for good liquor we shall not lack.\\
It will whet my brains and strengthen my back,\\
This jolly good cheer it must go to wrack—\\
God send us a merry new year.
\end{patverse}

\begin{patverse}
Come, give us more liquor when I do call,\\
I’ll drink to each one in this hall,\\
I hope that so loud I must not bawl,\\
But unto me lend an ear.\\
Good fortune to my master send,\\
And to my dame which is our friend,\\
God bless us all, and so I end—\\
And God send us a happy new year.
\end{patverse}
\end{dcverse}

The following version of the tune, from \textit{The Beggars' Opera}, 1728, is that
now best known. I have not found any lute or virginal copy which had this
second part. The earliest authority for it is \textit{The Dancing Master}, 1686, and it
may have been altered to suit the violin, as the older second part is rather low,
and less effective, for the~instrument. 
\pagebreak
%%233
%%===============================================================================

I have selected a few lines from a political song called \textit{The Trimmer}, to print
with this copy, because it has the burden, “Which nobody can deny.” It is one
of the many songs to the tune in \textit{Pills to purge Melancholy}.

\musicinfo{Boldly.}{Tune of Green Sleeves. Later copy.}
\lilypondfile[staffsize=15]{lilypond/233-greensleeves-later-copy}\normalsize

\musictitle{My Robin Is To The Greenwood Gone; Or, Bonny Sweet Robin.}

This is contained in Anthony Holborne’s \textit{Cittharn Schoole}, 1597; in Queen
Elizabeth’s Virginal Book; in William Ballet’s Lute Book; and in many other
manuscripts and printed books.

There are two copies in William Ballet’s Lute Book, and the second is entitled
“Robin \textit{Hood} is to the greenwood gone;” it is, therefore, probably the tune of a
ballad of Robin Hood, now lost.

Ophelia sings a line of it in \textit{Hamlet}—
\settowidth{\versewidth}{“For bonny sweet Robin is all my joy;”}
\begin{scverse}
“For bonny sweet Robin is all my joy;”
\end{scverse}
and in Fletcher’s \textit{Two Noble Kinsmen}, the jailer’s daughter, being mad, says,
“I~can sing twenty more\dots I can sing \textit{The Broom} and \textit{Bonny Robin}.” In
Robinson’s \textit{Schoole of Musicke} (1603), and in one of Dowland’s Lute Manuscripts, 
\pagebreak
%234
%%===============================================================================
(D.~d.,~2.~11, Cambridge), it is entitled, “Robin is to the greenwood gone; in
Addit. MSS. 17,786 (Brit. Mus.), “\textit{My} Robin,” \&c.

A ballad of “A dolefull adieu to the last Erle of Darby, to the tune of \textit{Bonny
sweet Robin}” was entered at Stationers’ Hall to John Danter on the 26th April,
1593; and in the \textit{Crown Garland of Golden Roses} is “A courtly new ballad of
the princely wooing of the fair Maid of London by King Edward;” as well as
“The fair Maid of London’s answer,” to the same tune. The two last were also
printed in black-letter by Henry Gosson, and are reprinted in Evans’ \textit{Old
Ballads}, iii. 8.

In “Good and true, fresh and new Christmas Carols,” \textsc{b.l.}, 1642, there is a
“Carol for St. Stephen’s day: tune of \textit{Bonny sweet Robin}” beginning—
\settowidth{\versewidth}{“Come, mad boys, be glad, boys, for Christmas is here,”}
\begin{scverse}
\vleftofline{“}Come, mad boys, be glad, boys, for Christmas is here,\\
And we shall be feasted with jolly good cheer,” \&c.
\end{scverse}

\musicinfo{Slowly and ad libitum.}{}
\lilypondfile{lilypond/234-my-robin-is-to-the-greenwood-gone-or-bonny-sweet-robin}\normalsize

\musictitle{With A Fading.}

In act iv., sc. 3, of Shakespeare’s \textit{Winter’s Tale}, the servant says of Autolycus,
“He hath songs for man or woman, of all sizes; no milliner can so fit his
customers with gloves: he has the prettiest love-songs for maids;\ldots\  with such
delicate burdens of \textit{dildos} and \textit{fadings}.”

In the Roxburghe Collection, i. 12, there is a ballad by L. P. (Laurence Price?),
entitled “The Batchelor’s Feast; or—
\settowidth{\versewidth}{Betwixt the batchelor’s pleasure and the married man’s trouble.}
\begin{scverse}The difference betwixt a single life and a double;\\
Betwixt the batchelor’s pleasure and the married man’s trouble.
\end{scverse}
To a pleasant new tune, called \textit{With a hie dildo dill}.” It begins thus:—
\indentpattern{00005656}
\begin{scverse}\begin{patverse}
“As I walkt forth of late, where grass and flowers spring,\\
I heard a batchelor within an harbour sing.\\
The tenor of his song contain’d much melodie:\\
It is a gallant thing to live in liberty.\\
\textit{With a hie, dildo, dill,\\
Hie do, dil dur lie};\\
It is a delightful thing\\
To live at liberty.”
\end{patverse}
\end{scverse}
There are six stanzas; and six more in  a second part (at p.~17 of the same 
\pagebreak
%%235
%%===============================================================================
volume), “printed at London for I. W.” (either I. Wright or I. White, who were
both ballad printers of the reigns of James I. and Charles I.)

In \textit{Choice Drollery}, 1656, p.~31, is another, which would require a different
tune, commencing—
\settowidth{\versewidth}{Of a woman that danc’d upon the rope.}
\begin{scverse}\begin{altverse}
\vleftofline{“}A story strange I will you tell,\\
But not so strange as true,\\
Of a woman that danc’d upon the rope.\\
And so did her husband too.\\
\textit{With a dildo, dildo, dildo,\\
With a dildo, dildo dee}.’’
\end{altverse}
\end{scverse}

In the Pepys Collection of Ballads, i. 224, is one by Robert Guy, printed for
H.~Gosson, and with the following title:—

\begin{scverse}\vin\vin\vin\vin “The Merry Forester.\\
Young men and maids, in country or in city\\
I crave your aids with me to tune this ditty;\\
Both new and true it is, no harm in this is,\\
But is composed of the word call’d kisses;\\
Yet meant by none, abroad loves to be gadding:\\
It goes unto the tune of \textit{With a fadding}.”
\end{scverse}
The first line is “Of late I chanc’d to be where I,” \&c.

Another song, which has the burden “with a fading,” will he found in
Shirley’s \textit{Bird in a Cage}, act iv., sc. 1 (1633). A third in \textit{Sportive Wit}, \&c.,
1656, p.~58. The last is also printed in \textit{Pills to purge Melancholy}, ii. 99 (1707),
with the tune, of which there are other copies in the same work.

There are also ballads to it, under the name of \textit{An Orange}, and \textit{With a
Pudding}. See Roxburghe Collection, ii. 16; \textit{Pills to purge Melancholy}, i. 90
(1707), \&c.

The \textit{Fading} is the name of an Irish dance, but\textit{ With a fading} (or \textit{fadding})
seems to he used as a nonsense-burden, like \textit{Derry dawn, Hey nonny, nonny no,} \&c.

\musicinfo{Tripping1y and in moderate time.}{}
\lilypondfile{lilypond/235-with-a-fading}\normalsize

\pagebreak
%%236
%%===============================================================================

\indentpattern{01018}
\begin{dcverse}\footnotesize
\begin{patverse}
You hawk, you hunt, you lie upon pallets,\\
You eat, you drink (the Lord knows how!);\\
We sit upon hillocks, and pick up our sallets,\\
And drink up a syllabub under a cow.\\
\textit{With a fading.}
\end{patverse}

\begin{patverse}
Your masks are made for knights and lords,\\
And ladies that go fine and gay;\\
We dance to such music the bagpipe affords,\\
And trick up our lasses as well as we may.\\
\textit{With a fading.}
\end{patverse}

\begin{patverse}
Your clothes are made of silk and satin,\\
And ours are made of good sheep’s gray;\\
You mix your discourse with pieces of Latin,\\
We speak our English as well as we may.\\
\textit{With a fading.}
\end{patverse}

\begin{patverse}
You dance Corants and the French Braul,\\
We jig the Morris upon the green,\\
And make as good sport in a country hall,\\
As you do before the King and the Queen.\\
\textit{With a fading.}
\end{patverse}
\end{dcverse}

\vspace{-\baselineskip}

\musictitle{How Should I Your True Love Know?}

\vspace{-0.5\baselineskip}
The late W. Linley (an accomplished amateur, and brother of the highly-gifted
Mrs. Sheridan) collected and published “the wild and pathetic melodies of
Ophelia, as he remembered them to have been exquisitely sung by Mrs. Forster,
when she was Miss Field, and belonged to Drury Lane Theatre;” and he says,
“the impression remained too strong on his mind to make him doubt the
correctness of the airs, agreeably to her delivery of them.” Dr.~Arnold also
noted them down from the singing of Mrs. Jordan, and Mr. Ayrton has followed
that version in his Annotations to Knight’s \textit{Pictorial Edition of Shakespeare}.
The notes of this air are the same in both; but in the former it is in \timesig{3}{4} time,
in the latter in common time. The melody is printed in common time in
\textit{The Beggars’ Opera} (1728) to “You’ll think, ere many days ensue,” and in
\textit{The Generous Freemason}, 1731.

Dr.~Percy selected some of the fragments of ancient ballads which are
dispersed through Shakespeare’s plays, and especially those sung by Ophelia in
\textit{Hamlet}, and connected them by a few supplemental stanzas into his charming
ballad, \textit{The Friar of Orders Gray}, the first line of which is taken from one, sung
by Petruchio, in \textit{The Taming of the Shrew}.

The following is the tune; but in singing Ophelia’s fragments, each line should
begin on the first of the bar, and not with the note before it. In the ballad-operas
it has the burden, \textit{Twang, lang, dildo dee} at the end, with two additional
bars of music, the same as to \textit{The Knight and Shepherd’s Daughter}. See p.~127.

\musicinfo{Moderate time and smoothly.}{}
\lilypondfile[staffsize=15]{lilypond/236-how-should-i-your-true-love-know}\normalsize


\settowidth{\versewidth}{At his head a green grass turf,}
\begin{dcverse}\footnotesize
\begin{altverse}
He is dead and gone, lady,\\
He is dead and gone;\\
At his head a green grass turf,\\
At his heels a stone.
\end{altverse}

\begin{altverse}
White his shroud as mountain snow,\\
Larded with sweet flowers,\\
Which bewept to the grave did go\\
With true love showers. 
\end{altverse}
\end{dcverse}
\pagebreak
%%237
%%===============================================================================

A parody on this song seems to be intended in Rowley’s \textit{A Match at Midnight},
1633, where the Welshman sings—
\settowidth{\versewidth}{Did hur not see hur true love-a}
\begin{scverse}
\vleftofline{“}Did hur not see hur true love-a\\
\vin As hur come from London?” \&c.
\end{scverse}

\musictitle{And Will He Not Come Again?}

This fragment, sung by Ophelia, was also noted down by W. Linley. It
appears to be a portion of the tune entitled \textit{The Merry Milkmaids} in \textit{The Dancing
Master}, 1650, and \textit{The Milkmaids’ Dumps} in several ballads. The following lines
in \textit{Eastward Roe}, 1605, resemble, and are probably a parody on, Ophelia’s song:—
\settowidth{\versewidth}{His head as white as milk,}
\indentpattern{00110}
\begin{scverse}
\begin{patverse}
\vleftofline{“}His head as white as milk,\\
All flaxen was his hair;\\
But now he is dead,\\
And lain in his bed,\\
And never will come again.”—\textit{Dodsley}, iv., 223.
\end{patverse}
\end{scverse}

\musicinfo{Very slowly and ad libitum.}{}
\lilypondfile{lilypond/237-and-will-he-not-come-again}\normalsize

\begin{scverse}
\begin{patverse}
His beard was white as snow,\\
All flaxen was his hair,\\
He is gone, he is gone,\\
And we cast away moan;\\
God ’a mercy on his soul.
\end{patverse}
\end{scverse}

\musictitle{O Death! Rock Me Asleep.}

In the second part of Shakespeare’s \textit{King Henry IV}., act ii., sc. 4, Pistol
snatching up his sword, exclaims—
\settowidth{\versewidth}{\textit{Then death rock me asleep, abridge my doleful days}!”}
\begin{scverse}
\vleftofline{“}What! shall we have incision? shall we imbrue?\\
Then \textit{death rock me asleep, abridge my doleful days}!”
\end{scverse}

This is in allusion to the following song, which is supposed to have been written
by Anne Boleyn. The words were printed by Sir John Hawkins in his \textit{History
of Music}, having been “communicated to him by a very judicious antiquary,”
then “lately deceased,” whose opinion was that they were written either by, or in
the person of, Anne Boleyn; “a conjecture,” he adds, “which her unfortunate
history renders very probable.” On this Ritson remarks, “It is, however, but a
conjecture: any other state prisoner of \pagebreak that period having an equal claim. 
%%238
%%===============================================================================
George, Viscount Rochford, brother to the above lady, and who suffered on her
account, ‘hath the fame,’ according to Wood, ‘of being the author of several
poems, songs, and sonnets, with other things of the like nature,’ and to him he
(Ritson) is willing to refer them.”—(\textit{Ancient Songs}, 1790, p.~120.)

The first stanza of the words, with the tune, is contained in a manuscript of
the latter part of Henry’s reign, formerly in the possession of Stafford Smith,
and now in that of Dr.~Rimbault. It is a single-voice part, in the diamond-headed
note, and without accompaniment. Another copy, with an accompaniment for the
lute, will be found in Addit. MSS. 4900, British Museum.

\musicinfo{Moderate time, and like recitative.}{}
\lilypondfile{lilypond/238-o-death-rock-me-asleep}\normalsize

\pagebreak
%%239
%%===============================================================================

\settowidth{\versewidth}{My dolour will not suffer strength}
\begin{dcverse}
\indentpattern{01014}
\begin{patverse}
My pains who can express?\\
Alas! they are so strong;\\
My dolour will not suffer strength\\
My life for to prolong.\\
Toll on, \&c.
\end{patverse}

\begin{patverse}
Alone in prison strong\\
I wail my destiny;\\
Woe worth this cruel hap that I\\
Should taste this misery.\\
Toll on, \&c.
\end{patverse}

\indentpattern{0101000222}
\begin{patverse}
Farewell my pleasures past,\\
Welcome my present pain;\\
I feel my torments so increase,\\
That life cannot remain.\\
Cease now the passing bell,\\
Rung is my doleful knell,\\
For the sound my death doth tell.\\
Death doth draw nigh,\\
Sound my end dolefully,\\
For now I die.
\end{patverse}
\end{dcverse}


\musictitle{Can You Not Hit It, My Good Man?}

The following lines are sung by Rosaline and Boyet in act iv., sc. 1, of \textit{Love’s
Labour Lost}. The tune was transcribed by Dr.~Rimbault from one of the MSS.
presented by Bishop Fell to the Music School at Oxford, and bearing a date of
1620. \textit{Canst thou not hit it} is mentioned as a dance in the play of \textit{Wily Beguiled},
written in the reign of Elizabeth. In 1579, “a ballat intytuled \textit{There is better
game, if you could hit it}” was licensed to Hughe Jaxon.

\musicinfo{Trippingly and moderately fast.}{}
\lilypondfile{lilypond/239-can-you-not-hit-it-my-good-man}\normalsize

\begin{center}{\makebox[1in]{\hrulefill}}\end{center}

The list of music illustrating Shakespeare might be largely increased, by
including in it catches, part-music, and the works of known composers, which do
not fall within the scope of the present collection. The admirers of Shakespeare
will be gratified to know that a work is in progress which will include not only
those, but also such of the original music to his dramas as can still be found.\dcfootnote{ %a
This work (to which Dr.~Rimbault has devoted many
years of zealous research) will be entitled “A Collection
of Ancient Music, illustrating the plays and poems of
Shakespeare.” The first portion will contain all that now
remains of the original music to his dramas, or which, if
not composed for the first representation of them, was
written during the life-time of the poet. The whole of
the music of \textit{The~Tempest} will be included in this part.
Another division will contain the old songs, ballads,
catches, \&c., inserted, or alluded to, by Shakespeare. The
dances will form the third part. It was owing to researches
on a subject so much akin to that of the present
Collection, that Dr.~Rimbault’s aid has been so peculiarly
valuable in this work.
} %end footnote

The three following ballads, with which I close the reign of Elizabeth, were
popular in the time of Shakespeare, but are not mentioned by the great poet.
\pagebreak
%%240
%%===============================================================================

\musictitle{Bara Faustus’ Dream.}

In the instrumental arrangements of this tune it is usually entitled \textit{Bara
Faustus} (or \textit{Barrow Foster’s}) \textit{Dream}; and when found as a song, it is generally
as, “\textit{Come, sweet love, let sorrow cease}.”

It will be found under the former name in Queen Elizabeth’s Virginal Book
(twice); in Rossiter’s \textit{Lessons for Consort}, 1609; and in Nederlandtsche Gedenck-
Clanck, 1626, under the latter in “Airs and Sonnets,”, MS., Trin. Col., Dublin
(F.~v.~13); in the MS. containing “It was a lover and his lass,” described at
p.~204; and in Forbes’ \textit{Cantus}, 1682.

\textit{Bara Faustus’ Dreame} was one of the tunes chosen for the \textit{Psalmes or Songs
of Sion}, \&c., 1642.


\musicinfo{Smoothly, and with expression.}{}
\lilypondfile[staffsize=15]{lilypond/240-bara-faustus-dream}\normalsize

\musictitle{The Spanish Pavan.}

Dekker, in his \textit{Knight’s Conjuring} (1607) thus apostrophises his opponent:
“Thou, most clear-throated singing man, with thy harp, to the twinkling of
which inferior spirits skipp’d like goats over the Welsh mountains, hadst privilege
(because thou wert a fiddler) to be saucy? Inspire me with thy cunning, and
guide me in true fingering, that I may strike those tunes which thou playd’st!
Lucifer himself danced a \textit{Lancashire Hornpipe} whilst thou wert there. If I can
but harp upon \textit{thy} string, he shall now, for my pleasure, tickle up \textit{The Spanish
Pavan}.” The tune of \textit{The Spanish Pavan} was very popular in the reigns of 
Elizabeth and James. One of the songs \pagebreak in Anthony Munday’s \textit{Banquet of}  
\markright{reign of elizabeth.}
%241
%===============================================================================
\textit{Daintie Conceits} 1588, is “to the note of \textit{The Spanish Pavin};” another in
part ii. of \textit{Robin Goodfellow}, 1628; and there are many in the Pepys and Roxburghe
Collections of Ballads.

It is mentioned as a dance in act iv., sc. 2, of Middleton’s \textit{Blurt, Master Constable},
1602; and in act i., sc. 2, of Ford’s ’\textit{Tis Pity}, 1633. In the former the
tune is played for Lazarillo to dance \textit{The Spanish Pavan}. The figure, which
differed from other Pavans, is described in Thoinot Arbeau’s \textit{Orchesographie}, 1589;
but as the tune there printed is wholly different from the following (which is
found in Queen Elizabeth’s Virginal Book, William Ballet’s Lute Book, Sir
J. Hawkins’ transcripts of Virginal Music, \&c.), I suppose this to be English,
although not a characteristic~air.

The ballad, “When Samson was a tall young man,” (of which the first stanza
is here printed) is in the Pepys Collection, i. 32; in the Roxburghe, i. 366; and
in Evans’ \textit{Old Ballads}, i. 283 (1810).\dcfootnote{ %a
The copies in the Pepys and Roxburghe Collections
differ. The former has no printer’s name; the latter
(which is followed by Evans) was printed “for the
assigns of T.~Symcocke.”
} %end footnote
It is parodied in \textit{Eastward Hoe}, the joint
production of Ben Jonson, Marston, and Chapman, act ii., sc. 1. The two first
lines are the same in the parody and the ballad.

\musicinfo{Moderate time.}{}
\lilypondfile{lilypond/241-the-spanish-pavan}\normalsize

\pagebreak
%%242
%%===============================================================================

\musictitle{Wigmore’s Galliard.}

The tune from William Ballet’s Lute Book. In Middleton’s \textit{Your five Gallants},
Jack says, “This will make my master leap out of the bed for joy, and dance
\textit{Wigmore’s Galliard} in his shirt about his chamber!” It is frequently mentioned
by other early writers, and there are many ballads to the tune. Among them
are “A most excellent new Dittie, wherein is shewed the wise sayings and wise
sentences of Solomon, wherein each estate is taught his dutie, with singular
counsell to his comfort and consolation” (a copy in the collection of the late
Mr. W. H. Miller, from Heber’s Library). “A most famous Dittie of the joyful
receiving of the Queen’s most excellent Majestie by the worthie citizens of
London, the 12th day of November, 1584, at her Grace’s coming to St. James’”
(a copy in the Collection of Mr. George Daniel). In the Pepys Collection, i. 455,
is “A most excellent Ditty called Collin’s Conceit,” beginning—
\settowidth{\versewidth}{“Conceits of sundry sorts there are.”}
\begin{scverse}
“Conceits of sundry sorts there are.”
\end{scverse}
Others are in the second volume of the Pepys Collection; in the Roxburghe; in
Anthony Munday’s \textit{Banquet of Daintie Conceits}; in Deloney’s \textit{Strange Histories},~1607,~\&c.

The following stanza is from the ballad of “King Henry the Second crowning
his son Henry, in his life-time,” \&c., by Deloney. The entire ballad is reprinted
by Evans (ii. 63), from \textit{The Garland of Delight}, but he omits the name of the tune.

\musicinfo{}{}
\lilypondfile{lilypond/242-wigmores-galliard}\normalsize

\pagebreak
%%243
%%===============================================================================

\musictitle{Good Fellows Must Go Learn To Dance.}

The following ballad is from a copy (probably unique) in the Collection of
Mr. George Daniel, of Canonbury. It may be sung to several of the foregoing
airs, but the name of the proper tune is not given on the copy.

\begin{center}\small\textsc{A New Ballad Intituled}
\end{center}

\musictitle{Good Fellows Must Go Learn To Dance.}

\begin{dcverse}
\settowidth{\versewidth}{The bridegroom would give 20 pound}
\begin{altverse}
Good fellows must go learn to dance,\\
The bridal is full near-a,\\
There is a Braule come out of France,\\
The trick’st you heard this year-a;\\
For I must leap, and thou must hop,\\
And we must turn all three-a,\\
The fourth must bounce it like a top,\\
And so we shall agree-a;\\
I pray thee, Minstrel, make no stop,\\
For we will merry be-a.
\end{altverse}

\begin{altverse}
The bridegroom would give 20 pound\\
The marriage-day were past-a;\\
You know while lovers are unbound,\\
The knot is slipp’ry fast-a.\\
A better man may come in place,\\
And take the bride away-a;\\
God send or Wilkin better grace,\\
Our pretty Tom doth say-a;\\
Good Vicar, axe the banns apace,\\
And haste the marriage-day-a.
\end{altverse}

\begin{altverse}
A band of bells in bawdrick wise\\
Would deck us in our kind-a;\\
A shirt after the Morris guise,\\
To flounce it in the wind-a;\\
A Whiffler for to make the way,\\
And May brought in with all-a,\\
Is braver than the sun, I say,\\
And passeth Round or Braule-a,\\
For we will trip so trick and gay,\\
That we will pass them all-a.
\end{altverse}

\begin{altverse}
Draw to dancing, neighbours all,\\
Good fellows, hip is best-a;\\
It skills not if we take a fall,\\
In honoring this feast-a.\\
The bride will thank us for our glee,\\
The world will us behold-a;\\
O where shall all this dancing be?\\
In Kent or in Cotswold-a?\\
Our lord doth know, then axe not me,\\
And so my tale is told-a.
\end{altverse}
\end{dcverse}

Imprinted at London in Flete Strete at the signe of the Faucon, by Wylliam
Gryffith, and are to be solde at his shoppe in S. Dunstones Church Yearde, 1569.  

\centerrule
\pagebreak

%%244
%%================================================
%\vspace*{-2\baselineskip}
\section*{REIGN OF JAMES I.}

\centerrule

The most distinguishing feature of chamber music, in the reign of James I.,
from that of his predecessor, was the rapidly-increasing cultivation of instrumental
music, especially of such as could be played in concert; and, coevally, the incipient
decline of the more learned, but less melodious descriptions of vocal music,
such as madrigals and motets.

During the greater part of the reign of Elizabeth, vocal music held an almost
undivided sway, and the practice of instrumental music, in private life, was
generally confined to solo performances, and to accompaniments for the voice.

The change of fashion, so far as I have been able to trace it, may be dated from
1599, in which year Morley printed a “First Booke of Consorte Lessons, made
by divers exquisite authors,” for six instruments to play together; and Anthony
Holborne a collection of “Pavans, Galliards, Almaines, and other short airs, both
grave and light, in five parts.” Morley’s publication consisted of favorite
subjects arranged for the Treble Lute, the Pandora,\dcfootnote{\scriptsizerr %a
There was a foreign instrument of the lute description,
with a great number of strings, called the Pand\textit{u}ra,
but I imagine the English Pandora to be the same instrument
as the \textit{B}andora. In Thomas Robinson’s “School
of Musicke, the perfect fingering of the Lute, \textit{Pandora},
Orpharion and Viol da Gamba” the music is noted on six
lines, for an instrument of six strings like the Lute. In
1613, Drayton and Sir William Leighton severally enumerated
the instruments in use in England. Drayton
names the “Pandore” among instruments strung with
wire. Sir William Leighton speaks of the “Bandore,”
but neither of \textit{both}. In 1609, Philip Rosseter printed a
set of “Lessons for Consort,” like Morley’s, and for the
same six instruments, if the Bandora be not an exception. 
It was a large instrument of the lute kind,
with the same number of strings (but in all probability of
wire), and invented in 1562 by John Rose, citizen of
London, dwelling in Bridewell. It was much used in
this reign, especially with the Cittern, to which it formed
the appropriate base.
} %end footnote
the Cittern, the (English)
Flute,\dcfootnote{\scriptsizerr %b
\textit{Fistula dulcis, seu Anglica}, and by some as the \textit{Flute
à bec}, has eight holes for the fingers, and a mouth-piece
at the end like a flageolet. Of the eight holes, six are in
a row in front, one at the end for the little finger
(added afterwards), and one at the back for the
thumb. The tone is soft, rich, and melodious, but less
brilliant than the present flute. The ordinary length is
rather more than two feet. I had three or four of different
sizes, the largest exceeding four feet in length. The
base flute must have been still longer. The modern
flute is blown like the old fife; or as in the ancient
sculpture of The Piping Fawn.
} %end footnote
and the Treble and Bass Viols. Holborne's was for Viols, for Violins, \dcfootnote{\scriptsizerr %c
The English flute, described by Mersenne as the
and the Treble and Bass Viols. Holborne’s was for Viols, for Violins,\textsuperscript{c} 
Under the name of “Violins“ the four different sizes
of the instrument are here comprehended. The word
Violoncello is of comparatively modem use. In Ben
Jonson’s \textit{Bartholomew Fair}, we find, “A set of these
Violins I would buy, too, for a delicate \textit{young noise}” (i.e
company of young musicians) “I have in the country;
they are every one a size less than another;—\textit{just like your
fiddles}—Act iii., sc. 1. Charles the Second’s famous
band of “four-and-twenty \textit{fiddlers}, all in a row,” consisted
of six violins, six counter-tenors, six tenors,
and six bases. The counter-tenor violin has become obsolete, 
because all the notes of its scale could be played
upon the violin or tenor.
} %end footnote
or for wind-instruments.

I know of no set of Madrigals printed during the reign of Elizabeth, which is
described on the title-page as “apt \textit{for Viols} and Voices”—it was fully understood
that they were for voices only;—but, from 1603, when James ascended the
throne, that mode of describing them became so general, that I have found but
two sets printed without it.\dcfootnote{\scriptsizerr %d
The exceptions are Bateson’s \textit{First Set of Madrigals},
1604, and Pilkington’s \textit{First Set}, 1613, but the second sets
of both authors are described as “apt for viols and voices.”
So are Wilbye’s \textit{Second Set}, 1609; Michael Este’s \textit{Eight Sets},
of various dates, and the Madrigals of Orlando Gibbons,
Robert Jones, John Ward, Henry Lichfield, Walter Porter,
as well as Byrd's \textit{Psalmes, Songs, and Sonnets}, 1611: Peerson’s
\textit{Motets or Grave Chamber Music}, 1630; and many
lighter kinds of music. See Rimbault’s \textit{Bibliotheca
Madrigaliana}, 8vo.,~1847.
} %end footnote
 

\renewcommand\rectoheader{reign of james i.}
\pagebreak
%%245
%%================================================

Between 1603 and 1609, Dowland printed his “Lacrimæ, or Seven Teares
figured in seven passionate Pavans, with divers other Pavans, Galliards, and
Almands.” This work, to which there are so many allusions by contemporary
Dramatists, was in five parts, for the Lute, Viols, or Violins. In 1609, Rossiter
printed his “Lessons for Consort” for the same six instruments as Morley. In
1611, Morley’s work was reprinted,\dcfootnote{\scriptsize %a
\looseness=-1 Twelve volumes of Dr.~Burney’s MS. extracts for his
\textit{History of Music} were formerly in my possession, and are
now in the British Museum. In one of them (Add. MSS.
11,587) are his extracts from Morley’s \textit{Consort Lessons}.
To “O mistress mine” (which I have printed at p.~209)
he appends the following note:—“If any melody or movement, 
besides the Hornpipe (\textit{a tune played by the Cornish
pipe, or pipe of Cornwall}), be truly native, it seems to be
this; \textit{which has the genuine drawl of our country clowns
and ballad singers in sorrowful ditties}, as the hornpipe has
the coarse and vulgar jollity of their mirth and merriment.” 
This criticism is a curiosity, and not less curious
is the judgment he passes on the Consort Lessons, after
scoring two out of the six parts (the Treble Viol and
Flute), and adding \textit{his own} base. Morley dedicates them
to the Lord Mayor and Aldermen, and Dr.~Burney says,
“Master Morley, supposing that the harmony which was
to be heard through the clattering of knives, forks, spoons,
and plates, with the jingling of glasses, and clamorous
conversation of a city feast, need not be very accurate and
refined, was not very nice in setting parts to these tunes,
\textit{if we may judge of the rest by what passes between the viol
and flute},” \&c. The whole of this passage is transferred
to his \textit{History of Music} (iii. 102, Note D, 1789), except
the qualification, “\textit{if} we may judge,” \&c. It was not
advisable to tell the reader \textit{how} he had formed his opinion
of a work that had formerly passed through two editions.\linebreak
\indent Among Dr.~Burney’s other criticisms of English Music
(for his History is essentially a critical one, and he has been
commonly quoted as an authority) are the following, which
are also directly connected with the subject of this book;—
In vol. ii., p.~553, he says, “It is related by Gio. Battista
Donado that the Turks have a limited number of tunes, to
which the poets of their country have continued to write
for ages; and the vocal music of our own country seems
long to have been \textit{equally circumscribed}; for, till the last
century, it seems as if the number of our secular and
popular melodies did not greatly exceed that of the Turks.”
In a note it is stated that the tunes of the Turks were in
all twenty-four; which were to depict melancholy, joy, or
fury; to be mellifluous or amorous. It may not, I hope,
be too presumptuous to say that Dr.~Burney knew very
little of the subject. In vol. iii., 143, after criticising a
work printed in 1614, and saying, “The Violin was now
\textit{hardly known by the English, in shape or name}” (although
Ben Jonson describes the instrument, at that very time,
as commonly sold with roast pigs in Bartholomew Fair,
and violins had certainly been used on the English
stage from its infancy. See, for instance, the tragedy of
\textit{Gorboduc, or Ferrex and Porrex}, acted by the gentlemen
of the Inner Temple before Queen Elizaheth, in 1561);
he adds, “And the low state of our regal music
in the time of Henry VIII., 1530, may be gathered
from the accounts given in Hall and Hollinshed’a
Chronicles, of a Masque at Cardinal Wolsey’s palace,
Whitehall, where the King was entertained with ‘\textit{a concert
of drums and fifes}’” He then says, “But this was
soft music compared with that of his heroic daughter
Elizaheth, who, according to Hentzner, \textit{used to be regaled
during dinner} “with twelve trumpets, and two kettle-drums; 
which, \textit{together with fifes, cornets, and side-drums},
made the hall ring for half an hour together.” I find
nothing of the kind in Hall’s Chronicle (there is a short
notice of a similar Masque at Cardinal Wolsey’s, in the
tenth year of Henry VIII., fol. 65, b. 1548, but no drums
and fifes); and Hollinshed, who takes the account from
Cavendish’s \textit{Life of Wolsey}, is speaking not of a “concert”
at the Cardinal’s, but of the manner of receiving the King
and some of his nobles, who came by water to a Masque;
firstly by firing off “divers chambers” (short guns that
make a loud report) at his landing, and then conducting
him up into the chamber “with such a noise of drums
and fleutes, as seldom had been heard the like.” Cavendish
says, “with such a number of drums and fifes as
I have seldom seen together at one time in any masque”
(Singer’s edit., 8vo., 1825); and, describing the masques
generally, says, “Then was there all kind of music and
harmony set forth, with excellent voices both of men and
children.” Sagudino, the Venetian Ambassador, who
describes a banquet given by Henry VIII., in honor of
the Flemish envoys, on the 7th July, 1517, says, “during
the dinner there were boys on a stage in the centre of the
hall, some of whom sang, and others played the flute, rebeck, 
and virginals, making the sweetest melody.” As to
Queen Elizabeth, I quote Hentzner’s words from the copy
used by Dr.~Burney: “During the time this guard, which
consists of the tallest and stoutest men that can he found
in all England, \textit{were bringing dinner}, twelve trumpets and
two kettle-drums made the hall ring for half an hour together.” 
(This was the loud music to give notice to prepare
for dinner, like the gong, or dinner-bell of the present
day, but the fifes, cornets, and side-drums, are of Dr.
Burney's invention.) “\textit{At the end} of all this ceremonial
a number of unmarried ladies appeared, who with particular
solemnity lifted the meat off the table, and conveyed
it into the Queen's inner and more private chamber, where,
after she had chosen for herself, the rest goes to the ladies
of the Court. The Queen dines and sups alone, with very
few attendants,” \&c. Hentzner also says, “Without the
city” (of London) “are some theatres where English
actors represent almost every day tragedies and comedies
to very numerous audiences: these are concluded with
\textit{excellent music}, variety of dances, and the excessive
applause of those that are present.” The original words
are “quas variis etiam saltationibus, suavissimâ adhibitâ
musicâ, magno cum populi applausu finire solent.” Again,
in summing up the character of the English in a few
lines, he says, “\textit{They excel in dancing and music, for they
are active and lively}, though of a thicker make than the
French.” Dr.~Burney, throughout his History, writes in
a similarly disparaging strain about English music and
English musicians, for which I am unable to account.
} %end footnote
and about the same time Orlando Gibbons
published his \textit{Fantasies of three parts for Viols}.\dcfootnote{\scriptsizer %
For the republication of these, and many other works of
the sixteenth and seventeenth centuries, the world is indebted to the Musical Atiquarian Society. 
The Madrigals
of Wilbye, Weelkes, Bennet, Bateson, and Gibbons;
the Ballets of Morley and Hilton; the four-part songs of
Dowland, and four Operas by Purcell; besides the first
music printed for the Virginals, the four-part Psalms by
Este, and various Anthems, \&c., \&c.
} %end footnote
  \pagebreak
%%246
%%================================================

Viols had six strings, and the position of the fingers was marked on the fingerboard
by frets, as in guitars of the present day. The “Chest of Viols” consisted
of three, four, five, or six of different sizes; one for the treble, others for the mean,
the counter-tenor, the tenor, and perhaps two for the base. Old English musical
instruments were commonly made of three or four different sizes, so that a player
might take any of the four parts that were required to fill up the harmony. So
Violins, Lutes, Recorders, Flutes, Shawms, \&c., have been described by some
writers in a manner which (to those unacquainted with this peculiarity) has
appeared irreconcileable with other accounts. Shakespeare (in \textit{Hamlet}) speaks of
the Recorder as a little pipe, and says, in \textit{A Midsummer Night's Dream}, “he hath
played on his prologue like a \textit{child} on a recorder;” but in an engraving of the
instrument,\dcfootnote{\scriptsizerr %a
See “The Genteel Companion for the Re\-corder,” by
Humphrey Salter, 1683. Recorders and (English) Flutes
are to outward appearance the same, although Lord Bacon,
in his \textit{Natural History}, cent, iii., sec. 221, says the Recorder
hath a less bore, and a greater above and below.
The number of holes for the fingers is the same, and the
scale, the compass, and the manner of playing, the same.
Salter describes the \textit{recorder} from which the instrument
derives its name, as situate in the upper part of it, \ie,
between the hole below the mouth and the highest hole
for the finger. He says, “Of the kinds of music, vocal
has always had the preference in esteem, and in consequence, 
the Recorder, \textit{as approaching nearest to the
sweet delightfulness of the voice}, ought to have first place
in opinion, as we see by the universal use of it confirmed.”
The Hautboy is considered now to approach most nearly
to the human voice, and Mr. Ward, the military instrument
manufacturer, informs me that he has seen “old
English Flutes” with a hole bored through the side, in
the upper part of the instrument, the holes being covered
with a thin piece of skin, like gold-beater's skin. I suppose
this would give somewhat the effect of the quill or
reed in the Hautboy, and that these were Recorders. In
the proverbs at Leckingfield (quoted ante Note \textsuperscript{b}, p.~35),
the Recorder is described as “desiring” the mean part,
but manifold fingering and stops bringeth high (notes)
from its clear tones. This agrees with Salter’s book. He
tells us the high notes are produced by placing the thumb
\textit{half} over the hole at the back, and blowing a little stronger.
Recorders were used for teaching birds to pipe.
} %end footnote
it reaches from the lip to the knee of the performer; and among
those left by Henry VIII. were Recorders of box, oak, and ivory, great and small,
two base recorders of walnut, and one \textit{great} base recorder. In the same catalogue
we find “flutes called Pilgrims’ staves,” which were probably six feet long.

Richard Braithwait, a writer of this reign, has “set down \textit{Some Rules for the
Government of the House of an Earl},” in which the Earl was to keep “five
musitions skillfull in that commendable sweete science,” and they were required
to teach the Earl’s children to sing, and to play upon the base-viol, the virginals,
the lute, and the bandora, or cittern. When he gave “great feasts,” the musicians
were to play, whilst the service was going to the table, upon Sackbuts,
Cornets, Shawms, and “such other instruments going with wind;”\dcfootnote{\scriptsizerr %b
In Middleton’s play, \textit{The Spanish Gipsy}, act ii., sc.~1,
is another allusion to the loud music while dinner was
being carried in, as well as a common pun upon sackbuts
and sack.

\textit{Alv}. “You must not look to \textit{have your dinner served in
with trumpets}”

\textit{Car}. “No, no, \textit{sack-buts} shall serve us.”
} %end footnote
and upon
“Viols, Violins, or other \textit{broken}\dcfootnote{\scriptsizerr %c
\looseness=-1 “Broken Music,” as is evident from this and other
passages, means what we now term “a string band.”
Shakespeare plays with the term twice: firstly in\textit{ Troilus
and Cressida}, act iii., sc.~1, proving that the musicians then,
on the stage were performing on stringed instruments;
and secondly in \textit{Henry V}., act v., sc. 2, where he says to
the French Princess Katherine, “Come, your answer in
broken music; for thy voice is music and thy English
broken.” The term originated probably from harps, lutes,
and such other stringed instruments as were played without
a bow, not having the capability to sustain a long note
to its full duration of time.
} %end footnote 
musicke,” during the repast.

The custom of retaining musicians in the service of families continued to the
time of the Protectorate. It was not confined to men of high rank (either in this
or the preceding century), but was general with the wealthy of all classes. 
\pagebreak
%%247
%%================================================
\noindent So the old merchant in Shirley’s \textit{Love Tricks} (licensed 1625) says, “I made a
ditty, and my musician, \textit{that I keep in my house to teach my daughter}, hath set it
to a very good air, he tells me.” At least one wealthy merchant of the reign of
Henry VIII. retained as many musicians in his service as are prescribed for the
household of an Earl in James’ reign. Sir Thomas Kytson, citizen and mercer,
built Hengrave Hall, in Suffolk, between the years 1525 and 1538, and at the
death of his son (towards the close of Elizabeth’s reign) inventories of all the furniture
and effects were taken, including those of “the chamber where the musicyons
playe,” and of the “instruments and books of musicke” it contained.\dcfootnote{ %a
\looseness=-1 \textit{History and Antiquities of Hengrave}, by John Gage,
F.S.A., fol., 1822. There are six viols in a chest; six
violins in a chest (in 1572 a treble violin cost 20s.); seven
recorders in a case; besides lutes, cornets, bandoras,
citterns, sackbuts, flutes, hautboys, a curtall (or short sort
of bassoon), a lysarden (base cornet, or serpent), a pair of
little virginals, a pair of double virginals, “a wind instrument
like a virginal,” and a pair of double organs.
} %end footnote a
With the
exception of those for the lute, all the books of instrumental music were in sets of
five (for music in five or more parts), as well as those containing the vocal music,
described as “old.” The number of musicians was perhaps increased by his son,
for in the household expenses of the year 1574, we find, “seven cornets bought
for the musicians;” and the viols, violins, and recorders, in the inventory, are
(like those of Henry VIII) in chests or cases containing six or seven of each;
whilst much of the vocal music required six, and some seven and eight, voices
to sing it. In 1575 he lent the services of Robert Johnson, Mus. Bac., one of
his musicians, to the Earl of Leicester, on the occasion of the pageants at
Kenilworth.

Although we have no old English book written for the purpose of describing the
musical instruments in use in former days, like those of Mersenne and Kircher
for France and Germany, we find in our translations of the Bible and the
Metrical Psalms, the names of all in general use at the times those translations
were made, for the Hebrew instruments are all rendered by the names of such as
were then commonly known. We are so accustomed to picture David playing
on the harp, that we are not easily reconciled to the French version of the
Psalms, in which, in translations of the same passages, the violin is the instrument
assigned to him; and what we translate lute, they render bagpipe (\textit{musette}).
It is not my purpose to enter upon a detailed account of musical instruments,\dcfootnote{ %b
Sir John Hawkins’ descriptions of musical instruments
are too much drawn from foreign sources. English
instruments often differed materially from those in use
abroad, as many do at the present day. I cannot agree
with his description of the Cittern (it has too many strings)
or of some others. The catalogue of musical instruments
left by Henry VIII. (Harl. MSS. 1419, fol. 200)
was unfortunately unknown to him, or it would have
explained many~difficulties.
} %end footnote 
but the curious in such matters will find in Sir William Leighton’s “Teares or
Lamentations of a sorrowful soule,” a long catalogue of those known at this period.
It is contained in “A thanksgiving to God, with magnifying of his holy name \textit{upon
all instruments}.\dcfootnote{ %c
A copy with music in the British Museum. Among
the instruments not mentioned by Drayton are the following, 
which I give in Sir William Leighton’s spelling:—
“Regalls, Simballs, Timbrell, Syrons, Crowdes, Claricoales, 
Dulsemers, Crouncorns, and Simfonie.” He mentions
the Drum after the Simphony, thereby apparently
drawing a distinction between them, but according to
Bartholomeus \textit{De Proprietatibus Rerum}, printed by
Wynken de Worde, the Simphony is “an instrument
of musyke... made of an holowe tree, closed in lether
in eyther syde, and mystrels betyth it wyth styckes.”
“Crouncorn” means, perhaps, Krumhorn or Cromhorn, a
crooked horn, in imitation of which we have a reed stop in
old organs called the Cromhorn, which is now corrupted
into Cremona. Henry VIII., at his death, left several
cases containing from four to seven Crumhorns in each.
} %end footnote
In the following lines from Song IV. in Drayton’s \textit{Poly-olbion},
printed in the same year (1613), many of those in common use are cited:— 
\pagebreak

%%248
%%================================================
\settowidth{\versewidth}{On which the practic’d hand with perfect’st fing’ring strikes,}
\begin{scverse}
\vleftofline{“}When now the British side scarce finished their song,\\
But th’ English, that repin’d to be delay’d so long.\\
All quickly at the hint, as with one free consent,\\
Struck up at once and sung, each to the instrument\\
(Of sundry sorts that were, as the musician likes).\\
On which the practic’d hand with perfect’st fing’ring strikes,\\
Whereby their height of skill might liveliest be exprest.\\
The trembling Lute some touch, some strain the Viol best.\\
In sets that there were seen, the music wondrous choice.\\
Some, likewise, there affect the Gamba with the voice,\\
To shew that England could variety afford.\\
Some that delight to touch the sterner wiry chord,\\
The Cithren, the Pandore, and the Theorbo strike:\\
The Gittern and the Kit the wand’ring fiddlers like.\\
So were there some again, in this their learned strife,\\
Loud instruments that lov’d, the Cornet\dcfootnote{ %a
Among Henry the Eighth’s instruments were “Gitteron
Pipes of ivory or wood, called Cornets” The Cornet
described by Mersenne is of a bent shape, like the segment
of a large circle, gradually tapering from the bottom to
the mouth-piece, The cornet was of a loud sound, but
in skilful hands could be modulated so as to resemble the
tones of the human voice. In Ben Jonson’s Masque of
\textit{Neptune’s Triumph}, the instruments employed were five
Lutes and three Cornets. In several other Masques, Lutes
and Cornets were the only instruments used. At the
Restoration, Cornets supplied the deficiency of boys'
voices in Cathedral Service. The base Cornet was of a
more serpentine form, and from four to five feet in length;
but Mersenne says, the Serpent (contorted to render it
more easy of carriage, as its length was six feet one inch)
was the genuine base of that instrument.
} %end footnote a
and the Fife,\\
The Hoboy, Sackbut deep, Recorder, and the Flute;\\
E’en from the shrillest Shawm unto the Cornamute.\\
Some blow the Bagpipe up, that plays the Country-Round;\\
The Tabor and the Pipe some take delight to sound.”\\
\attribution \textit{ The Sundry Musiques of England}.
\end{scverse}

In consequence of the almost universal cultivation of music in the sixteenth
century, and of the great employment and encouragement of musicians, so many
persons embraced music as a profession, that England overflowed with them.
Many travelled, and some were tempted by lucrative engagements to settle abroad.
Dowland, whose “touch upon the lute” was said to “ravish human sense,”
travelled through Italy, France, Germany, and the Netherlands, and about the
year 1600 became lutenist to the King of Denmark. On Dowland’s return to
England in 1607, Christian IV. begged of Lady Arabella Stuart (through the
Queen and Prince Henry) to allow Thomas Cutting, another famous lutenist, then
in her service, to replace him. Peter Phillips, better known on the continent
(where the greater part of his works were printed) as Pietro Philippi, accepted an
engagement as organist to the Arch-duke and Duchess of Austria, governors of
the Low Countries, and settled there. John Cooper spent much of his life in
Italy, and was called Coprario, or Cuperario, There were few, if any, Italian
composers or singers then in England,\dcfootnote{ %b
Alfonso Ferabosco, the elder, was born, of Italian
parents, at Greenwich. As he was brought up and lived
in England, he can scarcely he considered as an Italian
musician. Nicholas Lanier was an Italian by birth, and
came to England as an engraver. He settled here, and
became an eminent musician.
} %end footnote b
and the music of Italy was chiefly known
by the Madrigal, for the sacred music, as being for the service of the Mass, was
strictly~prohibited. 

\pagebreak
%%249
%%================================================

Anthony à Wood tells the following story of Dr.~John Bull:—While
travelling \textit{incognito} through France and Germany for the recovery of his health,
he heard of a famous musician belonging to the Cathedral of St. Omer, and
applied to him to see his works. The musician having conducted Bull to a vestry
or music-school adjoining the Cathedral, shewed him a lesson or song of forty
parts, and then made a vaunting challenge to any person in the world to add one
more part, supposing it so complete that it was impossible to correct or add to it.
Dr.~Bull having requested to be locked up for two or three hours, speedily added
forty more parts, whereupon the musician declared that “he that added those
forty parts must either be the devil or Dr.~John Bull.”\dcfootnote{ %a
Such exercises of learned ingenuity were common in
that day. Tallis wrote a Motet in forty parts, a copy of
which is now before me. It is for eight choirs, each of
five voices; the voices only coming together occasionally.
Dr.~Burney discredits Dr.~Bull’s feat as “impossible,”
but I am assured by Dr.~Rimbault and by Mr. Macfarren,
who have seen this Motet, that whether the story be true
or not, it was quite possible. In all cases the anecdote
may be taken as a proof of the very high reputation Dr.
Bull enjoyed.
} %end footnote a
In 1613, Bull (to
whom many offers of preferment at foreign courts had been previously made)
quitted England, and went to reside in the Netherlands, where he entered the
service of the Archduke.

The emigration of musicians was not confined to a few of the most eminent, for
we hear, indirectly, of many in the employ of foreign courts, whose movements
would not otherwise be recorded. Thus Taylor, the water-poet, who had just
described the Lutes, Viols, Bandoras, Recorders, Sackbuts, and Organs, in the
Chapel of the Graf (or Count) of Schomburg, says, “I was conducted an English
mile on my way by certain of my countrymen, my Lord’s musicians.”

We are indebted to foreign countries for the preservation of many of the works
of our best musicians of this age, as well as of our popular tunes. Dr.~Bull’s
music is chiefly to be found in foreign manuscripts.\dcfootnote{ %b
One foreign manuscript volume of Dr. Bull’s works
is now in my possession, and another in that of Mr.
Richard Clarke, who asserts that it contains “God save
the King,” of which more hereafter. The contents of
both are described in Ward’s \textit{Lives of the Gresham Professors}.
} %end footnote b
Dowland tells us that “some
part of his poor labours” had been printed in eight cities beyond the seas, viz.,
Paris, Antwerp, Cologne, Nuremburg, Frankfort, Leipzig, Amsterdam, and Hamburg. 
Much of the music printed in Holland in the seventeenth century was also
by English Composers. The right of printing music in England was a monopoly,
generally in the hands of one or two musicians,\dcfootnote{ %c
It was held by Tallis and Byrd from 1575 to 1596, then
by Morley and his assignee. See Introduction to Rimbault’s \textit{Bibliothica Madrigaliana}, 8vo., 1847.
} %end footnote c
and therefore very little, and
only such as they chose, could be printed. Hence the scarcity, as well as the
frequent imperfection, of these early~works.

In London, each ward of the city had its musicians; there was also the Finsbury
Music, the Southwark and the Blackfriars Music, as well as the Waits of
London and Westminster. Morley thus alludes to the Waits, in the dedication
of his \textit{Consort Lessons} to the Lord Mayor and Aldermen: “As the ancient
custom of this most honourable and renowned city hath been ever to retain and
maintain excellent and expert musicians to adorn your honours’ favours, feasts,
and solemn meetings: to those, your Lordships’ Wayts, I recommend the same.”
A “Wayte,” in the time of Edward IV., had to \textit{pipe} watch four times in the 
night, from Michaelmas to Shrovetide, and three in the summer, as well as to 
%%250
%%================================================
“make \textit{bon gayte}” at every chamber door; \pagebreak but Morley’s \textit{Consort Lessons}, as
before mentioned, required six instruments to play them,\dcfootnote{
A few specimens of the tunes of the waits of different
towns will be given under the reign of Charles~II.
} %end footnote
and the city bands are
commonly quoted as playing in six parts.\dcfootnote{
So in Heywood’s \textit{The English Traveller}, last scene of act i., 1633--
\settowidth{\versewidth}{\textit{Riot}. The best consort in the city for six parts.}
\begin{fnverse}
\vleftofline{“}\textit{Riot}. Fear not you shall have a full table.\\
\textit{Young L}. What, and music?\\
\textit{Riot}. The best consort in the city for six parts.\\
\textit{Young L}. We shall have songs, then?”\\
\end{fnverse}
} %end footnote

After the act of the 39th year of Elizabeth, which rendered all “minstrels
wandering abroad” liable to punishment as “rogues, vagabonds, and sturdy
beggars,” all itinerant musicians were obliged to wear cloaks and badges, with the
arms of some nobleman, gentleman, or corporate body, to denote in whose service
they were engaged, being thereby excepted from the operation of the act. So in
\textit{Ram Alley}, 1611, Sir Oliver says—
\settowidth{\versewidth}{Lightly, lightly, and by my knighthood’s spurs}
\begin{scverse}
\vin\vin\vin\vin “Musicians, on!\\
Lightly, lightly, and by my knighthood’s spurs\\
This year you shall have \textit{my protection},\\
And yet not buy your livery coats yourselves.”
\end{scverse}

And as late as 1699, we find in \textit{Historia Histrionic}a, “It is not unlikely that the
lords in those days, and persons of eminent quality, had their several gangs of
players, as some \textit{have now} of fiddlers, to whom they give cloaks and badges.”

Musicians in the service of noblemen and gentlemen seem to have held a
prescriptive right to go and perform to the friends and acquaintances of their
masters, whenever they wanted money: such visits were received as compliments,
and the musicians were rewarded in proportion to the rank of their masters.
Innumerable instances of this will be found in early books of household expenditure; 
but, in James’ reign, musicians not actually in employ presumed so far
upon the license, that their intrusion into all companies, and at all times, became
a constant subject of rebuke. Ben Jonson’s Club, the Apollo, which met at the
Devil tavern, chiefly for conversation, was obliged to make a law that no fiddler
should enter, unless requested.\dcfootnote{
The rules of this club, in Latin, will be found in Ben
Jonson’s Works. the following translation is by one of
his adopted poetical sons:—
\settowidth{\versewidth}{Let none but guests, or clubbers, hither come;}
\begin{fnverse}
\vleftofline{“}Let none but guests, or clubbers, hither come;\\
Let dunces, fools, sad sordid men, keep home,\\
Let learned, civil, merry men b’invited,\\
And modest, too; nor be choice ladies slighted.\\
Let nothing in the treat offend the guests;\\
More for delight than cost, prepare the feasts.\\
The cook and purvey’r must our palates know,\\
And none contend who shall sit high or low.\\
Our waiters must quick-sighted be, and dumb,\\
And let the drawers quickly hear and come.\\
Let not our wine be mix'd, but brisk and neat,\\
Or else the drinkers may the vintners beat.\\
And let our only emulation be,\\
Not drinking much, but talking wittily.\\
Let it be voted lawful to stir up\\
Each other with a moderate chirping cup;\\
Let not our company be, or talk too much;\\
On serious things, or sacred, let's not touch\\
With sated heads and bellies. Neither may\\
Fiddlers unask’d obtrude themselves to play.\\
With laughing, leaping, dancing, jests and songs,\\
And whate’er else to grateful mirth belongs,\\
Let’s celebrate our feasts: and let us see\\
That all our jests without reflection be.\\
Insipid poems let no man rehearse,\\
Nor any be compelled to write a verse.\\
All noise of vain disputes must be forborn,\\
And let no lover in a corner mourn.\\
To fight and brawl, like Hectors, let none dare,\\
Glasses or windows break, or hangings tear.\\
Whoe’er shall publish what’s here done or said,\\
From our society must be banished.\\
Let none by drinking do or suffer harm,\\
And, while we stay, let us be always warm.”\\
\vin \textit{Poems and Songs by Alexander Brome}, 8vo., 1661
\end{fnverse}
} %end footnote
Nevertheless, they were generally welcome, and
generally well paid; more especially, at merry-makings where their services were
ever required. In those days a wedding was of a much gayer character than \pagebreak
now. There was first the hunt’s-up, or morning song, to awake the bride; then  
%%251
%%================================================
the music to conduct her to church (young maids and bachelors following, with
garlands in their hands); the same from church; the music at dinner; and
singing, dancing, and merry-making throughout the evening. For those who had
no talent to write a hunt’s-up, there were songs ready printed (like “The Bride’s
Good-morrow,” in the Roxburghe Collection), but the hunt’s-up was not confined
to weddings, it was a usual compliment to young ladies, especially upon their
birthdays. The custom seems now to be continued only with princesses, and on
the last birthday of the Princess Royal, the court newsman, at a loss how to
describe this old English custom, gave it the name of a “Matinale.”

As to music at weddings, see the following allusions:—

“Then was there a fair bride-cup of silver and gilt carried before her [the
bride], wherein was a goodly braunch of rosemarie gilded very faire, hung about
with silken ribbonds of all colours; next there was \textit{a noyse\dcfootnote{
A noise of musicians means a company of musicians.
It is an expression frequently occurring: “those terrible
\textit{noyses}, with threadbare cloakes, that live by red lattices
and ivy-bushes” [that is by ale-houses and taverns],
“having authority to thrust into any man’s room, only
speaking but this—‘Will you have any musicke?’”---
Dekker's \textit{Belman of London}, 1608.
}
of musitians, that
played all the way before her}; after her came all the chiefest maydens of the
countrie, some bearing great bride-cakes, and some garlands of wheat finely
gilded, and so she past unto the church.”—Deloney’s \textit{Pleasant History of John
Winchcomb, in his younger years called Jacke of Newberie}.

“Come, come, we’ll to church presently. Prythee, Jarvis, \textit{whilst the musick
plays just upon the delicious close}, usher in the brides.”—Rowley’s \textit{A Match at
Midnight},~1633.

In Ben Jonson’s \textit{Tale of a Tub}, Turfe, the constable, “will let no music go afore
his child to church,” and says to his wife—
\settowidth{\versewidth}{Because you have entertained [musicians] all from Highgate,}
\begin{scverse}
\vleftofline{“}Because you have entertained [musicians] all from Highgate,\\
To shew your pomp, you’d have your daughters and maids\\
Dance o’er the fields like faies to church this frost.\\
I’ll have no \textit{rondels}, I, in the queen’s paths!\\
Let them scrape the gut at home, where they have fill’d it.”
\end{scverse}
And again, where Dame Turfe insists on having them to play at dinner, Clench
adds—
\begin{scverse}
\vleftofline{“}She is in the right, sir, vor your wedding dinner\\
Is starv’d without the music.”
\end{scverse}

Even at funerals musicians were in request: dirges were sung, and recorders the
instruments usually employed. It appears that the Blue-coat boys sang at City
Funerals;\dcfootnote{
See Brome’s \textit{City Wit}, act iii. sc. 1.
} 
being then taught music, as they \textit{should} be now. Music was not less
esteemed as a solace for grief, than as an excitement to merriment. Peacham says,
“the physicians will tell you that the exercise of music is a great lengthener of life,
by stirring and reviving the spirits, holding a secret sympathy with them; besides
it is an enemy to melancholy and dejection of mind; yea, a curer of some diseases.” 
(\textit{Compleat Gentleman}, 1622.) And Burton, “But I leave all declamatory
speeches in praise of divine music, I will confine myself to my proper subject:
besides that excellent power it hath to expel many other diseases, it is a sovereign
remedy against despair and melancholy, and will drive away the devil himself.” 
(\textit{Anatomy of Melancholy}.) So, in \textit{Henry IV}., Shakespeare says—  

\pagebreak
%%252
%%================================================

\settowidth{\versewidth}{Let there be no noise made, my gentle friends.}
\begin{scverse}
\vleftofline{“}Let there be no noise made, my gentle friends.\\
Unless some slow and favourable hand\\
Will whisper music to my weary spirit.”\\
\attribution \textit{Part II}, act iv., sc. 9.
\end{scverse}

Shakespeare purchased his house in Blackfriars, in 1612, from Henry Walker,
who is described in the deed as “Citizen and Minstrel, of London.” The price
paid was £140,\dcfootnote{
Shakespeare’s autograph, attached to the counterpart
of this deed, was sold by auction by Evans, on 24th May,
1841, for £155.}
which, considering the difference in the value of money, is equal
to, at least, £700 now. Of what class of “minstrel” Walker was, we know not,
but there were very few of any talent who had not the opportunity of saving money,
if so disposed. Even the itinerant fiddler who gave “a fytte of mirth for a groat,”
was well paid. The long ballads were usually divided into two or three “fyttes,”
and if he received a shilling per ballad, it would purchase as many of the necessaries
of life as five or six times that amount now. The groat was so generally his
remuneration, whether it were for singing or for playing dances, as to be
commonly called “fiddlers’ money,” and when the groat was no longer current,
the term was transferred to the sixpence.

It appears that in the reign of James, ballads were first collected into little
miscellanies, called Garlands, for we have none extant of earlier date. Thomas
Deloney and Richard Johnson (author of the still popular boys’ book, called \textit{The
Seven Champions of Christendom}) were the first who collected their scattered productions, 
and printed them in that form.

Deloney’s \textit{Garland of Good-will}, and Johnson’s \textit{Crown Garland of Golden Roses},
were two of the most popular of the class. They have been reprinted, with some
others, by the Percy Society, and the reader will find some account of the authors
prefixed to those works.

During the reign of Henry VIII., “the most pregnant wits” were employed
in compiling ballads.\dcfootnote{
See \textit{The Nature of the Four Elements}, written about
1517.
} 
Those in the possession of Captain Cox, described in
Laneham’s \textit{Letter from Kenilworth} (1575), as “all ancient,”\dcfootnote{
The list of Captain Cox’s ballads has been so often reprinted,
that I do not think it necessary to repeat it. The
reader will find it, with many others, in the introduction
to Ritson’s \textit{Ancient Songs}, as well as in more recently printed
books.
}
could not well be
of later date than Henry’s reign; and at Henry’s death we find, with the list of
musical instruments left in the charge of Philip van Wilder, “sondrie bookes and
\textit{skrolles of songes and ballattes}.” In the reign of James, however, poets rarely
wrote in ballad metre; ballad writing had become quite a separate employment,
and (from the evidently great demand for ballads) I should suppose it to have
been a profitable one. In Shakespeare’s \textit{Henry IV}., when Falstaff threatens
Prince Henry and his companions, he says, “An I have not ballads made on you
all, and sung to filthy tunes, let a cup of sack be my poison;” and after Sir
John Colvile had surrendered, he thus addresses Prince John: “I beseech your
grace, let it he booked with the rest of this day’s deeds; or by the Lord, I will
have it in a particular ballad else, with mine own picture at the top of it, Colvile
kissing my foot.”

To conclude this introduction, \pagebreak I have subjoined a few quotations to shew the  
%%253
%%================================================
universality of ballads, as well as their influence upon the public mind; but limiting
myself to dramatists, to Shakespeare’s contemporaries, and to one passage
from each, author.


In Ben Jonson’s \textit{Bartholomew Fair}, when Trash, the gingerbread-woman,
quarrels with Leatherhead, the hobby-horse seller, she threatens him—
\settowidth{\versewidth}{“I'll find a friend shall right me, and make a ballad of thee, and thy cattle all over.”}
\begin{scverse}“I'll find a friend shall right me, and make a ballad of thee, and thy cattle all over.”
\end{scverse}

In Heywood’s \textit{A Challenge for Beauty}, Valladaura says—
\settowidth{\versewidth}{But you must have some scurvy pamphlets and lewd}
\begin{scverse}“She has told all; I shall be balladed—\\
Sung up and down by minstrels.’’
\end{scverse}

In Fletcher’s \textit{Queen of Corinth}, Euphanes says—
\begin{scverse}\dots “and whate’er he be\\
Can with unthankfulness assoil me, let him\\
Dig out mine eyes, and sing my name in verse,\\
In ballad verse, at every drinking-house.’’
\end{scverse}

In Massinger’s \textit{Parliament of Love}, Chamont threatens Lamira—
\begin{scverse}\dots “I will have thee\\
Pictured as thou art now, and thy whole story\\
Sung to some villainous tune in a lewd ballad,\\
And make thee so notorious in the world,\\
That boys in the streets shall hoot at thee.”
\end{scverse}

In Chapman’s \textit{Monsieur d’Olive}, he says—
\begin{scverse}\vleftofline{“}I am afraid of nothing but I shall be balladed.”
\end{scverse}

In a play of Dekker’s (Dodsley, iii. 224) Matheo says—

“Sfoot, do you long to have base rogues, that maintain a Saint Anthony’s fire in
their noses by nothing but two-penny ale, make ballads of you?”

In Webster’s \textit{Devil's Law Case}, the officers are cautioned not to allow any to
take notes, because—
\begin{scverse}\vleftofline{“}\vleftofline{“}We cannot have a cause of any fame,\\
But you must have some scurvy pamphlets and lewd\\
Ballads engendered of it presently.”
\end{scverse}

In Ford’s \textit{Love's Sacrifice}, Fiormonda says—
\begin{scverse}\dots “Better, Duke, thou hadst been born a peasant;\\
Now boys will sing thy scandal in the streets,—\\
Tune ballads to thy infamy.”
\end{scverse}

In Marlow’s \textit{Edward II}., Mortimer says to the King—
\begin{scverse}\vleftofline{“}Libels are cast against thee in the street;\\
Ballads and rhymes made of thy overthrow.”
\end{scverse}

In Machin’s \textit{The Dumb Knigh}t—
\begin{scverse}\vleftofline{“}The slave will make base songs on my disgrace.”
\end{scverse}

In Middleton’s \textit{The Roaring Gir}l—
\begin{scverse}“O, if men’s secret youthful faults should judge ’em,\\
’Twould be the general’st execution\\
That e’er was seen in England!\\
There would be few left to sing the ballads,\\
There would be so much work.”
\end{scverse}

This is in allusion to the ballads on last dying speeches.  
\pagebreak
%%254
%%================================================

In the academic play of \textit{Lingua}, Phantastes says—

“O heavens! how am I troubled these latter times with poets—ballad-makers. Were
it not that I pity the printers, these sonnet-mongers should starve for conceits for all
Phantastes.”
\centerrule
The popular music of the time of Charles I. was so much like that of James,
as not to require separate notice. I have therefore included many ballads
of Charles’ reign in this division; but reserved those which relate to the troubles
and to the civil war, for the period of the Protectorate.
\centerrule
\musictitle{Upon A Summer’s-Day.}

In \textit{The Dancing Master}, from 1650 to 1665, and in \textit{Musick’s Delight on the
Cithren}, 1666, this is entitled “Upon a Summer’s-day;” and in later editions of
\textit{The Dancing Master}, viz., from 1670 to 1690, it is called “The Garland, or a
Summer’s-day.”

The song, “Upon a Summer’s-day” is in \textit{Merry Drollery Complete}, 1661,
p.~148. “The Garland” refers, in all probability, to a ballad in the Roxburghe
Collection, i. 22, or Pepysian, i. 300; which is reprinted in Evans’ \textit{Old Ballads},
iv. 345 (1810), beginning, “Upon a Summer’s \textit{time}.” It is more frequently
quoted by the last name in ballads. In the Pepys Collection, vol. i., is a
“Discourse between a Soldier and his Love;”--
\settowidth{\versewidth}{For land nor sea could make her stay behind.}
\begin{scverse}\vleftofline{“}Shewing that she did bear a faithful mind,\\
For land nor sea could make her stay behind.\\
\attribution To the tune of \textit{Upon a Summer time}”
\end{scverse}
It begins, “My dearest love, adieu.” And at p.~182 of the same volume,
“I smell a rat: to the tune of \textit{Upon a Summer tide}, or \textit{The Seminary Priest}.”
It begins, “I travell’d far to find.”

In the Roxburghe Collection, vol. i. 526, “The good fellow’s advice,” \&c., to
the tune of \textit{Upon a Summer time};” the burden of which is—
\settowidth{\versewidth}{Good fellows, great and small,}
\begin{scverse}\vleftofline{“}Good fellows, great and small,\\
Pray let me you advise\\
To have a care withall;\\
’Tis good to be merry and wise.”
\end{scverse}
And at p.~384 of the same volume, another by L.P., called “Seldom cleanely, or—
\settowidth{\versewidth}{Then lend your attention, while I do unfold}
\begin{scverse}A merry new ditty, wherein you may see\\
The trick of a huswife in every degree;\\
Then lend your attention, while I do unfold\\
As pleasant a story as ever was told.\\
\attribution To the tune of \textit{Upon a Summer's time}.”
\end{scverse}
It begins—
\settowidth{\versewidth}{I’ll tell you here a new conceit,}
\begin{scverse}
\vleftofline{“}Draw near, you country girls,\\
And listen unto me;\\
I’ll tell you here a new conceit,\\
Concerning huswifry.”
\end{scverse}

I have chosen a song which illustrates an old custom, instead of the original
words to this tune, because it is not desirable to reprint them. In \textit{Wit and}~ 
\pagebreak
%%255
%%================================================
\textit{Mirth}, 1707, the following song, entitled \textit{The Queen of May}, is joined to an
indifferent composition:—

\musicinfo{Slowly and smoothly.}{}
\lilypondfile{lilypond/255-upon-a-summers-day}\normalsize

\settowidth{\versewidth}{And homewards straight they went.}
\begin{dcverse}\begin{altverse}
From morning till the evening\\
Their controversy held,\\
And I, as judge, stood gazing on,\\
To crown her that excell’d.\\
At last when Phœbus’ steeds\\
Had drawn their wain away,\\
We found and crown’d a damsel\\
To be the Queen of May.
\end{altverse}

\begin{altverse}
Full well her nature from her\\
Face I did admire;\\
Her habit well became her,\\
Although in poor attire.\\
Her carriage was so good,\\
As did appear that day,\\
That she was justly chosen\\
To be the Queen of May.
\end{altverse}

\begin{altverse}
Then all the rest in sorrow,\\
And she in sweet content,\\
Gave over till the morrow,\\
And homewards straight they went.\\
But she, of all the rest,\\
Was hinder’d by the way,\\
For ev’ry youth that met her,\\
Must kiss the Queen of May.
\end{altverse}
\end{dcverse}


\musictitle{The Hunter In His Career.}

This is one of the songs alluded to in Walton’s \textit{Angler}. \textit{Piscator}. “I’ll
promise you I’ll sing a song that was lately made at my request by Mr. William
Basse, one that made the choice songs of ‘The Hunter in his career,’ and ‘Tom
of Bedlam,’ and many others of note.” The tune was translated from lute
tablature by Mr. G. F. Graham, of Edinburgh. It is taken from the “Straloch
Manuscript,” formerly in the possession of Mr. Chalmers, the date of which is
given in the original MS. from 1627 to 1629. It is also in the Skene MS., \&c.
A copy of the song is in the Pepys Collection, i. 452, entitled “Maister Basse \pagebreak
his careere, or The Hunting of the Hare. To a new court tune.” Printed for
%%256
%%================================================
E[liz.] A[llde]. On the same sheet is “The Faulconer’s Hunting; to the tune
of \textit{Basse his careere}.” The words are also in \textit{Wit and Drollery, Jovial Poems},
1682, p.~64, and in \textit{Old Ballads}, second edition, 1738, iii. 196.

\musicinfo{With spirit.}{}
\lilypondfile[staffsize=16]{lilypond/256-the-hunter-in-his-career}\normalsize

\settowidth{\versewidth}{Dapple-grey waxeth bay in his blood;}
\indentpattern{110110110110}
\begin{dcverse}\begin{patverse}
\vin Now bonny bay\\
In his foine waxeth gray;\\
Dapple-grey waxeth bay in his blood;\\
White-Lily stops\\
With the scent in her chaps,\\
And Black-Lady makes it good.\\
Poor silly Wat,\\
In this wretched state,\\
Forgets these delights for to hear\\
Nimbly she bounds\\
From the cry of the hounds,\\
And the music of their career.
\end{patverse}

\begin{patverse}
\vin Hills, with the heat\\
Of the gallopers’ sweat\\
Reviving their frozen tops,\\
{[And]} the dale’s purple flowers,\\
That droop from the showers\\
That down from the rowels drops.\\
Swains their repast,\\
And strangers their haste\\
Neglect, when the horns they do hear;\\
To see a fleet\\
Pack of hounds in a sheet,\\
And the hunter in his career.
\end{patverse}

\begin{patverse}
\vin Thus he careers,\\
Over heaths, over meres,\\
Over deeps, over downs, over clay;\\
Till he hath won\\
The noon from the morn,\\
And the evening from the day.\\
His sport then he ends,\\
And joyfully wends\\
Home again to his cottage, where\\
Frankly he feasts\\
Himself and his guests,\\
And carouses in his career. 
\end{patverse}
\end{dcverse}

\pagebreak
%%257
%%================================================

\renewcommand\rectoheader{reigns of james i. and charles i.}

\musictitle{Once I Loved A Maiden Fair.}

A copy of this ballad is in the Roxburghe Collection, i. 350, printed for the
assigns of Thomas Symcock. The tune is in \textit{The Dancing Master}, from 1650 to
1698; in Playford’s Introduction, 1664; in \textit{Musick’s Delight on the Cithren}, 1666;
in \textit{Apollo’s Banquet for the Treble Violin}, 1670; in the \textit{Pleasant Companion for
the Flageolet}, 1680;~\&c.

The first song in Patrick Carey’s \textit{Trivial Poems}, written in 1651 (“Fair one!
if thus kind you be”), is to the tune \textit{Once I lov’d a maiden fair}. It is also
alluded to in \textit{The Fool turn’d Critic}, 1678—“We have now such tunes, such
lamentable tunes, that would make me forswear all music. \textit{Maiden fair} and \textit{The
King’s Delight} are incomparable to some of these we have now.”

The ballad consists of twelve stanzas, from which the following are selected.

\musicinfo{Smoothly and in moderate time.}{}
\lilypondfile{lilypond/257-once-i-loved-a-maiden-fair}\normalsize

\settowidth{\versewidth}{That the church should make us one,}
\begin{dcverse}\begin{altverse}
Three times I did make it known\\
To the congregation,\\
That the church should make us one,\\
As priest had made relation.\\
Married we straight must be.\\
Although we go a begging;\\
Now, alas! ’tis like to prove\\
A very hopeless wedding.
\end{altverse}

\begin{altverse}
Happy he who never knew\\
What to love belonged;\\
Maidens wavering and untrue\\
Many a man have wronged.\\
Fare thee well! faithless girl,\\
I’ll not sorrow for thee;\\
Once I held thee dear as pearl,\\
Now I do abhor thee.
\end{altverse}
\end{dcverse}
\pagebreak
%%258
%%================================================

\musictitle{Gathering Peascods.}
This beautiful air is contained in all editions of \textit{The Dancing Master}, from
1650 to 1690. The two first bars are the same as “All in a garden green” (see
p.~111); but the resemblance continues no further, and that air is in phrases of
eight, and this of six bars.

Not having been able to discover the original words,, the following lines were
written to it by the late Mr. J. A. Wade; retaining the pastoral character, which
is indicated by its name.

\musicinfo{Moderate time, and sustained}{}
\lilypondfile{lilypond/258-gathering-peascods}\normalsize

\pagebreak
%%259
%%================================================

\settowidth{\versewidth}{Flow’rs sweet to gaze on, as the songs of birds to hear,}
\begin{dcverse}\indentpattern{010111}
\begin{patverse}
And as I wander in the blossom of the year,\\
By crystal waters’ flow,\\
Flow’rs sweet to gaze on, as the songs of birds to hear,\\
Spring up where e'er I go!	 \\
The violet agrees,\\
With the honey-suckle trees,
\end{patverse}

\columnbreak

\indentpattern{0110}
\begin{patverse}
To shed their balms around!—\\
Thus from the busy throng,\\
I careless roam along,\\
\vleftofline{’}Mid perfume and sweet sound.
\end{patverse}
\end{dcverse}

\musictitle{Lull Me Beyond Thee.}

This tune is in \textit{The Dancing Master}, from 1650 to 1690.

In the Pepys Collection, i. 372, there is a black-letter ballad entitled “The
Northern Turtle, wailing his unhappy fate in being deprived of his sweet mate:
to a new Northern tune, or \textit{A health to Betty}.” This is not the air of \textit{A health of
Betty}, and therefore I suppose it to be the “new Northern tune.” The first
stanza is here arranged to the music. The same ballad is the Roxburghe Collection, 
i. 319, as the second part to one entitled “The paire of Northerne Turtles:
\settowidth{\versewidth}{Whose love was firm till cruel death}
\begin{scverse}Whose love was firm till cruel death\\
Depriv’d them both of life and breath.”
\end{scverse}
That is also to “a new Northern tune,” and printed “for F. Coules, dwelling in
the Old Baily.” Coules printed about 1620 to 1628.

The following ballads are also to the tune:—

Pepys, i. 390—
\settowidth{\versewidth}{Which gives content unto a man’s life.}
\begin{scverse}\vleftofline{“}A constant wife, a kind wife,\\
Which gives content unto a man’s life.
\end{scverse}
To the tune of \textit{Lie lulling beyond thee}.'’ Printed for F. C[oules]. It begins—
\settowidth{\versewidth}{“Young men and maids, do lend me your aids.”}
\begin{scverse}“Young men and maids, do lend me your aids.”
\end{scverse}

Pepys i., and Roxburghe, i. 156—“The Honest Wooer,
\begin{scverse}His mind expressing, in plain and few terms,\\
By which to his mistris his love he confirms:”
\end{scverse}
to the tune of \textit{Lulling beyond her}, begins—
\begin{scverse}\begin{altverse}
\vleftofline{“}Fairest mistris, cease your moane,\\
Spoil not your eyes with weeping,\\
For certainly if one be gone,\\
You may have another sweeting.\\
I will not compliment with oaths,\\
Nor speak you fair to prove you;\\
But save your eyes, and mend your clothes,\\
For it is I that love you.”
\end{altverse}
\end{scverse}

Roxburghe, i. 416—“The two fervent Lovers,” \&c., “to the tune of \textit{The two
loving Sisters}, or \textit{Lulling beyond thee}.” Signed L.P.

Pepys, i. 427—
\begin{scverse}\vleftofline{“}A pleasant new ballad to sing both even and morn,\\
Of the bloody murther of Sir John Barley-Corne.
\end{scverse}
To the tune of \textit{Shall I lie beyond thee}.” Printed at London for H[enry] G[osson].
It commences thus:—
\begin{scverse}\vleftofline{“}As I went through the North country,\\
I heard a merry greeting,” \&c.
\end{scverse}

This excellent ballad has been reprinted by Evans (\textit{Old Ballads} iv. 214,
ed. 1810), from a copy in the Roxburghe Collection, “printed for John Wright.”
\pagebreak
%%260
%%================================================

\musicinfo{Smoothly and rather slow.}{}
\lilypondfile{lilypond/260-lull-me-beyond-thee}\normalsize

\musictitle{Come, Shepherds, Deck Your Heads.}

This is also one of the songs mentioned by old Isaak Walton.

\textit{Milkwoman}. “What song was it, I pray? was it ‘Come, shepherds, deck your
heads;’ or, ‘As, at noon, Dulcina rested;’ or, ‘Philida flouts me;’ or, Chevy
Chace;’ or, ‘Johnny Armstrong;’ or, ‘Troy Town?’”\footnote
{All will he found in this collection except “Johnny
Armstrong” of which (although an English song, and of
a Westmoreland man) I have not yet found the tune. The
words are in \textit{Wit restored}, 1658, and in \textit{Wit and Drollery,
Jovial Poems}, 1682, called “A Northern Ballet,” beginning—
\settowidth{\versewidth}{There dwelt a man in fair Westmorland,}
\begin{fnverse}
\begin{altverse}
\vleftofline{“}There dwelt a man in fair Westmorland,\\
Johnny Armstrong men did him call;\\
He had neither lands nor rents coming in,\\
Yet he kept eight score men in his hall.”
\end{altverse}
\end{fnverse}
%\columnbreak
There is also a Scotch ballad ahout the same hero.}

Izaak Walton was born in 1593, and married first Rachel Cranmer, niece of
that distinguished prelate, Thomas Cranmer, Archbishop of Canterbury, in 1624.

The air is found, under its English name, in \textit{Bellerophon, of Lust tot Wÿshed},
Amsterdam, 1622; and in \textit{Gesangh der Zeeden}, Amsterdam, 1662.\footnote
{There is another English tune under the same name,
to be found in two other collections, \textit{Nederlandtsche Gedenck-Clanck},
1626, and \textit{Friesche Lust-Hof}, 1634. I printed
it in \textit{National English Airs}, 1839, but think this rather
more like a ballad-tune, and it is of somewhat earlier
authority.}

The words (which Ritson said “are not known”) will be found in the Pepys
Collection, i. 366, entitled “The Shepherd’s Lamentation: to the tune of
\pagebreak
%%261
%%================================================
\textit{The plaine-dealing Woman}” On the other half of the sheet is “The second part
of \textit{The plaine-dealing Woman}” beginning—
\settowidth{\versewidth}{“Ye Sylvan Nymphs, come skip it,” \&c.}
\begin{scverse}
“Ye Sylvan Nymphs, come skip it,” \&c.
\end{scverse}

Imprinted at London for J. W. Sir Harris Nicolas prints the song. \textit{Come,
shepherds}, in his edition of Walton’s \textit{Angler}, from a MS. formerly in the possession
of Mr. Heber. A third copy will be found in MSS. Ashmole, No. 38,
art. 164.

\musicinfo{Moderate time.}{}
\lilypondfile{lilypond/261-come-shepherds-deck-your-heads}\normalsize

\settowidth{\versewidth}{The Satyrs strove to have her;}
\begin{dcverse}
\begin{altverse}
All ye forsaken wooers,\\
That ever care oppressed,\\
And all you lusty dooers,\\
That ever love distressed.\\
That losses can condole,\\
And altogether summon;\\
Oh! mourn for the poor soul\\
Of my plain-dealing woman.
\end{altverse}

\begin{altverse}
Fair Venus made her chaste,\\
And Ceres beauty gave her;\\
Pan wept when she was lost,\\
The Satyrs strove to have her;\\
Yet seem’d she to their view\\
So coy, so nice, that no man\\
Could judge, but he that knew\\
My own plain-dealing woman.
\end{altverse}

\begin{altverse}
At all her pretty parts\\
I ne’er enough can wonder;\\
She overcame all hearts,\\
Yet she all hearts came under;\\
Her inward mind was sweet,\\
Good tempers ever common;\\
Shepherd shall never meet\\
So plain a dealing woman.
\end{altverse}
\end{dcverse}
\pagebreak
%%262
%%================================================
%
\musictitle{There Was An Old Fellow At Waltham Cross.}

This is quoted as an old song in Brome’s play, \textit{The Jovial Crew}, which was
acted at the Cock-pit in Drury Lane, in 1641—“T’other old song for that.”
It is also in the \textit{Antidote to Melancholy}, 1661.

\textit{The Jovial Crew} was turned into a ballad-opera in 1731, and this song
retained. The tune was then printed under the name of \textit{Taunton Dean};
perhaps from a song commencing, “In Taunton Dean I was born and bred.”

The four last bars of the air are the prototype of \textit{Lilliburlero}, and still often
sung to the chorus,—

\begin{scverse}
“A very good song, and very well sung;\\
Jolly companions every one.”
\end{scverse}

The first part resembles \textit{Dargason} (see p.~65), and an air of later date, called
\textit{Country Courtship} (see Index).

\musicinfo{Boldly and in moderate time.}{}
\lilypondfile{lilypond/262-there-was-an-old-fellow-at-waltham-cross}\normalsize

\musictitle{Old Sir Simon The King.}

This tune is contained in Playford’s \textit{Musick’s Recreation on the Lyra Viol},
1652; in \textit{Musick’s Handmaid for the Virginals}, 1678; in \textit{Apollo’s Banquet for
the Treble Violin}; in \textit{The Division Violin}, 1685; in \textit{180 Loyal Songs}, 1684
and 1694; and in the seventh and all later editions of \textit{The Dancing Master}.

It it also in\textit{\textit{ Pills to purge Melancholy}}; in the \textit{Musical Miscellany}, 1721; in
many ballad-operas, and other works of later date.
\pagebreak
%%263
%%================================================

%\setlength{\columnsep}{1em}
%\twocolumnfootnotes
%\setlength{\columnsep}{1em}
Some of the ballads written to the tune have the following burden, which
appears to be the original:—
\indentpattern{3303}
\settowidth{\versewidth}{With his ale-dropt hose, and his malmsey nose,}
\begin{scverse}
\begin{patverse}
\vin\vin\vin \vleftofline{“}Says old Simon the king,\\
Says old Simon the king,\\
With his ale-dropt hose, and his malmsey nose,\\
Sing, hey ding, ding a ding, ding.”
\end{patverse}
\end{scverse}

From its last line, Ritson conjectured that the “Hey ding a ding” mentioned
in Laneham’s \textit{Letter from Kenilworth}, 1575, as one of the ballads “all ancient,”
then in the possession of Captain Cox, the Coventry mason, was \textit{Old Sir Simon}
under another name. So far as internal evidence can weigh, the tune may be of
even much greater antiquity, but we have no direct proof.

Mr. Payne Collier is of opinion that the ballad entitled \textit{Ragged and torn and
true}, was “first published while Elizabeth was still on the throne.” (See Collier’s
\textit{Roxburghe Ballads}, p.~26.) As it was sung to the tune of \textit{Old Simon the King},
the latter necessarily preceded it. This adds to the probability of Ritson’s conjecture. 
But, although we have ballads printed during the reign of James I., to
the tune of \textit{Old Simon}, I have not succeeded in discovering one of earlier date.

Sir John Hawkins, in the additional notes to his \textit{History of Music}, says, “It is
conjectured that the subject of the song was Simon Wadloe, who kept the Devil
(and St. Dunstan) Tavern, at the time when Ben Jonson’s Club, called the
Apollo Club,
\dcfootnote{
For the excellent rules of this Club, see Note, p.~250.
} %end footnote
met there.” The conjecture rests upon two lines of the inscription
over the door of the Apollo room—
\settowidth{\versewidth}{Cries Old Sym, the King of Skinkers.”}
\begin{scverse}
\vleftofline{“}Hang up all the poor hop-drinkers,\\
Cries Old Sym, the King of Skinkers.”
\end{scverse}
A skinker meaning one who serves drink. Sir John quotes the song in \textit{Pills to
purge Melancholy}, iii. 144. It has but one line of burden,—
\settowidth{\versewidth}{“Says old Simon the King; ”}
\begin{scverse}
“Says old Simon the King;”
\end{scverse}
and instead of the Devil tavern, the Crown is the tavern named in it. It appears
to be of too late a date for the original song. The Simon Wadloe
\dcfootnote{
A Latin “Epitaph upon Simon Wadloe, vintner,
dwelling at the Signe of the Devil and St. Dunstan,” will
he found in MS. Ashmole, No. 38 fol., art. 328; and in
Camden’s \textit{Remains}. It commences thus:—
%\settowidth{\versewidth}{“Apollo et cohors Musarum}
%\begin{verse}

\vin \vleftofline{“}Apollo et cohors Musarum\\
\vin\vin Bacchus vini et uvarum,” \&c.

%\end{verse}
} %end footnote
whom Ben
Jonson dubbed “King of Skinkers,” was buried in March, 1627,
\dcfootnote{
See Descriptive Catalogue of the Beaufoy Tokens, by
Jacob Henry Burn, 8vo.,1855. From the same book we learn
that \textit{John} Wadlow was proprietor of the Devil Tavern at
the Restoration. He is mentioned twice in Pepys’ Diary
(22nd April, 1661, and 25th Feb., 1664-5). The second
time as having made a fortune—gone to live like a prince
in the country,—there spent almost all he had got, and
finally returned to his old trade again.
}
and more
probably owed his title to having the same Christian name as the Simon of the
earlier song.

As there are two times, which differ considerably, it seems desirable, in the
case of a song once so popular, to print both. The first is from \textit{Musick’s
Recreation on the Lyra Viol}, 1652; and the viol was tuned to what was
termed the “bagpipe tuning,” to play it. To this I have adapted the song quoted
by Hawkins, but completing the burden as the music requires. I have no doubt
that “Old Simon the King” was changed to “Old \textit{Sir} Simon the King,” from
the want of another syllable to correspond with accent of the tune.

\pagebreak
%%264
%%================================================
\musicinfo{Cheerfully.}{First Tune.}
\lilypondfile{lilypond/264-old-sir-simon-the-king}\normalsize

\settowidth{\versewidth}{\vin He may hang up himself for shame,}
\indentpattern{0101010101013}
\begin{dcverse}\begin{patverse}
Considering in my mind,\\
I thus began to think:\\
If a man be full to the throat,\\
And cannot take off his drink,\\
If his drink will not go down,\\
He may hang up himself for shame,\\
So the tapster at the Crown;\\
Whereupon this reason I frame:\\
Drink will make a man drunk,\\
Drunk will make a man dry,\\
Dry will make a man sick,\\
And sick will make a man die,\\
Says Old Simon the King.
\end{patverse}

\begin{patverse}
If a man should he drunk to-night,\\
And laid in his grave to-morrow,\\
Will you or any man say\\
That he died of care or sorrow?\\
Then hang up all sorrow and care,\\
’Tis able to kill a cat,\\
\columnbreak
And he that will drink all right,\\
Is never afraid of that;\\
For drinking will make a man quaff,\\
And quaffing will make a man sing,\\
And singing will make a man laugh,\\
And laughing long life doth bring,\\
Says Old Simon the King.
\end{patverse}

\begin{patverse}
If a Puritan skinker do cry,\\
Dear brother, it is a sin\\
To drink unless you be dry,\\
Then straight this tale I begin:\\
A Puritan left his can,\\
And took him to his jug,\\
And there he played the man\\
As long as he could tug;\\
And when that he was spied,\\
Did ever he swear or rail?\\
No, truly, dear brother, he cried,\\
Indeed all flesh is frail.\\
Says Old Simon the King.
\end{patverse}
\end{dcverse}
\pagebreak
%%265
%%================================================


The above song dates before the Restoration, because there is a political parody
upon it among the King’s Pamphlets, Brit. Mus., dated January 19th, 1659,
commencing thus:—
\settowidth{\versewidth}{Not a man stands up for the Rump,” \&c.}
\begin{scverse}
\begin{altverse}
\vleftofline{“}In a humour of late I was\\
Ycleped a doleful dump;\\
Thought I, we’re at a fine pass,\\
Not a man stands up for the Rump,” \&c.
\end{altverse}
\end{scverse}
I suppose it to have been written only a short time before the return of Charles,
and that this \textit{Old Simon the King} is the same person alluded to in one of the
Catches in the \textit{Antidote to Melancholy}, 4to, 1661, beginning—

\settowidth{\versewidth}{Sure the heat of the toast your nose did so roast}
\begin{scverse}
\begin{altverse}
\vleftofline{“}Good Symon, how comes it your nose is so red,\\
And your cheeks and your lips look so pale?\\
Sure the heat of the toast your nose did so roast\\
When they were both soused in ale,” \&c.
\end{altverse}
\end{scverse}
And perhaps also in “An Epitaph on an honest citizen and true friend to all
claret drinkers,” contained in part ii. of Playford’s \textit{Pleasant Musical Companion},
4to,~1687—
\begin{scverse}
\vleftofline{“}Here lyeth Simon, cold as clay,\\
Who whilst he liv’d cried Tip away,” \&c.
\end{scverse}
At the end of this epitaph it is said—
\begin{scverse}
\vleftofline{“}Now although this same epitaph was long since given,\\
Yet Simon’s not dead more than any man living.”
\end{scverse}
He was, perhaps, an old man whose death had been long expected.

The tune was in great favour at, and after, the Restoration. Many of the
songs of the Cavaliers were sung to it; many by Martin Parker, and other
ballad-writers of the reigns of James and Charles; several by Wilmott, Earl of
Rochester; and others of still later date.

Penkethman, the actor, wrote a comedy called \textit{Love without Interest, or
The Man too hard for the Master} (1699), in which one of the characters says
satirically, “Who? he! why the newest song he has is \textit{The Children in the Wood},
or \textit{The London Prentice}, or some such like ditty, set to the \textit{new} modish tune of
\textit{Old Sir Simon the King}.”

The name of the tune, \textit{Old Simon the King}, is printed in much larger letters
than any other of the collection, on the title-page of “A Choice Collection of
Lessons, being excellently sett to the Harpsichord, by the two great masters,
Dr.~John Blow, and the late Mr. Henry Purcell,” printed by Henry Playford in
1705: it was evidently thought to be the great attraction to purchasers.

Fielding, in his novel of \textit{Tom Jones}, makes it Squire Western’s favorite tune.
He tells us, “It was Mr. Western’s custom every afternoon, as soon as he was
drunk, to hear his daughter play upon the harpsichord.\ldots\  He never relished
any music but what was light and airy; and, indeed, his most favorite tunes were
\textit{Old Sir Simon the King}, \textit{St. George he was for England}, and some others\dots The
Squire declared, if she would give him t’other bout of \textit{Old Sir Simon}, he would
give the gamekeeper his deputation the next morning. \textit{Sir Simon} was played
again and again, till the charms of music soothed Mr. Western to sleep.”—i. 169.

It was the tune rather than the words, that gave it so lengthened a popularity.
I have found the air commonly quoted under \pagebreak five other names; viz., as \textit{Ragged} 
%%266
%%================================================
\textit{and torn, and true}; as \textit{The Golden Age}; as \textit{I’ll ne’er be drunk again}; as \textit{When
this old cap was new}; and as \textit{Round about our coal-fire}. The first is from the
ballad called “Ragged and torn, and true; or The Poor Man’s Resolution: to
the tune of \textit{Old Simon the King}.” See Roxburghe Collection, i. 352; or Payne
Collier’s Roxburghe Ballads, p.~26.

The second from “The Newmarket Song, to the tune of \textit{Old Simon the King};”
and beginning with the line, “The Golden Age is come.” See \textit{180 Loyal Songs},
4th edition, 1694, p.~152.

The third from a song called “The Reformed Drinker;” the burden of which
is, “And ne’er be drunk again.” See \textit{ Pills to purge Melancholy}, ii. 47, 1707, or
iv. 47, 1719; also Ritson’s \textit{English Songs}, ii. 59, 1813.

The fourth from one entitled “Time’s Alteration:
\settowidth{\versewidth}{The old man’s rehearsal what brave things he knew,}
\begin{scverse}
\vleftofline{“}The old man’s rehearsal what brave things he knew,\\
A great while agone, when this old cap was new;
\end{scverse}
to the tune of \textit{Ile nere be drunke againe}.” Pepy’s Collection, i. 160; or Evans’
\textit{Old Ballads}, iii. 262. (The name of the tune omitted, as usual, by Evans.)

The fifth is the name commonly given to it in collections of country dances,
printed during the last century.

One of the best political songs to the tune is “The Sale of Rebellion’s
Household Stuff;” a triumph over the downfall of the Rump Parliament,
beginning—
\settowidth{\versewidth}{Which was warm and pleasant to sit in?” \&c.}
\begin{scverse}
\begin{altverse}
\vleftofline{“}Rebellion hath broken up house,\\
And hath left me old lumber to sell;\\
Come hither and take your choice,\\
I’ll promise to use you well.\\
Will you buy the old Speaker’s chair,\\
Which was warm and pleasant to sit in?” \&c.
\end{altverse}
\end{scverse}

This song has the old burden at full length. The auctioneer, finding no purchasers,
offers, at the end, to sell the whole “for an old song.” It has been reprinted
in Percy’s \textit{Reliques of Ancient Poetry}.

I have seen a song beginning—
\settowidth{\versewidth}{And young Sir Simon the Squire,”}
\begin{scverse}
\vleftofline{“}To old Sir Simon the King,\\
And young Sir Simon the Squire,”
\end{scverse}
but have mislaid the reference. The tune is called “\textit{To} old Sir Simon the King,”
in the first edition of the \textit{Beggars’ Opera}, 1728.

In the Roxburghe Collection, i. 468, one of the ballads by Martin Parker, to
the tune of \textit{Ragged and torn, and true}, is entitled “Well met, Neighbour, or—
\settowidth{\versewidth}{A dainty discourse, between Nell and Sis,}
\begin{scverse}
\vleftofline{“}A dainty discourse, between Nell and Sis,\\
Of men that do use their wives amiss.”
\end{scverse}
This might be revived with some benefit to the lower classes at the present day,
especially if the last line of the burden could be impressed upon them—
\settowidth{\versewidth}{\textit{Oh! such a rogue should be hang’d.}}
\begin{scverse}
\begin{altverse}
\vleftofline{“}Heard you not lately of Hugh,\\
How soundly his wife he bang’d?\\
He beat her all black and blue:\\
\textit{Oh! such a rogue should be hang’d.}”
\end{altverse}
\end{scverse}
\pagebreak
%%267
%%================================================
Farquhar’s song in the \textit{Beaux's Stratagem}, beginning—
\settowidth{\versewidth}{A trifling song you shall hear,}
\begin{scverse}
\begin{altverse}
\vleftofline{“}A trifling song you shall hear,\\
Begun with a trifle and ended;\\
All trifling people draw near,\\
And I shall be nobly attended,”
\end{altverse}
\end{scverse}
was written to this tune, and is printed to it in \textit{The Musical Companion, or Lady's
Magazine}, 8vo., 1741.

“The praise of St. David’s day: shewing the reason why the Welshmen honour
the leek on that day.” Beginning—
\settowidth{\versewidth}{Who list to read the deeds}
\begin{scverse}
\vleftofline{“}Who list to read the deeds\\
By valiant Welshmen done,” \&c.,
\end{scverse}
is also to the tune, under the name of \textit{When this old cap was new}.

The following is the ballad of “Ragged and torn, and true; or The Poor Man’s
Resolution,” set to the tune as it is found in \textit{The Dancing Master}, and other
violin copies, but omitting the variations.

\musicinfo{Cheerfully.}{Second Tune.}
\lilypondfile{lilypond/267-old-sir-simon-the-king-second-tune}\normalsize

\pagebreak
%%268
%%================================================

\settowidth{\versewidth}{Though my doublet be rent i’ th’ sleeves,}
\begin{dcverse}\begin{altverse}
I scorn to live by the shift,\\
Or by any sinister dealing;\\
I’ll flatter no man for a gift,\\
Nor will I get money by stealing.\\
I’ll be no knight of the post,\\
To sell my soul for a bribe;\\
Though all my fortunes be cross’d,\\
Yet I scorn the cheater’s tribe.\\
Then hang up sorrow and care,\\
It never shall make me rue;\\
What though my cloak be threadbare,\\
\textit{I'm ragged, and torn, and true}.
\end{altverse}

\begin{altverse}
A boot of Spanish leather\\
I’ve seen set fast in the stocks,\\
Exposed to wind and weather,\\
And foul reproach and mocks;\\
While I, in my poor rags,\\
Can pass at liberty still—\\
O, fie on these brawling brags,\\
When money is gotten so ill!\\
O, fie on these pilfering knaves!\\
I scorn to be of that crew;\\
They steal to make themselves brave—\\
\textit{I'm ragged, and torn, and true}.
\end{altverse}

\begin{altverse}
I’ve seen a gallant go by\\
With all his wealth on his back;\\
He looked as loftily\\
As one that did nothing lack.\\
And yet he hath no means\\
But what he gets by the sword,\\
Which he consumes on queans,\\
For it thrives not, take my word.\\
O, fie on these highway thieves!\\
The gallows will be their due—\\
Though my doublet be rent i’ th’ sleeves,\\
\textit{I’m raggedy and torn, and true}.
\end{altverse}

\begin{altverse}
Some do themselves maintain\\
With playing at cards and dice;\\
O, fie on that lawless gain,\\
Got by such wicked vice!\\
They cozen poor country-men\\
With their delusions vilde; [vile]\\
Yet it happens now and then\\
That they are themselves beguil’d;\\
For, if they be caught in a snare,\\
The pillory claims its due;—\\
Though my jerkin be worn and bare,\\
\textit{I'm ragged, and torn, and true}.
\end{altverse}

\begin{altverse}
I have seen some gallants brave\\
Up Holborn ride in a cart,\\
Which sight much sorrow gave\\
To every tender heart;\\
Then have I said to myself\\
What pity is it, for this,\\
That any man for pelf\\
Should do such a foul amiss.\\
O, fie on deceit and theft!\\
It makes men at the last rue;\\
Though I have but little left,\\
\textit{I’m raggedy and torn, and true}.
\end{altverse}

\begin{altverse}
The pick-pockets in a throng,\\
Either at market or fair,\\
Will try whose purse is strong,\\
That they may the money share;\\
But if they are caught i’ th’ action,\\
They’re carried away in disgrace,\\
Either to the House of Correction,\\
Or else to a worser place.\\
O, fie on these pilfering thieves?\\
The gallows will be their due;\\
What need I sue for reprieves?\\
\textit{I’m raggedy and torn, and true}.
\end{altverse}

\begin{altverse}
The ostler to maintain\\
Himself with money in’s purse,\\
Approves the proverb true,\\
And says, Grammercy horse;\\
He robs the travelling beast,\\
That cannot divulge his ill,\\
He steal's a whole handful, at least,\\
From every half-peck he should fill.\\
O, fie on these cozening scabs,\\
That rob the poor jades of their due!\\
I scorn all thieves and drabs—\\
\textit{I’m ragged, and torn, and true}.
\end{altverse}

\begin{altverse}
’Tis good to be honest and just,\\
Though a man be never so poor;\\
False dealers are still in mistrust,\\
They’re afraid of the officer’s door:\\
Their conscience doth them accuse,\\
And they quake at the noise of a bush;\\
While he that doth no man abuse,\\
For the law needs not care a rush.\\
Then well fare the men that can say,\\
I pay every man his due;\\
Although I go poor in array,\\
\textit{I’m ragged, and torn, and true}.
\end{altverse}

\end{dcverse}
\pagebreak
%%269
%%================================================

%
The following is the before-mentioned song, “The Reformed Drinker, or I’ll
ne’er be drunk again,” also to the tune of \textit{Old Sir Simon the King}.
\settowidth{\versewidth}{’Twill cherish and comfort the blood}
\begin{dcverse}\footnotesize\begin{altverse}
Come, my hearts of gold,\\
Let us be merry and wise;\\
It is a proverb of old,\\
Suspicion hath double eyes.\\
Whatever we say or do,\\
Let’s not drink to disturb the brain;\\
Let’s laugh for an hour or two,\\
And ne’er be drunk again.
\end{altverse}

\begin{altverse}
A cup of old sack is good\\
To drive the cold winter away;\\
’Twill cherish and comfort the blood\\
Most when a man’s spirits decay:\\
But he that drinks too much,\\
Of his head he will complain;\\
Then let’s have a gentle touch,\\
And ne’er be drunk again.
\end{altverse}

\begin{altverse}
Good claret was made for man,\\
But man was not made for it;\\
Let’s be merry as we can,\\
So we drink not away our wit;\\
Good fellowship is abus'd,\\
And wine will infect the brain:\\
But we’ll have it better us’d,\\
And ne’er be drunk again.
\end{altverse}

\begin{altverse}
When with good fellows we meet,\\
A quart among three or four,\\
’Twill make us stand on our feet,\\
While others lie drunk on the floor.\\
Then, drawer, go fill us a quart,\\
And let it be claret in grain;\\
’Twill cherish and comfort the heart—\\
But we’ll ne’er be drunk again.
\end{altverse}

\begin{altverse}
Here’s a health to our noble King,\\
And to the Queen of his heart;\\
Let’s laugh and merrily sing,\\
And he’s a coward that will start.\\
Here’s a health to our general,\\
And to those that were in Spain;\\
And to our colonel—\\
And we’ll ne’er be drunk again.
\end{altverse}

\begin{altverse}
Enough’s as good as a feast,\\
If a man did but measure know;.\\
A drunkard’s worse than a beast,\\
For he’ll drink till he cannot go.\\
If a man could time recall,\\
In a tavern that’s spent in vain,\\
We’d learn to be sober all,\\
And we’d ne’er be drunk again.
\end{altverse}
\end{dcverse}

\musictitle{The Beggar Boy.}

This tune is contained in \textit{The Dancing Master}, from 1650 to 1690.

The following ballads were sung to it:—

Roxburghe Collection, i. 528—“Trial brings Truth to light; or—
\settowidth{\versewidth}{A dainty new ditty of many things treating:}
\begin{scverse}The proof a pudding is all in the eating;\\
A dainty new ditty of many things treating:
\end{scverse}
to the tune of \textit{The Beggar Boy}” by Martin Parker; and beginning—
\settowidth{\versewidth}{Her tricks and devices he’s wise that well knows—}
\begin{scverse}\begin{altverse}
\vleftofline{“}The world hath allurements and flattering shows,\\
To purchase her lovers’ good estimation;\\
Her tricks and devices he’s wise that well knows—\\
The learn’d in this science are taught by probation,” \&c.
\end{altverse}
\end{scverse}
The burden is, “The proof of the pudding is all in the eating.”

In the Roxburghe Collection, i. 542—“The Beggar Boy of the North—
\settowidth{\versewidth}{Whose lineage and calling to the world is proclaim’d,}
\begin{scverse}Whose lineage and calling to the world is proclaim’d,\\
Which is to be sung to the tune so nam’d;”
\end{scverse}
beginning—
\begin{scverse}“From ancient pedigree, by due descent,\\
I well can derive my generation,” \&c.;
\end{scverse}
and the burden, “And cry, Good, your worship, bestow one token.”

In the Roxburghe, i. 450, and Pepys, i. 306—“The witty Western Lasse,” \&c.,
“to a new tune called \textit{The Beggar Boy}:” subscribed Robert Guy. This begins,
“Sweet Lucina, lend me thy ayde;” \pagebreak and in the Pepys Collection, i. 310, there is 
%%270
%%================================================
a ballad to the tune of \textit{Lucina}, entitled “A most pleasant Dialogue, or a merry
greeting hetween two Lovers,” \&c.; beginning, “Good morrow, fair Nancie,
whither so fast;” which I suppose to be also to the tune. It is subscribed C.R.
Printed at London for H[enry G[osson.]

The following is also from the Roxburghe Collection (i. 462), and is reprinted
in Collier’s \textit{Roxburghe Ballads}, p.~7.

\musicinfo{Slow \& very smoothly.}{}
\lilypondfile{lilypond/270-the-beggar-boy}\normalsize

\musictitle{The Boatman.}

This is a bagpipe tune, and might be harmonized with a drone base. In
\textit{Musick’s Recreation on the Viol, Lyra-way}, 1661, the viol is strung to the, “bagpipe
tuning,” to play it. It is to be found in every edition of \textit{The Dancing Master},
from the first to that of 1698. I have not discovered the song of \textit{The Boatman},
but have adapted a stanza from Coryat’s \textit{Crambe}, 1611, to the air. It resembles
\textit{Trip and go} (see p.~131), and the same words might be sung to it. The accent
of the tune seems intended to imitate the turning of the scull in boating.

In the Roxburghe Collection, ii. 496, \pagebreak is a ballad entitled “The wanton wife of
%%271
%%================================================
Castle-gate, or The Boatman’s Delight: to its own proper new tune;” but it
appears, from the following, which is the first stanza, that this air cannot have
been intended.
\settowidth{\versewidth}{For its neither grief nor sorrow}
\begin{dcverse}
\begin{altverse}
\vleftofline{“}Farewell both hawk and hind,\\
Farewell both shaft and bow,\\
Farewell all merry pastimes,\\
And pleasures in a row:
\end{altverse}

\begin{altverse}
Farewell, my best beloved.\\
In whom I put my trust;\\
For its neither grief nor sorrow\\
Shall harbour in my breast.”
\end{altverse}
\end{dcverse}

There is an air in Thomson’s \textit{Orpheus Caledonius} called \textit{The Boatman}, but wholly
different from this.

\musicinfo{In rowing time.}{}
\lilypondfile{lilypond/271-the-boatman}\normalsize


\musictitle{Sir Launcelot Du Lake.}

This second tune to the ballad, “When Arthur first in court began” (which
the black-letter copies, \textit{The Garland of Good-will}, \&c., direct to be sung to the
tune of \textit{Flying Fame}—see p.~199), was transcribed by Dr.~Rimbault, from the fly-leaf
of a rare book of Lessons for the Virginals, entitled \textit{Parthenia Inviolata}.

The ballad is quoted by Shakespeare, by Beaumout and Fletcher, by Marston,
\&c. It is founded on the romance of \textit{Sir Launcelot du Lake}, than which none
was more popular. Chaucer, in “The Nonne Prest his tale,” says—
\settowidth{\versewidth}{As the book is of Launcelot the Lake; ’’}
\begin{scverse}
\vleftofline{‘}This story is al so trewe, I undertake,\\
As the book is of Launcelot the Lake;’’
\end{scverse}
\pagebreak
%%272
%%================================================
\noindent and Churchard, in his “Replication to Camel’s objection,” says to him—
%\small %begin cramped page 272
\settowidth{\versewidth}{The most of your study hath been of Robyn Hood}
\begin{scverse}\vleftofline{“}The most of your study hath been of Robyn Hood:\\
And Bevis of Hampton and Syr Launcelet de Lake\\
Hath taught you, full oft, your verses to make.”
\end{scverse}
The ballad, entitled “The noble acts of Arthur of the Round Table, and of Sir
Launcelot du Lake,” is printed in Percy’s \textit{Reliques of Ancient Poetry}.

\musicinfo{Boldly and slow.}{}
\lilypondfile[staffsize=14]{lilypond/272-sir-launcelot-du-lake}\normalsize

%\vspace{-1.5\baselineskip}

\musictitle{The Spanish Gipsy.}

%\vspace{-0.5\baselineskip}

This is in every edition of \textit{The Dancing Master}, and in \textit{Musick’s Delight on
the Cithren},~1666.

It is found in the ballad-operas, such as \textit{The Bays’ Opera}, 1730, and \textit{The
Fashionable Lady}, 1730, under the name of \textit{Come, follow, follow me}.

The name of \textit{The Spanish Gipsy} is probably derived from a gipsies’ song in
Rowley and Middleton’s play of that name. It begins, “Come, follow your
leader, follow,” and the metre is suitable to the air.

In the Roxburghe Collection, i. 544, is a black-letter ballad, entitled “The
brave English Jipsie: to the tune of \textit{The Spanish Jipsie}. Printed for John
Trundle,” \&c. It consists of eighteen stanzas, and commencing—
\settowidth{\versewidth}{’Tis English Jipsies’ call.”}
\begin{scverse}\vleftofline{“}Come, follow, follow all,\\
’Tis English Jipsies’ call.”
\end{scverse}
And in the same volume, p.~408, one by M[artin] P[arker], called “The three
merry Cobblers,” of which the following are the first, eighth, fourteenth, and last
stanzas. (Printed at London for F. Grove.)
\settowidth{\versewidth}{Our trade excels most trades i’the land,}
\indentpattern{111100}
\begin{dcverse}\begin{patverse}
\vin Come, follow, follow me,\\
To the alehouse we’ll march all three.\\
Leave awl, last, thread, and leather,\\
And let’s go all together.\\
Our trade excels most trades i’the land,\\
For we are still on the mending hand.
\end{patverse}

\begin{patverse}
\vin Whatever we do intend,\\
We bring to a perfect \textit{end};\\
If any offence be past,\\
We make all well at \textit{last}.\\
We sit at work when others stand,\\
And still we are on the mending hand.
\end{patverse}

\begin{patverse}
\vin All day we merrily sing,\\
And customers to us do bring\\
Or unto us do send\\
Their boots and shoes to mend.\\
We have our money at first demand;\\
Thus still we are on the mending hand.
\end{patverse}

\begin{patverse}
\vin We pray for dirty weather,\\
And money to pay for leather,\\
Which if we have, and health,\\
A fig for worldly wealth.\\
Till men upon their heads do stand,\\
We still shall be on the mending hand.
\end{patverse}
\end{dcverse}
\pagebreak
%%273
%%================================================

The most popular Song to this tune was—
\settowidth{\versewidth}{Ye fairy elves that be,” \&c.}
\begin{scverse}
\vleftofline{“}Come, follow, follow me,\\
Ye fairy elves that be,” \&c.
\end{scverse}
It is the first in a tract entitled “A Description of the King and Queene of
Fayries, their habit, fare, abode, pompe, and state: being very delightful to the
sense, and full of mirth. London: printed for Richard Harper, and are to be
sold at his shop at the Hospitall Gate, 1635;” and the song was to be “sung
like to the \textit{Spanish~Gipsie}.”

The first stanza is here printed to the tune. The song will be found entire in
Percy’s \textit{Reliques of Ancient Poetry}, or Ritson’s \textit{English Songs}.

\musicinfo{Lightly, and in moderate time.}{}
\lilypondfile{lilypond/273-the-spanish-gipsy}\normalsize

\musictitle{The Friar In The Well.}

In Anthony Munday’s \textit{Downfall of Robert, Earl of Huntington} (written in
1597), where Little John expresses his doubts of the success of the play;
saying—
\settowidth{\versewidth}{No pleasant skippings up and down the wood;}
\begin{scverse}
\begin{altverse}
\vleftofline{“}Methinks I see no jests of Robin Hood;\\
No merry Morrices of Friar Tuck;\\
No pleasant skippings up and down the wood;\\
No hunting songs,” \&c.
\end{altverse}
\end{scverse}
The Friar answers, that “merry jests” have been shewn before, such as—
\begin{scverse}
\vleftofline{“}How the Friar fell into the well,\\
For love of Jenny, that fair, bonny belle,” \&c.
\end{scverse}
The title of this ballad is “The Fryer well fitted; or—
\begin{scverse}
\textit{A pretty jest} that once befell;\\
How a maid put a Fryer to cool in a well:
\end{scverse}
to a merry tune.”
\pagebreak
%%274
%%================================================

The tune is in \textit{The Dancing Master}, from 1650 to 1686, entitled \textit{The Maid
peept out at the window}, or \textit{The Frier in the Well}.

The ballad is in the Bagford Collection; in the Roxburghe (ii. 172); the
Pepys (iii. 145); the Douce (p.~85); and in \textit{Wit and Mirth, an Antidote to
Melancholy}, 8vo., 1682. Also, in an altered form, in\textit{ Pills to purge Melancholy},
1707, i. 340; or 1719, iii. 325. But not one of these contains the line, “The
maid peept out of the window.” I suppose, therefore, that the present has been
modelled upon some earlier version of the ballad, which I have not seen. The
story is a very old one, and one of the many against monks and friars, in which,
not only England but all Europe delighted.

\musicinfo{Gracefully.}{}
\lilypondfile{lilypond/274-the-friar-in-the-well}\normalsize


The story of the ballad may be told, with slight abbreviation. Firstly, the
Friar makes love to the Maid:---
\settowidth{\versewidth}{Tush, quoth the Friar, thou needst not doubt,}
\begin{scverse}
\vleftofline{“}But she denyed his desire,\\
And told him that she fear’d Hell-fire.\\
Tush, quoth the Friar, thou needst not doubt,\\
If thou wert in Hell, I could sing thee out.”
\end{scverse}
\pagebreak
%%275
%%================================================

The Maid pretends to be persuaded by his arguments, but stipulates that he shall
bring her an angel of money.

\begin{dcverse}\settowidth{\versewidth}{How she the Friar might beguile;}
\vleftofline{“}Tush, quoth the Friar, we shall agree,\\
No money shall part my love and me;\\
Before that I will see thee lack,\\
I’ll pawn my grey gown from my back.\\
The Maid bethought her of a wile,\\
How she the Friar might beguile;

While he was gone (the truth to tell),\\
She hung a cloth before the well.\\
The Friar came, as his covenant was,\\
With money to his bonny lass.\\
Good morrow, fair Maid, good morrow, quoth he,\\
Here is the money I promised thee.”
\end{dcverse}
The Maid thanks him, and takes the money, but immediately pretends that her
father is coming.

\begin{dcverse}\vleftofline{“}Alas! quoth the Friar, where shall I run,\\
To hide myself till he be gone?\\
Behind the cloth run thou, quoth she,\\
And there my father cannot thee see.\\
Behind the cloth the Friar crept,\\
And into the well on a sudden he leapt.\\
Alas! quoth he, I am in the well;\\
No matter, quoth she, if thou wert in Hell:\\
Thou sayst thou couldst sing me out of Hell,\\
Now, prythee, sing thyself out of the well.\\
The Friar sung on with a pitiful sound,\\
O help me out! or I shall he drown'd.\\
I trow, quoth she, your courage is cool’d;\\
Quoth the Friar, I never was so fool’d;\\
I never was served so before.\\
Then take heed, quoth she, thou com’st here no more;

Quoth he, for sweet St. Francis’ sake,\\
On his disciple some pity take;\\
Quoth she, St. Francis never taught\\
His scholars to tempt young maids to naught.\\
The Friar did entreat her still\\
That she would help him out of the well;\\
She heard him make such piteous moan,\\
She help’d him out, and bid him begone.\\
Quoth he, shall I have my money again,\\
Which from me thou hast before-hand ta’en?\\
Good sir, quoth she, there’s no such matter.\\
I’ll make you pay for fouling the water.\\
The Friar went all along the street,\\
Dropping wet, like a new-wash'd sheep;\\
Both old and young commended the Maid\\
That such a witty prank had play’d.”
\end{dcverse}

\musictitle{Sir Eglamore.}

This “merry tune” is another version of \textit{The Friar in the Well} (see the preceding). 
the ballad of \textit{Sir Eglamor}e is a satire upon the narratives of deeds
of chivalry in old romances. It is contained in \textit{The Melancholie Knight}, by
S[amuel] R[owlands], 1615; in the \textit{Antidote to Melancholy}, 1661; in \textit{Merry
Drollery Complete}, 1661; in Dryden’s \textit{Miscellany Poems}, iv. 104; in the Bagford
and Roxburghe Collections of Ballads; in Ritson’s \textit{Ancient Songs}; Evans’ \textit{Old
Ballads}; \&c., \&c.

It appears, with music, in part ii. of Playford’s \textit{Pleasant Musical Companion},
1687; in\textit{ Pills to purge Melancholy}; in Stafford Smith’s \textit{Musica Antiqua}; and the
tune, with other words, in \textit{180 Loyal Songs}, \&c.

The title of the ballad is, “Courage crowned with Conquest; or A brief relation
how that valiant Knight, and heroic Champion, Sir Eglamore, bravely fought
with and manfully slew a terrible, huge, great, monstrous Dragon. To a pleasant
new tune.” There are many variations in the copies from different presses.

The following songs were sung to \textit{Sir Eglamore}:—

“Sir Eglamore and the Dragon, or a relation how General Monk slew a most
cruel Dragon, Feb. 11, 1659.” \textit{See Loyal Songs written against the Rump
Parliament}.
\pagebreak
%%276
%%================================================

“Ignoramus Justice; or—
\settowidth{\versewidth}{To let knaves out and keep honest men in:}
\begin{scverse}
The English laws turn’d into a gin.\\
To let knaves out and keep honest men in:
\end{scverse}

to the tune of \textit{Sir Egledemore}.” London: printed for Allen Bancks, 1682.

“The Jacobite toss’d in a Blanket,” \&c. (Pepys Coll., ii. 292); beginning—
\settowidth{\versewidth}{You’ve brought yourselves and your cause to nought.}
\begin{scverse}
\vleftofline{“}I pray, Mr. Jacobite, tell me why, Fa la, \&c.,\\
You on our government look awry, Fa la, \&c.\\
With paltry hat, and threadbare coat,\\
And jaws as thin as a Harry groat.\\
You’ve brought yourselves and your cause to nought.\\
Fa-la, fa-la-la-la, Fa-la, lanky down dilly.”
\end{scverse}

In Rowland’s \textit{Melancholie Knight}, the ballad is thus prefaced:—
\begin{scverse}
“But that I turn, and overturn again.\\
Old books, wherein the worm-holes do remain;\\
Containing acts of ancient knights and squires\\
That fought with dragons, spitting forth wild fires.\\
The history unto you shall appear,\\
Even by myself, verbatim, set down here.”
\end{scverse}

\musicinfo{Gracefully.}{}
\lilypondfile{lilypond/276-sir-eglamore}\normalsize

\pagebreak
%%277
%%================================================

\settowidth{\versewidth}{The Knight did tremble, horse did quake;}
\begin{dcverse}\footnotesize A Dragon came out of his den,\\
Had slain, God knows how many men:\\
When he espied Sir Eglamore,\\
Oh! if you had but heard him roar!

Then the trees began to shake,\\
The Knight did tremble, horse did quake;\\
The birds betake them all to peeping,\\
It would have made you fall a weeping.

But now it is vain to fear,\\
For it must be fight dog, fight bear;\\
To it they go, and fiercely fight\\
A live-long day, from morn till night.

The Dragon had a plaguey hide,\\
And could the sharpest steel abide;\\
No sword would enter him with cuts,\\
Which vext the Knight unto the guts;

But, as in choler he did burn,\\
He watched the Dragon a good turn,\\
And as a yawning he did fall,\\
He thrust the sword in, hilt and all.

Then like a coward he [did] fly\\
Unto his den that was hard by,\\
And there he lay all night and roar’d:\\
The Knight was sorry for his sword;
\end{dcverse}
\settowidth{\versewidth}{He that will fetch it, let him take it.”}
\begin{scverse}\footnotesize But riding thence, said, I forsake it.\\
He that will fetch it, let him take it.”
\end{scverse}

Instead of the two last lines, in many of the copies, are the three following
stanzas:—
\vspace{-\baselineskip}
\settowidth{\versewidth}{When all was done, to the alehouse he went,}
\begin{dcverse}\footnotesize The sword, that was a right good blade,\\
As ever Turk or Spaniard made,\\
I, for my part, do forsake it,\\
And he that will fetch it, let him take it.

When all was done, to the alehouse he went,\\
And by and by his two-pence he spent;\\
\columnbreak
For he was so hot with tugging with the Dragon,\\
That nothing would quench him but a whole flagon,

Now God preserve our King and Queen,\\
And eke in London may he seen\\
As many knights, and as many more,\\
And all so good as Sir Eglamore.
\end{dcverse}

\musictitle{The Cobbler’s Jigg.}

This tune first appears in \textit{The Dancing Master}, in the seventh edition, printed
in 1686. It is there entitled \textit{The Cobbler’s Jigg}. More than sixty years before
it had been published in Holland, as an English song-tune, in \textit{Bellerophon}, 1622;
and in \textit{Nederlandtsche Gedenck-Clanck}, 1626. In the index to the latter, among
the “Engelsche Stemmen,” it is entitled “Cobbeler, of: Het Engelsch Lapperken.” 
All the English airs in these Dutch books have Dutch words adapted to
them; but as I do not know the English words which belong to this, I have
adapted an appropriate song from \textit{The Shoemaker's Holiday}, 1600.

In the Pepys Collection of Ballads, vol. i., No. 227, is one entitled “Round,
boyes, indeed! or \textit{The Shoomaker's Holy-day}:
\settowidth{\versewidth}{To fit both country, towne, and cittie, \&c.}
\begin{scverse}\footnotesize Being a very pleasant new ditty,\\
To fit both country, towne, and cittie, \&c.
\end{scverse}
To a pleasant new tune.” It is signed L.P. (Laurence Price?), and printed
for J.~Wright, who printed about 1620. This may prove to be the ballad.
I noted that it was in eighteen stanzas, but omitted to copy it.

Shoemakers called their trade “the gentle craft,” from a tradition that King
Edward IV., in one of his disguises, once drank with a party of shoemakers, and
pledged them. The story is alluded to in the old play, \textit{George a Greene, the
Pinner of Wakefield} (1599), when Jenkin says—
\settowidth{\versewidth}{Marry, because you have drank with the King,}
\begin{scverse}\footnotesize \vleftofline{“}Marry, because you have drank with the King,\\
And the King hath so graciously pledg’d you,\\
You shall no more be called shoemakers;\\
But you and yours, to the world’s end,\\
Shall be called the trade of the gentle graft.’’\\
\attribution Dodsley’s \textit{Old Plays}, v. iii., p.~45.
\end{scverse}

\pagebreak
%%278
%%================================================

“Would I had been created a shoemaker,” (says the servant in a play of Dekker’s)
“for all the \textit{gentle craft} are gentlemen every Monday by their copy, and scorn
then to work one true stitch.”—Dodsley’s \textit{Old Plays}, v. iii., p.~282.

Cobblers, too, were proverbially a merry set. In the opening scene of Ben
Jonson’s play, \textit{The case is altered}, Juniper, the cobbler, is discovered sitting at
work in his shop, and singing; and Onion, who is sent for him, has great difficulty
in stopping his song. When told that he must slip on his coat and go to
assist, because they lack waiters, he exclaims, “A pityful hearing! for now must I,
of a \textit{merry cobbler}, become a mourning creature.” (The family were in mourning).
Juniper is also represented as a small poet; and when, in the third act, Onion
goes to him again (the cobbler being in his shop, and singing, as usual), and
explains his distress because Valentine had not written the ditty he ordered of
him, Juniper says, “No matter, I’ll hammer out a ditty myself.”

\musicinfo{Jovially, and in moderate time.}{}
\lilypondfile{lilypond/278-the-cobblers-jigg}\normalsize

\settowidth{\versewidth}{Let’s sing a dirge for Saint Hugh’s soul.}
\indentpattern{01013434}
\begin{scverse}\begin{patverse}
Troll the bowl, the nut-brown bowl,\\
And here, kind mate, to thee!\\
Let’s sing a dirge for Saint Hugh’s soul.\\
And down it merrily.\\
Hey down a down, hey down a down,\\
Hey derry, derry, down a down;\\
Ho! well done, to me let come,\\
Ring compass, gentle joy.
\end{patverse}
\end{scverse}

\pagebreak	



%\changefontsize{1.03\defaultfontsize}
%\origpage{}
%279
%========================================

\musictitle{Down In The North Country.}

This tune was formerly very popular, and is to be found under a variety of
names, and in various shapes. In the second vol. of \textit{The Dancing Master} it is
entitled \textit{The Merry Milkmaids}. In \textit{The Merry Musician, or a Cure for the
Spleen}, i. 64, it is printed to the ballad, “The Farmer’s Daughter of merry
Wakefield.” That ballad begins with the line, “Down in the North Country;”
and the air is so entitled in the ballad-opera, \textit{A Cure for a Scold}, 1738. In
\textit{180 Loyal Songs}, third and fourth editions, 1684 and 1694, there are two songs,
and the tune is named \textit{Philander}. The first of the songs begins, “\textit{Ah, cruel
bloody fate},” and the second is “to the tune of \textit{Ah, cruel bloody fate};” by which
name it is also called in \textit{The Genteel Companion for the Recorder}, 1683, and
elsewhere.

One of M[artin] P[arker’s] ballads is entitled “Take time while ’tis offer’d;”
\settowidth{\versewidth}{For Tom has broke his word with his sweeting,}
\begin{scverse}
\vleftofline{“}For Tom has broke his word with his sweeting,\\
And lost a good wife for an hour’s meeting;\\
Another good fellow has gotten the lass,\\
And Tom may go shake his long ears like an ass.”
\end{scverse}
to the tune \textit{Within the North Country}”. (Roxburghe, i. 396.) It begins with
the line, “When Titan’s fiery steeds,” and the last stanza is—
\settowidth{\versewidth}{And I have raised my fortunes well—}
\begin{scverse}
\begin{altverse}
“Thus Tom hath lost his lass,\\
Because he broke his vow;\\
And I have raised my fortunes well—\\
\textit{The case is alter’d now}.’’
\end{altverse}
\end{scverse}

There are many ballads to the tune \textit{The case is altered}, and probably this is
intended.

In the Bagford Collection is “The True Lover’s lamentable Overthrow; or
The Damosel’s last Farewell,” \&c.: “to the tune of \textit{Cruel bloody fate};”
commencing—
\begin{scverse}
\vleftofline{“}You parents all attend\\
To what of late befell;\\
It is to you I send\\
These lines, my last farewell.” \&c.
\end{scverse}

In the Douce Collection, p.~245, “The West Country Lovers—
\begin{scverse}
\begin{altverse}
See here the pattern of true love,\\
Amongst the country blades,\\
Who never can delighted be,\\
But when amongst the maids:
\end{altverse}
\end{scverse}
tune of \textit{Philander}.”

The last is in black-letter, printed by J. Bonyers, at the Black Raven in Duck
Lane. A former possessor has written “Cruel bloody fate” under “Philander,”
as being the other name of the tune.

In the Roxburghe Collection, ii. 105,—“The Deceiver Deceived; or The
Virgin’s Revenge: to the tune of \textit{Ah, cruel bloody fate},” begins, “Ah, cruel maid,
give o’er.”

In \textit{A Cabinet of Choice Jewels}, 1688 (Wood’s Library, Oxford)—a “Carol for
Innocents’ Day: tune of \textit{Bloody fate}.”
\pagebreak


%280
%========================================

The song of \textit{Philander} is in \textit{ Pills to purge Melancholy}, ii. 252 (1707), or
iv. 284 (1719); in \textit{Wit and Drollery}, 1682; and a black-letter copy in the
Douce Collection, p.~74, entitled “The Faithfull Lover’s Downfall; or The
Death of fair Phillis, who killed herself for the loss of her Philander,” \&c.: to a
pleasant new play-house tune, or \textit{O cruel bloody fate}.” (Printed by T. Vere, at
the Angel in Giltspur Street.)

\musicinfo{Smoothly, and in moderate time.}{}
\lilypondfile{lilypond/280-o-cruel-bloody-fate}\normalsize

\settowidth{\versewidth}{Ah, I come! she cried, with a wound so wide, to need no second blow.}
\begin{scverse}Her poniard then she took, and graspt it in her hand,\\
And with a dying look, cried, Thus I fate command:\\
Philander, ah, my love! I come to meet thy shade below;\\
Ah, I come! she cried, with a wound so wide, to need no second blow.

In purple waves her blood ran streaming down the floor;\\
Unmov’d she saw the flood, and bless’d her dying hour:\\
Philander, ah Philander, still the bleeding Phillis cried;\\
She wept awhile, and forc’d a smile, then clos’d her eyes and died.
\end{scverse}

The following is the version called “Down in the North Country,” of which
there are also copies in Halliwell’s Collection (Cheetham Library, 1850), and in
Dr.~Burney’s Collection, Brit. Mus.
\pagebreak

%281
%========================================

\musicinfo{Cheerfully.}{}
\lilypondfile{lilypond/281-down-in-the-north-country}\normalsize


The following is the version of the same tune, which is entitled \textit{The Merry Milkmaids}
in the second volume of \textit{The Dancing Master}. It was formerly the custom
for milkmaids to dance before the houses of their customers in the month of May,
to obtain a small gratuity; and probably this tune, and \textit{The Merry Milkmaids in
green}, were especial favorites, and therefore named after them. To be a milkmaid
and to be merry were almost synonymous in the olden time. Sir Thomas
Overbury’s \textit{Character of a Milkmaid}, and some allusions to their songs, will be
found with the tune entitled \textit{The Merry Milkmaids in green}. The following
quotations relate to their music and dancing.
	    
In Beaumont and Fletcher’s play, \textit{The Coxcomb}, Nan, the milkmaid, says—
\settowidth{\versewidth}{And we serve a very good woman, and a gentlewoman;}
\begin{scverse}\vleftofline{“}Come, you shall e’en home with us, and be our fellow;\\
Our house is so honest!\\
And we serve a very good woman, and a gentlewoman;\\
And we live as merrily, and dance o’ good days\\
After even-song. Our wake shall be on Sunday:\\
Do you know what a wake is?—we have mighty cheer then,” \&c.
\end{scverse}

Pepys, in his Diary, 13th Oct., 1662, says, “With my father took a melancholy
walk to Portholme, seeing the country-maids, milking their cows there,
they being there now at grass; and to see with what mirth they come all home
together in pomp with their milk, and sometimes they have music go before them.”
\pagebreak
%282
%========================================
Again, on the 1st May, 1667, “To Westminster; on the way meeting many
milkmaids with their garlands upon their pails, dancing with a fiddler before
them; and saw pretty Nelly” [Nell Gwynne] “standing at her lodgings’ door in
Drury Lane, in her smock sleeves and bodice, looking upon one: she seemed a
mighty pretty~creature.”

In a set of prints, called \textit{Tempest's Cryes of London}, one is called “The Merry
Milkmaid, whose proper name was Kate Smith. She is dancing with her milkpail
on her head, decorated with silver cups, tankards, and salvers, borrowed for
the purpose, and tied together with ribbands, and ornamented with flowers. Of
late years, the plate, with other decorations, were placed in a pyramidical form,
and carried by two chairmen upon a wooden horse. The mikmaids walked before
it, and performed the dance without any incumbrance. Strutt mentions having
seen “these superfluous ornaments, with much more propriety, substituted by a
cow. The animal had her horns gilt, and was nearly covered with ribbands of
various colours, formed into bows and roses, and interspersed with green oaken
leaves and bunches of flowers.” \textit{Sports and Pastimes}, edited by Hone, p.~358.

\musicinfo{Lively.}{The Milkmaids’ Dance.}
\lilypondfile[staffsize=16]{lilypond/282-the-milkmaids-dance}\normalsize

\pagebreak

%283
%========================================

\musictitle{Morris Dance.}

This is entitled \textit{Engelsche Klocke-Dans} in three of the Collections published in
Holland: viz., in \textit{Bellerophon} (Amsterdam, 1622); \textit{Nederlandtsche GedenckClanck}
(Haerlem, 1626); and \textit{Friesche Lust-Hof} (Amsterdam, 1634.)

As “klok” signifies “bell,” and bells were worn in the morris, I suppose it to
have been a morris-dance. In the above-named collections, Dutch songs are
adapted to it, but I have no clue to the English words.

\musicinfo{Moderate time.}{}
\lilypondfile{lilypond/283-morris-dance}\normalsize

\musictitle{Amarillis Told Her Swain.}

This is found, under the name of \textit{Amarillis}, among the violin tunes in \textit{The
Dancing Master} of 1665, and in all later editions; in \textit{Musick’s Delight on the
Cithren}, 1666; in \textit{Apollo’s Banquet}, 1670; in the \textit{Pleasant Companion for the
Flageolet},~1680;~\&c.

The song, “Amarillis told her swain,” is in \textit{Merry Drollery complete}, 1670 (p.3).

The air is sometimes referred to as \pagebreak “Phillis on the new-made hay,” from a
%284
%========================================
ballad entitled “The coy Shepherdess; or Phillis and Amintas;” which was sung
to the tune of \textit{Amarillis}. See Roxburghe Collection, ii. 85.

Among the ballads to the air, are also the following:—

\settowidth{\versewidth}{Attend on human nature,” \&c.}
“The Royal Recreation of Jovial Anglers;” beginning—
\begin{scverse}
\vleftofline{“}Of all the recreations which\\
Attend on human nature,” \&c.\\
\attribution Roxburghe Collection.
\end{scverse}
Collier’s Roxburghe Ballads, p.~232; and \textit{Merry Drollery complete}, 1661 and
1670. It is also in \textit{ Pills to purge Melancholy}; but there set to the tune of
\textit{My father was born before me}.

“Love, in the blossom; or Fancy in the bud: to the tune \textit{Amarillis told her
swain}.” Roxburghe, ii. 315.

“Fancy’s Freedom; or true Lovers’ bliss: tune of \textit{Amarillis}, or \textit{Phillis on the
new-made hay}.” Roxburghe, iii. 114.

“The true Lovers’ Happiness; or Nothing venture, nothing have,” \&c.: tune
of \textit{Amintas on the new-made hay}; or \textit{The Loyal Lovers}.” Douce Collection, and
Roxburghe, ii. 486.

“The Cotsall Shepherds: to the tune of \textit{Amarillis told her swain},” in \textit{Folly in
print, or a Book of Rhymes}, 1667.

The following stanza, set to the tune, is the. first of the above-named ballad,
“The coy Shepherdess; or Phillis and Amintas:”—

\musicinfo{Smoothly, and in moderate time.}{}
\lilypondfile{lilypond/284-amarillis-told-her-swain}\normalsize

\pagebreak
%\changefontsize{0.99\defaultfontsize}

%285
%========================================

\musictitle{Cherrily And Merrily.}

In \textit{The Dancing Master} of 1652, this is entitled \textit{Mr. Webb’s Fancy}; and in
later editions \textit{Cherrily and merrily}.

In vol. xi. of the King’s Pamphlets, folio, there is a copy of a ballad written on
the violent dissolution of the Long Parliament by Cromwell, entitled “The Parliament
routed; or Here’s a house to be let:
\settowidth{\versewidth}{Shall be at peace, and give no way to warres:}
\begin{scverse}I hope that England, after many jarres,\\
Shall be at peace, and give no way to warres:\\
O Lord, protect the generall, that he\\
May be the agent of our unitie:
\end{scverse}
to the tune of \textit{Lucina}, or \textit{Merrily and cherrily}.” [June 3, 1653.] It has been
reprinted in \textit{Political Ballads}, Percy Society, No. 11, p.~126. The first stanza
is as follows:—
\settowidth{\versewidth}{Hot spirits are quenched, the tempest is layd,}
\begin{scverse}\begin{altverse}
\vleftofline{“}Cheer up, kind countrymen, be not dismay’d,\\
True news I can tell ye concerning the nation:\\
Hot spirits are quenched, the tempest is layd,\\
And now we may hope for a good reformation.”
\end{altverse}
\end{scverse}

The above is more suited to the tune of \textit{Lucina} (\ie, \textit{The Beggar Boy}, p.~269)
than to this air; I have therefore adapted a song from \textit{Universal Harmony}, 1746,
an alteration of the celebrated one by George Herbert.

\musicinfo{Smoothly, and in moderate time.}{}
\lilypondfile{lilypond/285-cherrily-and-merrily}\normalsize

\settowidth{\versewidth}{Storehouse where sweets unnumher’d lie,}
\begin{dcverse}\begin{altverse}
Sweet rose, so fragrant and so brave,\\
Dazzling the rash beholder’s eye,\\
Thy root is ever in its grave,\\
And thou, with all thy sweets, must die.
\end{altverse}

\begin{altverse}
Sweet Spring, so beauteous and so gay,\\
Storehouse where sweets unnumber’d lie,\\
Not long thy fading glories stay,\\
But thou, with all thy sweets, must die.
\end{altverse}

\begin{altverse}
Sweet love, alone, sweet wedded love,\\
To thee no period is assign’d;\\
Thy tender joys by time improve,\\
In death itself the most refin’d.
\end{altverse}
\end{dcverse}
\pagebreak

%286
%========================================

\musictitle{St. George For England.}

There are black-letter copies of this ballad in the Pepys and Bagford Collections. 
It is also in An \textit{Antidote to Melancholy}, 1661; in part ii. of \textit{Merry
Drollery Complete}, 1661 and 1670; in \textit{Wit and Drollery}, 1682; \textit{Pills to purge
Melancholy}, 1707 and 1719; \&c.

It is one of those offered for sale by the ballad-singer in Ben Jonson’s
comedy of \textit{Bartholomew Fair}.

Pepys, in his Diary, tells us of “reading a ridiculous ballad, made in praise of
the Duke of Albemarle, to the tune of \textit{St. George}—the tune being printed too;”
and adds, “I observe that people have great encouragement to make ballads of
him, of this kind. There are so many, that hereafter he will sound like Guy of
Warwick.” (6th March, 1667.)

Fielding, in his novel of \textit{Tom Jones}, speaks of \textit{St. George he was for England}
as one of Squire Western’s favorite tunes.

The ballad in the Pepys Collection (i. 87) is entitled “Saint George’s Commendation
to all Souldiers; or Saint George’s Alarum to all that profess martiall
discipline, with a memoriall of the Worthies who have been borne so high on the
wings of Fame for their brave adventures, as they cannot be buried in the pit of
oblivion: to \textit{a pleasant new tune}.” It was “imprinted at London, by W. W.,” in
1612, and is the copy from which Percy printed, in his \textit{Reliques of Ancient
Poetry}. It begins—“Why do we boast of Arthur and his Knightes.’’

In Anthony Wood’s Collection, at Oxford, No. 401, there is a modernization
of this ballad, entitled—
\settowidth{\versewidth}{St. George for England, and St. Dennis for France;}
\begin{scverse}
\vleftofline{“}St. George for England, and St. Dennis for France;\\
O hony soite qui mal y pance:
\end{scverse}
to an excellent new tune.” (Wood’s Ballads, ii. 118.) It is subscribed S. S., and
“printed for W. Gilbertson, in Giltspur Street;” from which it may be dated
about 1659.

As a specimen of the comparative modernization, I give the first stanza:—
\settowidth{\versewidth}{His sword with fame was crown’d;}
\begin{dcverse}
\begin{altverse}
\vleftofline{“}What need we brag or boast at all\\
Of Arthur and his Knights,\\
Knowing how many gallant men\\
They have subdued in fights.\\
For bold Sir Launcelot du Lake\\
Was of the table round;\\
And fighting for a lady’s sake,\\
His sword with fame was crown’d;\\
Sir Tarquin, that great giant,\\
His vassal did remain;\\
But St. George, St. George,\\
The Dragon he hath slain.\\
St. George he was for England,\\
St. Dennis was for France;\\
O hony soite qui mal y pance.”
\end{altverse}
\end{dcverse}

A copy of the \textit{old} ballad in the Bagford Collection is entitled “A new ballad
of St. George and the Dragon,” but there is also another of St. George and the
Dragon, which, Percy has printed in the \textit{Reliques}.

In \textit{180 Loyal Songs}, 1685 and 1694, there is “a new song on the instalment,of
Sir John Moor, Lord Mayor of London: tune, \textit{St. George for England}.” And in
\textit{Pills to purge Melancholy}, iii. 20 (1707), “A new ballad of King Edward and
Jane Shore,” to the same.
\pagebreak

%287
%========================================

As the ballad is contained in Percy’s \textit{Reliques}, as well as a witty second part,
written by John Grubb, and published in 1688, the first stanza only is here
printed with the music.

\musicinfo{Moderate time.}{}
\lilypondfile{lilypond/287-st-george-for-england}\normalsize 

\pagebreak

%288
%========================================

\musictitle{The Healths.}

This tune is in \textit{The Dancing Master}, from 1650 to 1690, and in \textit{Musick’s
Delight on the Cithren}, 1666.

In the first editions of \textit{The Dancing Master} it is entitled \textit{The Health}; in the
seventh and eighth, \textit{The Healths}, or \textit{The Merry Wassail}.

The following song, “Come, faith, since I’m parting,” was written by Patrick
Carey, a loyal cavalier, on bidding farewell to his hospitable entertainers at Wickham, 
in 1651. It is “to the tune of \textit{The Healths}.”

\musicinfo{Moderate time.}{}
\lilypondfile[staffsize=17]{lilypond/288-the-healths}\normalsize

%\vspace{-\baselineskip}

\settowidth{\versewidth}{He’ll make a brave man, you may see’t in his face;}
\indentpattern{0001}
\begin{dcverse}\footnotesizer\begin{patverse}
And first to Sir William, I’ll take’t on my knee;\\
He well doth deserve that a brimmer it be:\\
More brave entertainments none ere gave than he;\\
Then let his health go round.
\end{patverse}

\begin{patverse}
Next to his chaste lady, who loves him as life;\\
And whilst we are drinking to so good a wife,\\
The poor of the parish will pray for her life;\\
Be sure her health go round.
\end{patverse}

\begin{patverse}
And then to young Will, the heir of this place;\\
He’ll make a brave man, you may see’t in his face;\\
I only could wish we had more of the race;\\
At least let his health go round.
\end{patverse}

\begin{patverse}
To well-grac’d Victoria the next room we owe;\\
As virtuous she’ll prove as her mother, I trow,\\
And somewhat in huswifry more she will know;\\
O let her health go round!
\end{patverse}
\end{dcverse}
\pagebreak

%289
%========================================

\settowidth{\versewidth}{The most are good fellows, and love to carouse;}
\begin{dcverse}\footnotesizer\indentpattern{0001}
\begin{patverse}
To plump Bess, her sister, I drink down this cup:\\
Birlackins, my masters, each man must take’t up;\\
’Tis foul play, I bar it, to simper and sup,\\
When such a health goes round.
\end{patverse}

\begin{patverse}
And now, helter-skelter, to th rest of the house:\\
The most are good fellows, and love to carouse;\\
Who’s not, may go sneck-up;\dcfootnote{\textit{}
Sir Walter Scott prints this “sneake-up:” I~suppose
it should be “snecke-up”—a common expression,
equivalent to “go and be hanged.”}
 he’s not worth a louse\\
That stops a health i’ th’ round.
\end{patverse}

\begin{patverse}
To th’ clerk, so he’ll learn to drink in the morn;\\
To Heynous, that stares when he has quaft up his horn;\\
To Philip, by whom good ale ne’er was forlorn;\\
These lads can drink a round.
\end{patverse}

\begin{patverse}
John Chandler! come on, here’s some warm beer for you;\\
A health to the man that this liquor did brew:\\
Why Hewet! there’s for thee; nay take’t, ’tis thy due,\\
But see that it go round.
\end{patverse}

\begin{patverse}
Hot Coles is on fire, and fain would be quench’d;\\
As well as his horses, the groom must be drench’d:\\
Who’s else? let him speak, if his thirst he’d have stench’d,\\
Or have his health go round.
\end{patverse}

\begin{patverse}
And now to the women, who must not be coy;\\
A glass, Mistress Cary, you know’s but a toy;\\
Come, come, Mistress Sculler, no \textit{perdonnez moy},\\
It must, it must go round.
\end{patverse}

\begin{patverse}
Dame Nell, so you’ll drink, we’ll allow you a sop;\\
Up with’t, Mary Smith, in your draught never stop ;\\
Law, there now, Nan German has left ne’er a drop,\\
And so must all the round.
\end{patverse}

\begin{patverse}
Jane, Joan, Goody Lee, great Meg, and the less,\\
You must not he squeamish, but do as did Bess:\\
How th’ others are nam’d, if I could but guess,\\
I’d call them to the round.
\end{patverse}
\end{dcverse}

\settowidth{\versewidth}{And now, for my farewell, I drink up this quart,}
\begin{scverse}\footnotesizerrr
And now, for my farewell, I drink up this quart,\\
To you, lads aud lasses, e’en with all my heart;\\
May I find you ever, as now when we part,\\
\vin Each health still going round.
\end{scverse}

\musictitle{Mall Peatly.}

This tune is contained in \textit{Bellerophon, of Lust tot Wyshed}, Amsterdam, 1622;
in the seventh and later editions of \textit{The Dancing Master}; in \textit{Apollo's Banquet};
and in several of the ballad-operas.

In \textit{Bellerophon}, the first part is in common time, and the second in triple, like
a cushion dance; but it is not so in any of the above-named English copies,
which, however, are of later date.

D’Urfey wrote to it a song entitled \textit{Gillian of Croydon} (see \textit{Pills to purge
Melancholy}, ii. 46), and it is to be found under that name in some of the ballad-operas, 
such as \textit{The Fashionable Lady}, or \textit{Harlequin's Opera}, 1730; \textit{Sylvia}, or
\textit{The Country Burial}, 1731; \textit{The Jealous Clown}, 1730; \&c. There are also several
songs to it in the \textit{Collection of State Songs sung at the Mug-houses in London and
Westminster}, 1716. In \textit{Apollo’s Banquet}, the tune is entitled \textit{The Old Marinett},
or \textit{Mall Peatly}; in Gay’s \textit{Achilles}, \textit{Mo}ll Peatly.

Mall is the old abbreviation, of Mary. (See Ben Jonson’s \textit{English Grammar}.)

In \textit{Round about our coal-fire}, or \textit{Christmas Entertainments} (7th edit., 1734), it
is said, in allusion to Christmas, \pagebreak “This time of year being cold and frosty,
%290
%========================================
generally speaking, or when Jack-Frost commonly takes us by the nose, the
diversions are within doors, either in exercise or by the fire-side. Dancing is one
of the chief exercises—\textit{Moll Peatly} is never forgot;—this dance stirs the blood
and gives the males and females a fellow-feeling for each other’s activity, ability,
and agility: Cupid always sits in the corner of the room where these diversions
are transacting, and shoots quivers full of arrows at the dancers, and makes his
own game of them.”

\musicinfo{Gaily.}{}
\lilypondfile{lilypond/290-mall-peatly}\normalsize

\musictitle{Bobbing Joe, Oh Bobbing Joan.}

The tune of \textit{Bobbing Joe} will be found in every edition of \textit{The Dancing Master};
in \textit{Musick's Delight on the Cithren}, 1666; \&c.

It is sometimes entitled \textit{Bobbing Joan}, as by Carey in his \textit{Ballades} (1651); in
\textit{Polly}, 1729; in \textit{The Bay's Opera}, 1730; \textit{The Mad House}, 1737; \textit{A Cure for a
Scold}, 1738;~\&c.
\pagebreak


%291
%========================================

“New Bob-in-Jo” is mentioned as a tune in No. 38 of \textit{Mercurius Democritus,
or a True and Perfect Nocturnall}, December, 1652. (See King’s Pamphlets,
Brit. Mus.)

The song, “My dog and I,” is to the tune of \textit{My dog and I}, or \textit{Bobbing Joan}.
(A~copy in Mr. Halliwell’s Collection.)

The following is the ballad by Patrick Carey, “to the tune of \textit{Bobbing Joane}.”

\musicinfo{Cheerfully.}{}
\lilypondfile{lilypond/291-bobbing-joe-or-bobbing-joan}\normalsize

\settowidth{\versewidth}{Those beauties so, which were ensh}
\begin{dcverse}\indentpattern{0101220}
\begin{patverse}
It still was mine and others’ wonder\\
To see me court so eagerly;\\
Yet, soon as absence did me sunder\\
From those I lov’d, quite cured was I.\\
The reason was,\\
That my breast has,\\
Instead of heart, a looking-glass.
\end{patverse}

\begin{patverse}
And as those forms that lately shined\\
I’ th’ glass, are easily defac’d;\\
Those beauties so, which were enshrined\\
Within my breast, are soon displac’d:\\
Both seem as they\\
Would ne’er away;\\
Yet last but while the lookers stay.
\end{patverse}

\begin{patverse}
Then let no woman think that ever\\
In absence I shall constant prove;\\
Till some occasion does us sever\\
I can, as true as any, love:\\
But when that we\\
Once parted be,\\
Troth, I shall court the next I see.
\end{patverse}
\end{dcverse}

\musictitle{When The Stormy Winds Do Blow.}

The ballad, now known as \textit{You Gentlemen of England}, is an alteration of one
by M[artin] P[arker], a copy of which is in the Pepys Collection, i. 420; printed
at London for C. Wright. It is in black-letter, and entitled “Saylers for my
money: a new ditty composed in the praise of Saylers and Sea Affaires; briefly
shewing the nature of so worthy a calling, \pagebreak and effects of their industry: to the
%292
%========================================
tune of \textit{The~Joviall Cobler.}” Instead of “You gentlemen of England,” it begins,
“Countriemen of England,” \&c.

Ritson prints from a copy entitled “Neptune’s raging fury; or The Gallant
Seaman’s Sufferings. Being a relation of their perils and dangers, and of the
extraordinary hazards they undergo in their noble adventures: together with
their undaunted valour and rare constancy in all their extremites; and the
manner of their rejoycing on shore, at their return home. Tune of \textit{When the
stormy winds do blow}” (the burden of the song). A black-letter copy of this
version is in the Bagford Collection, printed by W. O[nley], temp. Charles II.;
and in one of the volumes of the Douce Collection, p.~168, printed by C. Brown
and T. Norris, and sold at the Looking Glass on London Bridge. A third in the
Roxburghe Collection, ii. 543. “\textit{Stormy winds}” is also in the list of ballads
printed by W.~Thackeray, about~1660.

On the accession of Charles II., we have, “The valiant Seaman’s Congratulation
to his Sacred Majesty King Charles the Second,” \&c.: to the tune of
\textit{Let us drink and sing, and merrily troul the bowl}, or \textit{The stormy winds do blow}, or
\textit{Hey, ho, my honey}.” (Black-letter, twelve stanzas; F. Grove, Snow Hill.)
It commences thus:--
\settowidth{\versewidth}{Great Charles, your English seamen,}
\begin{scverse}
\begin{altverse}
\vleftofline{“}Great Charles, your English seamen,\\
Upon our bended knee.\\
Present ourselves as freemen\\
Unto your Majesty.\\
Beseeching God to bless you\\
Where ever that you go;\\
So we pray, night and day,\\
When the stormy winds do blow.”
\end{altverse}
\end{scverse}

Although the option of singing it to three tunes is given, it is evident, from the
two last lines, that it was written to this.

Among the other ballads to the tune are, “The valiant Virgin, or Philip and
Mary: In a description of a young gentlewoman of Worcestershire (a rich gentleman’s 
daughter) being in love with a farmer’s son, which her father despising,
because he was poor, caus’d him to be press’d for sea: and how she disguised
herself in man’s apparel and follow’d him,” \&c. “To the tune of \textit{When the stormy
winds do blow};” (Roxburghe, ii. 546) beginning—
\begin{scverse}
\vleftofline{“}To every faithful lover\\
\vin That’s constant to her dear,” \&c.
\end{scverse}

In \textit{Poems by Ben Jonson, junior}, 8vo., 1672, is “The Bridegroom’s Salutation:
to the tune \textit{When the stormy winds do blow};” beginning—
\settowidth{\versewidth}{In all thy glories drest,” \&c.}
\begin{scverse}
\vleftofline{“}I took thee on a suddain,\\
\vin In all thy glories drest,” \&c.
\end{scverse}

In \textit{180 Loyal Songs}, 1686 and 1694, a bad version of the tune is printed to
“You Calvinists of England.”

There are fourteen stanzas in the copy of “You gentlemen” printed by Ritson,
in his \textit{English Songs}. The following shorter version is from one of the broadsides
with music, compared with another copy in \textit{Early Naval Ballads} (Percy Society,
No. 8, p.~34.)
\pagebreak

%293
%========================================

\musicinfo{Boldly.}{}
\lilypondfile{lilypond/293-when-the-stormy-winds-do-blow}\normalsize

\settowidth{\versewidth}{When the stormy winds do blow.}
\begin{dcverse}
\begin{altverse}
The sailor must have courage.\\
No danger he must shun;\\
In every kind of weather\\
His course he still must run;\\
Now mounted on the top-mast,\\
How dreadful ’tis below:\\
Then we ride, as the tide,\\
When the stormy winds do blow.
\end{altverse}

\begin{altverse}
If enemies oppose us,\\
And England is at war\\
With any foreign nation,\\
We fear not wound nor scar.\\
To humble them, come on, lads,\\
Their flags we’ll soon lay low;\\
Clear the way for the fray,\\
Tho’ the stormy winds do blow.
\end{altverse}

\begin{altverse}
Sometimes in Neptune’s bosom\\
Our ship is toss’d by waves,\\
And every man expecting\\
The sea to be our graves;\\
Then up aloft she’s mounted,\\
And down again so low,\\
In the waves, on the seas,\\
When the stormy winds do blow.
\end{altverse}

\begin{altverse}
But when the danger’s over,\\
And safe we come on shore,\\
The horrors of the tempest\\
We think of then no more;\\
The flowing bowl invites us,\\
And joyfully we go,\\
All the day drink away,\\
Tho’ the stormy winds do blow.
\end{altverse}
\end{dcverse}
\pagebreak

%294
%========================================
\musictitle{Red Bull.}

This tune is named after the Red Bull Playhouse, which formerly stood in
St. John Street, Clerkenwell. It was in use throughout the reigns of James I.
and Charles I., and perhaps before. At the Restoration, the King’s actors, under
Thomas Killigrew, played there until they removed to the new Theatre in Drury
Lane; and when Davenant produced his \textit{Playhouse to be Let}, in 1663, it was
entirely abandoned. (See Collier’s \textit{Annals of the Stage}.)

In the Roxburghe Collection, i. 246, is a ballad entitled “A mad kind of
wooing; or A Dialogue between Will the simple, and Nan the subtle, with their
loving agreement: to the tune of \textit{The new Dance at the Red Bull Playhouse}.”
It is black-letter, printed for the assigns of T. Symcocke, whose patent for
“printing of paper and parchment on the one side” was granted in 1620, and
assigned in the same year. Another copy of the ballad will be found in the
Pepys Collection, i.~276, “printed for H[enry] G[osson] on London Bridge.

The tune is contained in \textit{Apollo’s Banquet for the Treble Violin}, entitled \textit{The
Damsel’s Dance}; and in \textit{The Dancing Master}(1698), \textit{Red Bull}.

\musicinfo{Rather slow.}{}
\lilypondfile{lilypond/294-red-bull}\normalsize

The last eight bars are repeated for four more lines in the stanza. The whole
is reprinted in Evans’ \textit{Old Ballads}, i. 312 (1810).
\pagebreak

%295
%========================================

\musictitle{The Merry Milkmaids In Green.}

This is evidently the same air as \textit{And will he not come again}, one of the snatches
sung by Ophelia in \textit{Hamlet}, but in a different form (see p.~237). It is contained
in every edition of \textit{The Dancing Master}. In the eighteenth edition it is entitled
“The merry Milkmaids in \textit{green},” to distinguish it from another air of similar
name.

In Sir Thomas Overbury’s \textit{Character of a Milkmaid}, he says, “She dares go
alone, and unfold her sheep in the night, and fears no manner of ill, because she
means none: yet, to say truth, she is never alone, \textit{she is still accompanied with old
songs}, honest thoughts, and prayers, but short ones.”

In the “Character of a Ballad-monger,” in \textit{Whimzies, or a new Cast of
Characters}, 12mo., 1631, we find, “Stale ballad news, cashiered the city, must
now ride post for the country, where it is no less admired than a giant in a
pageant: till at last it grows so common there too, as every poor milkmaid\textit{ can
chant and chirp it under her cow}, which she useth, as a harmless charm, to make
her let down her milk.”

Maudlin, the milkmaid, in Walton’s \textit{Angler}, sings (among others) portions of
two ballads by Martin Parker, a well-known ballad-writer of the latter part of the
reign of James I., and during that of Charles and the Protectorate, and both are
to this tune. The first is—

“The Milkemaid’s Life; or—
\settowidth{\versewidth}{The praise of the milking paile to defend:}
\begin{scverse}
A pretty new ditty, composed and pen’d\\
The praise of the milking paile to defend:
\end{scverse}
to a curious new tune, called \textit{The Milkemaid’s Dumps}.” (Roxburghe Coll., i. 244,
or Collier’s \textit{Roxburghe Ballads}, 243.) Mr. Payne Collier remarks that the last
stanza but one proves it to have been written before “the downfall of May-games”
under the Puritans.

\settowidth{\versewidth}{That woods and fields possess,}
\begin{dcverse}
\indentpattern{0011011011110}
\begin{patverse}
You rural goddesses,\\
That woods and fields possess,\\
Assist me with your skill,\\
That may direct my quill\\
More jocundly to express\\
The mirth and delight.\\
Both morning and night.\\
On mountain or in dale,\\
Of those who choose\\
This trade to, use,\\
And through cold dews\\
Do never refuse\\
To carry the milking pail.
\end{patverse}

\begin{patverse}
The bravest lasses gay\\
Live not so merry as they;\\
In honest civil sort\\
They make each other sport,\\
As they trudge on their way.\\
Come fair or foul weather,\\
They’re fearful of neither—\\
Their courages never quail;\\
In wet and dry,\\
Though winds be high,\\
And dark’s the sky.\\
They ne’er deny\\
To carry the milking pail.
\end{patverse}

\begin{patverse}
Their hearts are free from care,\\
They never will despair;\\
Whatever may befall,\\
They bravely bear out all.\\
And Fortune’s frowns out-dare.\\
They pleasantly sing\\
To welcome the Spring—\\
’Gainst heaven they never rail;\\
If grass well grow.\\
Their thanks they show;\\
And, frost or snow,\\
They merrily go\\
Along with the milking pail.
\end{patverse}
\end{dcverse}
\pagebreak

%296
%========================================

\indentpattern{0011011011110}
\begin{dcverse}
\begin{patverse}
Bad idleness they do scorn;\\
They rise very early i’ th’ morn.\\
And walk into the field,\\
Where pretty birds do yield\\
Brave music on ev’ry thorn:\\
The linnet and thrush\\
Do sing on each bush.\\
And the dulcet nightingale\\
Her note doth strain\\
In a jocund vein,\\
To entertain\\
That worthy train\\
Which carry the milking pail.
\end{patverse}

\begin{patverse}
Their labour doth health preserve,\\
No doctors' rules they observe;\\
While others, too nice\\
In taking their advice,\\
Look always as though they would starve;\\
Their meat is digested.\\
They ne’er are molested,\\
No sickness doth them assail;\\
Their time is spent\\
In merriment;\\
While limbs are lent,\\
They are content\\
To carry the milking pail.
\end{patverse}

\begin{patverse}
Those lasses nice and strange,\\
That keep shops in the Exchange,\\
Sit pricking of clouts;\\
And giving of flouts;\\
They seldom abroad do range:\\
Then comes the green sickness\\
And changeth their likeness,\\
All this for want of good sale;\\
But ’tis not so,\\
As proof doth show,\\
By those that go\\
In frost and snow\\
To carry the milking pail.
\end{patverse}

\indentpattern{0011011011100}

\begin{patverse}
If they any sweethearts have\\
That do affection crave,\\
Their privilege is this,\\
Which many others miss:—\\
They can give them welcome brave.\\
With them they may walk,\\
And pleasantly talk,\\
With a bottle of wine or ale;\\
The gentle cow\\
Doth them allow,\\
As they know how.\\
God speed the plough,\\
And bless the milking pail.
\end{patverse}

\indentpattern{0011011011110}

\begin{patverse}
Upon the first of May,\\
With garlands fresh and gay;\\
With mirth and music sweet,\\
For such a season meet.\\
They pass their time away:\\
They dance away sorrow,\\
And all the day \textit{thorow},\\
Their legs do never fail;\\
They nimblely\\
Their feet do ply,\\
And bravely try\\
The victory,\\
In honour o’ th’ milking pail.
\end{patverse}

\begin{patverse}
If any think that I\\
Do practice flattery,\\
In seeking thus to raise\\
The merry milkmaids’ praise,\\
I’ll to them thus reply:\\
It is their desert\\
Inviteth my art\\
To study this pleasant tale;\\
In their defence,\\
Whose innocence\\
And providence\\
Gets honest pence\\
Out of the milking pail.
\end{patverse}
\end{dcverse}

There is another version of the above ballad in the Roxburghe Collection
(ii.~230), entitled “The innocent Country Maid’s Delight; or a Description of
the lives of the Lasses of London: set to \textit{an excellent Country Dance}.” It commences
with the lines quoted by the milkmaid from the above sixth stanza:
\settowidth{\versewidth}{That keep shop in the Exchange.”}
\begin{scverse}
\vleftofline{“}Some lasses are nice and strange\\
That keep shop in the Exchange.”
\end{scverse}
\pagebreak

%297
%========================================

The second ballad quoted by Maudlin is entitled “Keep a good tongue in your
head; or—
\settowidth{\versewidth}{But only her tongue breeds all her defect:}
\begin{scverse}
Here’s a good woman, in every respect,\\
But only her tongue breeds all her defect:
\end{scverse}

to the tune of \textit{The Milkmaids},” \&c. (Roxburghe Coll., i. 510, or Collier’s \textit{Roxburghe
Ballads}, 237.) From this I have selected a few stanzas to print with the
tune. It is sometimes referred to under its name, as in the following:—

“Hold your hands, honest men: for—
\settowidth{\versewidth}{Here’s a good wife hath a husband that likes her,}
\begin{scverse}
Here’s a good wife hath a husband that likes her,\\
In every respect, but only he strikes her;\\
Then if you desire to be held men complete.\\
Whatever you do, your wives do not beat.
\end{scverse}

To the tune of \textit{Keepe a good tongue},” \&c. (Roxburghe, i. 514.) The following
song by D’Urfey, entitled \textit{The Bonny Milkmaid}, was also written to the tune, but
had afterwards music composed to it for his play of \textit{Don Quixote}, and is so printed
in both editions of \textit{ Pills to purge Melancholy}, and in \textit{The Merry Musician, or
A Cure for the Spleen}, ii. 116. It is a rifacimento of Martin Parker’s song
printed above.

\settowidth{\versewidth}{That love green fields and woods,}
\begin{dcverse}\begin{patverse}
Ye nymphs and sylvan gods,\\
That love green fields and woods,\\
Where Spring, newly blown,\\
Herself does adorn\\
With flow’rs and blooming buds:\\
Come sing in the praise,\\
Whilst flocks do graze\\
In yonder pleasant vale,\\
Of those that choose\\
Their sleep to lose,\\
And in cold dews,\\
With clouted shoes,\\
Do carry the milking pail.
\end{patverse}

\begin{patverse}
The goddess of the morn\\
With blushes they adorn,\\
And take the fresh air,\\
Whilst linnets prepare\\
A concert in each green thorn.\\
The blackbird and thrush\\
On every bush,\\
And charming nightingale,\\
In merry vein\\
Their throats do strain\\
To entertain\\%
The jolly train\\
That carry the milking pail.
\end{patverse}

\begin{patverse}
When cold bleak winds do roar\\
And flow’rs can spring no more.\\
The fields that were seen\\
So pleasant and green\\
By Winter all candied o’er:\\
Oh! how the town lass\\
Looks, with her white face\\
And lips so deadly pale;\\
But it is not so\\
With those that go\\
Through frost and snow,\\
With cheeks that glow,\\
To carry the milking pail.
\end{patverse}

\begin{patverse}
The country lad is free\\
From fear and jealousy,\\
When upon the green\\
He is often seen\\
With a lass upon his knee;\\
With kisses most sweet\\
He does her greet,\\
And swears she'll ne’er grow stale;\\
While the London lass,\\
In every place.\\
With her brazen face,\\
Despises the grace\\
Of those with the milking pail.
\end{patverse}
\end{dcverse}

“The Merry Milkmaid’s Delight” was one of the ballads printed by
W. Thackeray, in the time of Charles II.

The following stanzas are selected from the ballad above-mentioned, “Keep
a good tongue in your head.”



\pagebreak

%298
%========================================

\musicinfo{Cheerfully.}{}
\lilypondfile{lilypond/298-the-merry-milkmaids-in-green}\normalsize

\indentpattern{0011011011110}
\settowidth{\versewidth}{Between her round chin and her n}
\begin{dcverse}\begin{patverse}
Her cheeks are red as the rose\\
Which June for her glory shows;\\
Her teeth in a row\\
Stand like a wall of snow\\
Between her round chin and her nose;\\
Her shoulders are decent,\\
Her arms white and pleasant,\\
Her fingers are small and long.\\
No fault I find,\\
But, in my mind,\\
Most womenkind\\
Must come behind:\\
O that she could rule her tongue!
\end{patverse}

\begin{patverse}
With eloquence she will dispute;\\
Few women can her confute.\\
\textit{She sings and she plays,\\
And she knows all the keys}\\
Of the vial de gambo, or lute.\\
She’ll dance with a grace,\\
Her Measures she’ll trace\\
As doth unto art belong;\\
She is a girl\\
Fit for an earl,\\
Not for a churl:\\
She were worth a pearl,\\
If she could but rule her tongue.
\end{patverse}
\end{dcverse}
\pagebreak%%

%299
%===================================================================


\settowidth{\versewidth}{O that she could rule her tongue!}
\begin{dcverse}
\begin{patverse}
Her needle she can use well,\\
In that she doth most excel;\\
She can spin and knit,\\
And every thing fit,\\
As all her neighbours can tell.\\
Her fingers apace\\
At weaving bone-lace\\
She useth all day long.\\
All arts that be\\
To women free,\\
Of each degree,\\
Performeth she:\\
O that she could rule her tongue!
\end{patverse}

\begin{patverse}
For huswifery she doth exceed;\\
She looks to her business with heed;\\
She’s early and late\\
Employ’d, I dare say’t,\\
To see all things well succeed.\\
She is very wary\\
To look to her dairy,\\
As doth to her charge belong;\\
Her servants all\\
Are at her call,\\
But she’ll so brawl\\
That still I shall\\
Wish that she could hold her tongue.
\end{patverse}
\end{dcverse}

\musictitle{The Queen’s Old Courtier.}

This ballad, which obtained a long and extensive popularity, seems to have
been first printed in the reign of James I. (by T. Symcocke).

Pepys thus refers to it in his Diary, under the date of 16th of June, 1668.
“Came to Newbery, and there dined, and music: a song of the Old Courtier of
Queen Elizabeth’s, and how he was changed upon the coming in of the King, did
please me mightily, and I did cause W. Hewer to write it out.” There are many
other versions of the ballad (sometimes entitled “The Old and New Courtier”),
and some are of greater length than others. Besides those in the great collections,
copies will be found in \textit{Le Prince d’Amour}, 1660; \textit{Antidote to Melancholy}, 1661;
\textit{Wit and Drollery}, 1682; Dryden’s \textit{Miscellany Poems}, iv., 108 (1716), \&c.

In \textit{Le Prince d’Amour}, and in \textit{Merry Drollery Complete}, 1661 and 1670, there
is a song of “An old \textit{Soldier} of the Queen’s commencing—
\settowidth{\versewidth}{With an old motley coat and a malmsey nose,”}
\begin{scverse}
\vleftofline{“}Of an old Soldier of the Queen’s,\\
With an old motley coat and a malmsey nose,”
\end{scverse}
and in \textit{Wit and Drollery}, 1682, p.~165, one entitled “Old Soldiers;” commencing—
\settowidth{\versewidth}{And we old fiddlers have forgot who they were,”}
\begin{scverse}
\vleftofline{“}Of old soldiers the song you would hear.\\
And we old fiddlers have forgot who they were,”
\end{scverse}
and at p.~282, “The new Soldier” (“With a new beard,” \&c.).

A ballad, written on the occasion of the overthrow of the Rump Parliament,
by General Monck, and dated Feb. 28, 1659, is amongst the King’s Pamphlets,
Brit. Mus. (folio broadsides, vol. xvi.). It“is entitled “Saint George and the
Dragon, anglice Mercurius Poeticus.” To the tune of “\textit{The old Souldier of the
Queen’s};” commencing—
\settowidth{\versewidth}{A dialogue between Haselrigg the baffled, and Arthur the furious}
\begin{scverse}
\vleftofline{“}News, news,—here’s the occurrences and a new Mercurius,\\
A dialogue between Haselrigg the baffled, and Arthur the furious,\\
With Ireton’s readings upon legitimate and spurious, \&c.”
\end{scverse}
It is reprinted in Wright’s Political Ballads (Percy Soc., No. 11).

In the reign of Charles II., “T. Howard, Gent.,” wrote and published “An
old song of the Old Courtiers of the \textit{King’s}, with a new song of a New Courtier of
\pagebreak
%300
%===================================================================
the King’s: to the tune of \textit{The Queen’s Old Courtier}.” A copy of this latter,
“printed for F. Coles,” is among the Roxburghe Ballads.

Dr.~King, in his “Preface to the Art of Cookery, in imitation of Horace’s Art
of Poetry,” declares his love “to the old British Hospitality, charity and valour,
when the arms of the family, the old pikes, muskets, and halberts, hung up in
the hall over the long table, and \textit{Chevy Chase}, and \textit{The Old Courtier of the Queen’s}
were placed over the carved mantle-piece, and beef and brown bread were carried
every day to the poor.” (Dr.~King’s Works, vol. iii.)

About the middle of the last century the ballad was revived and sung by
Mr. Vernon in Shadwell’s comedy, \textit{The Squire of Alsatia}, the burden being altered
to “Moderation and Alteration,” and, when comparing the young courtier to
the old, to—
\settowidth{\versewidth}{’Tis a wonderful alteration.”}
\begin{scverse}
\vleftofline{“}Alteration, alteration,\\
’Tis a wonderful alteration.”
\end{scverse}

Finally, it has been again revived, with further “alteration,” in the present
century, under the title of “The old English Gentleman.”

The ballad is to be chanted, \textit{ad libitum}, upon one note, except the final syllable
of each stanza, and the burden “Like an old Courtier,” \&c.

\musicinfo{To be sung ad. lib. upon one note.}{}
\lilypondfile{lilypond/300-the-queens-old-courtier}\normalsize

\settowidth{\versewidth}{Who every quarter pays her old servants their}
\begin{dcverse}With an old lady whose anger a good word assuages,\\
Who every quarter pays her old servants their wages\\
Who never knew what belonged to coachman, footmen, nor pages;\\
But kept twenty old fellows with blue coats and badges.\\
Like an old Courtier, \&c.

With an old study fill’d full of learned old books,\\
With an old reverend parson, you may judge him by his looks.\\
With an old buttery hatch worn quite off the hooks\\
And an old kitchen, that maintains half-a-dozen old cooks.\\
Like an old, \&c.

\end{dcverse}
\pagebreak


%301
%===================================================================
\settowidth{\versewidth}{But in the ensuing ditty you shall hear how he was inclin’d. Like a young Courtier, \&c.}
\begin{dcverse}\footnotesize
With an old hall hung about with guns, pikes, and bows,	\\
With old swords and bucklers that have stood many shrewd blows,\\
And an old frieze coat to cover his worship’s trunk hose,\\
And a cup of old sherry to comfort his copper nose. Like an old, \&c.

With an old fashion when Christmas was come,\\
To call in his neighbours with bagpipe and drum;\\
And good cheer enough to furnish every old room,\\
And old liquor able to make a cat speak and a man dumb. Like an old, \&c.

With an old huntsman, a falconer, and a kennel of hounds;	\\
Which never hunted nor hawked but in his own grounds;\\
Who like an old wise man kept himself within his own bounds,\\
And when be died, gave every child a thousand old pounds. Like an old, \&c.

But to his eldest son, his house and land he assigned,\\
Charging him in his will to keep the old bountiful mind,\\
To love his good old servants and to his neighbours be kind;\\
But in the ensuing ditty you shall hear how he was inclin’d. Like a young Courtier, \&c.

Like a young gallant newly-come to his land,\\
That keeps a brace of creatures at his command,\\
And takes up a thousand pound upon his own land\\
And lies drunk in a new tavern, ’till he can neither go nor stand. Like a young, \&c.

With a neat lady that is brisk and fair,\\
That never knew what belonged to good house-keeping or care,\\
But buys several fans to play with the wanton air,\\
And seventeen or eighteen dressings of other women’s hair. Like a young, \&c.

With a new hall built where the old one stood,\\
Wherein is burned neither coal nor wood.\\
And a shovelboard-table whereon meat never stood,\\
Hung round with pictures that do the poor no good. Like a young, \&c.

With a new study stuft full of pamphlets and plays;\\
With a new chaplain that swears faster than he prays ;\\
With a new buttery hatch that opens once in four or five days;\\
With a new French cook to make kickshaws and toys. Like a young, \&c.

With a new fashion when Christmas is come,\\
With a new journey up to London we must be gone,\\
And leave nobody at home but our new porter John,\\
Who relieves the poor with a thump on the back with a stone. Like a young, \&c.

With a gentleman usher whose carriage is complete;\\
With a footman, coachman, and page to carry meat;\\
With a waiting-gentlewoman whose dressing is very neat;\\
Who, when the master has dined, lets the servants not eat. Like a young, \&c.
\end{dcverse}

\settowidth{\versewidth}{With a new honour bought with the old gold,}
\begin{scverse}\footnotesizer 
With a new honour bought with the old gold,\\
That many of his father’s old manors had sold,\\
And this is the occasion that most men do hold,\\
That good house-keeping is now grown so cold. Like a young, \&c.
\end{scverse}

\musictitle{Joan, To The Maypole.}

This ballad is in the Roxburghe Collection, ii. 354, and Douce Collection,
p.~152. It is entitled “May-day Country Mirth; or The young Lads’ and
Lasses’ innocent Recreation, which is to be prized before courtly pomp and pastime: 
to an excellent new tune.” Dr.~Rimbault, in his “Little Book of Songs and
Ballads, gathered from Ancient Music-books,” prints a version “from a MS.
volume of old songs and music, formerly in the possession of the Rev. H. J.
Todd, dated 1630.” The same is in Evans’ \textit{Old Ballads} i. 245 (1810). Another
version will be found with the tune in \textit{ Pills to purge Melancholy}, ii. 145 (1707),
or iv. 145 (1719), with many more stanzas.
\pagebreak
%302
%===================================================================

\musicinfo{Gaily.}{}
\lilypondfile{lilypond/302-joan-to-the-maypole}\normalsize

\pagebreak
%303
%===================================================================

\settowidth{\versewidth}{Joan, shall we have a Hay or a Round,}
\indentpattern{00000101}
\begin{dcverse}\footnotesizer
\begin{patverse}
Joan, shall we have a Hay or a Round,\\
Or some dance that is new-found?\\
Lately I was at a Masque in the Court,\\
Where I saw of every sort,\\
Many a dance made in France,\\
Many a Braule, and many a Measure;\\
Gay coats, sweet notes,\\
Brave wenches—O ’twas a treasure.
\end{patverse}


But now, methinks, these courtly toys\\
Us deprive of better joys:\\
Gown made of gray, and skin soft as silk,\\
Breath sweet as morning milk;\\
O, these more please;\\
{[All]} these hath my Joan to delight me:\\
\vin False wiles, court smiles,\\
None of these hath my Joan to despite me.

\end{dcverse}

In \textit{ Pills to purge Melancholy}, the above second and third stanzas are replaced
by others, such as the following:—

\begin{dcverse}\footnotesizer
\begin{patverse}
Did you not see the Lord of the May\\
Walk along in his rich array?\\
There goes the lass that is only his;\\
See how they meet, ,and how they kiss!\\
Come Will, run Gill,\\
Or dost thou list to lose thy labour;\\
Kit, Crowd, scrape aloud,\\
Tickle up Tom with a pipe and tabor.
\end{patverse}

\begin{patverse}
Lately I went to a Masque at the Court,\\
Where I saw dances of every sort;\\
There they did dance with time and measure,\\
But none like a country-dance for pleasure;\\
They did dance as in France,\\
Not like the English lofty manner;\\
And every she must furnished be\\
With a feathered knack, when she’s hot for to fan her.
\end{patverse}

\begin{patverse}
But we, when we dance, and do happen to sweat,\\
Have a napkin in hand for to wipe off the wet;\\
And we with our lasses do jig it about,\\
Not like at Court, where they often are out;\\
If the tabor play, we jump away,\\
And turn, and meet our lasses to kiss 'em;\\
Nay, they will he as ready as we,\\
That hardly at any time can we miss 'em.
\end{patverse}
\columnbreak

\begin{patverse}
Come, sweet Joan, let us call a new dance,\\
That we before ’em may advance;\\
Let it be what you desire and crave,\\
And sure the same sweet Joan shall have.\\
She cried, and replied,\\
If to please me thou wilt endeavour,\\
Sweet Pig, the Wedding Jig,\\
Then, my dear, I’ll love thee for ever.
\end{patverse}

\begin{patverse}
There is not any that shall outvie\\
My litttle pretty Joan and I;\\
For I am sure I can dance as well\\
As Robin, Jenny, Tom, and Nell:\\
Last year we were here,\\
When rough Ralph he played us a Boree,\\
And we merrily\\
Thump’d it about, and gain’d the glory.
\end{patverse}

\begin{patverse}
And if we hold on as we begin,\\
Joan, thou and I the garland shall win;\\
Nay, if thou live till another day,\\
I’ll make thee Lady of the May.\\
Dance about, in and out,\\
Turn and kiss, and then for greeting;\\
Now, Joan, we have done,\\
Fare thee well till next merry meeting.
\end{patverse}
\end{dcverse}

\musictitle{Love Will Find Out The Way.}

This tune is contained in Playford’s \textit{Musick’s Recreation on the Lyra Viol},
1652; in \textit{Musick’s Delight on the Cithren}, 1666; in the Skene and several other
MSS.; also in \textit{ Pills to purge Melancholy}, vi. 86 (1719).

The words are in Percy’s \textit{Reliques}; Evans’ \textit{Old Ballads}, iii. 282 (1810); and
Rimbault’s \textit{Little Book of Songs and Ballads}, p.~137. All these versions differ.

Evans prints from a black-letter copy by F. Coules (whose ballads occasionally
bear dates which vary from 1620 to 1628); Rimbault from Forbes’ \textit{Cantus}, 1662,
with the second part from Coules’ copy; and Percy from a comparatively modern
edition.

The ballad is quoted in Brome’s \textit{Sparagus Garden}, acted in 1635, and its
popularity was so great, that “Love will find out the way” was taken as the
title to a play printed in 1661. Although stated on the title-page to be a
comedy by T. B., it was only Shirley’s \textit{Constant Maid}, under a new name.
\pagebreak
%304
%===================================================================

The air is still current, for in the summer of 1855, Mr. Jennings, Organist of
All Saints’ Church, Maidstone, noted it down from the wandering hop-pickers
singing a song to it, on their entrance into that town.

The title of the ballad, as printed by Coules, is “Truth’s Integrity; or
A curious Northern ditty, called \textit{Love will find out the way}: to a pleasant new
tune.” A later copy in the Douce Collection, p.~232, is entitled “A curious
Northern ditty, called \textit{Love will find out the way}.”

In the Roxburghe Collection, ii. 436, is a black-letter ballad of “Stephen and
Cloris; or The coy Shepherd and the kind Shepherdess: to a new play-house
tune, or \textit{Love will find out the way}.”

I suppose ballads which are said to be “to the tune of \textit{Over hills and high
mountains},” are also intended for this air; because the words of that ballad are
almost a paraphrase of this, and in the same measure. See the following stanza
from a copy in the Pepys Collection, iii. 165:—

%\vspace{-0.5\baselineskip}
\settowidth{\versewidth}{Ah! and down by the fountains,}
\begin{dcverse}\begin{altverse}
\vleftofline{“}Over hills and high mountains\\
Long time have I gone;\\
Ah! and down by the fountains,\\
By myself all alone;
\end{altverse}

\begin{altverse}
Through bushes and briers,\\
Being void of all care;\\
Through perils and dangers\\
For the loss of my dear.”
\end{altverse}
\end{dcverse}

%\vspace{-0.5\baselineskip}

There is, however, an air, entitled \textit{On yonder high mountains}, which may be intended,
and which will be found in this collection, under a later date.

Another black-letter ballad to the tune of \textit{Love will find out the way}, is entitled
“The Countryman’s new Care away;” commencing—

%\vspace{-0.5\baselineskip}
\begin{dcverse}\begin{altverse}
\vleftofline{“}If there were employments\\
For men, as have been;\\
And drums, pikes, and muskets,\\
I’ the field to he seen;
\end{altverse}

\begin{altverse}
And every worthy soldier\\
Had truly his pay;\\
Then might they be bolder\\
To sing Care away.”
\end{altverse}
\end{dcverse}

%\vspace{-0.5\baselineskip}

As the version of \textit{Love will find out the way} printed by Percy is the shortest,
consisting in all of but five stanzas, it is here coupled with the tune.

\musicinfo{Smoothly and not too fast.}{}
\lilypondfile{lilypond/304-love-will-find-out-the-way}\normalsize

\pagebreak
%305
%=================================================================== 

\settowidth{\versewidth}{Where the midge dares not venture,}
\begin{dcverse}\footnotesizer\begin{altverse}
Where there is no place\\
For the glow-worm to lie;\\
Where there is no space\\
For receipt of a fly;\\
Where the midge dares not venture,\\
Lest herself fast she lay;\\
If Love come, he will enter,\\
And soon find out his way.
\end{altverse}

\begin{altverse}
You may esteem him\\
A child for his might;\\
Or you may deem him\\
A coward from his flight.\\
But if she, whom Love doth honour,\\
Be conceal’d from the day,\\
Set a thousand guards upon her,\\
Love will find out the way.
\end{altverse}

\begin{altverse}
Some think to lose him,\\
By having him confin’d;\\
And some do suppose him,\\
Poor thing, to be blind;\\
But if ne’er so close ye wall him,\\
Do the best that you may,\\
Blind Love, if so ye call him,\\
Soon will find out his way.
\end{altverse}

\begin{altverse}
You may train the eagle\\
To stoop to your fist;\\
Or you may inveigle\\
The phœnix of the east;\\
The lioness, ye may move her\\
To give o’er her prey;\\
But you’ll ne’er stop a lover:\\
He will find out his way.
\end{altverse}
\end{dcverse}

\vspace{-1.5\baselineskip}
%\vspace{-1.5\baselineskip}
\musictitle{Stingo, Or Oil Of Barley.}
\vspace{-0.5\baselineskip}

This tune is contained in every edition of \textit{The Dancing Master}, and in many
other publications. It is often quoted under three, if not more, names.

In \textit{The Dancing Master}, from 1650 to 1690, it appears as \textit{Stingo}, or \textit{The Oyle
of Barley}.

The song, “A cup of old stingo” (\ie, old strong beer), is contained in \textit{Merry
Drollery Complete}, 1661 and 1670, and, if it be the original song, must be of a
date from thirty to forty (and perhaps more) years earlier than the book.

Traces of that doughty hero, Sir John Barleycorn, so famous in the days of
ballad-singing, are to be found as far back as the time of the Anglo-Saxons. In
the Exeter MS. (fol. 107) is an enigma in Anglo-Saxon verse, of which the
following is a literal translation;—

{\footnotesizer “A part of the earth is prepared beautifully with the hardest, and with the sharpest,
and with the grimest of the productions of men, cut and\ldots\  . (sworfen), turned and
dried, bound and twisted, bleached and awakened, ornamented and poured out, carried
afar to the doors of people; it is joy in the inside of living creatures, it knocks and
slights those, of whom before, while alive, a long while it obeys the will, and expostulateth
not; and then after death it takes upon it to judge, to talk variously. It is
greatly to seek by the wisest man, what this creature is.”—\textit{Essay on the State of
Literature and Learninq under the Anglo-Saxons, by Thomas Wright, Esq., M.A.,
F.S.A.}, p.~79, 8vo., 1839.}

In the Roxburghe Collection, i. 214, there is a black-letter ballad “to the tune
of \textit{Stingo},” which was evidently written in the reign of Charles I., by its
allusions to “the King’s great porter,” “Bankes’ Horse,” \&c. It is entitled,
“The Little Barley-Corn:
\settowidth{\versewidth}{Whose properties and vertues here}
\begin{scverse}\footnotesizer Whose properties and vertues here\\
Shall plainly to the world appeare;\\
To make you merry all the yeere.”
\end{scverse}

As it has been reprinted in Evans’ \textit{Old Ballads} i. 156 (1810), the first stanza
only is subjoined:—
\vspace{-0.5\baselineskip}
\begin{dcverse}\footnotesizer \begin{altverse}
“Come, and do not musing stand,\\
If thou the truth discern;\\
But take a full cup in thy hand,\\
And thus begin to learn:\\
Not of the earth, nor of the air,\\
At evening or at morn,\\
But, jovial boys, your Christmas keep\\
With the little barley-corn.”
\end{altverse}
\end{dcverse}
\noindent The ballad is divided into two parts, each consisting of eight stanzas.
\pagebreak
%306
%===================================================================

A second name for the tune is \textit{The Country Lass}, which it derived from a
ballad by Martin Parker. Copies of that ballad are in the Pepys Collection
(i. 268), and in the Roxburghe (i. 52). The former bears Martin Parker’s
initials, but no printer’s name; the latter was printed for the assigns of Thomas
Symcocke.

The copy in the Pepys Collection is entitled “The Countrey Lasse:
\settowidth{\versewidth}{To a dainty new note: which if you cannot hit,}
\begin{scverse}
To a dainty new note: which if you cannot hit,\\
There’s another tune which doth as well fit—\\
That’s \textit{The Mother beguil’d the Daughter}.”
\end{scverse}

\settowidth{\versewidth}{Although I am a countrey lasse,}
\begin{dcverse}\begin{altverse}
\vleftofline{“}Although I am a countrey lasse,\\
A loftie minde I beare-a;\\
I thinke myselfe as good as those\\
That gay apparrell weare-a.\\
My coat is made of comely gray,\\
Yet is my skin as soft-a,\\
As those that with the choicest wines\\
Do bathe their bodies oft-a.
\end{altverse}

\begin{altverse}
Downe, downe derry, derry downe,\\
Heigh downe, a downe, a downe-a,\\
A derry, derry, derry downe,\\
Heigh downe, a downe, a derry.”
\end{altverse}
\end{dcverse}

This is reprinted in Evans’ \textit{Old Ballads}, i. 41, and an altered copy will be found,
with the music, in \textit{ Pills to purge Melancholy}, ii. 165 (1707), or iv. 152 (1719).

The tune is referred to, under the above name, in a ballad by Laurence Price,
entitled “Good Ale for my money:
\settowidth{\versewidth}{The good fellowes resolution of strong ale,}
\begin{scverse}
The good fellowes resolution of strong ale,\\
That cures his nose from looking pale.
\end{scverse}
To the tune of \textit{The Countrey Lasse}.

\settowidth{\versewidth}{Be merry, my friends, and list awhile}
\begin{dcverse}\begin{altverse}
Be merry, my friends, and list awhile\\
Unto a merry jest,\\
It may from you produce a smile,\\
When you hear it exprest;\\
Of a young man lately married,\\
Which was a boone good fellow,
\end{altverse}

\begin{altverse}
This song in’s head he always carried,\\
When drink had made him mellow:\\
I cannot go home, nor will I go home,\\
It’s long of the \textit{oyle of barley};\\
I’ll tarry all night for my delight.\\
\textit{And go home in the morning early}.”
\end{altverse}
\end{dcverse}

A copy will be found in the Roxburghe Collection, i. 138.

Hilton wrought this tune into a catch for three voices, and published it in his
\textit{Catch that catch can}, in 1652; and it was afterwards reprinted in that form by
Playford in his \textit{Musical Companion}, 1667, 1673, \&c.

The first line of the catch is “I’se goe with thee, my sweet Peggy, my honey.”
The third part is to the tune of \textit{Stingo}, with the following words:—

\settowidth{\versewidth}{And what we doe neene shall know;}
\indentpattern{00101}
\begin{scverse}
\begin{patverse}
\vleftofline{“}Thou and I will foot it, Joe,\\
And what we doe neene shall know;\\
But taste the \textit{juice of barley}.\\
We’ll sport all night for our delight,\\
And \textit{home in the morning early}.”
\end{patverse}
\end{scverse}

\noindent The air is somewhat altered to harmonize with the other parts.

In the editions of \textit{The Dancing Master} which were printed \textit{after} 1690, the
name is changed from \textit{Stingo}, or \textit{The Oyle of Barley}, to \textit{Cold and raw}. This new
title was derived from a (so called) “New Scotch Song,” written by Tom
D’Urfey, which first appeared in the second book of \textit{Comes Amoris}, or \textit{The}
\pagebreak
%307
%===================================================================
\textit{Companion of Love}, printed by John Carr in 1688;\dcfootnote{\textit{}
a A few pages further in the same book there is another
“new Scotch song,” set by Mr. Akeroyd.

Ritson, in his \textit{Historical Essay on Scotish Song}, 1794,
says, “An inundation of \textit{Scotch songs}, so called, appears
to have been poured upon the town by Tom D’Urfey and
his Grub-street brethren, toward the end of the seventeenth
and in the beginning of the eighteenth century; of
which it is hard to say whether wretchedness of poetry,
ignorance of the Scotish dialect, or nastiness of ideas, is
most evident, or most despicable. In the number of
these miserable caricatures, the reader may be a little surprised
to find the favorite songs of \textit{De’ill take the Wars
that hurry'd Willy from me}; \textit{O Jenny, Jenny, where hast thou
been}? \textit{Young Philander wooed me lang; Farewell, my
bonny, witty, pretty Moggy; In January last; She rose and
let me in; Pretty Kate of Edinburgh; As I sat at my spinning
wheel; Fife, and a’ the lands about it; Bonny lad,
prithee lay thy pipe down; The bonny grey-ey'd morn;
’Twas within a furlong of Edinburgh town; Bonny Dundee;
O'er the hills and far away; By moonlight on the green;
What’s that to you}? and several others, which he has
been probably used to consider as genuine specimens of
Scotish song; as, indeed, most of them are regarded
even in Scotland.” Ritson's list might be very greatly
extended.}
 and, as frequently the case,
the air was a little altered for the words.

Of this song Sir John Hawkins relates the following anecdote in his \textit{History of
Music} (8vo., ii. 564):—

{\small
“This tune was greatly admired by Queen Mary, the consort of King William;
and she once affronted Purcell by requesting to have it sung to her, he being present.
The story is as follows: The Queen having a mind one afternoon to be entertained
with music, sent to Mr. Gosling, then one of her Chapel, and afterwards Sub-Dean of
St. Paul’s, to Henry Purcell, and to Mrs. Arabella Hunt, who had a very fine voice,
and an admirable hand on the lute, with a request to attend her; they obeyed her
commands; Mr. Gosling and Mrs. Hunt sung several compositions of Purcell, who
accompanied them upon the harpsichord; at length, the Queen beginning to grow
tired, asked Mrs. Hunt if she could not sing the ballad of ‘Cold and raw;’\dcfootnote{\textit{}
Sir John Hawkins, who relates the anecdote \textit{traditionally},
and who had evidently seen no older copy of the tune
than that contained in the Catch (as he elsewhere mentions
Hilton’s \textit{Catches} as Playford’s \textit{first} publication) calls
it “the old Scot's ballad,” but from the allusion to “the
next birth-day song,” it must have happened within four
years of the first publication. The term “old,” could
therefore only be applied, with propriety, to the music.}
 Mrs.
Hunt answered, Yes, and sung it to her lute. Purcell was all the while sitting at the
harpsichord unemployed, and not a little nettled at the Queen’s preference of a vulgar
ballad to his music; but seeing Her Majesty delighted with this tune, he determined
that she should hear it upon another occasion; and, accordingly, in the next birthday
song, viz., that for the year 1692, he composed an air ta the words, ‘May her
bright example chace vice in troops out of the land,’ the bass whereof is the tune to
‘Cold and raw.’”
}

In Anthony à Wood’s collection of broadsides (Ashmolean Library, vol. 417)
there are two ballads with music, bearing the date of December, 1688, and
printed to this tune. The first is “The Irish Lasses Letter; or her earnest
request to Teague, her dear joy: \textit{to an excellent new tune}.” The second is the
famous song of \textit{Lilliburlero}.

In the Douce Collection is a ballad called “The lusty Friar of Flanders: to
the tune of \textit{Cold and raw}.”

Horace Walpole mentions it under the same name in a letter to Richard West,
Esq., dated from Florence (Feb. 27, 1740), where, in speaking of the Carnival,
he says, “The Italians are fond to a degree of our Country Dances.\dcfootnote{\textit{}
This agrees with what I have been told about the book
entitled \textit{The Dancing Master} (the early editions of which
are extremely scarce in England), viz., that it is very well
known to the dealers in Italy, and that it may be procured
there with comparatively little trouble.}
 \textit{Cold and
raw} they only know by the tune; \textit{Blowzybella} is almost Italian, and \textit{Butter'd
Peas} is \textit{Pizelli al buro}.” (\textit{Letters of Walpole}, in vi. vols, 1840; vol. i. p.~32.)

The following is the song of “A cup of old stingo,” from \textit{Merry Drollery
Complete}, with the tune from \textit{The Dancing Master} of 1650.
\pagebreak
%308
%===================================================================

\musicinfo{Jovially.}{}
\lilypondfile{lilypond/308-stingo-or-oil-of-barley}\normalsize

\settowidth{\versewidth}{’Twill make him dance about a cross,}
\begin{dcverse}\begin{altverse}
’Twill make a man indentures make,\\
’Twill make a fool seem wise,\\
’Twill make a Puritan sociate,\\
And leave to be precise:\\
’Twill make him dance about a cross,\\
And eke to run the ring too,\\
Or anything he once thought gross,\\
Such virtue hath old stingo.
\end{altverse}

\begin{altverse}
’Twill make a constable oversee\\
Sometimes to serve a warrant,\\
’Twill make a bailiff lose his fee,\\
Though he be a knave-arrant;\\
’Twill make a lawyer, though that he\\
To ruin oft men brings, too,\\
Sometimes forget to take his fee,\\
If his head be lin’d with stingo.
\end{altverse}

\begin{altverse}
’Twill make a parson not to flinch,\\
Though he seem wondrous holy,\\
And for to kiss a pretty wench,\\
And think it is no folly;\\
’Twill make him learn for to decline\\
The verb that’s called \textit{Mingo},\\
’Twill make his nose like copper shine,\\
If his head be lin’d with stingo.
\end{altverse}

\begin{altverse}
’Twill make a weaver break his yarn,\\
That works with right and left foot,\\
But he hath a trick to save himself,\\
He’ll say there wanteth woof to’t;\\
’Twill make a tailor break his thread,\\
And eke his thimble ring too,\\
’Twill make him not to care for bread,\\
If his head be lin’d with stingo.
\end{altverse}

\begin{altverse}
’Twill make a baker quite forget\\
That ever corn was cheap,\\
’Twill make a butcher have a fit\\
Sometimes to dance and leap;\\
’Twill make a miller keep his room,\\
A health for to begin, too,\\
’Twill make him shew his golden thumb.\\
If his head be lin’d with stingo.
\end{altverse}

\begin{altverse}
’Twill make an hostess free of heart,\\
And leave her measures pinching,\\
’Twill make an host with liquor part\\
And bid him hang all flinching;\\
It’s so belov’d, I dare protest,\\
Men cannot live without it,\\
And where they find there is the best,\\
The most will flock about it.
\end{altverse}
\end{dcverse}
\pagebreak
%309
%===================================================================

\settowidth{\versewidth}{Though he be lame, he’ll prove his crutch,}
\begin{dcverse}\footnotesizerr\begin{altverse}
And, finally, the beggar poor,\\
That walks till he be weary,\\
Craving along from door to door,\\
With \textit{pre-commiserere};\\
If he do chance to catch a touch,\\
Although his clothes be thin, too,\\
Though he be lame, he’ll prove his crutch,\\
If his head be lin’d with stingo.
\end{altverse}

\begin{altverse}
Now to conclude, here is a health\\
Unto the lad that spendeth,\\
Let every man drink off his can,\\
And so my ditty endeth;\\
I willing am my friend to pledge,\\
For he will meet me one day;\\
Let’s drink the barrel to the dregs.\\
For the \textit{malt-man comes a Monday}.
\end{altverse}
\end{dcverse}

The last line has furnished the subject for a Scotch song.

The following is a later version of the tune. The copies in \textit{The Beggars’
Opera}, \textit{ Pills to purge Melancholy}, \textit{The Dancing Master}, and \textit{Midas} (1764), have
all slight differences, such as would occur from writing down a familiar tune from
memory. The words are Tom D’Urfey’s “last new Scotch song.” (See
\textit{Comes Amoris}, or \textit{The Companion of Love}, ii. 16, fol. 1688.)

\musicinfo{Gracefully.}{}
\lilypondfile{lilypond/309-last-new-scotch-song}\normalsize

\vspace{-\baselineskip}

\begin{dcverse}\footnotesizerr\begin{altverse}
Down I veil’d my bonnet low,\\
Thinking to show my breeding;\\
She returned a graceful bow—\\
A village far exceeding.
\end{altverse}

\begin{altverse}
I ask’d her where she went so soon,\\
I long’d to begin a parley,\\
She told me to the next market town\\
On purpose to sell her barley.\dcfootnote{\textit{}
However unobjectionable this song may have been in
Queen Mary’s time, the three remaining stanzas would
not be very courteously received in Queen Victoria's
\textit{Tempora mutantur}.}
\end{altverse}
\end{dcverse}
\pagebreak
%310
%===================================================================


\musictitle{What If A Day, Or A Month, Or A Year?}

Copies of this song are in the Roxburghe Collection, i. 116 and ii. 182, and
in \textit{The Golden Garland of Princely Delights}, third edition, 1620. In the
Roxburghe Ballads it is entitled “A Friend’s Advice, in an excellent ditty,
concerning the variable changes in this world” (printed by the assigns of Thomas
Symcocke); in \textit{The Golden Garland}, “The inconstancy of the world.”

The music is in a volume of transcripts of virginal music, by Sir John Hawkins;
in \textit{Logonomia Anglica}, by Alexander Gil, 1619; in \textit{Friesche Lust-Hof}, 1634; in
D. R. Camphuysen’s \textit{Stichtelycke Rymen}, 4to., Amsterdam, 1647; in the Skene
MS.; in Forbes’ \textit{Cantus}; \&c. The same words are differently set by Richard
Allison, in his \textit{Howre’s Recreation in Musicke}, 1608.

Gil (or Gill), who was Master of St. Paul’s School, refers to the song twice in his
\textit{Logonomia}. Firstly, “Hemistichium est, duobus constans dactylis, et choriambo;”
and secondly, “Ut in illo perbello cantico Tho. Campaiani, cujus mensuram, ut
rectius agnoscas, exhibeo cum notis.”

Thomas Campian, or Campion, to whom the poetry, and perhaps also the
music, is here ascribed, was by profession a physician; but he was also an eminent
poet and admirable musician. He flourished during the latter part of the
reign of Elizabeth and the greater portion of that of James I. Neither the words
nor music are, however, to be found in his printed collections.

According to the registers of St. Dunstan’s in the West, “Thomas Campion,
Doctor of Physicke,” was buried there on the 1st of March, 1619.\dcfootnote{%\scriptsizerr
Haslewood supposed him to have died in 1621, It
is strange that the name of so eminent a man should
have been omitted in the usual Biographical Dictionaries
and Universal Biographies. A short account of him is
given, with the reprint of his “Observations in the art
of English Poetry,” in Haslewood's “Ancient Critical
Essays upon English Poets and Poësy.” Haslewood
does not notice his four books of “Ayres,” printed in
1610 and 1612, which, with some others, are described in
Rimbault’s \textit{Bibliothica Madrigaliana}. He composed the
Psalm tune, called “Babylon’s streams,” which is still
in use. His \textit{Art of Descant} is contained in Playford’s
\textit{Introduction}.
}


In Camphuysen’s \textit{Stichtelycke Rymen} the song is entitled “\textit{Essex's Lamentation},
or \textit{What if a day}.”

Ritson, in a note to his \textit{Historical Essay on Scotish Song}, p.~57, says, “In a
curious dramatic piece, entitled \textit{Philotus}, printed at Edinburgh in 1603, by way
of finale, is \textit{Ane sang of the foure lufearis }(lovers), though little deserving that
title. It is followed by the old English song, beginning, ‘What if a day, or
a month, or a year?’ alluded to in \textit{Hudibras}, which appears to have been sung at
the end of the play, and was probably, at that time, new and fashionable.”

Mr. Halliwell, in a paper read before the Society of Antiquaries in Dec., 1840,
says, “It is a curious fact that one of the songs in Ryman’s well-known collection
of the fifteenth century, in the Cambridge Public Library, commences—
\settowidth{\versewidth}{Crowne my desyres wythe every delyghte; ’}
\begin{scverse}
\vleftofline{‘}What yf a daye, or nyghte, or howre,\\
Crowne my desyres wythe every delyghte; ’
\end{scverse}
and that in Sanderson’s Diary in the British Museum, MSS. Lansdowne 241,
fol.~49, temp. Elizabeth, are the two first stanzas of the song, more like the copy
in Ryman, and differing in its minor arrangements from the later version.
Moreover, that the tune in Dowland’s Musical Collection, in the Public Library,
Cambridge, is entitled ‘What if a day, or a \textit{night}, or an \textit{hour}?’ agreeing with
Sanderson’s copy.” Mr. Halliwell has reverted to the subject in \textit{Reliquæ Antiquæ},
i. 323, and ii. 123.
\pagebreak


%311
%===================================================================

“What if a day, or a month, or a year?” is mentioned as one of the tunes for
\textit{Psalms and Songs of Sion}, by W[illiam] S[atyer], 1642. See p.~319.

\musicinfo{Rather slow.}{}
\lilypondfile{lilypond/311-what-if-a-day-or-a-month-or-a-year}\normalsize

\pagebreak
%312
%===================================================================

\settowidth{\versewidth}{Th’ earth’s but a point of the world, and a man}
\indentpattern{000011111111}
\begin{dcverse}\begin{patverse}
Th’ earth’s but a point of the world, and a man\\
Is but a point of the earth’s compared centre:\\
Shall then the point of a point be so vain,\\
As to triumph in a silly point’s adventure?\\
All is hazard that we have,\\
Here is nothing biding;\\
Days of pleasure are as streams\\
Through fair meadows gliding.\\
Weal or woe, time doth go,\\
Time hath no returning;\\
Secret Fates guide our states\\
Both in mirth and mourning.
\end{patverse}

\begin{patverse}
What if a smile, or a beck, or a look,\\
Feed thy fond thoughts with many vain conceivings:\\
May not that smile, or that beck, or that look,\\
Tell thee as well they are all but false deceivings?\\
Why should beauty be so proud,\\
In things of no surmounting?\\
All her wealth is but a shroud,\\
Nothing of accounting.\\
Then in this there’s no bliss,\\
Which is vain and idle,\\
Beauty’s flow’rs have their hours,\\
Time doth hold the bridle.
\end{patverse}

\begin{patverse}
What if the world, with a lure of its wealth,\\
Raise thy degree to great place of high advancing;\\
May not the world, by a check of that wealth,\\
Bring thee again to as low despised changing?\\
While the sun of wealth doth shine\\
Thou shalt have friends plenty;\\
But, come want, they repine,\\
Not one abides of twenty.\\
Wealth (and friends), holds and ends,\\
As thy fortunes rise and fall:\\
Up and down, smile and frown,\\
Certain is no state at all.
\end{patverse}

\begin{patverse}
What if a grip, or a strain, or a fit,\\
Pinch thee with pain of the feeling pangs of sickness:\\
May not that grip, or that strain, or that fit,\\
Shew thee the form of thine own true perfect likeness?\\
Health is but a glance of joy,\\
Subject to all changes;\\
Mirth is but a silly toy,\\
Which mishap estranges.\\
Tell me, then, silly man,\\
Why art thou so weak of wit.\\
As to be in jeopardy.\\
When thou mayst in quiet sit?
\end{patverse}
\end{dcverse}

\musictitle{The Hemp-Dresser, or The London Gentlewoman.}

This tune has attained a long-enduring popularity. It is to be found in every
edition of \textit{The Dancing Master}, as well as in many other publications, and is
commonly known at the present day.

The name of \textit{The Hemp-dresser}, or \textit{The London Gentlewoman}, is derived from
an old song which was translated into Latin (together with \textit{Chevy Chace} and many
others) by Henry Bold, and published, after his death, in “Latine Songs with
their English,” 1685.

One of D’Urfey’s songs, commencing, “The sun had loos’d his weary team,”
was written to this air. It is printed, with music, in his third book of songs,
1685; in Playford’s third book of “Choice Ayres and Songs;” and in vol. i.
of all the editions of \textit{Pills to purge Melancholy}. In the first, it is entitled “A new
song set to a pretty country dance, called \textit{The Hemp-dresser}:” in the second, it
has the further prefix of “The Winchester Christening; The Sequel of the
Winchester Wedding. A new song,” \&c.

In \textit{The Beggars’ Opera}, 1728; \textit{The Court Legacy}, 1733; \textit{The Sturdy Beggars},
1733; and \textit{The Rival Milliners}, 1737, the time is named “The sun had loos’d
his weary team,” from D’Urfey’s song. In other ballad-operas, such as \textit{Penelope},
1728; and \textit{Love and Revenge}, or \textit{The Vintner outwitted}, n.d., it takes the name
of one beginning, “Jone stoop’d down.” Burns also wrote a song to it—“The
Deil’s awa wi’ the Exciseman.”
\pagebreak
%313
%===================================================================

In the “History of Robert Powel, the puppet-showman,” 8vo., 1715, \textit{The
Duke of York’s Delight; Welcome home, Old Rowley; The Knot;} and \textit{The Hemp-dressers}, 
are mentioned as favorite tunes called for by the company.

The song of \textit{The Hemp-dresser} consists of four stanzas, of which the two first
are as follows:—

\settowidth{\versewidth}{There was a London gentlewoman}
\indentpattern{01018}
\begin{dcverse}
\begin{patverse}
There was a London gentlewoman\\
That lov’d a country man-a;\\
And she did desire his company\\
A little now and then-a.\\
Fa la, \&c.
\end{patverse}

\begin{patverse}
This man he was a hemp-dresser,\\
And dressing was his trade-a;\\
And he did kiss the mistress, sir,\\
And now and then the maid-a.\\
Fa la, \&c.
\end{patverse}
\end{dcverse}

The first verse of D’Urfey’s song is here printed with the music.

\musicinfo{Gracefully.}{}
\lilypondfile{lilypond/313-the-hemp-dresser-or-the-london-gentlewoman}\normalsize

\musictitle{Since First I Saw Your Face.}

The following tune is by Thomas Ford, one of the musicians in the suite of
Prince Henry, the eldest son of James I. It is a song for one voice to the lute,
or for four without accompaniment, and contained in his \textit{Musicke of sundrie
Kindes} (fol. 1607.) The second part of a popular tune called \textit{Jamaica}, or \textit{My
father was born before me}, bears a resemblance to the second part of this.

In the \textit{Golden Garland of Princely Delight}, third edition, 1620, the song is
entitled, “Love’s Constancy.”
\pagebreak
%314
%===================================================================


Ford was not a great harmonist, but this song (now miscalled a madrigal) has
survived the works of many more learned composers, and is probably as popular
at the present day as when first written. The harmony of the modern copies is
not by Ford.

\musicinfo{Slow.}{}
\lilypondfile{lilypond/314-since-first-i-saw-your-face}\normalsize

\settowidth{\versewidth}{And your sweet beauty, past compare,}
\begin{dcverse}\begin{altverse}
If I admire or praise you too much,\\
That fault you may forgive me;\\
Or if my hands had stray’d to touch,\\
Then justly might you leave me.\\
I ask'd you leave, you bade me love,\\
Is’t now a time to chide me?\\
No, no, no, I’ll love you still,\\
What fortune e’er betide me.
\end{altverse}

\begin{altverse}
The sun, whose beams most glorious are,\\
Rejecteth no beholder;\\
And your sweet beauty, past compare,\\
Made my poor eyes the bolder.\\
When beauty moves, and wit delights,\\
And signs of kindness bind me,\\
There, O there, where’er I go,\\
I’ll leave my heart behind me.
\end{altverse}

\begin{altverse}
{[}If I have wronged you, tell me wherein,\\
And I will soon amend it;\\
In recompense of such a sin,\\
Here is my heart, I’ll send it.\\
If that will not your mercy move,\\
Then, for my life I care not;\\
Then, O then, torment me still,\\
And take my life, and spare not.{]}
\end{altverse}
\end{dcverse}

I have only found the last stanza in late copies, such as \textit{Wit's Interpreter},
third edition, 8vo., 1671.
\pagebreak
%315
%===================================================================

\musictitle{What Care I How Fair She Be?}

A copy of this song is in the Pepys Collection, i. 230, entitled “A new song of
a young man’s opinion of the difference between good and bad women. \textit{To a
pleasant new tune}.” (Printed at London for W. I.) It is also in the second part
of \textit{The Golden Garland of Princely Delights}, third edition, 1620, entitled “The
Shepherd’s Resolution. To the tune of \textit{The Young Man's Opinion}.” As the
name of the tune is here derived from the title of the ballad, it must have been
printed in ballad form before 1620, when it was published among T\textit{he Workes of
Master George Wither}.

The tune is in Heber’s Manuscript (described at p.~204), but, except for the
popularity of the words, it would scarcely be worth preserving. They were afterwards
reset by Mr. King, and are printed to his tune in \textit{ Pills to purge Melancholy}.

The first line of the copy in the Pepys Collection (unlike that in \textit{The Golden
Garland}) is, “Shall I \textit{wrestling} in dispaire.” In the same volume are the
following:—

Page 200.—“The unfortunate Gallant gull’d at London. To the tune of
\textit{Shall I wrastle in despair}.” (Printed for T. L.) Beginning—
\settowidth{\versewidth}{“From Cornwall Mount to London fair.”}
\begin{scverse}
“From Cornwall Mount to London fair.”
\end{scverse}

Page 316.—“This maid would give tenne shillings for a kisse. To the tune
of \textit{Shall I wrassle in despair}.” (Printed at London by I. White.) Beginning—
\begin{scverse}
“You young men all, take pity on me.”
\end{scverse}

Page 236.—“Jone is as good as my lady. To the tune of \textit{What care I how
fair she be}?” (Printed at London for A. M[ilbourn].) Beginning—
\begin{scverse}
“Shall I here rehearse the story.”
\end{scverse}

The following (which has been attributed, upon insufficient evidence, to Sir
Walter Raleigh) is in the same metre, and has the same burden as George
Wither’s~song:—

\settowidth{\versewidth}{Calling home the smallest part}
\indentpattern{00000011}
\begin{dcverse}
\begin{patverse}
Shall I, like an hermit, dwell\\
On a rock or in a cell?\\
Calling home the smallest part\\
That is missing of my heart,\\
To bestow it where I may\\
Meet a rival every day?\\
If she undervalues me,\\
What care I how fair she be.
\end{patverse}

\begin{patverse}
Were her tresses angel-gold;\\
If a stranger may be bold,\\
Unrebuked, unafraid,\\
To convert them to a braid,\\
And, with little more ado,\\
Work them into bracelets too;\\
If the mine be grown so free,\\
What care I how rich it be.
\end{patverse}

\begin{patverse}
Were her hands as rich a prize\\
As her hairs or precious eyes;\\
If she lay them out to take\\
Kisses, for good manners sake;\\
And let every lover skip\\
From her hand unto her lip;\\
If she seem not chaste to me.\\
What care I how chaste she be.
\end{patverse}

\begin{patverse}
No, she must be perfect snow,\\
In effect as well as show,\\
Warming but as snow-balls do,\\
Not, like fire, by burning too;\\
But when she by chance hath got\\
To her heart a second lot;\\
Then, if others share with me,\\
Farewell her, whate’er she be.
\end{patverse}
\end{dcverse}
\pagebreak
%316
%===================================================================

\musicinfo{Moderate time.}{}
\lilypondfile{lilypond/316-what-care-i-how-fair-she-be}\normalsize

\begin{dcverse}
\begin{patverse}
Shall my foolish heart be pin’d,\\
’Cause I see a woman kind?\\
Or a well-disposed nature,\\
Joined with a lovely feature?\\
Be she kind, or meeker than\\
Turtle-dove or pelican;\\
If she be not so to me,\\
What care I how kind she be.
\end{patverse}

\begin{patverse}
Shall a woman’s virtues move\\
Me to perish for her love?\\
Or, her well-deservings known,\\
Make me quite forget mine own?\\
Be she with that goodness blest,\\
Which may gain her name of Best;\\
If she be not so to me,\\
What care I how good she be.
\end{patverse}

\begin{patverse}
’Cause her fortune seems too high,\\
Shall I play the fool, and die?\\
He that bears a noble mind\\
If no outward help he find,\\
Think what with them he would do,\\
That without them dares to woo:\\
And, unless that mind I see,\\
What care I how great she be.
\end{patverse}

\begin{patverse}
Great, or good, or kind, or fair,\\
I will ne’er the more despair:\\
If she love me, this believe,\\
I will die ere she shall grieve.\\
If she slight me when I woo,\\
I can slight and let her go:\\
If she be not fit for me,\\
What care I for whom she be.
\end{patverse}
\end{dcverse}
\pagebreak
%317
%===================================================================


\musictitle{The New Royal Exchange.}

In \textit{The Dancing Master} of 1665 there are two tunes under very similar titles.
The first is \textit{The New Exchange}; the second, \textit{The New New-Exchange}. The first
is sometimes called \textit{Durham Stable};\dcfootnote{\scriptsizer
Strype, in his edition of Stow’s \textit{London}, book vi., p.~75,
says “In the place where certain old stables stood, belonging
to this house [Durham House], is the New Exchange;
being furnished with shops on both sides the walls, both
below and above stairs, for milliners, sempstresses, and
other trades, and is a place of great resort and trade for
the nobility and gentry, and such as have occasion for
such commodities.” It was opened April 11th, 1609, in
the presence of James I. and his Queen, and taken down
in 1737. Coutts’ Banking House now stands upon the
site. Pepys, in his Diary, 15th April, 1662, says,‘“With
my wife by coach to the New Exchange, to buy her some
things; where we saw some new-fashion pettycoats of
sarcenet, with a black broad lace printed round the bottom
and before; very handsome, and my wife had a mind to
one of them.”
}
 the second, which was more frequently
used as a ballad tune, is, in other editions, named \textit{The New Royal Exchange}.

In \textit{Wit and Drollery}, 1656, p.~110, is a song to this tune—“On the Souldiers
walking in the new Exchange to affront the Ladies.” It consists of four stanzas,
the first of which is here printed with the music.

In the same book, at p.~60, is another song of six stanzas beginning—

\vspace{-0.5\baselineskip}

\settowidth{\versewidth}{We’ll go no more to Tunbridge Well,}
\begin{dcverse}\footnotesizer\begin{altverse}
\vleftofline{“}We’ll go no more to Tunbridge Wells,\\
The journey is too far;\\
Nor ride in Epsom waggon, where\\
Our bodies jumbled are.\\
But we will all to the westward waters go,\\
The best that e’er you saw,\\
And we will have them henceforth call’d\\
The Kentish new-found Spa.\\
Then go, lords and ladies, whate’er you ail;\\
Go thither all that pleases;\\
For it will cure you, without fail,\\
Of old and new diseases.”
\end{altverse}
\end{dcverse}

%\vspace{-0.5\baselineskip}

In \textit{Westminster Drollery}, part ii, 1671, is a third song, “to the tune of \textit{I'll go
no more to the New Exchange};” beginning—

\vspace{-0.5\baselineskip}

\begin{dcverse}\footnotesizer\begin{altverse}
\vleftofline{“}Never will I wed a girl that's coy,\\
Nor one that is too free;\\
But she alone shall be my joy\\
That keeps a mean\dcfootnote{\scriptsizer
Mean, \ie, a middle course; the mean being the intermediate
part, or parts, between the treble and tenor. If
there were two means, as in the lute, the lower was called
the greater: the upper, the lesser mean.}
 to me.\\
For, if too coy, then I must court\\
For a kiss as well as any;\\
And if too free, I fear o’ th’ sport\\
I then may have too many,” \&c.
\end{altverse}
\end{dcverse}

%\vspace{-0.5\baselineskip}

In \textit{Wit Restored, in severall select Poems, not formerly publisht}, 1658, there are
two songs, \textit{The Burse of Reformation}, and \textit{The Answer}. The first commencing—

\vspace{-0.5\baselineskip}

\begin{dcverse}\footnotesizer\begin{altverse}
\vleftofline{“}We will go no more to the Old Exchange,\\
There’s no good ware at all;\\
Their bodkins, and their thimbles, too,\\
Went long since to Guildhall.\\
But we will go to the New Exchange,\\
Where all things are in fashion;\\
And we have it henceforth call’d\\
The Burse of Reformation.\\
Come, lads and lasses, what do you lack?\\
Here is ware of all prices;\\
Here’s long and short, here’s wide and straight;\\
Here are things of all sizes.
\end{altverse}
\end{dcverse}

\vspace{-0.5\baselineskip}

\noindent and the Answer—

\vspace{-0.5\baselineskip}

\begin{dcverse}\footnotesizer\begin{altverse}
\vleftofline{“}We will go no more to the New Exchange,\\
Their credit’s like to fall,\\
Their money and their loyalty\\
Is gone to Goldsmiths’ Hall.\dcfootnote{\scriptsizer
The place appointed for the reception of fines imposed
upon the Royalists; and for loans, etc., to the Puritanic
party.}\\
But we will keep our Old Exchange,\\
Where wealth is still in fashion,\\
Gold chaines and ruffes shalt beare the bell,\\
For all your reformation.\\
Look on our walls, and pillars too,\\
You’ll find us much the sounder:\\
Sir Thomas Gresham stands upright,\\
But Crook-back was your founder.”
\end{altverse}
\end{dcverse}

%\vspace{-0.5\baselineskip}

These have been reprinted in “Satirical Songs and Poems on Costume,” for the
Percy Society, by F. W. Fairholt, F.S.A. \pagebreak Another equally curious song for the
%318
%===================================================================
manners and fashions of the day, is “The New Exchange,” in \textit{Merry Drollery
Complete}, 1670, p.~134; commencing—

%\normalsize


\settowidth{\versewidth}{I'll go no more to the Old Exchange,}
\begin{dcverse}\begin{altverse}
\vleftofline{“}I'll go no more to the Old Exchange,\\
There's no good ware at all;\\
But I will go to the New Exchange,\\
Call’d Haberdashers’ Hall:\\
For there are choice of knacks and toys,\\
The fancy for to please;\\
For men and maids, for girls and boys,\\
And traps to catch the fleas.
\end{altverse}

\begin{altverse}
There you may buy a Holland smock,\\
That’s made without a gore,” \&c.
\end{altverse}
\end{dcverse}

\musicinfo{Lively.}{}
\lilypondfile{lilypond/318-the-new-royal-exchange}\normalsize

\pagebreak

%319===================================================================
\DFNsingle

\musictitle{The Fairest Nymph The Valleys.}{}

This, like \textit{In sad and ashy weeds} (p.~202), or like \textit{Fear no more the heat of the
sun}, in Shakespeare’s \textit{Cymbeline}, is a sort of dirge, a mourning or funeral song.
The copy in the Roxburghe Collection, i. 330, is entitled “The Obsequy of
Faire Phillida: with the Shepherds’ and Nymphs’ Lamentation for her losse.
\textit{To a new court tune}.” The music is contained in a MS. volume of virginal
music transcribed by Sir John Hawkins, and in Starter’s \textit{Friesche Lust-Hof},
1634, under its English name. In the library of the British Museum there is a
copy of “Psalmes or Songs of Sion, turned into the language and set to the tunes
of a Strange Land, by W[illiam] S[latyer], intended for Christmas Carols, and
fitted to divers of the most noted and common, but solemne tunes, every where
in this land familiarly used and knowne.” 1642. Upon this copy a former
possessor had written the names of the tunes to which they were designed to be
sung. These are, \textit{The fairest Nymph the valleys; All in a garden green; Bara
Faustus’ Dreame; Crimson velvet; What if a day, or a month, or a year? Fair
Angel of England; Dulcina; Walsingham}; and \textit{Jane Shore}.\footnote
{\centering All the tunes here named will be found in this Collection.}

\musicinfo{With expression.}{}
\lilypondfile{lilypond/319-the-fairest-nymph-the-valleys}\normalsize

\pagebreak
\DFNdouble
%320===================================================================

\settowidth{\versewidth}{Come, fatal sisters, leave your spools,}
\indentpattern{0122101221011221}
\begin{dcverse}
\begin{patverse}
The sheep for woe go bleating,\\
That they their goddess miss,\\
And sable ewes,\\
By their mourning, shew\\
Her absence, cause of this.\\
The nymphs leave off their dancing,\\
Pan’s pipe of joy is cleft,\\
For great his grief,\\
He shunneth all relief,\\
Since she from him is reft.\\
Come, fatal sisters, leave your spools,\dcfootnote{\textit{}
A spool to wind yarn upon.}\\
Leave ‘weaving’ altogether,\\
That made this flower to wither.\\
Let envy, that foul vipress,\\
Put on a wreath of cypress,\\
Sing sad dirges altogether.
\end{patverse}

\begin{patverse}
Diana was chief mourner\\
At these sad obsequies,\\
Who with her train\\
Went tripping o’er the plain,\\
Singing doleful elegies.\\
Menalchus and Amintas,\\
And many shepherds moe,\dcfootnote{\textit{}
More.}\\
With mournful verse,\\
Did all attend her hearse,\\
And in sable saddles go.\\
Flora, the goddess that us’d to beautify\\
Fair Phillis’ lovely bowers\\
With sweet fragrant flowers,\\
Now her grave adorned,\\
And with flowers mourned,\\
Tears thereon in vain she pours.
\end{patverse}

\begin{patverse}
Venus alone triumphed\\
To see this dismal day,\\
Who did despair\\
That Phillida the fair\\
Her laws would ne’er obey.\\
The blinded boy his arrows\\
And darts were vainly spent;\\
Her heart, alas,\\
Impenetrable was,\\
And to love would ne’er assent.\\
At which affront, Citharea repining,\\
Caus’d Death with his dart\\
To pierce her tender heart;\\
But her noble spirit\\
Doth such joys inherit,\\
‘As’ from her shall ne’er depart.
\end{patverse}
\end{dcverse}

\musictitle{Hunting The Hare.}

\settowidth{\versewidth}{Was al his lust, for no cost wolde he spare.”}
\begin{scverse}
\vleftofline{“}Of prikyng and of hunting for the Hare\\
Was al his lust, for no cost wolde he spare.”\\
\attribution \textit{Chaucer’s Description of a Monk}.
\end{scverse}

Hunting has always been so favorite an amusement with the English, that the
great variety of songs upon the subject will excite no surprise. Those I have
printed, of the reign of Henry VIII., relate either to deer or fox-hunting; but
Henry was no less careful of the minor sport, as may be seen by an act of
Parliament (passed anno 14-15 of his reign), entitled “An Act concerning
the Hunting of the Hare.” It recites that, “For as muche as oure Soveraigne
Lorde the Kinge, and other noblemen of this realme, before this time hath
used and exercised the game of huntynge the hare, for their disporte and
pleasure, which game is now decayed and almost utterly dystroied for that
divers parties of this realme, by reason of the \textit{trasinge in the snow}, have killed
and destroied, and dayly do kille and distroy the same hares, by fourteen or sixteen
upon a daye, to the dyspleasure of our Soveraigne Lorde the Kinge and
other nobleman,” \&c.; therefore the act fixes a penalty of six shillings and eightpence
(a large sum in comparison with the value of the hares in those days) for 
every one so killed. Henry seems, also, \pagebreak to have considered the sale of hunting-horns 
%321===================================================================
of sufficient importance, as a source of revenue, to affix an export duty of
four shillings per dozen upon them.\dcfootnote{\textit{}
This will be found in “The Rates of the Custome
House, both inwarde and outwarde, very necessarye
for all Merchantes to knowe. Imprinted at London, by
me, Rycharde Kele, dwellynge at the longe shoppe in the
Poultrye, under Saynt Myldreds Churche.” 1545. Among
the \textit{import} duties relating to music, will be found—
“Clarycordes, the payre, 2\textit{s}.; Harpe Strynges, the boxe,
10\textit{s}.; Lute Strynges, called Mynikins, the groce, 22\textit{d}.;
Orgons, the payre, \textit{ut sint in valore}; Wyer for Clarycordes,
the pound, 4\textit{d}.; Virginales, the payre, 3\textit{s}. 4\textit{d}.;
Whisteling Bellowes, the groc, 8\textit{s.}}


“A Songe of the huntinge and killinge of the Hare” was entered on the
registers of the Stationers’ Company, to Richard Jones, on June 1, 1577, but the
entry contains no clue to the words, or to the air.

The tune of the present song may be traced back to the reign of James I.;
but, both in his reign, and in that of his predecessor, hunting was so favorite a
sport, and hunting songs so generally popular, that the introduction of either on
the stage was thought a good means of assisting the success of a play.

Wood tells us that in Richard Edwardes’ comedy of \textit{Palæmon and Arcyte}
(which was performed before Queen Elizabeth, in Christ Church Hall, Oxford, on
the 2nd and 3rd September, 1566) “A cry of hounds was acted in the quadrant
upon the train of a fox, in the hunting of Theseus; with which the young
scholars, who stood in the remoter part of the stage and windows, were so much
taken and surprised, supposing it to be real, that they cried out, ‘There, there—he’s 
caught, he’s caught! ’ All which the Queen, merrily beholding, said,
‘Oh, excellent! These boys, in very truth, are ready to leap out of the windows
to follow the hounds.’”

James was passionately fond of hunting; and Anthony Munday, in his play,
\textit{The Downfall of Robert, Earl of Huntington}, thus deprecates his displeasure and
that of the audience for not having introduced hunting songs, or resorted to the
other usual expedients to ensure applause. In act iv., sc. 2, Little John says—
\settowidth{\versewidth}{No pleasant skippings up and down the wood;}
\begin{scverse}
\begin{altverse}
\vleftofline{“}Methinks I see no jests of Robin Hood;\\
No merry Morrices of Friar Tuck;\\
No pleasant skippings up and down the wood;\\
No hunting songs; no coursing of the buck.\\
Pray God this play of ours may have good luck,\\
And the King’s Majesty mislike it not.”
\end{altverse}
\end{scverse}

I have printed one song on hare-hunting, of James’ reign (\textit{Master Basse his
Careere}, or \textit{The New Hunting of the Hare}), at p.~256. Another song, entitled
“The Hunting of the Hare, with her last will and testament,
\settowidth{\versewidth}{As it was performed on Bamstead Downs,}
\begin{scverse}
As it was performed on Bamstead Downs,\\
By coney-catchers and their hounds,”
\end{scverse}
was printed by Coles, Vere, and Wright, and will be found in Anthony à Wood’s
Collection. It commences thus—
\settowidth{\versewidth}{Whose echo shall, throughout the sky,}
\begin{scverse}
\vleftofline{“}Of all delights that earth doth yield,\\
Give me a pack of hounds in field,\\
Whose echo shall, throughout the sky,\\
Make Jove admire our harmony,\\
And wish that he a mortal were,\\
To share the pastime we have here.”
\end{scverse}
No tune is indicated in the copy, and \pagebreak it could not have been sung to this air.
%322===================================================================

In \textit{Wit and Drollery}, and in several other publications, is a song, entitled
\textit{The Hunt}, commencing—
\settowidth{\versewidth}{Sweet is the breath, and fresh is the earth}
\begin{scverse}
\begin{altverse}
\vleftofline{“}Clear is the air, and the morning is fair,\\
Fellow huntsmen, come wind me your horn;\\
Sweet is the breath, and fresh is the earth\\
That melteth the rime from the thorn.”
\end{altverse}
\end{scverse}
\textit{Hunting the Hare} is also in the list of the songs and ballads printed by William
Thackeray, at the Angel in Duck Lane, in the early part of the reign of
Charles II., and it is, in all probability, the song to this tune (commencing—
\settowidth{\versewidth}{“Songs of shepherds, and rustical roundelays”),}
\begin{scverse}
“Songs of shepherds, and rustical roundelays”),
\end{scverse}
because the tune was then popular, and the words are to be found near that time
in \textit{Westminster Drollery}, part ii. (1672); as well as afterwards in \textit{Wit and
Drollery},~1682; in the \textit{Collection of Old Ballads}, 8vo., 1727; in \textit{Miscellany
Poems}, edited by Dryden, iii. 309 (1716); in Ritson’s, Dale’s, and other
Collections of English~Songs.

The first copy of the tune that I have discovered is in Playford’s \textit{Musick’s
Recreation on the Lyra Viol}, 1652; the second is in \textit{Musick’s Recreation on the
Viol, Lyra-way}, 1661. In both publications it is entitled \textit{Room for Cuckolds}.

Pennant speaking of Rychard Middleton (father of Sir Hugh Middleton), says,
“Thomas, the fourth son, became Lord Mayor of London, and was the founder of
the family of Chirk Castle. It is recorded that having married a young wife in
his old age, the famous song of \textit{Room for Cuckolds, here comes my Lord Mayor}!
was invented on the occasion.”—\textit{Pennant’s Tours in Wales}, ii. 152 (1810).
Thomas Middleton was Lord Mayor of London in 1614. Pennant gives the
Sebright MSS. as his authority for the anecdote.

In the Pepys Collection, i. 60, will be found, “A Scourge for the Pope;
satyrically scourging the itching sides of this obstinate brood, in England. To
the tune of \textit{Room for Cuckolds}.” It is one of Martin Parker’s early songs:
“Printed by John Trundle, at his shop in Smithfield,” and signed, “Per me,
Martin Parker.” Another song, which bears this title of the tune, is contained
in vol. xvi. of the King’s Pamphlets Brit. Mus., and dated in MS., 1659. It is
also quoted, by the same name, in \textit{Folly in print, or A Book of Rhymes}, 1667, in
the song, “Away from Romford, away, away.”

A third, and perhaps the earliest name for the air, is \textit{Room for Company};
apparently derived from a ballad in the Pepys Collection, i. 168, entitled and
commencing, “Room for Company, here comes good fellowes. \textit{To a pleasant new
tune}.” Imprinted at London for E. W. This was perhaps Edward White, a
ballad-printer of Elizabeth’s reign, and of the earliest part of that of James I.

In \textit{ Pills to purge Melancholy}, vi. 136, there is a song about the twelve great
Companies of the city of London, printed to this tune, and commencing—
\settowidth{\versewidth}{Room for gentlemen, here comes my Lord Mayor.”}
\begin{scverse}
\vleftofline{“}Room for gentlemen, here comes my Lord Mayor.”
\end{scverse}

In the Roxburghe Collection, i. 538, is, “The fetching home of May; or—
\begin{scverse}
\vleftofline{“}A pretty new ditty, wherein is made known,\\
How each lass doth strive for to have a green gown.
\end{scverse}

To the tune of \textit{Room for Company}.” \pagebreak Printed for J. Wright, jun., dwelling
%323===================================================================
at the upper end of the Old Bailey (about 1663). It is also contained in the \textit{Antidote to Melancholy}, 1661; and in \textit{ Pills to purge Melancholy}, ii. 26 (1707),
or iv. 26 (1719).

The first stanza is subjoined, with the earlier version of the tune.

\musicinfo{Smoothly, and in moderate time.}{}
\lilypondfile{lilypond/323-hunting-the-hare}\normalsize

In the \textit{Antidote to Melancholy}, and in \textit{ Pills to purge Melancholy}, the above song
is printed under the title of \textit{The Green Grown}, a name derived from the last line of 
each stanza of the song. In \textit{Musick à-la-Mode}; \pagebreak or \textit{The young Maid’s Delight:
%324===================================================================
containing five excellent new songs sung at the Drolls in Bartholomew Fair}, 1691,
there is another song, under the name of \textit{The Green Grown}, “to an excellent playhouse
tune.”

The tune of \textit{Hunting the Hare} is now in common use for comic songs, or for
such as require great rapidity of utterance; but it has also been employed as a
slow air. For instance, in Gay’s ballad-opera of \textit{Achilles}, 1733, it is printed
in \timesig{3}{4} time, and entitled “A Minuet.”

\musicinfo{Fast.}{Hunting the Hare.}
\lilypondfile[staffsize=16]{lilypond/324-hunting-the-hare-2}\normalsize

\settowidth{\versewidth}{The earth old and ample, they soon leave the air;}
\indentpattern{0101330330}
\begin{dcverse}\begin{patverse}
Stars quite tir’d with pastimes Olympical,\\
Stars and planets which beautiful shone,\\
Could no longer endure that men only shall\\
Swim in pleasures, and they but look on;\\
Round about horned\\
Lucina they swarmed,\\
And her informed how minded they were,\\
Each god and goddess,\\
To take human bodies,\\
As lords and ladies, to follow the hare.
\end{patverse}
\columnbreak

\begin{patverse}
Chaste Diana applauded the motion,\\
While pale Proserpina sat in her place,\\
To light the welkin, and govern the ocean,\\
While she conducted her nephews in chase:\\
By her example,\\
Their father to trample,\\
The earth old and ample, they soon leave the air;\\
Neptune the water,\\
And wine Liber Pater,\\
And Mars the slaughter, to follow the hare.
\end{patverse}
\end{dcverse}
\pagebreak
%325===================================================================


\settowidth{\versewidth}{Light god Cupid was mounted on Pegas}
\begin{dcverse}\footnotesize\begin{patverse}
Light god Cupid was mounted on Pegasus,\\
Lent by the Muses, by kisses and pray’rs;\\
Strong Alcides, upon cloudy Caucasus,\\
Mounts a centaur, which proudly him bears;\\
Postilion of the sky,\\
Light-heeled Mercury\\
Soon made his courser fly, fleet as the air;\\
Tuneful Apollo,\\
The kennel did follow,\\
And whoop and halloo, boys, after the hare.
\end{patverse}

\begin{patverse}
Drown’d Narcissus from his metamorphosis,\\
Rous’d by Echo, new manhood did take;\\
Snoring Somnus upstarted from Cimmeris,\\
Before, for a thousand years, he did not wake;\\
There was club-footed \\
Mulciber booted,\\
And Pan promoted on Corydon’s mare;\\
Proud Pallas pouted,\\
Loud Æolus shouted,\\
And Momus flouted, yet followed the hare.
\end{patverse}

\begin{patverse}
Hymen ushers the lady Astræa,\\
The jest took hold of Latona the cold;\\
Ceres the brown, with bright Cytherea;\\
Thetis the wanton, Bellona the bold;\\
Shame-fac’d Aurora,\\
With witty Pandora,\\
And Maia with Flora did company hear;\\
But Juno was stated\\
Too high to be mated,\\
Although she hated not hunting the hare.
\end{patverse}

\begin{patverse}
Three brown bowls to th’ Olympical rector,\\
The Troy-born boy presents on his knee;\\
Jove to Ph\oe bus carouses in nectar,\\
And Phœbus to Hermes, and Hermes to me;\\
Wherewith infused, \\
I piped and I mused,\\
In language unused, their sports to declare:\\
Till the house of Jove\\
Like the spheres did move:—\\
Health to those who love hunting the hare!
\end{patverse}
\end{dcverse}

\vspace{-0.75\baselineskip}

\musictitle{The Crossed Couple.}

\vspace{-0.75\baselineskip}

This tune is referred to under three names, viz,, \textit{The Crossed Couple}, \textit{Hyde
Park}, and \textit{Tantara rara tantivee}.

The ballad of “The Crost Couple: to a new Northern tune much in fashion,”
is in the Roxburghe Collection, ii. 94. In the same volume, at p.~379, is “News
from Hide Park,” \&c., “to the tune of \textit{The Crost Couple}.”

The burden of “News from Hide Park” (as will be seen by the verse printed
below with the music) is \textit{Tantara rara tantive}e; and in the Bagford Collection
(p.~170), the tune is quoted under that name, in “A pleasant Dialogue betwixt
two wanton Ladies of Pleasure; or, The Duchess of Portsmouth’s woful farewell
to her former felicity.” This ballad is a supposed conversation between Nell
Gwyn and Louise Renée de Penencourt de Quérouaille (vulgarly, Madame
Carwell), whom Charles II. created Duchess of Portsmouth.

Nell Gwynn was as popular with the ballad-singers, from her many redeeming
qualities, as the Duchess of Portsmouth (being a Roman Catholic, and supposed
to send large sums of money to her relations in France) was out of favour with
them.\dcfootnote{\textit{}
On the following page, in the same collection, there
is another Dialogue hetween the Duchess of Portsmouth
and Nell Gwyn, on the supposed intention of the former
to retire to France with the money she had acquired. It
is entitled, “Portsmouth’s Lamentation: Or a Dialogue
between two amorous Ladies, E. G. and D. P.
\settowidth{\versewidth}{Dame Portsmouth was design’d for France}
\begin{fnverse}
\vin \vleftofline{“}Dame Portsmouth was design’d for France\\
\vin But therein was prevented;\\
\vin Who mourns at this unhappy chance,\\
\vin And sadly doth lament it.\\
\vin\vin To the tune of \textit{Tom the Taylor}, or \textit{Titus Oates}’’
\end{fnverse}

It commences thus:—
\begin{fnverse}
\vin\vleftofline{“}I prithee, Portsmouth, tell me plain,\\
\vin\vin  Without dissimulation,\\
\vin  When dost thou home return again,\\
\vin\vin  And leave this English nation?\\
\vin Your youthful days are past and gone,\\
\vin\vin  You plainly may perceive it,\\
\vin Winter of age is coming on,\\
\vin\vin ’Tis true—you may believe it.”
\end{fnverse}
Nine stanzas, “Printed for C. Dennisson, at the Stationers
Arms, within Aldgate.”}
 The ballad commences thus:—
 
 %\backskip{0.5}
 
\settowidth{\versewidth}{Of a pleasant discourse that I heard at Pell-Mell,}
\begin{scverse}\begin{altverse}
“Brave gallants, now listen, and I will you tell,\\
With a fa la la, la fa, la la,\\
Of a pleasant discourse that I heard at Pell-Mell,\\
With a fa la la, la fa, la la, \&c.
\end{altverse}
\end{scverse}

\pagebreak
%326===================================================================


The ballad of \textit{News from Hide Park} is also printed, with the tune, in \textit{Pills to
purge Melancholy}, ii. 138 (1700 and 1707). Cunningham, in his \textit{Hand-book of
London}, says of Hyde Park:—“In 1550, the French Ambassador hunted there
with the King; in 1578, the Duke Casimer ‘killed a barren doe with his piece,
in Hyde Park, from amongst 300 other deer.’ In Charles the First’s reign, it
became celebrated for its foot and horse races round the Ring; in Cromwell’s
time, for its musters and coach races; in Charles the Second’s reign, for its drives
and promenades—a reputation which it still retains.” (Edit. 1850, p.~241.)
This ballad was printed in the reign of Charles II. The following are the three
first stanzas.

\musicinfo{Gaily.}{}
\lilypondfile[staffsize=15]{lilypond/326-the-crossed-couple}\normalsize

\settowidth{\versewidth}{The Park shone brighter than the skies,}
\indentpattern{01010001}
\begin{dcverse}\begin{patverse}
The Park shone brighter than the skies,\\
Sing tantara rara tantivee,\\
With jewels, and gold, and ladies’ eyes,\\
That sparkled and cried, “Come see me;”\\
Of all parts of England Hyde Park hath the name\\
For coaches, and horses, and persons of fame;\\
It look’d, at first sight, like a field full of flame,\\
Which made me ride up tantivee.
\end{patverse}

\begin{patverse}
There hath not been such a sight since Adam’s,\\
For perriwig, ribbon, and feather;\\
Hyde Park may be termed the market for madams,\\
Or lady-fair, choose yon whether. \\
Their gowns were a yard too long for their legs,\\
They show’d like the rainbow cut into rags,\\
A garden of flowers, or navy of flags,\\
When they did all mingle together.
\end{patverse}
\end{dcverse}

Another tune called \textit{Hide Park} is to be found in the earliest editions of \textit{The
Dancing Master}, and there are ballads in a different metre, such as “A new ditty
of a Lover, tost hither and thither, that cannot speak his mind when they are
together,” by Peter Lowberry (Roxburghe, i. 290); commencing thus:—
\pagebreak
%327===================================================================
\settowidth{\versewidth}{Alas! I am in love,}
\begin{dcverse}\footnotesizerrr\begin{altverse}
\vleftofline{“}Alas! I am in love,\\
And cannot speak it;\\
My mind I dare not move,\\
Nor ne’er can break it.
\end{altverse}

\begin{altverse}
She doth so far excel\\
All and each other,\\
My mind I cannot tell,\\
When we’re together.”
\end{altverse}
\end{dcverse}

In the Pepys Collection, i. 197, is a ballad, “The Defence of Hide Parke from
some aspersions cast upon her, tending to her great dishonour: \textit{To a curious new
Court tune}” It is in ten-line stanzas, and commences, “When glistering Phœbus.”
“Printed at London for H[enry] G[osson].” Also, at i. 188, “The praise of
London: or, A delicate new Ditty, which doth invite you to faire London City.
To the tune of the \textit{second part of Hide Parke}.”

In \textit{Westminster Drollery}, 1671, there is another song called “\textit{Hide Park}: the
tune, \textit{Honour invites you to delights—Come to the Court, and be all made Knights};”
commencing—

\settowidth{\versewidth}{Come, all you noble,}
\begin{scverse}\footnotesizerrr\vleftofline{“}Come, all you noble,\\
You that are neat ones,” \&c.
\end{scverse}

A copy of the ballad, \textit{Come to the Court, and be all made Knights}, will be found in
Addit. MSS., Brit. Mus., No. 5,832, fol. 205, entitled “Verses upon the Order
for making Knights of such persons who had 40\textit{l}. per annum, in King James
the First’s time.” Both James I. and Charles I. resorted to this obnoxious expedient
for raising money. According to John Philipot, Somerset Herald, in his
\textit{Perfect Collection or Catalogue of all Knights Batchelours made by King James,
since his coming to the Crown of England}, 1660, James I. created 2,323 Knights,
of whom 900 were made the first year of his reign.
\settowidth{\versewidth}{Carters, ploughmen, hedgers, and all;}
\begin{dcverse}\scriptsizerrr
\indentpattern{01010010}
\begin{patverse}
\vleftofline{“}Come all you farmers out of the country,\\
Carters, ploughmen, hedgers, and all;\\
Tom, Dick, and Will, Ralph, Roger, and Humphrey,\\
Leave off your gestures rusticall.\\
Bid all your home-spun russets adieu,\\
And suit yourselves in fashions new;\\
\textit{Honour invites you to delights—\\
Come all to Court, and be made Knights.}
\end{patverse}

\begin{patverse}
He that hath forty pounds per annum\\
Shall be promoted from the plough;\\
His wife shall take the wall of her grannum,\\
Honour is sold so dog-cheap now.\\
Though thou hast neither good birth nor breeding.\\
If thou hast money thou’rt sure of speeding.\\
\textit{Honour invites you}, \&c.
\end{patverse}

\begin{patverse}
Knighthood, in old time, was counted an honour,\\
Which the blest spirits did not disdain;\\
But now it is used in so base a manner,\\
That it’s no credit, but rather a stain.\\
Tush, it’s no matter what people do say,\\
The name of a Knight a whole village will sway.\\
\textit{Honour invites you}, \&c.
\end{patverse}
\columnbreak

\begin{patverse}
Shepherds, leave singing your pastoral sonnets,\\
And to learn compliments shew your endeavours;\\
Cast off for ever your two shilling bonnets,\\
Cover your coxcombs with three pound beavers.\\
Sell cart and tar-box, new coaches to buy,\\
Then, ‘Good, your worship,’ the vulgar will cry.\\
\textit{Honour invites you}, \&c.
\end{patverse}

\begin{patverse}
And thus unto worship being advanced,\\
Keep all your tenants in awe with your frowns,\\
And let your rents be yearly enhanced,\\
To buy your new-moulded madams new gowns.\\
Joan, Siss, and Nell, shall all be ladyfied,\\
Instead of hay-carts, in coaches shall ride.\\
\textit{Honour invites you}, \&c.
\end{patverse}

\begin{patverse}
Whatever you do, have a care of expences;\\
In hospitality do not exceed;\\
Greatness of followers belongeth to princes,\\
A coachman and footman are all that you need.\\
And still observe this—Let your servants meat lack,\\
To keep brave apparel upon your wife’s back.\\
\textit{Honour invites you},” \&c
\end{patverse}.
\end{dcverse}

\pagebreak
%328===================================================================


Another version of this ballad is printed in the Rev. Joseph Hunter’s \textit{History
of Sheffield} (p.~104), from “a small volume of old poetry in the Wilson Collections.”
It is there entitled, “Verses on account of King \textit{Charles the First} raising
money by Knighthood, 1630.” Shepherds are said to wear ten-penny, instead of
“two shilling,” bonnets in that version; and it has the following concluding
stanza;—
\settowidth{\versewidth}{Now to conclude and shut up my sonnet,}
\begin{scverse}
\begin{altverse}
\vleftofline{“}Now to conclude and shut up my sonnet,\\
Leave off the cart, whip, hedge-hill, and flail;\\
This is my counsel, think well upon it,\\
Knighthood and honour are now put to sale.\\
Then make haste quickly, and let out your farms,\\
And take my advice in blazing your arms.\\
\vin\vin\vin \textit{Honour invites you},” \&c.
\end{altverse}
\end{scverse}

The above would suit the tune of \textit{Hunting the Hare}.

\musictitle{New Mad Tom Of Bedlam, or Mad Tom.}

The earliest printed copy hitherto discovered of the music of this celebrated
song, which retains undiminished popularity after a lapse of more than two centuries, 
is to be found in the first edition of \textit{The English Dancing Master}, 1650-51.
This is one of the earliest known publications by Playford, before whose time music
was sparingly printed, and small pieces, such as songs, ballad and dance tunes, or
lessons for the virginals, were chiefly to be bought in manuscript, as they are in
many parts of Italy at the present time. In the first edition of \textit{The Dancing
Master} the tune is called \textit{Gray's-Inne Maske}, and in later editions (for instance,
the fourth, printed in 1670) \textit{Gray's-lnne Maske}; or, \textit{Mad Tom}. The blackletter
copies of the ballad, in the Pepys Collection (i. 502); in the Bagford
(643, m. 9, p.~52); and the Roxburghe (i. 299), are entitled \textit{New Mad Tom of
Bedlam}; or,—
\settowidth{\versewidth}{The Man in the Moone drinks claret}
\begin{scverse}
\vleftofline{“}The Man in the Moone drinks claret,\dcfootnote{\textit{}
The ballad is usually printed with another, which is also
entitled “The New Mad Tom; or, The Man in the Moon
drinks Claret, as it was lately sung at the Curtain, Holywell, 
to the same tune.” The Curtain Theatre (according
to Malone and Collier) was in disuse at the commencement
of the reign of Charles I. (1625). This ballad has
three long verses, in the same measure, and evidently intended
to be sung to the same music. The first is as
follows:—
\settowidth{\versewidth}{Bacchus, the father of drunken nowls,}
\indentpattern{00000000101220220112212200000}
\begin{fnverse}
\vleftofline{“}Bacchus, the father of drunken nowls,\\
\begin{patverse}
Full mazers, beakers, glasses, bowls,\\
Greezie flap-dragons, Flemish upsie freeze,\\
With health stab’d in arms upon naked knees;\\
Of all his wines he makes you tasters.\\
So you tipple like bumbasters;\\
Drink till you reel, a welcome he doth give;\\
O bow the boon claret makes you live;\\
Not a painter purer colours shows\\
Then what’s laid on by claret.\\
Pearl and ruby doth set out the nose,\\
When thin small beer doth mar it;\\
Rich wine is good,\\
It heats the blood,\\
It makes an old man lusty.\\
The young to brawl,\\
And the drawers up call,\\
Before being too much musty.\\
Whether you drink all or little,\\
Pot it so yourselves to wittle;\\
Then though twelve\\
A clock it be,\\
Yet all the way go roaring.\\
If the band\\
Of bills cry stand,\\
Swear that you must a ---\\
Such gambols, such tricks, such fegaries,\\
We fetch though we touch no canaries;\\
Drink wine till the welkin roars,\\
And cry out a --- of your scores.”
\end{patverse}
\end{fnverse}}\\
With powder'd beef, turnip, and carret,” \&c.\\
\attribution “The tune is \textit{Gray's-Inn Maske}”
\end{scverse}
% \pagebreak% moved as most of this footnote is on the next page

It was formerly the custom of gentlemen of the Inns of Court to hold revels
four times a year,\dcfootnote{\textit{}
\looseness=-1 Another curious custom, of obliging lawyers to \textit{dance}
four times a year, is quoted from Dugdale by Sir John
Hawkins. (\textit{History of Music}, vol. ii., p.~137.) “It is not
many years since the judges, in compliance with ancient
custom, danced annually on Candlemas-day. And, that
nothing might be wanting for their encouragement in this
excellent study (the law), they have very anciently 
 }
 and to represent masks and plays in their own Halls, or elsewhere.
 \pagebreak
 %329=================================================================== break above because...
\footnotetext[0]{\hspace{-2.5em}
had dancings for their recreations and delight, commonly
called Revels, allowed at certain seasons; and that, by
special order of the society, as appeareth in 9 Hen. VI.,
there should be four Revels that year, and no more,” \&c.
And again he says, “Nor were these exercises of dancing
merely permitted, but thought very necessary, as it seems,
and much conducing to the making of gentlemen more fit
for their books at other times; for, by an order made 6th
Feb. 7 Jac., it appears that the under-barristers were by
decimation put out of Commons for example’s sake, because
the whole bar offended by not dancing on the
Candlemas-day preceding, according to the ancient order
of this society, when the judges were present; with this,
that if the like fault were afterwards committed, they
should be fined or disbarred.”}

A curious letter on the subject of a mask, which for some unexplained
reason did not take place, may he seen in Collier’s \textit{History of Early Dramatic
Poetry and Annals of the Stage}, vol. i., p.~268. It is addressed to Lord
Burghley, by “Mr. Frauncis Bacon” (afterwards Lord Bacon), who in 1588 discharged
the office of Reader of Gray’s Inn. Many other curious particulars of
their masks may he found in the same work, and some in Sir J. Hawkins’ \textit{History
of Music}. For the Christmas Revels of the bar, see Mr. Payne Collier’s note to
Dodsley’s Old Plays, vol. vii., p.~311. Lawyers are now, generally speaking, a
music-loving class. The enjoyment of sweet sounds is to many the most acceptable
recreation after long study. They were also famous in former days for
songs and squibs. Some, too, were tolerable composers, for every one claiming to
be a gentleman learnt music. As their compositions are rather out of my present
subject, I will refer only to their rhyming propensities; and, although much more
ample illustration might be given, two passages from letters of John Chamberlain
to Sir Dudley Carleton, printed in \textit{The Court of James I}. (1849), will probably
suffice. On May 20, 1615, Chamberlain says, “On Saturday last the King went
again to Cambridge to see the play, \textit{Ignoramus}, which hath so nettled the lawyers,
that they are almost out of all patience; and the Lord Chief Justice [Sir E.
Coke] both openly at the King’s Bench, and divers other places, hath galled and
glanced at scholars with much bitterness; and \textit{there be divers Inns at Court have
made rhymes and ballads against them}, which they have answered sharply enough.”
(i. 363.) Again in the letter of Nov. 23, 1616, “Here is a bold rhyme of
our young gallants of Inns of Court against their old benchers, and a pretty
epigram upon the Lord Coke, and no doubt more will follow; for when men are
down, the very drunkards make rhymes and songs upon them.” (i. 444.)

The authorship of the music of this song has been a subject of contention; and
so little have dates been regarded, that it has long passed as the composition of
Henry Purcell, and is still published with his name. Walsh paved the way to
this error (in which Ritson and many others followed), by including it in
a collection of “Mr. Henry Purcell’s Favourite Songs, out of his most celebrated
\textit{Orpheus Britannicus,} and the rest of his works.” It is \textit{not} contained in
the \textit{Orpheus Britannicus} (which was published by Purcell’s widow), and the music
may still be seen as \textit{printed} eight years before Purcell’s birth.

In a note upon the passage before quoted from Walton’s \textit{Angler}, Sir J.
Hawkins adds, “This song, beginning, ‘Forth from my dark and dismal cell,’
with the music to it, \textit{set by Henry Lawes}, is printed in a book, entitled \textit{Choice
Ayres, Songs, and Dialogues to sing to the Theorbo-Lute and Bass Viol}, fol. 1675;
and in Playford’s \textit{Antidote against Melancholy}, 8vo.,~1669.”

\pagebreak
%330===================================================================

Sir John Hawkins must have had some reason, which he does not assign, for
attributing the composition to Henry Lawes. It is not contained in either of the
printed collections of Lawes’ songs, nor have I been able to find any copy with his
name attached to it. Sir John seems to be mistaken, because Lawes did not
enter the Chapel Royal until 1626, and the Curtain Theatre, at which \textit{one of the
songs to the tune were sung},\dcfootnote{\textit{}
Mr. Payne Collier, in a note to Heber’s Catalogue,
Part iv., p.~92, says that this song was sung at the Curtain
Theatre, about 1610. In \textit{Choice Ayres}, 2nd edition, fol.,
1675, the composer’s name is not given, and it is printed
without any base.}
 was in disuse at the commencement of the reign of
Charles I. (1625). We must therefore look to an earlier composer.

One of the Addit. MSS., Brit. Mus. (No. 10,444) is a collection of Mask-
tunes, and there are several in that collection entitled “Gray’s Inn.” See
Nos. 50, 51, 91, 99, \&c. If Nos. 50 and 99 are from the same Mask (which is
not improbable), Mad Tom may be the composition of Lawes’ master, John
Cooper, called “Cuperario” after his visit to Italy. No. 50, the first of the
above tunes, is there called “\textit{Cuperaree}, or Gray’s Inn;” No. 51, “Gray’s In
Anticke Masque;” and No. 99 (the tune in question), “Gray’s Inne Masque.”

There is an equal uncertainty about the authorship of the words. In Walton’s
Angler, 1653, Piscator says, “I’ll promise you I’ll sing a song that was lately
made at my request by Mr. William Basse, one that made the choice songs of
\textit{The Hunter in his career}, and \textit{Tom of Bedlam}, and many others of note.” There
are, however, so many \textit{Toms of Bedlam}, that it is impossible to determine, from
this passage, to which of them Isaak Walton refers.

In addition to the broadsides, and a copy in \textit{Le Prince d’Amour}, 1660, there is
in MSS. Harl., No. 7,332, a version in the handwriting of “Fearegod Barebone, of
Daventry, in the county of Northampton,” who, “beinge at many times idle, and
wanting imployment, bestoed his time with his penn and incke wrighting thease
sonnets, songes, and epigrames, thinkinge that it weare bettar so to doe for the
mendinge of his hand in wrighting, then worse to bestow his time.” Master
Fearegod Barebone was, no doubt, a puritanical hypocrite; and wrote this excuse
about improving his handwriting, to be prepared in case the book should fall into
“ungodly hands.” No other inference can be drawn from his selection of some
of the songs in the manuscript. \textit{Mad Tom}, however, is not one of those objectionable
ditties, and, as being the oldest copy, I have here followed his manuscript.
The tune is from \textit{The Dancing Master}, and differs somewhat from later versions.

\textit{Mad Tom} was employed as a ballad tune in \textit{Penelope}, 1728; and \textit{The Bay’s
Opera},~1730.

\musicinfo{Majestically.}{}
\lilypondfile{lilypond/330-new-mad-tom-of-bedlam-or-mad-tom}\normalsize

\pagebreak


%331===================================================================

\musicinfo{}{}
\lilypondfile[staffsize=15]{lilypond/331-new-mad-tom-of-bedlam-or-mad-tom-part-2}\normalsize

\pagebreak
%332===================================================================


\settowidth{\versewidth}{Last night I heard the dog-star bark;}
\begin{dcverse}\indentpattern{0000000001012212211101010000}
\begin{patverse}
Last night I heard the dog-star bark;\\
Mars met Venus in the dark;\\
Limping Vulcan het an iron bar,\\
And furiously he ran at the god of war.\\
Mars with his weapons beset him about,\\
But Vulcan’s temples had the gout,\\
And his horns did hang so in his light,\\
He could not see to aim his blows aright.\\
Mercury, the nimble post of heaven,\\
Came to see the quarrel;\\
Gor-bellied Bacchus, giant-like,\\
Bestrid a strong-beer barrel.\\
To me he drank,\\
I did him thank,\\
But I could get no cider;\\
He drank whole buts,\\
Till he brake his guts,\\
But mine be never the wider.\\
Poor Tom is very dry:\\
A little drink for charity!\\
Now, hark! I bear Actæon’s hounds,\\
The huntsman whoops and halloos;\\
Ringwood, Roister, Bowman, Jowler,\\
And all the troop do follow.\\
The Man in the Moon drinks claret,\\
Eats powder’d beef, turnip, and carrot,\\
But a cup of old Malaga sack\\
Will fire the bush at his back.
\end{patverse}
\end{dcverse}

It will be observed that the second verse of the above is not now sung.
Another \textit{Mad Tom}, composed by George Hayden, and commencing, “In my
triumphant chariot hurl’d,” has been added to the first, to make a bravura. There
are even different copies of George Hayden’s song, some having a \timesig{9}{4} movement at
the close, which others have not. Hayden was the author of the still favorite
duet, “As I saw fair Clora.” He flourished in the early part of last century.

\musictitle{Tom A Bedlam.}

In \textit{Le Prince d’Amour}, 1660, there are no less than three songs entitled
\textit{Tom of Bedlam}; also Bishop Corbet’s song, \textit{The distracted Puritan}, which is to
the tune of \textit{Tom of Bedlam}.

The first song (at p.~164) consists of eight stanzas, and commences thus:—
\settowidth{\versewidth}{And them I bore twelve leagues and more,}
\begin{dcverse}\begin{altverse}
\vleftofline{“}From the top of high Caucasus,\\
To Paul’s Wharf near the Tower,\\
In no great haste, I easily pass’d\\
In less than half an hour.\\
The gates of old Byzantium\\
I took upon my shoulders,
\end{altverse}

\begin{altverse}
And them I bore twelve leagues and more,\\
In spite of Turks and soldiers, \\
\textit{Sing, sing, and sob; sing, sigh, and be merry;\\
Sighing, singing, and sobbing;\\
Thus naked Tom away doth run,\\
And fears no cold nor robbing}.
\end{altverse}
\end{dcverse}

The second is at p.~167, and consists also of eight stanzas, of which the two
first are as follows:—
\settowidth{\versewidth}{And the spirits, that stand by the naked man}
\begin{dcverse}\begin{altverse}
\vleftofline{“}From the hag and hungry goblin,\\
That into rags would rend you, \\
And the spirits, that stand by the naked man\\
In the book of moons, defend you;\\
That of your five sound senses\\
You never be forsaken,\\
Nor travel from yourselves with Tom\\
Abroad to beg your bacon.\\
\textit{While I do sing, ‘Any food, any feeding,\\
Feeding, drink, or clothing!\\
Come, dame or maid, be not afraid,\\
Poor Tom will injure nothing}.’
\end{altverse}

\begin{altverse}
Of thirty bare years have I\\
Twice twenty been enraged;\\
And, of forty, been three times fifteen\\
In durance soundly caged;\\
On the lordly lofts of Bedlam,\\
With stubble soft and dainty,\\
Brave bracelets strong, and whips, ding-dong\\
And wholesome hunger plenty.\\
\textit{Yet did I sing, ‘Any food, any feeding,\\
Feeding, drink, or clothing!\\
Come, dame or maid, be not afraid,\\
Poor Tom will injure nothing}.’”
\end{altverse}
\end{dcverse}

Ritson, who has reprinted the above two songs, supposes them “to have been
written by way of burlesque on such sort of things.” (\textit{Ancient Songs}, p.~261, 1790.)
\pagebreak
%333===================================================================


The third song (p.~169) is now commonly known as \textit{Mad Tom}. It is in
another metre, and has a separate tune. (Ante p.~330.)

The fourth, commencing, “Am I mad, O noble Festus,” (p.~171), is here
printed to this~tune.

In the Roxburghe Collection, i. 42, there is a song on the tricks and disguises
of beggars, entitled “The cunning Northerne Begger:
\settowidth{\versewidth}{Who all the bystanders doth earnestly pray,}
\begin{scverse}Who all the bystanders doth earnestly pray,\\
To bestow a penny upon him to-day:
\end{scverse}
to the tune of \textit{Tom of Bedlam}.” The first stanza is as follows:—
\vspace{-0.5\baselineskip}
\settowidth{\versewidth}{And weare all ragged garments!xxxxxx}
\begin{dcverse}\indentpattern{010010100101001}
\begin{patverse}
“I am a lusty begger,\\
And live by others giving;\\
I scorne to worke,\\
But by the highway lurke,\\
And beg to get my living.\\
I’ll i’ th’ wind and weather,\\
And weare all ragged garments!\\
\columnbreak
Yet, though I’m bare,\\
I’m free from care,\\
A fig for high preferments, \\
\textit{But still will I cry, ‘Good, your worship, good sir.\\
Bestow one poor denier, sir;\\
Which, when I’ve got,\\
At the pipe and the pot,\\
I soon will it cashier, sir}.’”
\end{patverse}
\end{dcverse}

This copy of the ballad was printed “at London” for F. Coules, and may be
dated as of the reign of Charles, or James I.

In \textit{Wit and Drollery}, 1656 (p.~126), there is yet another \textit{Tom of Bedlam},
beginning—
\vspace{-0.5\baselineskip}
\settowidth{\versewidth}{Forth from the Elysian fields, a place of restless souls,}
\begin{scverse}\vleftofline{“}Forth from the Elysian fields, a place of restless souls,\\
\vin Mad Maudlin is come to seek her naked Tom,\\
Hell’s fury she controls,” \&c.
\end{scverse}
This is printed in an altered form, and with an imperfect copy of the tune, in \textit{
Pills to purge Melancholy}, ii. 192 (1700 and 1707), under the title of “Mad
Maudlin to find out Tom of Bedlam:”
\settowidth{\versewidth}{To find my Tom of Bedlam, ten thousand years I’ll travel;}
\begin{scverse}
\vleftofline{“}To find my Tom of Bedlam, ten thousand years I’ll travel;\\
Mad Maudlin goes, with dirty toes, to save her shoes from gravel.\\
\textit{Yet will I sing, Bonny boys, bonny mad boys, Bedlam boys are bonny;\\
They still go bare, and live by the air, and want no drink nor money}.”
\end{scverse}

The tune is again printed in \textit{ Pills to purge Melancholy}, iii. 13 (1707), to a song
“On Dr.~G[ill?], formerly master of St. Paul’s School,” commencing—
\settowidth{\versewidth}{In Paul’s Churchyard in London,xxxx}
\begin{scverse}\indentpattern{0011023223}
\begin{patverse}
\vleftofline{“}In Paul’s Churchyard in London,\\
There dwells a noble firker,\\
Take heed, you that pass,\\
Lest you taste of his lash,\\
For I have found him a jerker:\\
\textit{Still doth he cry, take him up, take him up, sir,\\
Untruss with expedition;\\
O the birchen tool\\
Which he winds in the school\\
Frights worse than the Inquisition}.”
\end{patverse}
\end{scverse}

In \textit{Loyal Songs written against the Rump Parliament}, 1731, ii. 272, we have
“The cock-crowing at the approach of a Free Parliament; or—
\settowidth{\versewidth}{Than fig, raisin, or stewed prune is:}
\begin{dcverse}\indentpattern{010110}
\begin{patverse}
\vin Good news in a ballat\\
More sweet to your pallat\\
Than fig, raisin, or stewed prune is:\\
A country wit made it,\\
Who ne’er got the trade yet,\\
And \textit{Mad Tom of Bedlam} the tune is.”
\end{patverse}
\end{dcverse}

\pagebreak
%334===================================================================


Among the King’s Pamphlets in the British Museum there are two songs to
this tune. The first (by a loyal Cavalier) is “Mad Tom a Bedlam’s desires of
Peace: Or his Benedicities for distracted England’s Restauration to her wits
again. By a constant though unjust sufferer (now in prison) for His Majesties
just Regality and his Country’s Liberty. S.F.W.B.” (Sir Francis Wortley,
Bart.) This is in the sixth vol. of folio broadsides, and dated June 27, 1648.

\vspace{-0.5\baselineskip}
\settowidth{\versewidth}{Yet still he cries for the King, for the good King;}
\begin{dcverse}\begin{altverse}
\vleftofline{“}Poor Tom hath been imprison’d,\\
With strange oppressions vexed;\\
He dares boldly say, they try'd each way\\
Wherewith Job was perplexed.
\end{altverse}

\begin{altverse}
Yet still he cries for the King, for the good King;\\
Tom loves brave confessors; \\
But he curses those that dare their King depose,\\
Committees and oppressors.” \&c.
\end{altverse}
\end{dcverse}
This has been reprinted in Wright’s \textit{Political Ballads}, for the Percy Society,
p.~102; and in the same volume, p.~183, is another, taken from the fifteenth vol.
of broadsides, entitled “A new Ballade, to an old tune,—\textit{Tom of Bedlam},” dated
January 17, 1659, and commencing, “Make room for an honest red-coat.”

Besides these, we have, in \textit{Wit and Drollery}, 1682, p.~184, \textit{Loving Mad Tom},
commencing, “I’ll bark against the dog-star;” and many other mad-songs in the
Roxburghe Collection, such as “\textit{The Mad Man’s Morrice};” “\textit{Love’s Lunacie, or
Mad Besse’s Vagary};” \&c., \&c.

Bishop Percy has remarked that “the English have more songs on the subject
of madness, than any of their neighbours.” For this the following reason has
been assigned by Mr. Payne Collier, in a note to Dodsley’s Collection of Old
Plays, ii.~4:—

“After the dissolution of the religious houses, where the poor of every denomination
were provided for, there was for many years no settled or fixed provision made to
supply the want of that care which those bodies appear always to have taken of their
distressed brethren. In consequence of this neglect, the idle and dissolute were
suffered to wander about the country, assuming such characters as they imagined were
most likely to insure success to their frauds, and security from detection. Among
other disguises, many affected madness, and were distinguished by the name of
\textit{Bedlam Beggars}. These are mentioned by Edgar, in \textit{King~Lear}:
\settowidth{\versewidth}{The country gives me proof and precedent,}
\begin{scverse}\vleftofline{“}The country gives me proof and precedent,\\
Of \textit{Bedlam} beggars, who, with roaring voices,\\
Stick in their numb’d and mortify’d bare arms\\
Pins, wooden pricks, nails, sprigs of rosemary;\\
And, with this horrible object, from low farms,\\
Poor pelting villages, sheep-cotes, and mills,\\
Sometime with lunatic bans, sometime with prayer,\\
Inforce their charity.”
\end{scverse}

In Dekker’s \textit{Bellman} of London, 1616, all the different species of beggars are
enumerated. Amongst the rest are mentioned \textit{Tom of Bedlam}’s band of mad caps,
otherwise called Poor Tom’s flock of wild geese (whom here thou seest by his black
and blue naked arms to be a man beaten to the world), and those wild geese, or hair
brains, are called Abraham men. An Abraham man is afterwards described in this
manner: ‘Of all the mad rascals (that are of this wing) the \textit{Abraham man} is the
most fantastick. The fellow (quoth this old Lady of the Lake unto me) that sate
half naked (at table to-day) from the girdle upward, is the best \textit{Abraham man} that 
ever came to my house, and the notablest villain: \pagebreak he swears he hath been in Bedlam,
%335===================================================================
and will talk frantickly of purpose: you see pins stuck in sundry places of his naked
flesh, especially in his arms, which pain he gladly puts himself to (being, indeed, no
torment at all, his skin is either so dead with some foul disease, or so hardened with
weather, only to make you believe he is out of his wits): he calls himself by the name
of \textit{Poor Tom}, and coming near any body, cries out, Poor Tom is a cold. Of these
\textit{Abraham men}, some be exceeding merry, and do nothing but sing songs, fashioned
out of their own brains, some will dance; others will do nothing but either laugh or
weep; others are dogged, and are sullen both in look and speech, that, spying but a
small company in a house, they boldly and bluntly enter, compelling the servants
through fear to give them what they demand, which is commonly \textit{Bacon}, or something
that will yield ready money.’”

The song of \textit{Tom of Bedlam} is alluded to in Ben Jonson’s \textit{The Devil is an Ass},
1616, act v., sc. 2. When Pug wishes to be thought mad, he says, “Your best
song’s Thom o’Bet’lem.”

The following copy of the tune is from a manuscript volume of virginal music,
formerly in the possession of Mr. Windsor, of Bath, and now in that of
Dr.~Rimbault. It is entitled \textit{Tom a Bedlam}. The words are from Bishop
Corbet’s song, \textit{The distracted Puritan}, which is printed entire in Percy’s \textit{Reliques
of Ancient Poetry}.

\musicinfo{Pompously.}{}
\lilypondfile{lilypond/335-tom-a-bedlam}\normalsize


\pagebreak
%336===================================================================
\DFNsingle

\settowidth{\versewidth}{In the house of pure Emanuel}
\begin{dcverse}\footnotesizerrr\indentpattern{001104}
\begin{patverse}
In the house of pure Emanuel\footnote
{Emanuel College, Cambridge, was originally a seminary of Puritans.}\\
I had my education,\\
Where my friends surmise\\
I dazzled my eyes\\
With the sight of revelation.\\
\textit{Boldly I preach}, \&c.
\end{patverse}

\begin{patverse}
They bound me like a bedlam,\\
They lash’d my four poor quarters;\\
Whilst this I endure,\\
Faith makes me sure\\
To be one of Fox’s martyrs.\\
\textit{Boldly I preach}, \&c.
\end{patverse}

\begin{patverse}
These injuries I suffer\\
Through antichrist’s persuasion:\\
Take off this chain,\\
Neither Rome nor Spain\\
Can resist my strong invasion.\\
\textit{Boldly I preach}. \&c.
\end{patverse}

\begin{patverse}
Of the beast’s ten horns (God bless us!)\\
I have knock’d off three already;\\
If they let me alone\\
I’ll leave him none:\\
But they say I am too heady.\\
\textit{Boldly I preach}, \&c.
\end{patverse}

\begin{patverse}
When I sack’d the seven hill’d city,\\
I met the great red dragon;\\
I kept him aloof\\
With the armour of proof,\\
Though here I have never a rag on.\\
\textit{Boldly I preach}, \&c.
\end{patverse}

\begin{patverse}
With a fiery sword and target,\\
There fought I with this monster:\\
But the sons of pride\\
My zeal deride,\\
And all my deeds misconster.\\
\textit{Boldly I preach}, \&c.
\end{patverse}

\begin{patverse}
I un-hors’d the Whore of Babel,\\
With the lance of Inspiration;\\
I made her stink,\\
And spill the drink\\
In her cup of abomination.\\
\textit{Boldly I preach}, \&c.
\end{patverse}

\begin{patverse}
I appear’d before the archbishop,\\
And all the high commission;\\
I gave him no grace,\\
But told him to his face,\\
That he favour’d superstition.\\
\textit{Boldly I preach}. \&c.
\end{patverse}
\end{dcverse}

\musictitle{Thomas, You Cannot.}

This tune is contained in Sir John Hawkins’ Transcripts of Virginal Music; in
the fourth and later editions of \textit{The Dancing Master}; in \textit{The Beggars’ Opera};
\textit{The Mock Doctor}; \textit{An Old Man taught Wisdom}; \textit{The Oxford Act}; and other
ballad-operas.

In some of the earlier editions of \textit{The Dancing Master}, it is entitled\textit{ Thomas,
you cannot}; in others, \textit{Tumas, I cannot}, or \textit{Tom Trusty}; in some of the ballad-operas
(for instance, \textit{The Generous Freemason}, and \textit{The Lover his own Rival}),
\textit{Sir Thomas, I cannot}.

In the Pepys Collection, i. 62, is a black-letter ballad (one of the many written
against the Roman Catholics after the discovery of the Gunpowder Plot, in~1605),
entitled “A New-yeeres-Gift for the Pope; O come see the difference plainly
decided between Truth and Falsehood:
\settowidth{\versewidth}{Not all the Pope’s trinkets, which here are brought forth,}
\begin{scverse}
Not all the Pope’s trinkets, which here are brought forth,\\
Can balance the bible, for weight, or for worth,” \&c.
\end{scverse}
“To the tune of \textit{Thomas you cannot}.” It commences thus:—
\begin{scverse}
\vleftofline{“}All you that desirous are to behold\\
The difference ’twixt falsehood and faith,” \&c.
\end{scverse}

In \textit{Grammatical Drollery}, by W. H (Captain Hicks), 1682, p.~75, is a song
commencing, “Come, my Molly, let us be jolly:” to the tune of \textit{Thomas,
I cannot}; and in Chetwood’s \textit{History of the Stage}, 8vo., 1749, a song on a
theatrical anecdote, by Mr. John Leigh (an actor, who died in 1726), of which
the following is the first stanza:—
\pagebreak
\DFNdouble
%337===================================================================

\musicinfo{Gaily.}{}
\lilypondfile{lilypond/337-thomas-i-cannot}\normalsize

I have not been successful in finding the song of \textit{Thomas, you cannot}, from
which the tune derives its name. In some copies (when there are no words), the
second part of the tune consists only of eight bars, instead of ten. See the
following from Sir J. Hawkins’ Transcripts of Virginal Music.

\musicinfo{Gaily.}{}
\lilypondfile{lilypond/337-thomas-i-cannot-hawkins}\normalsize

\pagebreak
%338===================================================================

\musictitle{When Daphne Did From Phœbus Fly.}

This tune is to be found in \textit{Nederlandtsche Gedenck-Clanck}, 1626; in\textit{ Friesche
Lust-Hof}, 1634; and in \textit{The Dancing Master}, from 1650 to 1690.

In the first named it is entitled \textit{Prins Daphne}; in the second,\textit{ When Daphne
did from Phœbus fly}; and in the last, \textit{Daphne, or The Shepherdess}.

A copy of the words will he found in the Roxburghe Collection, i. 388, entitled
“A pleasant new Ballad of Daphne: To a new tune.” Printed by the assignees
of Thomas Symcocke. It is on the old mythological story of Daphne turned into
a~laurel.

\musicinfo{Gracefully, and not too slow.}{}
\lilypondfile{lilypond/338-when-daphne-did-from-phoebus-fly}\normalsize
\pagebreak

%339
%===================================================================


\settowidth{\versewidth}{But still did neglect him the more he did moan;}
\begin{dcverse}\footnotesizerrr\indentpattern{01010101334}
\footnotesize
\begin{patverse}
She gave no ear unto his cry,\\
But still did neglect him the more he did moan;\\
Though he did entreat, she still did deny,\\
And earnestly pray him to leave her alone.\\
Never, never, cries Apollo,\\
Unless to love thou wilt consent,\\
But still, with my voice so hollow,\\
I’ll cry to thee, while life be spent.\\
But if thou turn to me,\\
’Twill prove thy felicity.\\
\textit{Pity, O Daphne, pity me}, \&c.
\end{patverse}

\begin{patverse}
Away, like Venus’s dove she flies,\\
The red blood her buskins did run all adown,\\
His plaintive love she still denies, \\
Crying, Help, help, Diana, and save my renown.\\
Wanton, wanton lust is near me,\\
Cold and chaste Diana, aid!\\
\columnbreak
Let the earth a virgin bear me,\\
Or devour me quick a maid.\\
Diana heard her pray,\\
And turn'd her to a Bay.\\
\textit{Pity, O Daphne, pity me}, \&c.
\end{patverse}

\indentpattern{010101013300}
\begin{patverse}
Amazed stood Apollo then, \\
While he beheld Daphne turn’d as she desir’d,\\
Accurs’d am I, above gods and men,\\
With griefs and laments my senses are tir’d.\\
Farewell! false Daphne, most unkind,\\
My love lies buried in thy grave,\\
Long sought I love, yet love could not find,\\
Therefore is this thy epitaph:\\
\vleftofline{“}This tree doth Daphne cover,\\
That never pitied Lover.” \\
Farewell, false Daphne, that would not pity me,\\
Although not my love, yet art thou my Tree.
\end{patverse}
\end{dcverse}

\vspace{-\baselineskip}

\musictitle{Come You Not From Newcastle?}

\vspace{-0.5\baselineskip}

This beautiful and very expressive melody is to be found in \textit{The Dancing
Master}, from 1650 to 1690, under the title of \textit{Newcastle}. In \textit{The Grub Street
Opera}, 1731, it is named \textit{Why should I not love my love}? from the burden of the
song. The following fragment of the first stanza is contained in the folio manuscript
formerly in the possession of Bishop Percy, p.~95. See Dr.~Dibdin’s
\textit{Decameron}, vol. 3.
\settowidth{\versewidth}{Come you not from Newcastle?}
\begin{dcverse}
\footnotesize
\begin{altverse}
\vleftofline{“}Come you not from Newcastle?\\
Come you not there away?\\
O met you not my true love,\\
Ryding on a bonny bay?
\end{altverse}

\begin{altverse}
Why should I not love my love?\\
Why should not my love love me?\\
{\quad *\quad *\quad *\quad *\quad *\quad *\quad }
\end{altverse}
\end{dcverse}
It is quoted in a little black-letter volume, called “The famous Historie of
Fryer Bacon: containing the wonderfull things that he did in his life; also the
manner of his death; with the lives and deaths of the two Conjurers, Bungye
and Vandermast. Very pleasant and delightfull to be read.” 4to., \textit{n.d.} “Printed
at London by A. E., for Francis Grove, and are to be sold at his shop at the
upper-end of Snow Hill, against the Sarazen’s Head:”—

“The second time, Fryer Bungy and he went to sleepe, and Miles alone to watch
the brazen head; Miles, to keepe him from sleeping, got a tabor and pipe, and being
merry disposed, sung this song to a Northern tune of \textit{Cam’st thou not from Newcastle}—

\vspace{-0.5\baselineskip}
%\settowidth{\versewidth}{To couple is a custome,}
\begin{dcverse}
\footnotesize
\begin{altverse}
\vleftofline{“}To couple is a custome,\\
All things thereto agree;\\
Why should not I then love?\\
Since love to all is free.
\end{altverse}

\begin{altverse}
But Ile have one that’s pretty,\\
Her cheekes of scarlet dye,\\
For to breed my delight,\\
When that I ligge her by.
\end{altverse}

\begin{altverse}
Though vertue be a dowry,\\
Yet Ile chuse money store:\\
If my love prove untrue,\\
With that I can get more.
\end{altverse}

\begin{altverse}
The faire is oft unconstant,\\
The blacke is often proud;\\
Ile chuse a lovely browne;\\
Come, fidler, scrape thy crowd.
\end{altverse}

\begin{altverse}
Come, fidler, scrape thy crowd,\\
For \textit{Peggie} the browne is she\\
Must be my bride; God guide\\
That Peggie and I agree.”
\end{altverse}
\end{dcverse}

\pagebreak
%340
%===================================================================
\DFNsingle

I have been favored by Mr. Barrett with a song, “O come ye from Newcastle?”
as still current in the North of England; but, doubting its antiquity, I have not
thought it desirable to print it in this collection.

\musicinfo{Rather slow, and with expression.}{}
\lilypondfile{lilypond/340-come-you-not-from-newcastle}\normalsize

\footnotetext[1]{\scriptsize
The two last lines are supplied from a song written to complete the fragment, by the late Mr. George Macfarren.}

\musictitle{Cuckolds All A Row.}

This tune is to be found in every edition of \textit{The Dancing Master}. Pepys
mentions it in the following account of a court ball, in the reign of Charles II.:—

“31 Dec., 1662. By and bye comes the King and Queene, the Duke and
Duchess, and all the great ones; and after seating themselves, the King takes out the
Duchesse of York; and the Duke, the Duchesse of Buckingham; the Duke of
Monmouth, my Lady Castlemaine; and so on, other lords other ladies; and they
danced the Bransle. After that, the King led a lady a single Coranto: and then the
rest of the lords, one after another, other ladies: very noble it was, and great pleasure
to see. Then to Country-dances; the King leading the first, which he called for,
which was \textit{Cuckolds all a row}, the old dance of England.”


It became a party tune of the Cavaliers, who sang the songs of \textit{Hey, boys, up}
\pagebreak
\DFNdouble
%341
%===================================================================
\textit{go we}, and \textit{London’s true character}, to it. The latter, abusing the Londoners for
taking part against the King, and commencing, “You coward-hearted citizens,”
is contained in \textit{Rats rhimed to death, or The Rump Parliament hanged in the
Shambles},~1660; and in both editions of \textit{Loyal Songs written against the Rump
Parliament}.

The tune is mentioned in the old song, \textit{O London is a fine town}; and one with
the burden is contained in \textit{Wit and Drollery}, 1661. The latter is reprinted (to
the tune of \textit{London is a fine town}) in \textit{ Pills to purge Melancholy}, ii. 77, 1700, and
iv. 77, 1719.

The following, on the miseries of married life, is from a black-letter ballad,
“printed by M.P. for Henry Gosson, on London Bridge, neere the gate,” and
signed Arthur Halliarg. A copy is in the Roxburghe Collection, i. 28; and
it is reprinted in Evans’ \textit{Old Ballads} i. 170 (1810). I haye omitted four stanzas,
the remainder being sufficient to tell the story. “The cruel Shrew; or The
Patient Man’s Woe:
\settowidth{\versewidth}{Straightway she such a noise will make}
\begin{scverse}
Declaring the misery and great pain,\\
By his unquiet wife, he doth daily sustain.”
\end{scverse}

To the tune of \textit{Cuckolds all a row}.

\musicinfo{Moderate time.}{}
\lilypondfile{lilypond/341-cuckolds-all-a-row}\normalsize
\DFNdouble
%
\pagebreak
%342
%===================================================================

\settowidth{\versewidth}{Her morning’s draught well spiced straight }
\begin{dcverse}\footnotesizerr\begin{altverse}
She never lins her bawling,\\
Her tongue it is so loud,\\
But always she’ll be railing,\\
And will not he controlled:\\
For she the breeches still will wear,\\
Although it breeds my strife;\\
If I were now a bachelor,\\
I’d never have a wife.
\end{altverse}

\begin{altverse}
Sometimes I go in the morning\\
About my daily work,\\
My wife she will be snorting,\\
And in her bed she’ll lurk,\\
Until the chimes do go at eight,\\
Then she’ll begin to wake,\\
Her morning’s draught well spiced straight \\
To clear her eyes she’ll take.
\end{altverse}

\begin{altverse}
As soon as she is out of bed,\\
Her looking-glass she takes,\\
(So vainly is she daily led),\\
Her morning’s work she makes\\
In putting on her brave attire,\\
That fine and costly be;\\
While I work hard in dirt and mire:\\
Alack what remedy?
\end{altverse}

\begin{altverse}
Then she goes forth a gossiping\\
Amongst her own comrades;\\
And then she falls a boosing\\
With all her merry blades;\\
When I come from my labour hard,\\
Then she’ll begin to scold,\\
And call me rogue without regard;\\
Which makes my heart full cold.
\end{altverse}

\begin{altverse}
When I, for quiet’s sake, desire\\
My wife for to be still,\\
She will not grant what I require,\\
But swears she’ll have her will;\\
Then if I chance to heave my hand,\\
Straightway she’ll murder cry;\\
Then judge all men that here do stand,\\
In what a case am I.
\end{altverse}

\begin{altverse}
And if a friend by chance me call\\
To drink a pot of beer,\\
Then she’ll begin to curse and brawl,\\
And fight, and scratch, and tear;\\
And swears unto my work she’ll send\\
Me straight without delay;\\
Or else with the same cudgel’s end,\\
She will me soundly pay.
\end{altverse}

\begin{altverse}
Then is not this a piteous cause,\\
Let all men now it try,\\
And give their verdicts, by the laws,\\
Between my wife and I;\\
And judge the cause, who is to blame.\\
I’ll to their judgment stand,\\
And be contented with the same,\\
And put thereto my hand.
\end{altverse}

\begin{altverse}
If I abroad go anywhere,\\
My business for to do,\\
Then will my wife anon be there\\
For to increase my woe;\\
Straightway she such a noise will make\\
With her most wicked tongue,\\
That all her mates, her part to take,\\
About me soon will throng.
\end{altverse}

\begin{altverse}
Thus am I now tormented still\\
With my most wretched wife;\\
All through her wicked tongue so ill,\\
I am weary of my life:\\
I know not truly what to do,\\
Nor how myself to mend,\\
This lingering life doth breed my woe,\\
I would ’twere at an end.
\end{altverse}

\begin{altverse}
O that some harmless honest man,\\
Whom death did so befriend.\\
To take his wife from off his hand,\\
His sorrows for to end,\\
Would change with me to rid my care.\\
And take my wife alive,\\
For his dead wife, unto his share!\\
Then I would hope to thrive.
\end{altverse}
\end{dcverse}

\musictitle{The Buff Coat Has No Fellow.}

In Fletcher’s play, \textit{The Knight of Malta}, act iii., sc. 1, there is a “Song by
the Watch,” commencing thus:—
\settowidth{\versewidth}{Sit, soldiers, sit and sing, the round is clear,}
\begin{scverse}
\vleftofline{“}Sit, soldiers, sit and sing, the round is clear,\\
And cock-a-loodle-loo tells us the day is near;\\
Each toss his can until his throat be mellow,\\
Drink, laugh, and sing \textit{The soldier has no fellow}.”
\end{scverse}

The last line is repeated in three out of the four verses or parts, and I suppose
\textit{The soldier has no fellow} to have been then a well-known song.

\pagebreak
%343
%===================================================================

Various ballads were written to a tune called \textit{The buff coat has no fellow} (see,
for instance, Pepys Coll., iii. 150; Roxburghe, i. 536; \&c.), and as the buff
coat was a distinguishing mark of the soldier of the seventeenth century, if the
words could be recovered, it might prove to be the song in question.

“In the reign of King James I.,” says Grose, “no great alterations were made
in the article of defensive armour except that the \textit{buff coat}, or jerkin, which was
originally worn under the cuirass, now became frequently a substitute for it, it
having been found that a good \textit{buff leather} would of itself resist the stroke of a
sword; this, however, only occasionally took place among the light-armed cavalry
and infantry, complete suits of armour being still used among the heavy horse.”—
\textit{Military Antiquities}, 1801, 4to., ii., 323. I have been favored with the following
note on the same subject by F. W. Fairholt, F.S.A.:—“The buff coat was
peculiarly indicative of the soldier. It first came into use in the early part of
the seventeenth century, when the heavier defensive armour of plate was discarded
by all but cavalry regiments. The infantry, during the great civil wars
of England, were all arrayed in buff coats; and in Rochester Cathedral are
still preserved some of these defensive coverings, as worn by Oliver’s soldiers
in their unwelcome visits there; as well as the bandoleers worn over them, to hold
the charges for muskets. The officers and cavalry at this time only added the
cuirass; the leather coat was frequently very thick and tough, and a defence
against a sword cut. The foreign, as well as the English armies, about this time,
discarded heavier armour; and the prints by Gheyn, of Low-Country troopers, as
well as those by Ciartes, of the soldiers of the French King, are all habited in
the buff coat, which displays, in the rigidity of its form, its innate strength.”
Grose gives an engraving of those that were worn over corslets, from one that
belonged to Sir Francis Rhodes, Part., of Balbrough Hall, Derbyshire, in the
time of Charles I.

The tune, \textit{The buff coat has no fellow}, is to be found in the fourth and every
subsequent edition of \textit{The Dancing Master};\dcfootnote{\textit{}
Mr. Stenhouse, in his notes to Johnson’s \textit{Scot’s Musical
Museum}, asserts that this air is to be found in Playford’s
\textit{Dancing Master} of 1657, a book which he quotes constantly,
and which, I am convinced, he never saw. Having
tested all his references to that work, I have no hesitation
in saying that not even one of the airs he mentions
is to he found in it. Mr. Stenhouse had before him one
of the last editions of vol. i. of \textit{The Dancing Master},
printed by Pearson and Young, between 1713 and 1725,
and consisting of 358 pages, to which only can \textit{all} of his
quotations be referred.}
 in the earlier editions as \textit{Buff coat},
and afterwards as \textit{Buff coat, or Excuse me}. The following list of ballad-operas, in
all of which songs may be found that were written to the tune, sufficiently proves
its former popularity:—\textit{Polly; The Lottery; An Old Man taught Wisdom; The
Intriguing Chambermaid; The Lovers’ Opera; The Bay’s Opera; The Lover his
own Rival; The Grub Street Opera; The Devil of a Duke, or Trapolin’s Vagaries;
The Fashionable Lady, or Harlequin’s Opera, The Generous Freemason; and
The Footman}.


This popularity extended to Ireland and Scotland; and although, in its old
form, purely English in character, the air has been claimed both as Irish and as
Scotch. T. Moore appropriated it, under the name of \textit{My husband’s a journey to 
Portugal gone}, although in the opinion of \pagebreak Dr.~Crotch, Mr. Wade, and others, “it is
%344
%===================================================================
not at all like an Irish tune.” In Scotland it has been claimed as \textit{The Deuks
dang o'er my Daddie}, and again disclaimed by Mr. George Farquhar Graham,
editor of Wood’s \textit{Songs of Scotland}, who “freely confesses his belief that the air
is not of Scottish origin.” iii.~165.

All the oldest copies of \textit{Buff coat} begin with three long notes, which seem to
require corresponding monosyllables for the commencement of the words. The
line I have quoted from \textit{The Knight of Malta} suggests a commencement somewhat
in the following manner:—

\musicinfo{Boldly.}{}
\lilypondfile{lilypond/344-the-buff-coat-has-no-fellow}\normalsize

I should add, that in some copies of \textit{The Dancing Master} the tune is in common~time.

In later versions, where the long notes at the commencement are split into
quavers (as in many of the ballad-operas), the bold character of the tune is lost,
and it becomes rather a pretty than a spirited air. This change seems to be
owing to the monosyllabic commencement having been discarded in the ballads
which were written to it: as, for instance, in the following, from the Roxburghe
Collection, i. 536:—“The merry Hostess; or—
\settowidth{\versewidth}{A pretty new ditty, compos’d on an hostess that lives in the city.}
\begin{scverse}
A pretty new ditty, compos’d on an hostess that lives in the city.\\
To wrong such an hostess it were a great pity,\\
By reason she caused this pretty new ditty.
\end{scverse}
To the tune of \textit{Buff coat has no fellow}.”
\settowidth{\versewidth}{Come all that love good company,}
\begin{dcverse}\begin{altverse}
\vleftofline{“}Come all that love good company,\\
And hearken to my ditty;\\
’Tis of a lovely hostess fine,\\
That lives in London city;
\end{altverse}

\begin{altverse}
Who sells good ale, nappy and stale,\\
And always thus sings she:\\
My ale was tunn’d when I was young,\\
And but little above my knee,” \&c.
\end{altverse}
\end{dcverse}
The above is printed in Evans’ Collection, i. 150 (1810).

\pagebreak
%345
%===================================================================

In several of the ballad-operas, the tune, whether under the name of \textit{Buff
coat}, or \textit{Excuse me}, commences thus (see, for instance, \textit{The Generous Freemason},
1731):—

\musicinfo{}{}
\lilypondfile{lilypond/345-the-buff-coat-has-no-fellow-begin-1}\normalsize

\noindent And in some more modern versions thus:--

\musicinfo{}{}
\lilypondfile{lilypond/345-the-buff-coat-has-no-fellow-begin-2}\normalsize

When the key-note is heard three times in equal succession at the end of a tune,
it is considered to be characteristic of Irish music; but that peculiarity often arises,
as in the last example, from too many syllables in the words adapted to the air.

\musictitle{A Begging We Will Go.}

In the Bagford Collection, a copy of this song, in black-letter, is entitled “The
Beggars’ Chorus in \textit{The Jovial Crew, to an excellent new tune}.” Brome’s comedy,
\textit{The Jovial Crew, or The Merry Beggars}, was acted at the Cockpit in Drury
Lane, in 1641, and I suppose the song to have been introduced, as it is
not contained in the printed copy of the play. One of the Cavaliers’ ditties,
“Col. John Okie’s Lamentation, or a Rumper cashiered,” is to the tune of
\textit{A begging we will go}. This was published on the 28th March, 1660, and a copy
may be seen among the King’s Pamphlets, Brit. Mus.

\textit{A begging we will go} is printed, with the music, in book v. of \textit{Choice Ayres
and Songs to sing to the Theorlo or Bass Viol}, fol. 1684; in \textit{180 Loyal Songs},
3rd edit., 1685; in \textit{ Pills to purge Melancholy}; \&c. It is sometimes entitled
\textit{The Jovial Beggars}.
\settowidth{\versewidth}{And a begging, we will go, we’ll go, we’ll go,}
\begin{dcverse}\footnotesizerr\begin{patverse}
\indentpattern{121200}
\vin \vleftofline{“}There was a jovial beggar,\\
He had a wooden leg,\\
Lame from his cradle,\\
And forced for to beg.\\
And a begging, we will go, we’ll go, we’ll go,\\
And a begging we will go!
\end{patverse}

\indentpattern{12125}
\begin{patverse}
\vin A bag for his oatmeal,\\
Another for his salt;\\
And a pair of crutches\\
To show that he can halt;\\
And a begging, \&c.
\end{patverse}

\begin{patverse}
\vin A bag for his wheat,\\
Another for his rye;\\
A little bottle by his side\\
To drink when he’s a dry, \&c.
\end{patverse}

\begin{patverse}
\vin Seven years I begg’d\\
For my old master \textit{Wild},\\
He taught me to beg\\
When I was but a child, \&c.
\end{patverse}

\begin{patverse}
\vin I begg’d for my master,\\
And got him store of pelf;\\
But now, Jove be praised,\\
I’m begging for myself, \&c.
\end{patverse}

\begin{patverse}
\vin In a hollow tree\\
I live, and pay no rent;\\
Providence provides for me,\\
And I am well content, \&c.
\end{patverse}

\begin{patverse}
\vin Of all the occupations,\\
A beggar’s life’s the best;\\
For whene’er he’s weary.\\
He’ll lay him down and rest, \&c.
\end{patverse}

\indentpattern{121200}
\begin{patverse}
\vin I fear no plots against me,\\
I live in open cell;\\
Then who would be a king\\
When beggars live so well.\\
And a begging we will go, we’ll go, we’ll go,\\
And a begging we will go!”
\end{patverse}
\end{dcverse}

\pagebreak
%346
%===================================================================

The tune was introduced into the ballad-operas of \textit{Polly, The Lovers, The
Quakers’ Opera, Don Quixote in England, The Court Legacy, The Rape of Helen,
The Humours of the Court, The Oxford Act, The Sturdy Beggars}, \&c.; and the
song is the prototype of many others, such as, “A bowling we will go,” “A fishing
we will go,” “A hawking we will go,” and “A hunting we will go.” The
last-named is printed in the sixth vol. of \textit{The Musical Miscellany}, 8vo., 1731.
It is still popular with those who take delight in hunting; and as the air is now
scarcely known by any other title, I have printed the words to the tune. In
\textit{The Musical Miscellany} it is entitled \textit{The Stag Chace}, and there are twenty-nine
verses; twelve are here omitted, being principally a description of the dogs,
and a catalogue of their names; indeed, it is presum’d that seventeen stanzas
will suffice.

\musicinfo{Gaily.}{}
\lilypondfile{lilypond/346-a-begging-we-will-go}\normalsize

\settowidth{\versewidth}{Thro’ bush and brake, o’er hedge and stake,}
\begin{dcverse}\begin{altverse}
I leave my bed betimes,\\
Before the morning’s grey;\\
Let loose my dogs, and mount my horse,\\
And halloo “come away!” \&c.
\end{altverse}

\begin{altverse}
The game’s no sooner rous’d,\\
But in rush the cheerful cry,\\
Thro’ bush and brake, o’er hedge and stake,\\
The noble beast does fly, \&c.
\end{altverse}

\begin{altverse}
In vain he flies to covert,\\
A num’rous pack pursue,\\
That never cease to trace his steps,\\
Even tho’ they’ve lost the view, \&c.
\end{altverse}

\begin{altverse}
Now sweetly in full cry\\
Their various notes they join;\\
Gods! what a concert’s here, my lads!\\
’Tis more than half divine, \&c.
\end{altverse}

\begin{altverse}
The woods, the rocks, and mountains,\\
Delighted with the sound,\\
To neighb’ring dales and fountains\\
Repeating, deal it round, \&c.
\end{altverse}

\begin{altverse}
A glorious chace it is,\\
We drive him many a mile,\\
O’er hedge and ditch, we go thro’ stitch,\\
And hit off many a foil, \&c.
\end{altverse}
\end{dcverse}
\pagebreak
%347
%===================================================================

\begin{dcverse}\begin{altverse}
And yet he runs it stoutly,\\
How wide, how swift he strains!\\
With what a skip he took that leap,\\
And scours o’er the plains! \&c.
\end{altverse}

\begin{altverse}
See how our horses foam!\\
The dogs begin to droop;\\
With winding horn, on shoulder borne,\\
’Tis time to cheer them up, \&c.
\end{altverse}

\begin{altverse}
Hark! Leader, Countess, Bouncer!\\
Cheer up my good dogs all;\\
To Tatler, hark! he holds it smart,\\
And answers ev’ry call, \&c.
\end{altverse}

\begin{altverse}
Up yonder steep I’ll follow.\\
Beset with craggy stones;\\
My lord cries, “Jack, you dog, come back,\\
Or else you’ll break your bones!” \&c.
\end{altverse}

\begin{altverse}
See, now he takes the moors,\\
And strains to reach the stream!\\
He leaps the flood, to cool his blood,\\
And quench his thirsty flame, \&c.
\end{altverse}

\begin{altverse}
He scarce has touch’d the bank,\\
The cry bounce finely in,\\
And swiftly swim across the stream,\\
And raise a glorious din, \&c.
\end{altverse}

\begin{altverse}
His legs begin to fail,\\
His wind and speed are gone,\\
He stands at bay, and gives ’em play,\\
He can no longer run, \&c.
\end{altverse}

\begin{altverse}
But vain are heels and antlers.\\
With such a pack set round,\\
Spite of his heart, they seize each part,\\
And pull him fearless down, \&c.
\end{altverse}

\begin{altverse}
Ha! dead, ’ware dead! whip off,\\
And take a special care;\\
Dismount with speed, and pray take heed,\\
Lest they his haunches tear! \&c.
\end{altverse}

\begin{altverse}
The sport is ended now,\\
We’re laden with the spoil;\\
As home we pass, we talk o’ th’ chace,\\
O’erpaid for all our toil, \&c.
\end{altverse}
\end{dcverse}

Many songs to the tune will be found in the publications enumerated above.
Others in the \textit{Songs sung at the Mug-houses in London and Westminster}, 1716; in
\textit{120 Loyal Songs}, 1684; and in the various collections of ballads. “The Church
Scuffle, or News from St. Andrew’s” is one of these; and contained in the collection
given to the Cheetham Library by Mr. Halliwell (No. 366).

\musictitle{The Noble Shirve.}

This tune is taken from a manuscript volume of virginal music, formerly in
the possession of Mr. Windsor, of Bath, and now in that of Dr.~Rimbault.

Although the transcript is of the seventeenth, the tunes are generally traceable
to the sixteenth century, and perhaps the latest are of the reign of James I.

I regret very much not having been able to find the ballad from which it
derives its name, for I imagine it would prove an interesting, and, probably, a
very early one.

“Shirve” is a very old form of “Shire-reeve,” or Sheriff; and I have not
been able to trace any other instance of its use so late as the seventeenth century.
It was then, almost universally, written “Shrieve.” The tune is one that—
like \textit{The Three Ravens} (ante p.~59), and \textit{The Friar in the Well} (p.~274)—
requires a burden at the end of the first and second lines of words, as well as at
the end. The third and fourth bars of music seem almost to speak the words
“dōwn-ă-down,” and “hĕy dōwn-ă-dōwn” (or some similar burden); and the
seventh and eighth, “dōwn, ă-dōwn, ă-dōwn-ā.”

These repeated burdens were more common in the sixteenth than in the
seventeenth century.

As every ballad-tune sounds the better for having words to it, I have taken one 
of the snatches of old songs sung by \pagebreak Moros, the fool, or jester, in Wager’s
%348
%===================================================================
interlude, \textit{The longer thou livest the more fool thou art}, 1568. It is not in the
precise measure—there should be two long syllables, instead of “out of Kent,”
in the second bar, \&c.—but I cannot find any old ballad, with similar burdens,
that corresponds exactly.

\musicinfo{Moderate time, and smoothly.}{}
\lilypondfile{lilypond/348-the-noble-shirve}\normalsize

\musictitle{Derry Down.}

This tune is referred to as \textit{The Abbot of Canterbury}; as \textit{Derry down}; as
\textit{A Cobbler there was}; and as \textit{Death and the Cobbler}.

Henry Carey, in his \textit{Musical Century}, 1740, i. 53, gives a song commencing—
\settowidth{\versewidth}{King George he was born in the month of October—}
\begin{scverse}
\vleftofline{“}King George he was born in the month of October—\\
’Tis a sin for a subject that month to be sober;”
\end{scverse}
which is to this tune; and he says, “The melody stolen from an \textit{old} ballad,
called \textit{Death and the Cobbler}.”

In Watts’ \textit{Musical Miscellany}, 1729, i. 94, is “A ballad to the \textit{old} tune, \textit{The
Abbot of Canterbury};” and, in the second volume of the same collection,
“\textit{Cobbler there was}, set by Mr. Leveridge,” who was then living. The tunes
are essentially the same, but Leveridge altered a few notes in the second part.


Dr.~Percy remarks that “the common popular \pagebreak ballad of \textit{King John and the
%349
%===================================================================
Abbot of Canterbury} seems to have been abridged and modernized about the time
of King James I., from one much older, entitled \textit{King John and the Bishop of
Canterbury}.” He adds that “the archness of the questions and answers hath
been much admired by our old ballad-makers; for, besides the two copies above
mentioned, there is extant another ballad on the same subject, entitled \textit{King
Olfrey and the Abbot}.” “Lastly, about the time of the civil wars, when the cry
ran against the bishops, some Puritan worked up the same story into a very
doleful ditty to a solemn tune, concerning \textit{King Henry and a Bishop}, with this
stinging moral”—
\settowidth{\versewidth}{Unlearned men hard matters out can find,}
\begin{scverse}
\vleftofline{“}Unlearned men hard matters out can find,\\
When learned bishops princes’ eyes do blind.”
\end{scverse}

A copy of the last is in the Douce Collection, fol. 110, entitled \textit{The King and
the Bishop}; another in the Pepys, i. 472; and a third in the Roxburghe, iii. 170.
It commences thus:—
\settowidth{\versewidth}{In Popish times, when bishops proud}
\begin{scverse}
\begin{altverse}
\vleftofline{“}In Popish times, when bishops proud\\
In England did bear sway,\\
Their lordships did like princes live,\\
And kept all at obey.”
\end{altverse}
\end{scverse}

The ballad of \textit{The old Abbot and King Olfrey} is in the Douce Collection, fol. 169.
Olfrey is supposed to be a corruption of Alfred.

Mr. Payne Collier, in his \textit{Extracts from the Registers of the Stationers’
Company}, i.~90, prints a ballad entitled \textit{The praise of Milkemaydes}, from
a manuscript of the time of James I., now in his possession. It is evidently the
same as \textit{A defence for Mylkemaydes against the terme of Mawken}, which was
entered on the Registers in 1563–4. Unfortunately neither the entry, nor Mr.
Collier’s manuscript, gives the name of the tune to which that ballad was sung.
I have a strong persuasion that it was to this air, for it has all the character of
antiquity, and I can find no other that would suit the words. The ballad
commences thus:—

\settowidth{\versewidth}{Passe not for rybaldes which mylkemaydes defame,}
\begin{scverse}
“Passe not for rybaldes which mylkemaydes defame,\\
And call them but Malkins, poore Malkins by name;\\
Their trade is as good as anie we knowe,\\
And that it is so I will presently showe.\\
\attribution Downe, a-downe, \&c.”
\end{scverse}

If, instead of “downe, a-downe, \&c.” we had the burden complete, “downe,
a-downe, downe, hey derry down,” I should feel no doubt of its being the air;
but the burden is not given at length in the manuscript. The second and sixth
stanzas allude to the singing of milkmaids—

\begin{scverse}
“They rise in the morning to heare the larke sing,\\
\textit{And welcome with ballettes the summer’s comming}\\
They goe to their kine, and their milking is done\\
Before that some sluggardes have lookt at the sunne.\\
\attribution Downe, a-downe, \&c.

In going to milking, or comming awaie,\\
They sing mery ballettes, or storyes they saye;\\
Their mirth is as pure and as white as their milke;\\
You cannot say that of your velvett and silke.\\
\attribution Downe, a-downe,” \&c.
\end{scverse}
\pagebreak
%350
%===================================================================

There are numberless songs and ballads to the tune, under one or other of its
names. Political songs will be found in the collections written against the Rump
Parliament; in those of the time of James II.; and again in “A Collection of
State Songs, \&c., that have been published since the Rebellion, and sung in the
several mug-houses in the cities of London and Westminster” (1716). One of
Shenstone’s ballads, \textit{The Gossiping}, is to the tune of \textit{King John and the Abbot of
Canterbury}, and is printed in his works, Oxford, 1737. Again, in \textit{The Asylum
for Fugitive Pieces}, 1789, there are several; and the tune is in common use at
the present day.

Dr.~Rimbault, in his \textit{Musical Illustrations to Percy’s Reliques of Ancient
Poetry}, prints from a MS. of the latter part of the seventeenth century, which
agrees with the copy in Watts’ \textit{Musical Miscellany}. Other copies will be found
in \textit{The Beggars’ Opera}, third edit., 1729; \textit{The Village Opera}, 1729; \textit{Penelope},
1728; \textit{The Fashionable Lady}, 1730; \textit{The Lover his own Rival}, 1736; \textit{The
Boarding-School, or The Sham Captain}, 1733; \textit{The Devil to pay}, 1731; \textit{The
Oxford Act; The Sturdy Beggars; Love and Revenge; The Jew decoy’d}; \&c.

I have printed two copies of the tune; the first being the commonly received
version, and the second taken from Watts’ \textit{Musical Miscellany}. These differ
materially, but intermediate versions will be found in \textit{The Beggars’ Opera}, and
some other of the above-mentioned works.

Both \textit{The King and the Abbot}, and \textit{The King and the Bishop}, are in the
catalogue of ballads, printed by Thackeray, in the reign of Charles II. The
copy of the former in the Bagford Collection is entitled “King John and the
Abbot of Canterbury, to the tune of \textit{The King and Lord Abbot}.” The story,
upon which these ballads are founded, can be traced back to the fifteenth
century.

\musicinfo{Moderate time.}{}
\lilypondfile{lilypond/350-derry-down}\normalsize
%

\pagebreak
%351
%===================================================================

\settowidth{\versewidth}{—Now from the third question thou must not shrink,}
\begin{dcverse}\scriptsizerrr
And I’ll tell you a story, a story so merry,\\
Concerning the Abbot of Canterbury;\\
How for his housekeeping, and high renown,\\
The king he sent for him to fair London town.

An hundred men, the king did hear say,\\
The abbot did keep in his house every day;\\
And fifty gold chains, without any doubt,\\
In velvet coats waited the lord abbot about.

How now, father abbot, I hear it of thee,\\
Thou keepest a far better house than me;\\
And from thy housekeeping and high renown,\\
I fear thou work’st treason against my crown.

My liege, quo’ the abbot, I would it were known,\\
I never spend nothing but what is my own;\\
And I trust that your grace will do me no dere,\\
For spending of my own true-gotten gear.

Yes, yes, father abbot, thy fault it is high,\\
And now for the same thou needest must die;\\
For, except thou canst answer me questions three,\\
Thy head shall be smitten from off thy bodỳ.

And first, quo’ the king, when I’m in this stead,\\
With my crown of gold so fair on my head,\\
Among all my liegemen so noble of birth,\\
Thou must tell me to one penny what I am worth.

And, secondly, tell me, without any doubt,\\
How soon I may ride the whole world about.\\
And at the third question thou must not shrink,\\
But tell me here truly what I do think.

O, these are hard questions for my shallow wit,\\
And I cannot answer your grace as yet:\\
But if you will give me but three weeks space,\\
I’ll do my endeavour to answer your grace.

Now three weeks’ space to thee will I give,\\
And that is the longest time thou hast to live;\\
For if thou dost not answer my questions three,\\
Thy lands and thy livings are forfeit to me.

Away rode the abbot all sad at that word,\\
And he rode to Cambridge and Oxenford;\\
But never a doctor there was so wise,\\
That could with his learning an answer devise.

Then home rode the abbot of comfort so cold,\\
And he met his shepherd a going to fold:\\
“How now, my lord abbot, you are welcome home, \\
What news do you bring us from good King John?”

“Sad news, sad news, shepherd, I must give;\\
That I have but three days longer to live;\\
For if I do not answer him questions three,\\
My head will be smitten from off my bodỳ.

The first is to tell him there in that stead,\\
With his crown of gold so fair on his head,\\
Among all his liegemen so noble of birth,\\
To within one penny of what he is worth.

The second, to tell him, without any doubt,\\
How soon he may ride this whole world about:\\
And at the third question I must not shrink,\\
But tell him there truly what he does think.”

“Now cheer up, sire abbot, did you never hear yet,\\
That a fool he may learn a wise man wit?\\
Lend me horse, and serving men, and your apparel,\\
And I’ll ride to London to answer your quarrel.

Nay frown not, if it hath been told unto me\\
I am like your lordship, as ever may be;\\
And if you will only but lend me your gown,\\
There’s none that shall know us at fair London town.”

“Now horses and serving-men thou shalt have,\\
With sumptuous array most gallant and brave;\\
With crozier, and mitre, and rochet, and cope,\\
Fit to appear ’fore our father the pope.”

“Now welcome, sire abbot, the king he did say,\\
’Tis well thou’rt come back to keep to thy day;\\
For and if thou canst answer my questions three,\\
Thy life and thy living both saved shall be.

And first, when thou seest me here in this stead,\\
With my crown of gold so fair on my head,\\
Among all my liegemen so noble of birth,\\
Tell me to one penny what I am worth.”

“For thirty pence our Saviour was sold\\
Among the false Jews, as I have been told;\\
And twenty-nine is the worth of thee,\\
For I think thou art one penny worser than he.”

The king he laughed, and swore by St. Bittel,\\
“I did not think I had been worth so little!\\
—Now, secondly tell me, without any doubt,\\
How soon I may ride this whole world about.”

“You must rise with the sun, and ride with the same,\\
Until the next morning he riseth again;\\
And then your grace need not make any doubt,\\
But in twenty-four hours you’ll ride it about.”

The king he laughed, and swore by St. Jone,\\
“I did not think it could be gone so soon!\\
—Now from the third question thou must not shrink,\\
But tell me here truly what I do think.”

\end{dcverse}
%
\pagebreak
%352
%===================================================================

\begin{dcverse}\footnotesizer
“Yea, that shall I do, and make your grace merry,\\
You think I’m the Abbot of Canterbury; \\
But I’m his poor shepherd, as plain you may see,\\
That am come to beg pardon for him and for me.”

The king he laughed, and swore by the mass,\\
“I’ll make thee lord abbot this day in his place.\\
\columnbreak
“Now nay, my liege, be not in such speed,\\
For, alack, I can neither write ne read.”

“Four nobles a week, then, I will give thee,\\
For this merry jest thou hast shown unto me;\\
And tell the old abbot, when thou comest home,\\
Thou hast brought him a pardon from good King John.”
\end{dcverse}

The following is a very different version of the tune, as printed in Watts’
\textit{Musical Miscellany}.

\musicinfo{Moderate time.}{}
\lilypondfile{lilypond/352-derry-down-2}\normalsize

The following punning prototype of the late T. Hood’s comic songs, should not
be omitted. It is entitled \textit{The Cobbler's End}:—

\begin{dcverse}\footnotesizer  A cobbler there was, and he liv’d in a stall,\\
Which serv’d him for parlour, for kitchen, and all;\\
No coin in his pocket, nor care in his pate,\\
No ambition had he, nor duns at his gate.\\
\vin Derry down, down, down, derry down.

Contented he work’d, and he thought himself happy, \\
If at night he could purchase a jug of brown nappy;\\
How he’d laugh then, and whistle, and sing, too, most sweet,\\
Saying just to a hair I have made both ends meet. \\
\vin Derry down, down, \&c.

But love the disturber of high and of low,\\
That shoots at the peasant as well as the beau,\\
He shot the poor cobbler quite thorough the heart:\\
I wish he had hit some more ignoble part.\\
\vin Derry down, down, \&c.

It was from a cellar this archer did play,\\
Where a buxom young damsel continually lay;\\
Her eyes shone so bright when she rose ev’ry day, \\
That she shot the poor cobbler quite over the way,\\
\vin Derry down, down, \&c.
\end{dcverse}
%
\pagebreak
%353
%===================================================================

\begin{dcverse}He sang her love songs as he sat at his work:\\
But she was as hard as a Jew or a Turk;\\
Whenever he spake, she would flounce and would fleer,\\
Which put the poor cobbler quite into despair.\\
\vin Derry down, down, \&c.

He took up his awl that he had in the world,\\
And to make away with himself was resolv’d;\\
%\columnbreak
He pierc’d through his body instead of his sole,\\
So the cobbler he died, and the bell it did toll.\\
\vin Derry down, down, \&c.

And now in good-will I advise as a friend,\\
All cobblers take warning by this cobbler’s end; \\
Keep your hearts out of love, for we find by what’s past,\\
That love brings us all to an end at the last.\\
\vin Derry down, down, \&c.
\end{dcverse}

\musictitle{Tom Tinker’s My True-Love.}

The tune of \textit{Tom Tinker's my true love} is mentioned in a black-letter tract,
called \textit{The World's Folly}, which was reprinted in \textit{The British Bibliographer},
edited by Sir Egerton Brydges:—“A pot of strong ale, which was often at his nose,
kept his face in so good a coulour, and his braine in so kinde a heate, as, forgetting
part of his forepassed pride, (in the good humour of grieving patience,) made him,
with a hemming sigh, ilfavourdly singe the ballad of \textit{Whilom I was}, to the tune of
\textit{Tom Tinker}.” (ii. 559). The tune is in \textit{The Dancing Master} from 1650 to
1698. About the latter period it seems to- have been rejected for another air
(under the same name), which is printed with the words in \textit{Pills to purge
Melancholy}, vi. 265; and was introduced in \textit{The Beggars’ Opera} for the song
\textit{Which way shall I turn me}?

The following tune is from\textit{The Dancing Master}:—

\musicinfo{Moderate time.}{}
\lilypondfile{lilypond/353-tom-tinkers-my-true-love}\normalsize

%
\pagebreak
%354
%===================================================================

The \textit{Tom Tinker} of \textit{The Beggars’ Opera}, and to which D’Urfey prints the
above words, is subjoined.

\musicinfo{Moderate time, and Smoothly.}{}
\lilypondfile{lilypond/354-tom-tinkers-my-true-love-2}\normalsize

\musictitle{Northern Nancy.}

This tune is contained in every edition of \textit{The Dancing Master}, after 1665.
It is evidently only another version of \textit{With my flock as walked I }(ante p.~157).\dcfootnote{\textit{}
I had not observed the identity of these tunes when
the former sheet went to press; otherwise I should have
compressed the account of them under one head. The
difference is chiefly in the two first bars, but even that
variation is diminished in the copy called\textit{ The faithful
Brothers}, to which I have referred at the former page.}
\settowidth{\versewidth}{Then plump \textit{Bobbing Joan} straight call’d for her own,}
\begin{scverse}
\begin{altverse}
\vleftofline{“}Then plump \textit{Bobbing Joan} straight call’d for her own,\\
And thought she frisk’d better than any,\\
Till Sisly, with pride, took the fiddler aside,\\
And hade him strike up Northern Nanny.”\\
\attribution \textit{Pills to purge Melancholy}, ii. 232, 1719.
\end{altverse}
\end{scverse}

In the Roxburghe Collection, i. 252, is a black-letter ballad, entitled “The
Map of Mock-Begger Hall, with his scituation in the spacious countrey called
\textsc{Anywhere}. To the tune of \textit{It is not your Northern Nanny}; or \textit{Sweet is the
lass that loves me}.” It commences thus:
\settowidth{\versewidth}{While \textit{Mock-Beggar Hall stands empty}.”}
\begin{dcverse}\begin{altverse}
\vleftofline{“}I read in ancient times of yore\\
That men of worthy calling\\
Built alms-houses and spittles store,\\
Which now are all down falling:
\end{altverse}

\begin{altverse}
And few men seek them to repair,\\
Nor is there one among twenty\\
That for good deeds will take any care,\\
While \textit{Mock-Beggar Hall stands empty}.”
\end{altverse}
\end{dcverse}

It consists of twelve stanzas, and “Printed at London for Richard Harper, neere
to the Hospitall Gate in Smithfield.”

In the same Collection, iii. 218, is another version of the same ballad, issued
by the same printer, but with variations in the imprint, in the number of stanzas,
and in the woodcut.

%
The first has a woodcut of a country mansion; the second of a castle. The
second has three additional stanzas, and variations in the remaining twelve.
The title commences, “Mock-Begger’s-Hall,” \pagebreak instead of “The Map of;” and
%355
%===================================================================
at the end, “London: Printed for Richard Harper, \textit{at the Bible and Harp} in
Smithfield.”

Mr. Payne Collier, who has reprinted the latter in his \textit{Roxburghe Ballads}, is of
opinion that, although Richard Harper printed during the Commonwealth, the
ballad itself is of the early part of the seventeenth century. (It contains the
same complaints of the decay of hospitality that are to be found in \textit{The Queen's
Old Courtier}.) The first stanza of the second ballad is here printed to the tune.

In the Roxburghe Collection, ii. 390, is another ballad, called \textit{The ruined
Lover}, \&c., “to the tune of \textit{Mock-Begger's Hall stands empty},” beginning—
\settowidth{\versewidth}{Mars shall to Cupid now submit,}
\begin{dcverse}
\begin{altverse}
\vleftofline{“}Mars shall to Cupid now submit,\\
For he hath gain’d the glory;\\
You that in love were never yet,\\
Attend unto my story;
\end{altverse}

\begin{altverse}
For it is new, ’tis strange and true,\\
As ever age afforded;\\
A tale more sad you never had\\
In any books recorded.”
\end{altverse}
\end{dcverse}

This was printed by W. Thackeray, temp. Charles II.

\textit{Northern Nancy} is one of the tunes called for by “the hob-nailed fellows” in
\textit{The Second Tale of a Tub}, 8vo., 1715.

\musicinfo{Rather slowly.}{}
\lilypondfile{lilypond/355-northern-nancy}\normalsize


\pagebreak
%356
%===================================================================

\musictitle{I Have But A Mark A Year.}
\vspace{-0.25\baselineskip}

This tune is to be found in \textit{ Pills to purge Melancholy}, ii. 116, 1700 and 1707;
or iv.~116,~1719. The ballad is by Martin Parker, and a copy is contained in
the Roxburghe Collection, i. 122. In the preface to the \textit{Pills}, Playford tells us
that the words of the songs “which are old have their rust generally filed from
them, which cannot but make them very agreeable.” This is one that has
undergone the process of “filing;” it is abbreviated, but certainly not improved,
by the operation. The copy in the Roxburghe Collection is entitled “A fair
portion for a fair Maid; or—
\vspace{-0.25\baselineskip}
\settowidth{\versewidth}{The thrifty maid of Worcestershire,}
\begin{dcverse}The thrifty maid of Worcestershire,\\
Who lives at London for a mark a year;

This mark was her old mother’s gift.\\
She teaches all maids how to thrift.
\end{dcverse}
\vspace{-0.25\baselineskip}

To the tune of \textit{Grammercy, Penny}.” (The first stanza is here printed with the
music.) \textit{Grammercy} (or \textit{God-a-mercy}), \textit{Penny}, derives its name from the burden
of another ballad, also in the Roxburghe Collection (i. 400), entitled “There’s
nothing to be had without money; or—
\vspace{-0.25\baselineskip}

\begin{dcverse}He that brings money in his hand,\\
Is sure to speed by sea and land;\\
But he that hath no coin in’s purse,

His fortune is a great deal worse;\\
Then happy are they that always have\\
A penny in purse, their credit to save.
\end{dcverse}
\vspace{-0.25\baselineskip}

\noindent To \textit{a new Northern tune}, or\textit{ The mother beguil’d the daughter}.” It commences thus:
\vspace{-0.25\baselineskip}

\begin{dcverse}“You gallants and you swagg’ring blades,\\
Give ear unto my ditty;\\
I am a boon-companion known\\
In country, town, and city;

I always lov'd to wear good clothes,\\
And ever scorned to take blows;\\
I am belov’d of all me knows,\\
But \textit{God-a-mercy penny}.”
\end{dcverse}
\vspace{-0.25\baselineskip}

This was “printed at London for H[enry] G[osson].” Six stanzas in the
first, and eight in the second part.

Another ballad, from the same press, is “The Praise of Nothing: to the tune
of \textit{Though I have but a marke a yeare}, \&c.” A copy in the Roxburghe Collection,
i. 328, and reprinted in Payne Collier’s \textit{Roxburghe Ballads}, p.~147. The
following lines are added to the title of the ballad:—
\settowidth{\versewidth}{Though some do wonder why I write the praise}
\begin{scverse}
\vleftofline{“}Though some do wonder why I write the praise\\
Of Nothing in these lamentable days,\\
When they have read, and will my counsel take,\\
I hope of Nothing they will Something make!”
\end{scverse}

The above contains much excellent advice.

Having traced the tune from \textit{I have but a mark a year} to \textit{God-a-mercy, Penny},
and from the latter to “\textit{a new Northern tune}, or \textit{The mother beguil’d the daughter},”
the following ballads may also be referred to it:—

Roxburghe, i. 238—“The merry careless lover: Or a pleasant new ditty, called
I love a lass since yesterday, and yet I cannot get her. To the tune of \textit{The
mother beguilde the daughter}."
\vspace{-0.25\baselineskip}

\settowidth{\versewidth}{Oft have I heard of many men}
\begin{dcverse}\begin{altverse}
\vleftofline{“}Oft have I heard of many men\\
Whom love hath sore tormented,\\
With grief of heart, and bitter smart,\\
And minds much discontented;\\
Such, love to me shall never be,\\
Distasteful, grievous, bitter!
\end{altverse}

\begin{altverse}
I have lov’d a lass since yesterday,\\
And yet I cannot get her.\\
But let her choose—if she refuse,\\
And go to take another,\\
I will not grieve, but still will be\\
\textit{The merry careless lover},” \&c.
\end{altverse}
\end{dcverse}


\pagebreak
%357
%===================================================================

\noindent Signed Robert Guy. Twelve stanzas. Printed at London for F. Coules, and
reprinted in Evans’ \textit{Old Ballads}, i. 176, 1810.

Roxburghe, i. 314, “A Peerless Paragon; or—
\settowidth{\versewidth}{Few so chaste, so beauteous, or so fair;}
\begin{scverse}
Few so chaste, so beauteous, or so fair;\\
For with my love I think none can compare.
\end{scverse}
To the tune of \textit{The mother beguild the daughter}.”
\settowidth{\versewidth}{In times of yore sure men did doat,}
\begin{dcverse}\begin{altverse}
\vleftofline{“}In times of yore sure men did doat,\\
And beauty never knew,\\
Else women were not of that note,\\
As daily come to view:
\end{altverse}

\begin{altverse}
For, read of all the faces then\\
That did most brightly shine,\\
Be judg’d by all true-judging men,\\
They were not like to mine.”
\end{altverse}
\end{dcverse}
This has no burden. It consists of thirteen stanzas. “Printed at London for
Thomas Lambert.”

Martin Parker’s ballad, “The Countrey Lasse,” to the tune of \textit{The mother
beguild the daughter}, has been quoted at p.~306, but it appears also to have had a
separate tune, which will be given hereafter.

\musicinfo{Cheerfully.}{}
\lilypondfile{lilypond/357-i-have-but-a-mark-a-year}\normalsize


\pagebreak
%358
%===================================================================

\musictitle{I Tell Thee, Dick, Where I Have Been.}

This celebrated ballad, by Sir John Suckling, was occasioned by the marriage
of Roger Boyle, the first Earl of Orrery (then Lord Broghill), with Lady
Margaret Howard, daughter of the Earl of Suffolk. The words are in the first
edition of Sir John Suckling’s works, 1646; in \textit{Wit’s Recreation}, 1654; in
\textit{Merry Drollery Complete}, 1661; \textit{Antidote to Melancholy}, 1661; in\textit{ The Convivial
Songster}, 1782; in Ritson’s \textit{Ancient Songs}, p.~223; and Ellis’ \textit{Specimens of Early
English Poets}, iii. 248.

The tune is in \textit{A Choice Collection of 180 Loyal Songs}, third edit., 1685; in
\textit{Pills to purge Melancholy}, vol. i., 1699 and 1707; in \textit{The Convivial Songster},
1782, \&c.

The following were written to the tune:—

1.\textit{ The Cavalier’s Complaint}. A copy in the Bagford Collection (643, m. 11,
p.~23) dated 1660; and one in the King’s Pamphlets, No. 19, fol., 1661; others
in \textit{Antidote to Melancholy}; \textit{Merry Drollery}, 1670; \textit{The New Academy of Compliments},
1694 and 1713; and Dryden’s \textit{Miscellany Poems}, vi. 352; \&c.

\settowidth{\versewidth}{Come, Jack, let’s drink a pot of ale,}
\begin{dcverse}\indentpattern{001001}
\begin{patverse}
\vleftofline{“}Come, Jack, let’s drink a pot of ale,\\
And I will tell thee such a tale,\\
Shall make thine ears to ring;\\
My coin is spent, my time is lost,\\
And I this only fruit can boast—\\
That once I saw my King.
\end{patverse}

\begin{patverse}
But this doth most afflict my mind—\\
I went to court in hope to find\\
Some of my friends in place;\\
And, walking there, I had a sight\\
Of all the crew—but, by this light,\\
I hardly knew one face!
\end{patverse}

\begin{patverse}
S’life, of so many noble sparks,\\
Who on their bodies bear the marks\\
Of their integrity,\\
And suffer’d ruin of estate,\\
It was my damn’d unhappy fate\\
That I not one could see.
\end{patverse}

\begin{patverse}
Not one, upon my life, among\\
My old acquaintance, all along\\
At Truro, and before;\\
And I suppose the place can shew\\
As few of those whom thou didst know\\
At York, or Marston-Moor.
\end{patverse}

\begin{patverse}
But, truly, there are swarms of those\\
Whose chins are beardless, yet their hose\\
And buttocks still wear muffs;\\
Whilst the old rusty Cavalier\\
Retires, or dares not once appear.\\
For want of coin and cuffs.
\end{patverse}

\begin{patverse}
When none of these I could descry,\\
(Who better far deserv’d than I,)\\
Calmly did I reflect;\\
Old services, by rule of state,\\
Like almanacks, grow out of date;\\
What then can I expect?
\end{patverse}

\begin{patverse}
Troth, in contempt of fortune’s frown,\\
I’ll get me fairly out of town,\\
And in a cloister pray\\
That since the stars are yet unkind\\
To Royalists, the King may find\\
More faithful friends than they.’’
\end{patverse}
\end{dcverse}

2. \textit{An Echo to the Cavalier’s Complaint}. Copies in The \textit{Antidote to Melancholy},
1661; \textit{Merry Drollery Complete}, 1670; \textit{New Academy of Compliments}; \&c.

\begin{dcverse}\indentpattern{001001}
\begin{patverse}
\vleftofline{“}I marvel, Dick, that having been\\
So long abroad, and having seen\\
The world, as thou hast done,\\
Thou shouldst acquaint me with a tale\\
As old as Nestor, and as stale\\
As that of priest and nun.
\end{patverse}

\begin{patverse}
Are we to learn what is a court?\\
A pageant made for Fortune’s sport,\\
Where merits scarce appear;\\
For bashful merit only dwells\\
In camps, in villages, and cells;\\
Alas! it dwells not there.
\end{patverse}
\end{dcverse}

\pagebreak
%359
%===================================================================

\begin{dcverse}\indentpattern{001001}\footnotesize
\begin{patverse}
Desert is nice in its address,\\
And merit oft-times doth oppress,\\
Beyond what guilt would do;\\
But they are sure of their demands\\
That come to court with golden hands,\\
And brazen faces too.
\end{patverse}

\begin{patverse}
The King, they say, doth still profess\\
To give his party some redress,\\
And cherish honesty;\\
But his good wishes prove in vain,\\
Whose service with his servant’s gain\\
Not always doth agree.
\end{patverse}

\begin{patverse}
All princes, be they ne’er so wise,\\
Are fain to see with others’ eyes,\\
But seldom hear at all;\\
And courtiers find’t their interest\\
In time to feather well their nest,\\
Providing for their fall.
\end{patverse}

\begin{patverse}
Our comfort doth on time depend;\\
Things, when they are at worst, will mend:\\
And let us but reflect\\
On our condition th’other day,\\
When none but tyrants bore the sway,\\
What, then, did we expect?
\end{patverse}

\begin{patverse}
Meanwhile, a calm retreat is best;\\
But discontent, if not supprest,\\
Will breed disloyalty.\\
This is the constant note I sing,—\\
I have been faithful to my king,\\
And so shall ever be.
\end{patverse}
\end{dcverse}

3. \textit{Upon Sir John Suckling’s 100 Horse}. Contained in \textit{Le Prince d’Amour}, or
\textit{The Prince of Love}, 1660, p.~148. Sir John raised a magnificent regiment of
cavalry at his own expense (12,000\textit{l}.), in the beginning of our civil wars, which
became equally conspicuous for cowardice and finery. They rendered him the
subject of much ridicule; and although he had previously served in a campaign
under Gustavus Adolphus—during which he was present at three battles, five
sieges, and as many skirmishes—his military reputation did not escape.
\indentpattern{001001}
\begin{dcverse}\footnotesize
\begin{patverse}
“I tell thee, Jack, thou gav’st the King\\
So rare a present, that nothing\\
Could welcomer have been;\\
A hundred horse! beshrew my heart,\\
It was a brave heroic part,\\
The like will scarce be seen.
\end{patverse}

\begin{patverse}
For ev’ry horse shall have on’s back\\
A man as valiant as Sir Jack,\\
Although not half so witty:\\
Yet I did hear the other day\\
Two tailors made seven run away,\\
Good faith, the more’s the pity.” \&c.
\end{patverse}
\end{dcverse}
There are seven stanzas, and then “An Answer” to it.\dcfootnote{\textit{}
These were not the only satires Sir John Suckling
had to bear. There were, at least, two others. One, to
the tune of \textit{John Dory}, begins—
\settowidth{\versewidth}{Had you seen but his look, you would sweare by the book,}
\indentpattern{0323}
\begin{fnverse}
\begin{patverse}
\vin\vin\vleftofline{“}Sir John got on a bonny brown beast,\\
To Scotland for to ride-a;\\
A brave buff coat upon his back,\\
And a short sword by his side-a.”
\end{patverse}
\end{fnverse}
The other—
\indentpattern{232323253}
\begin{fnverse}
\begin{patverse}
\vin\vin\vleftofline{“}Sir John got him an ambling nag,\\
To Scotland for to go,\\
With a hundred horse, without remorse,\\
To keep ye from the foe;\\
No carpet knight ever went to fight\\
With half so much bravado;\\
Had you seen but his look, you would\\ 
sweare by the book,\\
He’d ha’ conquer’d the whole Armado.”
\end{patverse}
\end{fnverse}
There are also two other versions of the latter; the one
beginning, “Then as it fell out on a holiday,” (see “Censura
Literaria,” vol. vi., p.~269) and the other in Percy's
\textit{Reliques of Ancient Poetry}, vol. ii., p.~326.}


4 and 5. \textit{A ballad on a Friend’s Wedding}, and \textit{Three Merry Boys of Kent},
in \textit{Folly in Print, or a Book of Rhymes}, 1667.

6. \textit{A new ballad, called The Chequers Inn}, in \textit{Poems on State Affairs}, iii. 57,
1704, It begins:—
\vspace{-\baselineskip}

\settowidth{\versewidth}{I tell thee, Dick, where I have been,}
\begin{scverse}\footnotesize
\vleftofline{“}I tell thee, Dick, where I have been,\\
Where I the Parliament have seen,” \&c.
\end{scverse}

7. \textit{A Christmas Song}, when the Rump Parliament was first dissolved, \textit{Loyal
Songs}, ii. 99, 1731.

Besides these, there is one in Carey’s \textit{Trivial Poems}, 1651; three in \textit{180 Loyal
Songs}, 1685; \&c.

%
“The grace and elegance of Sir John Suckling’s songs and ballads are inimitable.”
“They have a touch,” says Phillips, “of a gentle spirit, and seem
\pagebreak
%360
%===================================================================
to savour more of the grape than the lamp.” The author of the song above
quoted from \textit{Folly in Print}, says—
\textit{I do not write to get a name,}
%\vspace{-0.25\baselineskip}

\begin{dcverse}\vleftofline{“}I do not write to get a name,\\
At best this is but ballad-fame;\\
And Suckling hath shut up that door,\\
To all hereafter, as before.”
\end{dcverse}

%\vspace{-0.25\baselineskip}
Sir John died in 1641, at the early age of twenty-eight. The ballad is a
countryman’s description of a wedding.

\musicinfo{Smoothly.}{}
\lilypondfile[staffsize=14]{lilypond/360-i-tell-thee-dick-where-i-have-been}\normalsize

\vspace{-\baselineskip}
\begin{dcverse}\indentpattern{001}
\begin{patverse}
“At Charing Cross, hard by the way\\
Where we, thou know’st, do sell our hay,\\
There is a house with stairs;
\end{patverse}

\begin{patverse}
And there did I see, coming down,\\
Such folk as are not in our town,\\
Forty, at least, in pairs.”
\end{patverse}
\end{dcverse}
\vspace{-0.25\baselineskip}

There, are twenty-two stanzas, but some lines of the ballad might now be
considered objectionable. I have, therefore, extracted the following—a part of
the description of the bride:—
%\vspace{-0.25\baselineskip}

\settowidth{\versewidth}{Would not stay on which they did bring,}
\begin{dcverse}\indentpattern{001001}
\begin{patverse}
The maid—and thereby hangs a tale—\\
For such a maid no Whitsun-ale\\
Could ever yet produce:\\
No grape that’s kindly ripe could be\\
So round, so plump, so soft as she,\\
Nor half so full of juice.
\end{patverse}

\begin{patverse}
Her finger was so small, the ring\\
Would not stay on which they did bring,\\
It was too wide a peck:\\
And, to say truth, (for out it must,)\\
It lookt like the great collar (just)\\
About our young colt’s neck.
\end{patverse}

\indentpattern{0010010001001}
\begin{patverse}
Her feet beneath her petticoat,\\
Like little mice stole in and out,\\
As if they fear’d the light;\\
But, oh! she dances such a way,\\
No sun upon an Easter-day\\
Is half so fine a sight.\\
\quad\quad *\quad\quad *\quad\quad *\quad\quad *\quad\quad *\quad\quad *\\
Her cheeks so rare a white was on,\\
No daisy makes comparison;\\
(Who sees them is undone;)\\
For streaks of red were mingled there,\\
Such as are on a Kath’rine pear,\\
The side that’s next the sun.
\end{patverse}

\indentpattern{001001}
\begin{patverse}
Her lips were red, and one was thin,\\
Compar’d to that was next her chin;\\
Some bee had stung it newly:\\
But, Dick, her eyes so guard her face,\\
I durst no more upon them gaze.\\
Than on the sun in July.
\end{patverse}
\end{dcverse}

%
\pagebreak

%361 842pts x 570pts - 770 x 428 
%===================================================================

\musictitle{The Court Lady.}

The first ballad in the \textit{Collection of Old Ballads}, 8vo., 1727, vol. i., is “The
unfortunate Concubine, or Rosamond’s Overthrow; occasioned by her brother’s
praising her beauty to two young knights of Salisbury, as they rid on the road.
To the tune of \textit{The Court Lady}.” I have not found the ballad of \textit{The Court Lady},
but the tune is contained in \textit{The Dancing Master}, from 1650 to 1698, under the
name of \textit{Confess}, or \textit{The Court Lady}.

This ballad of Fair Rosamond is so exceedingly long (twenty-six stanzas of
eight lines, and occupying ten pages in vol. ii. of Evans’ \textit{Old Ballads}, where it is
reprinted), that the first, third, and fourth stanzas only, are here subjoined.

\musicinfo{Moderate time.}{}
\lilypondfile{lilypond/361-the-court-lady}\normalsize

\settowidth{\versewidth}{As three young knights of Salisbury}
\begin{dcverse}\begin{altverse}
As three young knights of Salisbury\\
Were riding on the way,\\
One boasted of a fair lady,\\
Within her bower so gay:\\
I have a sister, Clifford swears,\\
But few men do her know;\\
Upon her face the skin appears\\
Like drops of blood on snow.
\end{altverse}

\begin{altverse}
My sister’s locks of curled hair\\
Outshine the golden ore;\\
Her skin for whiteness may compare\\
With the fine lily flow’r;\\
Her breasts are lovely to behold,\\
Like to the driven snow;\\
I would not, for her weight in gold,\\
King Henry should her know, \&c.
\end{altverse}
\end{dcverse}
\pagebreak
%362
%===================================================================

\musictitle{Gather Your Rosebuds While You May.}

This song is in Playford’s \textit{Ayres and Dialogues}, 1659, p.~101; in Playford’s
\textit{Introduction to Music}, third edit., 1660; in \textit{Musick’s. Delight on the Cithren}, 1666;
and in \textit{The Musical Companion}, 1667. The music is the composition of William
Lawes; the poetry by Herrick. It became popular in ballad-form, and is in the
list of those printed by W. Thackeray, at the Angel in Duck Lane, as well as
in \textit{Merry Drollery Complete}, 1670. It has been reprinted (from a defective
copy) in Evans’ \textit{Old Ballads}, iii. 287, 1810. Herrick addresses it “To the
Virgins, to make much of time.” \textit{Hesperides}, i. 110,1846.

\musicinfo{}{}
\lilypondfile{lilypond/362-gather-your-rosebuds-while-you-may}\normalsize

\settowidth{\versewidth}{The glorious lamp of heaven, the sun,}
\begin{dcverse}\begin{altverse}
The glorious lamp of heaven, the sun,\\
The higher he is getting,\\
The sooner will his race be run,\\
And nearer he’s to setting.
\end{altverse}

\begin{altverse}
That age is best which is the first,\\
When youth and blood are warmer;\\
But being spent, the worse and worst\\
Times still succeed the former.
\end{altverse}

\begin{altverse}
Then be not coy, but use your time,\\
And, while ye may, go marry;\\
For having once but lost your prime,\\
You may for ever tarry.
\end{altverse}
\end{dcverse}

\musictitle{Three Merry Boys Are We.}

This is properly a round, and composed by William Lawes, who was appointed
Gentleman of the Chapel Royal in 1602. He became afterwards one of Charles
the First’s Chamber Musicians, and was killed fighting for his cause in 1645.

It is to he found in Hilton’s \textit{Catch that catch can}, 1652; in Playford’s \textit{Musical
Companion}; in \textit{Musick’s Delight on the Cithren}; \&c. The words have been
adduced by Sir John Hawkins to illustrate the \textit{Three merry men are we} quoted
by Shakespeare. See note to \textit{Twelfth Night}, act ii., sc. 3.

In \textit{Merry Drollery Complete}, 1670, is a parody on this, entitled “The Cambridge
Droll”—

\settowidth{\versewidth}{For three merry boys, and three merry boys,}
\indentpattern{101101}
\begin{dcverse}
\begin{patverse}
\vin “The ’proctors are two and no more,\\
Then hang them, that makes them three;\\
The taverns are but four,\\
\vin I wish they were more for me:\\
For three merry boys, and three merry boys,\\
And three merry boys are we.”
\end{patverse}
\end{dcverse}

%
\pagebreak%363
%===================================================================

\musicinfo{Boldly.}{}
\lilypondfile{lilypond/363-three-merry-boys-are-we}\normalsize

\settowidth{\versewidth}{And three merry girls, and three merry girls,}
\begin{dcverse}\begin{altverse}
The virtues they were seven,\\
And three the greater he;\\
The Caesars they were twelve,\\
And the fatal sisters three.\\
And three merry girls, and three merry girls,\\
And three merry girls are we.
\end{altverse}
\end{dcverse}

Another \textit{Three merry boys are we} has been already quoted (ante p.~216).

\musictitle{Cupid’s Courtesy.}

Copies of this ballad are in the Roxburghe Collection, ii. 58; and in the Douce
Collection, p.~27. It is also printed entire, with the tune, in \textit{Pills to purge
Melancholy}, vi. 43.

The copy in the Roxburghe Collection may be dated as of the reign of
Charles II., being “printed by and for W. O[nley], for A[lexander] M[ilbourne],
and sold by the booksellers;” but Mr. Payne Collier, who reprints it in his \textit{Book
of Roxburghe Ballads}, p.~80, mentions “a manuscript copy, dated 1595,” as still
extant. The words are in the same metre as \textit{Phillida flouts me}, and \textit{Lady lie near
me} (ante pages 183 and 185), but the stanzas are shorter, being of eight instead
of twelve lines. The ballad is entitled “\textit{Cupid’s Courtesie}; or The young Gallant
foil’d at his own weapon. To a most pleasant \textit{Northern tune}.”

In another volume of the Douce Collection (p.~264) is “The Young Man’s
Vindication against The Virgin’s Complaint. Tune of \textit{The Virgin’s Complaint},
or \textit{Cupid’s Courtesie};” commencing—
\settowidth{\versewidth}{Sweet virgin, hath disdain}
\begin{dcverse}\begin{altverse}
\vleftofline{“}Sweet virgin, hath disdain\\
Mov’d you to passion,—\\
Ne’er to love man again,\\
But for the fashion?” \&c.
\end{altverse}
\end{dcverse}

%
\pagebreak
%364
%===================================================================

This is also in eight-line stanzas (black-letter); and a former possessor has
penciled against the name of the tune, “\textit{I am so deep in love}” I have referred
to \textit{I am so deep in love} (ante p.~183) as probably another name for \textit{Phillida flouts
me}, but on this authority it should rather be to \textit{Cupid's Courtesy}:

\musicinfo{Smoothly.}{}
\lilypondfile{lilypond/364-cupids-courtesy}\normalsize

\settowidth{\versewidth}{“Little boy, tell me why thou art here diving;}
\begin{dcverse}\footnotesizerr
“Little boy, tell me why thou art here diving;\\
Art thou some runaway, and hast no biding?”\\
“I am no runaway; Venus, my mother,\\
She gave me leave to play, when I came hither.”

“Little boy, go with me, and be my servant;\\
I will take care to see for thy preferment.” \\
“If I with thee should go, Venus would chide me,\\
And take away my bow, and never abide me.”

“Little boy, let me know what’s thy name termed,\\
That thou dost wear a bow, and go’st so armed?”\\
“You may perceive the same with often changing,\\
Cupid it is my name; I live by ranging.”

“If Cupid be thy name, that shoots at rovers,\\
I have heard of thy fame, by wounded lovers:\\
Should any languish that are set on fire\\
By such a naked brat, I much admire.”
%\columnbreak

“If thou dost but the least at my laws grumble,\\
I’ll pierce thy stubborn breast, and make thee humble:\\
If I with golden dart wound thee but surely,\\
There’s no physician’s art that e’er can cure thee.”

“Little boy, with thy bow why dost thou threaten?\\
It is not long ago since thou wast beaten.\\
Thy wanton mother, fair Venus, will chide thee:\\
When all thy arrows are gone, thou may’st go hide thee.”

“Of powerful shafts, you see, I am well stored,\\
Which makes my deity so much adored:\\
With one poor arrow now I’ll make thee shiver,\\
And bend unto my how, and fear my quiver.”
\end{dcverse}


\pagebreak
%365
%===================================================================

\settowidth{\versewidth}{“Although thou call’st me blind, surely I’ll hit thee,}
\begin{dcverse}\footnotesizerr
“Dear little Cupid, be courteous and kindly:\\
I know thou canst not hit, but shootest blindly.” \\
“Although thou call’st me blind, surely I’ll hit thee,\\
That thou shalt quickly find; I’ll not forget thee.”

Then little Cupid caught his bow so nimble,\\
And shot a fatal shaft which made me tremble.\\
“Go, tell thy mistress dear thou canst discover\\
What all the passions are of a dying lover.”

And now his gallant heart sorely was bleeding,\\
And felt the greatest smart from love proceeding:\\
He did her help implore whom he affected,\\
But found that more and more him she rejected.

For Cupid with his craft quickly had chosen,\\
And with a leaden shaft her heart had frozen;\\
Which caus’d this lover more sadly to languish,\\
And Cupid’s aid implore to heal his anguish.

He humbly pardon crav’d for his offence past,\\
And vow’d himself a slave, and to love stedfast.\\
His pray’rs so ardent were, whilst his heart panted,\\
That Cupid lent an ear, and his suit granted.

For by his present plaint he was regarded,\\
And his adored saint his love rewarded.\\
And now they live in joy, sweetly embracing,\\
And left the little boy in the woods chasing.
\end{dcverse}

\musictitle{Have At Thy Coat, Old Woman.}

This tune is contained in every edition of \textit{The Dancing Master}, and in \textit{Musick’s
Delight on the Cithren}, 1666.

A copy of the ballad from which it derives the above name is in the Pepys
Collection, i. 284. It is—
\settowidth{\versewidth}{Who married a young man to her own undoing.}
\begin{scverse}
\vleftofline{“}A merry new song of a rich widow’s wooing,\\
Who married a young man to her own undoing.
\end{scverse}
To the tune of \textit{Stand thy ground, old Harry}.” It is a long ballad, in black-letter, 
“printed at London for T. Langley,” and commences thus:—
\settowidth{\versewidth}{Which makes me cry, with a love-sick sigh,}

\begin{dcverse}\begin{altverse}
“I am so sick for love,\\
As like was never no man, \\
Which makes me cry, with a love-sick sigh,\\
Have at thy coat, old woman.\\
\end{altverse}

Have at thy coat, old woman,\\
Have at thy coat, old woman,\\
Here and there, and everywhere,\\
Have at thy coat, old woman.”
\end{dcverse}


I have not found the ballad, \textit{Stand thy ground, old Harry}; but there is another
to the tune, under that name, in the same volume, i. 282—“A very pleasant
new ditty, to the tune of \textit{Stand thy ground, old Harry}; commencing, “Come,
hostess, fill the pot.” Printed at London for H. Gosson.

A song, commencing, “My name is honest Harry,” to the tune of \textit{Robin
Rowser}, which is in the same metre, is contained in \textit{Westminster Drollery}, 1671
and 1674; and in Dryden’s \textit{Miscellany Poems}, iv. 119. I imagine that \textit{Stand
thy ground, old Harry}, and \textit{My name is honest Harry}, are to the same tune,
although I cannot prove it. The words of the latter suit the air so exactly, that
I have here printed them with the music.

Whitlock, in his \textit{Zootomia; or Observations on the Present Manners of the
English}, 12mo., 1654, p.~45, commences his character of a female quack, with
the line, “\textit{And have at thy coat, old woman}.” In \textit{Vox Borealis}, 4to., 1641, we
find, “But all this sport was little to the court-ladies, who began to be very
melancholy for lack of company, till at last some young gentlemen revived an old
game, called \textit{Have at thy coat, old woman}.”


\pagebreak
%366
%===================================================================

\musicinfo{Merrily.}{}
\lilypondfile{lilypond/366-have-at-thy-coat-old-woman}\normalsize

\settowidth{\versewidth}{Fresh and gay as the flowers in May,}
\begin{dcverse}\begin{altverse}
My love is blithe and buxom,\\
And sweet and fine as can be,\\
Fresh and gay as the flowers in May,\\
And looks like Jack-a-dandy.
\end{altverse}

\begin{altverse}
And if she will not have me.\\
That am so true a lover,\\
I’ll drink my wine, and ne’er repine,\\
And down the stairs I’ll shove her.
\end{altverse}

\begin{altverse}
But if that she will love me,\\
I’ll be as kind as may be;\\
I’ll give her rings and pretty things,\\
And deck her like a lady.
\end{altverse}

\begin{altverse}
Her petticoat of satin,\\
Her gown of crimson tabby,\\
Lac’d up before, and spangled o’er,\\
Just like a Bart’lemew baby.
\end{altverse}

\begin{altverse}
Her waistcoat shall be scarlet,\\
With ribbons tied together;\\
Her stockings of a Bow-dyed hue,\\
And her shoes of Spanish leather.
\end{altverse}

\begin{altverse}
Her smock o’ th’ finest holland,\\
And lac’d in every quarter;\\
Side and wide, and long enough,\\
To hang below her garter.
\end{altverse}

\begin{altverse}
Then to the church I’ll have her,\\
Where we will wed together;\\
And so come home when we have done,\\
In spite of wind and weather.
\end{altverse}

\begin{altverse}
The fiddlers shall attend us,\\
And first play \textit{John come kiss me};\\
And when that we have danc’d a round,\\
They shall play \textit{Hit or miss me}.
\end{altverse}

\begin{altverse}
Then hey for little Mary,\\
Tis she I love alone, sir;\\
Let any man do what he can,\\
I will have her or none, sir.
\end{altverse}
\end{dcverse}

\musictitle{A Health To Betty.}

This tune is contained in every edition of \textit{The Dancing Master}, and in \textit{Musick’s
Delight on the Cithren}.

D’Urfey prints “The Female Quarrel: Or a Lampoon upon Phillida and
Chloris, to the tune of a country dance, call’d \textit{A health to Betty},” \textit{Pills}
ii. 110, 1719.

In the Pepys Collection, i, 274, is a ballad—“Four-pence-half-penny-farthing;
or A woman will have the oddes;” signed M[artin] P[arker]. “Printed at
London for C. W. To the tune of \textit{Bessy Bell [she doth excell]}, or \textit{A health to
Betty}.” The first verse is here printed to the tune.

In the same Collection, ii. 372, is “The Northern Turtle:
\settowidth{\versewidth}{In being deprived of his sweet mate.}
\begin{scverse}
Wayling his unhappy fate,\\
In being deprived of his sweet mate.
\end{scverse}


\pagebreak
%367
%===================================================================

To a new Northern tune, or \textit{A health to Betty}.” Printed at London for J. H.,
and beginning— “As I was walking all alone.”

In the Roxburghe Collection, i. 318, “The \textit{pair} of Northern Turtles—
\settowidth{\versewidth}{Deprived them both of life and breath.”}
\begin{scverse}
Whose love was firm, till cruel death\\
Deprived them both of life and breath.”
\end{scverse}

This is also “to a new Northern tune, or \textit{A health to Betty},” and commences—
\settowidth{\versewidth}{Farewell, farewell, my dearest dear}
\begin{scverse}
\begin{altverse}
\vleftofline{“}Farewell, farewell, my dearest dear,\\
All happiness wait on thee.”
\end{altverse}
\end{scverse}

\musicinfo{Gracefully.}{}
\lilypondfile{lilypond/367-a-health-to-betty}\normalsize

\musictitle{Shackley-Hay.}

The only copy I have found of this tune is in the Skene Manuscript, temp.
Charles~I.

It seems to derive its name from “A most excellent song of the love of young
Palmus and faire Sheldra, with their unfortunate love.” Copies of this, “to the
tune of \textit{Shackley-hay},” are in the Pepys Collection, i. 350; in the Roxburghe,
i.~436 and 472; the Bagford, fol. 75; and it is reprinted in Evans’ \textit{Old
Ballads}, i. 50.

In the Pepys Collection, i. 344, is a ballad of “Leander’s love to Hero. To
the tune of \textit{Shackley-hay}” beginning—
\settowidth{\versewidth}{“Two famous lovers once there was.”}
\begin{scverse}
“Two famous lovers once there was.”
\end{scverse}

In Westminster Drollery, 1671 and 1674, “A Song of the Declensions. The
tune is \textit{Shackle de hay},” and the same, with two others, in \textit{Grammatical Drollery},
by W.~H. (Captain Hicks), 1682.

In the Roxburghe Collection, ii. 244, and the Douce Collection, p.~109, is
“The Knitter’s Job: Or the earnest suitor of Walton town to a fair maid, with
her modest answers, and conclusion of their intents. To the tune of \textit{Shackley-hey}.”
It commences thus:—
\begin{dcverse}\indentpattern{0101005}
\begin{patverse}
“Within the town of Walton fair,\\
A lovely lass did dwell;\\
Both carding, spinning, knitting yarn,\\
She could do all full well.\\
This maid she many suitors had,\\
And some were good, and some were bad.\\
Fa, la la la la, \&c.
\end{patverse}
\end{dcverse}


\pagebreak%368
%===================================================================

The Canaries (a dance “with sprightly fire and motion,” alluded to by
Shakespeare, and which, under that name, seems always to have had the same
tune) is called “The Canaries, or The \textit{Hay}” in \textit{Musick's Handmaid}, 1678.
The figure of \textit{The Hay} was also frequently danced in country-dances; but
\textit{Shackley-hay} is the name of a place in the ballad. It is very long—twenty-four
stanzas of eight lines—I have, therefore, selected nine from the first part. The
second recounts young Palmus’s going to sea in an open boat, through fair
Sheldra’s disdain; his being wrecked and drowned, and the sea-nymphs falling in
love with him.

\musicinfo{Smoothly.}{}
\lilypondfile{lilypond/368-shackley-hay}\normalsize

\indentpattern{01010000}
\settowidth{\versewidth}{But all in vain she did complain,}
\begin{dcverse}\begin{patverse}
But all in vain she did complain,\\
For nothing could him move,\\
Till wind did turn him back again,\\
And brought him to his love.\\
When she saw him thus turn’d by fate,\\
She turn’d her love to mortal hate;\\
Then weeping, to her he did say,\\
I’ll live with thee at Shackley-hay.
\end{patverse}

\begin{patverse}
No, no, quoth she, I thee deny,\\
My love thou once did scorn,\\
And my prayers wouldst not hear,\\
But left me here forlorn.\\
And now, being turn’d by fate of wind,\\
Thou thinkst to win me to thy mind;\\
Go, go, farewell! I thee deny,\\
Thou shalt not live at Shackley-hay.
\end{patverse}
\end{dcverse}

\pagebreak%369
%===================================================================

\begin{dcverse}\footnotesizerrr\begin{patverse}
If that thou dost my love disdain,\\
Because I live on seas;\\
Or that I am a ferry-man\\
My Sheldra doth displease,\\
I will no more in that estate\\
Be servile unto wind and fate,\\
But quite forsake boats, oars, and sea,\\
And live with thee at Shackley-hay.
\end{patverse}

\begin{patverse}
To strew my boat, for thy avail,\\
I’ll rob the flowery shores;\\
And whilst thou guid’st the silken sail,\\
I’ll row with silv’ry oars;\\
And as upon the streams we float,\\
A thousand swans shall guide our boat;\\
And to the shore still will I cry,\\
My Sheldra comes to Shackley-hay.
\end{patverse}

\begin{patverse}
And, walking lazily to the strand,\\
We’ll angle in the brook,\\
And fish with thy white lily hand,\\
Thou need’st no other hook;\\
To which the fish shall soon be brought,\\
And strive which shall the first be caught;\\
A thousand pleasures will we try,\\
As we do row to Shackley-hay.

\end{patverse}
\begin{patverse}
And if we be opprest with heat,\\
In mid-time of the day,\\
Under the willows tall and great\\
Shall be our quiet bay;\\
Where I will make thee fans of boughs,\\
From Phœbus’ beams to shade thy brows;\\
And cause them at the ferry cry,\\
A boat, a boat, to Shackley-hay!
\end{patverse}

\begin{patverse}
A troop of dainty neighbouring girls\\
Shall dance along the strand,\\
Upon the gravel all of pearls,\\
To wait when thou shalt land;\\
And cast themselves about thee round,\\
Whilst thou with garlands shalt be crown’d;\\
And all the shepherds with joy shall cry,\\
O Sheldra, come to Shackley-hay!
\end{patverse}

\begin{patverse}
Although I did myself absent,\\
’Twas but to try thy mind;\\
And now thou may’st thyself repent,\\
For being so unkind.—\\
No! now thou art turn’d by wind and fate,\\
Instead of love thou hast purchas’d hate,\\
Therefore return thee to the sea,\\
And bid farewell to Shackley-hay.
\end{patverse}
\end{dcverse}

\musictitle{Franklin Is Fled Away.}

Copies of this ballad are in the Pepys Collection, ii. 76; the Roxburghe,
ii. 348; the Bagford, 643, m. 10, p.~69; and the Douce, fol. 222.

In the same volume of the Bagford Collection, p.~139, is “The two faithful
Lovers. To the tune of \textit{Franklin is fled away};” commencing—
\settowidth{\versewidth}{Farewell, my heart’s delight,}
\indentpattern{0202}
\begin{dcverse}\begin{patverse}
\vleftofline{“}Farewell, my heart’s delight,\\
Ladies, adieu!
\end{patverse}

\begin{patverse}
I must now take my flight,\\
Whate’er ensue.”
\end{patverse}
\end{dcverse}

The tune is contained in \textit{Apollo’s Banquet for the Treble Violin}, 1669; in
\textit{180 Loyal Songs}, 1685 and 1694; and in \textit{ Pills to purge Melancholy}, iii. 208, 1707;
sometimes under the name of \textit{Franklin is fled away}, and at others as \textit{O hone,
O hone}, the burden of the ballad. This burden is derived from the Irish lamentation, 
to which there were many allusions in the sixteenth and seventeenth
centuries, as in Marston’s \textit{Eastward Hoe}, act v., sc. 1; or in Gayton’s \textit{Festivous
Notes upon Don Quixote}, 1654, p.~57,—“Who this night is to be rail’d upon by
the black-skins, in as lamentable noyse as the wild Irish make their \textit{O hones}.”
A different version of the tune will be found in the ballad opera of \textit{The Jovial
Crew}, 1731, under the name of \textit{You gallant ladies all}.

A variety of songs and ballads, which were sung to it, will be found in the
above-named collections of ballads; in the \textit{180 Loyal Songs}; in Patrick Carey’s
\textit{Trivial Poems}, 1651; and in \textit{ Pills to purge Melancholy}.

The tune is one of the many from which \textit{God save the King} has been said to be
derived.
\pagebreak%370
%===================================================================


The title of the original ballad is “A mournful Caral: Or an Elegy lamenting
the tragical ends of two unfortunate faithful Lovers, Franklin and Cordeli\textit{us}; he
being slain, she slew herself with her dagger. To a new tune called \textit{Franklin is
fled away}.”

\musicinfo{Moderate time.}{}
\lilypondfile{lilypond/370-franklin-is-fled-away}\normalsize

\settowidth{\versewidth}{Franklin is fled and gone, O hone, O hone!}
\indentpattern{00220}
\begin{scverse}
\begin{patverse}
Franklin is fled and gone, O hone, O hone!\\
And left me here alone, O hone, O hone!\\
Franklin is fled away,\\
The glory of the May;\\
Who can but mourn and say, O hone, O hone!
\end{patverse}
\end{scverse}

There are six stanzas in the first, and eight in the second part. Black-letter.
Printed for M. Coles, W. Thackeray, \&c.

\musictitle{Queen Dido, Or Troy Town.}

“A ballett intituled \textit{The Wanderynge Prince}” was entered on the Registers
of the Stationers’ Company in 1564-5. This was, no doubt, the “Proper new
ballad, intituled \textit{The Wandering Prince of Troy}: to the tune of \textit{Queen Dido},” of
which there are two copies in the Pepys Collection (i. 84 and 548). Of these
copies, the first, being printed by John Wright, is probably not of earlier date
than 1620; and the second, by Clarke, Thackeray, and Passinger, after 1660.

The ballad has been reprinted in Percy’s \textit{Reliques of Ancient Poetry}, iii. 192,
1765; and in Ritson’s \textit{Ancient Songs}, ii. 141, 1829. Its extensive popularity
will be best shown by the following quotations:—“You ale-knights, you that
devour the marrow of the malt, and drink whole ale-tubs into consumptions; that
\pagebreak%371
%===================================================================
sing \textit{Queen Dido} over a cup, and tell strange news over an ale-pot\ldots\  you shall be
awarded with this punishment, that the rot shall infect your purses, and eat out
the bottom before you are aware.”—\textit{The Penniless Parliament of threadbare
Poets}, 1608. (Percy Soc. reprint, p.~44.)
\settowidth{\versewidth}{With the old footman for singing \textit{Queen Dido}?”}
\begin{scverse}
\vleftofline{Frank.—“}These are your eyes!\\
Where were they, Clora, when you fell in love\\
With the old footman for singing \textit{Queen Dido}?”\\
\attribution Fletcher’s \textit{The Captain}, act iii., sc. 3.
\end{scverse}
Fletcher again mentions it in act i., sc. 2, of \textit{Bonduca}, where Petillius says of
Junius that he is “in love, indeed in love, most lamentably loving,—to the tune
of \textit{Queen Dido}.” At a later date, Sir Robert Howard (speaking of himself)
says, “In my younger time I have been delighted with a ballad for its sake;
and ’twas ten to one but my muse and I had so set up first: nay, I had almost
thought that \textit{Queen Dido}, sung that way, was some ornament to the pen of Virgil.
I was then a trifler with the lute and fiddle, and perhaps, being musical, might
have been willing that \textit{words} should have their tones, unisons, concords, and
diapasons, in order to a poetical gamuth.”—\textit{Poems and Essays}, 8vo., 1673.

A great number of ballads were sung to the tune, either under the name of
\textit{Queen Dido} or of \textit{Troy Town}. Of these I will only cite the following:—

“The most excellent History of the Duchess of Suffolk’s Calamity. To the
tune of \textit{Queen Dido};” commencing—
\settowidth{\versewidth}{That prudent prince, King Edward, away.”}
\begin{scverse}
\vleftofline{“}When God had taken for our sin\\
\vin That prudent prince, King Edward, away.”
\end{scverse}
Contained in \textit{Strange Histories, or Songes and Sonets}, \&c., 1607; in the \textit{Crown
Garland of Golden Roses}, 1659; in the Pepys Collection, i. 544; and reprinted
in Evans’ \textit{Old Ballads}, iii. 135.

“Of the Inconveniences by Marriage. To the tuno of \textit{When Troy towne};”
beginning—\qquad\qquad “Fond, wanton youth makes love a god.”

\noindent Contained in \textit{The Golden Garland of Princely Delights}, third edit., 1620; also
set to music by Robert Jones, and printed in his \textit{First Booke of Ayres}, fol., 1601.

“The lamentable song of the Lord Wigmore, Governor of Warwick Castle,
and the fayre Maid of Dunsmoore,” \&c.; beginning—
\settowidth{\versewidth}{In Warwickshire there stands a downe,}
\begin{scverse}
\vleftofline{“}In Warwickshire there stands a downe,\\
And Dunsmoore-heath it hath to name;”
\end{scverse}
which, in the \textit{Crown Garland of Golden Roses}, 1612, is to the tune of \textit{Diana [and
her darlings dear]}; but in the copy in the Bagford Collection is to the tune of
\textit{Troy Town}. (Reprinted by Evans, iii. 226.)

“The Spanish Tragedy: containing the lamentable murder of Horatio and
Belimperia; with the pitiful death of old Hieronimo. To the tune of \textit{Queen
Dido}; beginning—\qquad\qquad “You that have lost your former joys.”

\noindent Printed at the end of the play of \textit{The Spanish Tragedy}, in Dodsley’s \textit{Old Plays},
iii.~203, 1825; and by Evans, iii. 288.

“A Looking-glass for Ladies; or a Mirror for Married Women. Tune, \textit{Queen
Dido}, or \textit{Troy Town};” commencing—
\settowidth{\versewidth}{“When Greeks and Trojans fell at strife.”}
\begin{scverse}
“When Greeks and Trojans fell at strife.”
\end{scverse}
\pagebreak%372
%===================================================================
Reprinted by Percy, under the name of \textit{Constant Penelope}, from a copy in the
Pepys Collection.

“The Pattern of True Love; or Bowes’ Tragedy,” written in 1717, and printed
in Ritson’s \textit{Yorkshire Garland}.

The last shows its popularity at a late period.

The only tune I can find for the ballad, \textit{The Wandering Prince of Troy}, is the
composition of Dr.~Wilson. It is adopted in \textit{ Pills to purge Melancholy}, iii. 15,
1707, and iv. 266, 1719; and is the \textit{Troy Town} of the ballad-operas, such as
\textit{Polly}, 1729, \&c. The ballad was entered at Stationers’ Hall before Dr.~Wilson
was born; therefore this cannot be the original tune,—unless he merely arranged
it for three voices, which we have no reason for supposing. It is printed in his
“\textit{Cheerful Ayres or Ballads}, first composed for one single voice, and since set
for three voices,” Oxford, 1660. Dr.~Rimbault has recently identified Dr.
Wilson with the “Jack Wilson” who was a singer on the stage in Shakespeare’s
time. It is possible, therefore, that he may have sung the ballad on the stage,
according to the custom of those days. Wilson was created Doctor, at Oxford,
in 1644, and died in his seventy-ninth year, \ad 1673.

There is also a song of \textit{Queen Dido}, but, being in a different metre, it could
not be sung to the same air. (See Index.) In the following, I have adopted
Dr.~Percy’s copy of the ballad, after the first stanza, which is printed with the
tune. It consists of twenty-three verses, of which eleven are subjoined; ending
with the first climax—Dido’s death.

\musicinfo{Moderate time.}{}
\lilypondfile{lilypond/372-queen-dido-or-troy-town}\normalsize

\pagebreak%373
%===================================================================


\settowidth{\versewidth}{That thou, poor wandering prince, hast had.}
\indentpattern{010100}
\begin{dcverse}\footnotesizer
\begin{patverse}
Æneas, wandering prince of Troy,\\
When he for land long time had sought,\\
At length arriving with great joy,\\
To mighty Carthage walls was brought;\\
Where Dido queen, with sumptuous feast,\\
Did entertain that wandering guest.
\end{patverse}

\begin{patverse}
And, as in hall at meat they sate,\\
The queen, desirous news to hear,\\
Says, ‘Of thy Troy’s unhappy fate\\
Declare to me, thou Trojan dear:\\
The heavy hap and chance so bad,\\
That thou, poor wandering prince, hast had.
\end{patverse}

\begin{patverse}
And then anon this comely knight,\\
With words demure, as he could well,\\
Of his unhappy ten years’ fight,\\
So true a tale began to tell,\\
With words so sweet, and sighs so deep,\\
That oft he made them all to weep.
\end{patverse}

\begin{patverse}
And then a thousand sighs he fet,\\
And every sigh brought tears amain;\\
That where he sate the place was wet,\\
As though he had seen those wars again:\\
So that the queen, with ruth therefore,\\
Said, worthy prince, enough, no more.
\end{patverse}

\begin{patverse}
And then the darksome night drew on,\\
And twinkling stars the sky bespread;\\
When he his doleful tale had done,\\
And every one was laid in bed:\\
Where they full sweetly took their rest,\\
Save only Dido’s boiling breast.
\end{patverse}

\begin{patverse}
This silly woman never slept,\\
But in her chamber, all alone,\\
As one unhappy, always wept,\\
And to the walls she made her moan;\\
That she should still desire in vain\\
The thing she never must obtain.
\end{patverse}

\begin{patverse}
And thus in grief she spent the night,\\
Till twinkling stars the sky were fled,\\
And Phœbus, with his glistering light,\\
Through misty clouds appeared red;\\
Then tidings came to her anon,\\
That all the Trojan ships were gone.
\end{patverse}

\begin{patverse}
And then the queen, with bloody knife,\\
Did arm her heart as hard as stone,\\
Yet, something loth to loose her life.\\
In woful wise she made her moan;\\
And, rolling on her careful bed,\\
With sighs and sobs these words she said .
\end{patverse}

\begin{patverse}
O wretched Dido, queen! quoth she,\\
I see thy end approacheth near;\\
For he is fled away from thee,\\
Whom thou didst love and hold so dear:\\
What! is be gone, and passed by?\\
O heart, prepare thyself to die.
\end{patverse}

\begin{patverse}
Though reason says, thou shouldst forbear,\\
And stay thy hand from bloody stroke,\\
Yet fancy bids thee not to fear,\\
Which fetter’d thee in Cupid’s yoke.\\
Come death, quoth she, resolve thy smart!\\
And with those words she pierced her heart.
\end{patverse}
\end{dcverse}

\musictitle{Remember, O Thou Man.}

This Christmas Carol is the last of the “Country Pastimes” in “Melismata:
Musicall Phansies fitting the Court, Citie, and Countrey Humours,” edited by
Ravenscroft, 4to., 1611. It is paraphrased in “Ane compendious booke of
Godly and Spirituall Songs\ldots\  with sundrie\ldots\  ballates changed out of prophaine
Sanges,” \&c., printed by Andro Hart, in Edinburgh, in 1621.
\settowidth{\versewidth}{That I thy saull from Sathan wan,}
\begin{dcverse}\vleftofline{“}Remember, man, remember, man,\\
That I thy saull from Sathan wan,\\
\columnbreak
And hes done for thee what I can,\\
\vin Thow art full deir to me,” \&c.\\
\vleftofline{\textit{Scottish Poems of t}}\textit{he Sixteenth Century}, ii. 188, 1801.
\end{dcverse}

From \textit{Melismata} the carol was copied into Forbes’ \textit{Cantus}, and taught in the
Music School at Aberdeen. Some years ago, the latter work was sold for a
comparatively high price at public auctions in London (about 10\textit{l}.), and chiefly
on the reputation of containing, in this carol, the original of \textit{God save the King},
The report originated with Mr. Pinkerton, who asserted in his \textit{Recollections of
Paris}, ii. 4, that “the supposed national air is a mere transcript of a Scottish
Anthem” contained in a collection printed in 1682. Forbes’ \textit{Cantus} is comparatively
useless to a musician, since it contains \textit{only} the “cantus,” or treble voice
\pagebreak%374
%===================================================================
part of English compositions, which were written, and should be, in three, four,
or five parts. There are, also, a few ballad tunes, such as “Satan, my foe,” to
\textit{Fortune, my foe}; “Shepherd, saw thou not,” to \textit{Crimson Velvet}, \&c.; and, in the
last edition, 1682, some Italian songs, and “new English Ayres,” in three parts
complete. The two former editions were printed at Aberdeen, in 1662 and 1666.
%

\musicinfo{Moderate time.}{}
\lilypondfile{lilypond/374-remember-o-thou-man}\normalsize

\settowidth{\versewidth}{Remember God’s goodness, how be sent his Son, doubtless,}
\begin{dcverse}\scriptsizerrr
\begin{altverse}
Remember Adam’s fall, O thou man, \&c.,\\
Remember Adam’s fall, from heaven to hell;\\
Remember Adam’s fall, how we were condemned all\\
In hell perpetual there for to dwell.
\end{altverse}

\begin{altverse}
Remember God’s goodness, O thou man, \&c.,\\
Remember God’s goodness and his promise made;\\
Remember God’s goodness, how be sent his Son, doubtless,\\
Our sins for to redress;—Be not afraid.
\end{altverse}

\begin{altverse}
The angels all did sing, O thou man, \&c.,\\
The angels all did sing upon the shepherd’s hill;\\
The angels all did sing praises to our heavenly King,\\
And peace to man living, with a good will.
\end{altverse}

\begin{altverse}
The shepherds amazed were, O thou man, \&c.,\\
The shepherds amazed were, to hear the angels sing;\\
The shepherds amazed were, how it should  come to pass\\
That Christ, our Messias, should be our King.
\end{altverse}

\begin{altverse}
To Bethlem they did go, O thou man, \&c.,\\
To Bethlem they did go, the shepherds three;\\
To Bethlem they did go, to see wh’er it were so or no,\\
Whether Christ were born or no, to set man free.
\end{altverse}

\begin{altverse}
As the angels before did say, O thou man, \&c.,\\
As the angels before did say, so it came to pass;\\
As the angels before did say, they found a babe where it lay,\\
In a manger, wrapt in hay, so poor he was.
\end{altverse}

\begin{altverse}
In Bethlem he was born, O thou man, \&c.,\\
In Bethlem he was born for mankind’s sake;\\
In Bethlem he was born, for us that were forlorn,\\
And therefore took no scorn our flesh to take.
\end{altverse}

\begin{altverse}
Give thanks to God always, O thou man, \&c.,\\
Give thanks to God always with heart most joyfully;\\
Give thanks to God alway, for this our happy day—\\
Let all men sing and say, Holy, holy.
\end{altverse}
\end{dcverse}

\pagebreak%375
%===================================================================



\musictitle{The Country Lass.}

This is the tune to which, with slight alteration, \textit{Sally in our Alley} is now sung.
Henry Carey, the author of that song, composed other music for it, which is
introduced four times in his \textit{Musical Century}. Carey’s tune is the \textit{Sally in our
Alley} of the ballad-operas that were printed from 1728 to 1760; but from the
latter period its popularity seems to have waned, and, at length, his music was
entirely superseded by this older ballad-tune.

\textit{The Countrey Lasse}, from which it derives its name, was to be sung to “a dainty
new note;” but, if unacquainted with that, the singer had the option of another
tune—\textit{The mother beguil’d the daughter}. In \textit{ Pills to purge Melancholy}, ii. 165,
1700 and 1707, it is printed (in an abbreviated form) to the one; and in \textit{The
Merry Musician, or a Cure for the Spleen},\dcfootnote{\textit{}
The first volume of The Merry Musician is dated
1716; but the second, third, and fourth, being engraved,
not set up in type like the first, bear no dates.}
 iii. 9, to the other.

In \textit{The Devil to pay}, 8vo., 1731, where Carey’s tune is printed at p.~35, as
\textit{Charming Sally}, this will be found, as \textit{What tho’ I am a Country Lass}, at p.~50.
Being unfit for dancing, the air is not contained in \textit{The Dancing Master}.

I have quoted the full title of the ballad of \textit{The Country Lass} at p.~306. The
copy in the Roxburghe Collection, i. 52, being printed by the assigns of Thomas
Symcocke, would date in or after 1620, the year of that assignment. The copy in
the Pepys Collection, i. 268, is, perhaps, an original copy. It bears the initials
of Martin Parker, the famous ballad-writer, and is evidently more correctly
printed.

The versions in \textit{ Pills to purge Melancholy}, and in \textit{The Merry Musician}, have each
had “the rust of antiquity filed from them,” and, as usual, without any improvement. 
The two first stanzas are nearly the same as in the old ballad; but the
three remaining have been re-written. The older ballad is reprinted by Evans,
i. 41, from the Roxburghe copy.

The “a” at the end of each alternate line is a very old expedient of the
ballad-maker for fitting his words to music, when an extra syllable was required.
The reader may have observed it already in \textit{John Dory, Jog on the footpath way.
Good fellows must go learn to dance}, and others. The custom is thus reproved in
“\textit{A Discourse of English Poetrie}, by William Webbe, graduate,” 1586:—“If
I let passe the \textit{un-countable rabble of ryming ballet-makers}, and compylers of
sencelesse sonets (who be most busy to stuffe every stall full of grosse devises
and unlearned pamphlets), I trust I shall, with the best sort, be held excused.
For though many such can frame an alehouse song of five or six score verses,
hobbling uppon some tune of a \textit{Northern Jygge}, or \textit{Robyn Hoode}, or \textit{La
Lubber}, \&c.: and perhappes observe just number of sillables, eight in one
line, sixe in an other, and therewithall\textit{ an ‘a’ to make a jercke in the ende}: yet
if these might be accounted poets (as it is sayde some of them make meanes to
be promoted to the Lawrell), surely we shall shortly have whole swarmes of
poets; and every one that can frame a booke in ryme, though, for want of
matter, it be but in commendations of copper noses or bottle ale, wyll catch at \pagebreak
the garlande due to poets—whose \textit{potticall} (poeticall, I should say) \textit{heades},
%376
%===================================================================
I would wyshe, at their worshipfull commencements, might, in steede of lawrell,
be gorgiously garnished with fayre greene \textit{barley}, in token of their good affection
to our Englishe malt.”
%

The following verses are selected from the older copy of the ballad. In the
\textit{Pills}, and \textit{Merry Musician}, the burden, which requires the repetition of the first
part of the tune, is omitted:—

\musicinfo{Gracefully.}{}
\lilypondfile[staffsize=15]{lilypond/376-the-country-lass}\normalsize

\settowidth{\versewidth}{Close by a crystal fountain side,}
\begin{dcverse}\footnotesizer
\begin{altverse}
What, though I keep my father’s sheep,\\
A thing that must he done-a,\\
A garland of the fairest flow’rs\\
Shall shroud me from the sun-a;\\
And when I see them feeding by,\\
Where grass and flowers spring-a,\\
Close by a crystal fountain side,\\
I sit me down and sing-a.
\end{altverse}

\begin{altverse}
Dame Nature crowns us with delight\\
Surpassing court or city,\\
We pleasures take, from morn to night,\\
In sports and pastimes pretty:\\
Your city dames in coaches ride\\
Abroad for recreation.\\
We country lasses hate their pride,\\
And keep the country fashion.
\end{altverse}

\begin{altverse}
I care not for the fan or mask.\\
When Titan’s heat reflecteth,\\
A homely hat is all I ask,\\
Which well my face protecteth;\\
Yet am I, in my country guise,\\
Esteem’d a lass as pretty,\\
As those that every day devise\\
New shapes in court or city.
\end{altverse}

\begin{altverse}
Then do not scorn the country lass,\\
Though she go plain and meanly;\\
Who takes a country wench to wife\\
(That goeth neat and cleanly),\\
Is better sped, than if he wed\\
A fine one from the city,\\
For there they are so nicely bred,\\
They must not work for pity.
\end{altverse}
\end{dcverse}
\pagebreak%377
%===================================================================
%

\musictitle{Maying-Time.}

In \textit{The Golden Garland of Princely Delights}, 3rd edit., 1620, this is entitled
“The Shepherd’s Dialogue of Love between Willy and Cuddy: To the tune of
\textit{Maying-time}.” It is also in Dryden’s \textit{Miscellany Poems}, vi. 337, and in Percy’s
\textit{Reliques of Ancient Poetry}. Percy entitles it “The Willow Tree: a Pastoral
Dialogue.”

The tune is in a manuscript dated 1639, in the Advocates’ Library, Edinburgh;
in the Skene MS.; and in all the editions of Forbes’ \textit{Cantus}.

\musicinfo{}{}
\lilypondfile[staffsize=15]{lilypond/377-maying-time}\normalsize

\begin{dcverse}\footnotesizerr
\qquad\qquad\qquad \textsc{willy.}\\
Phillis! she that lov’d thee long?\\
Is she the lass hath done thee wrong?\\
She that lov’d thee long and best,\\
Is her love turned to a jest?

\qquad\qquad\qquad \textsc{cuddy}.\\
She that long true love profest,\\
She hath robb’d my heart of rest:\\
For she a new love loves, not me;\\
Which makes we wear the willow-tree.

\qquad\qquad\qquad \textsc{willy}.\\
Come then, shepherd, let us join,\\
Since thy hap is like to mine:\\
For the maid I thought most true\\
Me hath also bid adieu.

\qquad\qquad\qquad \textsc{cuddy}.\\
Thy hard hap doth mine appease,\\
Company doth sorrow ease:\\
Yet, Phillis, still I pine for thee,\\
And still must wear the willow-tree.

\qquad\qquad\qquad \textsc{willy}.\\
Shepherd, be advis’d by me,\\
Cast off grief and willow-tree:\\
For thy grief brings her content,\\
She is pleas’d if thou lament.

\qquad\qquad\qquad \textsc{cuddy}.\\
Herdsman, I’ll be rul’d by thee,\\
There lies grief and willow-tree:\\
Henceforth I will do as they,\\
And love a new love every day.
\end{dcverse}
\pagebreak%378
%===================================================================
%

\musictitle{Never Love Thee More.}

This song, commencing, “My dear and only love, take heed,” is contained in a
manuscript volume of songs and ballads, with music, dated 1659, in the handwriting
of John Gamble, the composer. The MS. is now in the possession of
Dr.~Rimbault.

Gamble published some of his own works in 1657 and 1659, but this seems to
have been his common-place book. It contains the songs Dr.~Wilson composed
for Brome’s play, \textit{The Northern Lass}, and many compositions of H. and W.
Lawes, as well as common songs and ballads. The last are usually noted down
without bases; but, in some instances, the space intended for the tune is unfilled.

In the Pepys Collection, i. 256, is “The Faythfull Lover’s Resolution; being
forsaken of a coy and faythless dame. To the tune of \textit{My dear and only love, take
heed};” commencing, “Though booteles I must needs complain.” “Printed
at London for P. Birch.”

In the same volume, i. 280—“Good sir, you wrong your Britches;—pleasantly
discoursed by a witty youth and a wily wench. To the tune of \textit{O no, no, no, not
yet}, or \textit{Ile never love thee more};” commencing, “A young man and a lasse of
late.” “Printed at London for J[ohn] T[rundle].”

At p.~378—“Anything for a quiet life; or The Married Man’s Bondage,” \&c.
“To the tune of \textit{O no, no, no, not yet}, or \textit{Ile never love thee more}” Printed at
London by G. P.

And at p.~394—“’Tis not otherwise: Or The Praise of a Married Life. To
the tune of \textit{Ile never love thee more};” commencing, “A young man lately did
complaine.” Printed at London by G. B.

The above quotations tend to prove the tune to be of the time of James I.
Philip Birch, the publisher of the first ballad, had a “shop at the Guyldhall”
in 1618, when he published “Sir Walter Rauleigh his Lamentation,” to which
I have referred at p.~175. John Trundle, the publisher of the second, was dead
in 1628; the ballads were then printed by “M. T., widdow.” Trundle is
mentioned as a ballad-printer in Ben Jonson’s \textit{Every man in his humour}, 1598.

In the Roxburghe Collection, ii. 574, is “A proper new ballad, being the
regrate [regret] of a true Lover for his Mistris unkindness. To a new tune, \textit{Ile
ever love thee more}.” The rude orthography of this seems to mark it as an early
ballad; but, unfortunately, the printer’s name is cut away. It commences thus:
\settowidth{\versewidth}{Which could thy hands inshrine;}
\begin{dcverse}\begin{altverse}
\vleftofline{“}I wish I were those gloves, dear heart.\\
Which could thy hands inshrine;\\
Then should no sorrow, grief, or smart.\\
Molest this heart of mine,” \&c.;
\end{altverse}
\end{dcverse}
\noindent and consists of twenty-one stanzas of eight lines; thirteen in the first part, and
eight in the second.

In the same collection, and in Mr. Payne Collier’s \textit{Roxburghe Ballads}, p.~227, is
“The Tragedy of Hero and Leander. To \textit{a pleasant new tune}, or \textit{I will never love
thee more}.” The last was “printed for R. Burton, at the Horse-shoe in West-Smithfield,
neer the Hospital-gate;” and the copy would, therefore, date in the
reign of Charles I., or during the Commonwealth.
\pagebreak%379
%===================================================================

James Graham, Marquis of Montrose, also wrote “Lines” to this tune,
retaining a part of the first line, and the burden of each verse, “\textit{I’ll never love
thee more}.” It is “An Address to his Mistress,” and commences—
\settowidth{\versewidth}{My dear and only love, I \textit{pray}}
\begin{scverse}
\begin{altverse}
\vleftofline{“}My dear and only love, I \textit{pray}\\
This noble world of thee, \&c.
\end{altverse}
\end{scverse}
Like “My dear and only love, \textit{take heed} it consists of five stanzas; and must
have been written after the establishment of the Committees and the Synod of
Divines at Westminster (1643), because he refers to both in the song.

Watson in his \textit{Collection of Scotch Poems}, part iii., 1711, printed one of the
extended versions of “My dear and only love, \textit{take heed},” as a “second part” to
the Marquis of Montrose’s song; but it cannot have been written by him, as he
was only born in 1612. Neither Ritson, Robert Chambers, nor Peter Cunningham,
have followed this error; but it has been reproduced in \textit{Memoirs of Montrose},
Edinburgh, 1819.

It was, no doubt, the Marquis of Montrose’s song that made the tune popular
in Scotland. It is found, under the name of \textit{Montrose Lyns}, in a manuscript of
lyra-viol music, dated 1695, recently in the possession of Mr. A. Blaikie. The
tune has, therefore, been included in collections of Scottish music; but “My dear
and only love, \textit{take heed}” continued to be the popular song in England, and from
that it derives its name. In English ballads it is called “A rare Northern
tune,”\dcfootnote{\textit{}
In ballad-phrase, the terms “Northern” and ‘‘North-country”
were often applied to places within a hundred
miles of London. Percy describes the old ballad of \textit{Chevy
Chace} as written in “the coarsest and broadest Northern
dialect,” although Richard Sheale, the author of that version,
was a minstrel residing in Tamworth, and in the
service of the Earl of Derby. Puttenham thus notices
the difference of speech prevailing in his time beyond the
Trent:—“Our [writer] therefore at these days shall not
follow Piers Plowman, nor Gower, nor Lydgate, nor yet
Chaucer, for their language is now out of use with us:
neither shall he take the terms of North-men, such as they
use in dayly talke (whether they be noble men or gentlemen,
or of their best clarkes, all is a matter), nor in effect
any speach used beyond the river of Trent: though no man
can deny but theirs is the purer English Saxon at this day,
yet it is not so courtly nor so current as our Southerne
English is, no more is the far Western man’s speach:
ye shall therefore take the usuall speach of the Court,
and that of London and the shires lying about London,
within sixty miles, and not much above.” (\textit{Arte of
English Poesie}.) Many of the characters in plays of the
seventeenth century, such as Brome’s \textit{Northern Lass},
speak in a dialect which might often pass for Scotch with
those who are unacquainted with the language of the
time.}
 and I have never yet found that term applied to a Scotch air. Besides
Gamble’s manuscript, which contains both the words and air, the words will be
found in the first and second editions of \textit{Wit and Drollery}, 1656 and 1661,
(there entitled “A Song”); in \textit{ Pills to purge Melancholy}, 1700, 1707, and 1719.
The tune was first added to \textit{The Dancing Master} in 1686, and is contained in
every subsequent edition, in a form more appropriate to dancing than the
earlier copy.

Some of the ballads are of a later date than the Marquis of Montrose’s
song, such as “Teach me, Belissa, what to do:” to the tune of “\textit{My dear and
only love, take heed}” in \textit{Folly in print}, 1667; “A Dialogue between Tom and
Dick,” in \textit{Rats rhimed to death}, 1660; “The Swimming Lady,” in the Bagford,
others in Roxburghe and Pepys Collections; but I have already cited enough to
prove that it was a very popular air, and popular before the Marquis of Montrose’s
song can have been written.

A copy of the ballad, consisting of four verses in the first, \pagebreak and five in the
%380
%===================================================================
second part, is contained in the Douce Collection, p.~102, entitled “\textit{Ile never love
thee more}: Being the Forsaken Lover’s Farewell to his fickle Mistress. To \textit{a rare
Northern tune}, or \textit{Ile never love thee more}.“It commences, “My dear and only
\textit{joy}, take heed;” and the second part, “Ile lock myself within a cell.” Having
been “Printed for W. Whitwood, at the Golden Lyon in Duck Lane,” this Copy
may be dated about 1670. It is also in the list of those printed by W.
Thackeray at the same period. The copies in \textit{Wit and Drollery}, and in Gamble’s
MS., consist only of five stanzas.

The following copy of the tune is taken from Gamble’s MS.; the words are the
first, second, and fourth stanzas, in the order in which they stand in \textit{Wit and
Drollery}; or first, third, and fourth, in the MS. All the old copies above cited
have verbal differences, as well as differences of arrangement.

\musicinfo{Rather slowly and smoothly.}{}
\lilypondfile{lilypond/380-never-love-thee-more}\normalsize


\pagebreak%381
%===================================================================

\settowidth{\versewidth}{Nor smoothness of their language plot}
\begin{dcverse}\begin{altverse}
Let not their oaths, by volleys shot,\\
Make any breach at all,\\
Nor smoothness of their language plot\\
A way to scale the wall;\\
No balls of wild-fire-love consume\\
The shrine which I adore;\\
For, if such smoke about it fume,\\
I’ll never love thee more.
\end{altverse}

\begin{altverse}
Then if by fraud or by consent,\\
To ruin thou shouldst come,\\
I’ll sound no trumpet as of wont,\\
Nor march by beat of drum;\\
But fold my arms, like ensigns, up,\\
Thy falsehood to deplore,\\
And, after such a bitter cup,\\
I’ll never love thee more.
\end{altverse}
\end{dcverse}

\settowidth{\versewidth}{The Merchantman.}

The ballad of the \textit{Merchantman and the Fiddler’s Wife} is in the list of those
printed by Thackeray, in the reign of Charles II. It is also printed in \textit{Pills to
purge Melancholy}, iii. 153, 1707, to the following “pleasant Northern tune.”

It commences with the line, “It was a rich Merchantman,” and the ballad of
“George Barnwell” was to be sung to the tune of \textit{The rich Merchantman}. (See
Roxburghe Collection, iii. 26.) Percy prints it from another copy in the Ashmole
Collection, where the tune is entitled “\textit{The Merchant}.”

There must either be another tune called \textit{A rich Merchantman}, or else only
half the air is printed in \textit{ Pills to purge Melancholy}; for, although eight bars of
music suffice for the above-named, which are in short stanzas of four lines,
sixteen, at least, are required for other ballads, which are in stanzas of eight,
and have occasionally a burden of four more. It is not unusual to find only
the half of a tune printed in the \textit{Pills} (see, for instance, \textit{Tom of Bedlam, Green
Sleeves}, \&c.), but I know of no other version of this tune, and therefore have not
the means of testing it.

“A song of the strange Lives of two young Princes of England, who became
shepherds on Salisbury Plain, and were afterwards restored to their former
estates: To the tune of \textit{The Merchant Man}”—As contained in \textit{The Golden Garland
of Princely Delights}, 3rd edit., 1620, as well as in \textit{Old Ballads}, 2nd edit.,
iii. 5, 1738. It is in stanzas of eight lines (commencing, “In kingly Stephen’s
reign”), and reprinted, omitting the name of the tune, in Evans’ \textit{Old Ballads}
ii. 53, 1810.

“A most sweet song of an English Merchant, born at Chichester: To an
excellent new tune”—has the additional burden of four lines, and is probably
the earliest. It commences thus:—
\settowidth{\versewidth}{Did kill a man at Embden towne}
\indentpattern{010110102323}
\begin{dcverse}\begin{patverse}
\vleftofline{“}A rich merchant man there was,\\
That was both grave and wise,\\
Did kill a man at Embden towne\\
Through quarrels that did rise.\\
Through quarrels that did rise,\\
The German he was dead,\\
And for this fact the merchant man\\
Was judg’d to lose his head.\\
\textit{A sweet thing is love,\\
It rules both heart and mind,\\
There is no comfort in this world\\
‘Like’ women that are kind.”}
\end{patverse}
\end{dcverse}

Of this various copies are extant, and all apparently very corrupt. One in the
Roxburghe Collection, i. 104, is “Printed at London for Francis Coules;”
a second, in the Bagford Collection, printed for A. P.; a third, in the Pepys
Collection, by Clarke, Thackeray, and Passinger. \pagebreak Evans reprints from the last.
%382
%===================================================================
(See \textit{Old Ballads} i.~28, 1810.) It is parodied in act iii. of Rowley’s comedy,
\textit{A new Wonder, a Woman never vext}, 1632; and quoted in\textit{ The Triumphant
Widow}, 1677:—
\settowidth{\versewidth}{There was a rich merchant man,}
\begin{dcverse}\begin{altverse}
\vleftofline{“}There was a rich merchant man,\\
That was both great and wise,\\
He kill’d a man in Athens town,\\
Great quarrels there did arise,” \&c.
\end{altverse}
\end{dcverse}


\textit{A rich Merchantman} is one of the tunes to a song in \textit{The Famous Historie of
Fryer Bacon}, \textsc{b.l.}, 4to, n.d.; and \textit{There was a rich Merchantman} to a ballad in
the Pepys Collection, ii. 190. Others (under the one name or the other) will
be found in the Roxburghe Collection, i. 286 and 444, ii. 242, \&c.

\musicinfo{}{}
\lilypondfile{lilypond/382-the-merchantman}\normalsize

\musictitle{Fair Margaret And Sweet William.}

Copies of this ballad are in the Douce Collection, fol. 72, and in the Collection
of Mr. George Daniel; also in Percy’s \textit{Reliques of Ancient Poetry}.

Percy says, “This seems to be the old song quoted in Fletcher’s \textit{Knight of the
Burning Pestle}, acts ii. and iii.; although the six lines there preserved are somewhat
different from those in the ballad as it stands at present. The lines preserved
in the play are this distich:—
\settowidth{\versewidth}{You are no love for me, Margaret,}
\begin{scverse}
\vleftofline{“}You are no love for me, Margaret,\\
I am no love for you;”
\end{scverse}
and the following stanza:—
\settowidth{\versewidth}{In came Margaret’s grimly ghost,}
\begin{dcverse}\begin{altverse}
\vleftofline{“}When all was grown to dark midnight,\\
And all were fast asleep,\\
In came Margaret’s grimly ghost,\\
And stood at William’s feet.”
\end{altverse}
\end{dcverse}

Percy adds that “these lines have acquired an importance by giving birth to one
of the most beautiful ballads in our own or any other language”—“Margaret’s
Ghost” by Mallet.

Mallet’s ballad attained deserved popularity. It was printed in various forms
on half-sheets with music, and in Watts’ \textit{Musical Miscellany}, ii. 84, 1729. The
air became known by its name, and is so published in \textit{The Village Opera}, 1729,
and in \textit{The Devil to pay}, 1731.

It was not, however, printed exclusively to this tune. \pagebreak Thomson published it
%383
%===================================================================
in his \textit{Orpheus Caledonius}, and described it, with his usual inaccuracy, as “an
old Scotch ballad, with the original Scotch tune;”—“old,” although (on the
authority of Dr.~Johnson) it was first printed in Aaron Hill’s \textit{Plain Dealer},
No. 36, July 24, 1724, and Thomson’s \textit{Orpheus} was published within six months
of that time—viz., on January 5,1725. The “original Scotch tune” of Thomson
is a version of “Montrose’s lines,” or \textit{Never love thee more}.

Another point deserving notice in the old ballad, is that one part of it has
furnished the principal subject of the modern burlesque ballad, “Lord Lovel,” and
another that of T. Hood’s song, “Mary’s Ghost.”

The copy in the Douce Collection is entitled “Fair Margaret’s Misfortune; or
Sweet William’s frightful dreams on his wedding night: With the sudden death
and burial of those noble lovers. To an excellent new tune.”

The following version of the words is from \textit{Percy’s Reliques of Ancient
Poetry}:—

\musicinfo{With expression.}{}
\lilypondfile{lilypond/383-fair-margaret-and-sweet-william}\normalsize

\settowidth{\versewidth}{There she spied sweet William and his bride,}
\begin{dcverse}\begin{altverse}
I see no harm by you, Margaret,\\
And you see none by me;\\
Before to-morrow at eight o’ the clock\\
A rich wedding you shall see.
\end{altverse}

\begin{altverse}
Fair Margaret sat in her bower-window,\\
Combing her yellow hair;\\
There she spied sweet William and his bride,\\
As they were a riding near.
\end{altverse}

\begin{altverse}
Then down she laid her ivory comb,\\
And braided her hair in twain;\\
She went alive out of her bower,\\
But ne'er came alive in’t again.
\end{altverse}

\begin{altverse}
When day was gone, and night was come,\\
And all men fast asleep,\\
Then came the spirit of fair Marg’ret,\\
And stood at William’s feet.
\end{altverse}
\end{dcverse}
\pagebreak%384
%===================================================================

\begin{dcverse}\begin{altverse}
Are you awake, sweet William? she said,\\
Or, sweet William, are you asleep?\\
God give you joy of your gay bride-bed,\\
And me of my winding-sheet.
\end{altverse}

\begin{altverse}
When day was come, and night was gone,\\
And all men wak’d from sleep,\\
Sweet William to his lady said,\\
My dear, I have cause to weep,
\end{altverse}

\begin{altverse}
I dreamt a dream, my dear lady,\\
Such dreams are never good;\\
I dreamt my bower was full of red wine,\\
And my bride-bed full of blood.
\end{altverse}

\begin{altverse}
Such dreams, such dreams, my honoured Sir,\\
They never do prove good;\\
To dream thy bower was full of red wine,\\
And thy bride-bed full of blood.
\end{altverse}

\begin{altverse}
He called up his merry men all,\\
By one, by two, and by three;\\
Saying, I’ll away to fair Marg’ret’s bower,\\
By leave of my lady.
\end{altverse}

\begin{altverse}
And when he came to fair Marg’ret’s bower,\\
He knocked at the ring;\\
And who so ready as her seven brethren.\\
To let sweet William in.
\end{altverse}

\begin{altverse}
Then he turned up the covering sheet,\\
Pray let me see the dead;\\
Methinks she looks all pale and wan,\\
She hath lost her cherry red.
\end{altverse}

\begin{altverse}
I'll do more for thee, Margaret,\\
Than any of thy kin;\\
For I will kiss thy pale wan lips,\\
Though a smile I cannot win.
\end{altverse}

\begin{altverse}
With that be spake the seven brethren,\\
Making most piteous moan:\\
You may go kiss your jolly brown bride,\\
And let our sister alone.
\end{altverse}

\begin{altverse}
If I do kiss my jolly brown bride,\\
I do but what is right;\\
I ne’er made a vow to yonder poor corpse\\
By day, nor yet by night.
\end{altverse}

\begin{altverse}
Deal on, deal on, my merry men all,\\
Deal on your cake and your wine;\\
For whatever is dealt at her funeral to-day,\\
Shall be dealt to-morrow at mine.
\end{altverse}

\begin{altverse}
Fair Margaret died to-day, to-day,\\
Sweet William died the morrow;\\
Fair Margaret died for pure true love,\\
Sweet William died for sorrow.
\end{altverse}

\begin{altverse}
Margaret was buried in the lower chancel,\\
And William in the higher;\\
Out of her breast there sprang a rose,\\
And out of his a brier.
\end{altverse}

\begin{altverse}
They grew till they grew unto the church-top,\\
And then they could grow no higher;\\
And there they tied in a true lover’s knot,\\
Which made all the people admire.
\end{altverse}

\begin{altverse}
Then came the clerk of the parish,\\
As you the truth shall hear,\\
And by misfortune cut them down,\\
Or they now had been there.
\end{altverse}
\end{dcverse}
\vfill
\center\textsc{end of volume the first.}
\vfill

%\newpage \begingroup \parindent 0pt \parskip 2ex \def\enotesize{\normalsize} \theendnotes \endgroup


\pagebreak