\musictitle{“Dargason.”\dcfootnote{
This tune is inserted in Jones’ \textit{Musical and Poetical
Relics of the Welsh Bards}, p. 129, under the name of “The
melody of Cynwyd;” and some other curious coincidences
occur in the same work. At page 172, the tune called
“The Welcome of the Hostess” is evidently our “Mitter
Rant.” At page 176, the tune called “Flaunting two,”
is,the country dance of “The Hemp Dresser, or the London
Gentlewoman.” At page 129, “The Delight of the
men of Dovey,” appears to he an inferior copy of
“Green Sleeves.” At page 174, is “Hunting the Hare,”
which we also claim. At page 162, “The Monks’ March”
(of which Jones says, “Probably the tune of the Monks
of Bangor, when they marched to Chester, about the year
603,”) is “\textit{General} Monk’s March,” published by Playford, 
and the quick part, “The Rummer;” and at page
142, the air called “White Locks” is evidently Lord
Commissioner \textit{Whitelocke’s} coranto, an account of which,
with the tune, is contained in Sir J. Hawkins’ \textit{History of
Music}, vol. iv. page 51, and in Burney’s \textit{History of Music},
vol. iii. page 378. In several of these, particularly in the
last, which is identified by the second part of the tune
(and especially by a very different version, under the same
name, in Parry’s \textit{Cambrian Harmony}, published about
fifty years ago), there is considerable variation, as may he
expected in tunes traditionally preserved for so long a
time, but their identity admits of little question. In
vol. ii., at p. 25, “The Willow Hymn” is, “By the osiers
so dank.” At p. 44, “The first of August” is, “Come,
jolly Bacchus,” with a little admixture of “In my cottage
near a wood. “At page 33, a tune called “The Britons,”
which is in \textit{The Dancing Master} of 1696, is claimed. At
p. 45, “Mopsy’s Tune, the old way,” is “The Barking
Barber,” and “Prestwich Bells” is “Talk no more of
Whig or Tory,” contained in many collections. At vol. iii.,
p. 15, “The Heiress of Montgomery” is another version
of “As down in the meadows.” At p. 16, “Captain
Corbett” is “Of all comforts I miscarried;” and at p. 49,
“If love’s a sweet passion,” is claimed.” In addition
to these, Mr. Jones has himself noticed a coincidence
between the tune called “The King’s Note,” (vol. iii.)
and “Pastyme with good Company.” Such mistakes will
always occur when an editor relies solely on tradition.}}

In Ritson’s Ancient Songs, class 4 (from the reign of Edward VI. to Elizabeth)
is “A merry ballad of the Hawthorn tree,” to be sung to the tune of \textit{Donkin
Dargeson}. This curiosity is copied from a miscellaneous collection in the Cotton
Library (Vespasian A. 25), and Ritson remarks, “This tune, whatever it was,
appears to have been in use till after the Restoration.” I have found several
copies of the tune; one is in the Public Library, Cambridge, among Dowland’s
manuscripts. The copy here given is from the Dancing Master, 1650-51, where
it is called Dargason, or the Sedany. The Sedany was a country dance, the figure
of which is described in the \textit{The Triumph of Wit, or Ingenuity displayed}, p. 206.
In Ben Jonson’s \textit{Tale of a Tub}, we find, “But if you get the lass from \textit{Dargison},
what will you do with her?” Gifford, in a note upon this passage, says, “In
some childish book of knight-errantry, which I formerly read, but which I cannot
now recall to mind, there is a dwarf of this name (Dargison), who accompanies a
lady, of great beauty and virtue, through many perilous adventures, as her guard
and guide.” In the \textit{Isle of Gulls}, played by the children of the Revels, in the
Black Fryars, 1606, may be found the following scrap, possibly of the original
ballad:
\begin{scverse}\vleftofline{“}An ambling nag, and a-down, a-down,\\
We have borne her away to \textit{Dargison}”
\end{scverse}
See also “Oft have I ridden upon my grey nag,” page 63. In the Douce collection
of Ballads (fol. 207), Bodleian Library, as well as in the Pepysian, is a song
called “The Shropshire Wakes, or hey for Christmas, being the delightful sports
of most countries, to the tune of \textit{Dargason}.” It begins thus:
\begin{scverse}“Come Robin, Ralph, and little Harry,\\
And merry Thomas to our green;\\
Where we shall meet with Bridget and Sary,\\
And the finest girls that e’er were seen.\\
Then hey for Christmas a once year,\\
When we have cakes, with ale and beer,\\
For at Christmas ‘every day’\\
Young men and maids ‘may dance away,’” \&c.
\end{scverse}
\pagebreak
%065
There are sixteen verses in the song. The tune is one of those which only end
when the singer is exhausted; for although, strictly speaking, it consists of but
eight bars (and in the seventh edition of \textit{The Dancing Master }only eight bars are
printed), yet, from never finishing on the key-note, it seems never to end. Many
of these short eight-bar tunes terminate on the fifth of the key, but when longer
melodies were used, such as sixteen bars, they generally closed with the key-note.
There were, however, exceptions to the rule, especially among dance tunes, which
required frequent repetition.

\musicinfo{Pastoral character.}{A Mery Ballet of the Hathorne Tre.}
\medskip

\includemusic{chappellV1015.pdf}

\backskip{1}

\settowidth{\versewidth}{But how, an they chance to cut thee do,}

\begin{dcverse}The tree made answer by and by,\\
I have cause to grow triumphantly,\\
The sweetest dew that ever be seen,\\
Doth fall on me to keep me green.

Yea, quoth the maid, but where you grow\\
You stand at hand for every blow,\\
Of every man for to be seen,\\
I marvel that you grow so green.

Though many one take flowers from me,\\
And many a branch out of my tree;\\
I have such store they will not be seen,\\
For more and more my twigs grow green.

But how, an they chance to cut thee down,\\
And carry thy branches into the town?\\
Then they will never more be seen\\
To grow again so fresh and green.
\end{dcverse}
\pagebreak
%066

\settowidth{\versewidth}{Though that you do it is no hoot,}
\begin{dcverse}
Though that you do it is no boot,\\
Although they cut me to the root,\\
Next year again I will be seen\\
To hud my branches fresh and green.

And you, fair maid, can not do so,\\
For ‘when your beauty once does go,’\\
Then will it never more be seen,\\
As I with my branches can grow green.

The Maid with that began to blush,\\
And turn’d her from the hawthorn bush;\\
She thought herself so fair and clean,\\
Her beauty still would ever grow green.\\
\hspace{8em}* * * * *\\
But after this never I could hear\\
Of this fair maiden any where,\\
That ever she was in forest seen\\
To talk again with the hawthorn green.
\end{dcverse}


The above will be found in Ritson’s Ancient Songs, in Evans’ Collection of Old
Ballads (vol. i., p. 342, 1810), and in Peele’s Works, vol. ii., p. 256, edited by
Dyce. It is included in the last named work, because in the MS. the name of
“G. Peele” is appended to the song, but by a comparatively modern hand. The
Rev. Alexander Dyce does not believe Peele to have been the author, and Ritson,
who copied from the same manuscript, does not mention his name.

\musictitle{Shall I Go Walk the Woods So Wild?}

This is mentioned in the Life of Sir Peter Carew as one of the Freemen’s Songs,
which he used to sing with Henry VIII.—(See page 52). It must have enjoyed
an extensive and long-continued popularity, for there are three different arrangements
of it in Queen Elizabeth’s Virginal Book, all by Byrde; it is in Lady
Neville’s Virginal Book; in \textit{Pammelia} (1609) it is one of the three tunes that
could be sung together; and it is in \textit{The Dancing Master}, from the first edition,
in 1650, to that of 1690. In the edition of 1650, it is called \textit{Greenwood}, and in
some of the later copies, \textit{Greenwood, or The Huntsman}.

There were probably different words to the tune, because in the Life of Sir
Peter Carew it is called “\textit{As I walked} the woods so wild;” in Lady Neville’s
Virginal Book, “\textit{Will you walk} the woods so wild?” and in \textit{Pammelia}, “\textit{Shall
I go walk},” \&c.

\musicinfo{Moderate time.}{}
\smallskip

\includemusic{chappellV1016.pdf}

\pagebreak
%067

\musictitle{JOHN DORY.}

This celebrated old song is inserted among the \textit{Freemen’s Songs} of three voices
in \textit{Deuteromelia}, 1609. It is also to be found in Playford’s \textit{Musical Companion},
1687, and for one voice in \textit{Wit and Mirth, or Pills to Purge Melancholy}, vol. i.,
1698 and 1707. It is, however, much older than any of these books. Carew,
in his Survey of Cornwall, 1602, p. 135, says, “The prowess of one Nicholas,
son to a widow near Foy, is descanted upon in an \textit{old three-man's song}, namely,
how he fought bravely at sea, with one John Dory (a Genowey, as I conjecture),
set forth by John, the French King, and after much blood shed on both sides, took
and slew him,” \&c. Carew was born in 1555. The only King John of France
died a prisoner in England, in 1364. In the play of \textit{Gammer Gurton's Needle}
there is a song, “I cannot eat but little meat,” which was sung \textit{to the tune of
John Dory}. The play was printed in 1575, but the song appears to be older.
(See page 72). Bishop Corbet thus mentions \textit{John Dory}, with others, in his
“Journey to Fraunce:”

\settowidth{\versewidth}{“But woe is me! the guard, those men of warre,}
\begin{scverse}
\vleftofline{“}But woe is me! the guard, those men of warre,\\
Who but two weapons use, beef and the barre,\\
Begun to gripe me, knowing not the truth,\\
That I had sung \textit{John Dory} in my youth;\\
Or that I knew the day when I could chaunt,\\
\textit{Chevy}, and \textit{Arthur}, or \textit{The Siege of Gaunt}.”
\end{scverse}

Bishop Earle, in his “Character of a Poor Fiddler,” says, “Hunger is the greatest
pains he takes, except a broken head sometimes, and labouring \textit{John Dory}.” In
Fletcher’s comedy \textit{The Chances}, Antonio, a humourous old man, receives a wound,
which he will only suffer to be dressed on condition that the song of \textit{John Dory} be
sung the while, and he gives 10\textit{s}. to the singers. It is again mentioned by
Fletcher in \textit{The Knight of the Burning Pestle}; by Brathwayte in \textit{Drunken
Barnaby's Journal}; in \textit{Vox Borealis, or the Northern Discoverie}, 1641; in some
verses on the Duke of Buckingham, 1628:

\begin{scverse}
\vleftofline{“}Then Viscount Slego telleth a long storie\\
Of the supplies, as if he sung \textit{John Dorie};”
\end{scverse}

\noindent and twice by Gayton, in his \textit{Pleasant Notes upon Don Quixote}, 1654.

A parody was made upon it by Sir John Mennis, on the occasion of Sir John
Suckling’s troop of horse, which he raised for Charles I., running away in the
civil war, and it was much sung by the Parliamentarians at the time. In will be
found in \textit{Wit Restored}, 1658, entitled “Upon Sir John Suckling’s most warlike
preparation for the Scottish War,” and begins—

\begin{scverse}
“Sir John got him an ambling nag.”
\end{scverse}

In the epilogue to a farce called the \textit{Empress of Morocco}, 1674, intended to
ridicule a tragedy of the same name by Elk. Settle, and Sir W. Davenant’s
alteration of \textit{Macbeth} (which had been lately revived with the addition of music
by Mathew Locke), “the most renowned and melodious song of John Dory was
to be heard in the air, sung in parts by spirits, to raise the expectation and charm 
the audience with thoughts sublime and worthy of the heroic scene which follows.” %\origpage{}
It is quoted in \textit{Folly in print}, 1667; \pagebreak 
in \textit{Merry Drollery complete}, 1670; and in
%068
many songs. Dryden refers to it, as one of the most hackneyed in his time,
in one of his lampoons:
\settowidth{\versewidth}{“But Sunderland, Godolphin, Lory,}
\begin{scverse}%\vskip -12pt plus 6pt minus 6pt
\vleftofline{“}But Sunderland, Godolphin, Lory,\\
These will appear such chits in story,\\
’Twill turn all politics to jest,\\
To be repeated, like \textit{John Dory},\\
When fiddlers sing at feasts.”
\end{scverse}
The above lines were also printed under the name of the “Earl of Rochester.”

The name of the fish called John Dory, corrupted from dorée or dorn, is
another proof of the great popularity of this song.


\musicinfo{Cheerfully.}{}

\includemusic{chappellV1017.pdf}

\backskip{1}

\settowidth{\versewidth}{And when John Dory to Paris was come}

\begin{dcverse}\begin{altverse}
And when John Dory to Paris was come,\\
A little before the gate-a;\\
John Dory was fitted, the porter was witted,\\
To let him in thereat-a.
\end{altverse}

\begin{altverse}
The first man that John Dory did meet,\\
Was good King John of France-a:\\
John Dory could well of his courtesie,\\
But fell down in a trance-a.
\end{altverse}

\begin{altverse}
A pardon, a pardon, my liege and king,\\
For my merry men and me-a:\\
And all the churls in merry England\\
I’ll bring them bound to thee-a.
\end{altverse}

\begin{altverse}
And Nichol was then a Cornish man,\\
A little beside Bohyde-a;\\
And he manned forth a good black bark,\\
With fifty good oars on a side-a.
\end{altverse}

\begin{altverse}
Run up, my boy, into the main top,\\
And look what thou canst spy-a;\\
Who, ho! who, ho! a good ship I do see,\\
I trow it be John Dory-a.
\end{altverse}

\begin{altverse}
They hoist their sails, both top and top,\\
The mizen and all was tried-a;\\
And every man stood to his lot,\\
Whatever should betide-a.
\end{altverse}

\begin{altverse}
The roaring cannons then were plied,\\
And dub-a-dub went the drum-a;\\
The braying trumpets loud they cried,\\
To courage both all and some-a.
\end{altverse}

\begin{altverse}
The grappling hooks were brought at length,\\
The brown bill and the sword-a:\\
John Dory at length, for all his strength,\\
Was clapt fast under board-a.
\end{altverse}
\end{dcverse}\normalsize

\pagebreak
%069


\musictitle{Sellenger's Round, or The Beginning of the World.}
\musicinfo{Smoothly and in moderate time.}{}
\bigskip

\includemusic{chappellV1018.pdf}


\pagebreak
%070

This tune, which Sir John Hawkins thought to be “the oldest country-dance
tune now extant” (an opinion to which I do not subscribe), is to be found in
Queen Elizabeth’s and Lady Neville’s Virginal Books, in \textit{Music’s Handmaid},
1678, \&c. It is difficult to say from whom it derived its name. It might be from
“Sir Thomas Sellynger,” who was buried in St. George’s Chapel, Windsor,
before the year 1475, as appears by a brass plate there; or from Sir Antony
St. Leger, whom Henry VIII. appointed Lord Deputy of Ireland in 1540.

In \textit{Bacchus' Bountie} (4to., 1593), we find this passage: “While thus they
tippled, the fiddler he fiddled, and the pots danced for joy the old hop-about
commonly called \textit{Sellengar’s Round}.” In Middleton’s \textit{Father Hubburd’s Tales}
(1604):—“Do but imagine now what a sad Christmas we all kept in the country,
without either carols, wassail bowls, \textit{dancing of Sellenger's Round in moonshine
nights about Maypoles}, shoeing the mare, hoodman-blind, hot cockles, or any of our
Christmas gambols,—no, not not so much as choosing king and queen on Twelfth
Night!” In Heywood’s \textit{Fair Maid of the West}, part ii.:—“They have so tired
me with their moriscoes [morris dances], and I have so tickled them with our
country dances, \textit{Sellenger’s Round} and \textit{Tom Tiler}. We have so fiddled it!”

A curious reason for the second name to this tune is given in the comedy
of \textit{Lingua}, 1607. \textit{Anamnestes}: “By the same token the first tune the planets
played; I remember Venus, the treble, ran sweet division upon Saturn, the base.
The first tune they played was \textit{Sellenger’s Round}, in memory whereof, ever since,
it hath been called \textit{The Beginning of the World}.” On this, \textit{Common Sense} asks:
“How comes it we hear it not now? and \textit{Memory}, another of the characters,
says: “Our ears are \textit{so well acquainted with the sound}, that we never mark it.”
In Shirley’s \textit{Lady of Pleasure}, Lady Bornwell says that, “to hear a fellow make
himself merry and his horse with whistling \textit{Sellenger’s Round}, and to observe with
what solemnity they keep their wakes, moriscoes, and Whitsun-ales, are the \textit{only}
amusements of the country.”

It is mentioned as \textit{The Beginning of the World} by Deloney in his history of
Jack of Newbury, and .the times to which he refers are those of Henry VIII.;
but, so great was its popularity, that it is mentioned three or four times by
Heywood; also by Ben Jonson, by Taylor the water-poet, by Fletcher, Shirley,
Brome,Farquhar, Wycherley, Morley (1597), Clieveland (1677),Marmion (1641);
by the author of \textit{The Return from Parnassus}, and by many other writers.

There is a wood-cut of a number of young men and women dancing \textit{Sellenger’s
Round}, with hands joined, round a Maypole, on the title page of a black letter
garland, called “The new Crown Garland of princely pastime and mirth,” printed
by J. Back, on London Bridge. In the centre are two musicians, the one playing
the fiddle, the other the pipe, with the inscription, “Hey for Sellenger’s Round!”
above them.

As the dance was so extremely popular, I shall, in this instance, give the figure
from the \textit{The Dancing Master} of 1670, where it is described as a round dance
“for as many as will.”

“Take hands, and go round twice: \pagebreak
back again. All set and turn sides: that
%071
again. Lead all in a double forward and back: that again. Two singles and a
double back, set and turn single: that again. Sides all: that again. Arms all:
that again. As before, as before.” Country dances were formerly danced as
often in circles as in parallel lines.

The following songs were sung to the tune:—“The merry wooing of Robin and
Joan, the West-country Lovers, to the tune of the Beginning of the World, or
Sellenger’s Round.”—\textit{Roxburgh Collection}. “The Fair Maid of Islington, or the
London Vintner over-reached,” in the \textit{Bagford Collection}. “Robin’s Courtship,”
in \textit{Wit Restored}, 1658.

As a specimen of old harmony, I have added the arrangement of Sellenger’s
Round by Byrd, from Queen Elizabeth’s Virginal Book. Having an instrument
that would not sustain the tone (for the virginals, like the harpsichord, only
twitted the wires with a quill) it is curious to see how he has filled up the harmony
by an inner part, that seems intended to imitate the prancing of the hobby-horse.
The hobby-horse was the usual attendant on May-day and May Games.


\musicinfo{In moderate time.}{with the old harmony by byrd.}
\smallskip

\includemusic{chappellV1019.pdf}
\begin{center}
\scriptsize $^{a}$ Hobby-horse.
\end{center}
\normalsize

\pagebreak
%072

\musictitle{I CANNOT EAT BUT LITTLE MEAT.}

This song was sung in “a right pithy, pleasant, and merry comedy,” called
\textit{Gammer Gurton’s Needle}, which was printed in 1575, but the Rev. Alex. Dyce
has given a copy of double length from a manuscript in his possession, and
“certainly of an earlier date than the play.” It may be seen in his account of
Skelton and his writings, vol. i., p. 7. I have selected four from the eight
verses, as sufficiently long for singing. Warton calls it “the first drinking song of
any merit in our language.” In early dramas it was the custom to sing old songs,
or to play old tunes, both at the commencement and at the end of the acts. For
instance, in \textit{Summer’s Last Will and Testament}, which was performed in 1593,
the direction to the actors in the Prologue is to begin the play with “a fit
of mirth and an old song:” and at the end of the comedy, \textit{Ram Alley}, “strike up
music; let’s have an old song.” In \textit{Peele’s Arraignment of Paris}, Venus “singeth
an old song, called \textit{The wooing of Colman}.” In Marston’s \textit{Antonio and Mellida},
Feliche sings “the old ballad, \textit{And was not good king Solomon}.” To these instances
many others might be added; indeed, in the very play (\textit{Gammer Gurto}n),
at the end of the second act, Diccon says:—

\settowidth{\versewidth}{“In the mean time, fellows, pipe up your fiddles, I say take them}
\begin{scverse}
“In the mean time, fellows, pipe up your fiddles, I say take them\\
And let your friends have such mirth as ye can make them.”
\end{scverse}

The tune is printed in Stafford Smith’s Musica Antiqua, and in Ritson’s English
Songs. Ritson says: “Set, four parts in one, by Mr. Walker, before the year
1600.” And Smith, not knowing, I suppose, who Mr. Walker was, seems to have
guessed Weelkes; but it is the old tune of John Dory in common time.


\musicinfo{In moderate time, and well marked.}{}

\includemusic{chappellV1020.pdf}


\settowidth{\versewidth}{But belly, God send thee good ale enough,}
\begin{dcverse}\begin{altverse}
Though I go bare, take ye no care,\\
‘For I am never’ cold:\\
I stuff my skin so full within\\
Of jolly good ale and old.\\
Back and side, go bare, go bare,\\
Both foot and hand go cold:\\
But belly, God send thee good ale enough,\\
Whether it be new or old.
\end{altverse}

\begin{altverse}
I love no roast, but a nut-brown toast,\\
And a crab laid in the fire,\\
A little bread shall do me stead,\\
Much bread I never desire.\\
No frost nor snow, nor wind, I trow,\\
Can hurt me, if it would;\\
I am so wrapp’d, so thoroughly lapp’d\\
With jolly good ale and old.\\
\hspace{6em}Back and side, \&c.
\end{altverse}
\end{dcverse}
\pagebreak
%073

\begin{dcverse}
\begin{altverse}
I care right nought, I take no thought\\
For clothes to keep me warm,\\
Have I good drink I surely think\\
‘That none’ can do me harm.\\
For truly then I fear no man,\\
‘Though never he’ so bold,\\
When I am arm’d and thoroughly warm’d\\
With jolly good ale and old.\\
\hspace{6em}Back and side, \&c.
\end{altverse}

\begin{altverse}
Now let them drink till they nod and wink,\\
Even as good fellows should do,\\
They shall not miss to have the bliss\\
Good ale doth bring men to;\\
And all poor souls that scour black bowls,\\
Or have them lustily troled,\\
God save the lives of them and their wives,\\
Whether they be young or old.\\
\hspace{6em}Back and side, \&c.
\end{altverse}
\end{dcverse}
\normalsize


\musictitle{Hanskin, or Half Hannikin.}

In Queen Elizabeth’s Virginal Book there is a tune called \textit{Hanskin}, and in all
the early editions of \textit{The Dancing Master}, viz., from 1650 to 1690, one called
\textit{Half Hannikin}. Hankin or Hannikin was the common name of a clown:

\settowidth{\versewidth}{“Thus for her love and loss poor \textit{Hankin} dies;}
\begin{scverse}
“Thus for her love and loss poor \textit{Hankin} dies;\\
His amorous soul down flies\\
To th’ bottom of the cellar, there to dwell:\\
Susan, farewell, farewell!”—\textit{Musarum Delicicæ}, 1655.
\end{scverse}

And \textit{Hankin Booby} was used as term of contempt. Nash, meaning to call his
opponent a Welsh clown, calls him a “Gobin a Grace ap Hannikin,” and says,
“No vulgar respects have I, what Hoppenny Hoe and his fellow \textit{Hankin Booby}
think of me.” (\textit{Have with you to Saffron-Waldon}, 1596.)

We find \textit{Hankin Booby} mentioned as a tune in the interlude of Thersytes,
which was written in 1537:

\settowidth{\versewidth}{“And we wyll have minstrelsy}
\begin{scverse}
“And we wyll have minstrelsy\\
That shall pype \textit{Hankin boby}.”
\end{scverse}

\noindent Skelton, in his \textit{Ware the Hauke}, says:
\settowidth{\versewidth}{“With troll, cytrace, and trovy}
\begin{dcverse}“With troll, cytrace, and trovy,\\
They ranged, \textit{hankin bovy},\\
My churche all aboute.\\
This fawconer then gan showte,\\
These be my gospellers,

These be my pystillers, [epistlers]\\
These be my querysters [choristers]\\
To help me to synge,\\
My hawkes to mattens rynge.\\
\hspace{-1em}\textit{Skelton’s Works, Ed. Dyce}, vol. i., p. 159.
\end{dcverse}

\noindent By an extract from Sir H. Herbert’s office-book of revels and plays performed at
Whitehall at Christmas, 1622-3, quoted by Mr. Collier, in his Annals of the
Stage, we find that on Sunday, 19th Jan., 1623, after the performance of Ben
Jonson’s masque, \textit{Time Vindicated}, “The Prince did lead the measures with the
French Ambassador’s wife,” and “the measures, braules, corrantos, and galliards,
being ended, the masquers, with the ladies, did daunce two countrey dances,
namely, \textit{The Soldier’s Marche} and \textit{Huff Hamukin}.” I believe that by \textit{Huff
Hamukin}, \textit{Half Hannikin} is intended, the letters are so nearly alike in form, and
might be so easily mistaken. In Brome’s \textit{Jovial Crew}, 1652,—“Our father is so
pensive that he makes us even sick of his sadness, that were wont to ‘See my
gossip’s cock to-day,’ mould cocklebread, daunce \textit{Clutterdepouch} and \textit{Hannykin
booby}, bind barrels, or do anything before him, and he would laugh at us.”

The tune called \textit{Hanskin} in Queen \pagebreak 
Elizabeth’s Virginal Book is the same as
%074
“Jog on, the foot-path way,” and will be found in this collection among the airs
that are mentioned by Shakspeare. The following is \textit{Half Hannikin}, from \textit{The
Dancing Master}.

\includemusic{chappellV1021.pdf}

\musictitle{Malt’s Come Down.}

This is one of the tunes in Queen Elizabeth’s Virginal Book, where it is
arranged by Byrd. The words are from \textit{Deuteromelia}, 1609, but it appears that
Ravenscroft, in arranging it as a round, has taken only half the tune.

\includemusic{chappellV1022.pdf}

\settowidth{\versewidth}{The greatest drunkards in the town}
\indentpattern{006}
\begin{scverse}
\begin{patverse}
The greatest drunkards in the town\\
Are very glad that malt’s come down.\\
Malt’s come down, \&c.
\end{patverse}
\end{scverse}
\pagebreak
%075

\musictitle{of all the birds.}

In Beaumont and Fletcher’s play, \textit{The Knight of the Burning Pestle}, Old
Merrythought sings many snatches of old songs, and among others—

\begin{scverse}
“Nose, nose, jolly red nose\\
And who gave thee this jolly red nose? \\
Cinnamon, ginger, nutmegs and cloves\\
And they gave me this jolly red nose;”
\end{scverse}

\noindent which are the four last lines of this song. It is one of the King Henry’s Mirth
or Freemen’s Songs in \textit{Deuteromelia}, 1609.

\includemusic{chappellV1023.pdf}

\pagebreak
%076


\musictitle{Who's the fool now}

This tune is in Queen Elizabeth’s Virginal Book, and it is one of the Freemen’s
Songs in \textit{Deuteromelia}, 1609. It was entered on the books of the Stationer’s
Company as a ballad in 1588, when Thomas Orwyn had a license to print it; and
it is alluded to in Dekker’s comedy, \textit{Old Fortunatus}, where \textit{Shadow} says: “Only
to make other idiots laugh, and wise men to cry ‘\textit{Who's the fool now}?’” which
is the burden of every verse. It is thought to be a satire upon those who tell
wonderful stories.

\includemusic{chappellV1024.pdf}


\indentpattern{04040014}
\begin{dcverse}\begin{patverse}
I saw the man in the moon;\\
Fie! man, fie!\\
I saw the man in the moon;\\
Who’s the fool now?\\
I saw the man in the moon\\
Clouting of St. Peter’s shoon;\\
Thou hast well drunken, man—\\
Who’s the fool now?
\end{patverse}

\begin{patverse}
I saw a hare chase a hound;\\
Fie! man, fie!\\
I saw a hare chase a hound;\\
Who’s the fool now?\\
I saw a hare chase a hound,\\
Twenty miles above the ground;\\
Thou hast well drunken, man—\\
Who’s the fool now?
\end{patverse}

\begin{patverse}
I saw a goose ring a hog;\\
Fie! man, fie!\\
I saw a goose ring a hog;\\
Who’s the fool now?\\
I saw a goose ring a hog,\\
And a snail bite a dog;\\
Tho hast well drunken, man—\\
Who’s the fool now?
\end{patverse}

\begin{patverse}
I saw a mouse catch a cat;\\
Fie! man, fie!\\
I saw a mouse catch a cat;\\
Who’s the fool now?\\
I saw a mouse catch a cat,\\
And the cheese eat the rat;\\
Thou hast well drunken, man—\\
Who’s the fool now?
\end{patverse}
\end{dcverse}

\pagebreak
%077

\musictitle{We be soldiers three.}

This is also one of the King Henry’s Mirth or Freemen’s Songs in \textit{Deuteromelia},
1609, and will be found as a song in \textit{Wit and Mirth, or Pills to Purge
Melancholy}, vol. i., 1698 and 1707.

\includemusic{chappellV1025.pdf}

\begin{dcverse}
\begin{altverse}
Here, good fellow, I drink to thee,\\
\textit{Pardona moy, je vous an pree;}\footnotemark\\
To all good fellows, wherever they be,\\
With never a penny of money.
\end{altverse}

\begin{altverse}
And he that will not pledge me this,\\
\textit{Pardona moy, je vous an pree,}\\
Pays for the shot whatever it is,\\
With never a penny of money.
\end{altverse}
\end{dcverse}


\begin{scverse}
\begin{altverse}
Charge it again, boy, charge it again,\\
\textit{Pardona moy, je vous an pree;}\\
As long as there is any ink in thy pen,\\
With never a penny of money.
\end{altverse}
\end{scverse}

\footnotetext{\scriptsize “These \textit{pardonnez-moy's} who stand so much on the
new form.”—\textit{Romeo and Juliet}, act ii., sc. 4. Dr. Johnson
in a note says; “\textit{Pardonnez moi} became the language
of doubt or hesitation among men of the sword, when the
point of honour was grown so delicate that no other mode
of contradiction would be endured.”}

\musictitle{We be Three Poor Mariners}

This is one of the King Henry’s Mirth or Freemen’s Songs in \textit{Deuteromelia},
1609, and is to be found as a dance tune in the Skene MS. (about 1630), called
\textit{Brangill of Poictu},—\ie, Branle, or Braule of Poictu.

Braules\dcfootnote{Braules, which, Mr. M. Mason observes, seem to be
what we now call cotillons, are described by Philips as
“a kind of dance in which several persons danced together
in a ring, holding one another by the hand.” In Marston’s
play of \textit{The Malcontent} there is a minute, but perhaps
not now very intelligible description of the figures. See
Dodsley's Collection of old Plays, vol.~iv. Braules are
alluded to by Shakespeare, Ben Jonson, Massinger, and
others.} were dances much in vogue with the upper classes during the sixteenth
and seventeenth centuries. Their being danced at Whitehall in 1623, has
been mentioned at page 73; and Pepys speaks of them at the Court of Charles II.
Branle de Poictu is explained by Morley (1597) as meaning the Double Branle,
in contradistinction to the French Branle, or Branle-Simple.

Another Branle de Poictu (quite a different tune) will be found in the Straloch
Manuscript, for the name was given to \pagebreak any air used for the dance. It was so
%078
usual in England, formerly, to make dances out of such song and ballad tunes as
were of a sufficiently cheerful character, that nearly \textit{every} air in the \textit{first} edition
of \textit{The Dancing Master}, 1650-51, can be proved to be that of a song or “ballet”
of earlier date than the book. It has for that reason been so valuable an aid in
the present collection. About 1690, tunes composed \textit{expressly} for dancing were
becoming more general, and in the editions of \textit{The Dancing Master} from 1715
to 1728, the song and the dance tunes are nearly equally divided.

\includemusic{chappellV1026.pdf}

\settowidth{\versewidth}{\vin Come pledge me on the ground, aground, aground.}
\begin{scverse}\begin{altverse}
We care not for those martial men\\
That do our states disdain;\\
But we care for the merchantmen\\
Who do our states maintain.\\
To them we dance this round, around, around,\\
To them we dance this round;\\
And he that is a bully [jolly] boy,\\
Come pledge me on the ground, aground, aground.
\end{altverse}
\end{scverse}
\pagebreak
%079

\musictitle{My Little Pretty One}

This ancient melody is also transcribed from a MSS. of the time of Henry the
Eighth (No. 4900, Additional MS., Brit. Mus.). The original is, as usual, without
bars, but with an accompaniment in tablature for the lute. In the same
volume are songs by John Taverner, Shepherde, Heywood, \&c. It has the same
peculiarity as the dance tune at page 27, each part consisting of nine bars. A
song called “My little pretty one” is in the Koxburgh Collection of Ballads,
“to a pleasant new tune,” but the measure is different.

\includemusic{chappellV1027.pdf}

\musictitle{Robin, Lend to Me thy Bow}

This song is still known in some parts of the country, and was written down for
me by a friend, in Leicestershire, some years ago. In the “very mery and pithie
commedie” called \textit{The longer thou livest the more fool thou art}, there is a
stage direction—“Here entreth Moros, counterfaiting a vaine gesture and foolish
countenance, synging the foote [burden] of many songes, as fooles were wont.”
Among the burdens is the following:—

\begin{scverse}
\vleftofline{“}Robin, lende me thy bowe, thy bowe,\\
Robin, the bow, \textit{Robin, lend to me thy bow-a}.”
\end{scverse}
\pagebreak
%080

The play was entered at Stationers’ Hall in 1568-9. “That it was a popular
song in the beginning of Queen Elizabeth’s reign appears also from its being
mentioned, amongst others, in a curious old musical piece (MS. Harl. 7578),
containing the description and praises of the city of Durham, written about that
time.” It is to be found as one of the “pleasant roundelayes” in \textit{Pammelia},
1609, and has likewise been printed by Ritson, in his Ancient Songs. The tune
differs slightly from the copy in \textit{Pammelia}, but I think for the better.


\musicinfo{Smoothly and slow.}{}

\includemusic{chappellV1028.pdf}


\begin{scverse}
\begin{altverse}
And whither will thy Lady go?\\
Sweet Wilkin, tell it unto me;\\
And thou shalt have my hawke, my hound, and eke my bow,\\
To wait on thy Lady.
\end{altverse}

\begin{altverse}
My Lady will to Uppingham,\footnotemark\\
To Uppingham forsooth will shee;\\
And I myself appointed for to be the man\\
To wait on my Lady.
\end{altverse}

\begin{altverse}
Adieu, good Wilkin, all beshrewde,\\
Thy hunting nothing pleaseth mee;\\
But yet beware thy babling hounds stray not abroad\\
For ang’ring of thy Lady.
\end{altverse}

\begin{altverse}
My hounds shall be led in the line,\\
So well I can assure it thee;\\
Unless by straine of view some pursue I may finde,\\
To please my sweet Ladye.
\end{altverse}

\begin{altverse}
With that the Lady shee came in,\\
And will’d them all for to agree;\\
For honest hunting never was accounted sinne.\\
Nor never shall for mee.
\end{altverse}
\end{scverse}

\footnotetext{\scriptsize A market-town in Rutlandshire.}

\pagebreak
%081

\musictitle{Who Liveth So Merry in All This Land?}

This is also one of the \textit{King Henry’s Mirth or Freemen’s Songs}, in \textit{Deuteromelia}.
In the first year of the Registers of the Stationers’ Company (1557-58)
there is an entry of a license to Mr. John Wallye and Mrs. Toye to print a
“Ballette” called
\begin{scverse}
“Who lyve so mery and make such sporte,\\
As thay that be of the poorest sorte?”
\end{scverse}

These lines will be found in the last verse of the song, and were probably printed
at the head of it as the title. Ballets were songs of a cheerful character, which
being “sung to a ditty may likewise be danced.” So the “Merry Ballet of the
Hawthorn Tree” (see page 64), was to be sung to the tune of \textit{Dargason}, which is
also mentioned as a dance tune.

The following song will also be found in \textit{Wit and Drollery, Jovial Poems}, p. 252,
and in \textit{Wit and Mirth, or\textit{ Pills to purge Melancholy}}, vol. i., 1698 and 1707. In
\textit{Wit and Drollery}, as well as in \textit{Deuteromelia}, the third and fourth lines of each
verse are marked to be sung in chorus.


\musicinfo{Moderate time.}{}
\bigskip

\includemusic{chappellV1029.pdf}


\begin{scverse}\indentpattern{0011}
\begin{patverse}
The broom-man maketh his living most sweet.\\
With carrying of brooms from street to street.\\
\vleftofline{Chorus.—}Who would desire a pleasanter thing\\
Than all the day long to do nothing but sing?
\end{patverse}

The chimney-sweeper all the long day,\\
He singeth and sweepeth the soot away;\\
\vleftofline{Ch.—}Yet when he comes home, although he be weary,\\
With his pretty, sweet wife he maketh full merry.

The cobbler he sits cobbling till noon,\\
And cobbles his shoes till they be done;\\
\vleftofline{Ch.—}Yet doth he not fear, and so doth say,\\
For he knows that his work will soon decay.

The merchantman he doth sail on the seas,\\
And lie on the ship-board with little ease;\\
\vleftofline{Ch.—}For always he doubts that the rocks are near,—\\
How can he he merry and make good cheer?
\end{scverse}

\pagebreak
%82

\begin{scverse}
The husbandman all day goeth to plough,\\
And when he comes home he serveth his sow;\\
\vleftofline{Ch.—}He moileth and toileth all the long year,—\\
How can he he merry and make good cheer?

The serving-man waiteth from street to street,\\
Either blowing his nails or beating his feet;\\
\vleftofline{Ch.—}Yet all that serves for, four angels\scfootnote{
\scriptsize The angel was a gold coin worth about ten shillings, so named 
from having the representation of an angel upon it.} a year,\\
Impossible ’tis that he make good cheer.


Who liveth so merry and maketh such sport\\
As those that be of the poorest sort?\\
\vleftofline{Ch.—}The poorest sort, wheresoever they be,\\
They gather together by one, two, and three.

And every man will spend his penny,\\
What makes such a shot among a great many.
\end{scverse}



\musictitle{To-Morrow the Fox Will Come to Town, or Trenchmore.}

In \textit{The Dancing Master} this tune is called \textit{Trenchmore}. In \textit{Deuteromelia} it is
one of the \textit{King Henry’s Mirth or Freemen’s Songs}, under the name of “Tomorrow
the fox will come to town.”

In a Morality, by William Bulleyn, called \textit{A Dialogue both pleasant and pietyfull,
wherein is a goodly regimen against the fever pestilence}, \&c., 1564, a minstrel
is thus described: “There is one lately come into the hall, in a green Kendal coat,
with yellow hose; a beard of the same colour, only upon the upper lip; a russet
hat, with a great plume of strange feathers; and a brave scarf about his neck;
in cut buskins. He is playing at the \textit{trea trippe} with our host’s son; he playeth
trick upon the gittern, daunces \textit{Trenchmore} and \textit{Heie de Gie}, and telleth news
from Terra Florida.”

Taylor, the water-poet, in \textit{A Merry Wherry-ferry Voyage}, says:
\begin{scverse}
“Heigh, \textit{to the tune of Trenchmore} I could write\\
The valiant men of Cromer’s sad affright;”
\end{scverse}
\noindent and in \textit{A Navy of Land Ships}, 1627, “Nimble-heel’d mariners, like so many
dancers, capering a morisco [morris dance], or \textit{Trenchmore} of forty miles long,
to the tune of ‘Dusty, my dear,’ ‘Dirty, come thou to me,’ ‘Dun out of the mire,’
or ‘I wail in woe and plunge in pain:’ all these dances have no other music.”
Deloney, in his \textit{History of the gentle craft}, 1598, says: “like one dancing the
\textit{Trenchmore}, he stamp’d up and down the yard, holding his hips in his hands.”

Burton, in his \textit{Anatomy of Melancholy}, 1621, says that mankind are at no
period of their lives insensible to dancing. “Who can withstand it? be we young
or old, though our teeth shake in our heads like Virginal Jacks, or stand parallel
asunder like the arches of a bridge,—there is no remedy: we must dance \textit{Trenchmore}
over tables, chairs, and stools.” The following amusing description is from
Selden’s \textit{Table Talk}:

“The court of England is much alter’d. At a solemn dancing, first you had the
grave measures, then the corantoes and the galliards, and this kept up with ceremony; 
and \textit{at length to Trenchmore} and \pagebreak \textit{the Cushion Dance}: then all the company dances, 
%083
lord and groom, lady and kitchen maid, no distinction. So in our court in Queen
Elizabeth’s time, \textit{gravity and state were kept up}. In King James’s time things were
pretty well, but in King Charles’s time, there has been \textit{nothing but} Trenchmore and
the Cushion Dance, omnium gatherum, tolly polly, hoite come toite.”



\textit{Trenchmore} is mentioned also in Stephen Gosson’s Schoole of Abuse, 1579; in
Heywood’s \textit{A Woman Killed with Kindness}, 1600; in Chapman’s \textit{Wit of a
Woman}, 1604; in Barry’s \textit{Ram Alley}, 1611; in Beaumont and Fletcher’s \textit{Island
Princess}; in Weelkes’ \textit{Ayres or Phantasticke Sprites}, 1608; and in 1728 was
still to he found in \textit{The Dancing Master}. In the comedy of \textit{The Rehearsal},
1672, the earth, sun, and moon, are made to dance \textit{the Hey} to the tune of
\textit{Trenchmore}.

Several political songs were sung to it, one of which is in the collection of
“Poems on Affairs of State, from 1640 to 1704.” In the Roxburghe Collection
of Ballads is one called “The West-country Jigg, or a Trenchmore Galliard,”
“Four-and-twenty lasses went over Trenchmore Lee.”

The following is the song in \textit{Deuteromelia}.


\musicinfo{Moderate time.}{}
\smallskip

\includemusic{chappellV1030.pdf}


\normalsize

\pagebreak
%084

\settowidth{\versewidth}{He’ll steal the cock out from his flock,}
\indentpattern{01014}
\begin{dcverse}\begin{patverse}
He’ll steal the cock out from his flock,\\
Keep, keep, keep, keep, keep;\\
He'll steal the cock e’en from his flock,\\
O keep you all well there.\\
I must desire you, \&c.
\end{patverse}

\begin{patverse}
He’ll steal the hen out of the pen,\\
Keep, keep, \&c.;\\
He’ll steal the hen out of the pen,\\
O keep you all well there.\\
I must desire you, \&c.
\end{patverse}

\begin{patverse}
He’ll steal the duck out of the brook,\\
Keep, keep, \&c.;\\
He’ll steal the duck out of the brook,\\
O keep you all well there.\\
I must desire you, \&c.
\end{patverse}

\begin{patverse}
He’ll steal the lamb e’en from his dam,\\
Keep, keep, \&c.;\\
He’ll steal the lamb e’en from his dam,\\
O keep you all well there.\\
I must desire you, \&c.
\end{patverse}
\end{dcverse}

\musictitle{The Shaking of the Sheet, or the Dance of Death.}

This is frequently mentioned by writers in the sixteenth and seventeenth centuries, 
both as a country dance and as a ballad tune. In the recently-discovered
play of \textit{Misogonus}, produced about 1560,\dcfootnote{
See Collier’s \textit{History of Early Dramatic Poetry}, v. 2,
p.~474.} \textit{The Shaking of the Sheets}, \textit{The Vicar
of St. Fools}, and \textit{The Catching of Quails}, are mentioned as country dances.\dcfootnote{ 
Sometimes it is called \textit{The Night Piece}, or \textit{The Shaking
of the Sheets}.}
There is a manuscript copy of the ballad in the British Museum (Add. MSS.
No. 15,225), in which it is ascribed to Thomas Hill; and printed copies, in black
letter, are to be found in the Roxburghe Collection (i., 499), and in that of
Anthony à Wood, in the Ashmolean Museum, Oxford (vol. 401., f. 60). In
1568-9, it was entered at Stationers’ Hall to John Awdelay (see \textit{Collier’s
Extracts}, vol. i., p. 195).

\textit{Dance after my pipe}, which is the second title ,of the ballad, seems to have
been a proverbial expression. In Ben Jonson’s \textit{Every man out of his humour},
Saviolina says: “Nay, I cannot stay to \textit{dance after your pipe}.” In \textit{Vox Borealis},
1641,—“I would teach them to sing another song, and make them \textit{dance after
my pipe}, ere I had done with them.” And in Middleton’s \textit{The World Lost at
Tennis},—“If I should \textit{dance after your pipe} I should soon dance to the devil;”
and so in many other instances.

In \textit{The Meeting of Gallants at an Ordinary}, the host, describing a young man
who died of the plague, in London, in 1603, says: “But this youngster \textit{daunced
the shaking of one sheete} within a few daies after” (Percy Soc, Reprint, p. 20);
and in \textit{A West-country Jigg}, or a \textit{Trenchmore Galliard}, verse 5:

\settowidth{\versewidth}{“The piper he struck up,}
\begin{scverse}
\begin{altverse}
“The piper he struck up,\\
And merrily he did play\\
\textit{The Shaking of the Sheets},\\
And eke \textit{The Irish Hay}.”
\end{altverse}
\end{scverse}

The tune is also mentioned in Lilly’s \textit{Pappe with a Hatchet}, 1589; in Gosson’s
\textit{Schoole of Abuse}, 1579; by Rowley, Middleton, Taylor the water-poet, Marston,
Massinger, Heywood, Dekker, Shirley, \&c., \&c.

There are two tunes under this name, the one in William Ballet’s Lute Book,
which is the same as printed by Sir John Hawkins in his \textit{History of Music}
(vol. 2, p. 934, 8vo. edit.); the other, and in all probability the more popular one,
is contained in numerous publications,\dcfootnote{
The tune of \textit{The Catching of Quails} is also in
\textit{The Dancing
Master}.} from \textit{The Dancing Master} of 1650-51, to
\textit{The Vocal Enchantress} of 1783.
\normalsize

\pagebreak
%085
Many ballads were sung to it, and among them, \textit{King Olfrey and the old Abbot},
which is on the same story as \textit{King John and the Abbot of Canterbury}; and \textit{The
Song of the Caps}, in the Roxburgh Collection, which is also, in an altered form,
in \textit{Wit and Mirth, or Pills to Purge Melancholy}.

The following ballad is from a black-letter copy, in the Ashmolean Museum.


\musictitle{The Doleful Dance and Song of Death}
\musicsubtitle{intituled Dance After My Pipe.---to a pleasant new tune}

\musicinfo{Moderate time.}{}
\smallskip

\includemusic{chappellV1031.pdf}


\settowidth{\versewidth}{Merchants, have you made your mart in France,}
\indentpattern{0101000}
\begin{dcverse}\begin{patverse}
Bring away the beggar and the king,\\
And every man in his degree;\\
Bring away the old and youngest thing,\\
Come all to death, and follow me;\\
The courtier with his lofty looks,\\
The lawyer with his learned books,\\
The banker with his baiting hooks.
\end{patverse}

\begin{patverse}
Merchants, have you made your mart in France,\\
In Italy, and all about,\\
Know you not that you and I must dance,\\
Both our heels wrapt in a clout;\\
What mean you to make your houses gay,\\
And I must take the tenant away,\\
And dig for your sake the clods of clay?
\end{patverse}
\end{dcverse}

\pagebreak
%086

\begin{dcverse}
\begin{patverse}
Think you on the solemn ’sizes past,\\
How suddenly in Oxfordshire\\
I came, and made the judges all aghast,\\
And justices that did appear,\\
And took both Bell and Barham away,\footnotemark\\
And many a worthy man that day,\\
And all their bodies brought to clay.
\end{patverse}

\begin{patverse}
Think you that I dare not come to schools,\\
Where all the cunning clerks be most;\\
Take I not away both wise and fools,\\
And am I not in every coast?\\
Assure yourselves no creature can\\
Make Death afraid of any man,\\
Or know my coming where or whan.
\end{patverse}

\begin{patverse}
Where be they that make their leases strong,\\
And join about them land to land,\\
Do you make account to live so long,\\
To have the world come to your hand?\\
No, foolish nowle, for all thy pence,\\
Full soon thy soul must needs go hence;\\
Then who shall toyl for thy defence?
\end{patverse}

\begin{patverse}
And you that lean on your ladies’ laps,\\
And lay your heads upon their knee,\\
‘May think that you’ll escape, perhaps,\\
And need not come to dance with me.’\\
But no! fair lords and ladies all,\\
I will make you come when I do call,\\
And find you a pipe to dance withall.
\end{patverse}

\begin{patverse}
And you that are busy-headed fools,\\
To brabble for a pelting straw,\\
Know you not that I have ready tools\\
To cut you from your crafty law?\\
And you that falsely buy and sell,\\
And think you make your markets well,\\
Must dance with Death wheresoe’er you dwell.
\end{patverse}

\begin{patverse}
Pride must have a pretty sheet, I see,\\
For properly she loves to dance;\\
Come away my wanton wench to me,\\
As gallantly as your eye doth glance;\\
And all good fellows that flash and swash\\
In reds and yellows of revell dash,\\
I warrant you need not be so rash.
\end{patverse}

\begin{patverse}
For I can quickly cool you all,\\
How hot or stout soever you be,\\
Both high and low, both great and small,\\
I nought do fear your high degree;\\
The ladies fair, the beldames old,\\
The champion stout, the souldier bold,\\
Must all with me to earthly mould.
\end{patverse}

\begin{patverse}
Therefore take time while it is lent,\\
Prepare with me yourselves to dance;\\
Forget me not, your lives lament,\\
I come oft-times by sudden chance.\\
Be ready, therefore,—watch and pray,\\
That when my minstrel pipe doth play,\\
You may to heaven dance the way.
\end{patverse}
\end{dcverse}
\footnotetext{\scriptsize
Anthony à Wood observes: “This solemn Assize,
mentioned in the foregoing page, was kept in the Courthouse
in the Castle-yard at Oxon, 4 Jul., 1577. The Judges
who were infected and dyed with the dampe, were Sir
Rob. Bell, Baron of the Exchequer, and Sir Nich. Barham, 
Serjeant at Lawe.” See Hist, et Antiq. Univ. Oxon.
lib. i. sub an. 1577. This verse, therefore, cannot have
been in the ballad entered to Awdelay, in 1568-9.}

\musictitle{Wolsey’s Wild.}

This tune is called \textit{Wolsey's Wild} in Queen Elizabeth’s Virginal Book, but in
William Ballet’s Lute Book\dcfootnote{
This highly interesting manuscript, which is in the
library of Trinity College, Dublin (D. I. 21), contains a
large number of the popular tunes of the sixteenth century. 
“Fortune my foe,” “Peg a Ramsey,” “Bonny
sweet Robin,” “Calleno,” “Lightie love Ladies,” “Green
Sleeves,” “Weladay” (all mentioned by Shakspeare),
besides “The Witches Dawnce,” “Thehunt is up,” “The
Shaking of the Shetes,” “The Quadran Pavan,” “a Hornpipe,” 
“Robin Reddocke,” “Barrow Foster’s Dreame,”
“Dowland’s Lachrimæ,” “Lusty Gallant,” The Blacksmith,” 
“Rogero,” “Turkeyloney,” “Staynes Morris,”
“Sellenger’s Rownde,” “All flowers in brome,” “Baloo,”
“Wigmore’s Galliard,” “Robin Hood is to the greenwood
gone,” \&c., \&c.,are to be found in it. “Queen Mariees
Dump” (in whose reign it was probably commenced)
stands first in the book. The tunes are in lute tablature,
a style of notation now obsolete, in which the letters of
the alphabet up to K are used to designate the strings and
frets of the instrument.} it is called \textit{Wilson's Wile}, and in \textit{Musick's Delight
on the Cithren}, 1666, \textit{Wilson's Wild}. In the Bagford Collection of Ballads,
Brit. Mus., there is one called “A proper newe sonet, declaring the Lamentation
of Beccles, a town in Suffolk,” \&c., by T. D. (Thomas Deloney), to \textit{Wilson's Tune},
and dated 1586, but it does not appear, from the metre, to have been intended
for this air. Another “proper new ballad” \pagebreak to \textit{Wilson's New Tune} is in the
%087
Library of the Society of Antiquaries. It is on Ballard and Babington’s conspiracy,
and was written just after their execution, in 1586. \textit{Wilson’s Delight},
\textit{Arthur a Bradley}, and \textit{Mall Dixon’s Round}, are mentioned as popular tunes in
Braithwaite’s \textit{Strappado for the Devil},~1615.

The song, “Quoth John to Joan,” or “I cannot come every day to woo,” is
certainly as old as the time of Henry VIII., because the first verse is to be found
elaborately set to music in a manuscript of that date, formerly in the possession
of Stafford Smith (who'printed the song in Musica Antiqua, vol. i., p. 32), and now
in that of Dr. Rimbault. There are two copies of the words in vol. ii. of the
Roxburghe Collection of Ballads, and it is in all the editions of \textit{Wit and Mirth, or\textit{
Pills to purge Melancholy}}, from 1698 to 1719. In \textit{Wit’s Cabinet}, 1731, it is
called “The Clown’s Courtship, sung to the King at Windsor.”


\musicinfo{Moderate time.}{}
\bigskip

\includemusic{chappellV1032.pdf}



\settowidth{\versewidth}{In the nook of the chimney, instead of a bag.}
\begin{dcverse}I’ve corn and hay in the barn hard by,\\
And three fat hogs pent up in the sty;\\
I have a mare, and she is coal-black,\\
I ride on her tail to save her back.\\
\hspace{6em}Then say, my Joan, \&c.

I have a cheese upon the shelf,\\
And I cannot eat it all myself;\\
I’ve three good marks that lie in a rag,\\
In the nook of the chimney, instead of a bag.\\
\hspace{6em}Then say, my Joan, \&c.
\end{dcverse}

\begin{scverse}To marry I would have thy consent,\\
But, faith, I never could compliment;\\
I can say nought but “hoy, gee ho,”\\
Words that belong to the cart and the plough:\\
\hspace{6em}Then say, my Joan, say, my Joan, will that not do,\\
\hspace{6em}I cannot come every day to woo.
\end{scverse}

\pagebreak
%088

\musictitle{The Marriage of the Frog and the Mouse.}

In Wedderburn’s \textit{Complaint of Scotland}, 1549, one of the songs sung by the
shepherds is \textit{The frog cam to the myl dur} [mill-door]. In 1580, a ballad of
“A most strange wedding of the frog and the mouse” was licensed to Edward
White, at Stationers’ Hall: and in 1611, this song was printed with music, among
the “Country Pastimes,” in \textit{Melismata}. It is the progenitor of several others;
one beginning—
\settowidth{\versewidth}{“There was a frog lived in a well,}
\begin{scverse}
“There was a frog lived in a well,\\
And a farce mouse in a mill;”
\end{scverse}
another, “A frog he would a-wooing go;” a third in \textit{Pills to purge Melancholy}, \&c., \&c.


\musicinfo{Moderate time.}{}
\smallskip

\includemusic{chappellV1033.pdf}

\backskip{1}

\settowidth{\versewidth}{The frogge would a-wooing ride,}

\begin{dcverse}The frogge would a-wooing ride,\\
\hspace{4em}Humble-dum, humble-dum;\\
Sword and buckler by his side,\\
\hspace{4em}Tweedle, tweedle, twino.\\
When upon his high horse set,\\
\hspace{4em}Humble-dum, \&c.,\\
His boots they shone as black as jet,\\
\hspace{4em}Tweedle, \&c.

When he came to the merry mill pin,\\
Lady Mouse beene you within?\\
Then came out the dusty mouse:\\
I am lady of this house;\columnbreak

Hast thou any mind of me?\\
I have e’en great mind of thee.\\
Who shall this marriage make?\\
Our lord, which is the rat.

What shall we have to our supper?\\
Three beans in a pound of butter.\\
But, when supper they were at,\\
The frog, the mouse, and e’en the rat,

Then came in Gib, our cat,\\
And caught the mouse e’en by the back.\\
Then did they separate:\\
The frog leapt on the floor so flat;
\end{dcverse}


\begin{scverse}\smallskip
Then came in Dick, our drake,\\
And drew the frog e’en to the lake;\\
The rat he ran up the wall,\\
‘And so the company parted all.’
\end{scverse}

\musictitle{The Cramp.}

This is one of the three country dance tunes arranged to be sung together in
\textit{Pammelia}, and is frequently referred to as a ballad tune.

In the Ashmolean library, in the same manuscript volume with \textit{Chevy Chace}
(No. 48), is a ballad by Elderton, describing the articles sold in the market in
time of Lent. The observance of Lent was compulsory in those days, and it was
by no means palatable to all. In 1570, \pagebreak William Pickering had a license to print
%089
a ballad, entitled \textit{Lenton Stuff}, which was, in all probability,.the same. Elderton’s
ballad is called—
\begin{scverse}
“A new ballad, entitled \textit{Lenton Stuff},\\
For a little money ye may have enough;”\\
\end{scverse}

\begin{center} to the tune of \textit{The Cramp}.\end{center}

\begin{scverse}
“Lenton stuff is come to the town,\\
\vin The cleansing week comes quickly;\\
You know well enough you must kneel down,\\
\vin Come on, take ashes trickly;\\
That neither are good flesh nor fish,\\
But dip with Judas in the dish,\\
And keep a rout not worth a ryshe” [rush].\\
\vin\vin\vin\vin\vin\vin{[Heigh ho! the cramp-a.]}
\end{scverse}

It is not noticed by Ritson in his list of Elderton’s ballads, Bibl. Poet. p. 195-8;
but Mr. Halliwell has printed it in the volume containing \textit{The Marriage of Wit
and Wisdom}, for the Shakespeare Society. The following is from \textit{Pammelia}.


\musicinfo{Moderate time.}{}
\smallskip

\includemusic{chappellV1034.pdf}


\pagebreak
%090

\musictitle{I Have House and Land in Kent.}

This song, which is one of the “Country Pastimes,” in \textit{Melismata}, 1611, is on
the same subject as \textit{Quoth John to Joan}, page 87. The tune begins like \textit{The
Three Ravens}, but is in quicker time. In \textit{Melismata} it is called \textit{A Wooing Song
of a Yeoman of Kent's son}, and the words are given in the Kentish dialect.


\musicinfo{Moderate time.}{}
\smallskip

\includemusic{chappellV1035.pdf}


\indentpattern{121200}
\settowidth{\versewidth}{\textit{Chorus}.—For he can bravely clout his shoone,}
\begin{dcverse}\begin{patverse}
\vin Ich am my vather's eldest zonne,\\
My mother eke doth love me well;\\
For ich can bravely clout my shoone,\\
And ich full well can ring a bell.\footnote{
Bell-ringing was formerly a great amusement of the
English, and the allusions to it are of frequent occurrence.
Numerous payments to bell-ringers are generally to be
found in Churchwardens’ accounts of the 16th and 17th
centuries.}\\
\textit{Chorus}.—For he can bravely clout his shoone,\\
\vinphantom{\textit{Chorus}.—}And he full well can ring a bell.
\end{patverse}

\begin{patverse}
\vin My vather he gave me a bogge,\\
My mouther she gave me a zow;\\
I have a godvather dwells there by,\\
And he on me bestowed a plow.\\
\textit{Chorus}.—He has a godvather dwells there by,\\
\vinphantom{\textit{Chorus}.—}And he on him bestowed a plow.
\end{patverse}

\begin{patverse}
\vin One time I gave thee a paper of pins,\\
Anoder time a taudry lace;\\
And if thou wilt not grant me love,\\
In truth ich die bevore thy vace.\\
\textit{Chorus}.—And if thou wilt not grant his love,\\
\vinphantom{\textit{Chorus}.—}In truth he’ll die bevore thy face.
\end{patverse}

\begin{patverse}
\vin Ich have beene twise our Whitson lord,\\
Ich have had ladies many vare;\\
And eke thou hast my heart in hold,\\
And in my mind zeemes passing rare.\\
\textit{Chorus}.—And eke thou hast his heart in hold,\\
\vinphantom{\textit{Chorus}.—}And in his mind zeemes passing rare.
\end{patverse}

\begin{patverse}
\vin Ich will put on my best white slopp,\\
And ich will wear my jellow hose,\\
And on my head a good gray hat,\\
And in’t ich stick a lovely rose.\\
\textit{Chorus}.—And on his head a good gray hat,\\
\vinphantom{\textit{Chorus}.—}And i'nt he'll stick a lovely rose.
\end{patverse}

\begin{patverse}
\vin Wherefore cease off, make no delay,\\
And if you’ll love me, love me now;\\
Or else ich zeek zome oder where,\\
For I cannot come every day to woo.\\
\textit{Chorus}.—Or else he'll zeek zome oder where,\\
\vinphantom{\textit{Chorus}.—}For he cannot come every day to woo.
\end{patverse}
\end{dcverse}



\pagebreak
%091

\musictitle{Lusty Gallant.}

This tune, which was extremely popular in former times, is to be found in
William Ballet’s Lute Book. It resembles “Now foot it as I do, Tom, boy, Tom,”
which is one of three country dances, arranged to be sung together as a round, in
\textit{Pammelia}.

Nicholas Breton mentions \textit{Old Lusty Gallant} as a dance tune in his \textit{Works of
a Young Wit}, 1577:\qquad \qquad \qquad\dots “by chance,
\begin{scverse}
Our banquet done, we had our music by,\\
And then, you know, the youth must needs go dance,\\
First galliards—then larousse, and heidegy—\\
\textit{Old Lusty Gallant—All flowers of the broom};\\
And then a hall, for dancers must have room;”
\end{scverse}
and Elderton, wrote, “a proper new balad in praise of my Ladie Marques, whose
death is bewailed,” to the tune of \textit{New Lusty Gallant}. A copy of that ballad is
in the possession of Mr. George Daniel, of Canonbury; but I assume it to have
been intended for another air, because there are seven lines in each stanza. The
following is the first:—

\begin{scverse}
\vleftofline{“}Ladies, I thinke you marvell that\\
I writ no mery report to you:\\
And what is the cause I court it not\\
So merye as I was wont to dooe?\\
Alas! I let you understand\\
It is no newes for me to me to show\\
The fairest flower of my garland.”
\end{scverse}
If sung to this tune, the last line of each stanza would require repetition.

Nashe, in his \textit{Terrors of the Night}, 1594, says, “After all they danced \textit{Lusty
Gallant}, and a drunken Danish levalto or two.”

There is a song beginning, “Fain would I have a pretie thing to give unto my
ladie” (to the tune of \textit{Lusty Gallant}), in \textit{A Handefull of Pleasant Delites}, and
although that volume is not known to have been printed before 1584, it seems to
have been entered at Stationers’ Hall as early as 1565-6. \textit{Fain would I}, \&c.,
must have been written, and have attained popularity, either in or before the
year 1566, because, in 1566-7, a moralization, called \textit{Fain would I have a godly
thing to shew unto my lady}, was entered, and in MSS. Ashmole\dcfootnote{
Mr. W. H. Black, in his Catalogue of the Ashmolean
MSS., describes this volume as ‘‘written in the middle of
the sixteenth century”-—(it is the manuscript which contains
\textit{Chevy Chace}). Mr. Halliwell has printed the ballad
of \textit{Troilus and Creseida}, in the volume containing The
\textit{Marriage of Wit and Wisdom}, for the Shakespeare Society.} 48, fol. 120, is a
ballad of \textit{Troilus and Creseida}, beginning—
\begin{scverse}
\vleftofline{“}When Troilus dwelt in Troy town,\\
A man of noble fame-a”—
\end{scverse}
to the tune of \textit{Fain would I find some pretty thing}, \&c., so that, from the popularity
of the ballad, the tune had become known by its name also.

I have not found any song called \textit{Lusty Gallant}: perhaps it is referred to in
Massinger’s play, \textit{The Picture}, where Ferdinand says:

\pagebreak
%092
\begin{scverse}
\vin\vin\vin\vin\vin\vin ---“is your Theorbo\\
Turn’d to a distaff, Signior, and your voice,\\
With which you chanted \textit{Room for a lusty Gallant},\\
Tuned to the note of \textit{Lachrymæ}?”\scfootnote{
\textit{Lachrymæ}, a tune often referred to, composed by Dowland.}
\end{scverse}


The ballad of “A famous sea-fight between Captain Ward and the Rainbow”
(in the Roxburghe Collection) “to the tune of Captain Ward,” \&c., begins, “Strike
up, you lusty Gallants.”

In the \textit{Gorgeous Gallery of gallant Inventions}, 1578, there is a “proper dittie,”
to the tune of \textit{Lusty Gallant}; and Pepys mentions a song with the burden of
“St. George for England,” to the tune of \textit{List, lusty Gallants}.


\musicinfo{Moderate time.}{}
\smallskip

\includemusic{chappellV1036.pdf}


\settowidth{\versewidth}{Twenty journeys would I make,}
\begin{dcverse}
\begin{altverse}
Twenty journeys would I make,\\
And twenty days would hie me,\\
To make adventure for her sake,\\
To set some matter by me.
\end{altverse}

\begin{altverse}
Some do long for pretty knacks.\\
And some for strange devices;\\
God send me what my lady lacks,\\
I care not what the price is.
\end{altverse}
\end{dcverse}

There are eight more stanzas, which will be found in Evans’ \textit{Old Ballads}, vol. 1,
p. 123, edit. 1810, or in the reprint of \textit{A Handefull of Pleasant Delites}.

\musictitle{By a Bank as I Lay.}

In the Life of Sir Peter Carew, before quoted (page 52), “By the bank as
I lay” is mentioned as one of the \textit{Freemen’s Songs} which Sir Peter used to sing
with Henry VIII.; and this is one of the \textit{King Henry’s Mirth or Freemen’s Songs}
in \textit{Deuteromelia}. In Laneham’s letter from Kenilworth, 1565, “By a bank as
I lay” is included in the “bunch of ballads and songs, all ancient,” which were
then in the possession of Captain Cox, the Mason of Coventry. In Wager’s interlude, 
\textit{The longer thou livest the more fool thou art}, 1568, Moros sings the two
following lines:—
\begin{scverse}
“By a bank as I lay, I \textit{lay}.\\
\textit{Musing on things past}, heigh ho!”
\end{scverse}

In Royal MSS. Append. 58, there is another \pagebreak song, of which the first line is the
%093
same, but the second differs; and the music to it is not of the light and popular
class called \textit{Freemen’s Songs}, but a studied composition. The words of the latter
have been printed by Mr. Payne Collier, in his Extracts from the Registers of
the Stationers’ Company, vol. i., page 193. They are in the same metre, and
therefore might also be sung to this tune.

The last line of the song, as printed in \textit{Deuteromelia}, is “And save noble \textit{James}
our king,” because the book was printed in his reign.


\musicinfo{Moderate time.}{}
\medskip

\includemusic{chappellV1037.pdf}


\indentpattern{00110}
\settowidth{\versewidth}{To hear the bird how merrily she could sing,}
\begin{dcverse}\begin{patverse}
\vin O the gentle nightingale,\\
The lady and the mistress of all musick.\\
She sits down ever in the dale;\\
Singing with her notès smale [small],\\
And quavering them wonderfully thick.
\end{patverse}

\begin{patverse}
\vin Oh, for joy, my spirits were quick,\\
To hear the bird how merrily she could sing,\\
And I said, good Lord, defend\\
England, with thy most holy hand,\\
And save noble ‘Henry’ our king.
\end{patverse}
\end{dcverse}

\musictitle{Rogero.}

This tune is to be found among Dowland’s Manuscripts,\dcfootnote{
The references to these Manuscripts are, D. d. 2. 11.
--D. d. 3. 18.--D. d. 4. 23.—D. d. 9. 33.—D. d. 14. 24.,
\&c. Some appear to be in the handwriting of Dowland,
the celebrated lutenist of Elizabeth's reign. The tune of
\textit{Rogero} is in three or four of them.} in the public library,
Cambridge; in William Ballet’s Lute Book, and in Dallis’ Lute Book, both in
the library of Trinity College, Dublin.

The first entry in Mr. Payne Collier’s Extracts from the Registers of the
Stationers’ Company, is to William Pickering,  “Ballett called \textit{Arise and wake}”
(1557). In the Roxburghe Collection of Ballads, there is one commencing,
“Arise and \textit{a}wake,” entitled—

\begin{scverse}
“A godly and Christian A.B.C.,\\
Shewing the duty of every degree,”
\end{scverse}
to the tune of \textit{Rogero}. It may be the ballad referred to, although the copy in the
Roxburghe Collection was printed at a later date. In the same year, 1557, there
is an entry of “A Ballett of the A.B.C. of a Priest, called Hugh Stourmy,”
and another of “The aged man’s A.B.C.”
\pagebreak
%094

\textit{Rogero} is mentioned as a dance tune in Stephen Gosson’s \textit{School of Abuse},
1579; in Heywood’s \textit{A woman killed with kindness} (acted before 1604); and in
Nashe’s \textit{Have with you to Saffron-Walden}, 1596; also by Dekker, in \textit{The Shoemaker’s
Holiday}, \&c.

Many ballads were sung to the tune of \textit{Rogero}. In the first volume of the
Roxburghe Collection, for instance, there are at least four.\dcfootnote{
See folios 130, 258, 482, and 492.}  Others in the
Pepysian Collection; in \textit{The Grown Garland of Golden Roses}, 1612; in Deloney’s
\textit{Strange Histories},\dcfootnote{
\textit{The Crown Garland} and \textit{Strange Histories} have been
reprinted by the Percy Society.} 1607; in Percy’s \textit{Reliques of Ancient Poetry}; and in Evans’
\textit{Old Ballads}. \textit{Arise and awake} is also referred to as a ballad tune.

The following, which is entitled “The valiant courage and policy of the
Kentishmen with long tails, whereby they kept their ancient laws and customs,
which William the Conqueror sought to take from them \dcfootnote{
Evans, who prints this ballad from another copy (\textit{The
Garland of Delight}) extracts the following account of the
event which gave rise to it, from \textit{The Lives of the three
Norman Kings of England}, by Sir John Heyward, 4to, 1613,
p.~97: “Further, by the counsel of Stigand, Archbishop
of Canterbury, and of Eglesine, Abbot of St. Augustine’s
(who at that time were the chief governors of Kent), as the
King was riding towards Dover, at Swanscombe, two
miles from Gravesend, the Kentishmen came towards him
armed and bearing boughs in their hands, as if it had been
a moving wood; they enclosed him upon the sudden, and
with a firm countenance, but words well tempered with
modesty and respect, they demanded of him the use of
their ancient liberties and laws: that in other matters
they would yield obedience unto him: that without this
they desired not to live. The king was content to strike
sail to the storm, and to give them a vain satisfaction for
the present; knowing right well that the general customs
and laws of the residue of the realm would in short time
overflow these particular places. So pledges being given
on both sides, they conducted him to Rochester, and
yielded up the county of Kent, and the castle of Dover
into his power.”}—to the tune of \textit{Rogero}
is from \textit{Strange Histories}, \&c., 1607. It was written by Deloney, “the ballading
silk-weaver,” who died in or before 1600.


\musicinfo{Boldly and marked.}{}

\includemusic{chappellV1038.pdf}


\settowidth{\versewidth}{Which being done, he changed quite}
\begin{dcverse}\begin{altverse}
On Christmas-day in solemn sort\\
Then was he crowned here,\\
By Albert archbishop of York,\\
With many a noble peer.
\end{altverse}

\begin{altverse}
Which being done, he changed quite\\
The customs of this land,\\
And punisht such as daily sought\\
His statutes to withstand:
\end{altverse}

\begin{altverse}
And many cities he subdued,\\
Fair London with the rest;\\
But Kent did still withstand his force,\\
And did his laws detest.
\end{altverse}

\begin{altverse}
To Dover then he took his way,\\
The castle down to fling,\\
Which Arviragus builded there,\\
The noble British king.
\end{altverse}
\end{dcverse}

\pagebreak
%095
%\changefontsize{0.99\defaultfontsize}

\begin{dcverse}\footnotesizerr
\begin{altverse}
Which when the brave archbishop bold\\
Of Canterbury knew,\\
The abbot of Saint Augustines eke,\\
With all their gallant crew,
\end{altverse}

\begin{altverse}
They set themselves in armour bright,\\
These mischiefs to prevent,\\
With all the yeomen brave and bold\\
That were in fruitful Kent.
\end{altverse}

\begin{altverse}
At Canterbury did they meet\\
Upon a certain day,\\
With sword and spear, with bill and bow,\\
And stopt the conqueror’s way.
\end{altverse}

\begin{altverse}
Let us not live like bond-men poor\\
To Frenchmen in their pride,\\
But keep our ancient liberty,\\
What chance so e’er betide,
\end{altverse}

\begin{altverse}
And rather die in bloody field,\\
In manlike courage prest (ready),\\
Than to endure the servile yoke,\\
Which we so much detest.
\end{altverse}

\begin{altverse}
Thus did the Kentish commons cry\\
Unto their leaders still,\\
And so march’d forth in warlike sort,\\
And stand at Swanscomb hill:
\end{altverse}

\begin{altverse}
Where in the woods they hid themselves,\\
Under the shady green,\\
Thereby to get them vantage good,\\
Of all their foes unseen.
\end{altverse}

\begin{altverse}
And for the conqueror’s coming there,\\
They privily laid wait,\\
And thereby suddenly appal'd\\
His lofty high conceit;
\end{altverse}

\begin{altverse}
For when they spied his approach,\\
In place as they did stand,\\
Then marched they, to hem him in,\\
Each one a bough in hand,
\end{altverse}

\begin{altverse}
So that unto the conqueror’s sight,\\
Amazed as he stood,\\
They seem’d to be a walking grove,\\
Or else a moving wood.
\end{altverse}

\begin{altverse}
The shape of men he could not see,\\
The boughs did hide them so:\\
And now his heart for fear did quake,\\
To see a forest go;
\end{altverse}

\begin{altverse}
Before, behind, and on each side,\\
As he did cast his eye,\\
He spied those woods with sober pace\\
Approach to him full nigh:
\end{altverse}

\begin{altverse}
But when the Kentish-men had thus\\
Enclos’d the conqueror round,\\
Most suddenly they drew their swords,\\
And threw the boughs to ground;
\end{altverse}

\begin{altverse}
Their banners they display’d in sight,\\
Their trumpets sound a charge,\\
Their rattling drums strike up alarms,\\
Their troops stretch out at large.
\end{altverse}

\begin{altverse}
The conqueror, and all his train,\\
Were hereat sore aghast,\\
And most in peril, when they thought\\
All peril had been past.
\end{altverse}

\begin{altverse}
Unto the Kentish men he sent,\\
The cause to understand,\\
For what intent, and for what cause,\\
They took this war in hand;
\end{altverse}

\begin{altverse}
To whom they made this short reply,\\
For liberty we fight,\\
And to enjoy king Edward’s laws,\\
The which we hold our right.
\end{altverse}

\begin{altverse}
Then said the dreadful conqueror,\\
You shall have what you will,\\
Your ancient customs and your laws,\\
So that you will be still:
\end{altverse}

\begin{altverse}
And each thing else that you will crave\\
With reason, at my hand,\\
So you will but acknowledge me\\
Chief king of fair England.
\end{altverse}

\begin{altverse}
The Kentish men agreed thereon.\\
And laid their arms aside,\\
And by this means king Edward’s laws\\
In Kent do still abide;
\end{altverse}
\end{dcverse}

\settowidth{\versewidth}{And in no place in England else}
\begin{scverse}\smallskip
\begin{altverse}
And in no place in England else\\
These customs do remain,\\
Which they by manly policy\\
Did of duke William gain.
\end{altverse}
\end{scverse}

\musictitle{Turkeyloney.}

The figure of the dance called \textit{Turkeyloney} is described with others in a manu\-script
in the Bodleian Library (MS. Rawl. Poet. 108), which was written about
1570. Stephen Gosson, in his \textit{Schoole of Abuse, containing a pleasant Invective
against Poets, Pipers, Players, Jesters}, \&c., \pagebreak 1579, alludes to the tune as one of
%096
the most popular in his day. He says, “Homer, with his music, cured the sick
soldiers in the Grecians’ camp, and purged every man’s tent of the plague.
Think you that those miracles could be wrought with playing dances, dumps,
pavans, galliards, fancies, or new strains? They never came where this grew,
nor knew what it meant\dots Terpander neither piped \textit{Rogero}, nor \textit{Turkeloney},
when he ended the brabbles at Lacedemon, but, putting them in mind of Lycurgus’
laws, taught them to tread a better measure:” but, “if you enquire how many
such poets and pipers we have in our age, I am persuaded that every one of them
may creep through a ring, or dance the wild morris in a needle’s eye. We have
infinite poets and pipers, and such peevish cattle among us in England, that live
by merry begging, maintained by alms, and privily encroach upon every man’s
purse, but if they in authority should call an account to see how many Chirons,
Terpandri, and Homers are here, they might cast the sum without pen or
counters, and sit down with Rachel to weep for her children, because they are not.”

Turkeylony is also mentioned, as a dance tune, in Nashe’s \textit{Have with you to
Saffron-Walden}, 1596; and the music will be found in William Ballet’s Lute
Book, described in a note at page 86.

The words here coupled with the tune are taken from a manuscript in the
possession of Mr. Payne Collier. Although the manuscript is of the reign of
James I., the “ballett” \textit{Yf ever I marry, I will marry a mayde}, was entered
at Stationers’ Hall as early as 1557-8. The name of the air to which it should
be sung is neither given in the MS., nor in the entry at Stationers’ Hall; but the
words and music agree so well together, that it is very probable the ballet was
written to this tune.


\musicinfo{In moderate time, and smoothly.}{}
\smallskip

\includemusic{chappellV1039.pdf}

\pagebreak
%097
\settowidth{\versewidth}{And cost ne’er so much, she will ever go brave, [gaily dress’d]}
\begin{scverse}
A maid is so sweet, and so gentle of kind,\\
That a maid is the wife I will choose to my mind;\\
A widow is froward, and never will yield;\\
Or if such there be, you will meet them but seeld. [seldom]

A maid ne’er complaineth, do what so you will;\\
But what you mean well, a widow takes ill:\\
A widow will make yon a drudge and a slave,\\
And cost ne’er so much, she will ever go brave, [gaily dress’d]

A maid is so modest, she seemeth a rose,\\
When first it beginneth the bud to unclose;\\
But a widow full blowen, full often deceives,\\
And the next wind that bloweth shakes down all her leaves.

That widows be lovely I never gainsay,\\
But too well all their beauty they know to display;\\
But a maid hath so great hidden beauty in store,\\
She can spare to a widow, yet never be poor.

Then, if ever I marry, give me a fresh maid,\\
If to marry with any I be not afraid;\\
But to marry with any it asketh much care,\\
And some bachelors hold they are best as they are.
\end{scverse}
\vfill
\centerrule
\vfill

\pagebreak
