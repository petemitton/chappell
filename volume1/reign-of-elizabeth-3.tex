
%144
\changefontsize{0.98\defaultfontsize}

\musictitle{Who List To Lead A Soldier’s Life.}

This tune is in \textit{The Dancing Master}, from 1650 to 1725, called “A soldier’s
life, or “Who list to lead a soldier’s life.” There were, evidently, two tunes
under the same name (one of which I have not discovered), because some of the
ballads could not be sung to this air. In Peele’s \textit{Edward I}., 1593, we find,
“Enter a harper and sing, to the tune of \textit{Who list to lead a soldier’s life}, the
following:—
\settowidth{\versewidth}{And play the men both great and small,” \&c.}
\begin{scverse}\vleftofline{“}Go to, go to, you Britons all,\\
And play the men both great and small,” \&c.;
\end{scverse}
and in Deloney’s \textit{Strange Histories}, 1607—
\settowidth{\versewidth}{In woeful wars had victorious been,” \&c;}
\begin{scverse}\vleftofline{“}When Isabell, fair England’s queen,\\
In woeful wars had victorious been,” \&c;
\end{scverse}
neither of which could be sung to \textit{this} air, but “A Song of an English Knight,
that married the Royal Princess, Lady Mary, sister to Henry VHL, which Knight
was afterwards made Duke of Suffolk;” beginning—
\settowidth{\versewidth}{“Eighth Henry ruling in this land,}
\begin{scverse}\begin{altverse}
\vleftofline{“}Eighth Henry ruling in this land,\\
He had a sister fair;”
\end{altverse}
\end{scverse}
and “A Song of the Life and Death of King Richard III., who, after many
murders by him committed, \&c., was slain at the battle of Bosworth, by
Henry VII., King of England;” beginning—
\settowidth{\versewidth}{“In England once there reigned a king,}
\begin{scverse}\begin{altverse}
\vleftofline{“}In England once there reigned a king,\\
A tyrant fierce and fell,”\dcfootnote{\textit{}
These two ballads have been reprinted by Evans in
\textit{Old Ballads} vol. iii., 30 and 84 (1810); but he has omitted
the names of the tunes to which they were to be sung, not
only in these, but in numberless other instances.}
\end{altverse}
\end{scverse}
as well as several others, are exactly fitted to the tune.

Ophelia’s Song, “Good morrow, ’tis St. Valentine’s day,” and the traditional
air to “Lord Thomas and Fair Ellinor,” are only different versions of this.

In the Pepys Collection, vol i., is a black-letter ballad of “The joyful peace
concluded between the King of Denmark and the King of Sweden, by the means of
our most worthy sovereign James,” \&c., to the tune of “Who list to lead a
soldier’s life;” dated 1613.

In \textit{The Miseries of inforced Marriage} (Dodsley’s Old Plays, vol. v.), the song,
“Who list to have a lubberly load,” was, perhaps, a parody on “Who list to lead
a soldier’s life,” the words of which I have not been successful in finding.

\musicinfo{Gracefully.}{}

\includemusic{chappellV1061.pdf}

\pagebreak
\changefontsize{\defaultfontsize}
%145

\musictitle{Lord Thomas and Fair Ellinor.}

This traditional version of the tune of \textit{Lord Thomas and Fair Ellinor} is taken
from Sandys’ Collection of Christmas Carols. It is, evidently, the air of \textit{Who
list to lead a soldier's life}? adapted for words of a somewhat different measure.
(See the opposite page.)

At p. 17 of Ritson’s \textit{Observations on the Minstrels}, in enumerating the probable
“causes of the rapid decline of the minstrel profession, since the time of Elizabeth,”
he says, “It is conceived that a few individuals, resembling the character,
might have been lately, and may possibly be still found, in some of the least
polished or less frequented parts of the kingdom. It is not long since the public
papers announced the death of a person of this description, somewhere in Derbyshire;
and another was within these two years to be seen in the streets of London;
he played on an instrument of the rudest construction, which he, properly enough,
called a \textit{hum-strum}, and chanted (amongst others) the old ballad of \textit{Lord Thomas
and Fair Ellinor}, which, by the way, has every appearance of being originally a
minstrel song.”

The ballad will be found in book i., series 3, of Percy’s \textit{Reliques of Ancient
Poetry}, and it is one of those still kept in print in Seven Dials. The black-letter
copies direct it to be sung “to a pleasant new tune.” See Douce Collection, i. 121.

\musicinfo{Gracefully.}{}

\includemusic{chappellV1062.pdf}

\musictitle{The Friar and The Nun.}

In Henry Chettle’s \textit{Kind-hart's Dreame}, 1592, two lines are quoted from the
ballad of “The Friar and the Nun.” The tune is in \textit{The Dancing Master,} from
1650 to 1725; in \textit{Musick's Delight on the Cithren}, 1666; in \textit{Pills to purge
Melancholy}; and in many of the ballad-operas, such as \textit{The Beggars' Opera}, \textit{The
Devil to pay}, \textit{The Jovial Crew}, \&c. Henry Carey wrote a song to the tune in his
\textit{Honest Yorkshireman}, 1735, and there are three, or more, in \textit{Pills to purge Melancholy}.
 In vol. ii. of some editions, and vol. iv. of others, the title and tune of
“The Friar and the Nun” are printed by mistake with the song of “Fly, merry 
news,” which has no reference to them. \pagebreak The ballad of \textit{The London Prentice} was
%146
occasionally sung to it, and in some of the ballad-operas the tune bears that name.
In \textit{The Plot}, 1735, it is called “The merry songster.” The composer of the
modern song, “Jump, Jim Crow,” is under some obligations to this air.

Henry Carey’s song is called “The old one outwitted,” and begins—
\begin{scverse}
\begin{altverse}
\vleftofline{“}There was a certain usurer,\\
He had a pretty niece,” \&c.
\end{altverse}
\end{scverse}

In \textit{The Beggars’ Opera}, the name of “All in a misty morning” is given to
the tune, from the first line of a song called \textit{The Wiltshire Wedding}, which will
be found in \textit{Pills to purge Melancholy}, iv. 148, or ii. 148. There are fifteen
verses, of which the following nine suffice to tell the story.

\musicinfo{Quick.}{}

\includemusic{chappellV1063.pdf}
\vspace{-2\baselineskip}

\settowidth{\versewidth}{With how d’ye do? and how d’ye do?}
\begin{dcverse}\begin{altverse}
The rustic was a thresher,\\
And on his way he hied,\\
And with a leather bottle\\
Fast buckled by his side;\\
And with a cap of woollen,\\
Which covered cheek and chin;\\
With how d’ye do? and how d’ye do?\\
And how d’ye do? again.
\end{altverse}

\begin{altverse}
I went a little further,\\
And there a met a maid\\
Was going then a milking,\\
A milking, sir, she said;\\
Then I began to compliment,\\
And she began to sing:\\
With how d’ye do? \&c.
\end{altverse}

\begin{altverse}
This maid, her name was Dolly,\\
Cloth’d in a gown of gray,\\
I, being somewhat jolly,\\
Persuaded her to stay:\\
Then straight I fell to courting her,\\
In hopes her love to win,\\
With how d’ye do? \&c.
\end{altverse}

\begin{altverse}
I told her I would married be,\\
And she should be my bride,\\
And long we should not tarry,\\
With twenty things beside:\\
“I’ll plough and sow, and reap and mow,\\
Whilst thou shalt sit and spin,”\\
With how d’ye do? \&c.
\end{altverse}
\end{dcverse}
\pagebreak
%147


\settowidth{\versewidth}{With how d’ye do? and how d’ye}
\begin{dcverse}\begin{altverse}
\vleftofline{“}Kind sir, I have a mother,\\
Besides, a father, still,\\
And so, before all other.\\
You must ask their good will;\\
For if I be undutiful\\
To them, it is a sin;”\\
With how d’ye do? \&c.
\end{altverse}

\begin{altverse}
Now, there we left the milking-pail,\\
And to her mother went,\\
And when we were come thither,\\
I asked her consent;\\
I doff’d my hat, and made a leg,\\
When I found her within;\\
With how d’ye do? \&c.
\end{altverse}

\begin{altverse}
Her dad came home full weary,\\
(Alas! he could not choose;)\\
Her mother being merry,\\
She told him all the news.\\
Then he was mighty jovial too,\\
His son did soon begin\\
With how d’ye do? \&c.
\end{altverse}

\begin{altverse}
The parents being willing,\\
All parties were agreed,\\
Her portion, thirty shilling;\\
We married were with speed.\\
Then Will, the piper, he did play,\\
Whilst others dance and sing;\\
With how d’ye do? and how d’ye do?\\
And how d’ye do? again.
\end{altverse}
\end{dcverse}

\musictitle{John, Come Kiss Me Now.}

This favorite old tune will be found in Queen Elizabeth’s Virginal Book; in
Playford’s \textit{Introduction}; in \textit{Apollo’s Banquet for the Treble Violin}; and in the
\textit{First part of the Division Violin, containing a collection of Divisions upon several
excellent grounds}, printed by Walsh; as well as Playford’s \textit{Division Violin} (1685.)
In \textit{Pills to purge Melancholy}, vol. iii., 1707., and vol. v., 1719, it is adapted to a
song called \textit{Stow, the Friar}. It is mentioned in Heywood’s \textit{A Woman kill’d with
Kindness}, 1600:
\settowidth{\versewidth}{“I come to dance, not to quarrel: come, what shall it be? \textit{Rogero}?}
\begin{scverse}
\textit{Jack Slime}.—“I come to dance, not to quarrel: come, what shall it be? \textit{Rogero}?\\
\textit{Jenkin}.—“\textit{Rogero}, no; we will dance \textit{The Beginning of the World}.\\
\textit{Sisly}.—“I love no dance so well as \textit{John, come kiss me now}.”
\end{scverse}
In ’\textit{Tis merry when Gossips meet}, 1609:
\settowidth{\versewidth}{Such dauncing, coussen, you would hardly thinke it;}
\begin{scverse}
\vleftofline{\textit{Widow}.—“}No musique in the evening did we lacke;\\
Such dauncing, coussen, you would hardly thinke it;\\
Whole pottles of the daintiest burned sack,\\
’Twould do a wench good at the heart to drinke it.\\
Such store of tickling galliards, I do vow;\\
Not an old dance, but \textit{John, come kisse me now}.
\end{scverse}
In a song in \textit{Westminster Drollery}, 1671 and 1674, beginning, “My name is
honest Harry:”
\settowidth{\versewidth}{And when that we have danc’d a round,}
\begin{scverse}
\vleftofline{“}The fidlers shall attend us,\\
And first play, \textit{John, come kisse me};\\
And when that we have danc’d a round,\\
They shall play, \textit{Hit or misse me}.”\dcfootnote{\textit{}
\textit{Hit or miss} is a tune in \textit{The Dancing Master} of 1650,
and later editions. It is referred to by Whitlock, in his
\textit{Zootamia, or present Manners of the English}, 12mo., 1654,
where he speaks of one whose practice in physic is
“nothing more than the country dance called \textit{Hit or
misse}.”
}
\end{scverse}

In Burton’s \textit{Anatomy of Melancholy}, 1621: “Yea, many times this \textit{love} will
make old men and women, that have more toes than teeth, dance \textit{John, come kiss
me now}.” It is also mentioned in \textit{The Scourge of Folly}, 8vo. (n.d.); in Brathwayte’s
\textit{Shepherd’s Tales}, 1623; in \textit{Tom Tiler and his Wife}, 1661; and in Henry
Bold’s \textit{Songs and Poems}, 1685.
\pagebreak
%148

\begin{scverse}\vleftofline{“}In \textit{former} times’t hath been upbraided thus,\\
That barber’s music was most barbarous;\\
For that the cittern was confin’d unto\\
\vleftofline{‘}The Ladies’ Fall,’ or ‘\textit{John, come kiss me now},’\\
\vleftofline{‘}Green Sleeves and Pudding Pyes,’ ‘The P-------’s Delight,’\\
\vleftofline{‘}Winning of Bulloigne,’ ‘Essex’s last Good-night,’ \&c.”
\end{scverse}

From lines “On a Barber who became a great Master of Musick.” The ground of
\textit{John, come kiss me now}, was a popular theme for fancies and divisions (now
called fantasias' and variations) for the virginals, lute, and viols. In the
Virginal Book, only the first part of the tune is taken, and it is doubtful if it
then had any second part; the copy we have given is from Playford’s and Walsh’s
\textit{Division Violin}. It is one of the songs parodied in Andro Hart’s \textit{Compendium
of Godly Songs}, before mentioned, on the strength of which the tune has been
claimed as Scotch, although it has no Scotch character, nor has hitherto been
found in any old Scotch copy. Not only are all the other tunes to the songs in the
\textit{Compendium}, of which any traces are left, English, but what little secular music
was printed in Scotland until the eighteenth century, was entirely English or
foreign. The following are the first, second, and twenty-first stanzas of the
“Godly Song”:—
\settowidth{\versewidth}{My prophites call, my preacher's cry,}
\begin{dcverse}\begin{altverse}
John, come kisse me now;\\
John, come kisse me now,\\
Johne, come kisse me by and by,\\
And make no more adow.
\end{altverse}

\begin{altverse}
The Lord thy God I am,\\
That John dois thee call;\\
\end{altverse}
\begin{altverse}
John represents man,\\
By grace celestiall.
\end{altverse}

\begin{altverse}
My prophites call, my preacher's cry,\\
John, come kisse me now;\\
John, come kisse me by and by,\\
And make no more adow.
\end{altverse}
\end{dcverse}

\musicinfo{Rather slow and stately.}{}

\smallskip
\includemusic{chappellV1064.pdf}

\pagebreak
%149

\musictitle{All You That Love Good Fellows, or The London Prentice.}

The tunes called \textit{Nancie} in Queen Elizabeth’s Virginal Book; \textit{Eduward
Nouwels}, in Bellerophon (Amsterdam, 1622, p. 115); \textit{Sir Eduward Nouwel’s
Delight}, in \textit{Friesche Lust-hof}, 1634; and \textit{The London Prentice}, in \textit{Pills to purge
Melancholy} (vi., 342), and in \textit{The Devil to pay}, 1731, are the same: but the two
last contain only sixteen bars, while all the former consist of twenty-four.

The following is the version called \textit{Sir Edward Noels Delight}.

\medskip
\musicinfo{In marching time.}{}

\includemusic{chappellV1065.pdf}

The ballad of “The honour of a London Prentice: being an account of his
matchless manhood, and brave adventures done in Turkey, and by what means he
married the king’s daughter,” is evidently a production of the reign of Elizabeth.
The apprentice maintains her to be “the phœnix of the world,” “the pearl of
princely majesty,” \&c., against “a score of Turkish Knights,” whom he overthrows
at tilt.

The ballad is printed in Ritson’s \textit{English Songs} (among the Ancient Ballads),
and in Evans’ \textit{Old Ballads}, vol. iii., 178. Copies will also be found in the Bagford,
Roxburghe (iii. 747), and other Collections. It was “to be sung to the tune
of \textit{All you that love good fellows} under which name the air is most frequently
mentioned.
\pagebreak
%150

\changefontsize{1.03\defaultfontsize}

Bishop Earle, in his \textit{Micosmography}, 1628, in giving the character of a Potpoet,
says, “He is a man now much employed in commendations of our navy, and
a bitter inveigher against the Spaniard. His frequentest works go out in single
sheets, and are chanted from market to market to a vile tune, and a worse throat;
whilst the poor country wench melts, like her butter, to hear them. And these
are the stories of some men of Tyburn, or A strange monster out of Germany.”
One of these ballads of “strange monsters out of Germany” will be found in the
Bagford and in the Pepys Collection (ii. 66), “to the tune of \textit{All you that love
good fellows}.” It is entitled “Pride’s fall: or a warning for all English women
by the example of a strange monster born late in Germany, by a merchant’s
proud wife of Geneva.” The ballad, evidently a production of the reign of
James I.,\dcfootnote{\textit{}
See Fairholt's \textit{Satirical Song}s and \textit{Poems on Costume}, p. 107; printed for the Percy Society.
} is perhaps the one alluded to by Bishop Earle.

There are other ballads about London apprentices; one of “The honors achiev\-ed
in Fraunce and Spayne by four prentises of London,” was entered to John
Danter, in 1592. “Well, my dear countrymen, \textit{What-d’ye-lacks}” (as apprentices
were frequently called, from their usual mode of inviting custom), “I’ll have you
chronicled, and all to be praised, and sung in sonnets, and bawled in new brave
ballads, that all tongues shall troul you \textit{in sœecula seculorum}.”—\textit{Beaumont and
Fletcher's Philaster}.

One of the ballads to the tune of “the worthy London prentice” relates
to a very old superstition, and will recall to us the “Out, damned spot!” in
\textit{Macbeth}. It is entitled the “True relation of Susan Higges, dwelling in Risborow,
a towne in Buckinghamshire, and how she lived twenty years by robbing
on the high wayes, yet unsuspected of all that knew her; till at last coming to
Messeldon, and there robbing and murdering a woman, which woman knew her,
and standing by her while she gave three groanes, \textit{she spat three drops of blood in
her face, which never could be washt out}, by which she was knowne, and executed
for the aforesaid murder, at the assises in Lent at Brickhill.” A copy is in the
Roxburghe Collection, i. 424; also in Evans’ \textit{Old Ballads}, i. 203 (1810).

I have not found any song or ballad commencing “All you that \textit{love} good
fellows,” although so frequently quoted as a tune; but there are several “All you
that \textit{are},” and “All you that \textit{be} good fellows,” which, from similarity of metre,
I assume to be intended for the \textit{same air}.

In a chap-book called “The arraigning and indicting of Sir John Barleycorn,
knight; newly composed by a well-wisher to Sir John, and all that love him,” are
two songs, “All you that are good fellows,” and “All you that \textit{be} good fellows,”
“to the tune of \textit{Sir John Barleycorn}, or \textit{Jack of all trades}.” Lowndes speaks of
this tract as printed for T. Passenger in 1675, and of the author as Thomas
Robins; but there are Aldermary and Bow Church-yard editions of later date.

Another “All you that \textit{are} good fellows” is here printed to the shorter copy of
the tune. It is from a little black-letter volume (in Wood’s library, Ashmolean
Museum) entitled “Good and true, fresh and new Christmas Carols,” \&c.,
printed by E. P. for Francis Coles, dwelling in the Old Bailey, 1642. It is one
%151
of the merry Christmas carols, and to \pagebreak be sung to the tune of “All you that
are good fellows.”

\musicinfo{In marching time.}{}

\includemusic{chappellV1066.pdf}

\settowidth{\versewidth}{Plum-porridge, roast beef, and minc’d pies,}
\begin{dcverse}\begin{altverse}
This is a time of joyfulness,\\
And merry time of year,\\
When as the rich with plenty stor’d\\
Do make the poor good cheer.\\
Plum-porridge, roast beef, and minc’d pies,\\
Stand smoking on the board;\\
With other brave varieties,\\
Our master doth afford.
\end{altverse}

\begin{altverse}
Our mistress and her cleanly maids\\
Have neatly play’d the cooks;\\
Methinks these dishes eagerly\\
At my sharp stomach looks,\\
As though they were afraid\\
To see me draw my blade;\\
But I revenged on them will be,\\
Until my stomach’s stay’d.
\end{altverse}

\begin{altverse}
Come fill us of the strongest,\\
Small drink is out of date;\\
Methinks I shall fare like a prince,\\
And sit in gallant state:\\
This is no miser’s feast,\\
Although that things be dear;\\
God grant the founder of this feast\\
Each Christmas keep good cheer.
\end{altverse}

\begin{altverse}
This day for Christ we celebrate,\\
Who was born at this time;\\
For which all Christians should rejoice,\\
And I do sing in rhyme.\\
When you have given thanks,\\
Unto your dainties fall,\\
Heav’n bless my master and my dame,\\
Lord bless me, and you all.
\end{altverse}
\end{dcverse}
\pagebreak
%152

\musictitle{The British Grenadiers.}

The correct date of this fine old melody appears altogether uncertain, as it
is to be found in different forms at different periods; but it is here placed in juxtaposition
to \textit{Sir Edward Noel’s Delight}, and \textit{All you that love good fellows}, or
\textit{The London Prentice}, because evidently derived from the same source. The
commencement of the air is also rather like \textit{Prince Rupert’s March}, and the end
resembles \textit{Old King Cole}, with the difference of being major instead of minor.
Next to the National Anthems, there is not any tune of a more spirit-stirring
character, nor is any one more truly characteristic of English national music.
This version of the tune is as played by the band of the Grenadier Guards. The
words are from a copy about a hundred years old, with the music.

\musicinfo{March.}{}

\includemusic{chappellV1067.pdf}

\settowidth{\versewidth}{Sing tow, row, row, row, row, row, for the British Grenadiers.}
\begin{scverse}
Those heroes of antiquity ne’er saw a cannon ball,\\
Or knew the force of powder to slay their foes withal;\\
But our brave boys do know it, and banish all their fears,\\
Sing tow, row, row, row, row, row, for the British Grenadiers.\\
\vin\vin\vin\vin\vin\vin\textit{Chorus}.—But our brave boys, \&c.
\end{scverse}
\pagebreak
%153


\begin{scverse}
Whene’er we are commanded to storm the palisades,\\
Our leaders march with fusees, and we with hand grenades,\\
We throw them from the glacis, about the enemies’ ears,\\
Sing tow, row, row, row, row, row, the British Grenadiers.\\
\vin\vin\vin\vin\vin\vin\textit{Chorus}.—We throw them, \&c.

And when the siege is over, we to the town repair,\\
The townsmen cry Hurra, boys, here comes a Grenadier,\\
Here come the Grenadiers, my boys, who know no doubts or fears.\\
Then sing tow, row, row, row, row, row, the British Grenadiers.\\
\vin\vin\vin\vin\vin\vin\textit{Chorus}.—Here come the, \&c.

Then let us fill a bumper, and drink a health to those\\
Who carry caps and pouches, and wear the louped clothes,\\
May they and their commanders live happy all their years,\\
With a tow, row, row, row, row, row, for the British Grenadiers.\\
\vin\vin\vin\vin\vin\vin\textit{Chorus}.—May they, \&c.
\end{scverse}

\musictitle{The Cushion Dance.}

The Cushion Dance was in favour both in court and country in the reign of
Elizabeth, and is occasionally danced even at the present day. In Lilly’s \textit{Euphues},
1580, Lucilla, says, “Trulie, Euphues, you have \textit{mist the cushion}, for I was neither
angrie with your long absence, neither am I well pleased at your presence.” This
is, perhaps, in allusion to the dance, in which each woman selected her partner by
placing the cushion before him. Taylor, the water-poet, calls it “a pretty little
provocatory dance,” for he before whom the cushion was placed, was to kneel and
salute the lady. In Heywood’s \textit{A Woman kill’d with Kindness}, (which Henslow
mentions in his diary, in 1602), the dances which the country people call for are,
\textit{Rogero}; \textit{The Beginning of the World}, or \textit{Sellenger’s Round}; \textit{John, come kiss me
now}; \textit{Tom Tyler}; \textit{The hunting of the Fox}; \textit{The Hay}; \textit{Put on your smock a
Monday}; and \textit{The Cushion Dance}; and Sir Francis thus describes their style of
dancing:—
\settowidth{\versewidth}{Made with their high shoes: though their skill be small,}
\begin{scverse}
\vleftofline{“}Now, gallants, while the town-musicians\\
Finger their frets within; and the mad lads\\
And country lasses, every mother’s child.\\
With nosegays and bride-laces in their hats,\\
Dance all their country measures, rounds, and jigs,\\
What shall we do? Hark! they’re all on the hoigh;\\
They toil like mill-horses, and turn as round;\\
Marry, not on the toe: aye, and they caper,\\
But not without cutting; you shall see, to-morrow,\\
The hall floor peck’d and dinted like a mill-stone,\\
Made with their high shoes: though their skill be small,\\
Yet they tread heavy where their hob-nails fall.”
\end{scverse}

When a partner was selected in the dance, he, or she, sang “Prinkum-prankum
is a fine dance,” \&c.; which line is quoted by Burton, in his \textit{Anatomy
of Melancholy}; and, “No dance is lawful but Prinkum-prankum,” in \textit{The Muses’
Looking-glass}, 1638.

In the \textit{Apothegms of King James, \pagebreak the Earl of Worcester}, \&c., 1658, a wedding
%154
entertainment is spoken of: and, “when the masque was ended, and time had
brought in the supper, \textit{the cushion led the dance} out of the parlour into the hall.”
Selden, speaking of \textit{Trenchmore} and \textit{The Cushion Dance} in Queen Elizabeth’s
time, says, “Then all the company dances, lord and groom, lady and kitchen-maid, 
no distinction.”—(See ante p. 82.) In \textit{The Dancing Master} of 1686, and
later editions, the figure is thus described:—

“This dance is begun by a single person (either man or woman), who, taking a
cushion in hand, dances about the room, and at the end of the tune, stops and sings,
‘This dance it will no further go.” The musician answers, ‘I pray you, good Sir,
why say you so?’—\textit{Man}. ‘Because Joan Sanderson will not come too.’—\textit{Musician}.
‘She must come too, and she shall come too, and she must come whether she will
or no.’ Then he lays down the cushion before the woman, on which she kneels, and
he kisses her, singing ‘Welcome, Joan Sanderson, welcome, welcome.’ Then she
rises, takes up the cushion, and both dance, singing, ‘Prinkum-prankum is a fine dance,
and shall we go dance it once again, once again, and once again, and shall we go dance
it once again.’ Then making a stop, the woman sings as before, ‘This dance it will
no further go.’—\textit{Musician}. ‘I pray you, madam, why say you so?'— \textit{Woman}. ‘Because
John Sanderson will not come too.’—\textit{Musician}. ‘He must come too, and he shall
come too, and he must come whether he will or no.’ And so she lays down the
cushion before a man, who kneeling upon it, salutes her; she singing, ‘Welcome,
John Sanderson, welcome, welcome.’ Then he taking up the cushion, they take
hands, and dance round, singing as before. And thus they do, till the whole company
are taken into the ring; and if there is company enough, make a little ring in its
middle, and within that ring, set a chair, and lay the cushion in it, and the first man
set in it. Then the cushion is laid before the first man, the woman singing, ‘This
dance it will no further go;’ and as before, only instead of ‘Come too,’ they sing, ‘Go
fro;’ and instead of ‘Welcome, John Sanderson,’ they sing, ‘Farewell, John Sanderson,
farewell, farewell;’ and so they go out one by one as they came in. \textsc{Note}.—\textit{The
women are kissed by all the men in the ring at their coming and going out, and likewise
the men by all the women}.”

This agreeable pastime tended, without doubt, to popularize the dance.

One of the engravings in Johannis de Brunes \textit{Emblemata} (4to., Amsterdam,
1624, and 1661) seems to represent the Cushion Dance. The company being
seated round the room, one of the gentlemen, hat in hand, and with a cushion held
over the left shoulder, bows to a lady, and seems about to lay the cushion at her
feet.

In 1737, the Rev. Mr. Henley, or “Orator Henley,” as he called himself,
advertised in the \textit{London Daily Post} that he would deliver an oration on the
subject of the Cushion Dance.

A political parody is to be found in \textit{Poems on Affairs of State, from 1640 to
1704}, called, “The Cushion Dance at Whitehall, by way of Masquerade. To the
tune of Joan Sanderson.”

\settowidth{\versewidth}{The trick of trimming is a fine trick,}
\begin{scverse}
\textit{\vleftofline{Enter God}frey Aldworth, followed by the King and Duke.}\\
\vleftofline{\textit{King}. “}The trick of trimming is a fine trick,\\
And shall we go try it once again?\\
\vleftofline{\textit{Duke}. “}The plot it will no further go.\\
\vleftofline{\textit{King}. “}I pray thee, wise brother, why say you so,” \&c.
\end{scverse}
\pagebreak
%155

The tunes of Cushion-Dances (like Barley-Breaks) have the first part in
\timesig{3}{4}, and the last in \timesig{6}{8} time. The earliest printed copy I have found is in \textit{Tablature
de Luth, intitulé Le Secret des Muses}, 4to., Amsterdam, 1615, where it is
called \textit{Gaillarde Anglaise}. In \textit{Nederlandtsche Gedenck-Clanck}, Haerlem, 1626,
the same air is entitled \textit{Gallarde Suit Margrie}t, which being intended as English,
may he guessed as “Galliard, Sweet Margaret.” It is the following:—

\musicinfo{Slow.}{}

\includemusic{chappellV1068.pdf}

The Galliard (a word meaning brisk, gay; and used in that sense by Chaucer)
is described by Sir John Davis as a swift and wandering dance, with lofty turns
and capriols in the air. Thoinot Arbeau, in his \textit{Orchesographie}, 1589, says that,
\textit{formerly}, when the dancer had taken his partner for the galliard, they first placed 
themselves at the end of the room, and, \pagebreak after a bow and curtsey, they walked once
%156
or twice round it. Then the lady danced to the other end, and remained there
dancing, while the gentleman followed; and presenting himself before her, made
some steps, and then turned to the right or left. After that she danced to the
other end, and he followed, doing other steps; and so again, and again. “But
now,” says he, “in towns they dance it tumultuously, and content themselves
with making the five steps and some movements without any design, caring only
to be in position on the sixth of the bar” (pourvu qu’ils tombent en cadence).
In the four first steps, the left and right foot of the dancer were raised alternately,
and on the fifth of the bar he sprang into the air, twisting round, or capering, as
best he could. The repose on the sixth note gave more time for a lofty spring.\dcfootnote{\textit{}
Narea, in his Glossary, refers to \textit{Cinque pace}, but that
was a dance in common time; four steps to the four beats
of the bar, and the fifth on a long note at the commencement
of the second bar.}
“Let them take their pleasures,” says Burton, in his \textit{Anatomy of Melancholy};
“young men and maids flourishing in their age, fair and lovely to behold, well
attired, and of comely carriage, dancing a Greek \textit{Galliarde}, and, \textit{as their dance
requireth}, keep their time, now turning, now tracing, now apart, now altogether,
now a curtesie, then a caper, \&c., it is a pleasant sight.”

The following tune is from \textit{The Dancing Master} of 1686, called “\textit{Joan Sanderson,
or The Cushion Dance}, an old Round Dance.”

\includemusic{chappellV1069.pdf}

\pagebreak
%157

Reverting to the pavan and galliard, Morley says, “The pavan” (derived from
pavo, a peacock) “for grave dancing; galliards, which usually follow pavans, they
are for a lighter and more stirring kind of dancing.” The pavan was sometimes
danced by princes and judges in their robes, and by ladies with long trains held up
behind them; but usually the galliard followed the pavan, much in the same manner
as the gavotte follows the minuet. Butler, in his \textit{Principles of Musick}, 1636, says,
“Of this sort (the Ionic mood) are pavans, invented for a slow and soft kind of
dancing, altogether in duple proportion [common time]. Unto which are framed
galliards for more quick and nimble motion, always in triple proportion: and,
therefore, the triple is oft called galliard time, and the duple pavan time. In this
kind is also comprehended the \textit{infinite multitude of Ballads}, set to sundry pleasant
and delightful tunes by cunning and witty composers, with \textit{country dances} fitted
unto them,... and which surely might and would be more freely permitted by
our sages, were they used as they ought, only for health and recreation.”—(p.~8.)
At this time Puritanism was nearly at its height.

\musictitle{With My Flock As Walked I.}

Stafford Smith found this song, with the tune, in a manuscript of about the year
1600, and printed it in his \textit{Musica Antiqua}, p. 57. I discovered a second copy of
the tune in Elizabeth Rogers’ MS. Virginal book, in the British Museum, under
the name of \textit{The faithful Brothers}.

The song is evidently in allusion to Queen Elizabeth, and in the usual complimentary
style to her beauty, to her vow of virginity, \&c.

\musicinfo{Gracefully.}{}

\includemusic{chappellV1070.pdf}

\pagebreak
%158
\changefontsize{0.99\defaultfontsize}

\settowidth{\versewidth}{That they themselves might move}
\begin{dcverse}Such a face she had for to\\
Invite any man to love her;\\
But her coy behaviour taught\\
That it was but in vain to move her;\\
For divers so this dame had wrought\\
That they themselves might move her.\scfootnote
{This line is evidently incorrect, but I have no other copy to refer to.}


Phœbus for her favour spent\\
His hair, her fair brows to cover;\\
Venus’ cheek and lips were sent,\\
That Cupid and Mars might move her;\\
But Juno, alone, her nothing lent,\\
Lest Jove himself should love her.

Though she be so pure and chaste,\\
That nobody can disprove her;\\
So demure and straightly cast,\\
That nobody dares to move her;\\
Yet is she so fresh and sweetly fair\\
That I shall always love her.

Let her know, though fair she be,\\
That there is a power above her;\\
Thousands more enamoured shall be,\\
Though little it will move her;\\
She still doth vow virginity,\\
When all the world doth love her.
\end{dcverse}
\musictitle{Go No More A Rushing.}

This tune is called \textit{Go no more a rushing}, in a MS. Virginal Book of Byrd’s
arrangements and compositions, in the possession of Dr. Rimbault; and \textit{Tell me,
Daphne}, in Queen Elizabeth’s Virginal Book.

\musicinfo{Moderate time.}{}

\includemusic{chappellV1071.pdf}

\musictitle{The Blind Beggar’s Daughter of Bethnal Green.}

This tune was found by Dr. Rimbault in a MS. volume of Lute Music, written
by Rogers, a celebrated lutenist of the reign of Charles II., in the library at
Etwall Hall, Derbyshire. It is there called \textit{The Cripple}, and the ballad of
\textit{The stout Cripple of Cornwall} is directed to be sung to the tune of \textit{The blind
Beggar}. See Roxburghe Collection, i. 389, and Bagford, i. 32. It is also in
Evans’ \textit{Old Ballads}, i. 97 (1810); but, as too frequently the case, the name of
the tune to which it was to be sung, is there omitted.
\pagebreak
%159

Pepys, in his diary, 25th June, 1663, speaks of going with Sir William and
Lady Batten, and Sir J. Minnes, to Sir W. Rider’s, at Bednall Green, to dinner,
“a fine place;” and adds, “This very house was built by the blind Beggar of
Bednall Green, so much talked of and sang in ballads; but they say it was only
some outhouses of it.” The house was called Kirby Castle, then the property of
Sir William Ryder, Knight, who died there in 1669.

“This popular old ballad,” says Percy, “was written in the reign of Elizabeth,
as appears not only from verse 23, where the arms of England are called the
‘Queenes armes;’ but from its tune being quoted in other old pieces written in
her time. See the ballad on \textit{Mary Ambree},” \&c.

In a black-letter book called \textit{The World’s Folly}, we read that “a dapper fellow,
that in his youth had spent more than he got, on his person, fell to singing
\textit{The blind Beggar}, to the tune of \textit{Heigh ho}!”—(\textit{Brit. Bibliographer}, ii. 560.)

In the “Collection of Loyal Songs written against the Rump Parliament,” and
in “Rats rhimed to death, or the Rump Parliament hang’d up in the shambles”
(1660), are many songs to the tune of \textit{The blind Beggar}, as well as in the King’s
Pamphlets, Brit. Museum.

Among them, “A Hymn to the gentle craft, or Hewson’s lamentation”
(a satire on Lord Hewson, one of Cromwell’s lords, who had been a cobbler,
and had but one eye), and “The second Martyrdom of the Rump.”

The tune was sometimes called \textit{Pretty Bessy}, and a ballad to be sung to it,
under that name, is in the Roxburghe Collection, i. 142.

\musicinfo{Moderate time and with expression.}{}

\includemusic{chappellV1072.pdf}

\pagebreak
%160

The ballad of The blind Beggar will be found in Percy’s \textit{Reliques}, book ii.,
series 2; in the Roxburghe Collection, i. 10; and in Dixon’s \textit{Songs of the Peasantry
of England}. It is still kept in print in Seven Dials, and sung about the country,
but to the following tune.

\musicinfo{Moderate time and with expression.}{}
\medskip

\includemusic{chappellV1073.pdf}

\musictitle{Cock Lorrel, Or Cook Lawrel.}

This tune is in the \textit{Choice Collection of 180 Loyal Songs}, \&c. (3rd edit. 1685),
and in \textit{Pills to purge Melancholy}, as well as in every edition of \textit{The Dancing
Master}, from 1650 to 1725. In \textit{The Dancing Master} it is called \textit{An old man is
a bed full of bones}, from a song, of which four lines are quoted in Rowley’s
\textit{A Match at Midnight}, act i., sc. 1., and one in Shirley’s \textit{The Constant Maid},
act ii., sc. 2., where the usurer’s niece sings it.

The song of \textit{Cook Lorrel} is in Ben Jonson’s masque, \textit{The Gipsies metamorphosed}. 
Copies are also in the Pepys Collection of Ballads; in Dr. Percy’s folio
MS., p.~182;\scfootnote
{See Dr. Dibdin’s Decameron, vol. 3.} and, with music, in \textit{Pills to purge Melancholy}. It is a satire upon
rogues and knaves of all classes supposed to be doomed to perdition. Cook
Lorrel, a notorious rogue, invites his Satanic Majesty into the Peak in Derbyshire
to dinner; and he, somewhat inconvenienced by the roughness of the
road, commences by feasting on the most delicate sinner:

\settowidth{\versewidth}{His stomach was queasie (for, riding there coach’d,}
\begin{scverse}
\begin{altverse}
\vleftofline{“}His stomach was queasie (for, riding there coach’d,\\
The jogging had caused some crudities rise);\\
To help it he called for a Puritan \textit{poach’d},\\
That used to turn up the \textit{eggs} of his eyes, \&c.”
\end{altverse}
\end{scverse}
\pagebreak
%161

Wynken de Worde printed a tract called \textit{Cocke Lorrel’s Bote}; in which persons
of all classes, and, among them the \textit{Mynstrelles}, are summoned to go on board
his Ship of Fools. \textit{Cock Lorels’s Boat} is mentioned in a MS. poem in the
Bodleian Library, called \textit{Doctour Double Ale}, and in John Heywood’s \textit{Epigrams
upon 300 Proverbs}, 1566 (in the Epigram upon a Busy-body, No. 189).

In S. Rowland’s \textit{Martin Markhall, his defence and answer to the Bellman of
London}, 1610, is a list of rogues by profession, in which \textit{Cock Lorrel} stands
second. He is thus described: “After him succeeded, by the general council,
one Cock Lorrell, the most notorious knave that ever lived. By trade he was a
tinker, often carrying a pan and hammer for shew; but when he came to a good
booty, he would cast his profession in a ditch, and play the padder.” In 1565,
a book was printed called \textit{The Fraternitye of Vacabondes; whereunto also is
adjoyned the twventy-five orders of knaves: confirmed for ever by Cocke Lorell}.

In \textit{Satirical Poems} by Lord Rochester (Harl. MSS., 6913) there is a ballad to
the tune of \textit{An old man is a bed full of bones}, but the air is far more generally
referred to by the name of \textit{Cock Lorrel}.

In the “Collection of Loyal Songs written against the Rump Parliament”
there are many to this air, such as “The Rump roughly but righteously
handled;” “The City’s Feast to the Lord Protector;” “St. George for England”
(commencing, “The Westminster Rump hath been little at ease”); \&c., \&c.
Others in the King’s Pamphlets, Brit. Mus.; in the \textit{Collection of 180 Loyal
Songs}, 1685; in \textit{Poems on Affairs of State}, vol. i., 1703; and in the Roxburghe
Collection of Ballads.

A tune called \textit{The Painter} is sometimes mentioned, and it appears to be
another name for this air, because the ballad of “The Painter’s Pastime: or a
woman defined after a new fashion,” \&c., was to be sung to the tune of \textit{Cook
Laurel}. A black-letter copy is in the Douce Collection (printed by P. Brooksby,
at the Golden Ball, \&c.).

Some copies of the tune are in a major, others in a minor key. The four lines
here printed to it are from an \textit{Antidote to Melancholy}, 1651, for, although some
of the ballads above quoted are witty, they would not be admissible in the
present day.

\includemusic{chappellV1074.pdf}

\pagebreak
%162

\musictitle{Fortune My Foe.}

The tune of \textit{Fortune} is in Queen Elizabeth’s Virginal Book; in William
Ballet’s MS. Lute Book; in Vallet’s \textit{Tablature de Luth}, book i., 1615, and
book ii., 1616; in \textit{Bellerophon}, 1622; in \textit{Nederlandtsche Gedenck-Clanck}, 1626; in
Dr. Camphuysen’s \textit{Stichtelycke Rymen}, 1652; and in other more recent publications. 
In the Dutch books above quoted, it is always given as an English air.

A ballad “Of one complaining of the mutability of Fortune” was licensed to
John Charlewood to print in 1565-6 (See Collier’s \textit{Ex. Reg. Stat. Comp}., p. 139).
A black-letter copy of “A sweet sonnet, wherein the lover exclaimeth against
Fortune for the loss of his lady’s favour, almost past hope to get it again, and in
the end receives a comfortable answer, and attains his desire, as may here appear:
to the tune of \textit{Fortune my foe},” is in the Bagford Collection of Ballads (643 m.,
British Museum). It begins as follows:—

\musicinfo{Slow.}{}

\includemusic{chappellV1075.pdf}

There are twenty-two stanzas, of four lines each, in the above.

\textit{Fortune my foe} is alluded to by Shakespeare in \textit{The Merry Wives of Windsor},
act ii., sc. 3; and the old ballad of \textit{Titus Andronicus}, upon which Shakespeare
founded his play of the same name, was sung to the tune. A copy of that ballad
is in the Roxburghe Collection, i. 392, and reprinted in Percy’s \textit{Reliques}.

Ben Jonson alludes to \textit{Fortune my foe}, in \textit{The case is altered}, and in his masque
\textit{The Gipsies Metamorphosed}; Beaumont and Fletcher, in \textit{The Custom of the
Country}, \textit{The Knight of the Burning Pestle}, and \textit{The Wild Goose Chase}; Lilly
gives the first verse in his \textit{Maydes Metamorphosis}, 1600; Chettle mentions the
tune in \textit{Kind-hart's Dreame},~1592; Burton, in his \textit{Anatomy of Melancholy}, 1621;
Shirley, in \textit{The Grateful Servant},~1630; \pagebreak Brome, in his \textit{Antipodes}, 1638. See
%164
also Lodge’s \textit{Rosalind}, 1590; \textit{Lingua}, 1607; \textit{Every Woman in her humour}, 1609;
\textit{The Widow’s Tears}, 1612; Henry Hutton’s \textit{Follie’s Anatomie}, 1619; \textit{The two
merry Milkmaids}, 1620; \textit{Vox Borealis}, 1641; \textit{The Rump, or Mirror of the
Times}, 1660; \textit{Tom’s Essence}, 1677, \&c. In Forbes’ \textit{Cantus}, 1682, is a parody
on \textit{Fortune my foe}, beginning, \textit{Satan my foe, full of iniquity}, with which the tune
is there printed.

One reason for the great popularity of this air is that “the metrical lamentations
of extraordinary criminals have been usually chanted to it for upwards of
these two hundred years.” Rowley alludes to this in his \textit{Noble Soldier}, 1634:
\settowidth{\versewidth}{The King! shall I be bitter ’gainst the King?}
\begin{scverse}
\vleftofline{“}The King! shall I be bitter ’gainst the King?\\
I shall have scurvy ballads made of me,\\
Sung to \textit{the hanging tune}!”
\end{scverse}
And in “The penitent Traytor: the humble petition of a Devonshire gentleman,
who was condemned for treason, and executed for the same, anno 1641,” the
last verse but two runs thus:
\begin{scverse}
\vleftofline{“}How could I bless thee, couldst thou take away\\
My life and infamy both in one day?\\
But this in ballads will survive I know,\\
Sung to that \textit{preaching tune, Fortune my foe}."
\end{scverse}
The last is from “Loyal Songs written against the Rump Parliament.”

Deloney’s ballad, “The Death of King John,” in his \textit{Strange Histories}, 1607;
and “The most cruel murder of Edward V., and his brother the Duke of York,
in the Tower, by their uncle, the Duke of Gloucester” (reprinted in Evans’ \textit{Old
Ballads}, iii. 13, ed. 1810), are to this tune; but ballads of this description which
were sung to it are too many for enumeration. In the first volume of the Roxburghe
Collection, at pages 136, 182, 376, 392, 486, 487, 488, and 490, are
ballads to the tune of \textit{Fortune}, and all about murders, last dying speeches, or some
heavy misfortunes.

In the Pepys’ Collection, i. 68, is a ballad of “The lamentable burning of the
city of Cork, by the lightning which happened the last day of May, 1622, after
the prodigious battle of the stares” (\ie, starlings), “which fought most strangely
over and near the city the 12th and 14th May, 1621.”

Two other ballads require notice, because the tune is often referred to under
their names, \textit{Dr. Faustus}, and \textit{Aim not too high}. The first, according to the title
of the ballad, is “The Judgment of God shewed upon Dr. John Faustus: tune,
\textit{Fortune my foe}.” A copy is in the Bagford Collection.\scfootnote
{It is also printed in my National English Airs, quarto, part i., 1838.}
It is illustrated by two
woodcuts at the top: one representing Dr. Faustus signing the contract with the
devil; and the other shewing him standing in a magic circle, with a wand in his
left hand, and a sword with flame running up it, in his right: a little devil
seated on his right arm. Richard Jones had a licence to print the ballad “of the
life and deathe of Dr. Faustus, the great cungerer,” on the 28th Feb., 1588-9.

In the Roxburghe Collection, i. 434, is “Youth’s warning piece,” \&c., “to the
tune of \textit{Dr. Faustus};” printed for A. K., 1636. And in Dr. Wild’s\textit{ Iter
Boreal}e, 1671, “The recantation of a penitent Proteus,” \&c., to the tune of
\textit{Dr. Faustus}.
\pagebreak
%164

The other name is derived from—
\settowidth{\versewidth}{So give Him thanks that shall encrease it still,”}
\begin{scverse}
\vleftofline{“}An excellent song, wherein you shall finde\\
Great consolation for a troubled mind.
\end{scverse}
To the tune of \textit{Fortune my foe}.” Commencing thus:
\begin{scverse}
\vleftofline{“}\textit{Ayme not too hie} in things above thy reach;\\
Be not too foolish in thine owne conceit;\\
As thou hast wit and worldly wealth at will,\\
So give Him thanks that shall encrease it still,”\&c.
\end{scverse}

This ballad is also in the Roxburghe Collection, i. 106., printed by the “Assignes
of Thomas Symcocke:” and, in the same, others to the tune of \textit{Aim not too high}
will be found, viz., in vol. i., at pages 70, 78, 82, 106, 132, and 482; in vol. ii.,
at pages 128, 130, 189, 202, 283, 482, and 562, \&c.

In the Douce Collection there is a ballad of “The manner of the King’s”
[Charles the First’s] “Trial at Westminster Hall,” \&c.; “the tune is \textit{Aim not
too high}.”

\musictitle{Death And The Lady.}

\textit{Death and the Lady} is one of a series of popular ballads which had their rise
from the celebrated \textit{Dance of Death}. A \textit{Dance of Death} seems to be alluded to
in \textit{The Vision of Pierce Plowman}, written about 1350:
\begin{scverse}
\vleftofline{“}Death came driving after, and al[l] to dust pashed\\
Kyngs and Kaisars, Knights and Popes;”
\end{scverse}
but the subject was rendered especially popular in England by Lydgate’s free
translation from a French version of the celebrated German one by Machaber.

Representations of \textit{The Dance of Death} were frequently depicted upon the
walls of cloisters and cathedrals. Sir Thomas More speaks of one “pictured in
Paules,” of which Stow, in his \textit{Survey of London}, gives the following account:—
“John Carpenter, town clerk of London in the reign of Henry VI., caused, with
great expense, to be curiously painted upon board, about the north cloister of
Paul’s, a monument of Death leading all estates, with the speeches of Death, and
answer of every state. This cloister was pulled down in 1549.”

On the walls of the Hungerford Chapel in Salisbury Cathedral was a painting
executed about 1460, representing Death holding conversation with a young
gallant, attired in the fullest fashion, who thus addresses him:—
\settowidth{\versewidth}{Alasse, Dethe, alasse! a blessful thing thou were}
\begin{scverse}
\vleftofline{“}Alasse, Dethe, alasse! a blessful thing thou were\\
If thou woldyst spare us in our lustynesse,\\
And cum to wretches that bethe of he[a]vy chere,\\
When they thee clepe [call] to slake their dystresse.\\
But, owte alasse! thyne owne sely self-willdnesse\\
Crewelly we[a]rieth them that sighe, wayle, and weepe,\\
To close their eyen that after thee doth clepe.’’
\end{scverse}
To which Death gloomily replies:
\begin{scverse}
\vleftofline{“}\vleftofline{“}Graceles Gallante, in all thy luste and pryde\\
Remembyr that thou ones schalte dye;\\
De[a]th shold fro’ thy body thy soule devyde,\\
Thou mayst him not escape, certaynlỳ.
\end{scverse}
\pagebreak
%165
\changefontsize{0.97\defaultfontsize}

\begin{scverse}
To the de[a]de bodys cast downe thyne eye,\\
Behold them well, consyder and see,\\
For such as they are, such shalt thou be.”
\end{scverse}

Among the Roxburghe Ballads is one entitled “Death’s uncontrollable summons,
or the mortality of mankind; being a dialogue between Death and a young
man,” which very much resembles the verses in the Hungerford Chapel, above
quoted. We have also “The dead man’s song,” reprinted in Evans’ Collection,
“Death and the Cobbler,” and “Death’s Dance,” proving the popularity of these
moralizations on death. Another “Dance and Song of Death,” which was
licensed in 1568, has been printed at page 85.

In the Douce Collection is a black-letter copy of “The midnight messenger, or
a sudden call from an earthly glory to the cold grave, in a dialogue between Death
and a rich man,” \&c., beginning—
\begin{scverse}\vleftofline{“}Thou wealthy man, of large possessions here,\\
Amounting to some thousand pounds a year,\\
Extorted by oppression from the poor,\\
The time is come that thou shalt be no more,” \&c.;
\end{scverse}
which is reprinted in Dixon’s \textit{Songs of the Peasantry}, \&c.

In Mr. Payne Collier’s MS. volume, written in the reign of James I., is a
dialogue of twenty-four stanzas, between “Life and Death,” commencing—
\settowidth{\versewidth}{Nay, what art thou, that I should give}
\begin{scverse}\vleftofline{\textit{Life}.—}“Nay, what art thou, that I should give\\
\vin To thee my parting breath?\\
Why may not I much longer live?”\\
\vleftofline{\textit{Death}.—}\vin “Behold! my name is Death.”\\
\vleftofline{\textit{Life}.—}“I never have seen thy face before;\\
\vin Now tell me why thou came:\\
I never wish to see it more—\\
\vleftofline{\textit{Death}.—}\vin “Behold! Death is my name,” \&c.
\end{scverse}

The following “Dialogue betwixt an Exciseman and Death” is from a copy in
the Bagford Collection, dated 1659.

\begin{dcverse}Upon a time when Titan’s steeds were driven\\
To drench themselves against the western heaven;\\
And sable Morpheus had his curtains spread,\\
And silent night had laid the world to bed,\\
’Mongst other night-birds which did seek for prey,\\
A blunt exciseman, which abhorr’d the day,\\
Was rambling forth to seeke himself a booty\\
’Mongst merchants’ goods which had not paid the duty:\\
But walking all alone, Death chanc’d to meet him,\\
And in this manner did begin to greet him.\\
\vin\vin\vin\vin \textsc{death}.\\
Stand, who comes here? what means this knave to peepe\\
And sculke abroad, when honest men should sleepe?\\
Speake, what’s thy name? and quickly tell me this,\\
Whither thou goest, and what thy bus’ness is?\\
\vin\vin\vin\vin \textsc{exciseman}.\\
Whate’er my bus’ness is, thou foule-monthed scould,\\
I’de have you know I scorn to be coutroul’d\\
By any man that lives; much less by thou,\\
Who blurtest out thou knowst not what, nor how;\\
I goe about my lawful bus’ness; and\\
I’le make you smarte for bidding of mee stand.\\
\vin\vin\vin\vin \textsc{death}.\\
Imperious cox-combe! is your stomach vext?\\
Pray slack your rage, and harken what comes next:\\
I have a writt to take you up; therefore,\\
To chafe your blood, I bid yon stand, once more.
\end{dcverse}

\pagebreak
%166
\changefontsize{1.01\defaultfontsize}

\begin{dcverse}\footnotesize
\vin\vin\vin\vin \textsc{exciseman}.

A writt to take mee up! excuse mee, sir,\\
You doe mistake, I am an officer\\
In publick service, for my private wealth;\\
My bus’ness is, if any seeke by stealth\\
To undermine the states, I doe discover\\
Their falsehood; therefore hold your hand,— give over.

\vin\vin\vin\vin \textsc{death}.

Nay, fair and soft! ’tis not so quickly done\\
As you conceive it is: I am not gone\\
A jott the sooner, for your hastie chat\\
Nor bragging language; for I tell you flat\\
’Tis more than so, though fortune seeme to thwart us,\\
Such easie terms I don’t intend shall part us.\\
With this impartial arme I’ll make you feele\\
My fingers first, and with this shaft of steele\\
I’le peck thy bones! as thou alive wert hated,\\
So dead, to doggs thou shalt be segregated.

\vin\vin\vin\vin \textsc{exciseman}.

I’de laugh at that; I would thou didst but dare\\
To lay thy fingers on me; I’de not spare\\
To hack thy carkass till my sword was broken,\\
I’de make thee eat the wordes which thou hast spoken;\\
All men should warning take by thy transgression,\\
How they molested men of my profession.\\
My service to the states is so welle known,\\
That I should but complaine, they’d quickly owne\\
My publicke grievances; and give mee right\\
To cut your eares, before to-morrow night.

\vin\vin\vin\vin \textsc{death}.

Well said, indeed! but bootless all, for I\\
Am well acquainted with thy villanie;\\
I know thy office, and thy trade is such,\\
Thy service little, and thy gaines are much:\\
Thy braggs are many; but ’tis vaine to swagger,\\
And thinke to fighte mee with thy guilded dagger:\\
As I abhor thy person, place, and threate,\\
So now I’le bring thee to the judgement seate.

\columnbreak
\vin\vin\vin\vin \textsc{exciseman}.

The judgement seate! I must confess that word\\
Doth cut my heart, like any sharpnèd sword:\\
What! come t’ account! methinks the dreadful sound\\
Of every word doth make a mortal wound,\\
Which sticks not only in my outward skin,\\
But penetrates my very soule within.\\
’Twas least of all my thoughts that ever Death\\
Would once attempt to stop excisemen’s breath.\\
But since ’tis so, that now I doe perceive\\
You are in earnest, then I must relieve\\
Myself another way: come, wee’l be friends,\\
If I have wrongèd thee, I’le make th’ amendes.\\
Let’s joyne together; I’le pass my word this night\\
Shall yield us grub, before the morning light.\\
Or otherwise (to mitigate my sorrow),\\
Stay here, I’le bring you gold enough tomorrow.

\vin\vin\vin\vin \textsc{death}.

To-morrow’s gold I will not have; and thou\\
Shalt have no gold upon to-morrow: now\\
My final writt shall to th’ execution have thee,\\
All earthly treasure cannot help or save thee.

\vin\vin\vin\vin \textsc{exciseman}.

Then woe is mee! ah! how was I befool’d!\\
I thought that gold (which answereth all things) could\\
Have stood my friend at any time to baile mee!\\
But griefe growes great, and now my trust doth faile me.\\
Oh! that my conscience were but clear within,\\
Which now is rackèd with my former sin;\\
With horror I behold my secret stealing,\\
My bribes, oppression, and my graceless dealing;\\
My office-sins, which I had clean forgotten,\\
Will gnaw my soul when all my bones are rotten:\\
I must confess it, very griefe doth force mee,\\
Dead or alive, both God and man doth curse mee,\\
Let all excisemen hereby warning take,\\
To shun their practice for their conscience sake.
\end{dcverse}

Of all the ballads on the subject of Death, the most popular, however, was
\textit{Death and the Lady}. In Mr. George Daniel’s Collection there is a ballad
“imprinted at London by Alexander Lacy” (about 1572), at the end of which
is a still older woodcut, representing \textit{Death and the Lady}. It has been used as
an ornament to fill up a blank in one to which it bears no reference; but was, in
all probability, engraved for this, or one \pagebreak on the same subject. The tune is in
%167
Henry Carey’s \textit{Musical Century}, 1738. He calls it “\textit{the old tune} of Death and
the Lady.” Also in \textit{The Cobbler's Opera}, 1729; \textit{The Fashionable Lady}; and
many others about the same date.

The oldest copies of \textit{Aim not too high} direct it to be sung to the tune of \textit{Fortune},
but there is one class of ballads, said to be to the tune of \textit{Aim not too high}, that
could not well be sung to that air. The accent of \textit{Fortune my foe} is on the first
syllable of each line; exactly agreeing with the tune. But these ballads on
Death have the accent on the second, and agree with the tune of \textit{Death and the
Lady}. See, for instance, the four lines above quoted from \textit{The Dialogue between
Death and the rich man}, which the black-letter copies direct to be sung to the
tune of \textit{Aim not too high}. I believe, therefore, that \textit{Aim not too high} had either
a separate tune, which is the same I find under the name of \textit{Death and the Lady},
or else, \textit{Fortune}, being altered by the singer for the accent of those ballads, and
sung in a major key, gradually acquired a different shape. (Many of these airs
are found both in major and minor keys.) This would account for \textit{Fortune} and
\textit{Aim not too high} being so frequently cited as different tunes in ballads printed
about the same period.

I suppose, then, that ballads to the tune of \textit{Aim not too high} may be either
to \textit{Fortune}, or \textit{Death and the Lady}; a point to be determined generally by the
accent of the words.

The ballad of \textit{Death and the, Lady} is printed in a small volume entitled \textit{A Guide
to Heaven}, 12mo., 1736; and it is twice mentioned in Goldsmith’s popular tale,
\textit{The Vicar of Wakefield}, first printed in 1776.

\musicinfo{Slow.}{}

\includemusic{chappellV1076.pdf}

\pagebreak
%168

\begin{dcverse}\scriptsizerrr
\vin\vin\vin\vin\vin LADY.

What bold attempt is this? pray let me know\\
From whence you come, and whither I must go!\\
Shall I, who am a lady, stoop or bow\\
To such a pale-fac’d visage? Who art thou?

\vin\vin\vin\vin\vin DEATH

Do you not know me? Well, I’ll tell you, then:\\
’Tis I who conquer all the sons of men!\\
No pitch of honour from my dart is free;\\
My name is Death! have you not heard of me?

\vin\vin\vin\vin\vin LADY.

Yes, I have heard of thee time after time;\\
But, being in the glory of my prime,\\
I did not think you would have called so soon.\\
Why must my sun go down before its noon?

\vin\vin\vin\vin\vin DEATH

Talk not of noon! you may as well be mute;\\
This is no more the time for to dispute:\\
Your riches, jewels, gold, and garments brave—\\
Houses and lands, must all new masters have.\\
Though thy vain heart to riches was inclin’d,\\
Yet thou must die, and leave them all behind.

\vin\vin\vin\vin\vin LADY.

My heart is cold; I tremble at the news!\\
Here’s bags of gold if thou wilt me excuse,\\
And seize on them: and finish thou the strife\\
Of those that are most weary of their life.\\
Are there not many bound in prison strong,\\
In bitter grief of soul have languish’d long?\\
All such would find the grave a place of rest\\
From all the griefs by which they are opprest.\\
Besides, there’s many both with hoary head,\\
And palsied joints, from which all strength is fled.\\
Release thou those, whose sorrows are so great,\\
But spare my life to have a longer date.

\vin\vin\vin\vin\vin DEATH

Though they, by age, are full of grief and pain,\\
Yet their appointed time they must remain.\\
I come to none before their warrant’s seal’d,\\
And when it is, all must submit and yield;\\
I take no bribe, believe me this is true;\\
Prepare yourself, for now I come for you.

\vin\vin\vin\vin\vin LADY.

Be not severe! O Death! let me obtain\\
A little longer time to live and reign!\\
Fain would I stay, if thou my life wilt spare,\\
I have a daughter, beautiful and fair;\\
I’d live to see her wed, whom I adore;\\
Grant me but this, and I will ask no more.

\vin\vin\vin\vin\vin DEATH

This is a slender, frivolous excuse,\\
I have you fast, and will not let you loose;\\
Leave her to Providence, for you must go\\
Along with me, whether you will or no.\\
I, Death, command e’en kings to leave their crown,\\
And at my feet they lay their sceptres down.\\
If unto kings this favour I don’t give,\\
But cut them off, can you expect to live\\
Beyond the limits of your time and space?\\
No I I must send you to another place.

\vin\vin\vin\vin\vin LADY.

You learned doctors, now express your skill,\\
And let not Death of me obtain his will;\\
Prepare your cordials, let me comfort find,\\
And gold shall fly like chaff before the wind!

\vin\vin\vin\vin\vin DEATH

Forbear to call, their skill will never do,\\
They are but mortals here, as well as you;\\
I gave the fatal wound, my dart is sure;\\
’Tis far beyond the doctor’s skill to cure.\\
How freely can you let your riches fly\\
To purchase life, rather than yield to die!\\
But while you flourish’d here in all your store,\\
You would not give one penny to the poor,\\
Who in God’s name their suit to you did make;\\
You would not spare one penny for His sake.\\
The Lord beheld wherein you did amiss,\\
And calls you hence to give account for this.

\vin\vin\vin\vin\vin LADY.

Oh, heavy news! must I no longer stay?\\
How shall I stand at the great judgment day.”\\
Down from her eyes the crystal tears did flow:\\
She said, “None knows what now I undergo.\\
Upon a bed of sorrow here I lie.\\
My carnal life makes me afraid to die;\\
My sins, alas! are many, gross, and foul,\\
Lord Jesus Christ have mercy on my soul!\\
And though I much deserve thy righteous frown.\\
Yet pardon, Lord, and send a blessing down!”

Then, with a dying sigh, her heart did break,\\
And she the pleasures of this world forsake.\\
Thus do we see the high and mighty fall,\\
For cruel death shows not respect at all\\
To any one of high or low degree:\\
Great men submit to death, as well as we.\\
If old or young, our life is but a span—\\
A lump of clay—so vile a creature’s man.\\
Then happy they whom Christ has made his care—\\
Die in the Lord, and ever blessed are!
\end{dcverse}
\pagebreak
%169

\musictitle{The King And The Miller Of Mansfield.}

This tune was found by Dr. Rimbault in a MS. volume of virginal music in the
possession of T. Birch, Esq., of Repton, Derbyshire. The black-letter copies of
the ballad of \textit{King Henry II. and the Miller of Mansfield}, direct it to be sung to
the tune of \textit{The French Levalto}, and, as the air was found under that name, it
\textit{may} be a French tune, although neither Dr. Rimbault nor I think it so. The
progression of the last four notes in each part is very English in character.

There are copies of the ballad in the Roxburghe Collection (v. i. 178 and 228);
in the Bagford (p. 25); and in the Pepys. It is also in \textit{Old Ballads}, 1727,
v. i., p. 53; and in Percy’s \textit{Reliques}, series 3, book ii. \textit{The French Levalto} is
frequently referred to as a ballad tune.

\musicinfo{Rather slow and gracefully.}{}
\medskip

\includemusic{chappellV1077}

\pagebreak
%170

\settowidth{\versewidth}{All a long summer’s day rode the king pleasantlye,}
\indentpattern{010100}
\begin{scverse}
\begin{patverse}
All a long summer’s day rode the king pleasantlye,\\
With all his princes and nobles eche one;\\
Chasing the hart and hind, and the bucke gallantlye,\\
Till the darke evening forc’d all to turne home.\\
Then at last, riding fast, he had lost quite\\
All his lords in the wood, late in the night.
\end{patverse}

\begin{patverse}
Wandering thus wearilye, all alone, up and downe,\\
With a rude miller he mett at the last:\\
Asking the ready way unto faire Nottingham;\\
Sir, quoth the miller, I meane not to jest,\\
Yet I thinke, what I thinke, sooth for to say,\\
You doe not lightlye ride out of your way.
\end{patverse}

\begin{patverse}
Why, what dost thou thinke of me, quoth our king merrily\\
Passing thy judgment upon me so briefe?\\
Good faith, sayd the miller, I meane not to flatter thee;\\
I guess thee to be but some gentleman thiefe;\\
Stand thee backe, in the darke; light not adowne,\\
Lest that I presentlye crack thy knaves crowne. \&c.
\end{patverse}
\end{scverse}

\musictitle{Little Musgrave and Lady Barnard.}

This ballad is quoted in Fletcher’s \textit{Knight of the Burning Pestle}, and \textit{Monsieur
Thomas}; in \textit{The Varietie}, 1649; and in Davenant’s \textit{The Wits}, where Twack, an
antiquated beau, boasting of his qualifications, says—

“Besides, I sing \textit{Little Musgrove}; and then
For \textit{Chevy Chase} no lark comes near me.”

A copy of the ballad is in the Bagford Collection, entitled “A lamentable
ballad of Little Musgrove and the Lady Barnet, to an excellent new tune.” It is
also in \textit{Wit restored}, 1658; in Dryden’s \textit{Miscellany Poems}, iii. 312 (1716); and
in Percy’s \textit{Reliques}, series 3, book i.

The tune is the usual traditional version.

\musicinfo{Gracefully.}{}

\includemusic{chappellV1078.pdf}

\pagebreak
%171

\musictitle{The Gipsies’ Round.}

The tune from Queen Elizabeth’s Virginal Book.

Whenever gipsies are introduced in old plays, we find some allusions to their
singing, dancing, or music, and generally a variety of songs to be sung by them.
In Middleton’s \textit{Spanish Gipsy}, Roderigo, being invited to turn gipsy, says—
\settowidth{\versewidth}{I can neither dance, nor sing; but if my pen}
\begin{scverse}
\vleftofline{“}I can neither dance, nor sing; but if my pen\\
From my invention can strike music tunes,\\
My head and brains are yours.”
\end{scverse}
In other words, “I think I can invent tunes, and therefore have one qualification
for a gipsy, although I cannot dance, nor sing.”

By \textit{Round} is here meant a country dance. Country dances were formerly danced
quite as much in rounds as in parallel lines; and in the reign of Elizabeth were
in favour at court, as well as at the May-pole. In the Talbot papers, Herald’s
College, is a letter from the Earl of Worcester to the Earl of Shrewsbury, dated
Sep. 19th, 1602, in which he says, “We are frolic here in court; much dancing
in the privy chamber of country dances before the Queen’s Majesty, who is
much pleased therewith.”—(Lodge, iii. 577.)

\musicinfo{Boldly.}{}

\includemusic{chappellV1079.pdf}

\musictitle{The Legend Of Sir Guy.}

This ballad was entered to Richard Jones on Jan. 5th, 1591-2, as “A plesante
songe of the valiant actes of Guy of Warwicke, to the tune of \textit{Was ever man so
lost in love}.” The copy in the Bagford Collection (p. 19) is entitled “A pleasant
song of the valiant deeds of chivalry achieved by that noble knight, Sir Guy of
Warwick, who, for the love of fair Phillis, \pagebreak became a hermit, and died in a cave of
%172
a craggy rock, a mile distant from Warwick. Tune, \textit{Was ever man}, \&c.” Other
copies are in the Pepys Collection; Roxburghe, iii. 50; and in Percy’s \textit{Reliques},
series 3, book~ii.

It is quoted in Fletcher’s \textit{Knight of the Burning Pestle}, act ii., sc. 8; and in
\textit{The little French Lawyer}, act ii., sc. 3.

William of Nassyngton (about 1480) mentions stories of Sir Guy as usually
sung by minstrels at feasts. (See ante page 45.) Puttenham, in his \textit{Art of
Poetry}, 1589, says they were then commonly sung to the harp at Christmas
dinners and bride-ales, for the recreation of the lower classes. And in Dr. King’s
\textit{Dialogues of the Dead}, “It is the negligence of our ballad singers that makes us to
be talked of less than others: for who, almost, besides \textit{St. George, King Arthur,
Bevis, Guy}, and \textit{Hickathrift}, are,in the chronicles.”—(Vol. i., p. 153.)

This tune is from the ballad-opera of \textit{Robin Hood}, 1730, called \textit{Sir Guy}.

\musicinfo{Slow.}{}

\includemusic{chappellV1080.pdf}

\pagebreak
%173

\musictitle{Loth To Depart.}

Tune from Queen Elizabeth’s Virginal Book, where it is arranged by Giles
Farnaby.

In Beaumont and Fletcher’s \textit{Wit at several Weapons}, act ii., sc. 2, Pompey
makes his exit singing \textit{Loath to depart}. In Middleton’s \textit{The Old Law}, act iv.,
sc. 1, “The old woman is \textit{loath to depart}; she never sung other tune in her life.”
In the ballad of \textit{Arthur of Bradley}, which exists in black-letter, and in the \textit{Antidote
to Melancholy}, 1661, are the following lines:—
\settowidth{\versewidth}{Then Will and his sweetheart}
\begin{scverse}
\vleftofline{“}Then Will and his sweetheart\\
Did call for \textit{Loth to depart.}”
\end{scverse}
Also in Chapman’s \textit{Widow’s Tears}, 1612; \textit{Vox Borealis}, 1641; and many others.

A \textit{Loth to depart} was the common term for a song sung, or a tune played, on
taking leave of friends. So in a \textit{Discourse on Marine Affairs} (Harl. MSS.,
No.~1341) we find, “Being again returned into his barge, after that the trumpets
have sounded a \textit{Loathe to departe}, and the barge is fallen off a fit and fair birth
and distance from the ship-side, he is to be saluted with so many guns, for an
adieu, as the ship is able to give, provided that they be always of an odd
number.”—(Quoted in a note to Teonge’s Diary, p. 5.) In Tarlton’s \textit{News out of
Purgatory}, (about 1589), “And so, with a \textit{Loath to depart}, they took their
leaves;” and in the old play of \textit{Damon and Pithias}, when Damon takes leave,
saying, “Loth am I to depart,” he adds, “O Music, sound my doleful plaints
when I am gone away,” and the regals play “a mourning song.”

The following are the words of a round in \textit{Deuteromelia}, 1609:—
\settowidth{\versewidth}{Sing with thy mouth, sing with thy heart.}
\begin{scverse}
“Sing with thy mouth, sing with thy heart.\\
Like faithful friends, sing \textit{Loath to depart};\\
Though friends together may not always remain.\\
Yet \textit{Loath to depart} sing once again.”
\end{scverse}

The four lines here printed to the tune, are part of a song called “Loth to
depart,” in \textit{Wit’s Interpreter}, 1671. It is also in \textit{The Loyal Garland}; and, with
some alteration, in Dryden’s \textit{Miscellany Poems}, iv., 80. It is there attributed to
Mr. J. Donne.

\musicinfo{Slow.}{}

\includemusic{chappellV1081.pdf}

\pagebreak
%174

\musictitle{Queen Eleanor’s Confession.}

This is the traditional tune to the ballad which is printed in Percy’s \textit{Reliques
of Ancient Poetry} (No. 8, series ii., book 2). A copy is in the Bagford Collection,
i.~26, to be sung to “a pleasant new tune.”

\musicinfo{Moderate time.}{}

\includemusic{chappellV1082.pdf}

\musictitle{Essex’s Last Good-Night, or Well-A-Day.}

This air is contained in Elizabeth Rogers’ MS. Virginal Book (Brit. Mus.);
and in a transcript of virginal music made by Sir John Hawkins, now in the possession
of Dr. Rimbault. In the former it is entitled \textit{Essex’s last Good-night}, and
there are but eight bars in the tune; the latter is called \textit{Well-a-day}, and consists
of sixteen bars.

The ballad of \textit{Essex's last Good-night} is in the Pepys Collection, i. 106; and
Roxburghe, i. 101, and 185. In the Pepys Collection it is called “A lamentable
new ballad upon the Earl of Essex his death; to the tune of \textit{The King's last
Good-night}.” In the Roxburghe, i. 101, to the tune of \textit{Essex's last Good-night}.
It is printed in Evans’ \textit{Old Ballads}, iii. 167 (1810); but, as usual, without the
name of the tune. The first verse of the Pepys copy is as follows:—
\settowidth{\versewidth}{All you that cry O hone, O hone! [alas],}
\begin{scverse}
\begin{altverse}
\vleftofline{“}All you that cry O hone, O hone! [alas],\\
Come now and sing O Lord with me;\\
For why our jewel is from us gone,\\
The valiant knight of chivalry.\\
Of rich and poor belov’d was he;\\
In time an honorable knight;\\
When by our laws condemn’d was he.\\
And lately took his last \textit{Good-night}.”
\end{altverse}
\end{scverse}

This is on the death of Walter Devereux, Earl of Essex (father of Queen Elizabeth’s
favorite), who died in Dublin, in 1576. Another on the same subject, and
in the same metre, has been printed by Mr. Payne Collier, in his \textit{Extracts from
the Registers of the Stationers' Company}, ii. 35; beginning thus:—
\pagebreak
%175

\settowidth{\versewidth}{Lament, each subject, and the head}
\begin{scverse}
\begin{altverse}
\vleftofline{“}Lament, lament, for he is dead\\
Who serv’d his prince most faithfully;\\
Lament, each subject, and the head\\
Of this our realm of Brittany.\\
Our Queen has lost a soldier true;\\
Her subjects lost a noble friend:\\
Oft for his queen his sword he drew,\\
And for her subjects blood did spend,” \&c.
\end{altverse}
\end{scverse}

The ballad of \textit{Well-a-day} is entitled “A lamentable dittie composed upon the
death of \textit{Robert} Lord Devereux, late Earle of Essex, who was beheaded in the
Tower of London, upon Ash Wednesday, in the morning, 1601. To the tune of
\textit{Well-a-day}. Imprinted at London for Margret Allde, \&c., 1603. Reprinted in
Payne Collier’s \textit{Old Ballads}, 124, 8vo., 1840; and in Evans’, iii. 158. Copies
are also in the Bagford and Roxburghe Collections (i. 184); and Harl. MSS.,
293. The first verse is here given with the tune.

The ballads to the tune of \textit{Essex's last Good-night} are in quite a different metre
to those which were to be sung to \textit{Well-a-day}; and either the melody consisted
originally of but eight bars, and those bars were repeated for the last four lines
of each stanza, or else the second part differed from my copy.

\textit{Well-a-day} seems to be older than the date of the death of either Earl, because,
in 1566-7, Mr. Wally had a license to print “the second Well-a-day” (\textit{Ex. Reg.
Stat}., i. 151.); and, in 1569-70, Thomas Colwell, to print “A new Well-a-day,
As plain, Mr. Papist, as Dunstable way.”

To “sing well-away” was proverbial even in Chaucer’s time; for in the prologue
to the Wife of Bath’s Tale, speaking of her husbands, she says (lines
5597-600)
\settowidth{\versewidth}{I sette [t]hem so on werke, by my fay!}
\begin{scverse}
\vleftofline{“}I sette [t]hem so on werke, by my fay!\\
That many a night thay \textit{songen weylaway}.\\
The bacoun was nought fet for hem, I trowe,\\
That som men fecche in Essex at Dunmowe.” \dcfootnote{\textit{}
The claiming the Flitch of Bacon at Dunmow was
a custom to which frequent allusions are made in the
fourteenth and fifteenth centuries. See also a song in
\textit{Reliquiæ Antiquæ}, ii.~29.}
\end{scverse}
And in the Shipman’s Tale, “For I may synge allas and waylaway that I was
born.” So in the \textit{Owl and the Nightingale}, one of our earliest original poems, the
owl says to the nightingale—
\settowidth{\versewidth}{Thu singest a night, and noght a dai,}
\begin{scverse}
\vleftofline{“}Thu singest a night, and noght a dai,\\
And al thi song is wail awai.”
\end{scverse}
In the sixteenth century we find a similar passage in Nicholas Breton’s \textit{Farewell
to town}—
\begin{scverse}
\vleftofline{“}I must, ah me! wretch, as I may,\\
Go sing the song of \textit{Welaway}.”
\end{scverse}

The ballads sung to one or other of these tunes are very numerous. Among
them are—

“Sir Walter Rauleigh his Lamentation,” \&c., “to the tune of \textit{Well-a-day}.
Pepys Collection, i. 111, \textsc{b. l.}

“The arraignment of the Devil for stealing away President Bradshaw.” Tune,
\textit{Well-a-day, well-a-day}. (King’s Pamphlets, vol. 15, or Wright’s \textit{Political
Ballads}, 139.)
\pagebreak
%176

“The story of {\blackletter M} May-day, \&c., and how Queen Catherine begged the lives of
2,000 London apprentices,” Tune \textit{Essex's Good-night}. (\textit{Crown Garland of
Golden Roses}, or Evans, iii. 76.)

“The doleful death of Queen Jane, wife of Henry VIII.,” \&c. “Tune,
\textit{The Lamentation of the Lord of Essex}.” (\textit{Crown Garland}, or Evans, iii. 92.)

A Carol, to the tune of Essex’s last Good-night, dated 1661. (Wright’s
Carols.)—
\settowidth{\versewidth}{All you that in this house be here,}
\begin{scverse}\begin{altverse}
\vleftofline{“}All you that in this house be here,\\
Remember Christ that for us died;\\
And spend away with modest cheer,\\
In loving sort this Christmas-tide,” \&c.
\end{altverse}
\end{scverse}

Several other tunes were named after the Earl of Essex. In Dr. Camphuysen’s
\textit{Stichtelyche Rymen} (4to., Amsterdam, 1647) is one called \textit{Essex's Galliard}, and
another \textit{Essex's Lamentation}. The last is the same air as \textit{What if a day, or a
month, or a year}.

In \textit{The World's Folly} (\textsc{b.l.}) a widow “would sing \textit{The Lamentation of a Sinner},
to the tune of \textit{Well-a-daye}."
\normalsize % end cramped page

\musicinfo{Slow.}{}

\includemusic{chappellV1083.pdf}

\musictitle{The Fit’s Come On Me Now.}

This song is quoted by Valentine in Beaumont and Fletcher’s \textit{Wit without
money}, act v., sc. 4., where a verse is printed.

One of my friends recollects his nurse singing a ballad with the burden—
\settowidth{\versewidth}{I must and will get married,}
\begin{scverse}
\vleftofline{“}I must and will get married,\\
The fit’s upon me now.”
\end{scverse}
\pagebreak
%177

The tune is from the seventh edition of \textit{The Dancing Master}. In some later,
editions it is called \textit{The Bishop of Chester’s Jig, or The fit’s come on me now}.

\musicinfo{Cheerfully}{}

\includemusic{chappellV1084.pdf}

\musictitle{Mall Sims.}

This favorite old dance tune is in Queen Elizabeth’s Virginal Book; in Morley’s
\textit{Consort Lessons}, 1599 and 1611; in Rossiter’s \textit{Consort Lessons}, 1609; in Vallet’s
\textit{Tablature de Luth, intitulé Le Secret des Muses}, book i., 4to., Amsterdam, 1615,
entitled “Bal Anglois, Mal Simmes;” also in the second book of the same work,
1616; in \textit{Nederlandtsche Gedenck-Clank}, 1626; in Camphuysen’s \textit{Stichtelycke
Rymen}, 1647 (called “The English Echo, or Malsims”); in the Skene MS., \&c.

It is most likely one of the old harpers’ tunes, as it has quite the character of
harp music. In Rossiter’s \textit{Consort Lessons}, 1609, in which the names of the composers
are given to every other air, this is marked \textit{Incertus}: and if unknown
then, it is probably much older than the date of the book.

In \textit{Wit Restor’d}, 1658, is the ballad of “The Miller and the King’s Daughters,”
written by Dr. James Smith, in which this tune is mentioned:
\settowidth{\versewidth}{What did he doe with her two shinnes?}
\begin{scverse}
\vleftofline{“}What did he doe with her two shinnes?\\
Unto the violl they danc’t \textit{Moll Syms}.”
\end{scverse}
\pagebreak
%178

\musicinfo{Pompously}{}

\includemusic{chappellV1085.pdf}

\musictitle{Crimson Velvet.}

This tune is found in one of the Dutch collections, \textit{Friesche Lust-Hof}, by Jan
Jansz Starter, the edition printed at Amsterdam in 1634. It is called “’Twas a
youthful Knight, which loved a galjant Lady,” which is the first line of the
ballad of “Constance of Cleveland: to the tune of \textit{Crimson Velvet}.” The
ballad is in the Roxburghe Collection, iii. 94, and in Collier’s \textit{Roxburghe
Ballads}, p. 163.

The tune of \textit{Crimson Velvet} was, as Mr. Collier remarks, “highly popular in
the reigns of Elizabeth and her successor.” Among the ballads that were sung
to it, are “The lamentable complaint of Queen Mary, for the unkind departure of
King Philip, in whose absence she fell sick and died;” beginning—
\settowidth{\versewidth}{Mary doth complain.}
\begin{scverse}\vleftofline{“}Mary doth complain.\\
Ladies, be you moved\\
With my lamentations\\
And my bitter groans,” \&c.
\end{scverse}
A copy in the \textit{Crown Garland of Golden Roses} (reprint of edit. of 1659, p. 69).

“An excellent ballad of a prince of \pagebreak England’s courtship to the King of
%179
France’s daughter, and how the prince was disasterously slain; and how the
aforesaid princess was afterwards married to a forrester;” commencing—
\settowidth{\versewidth}{When fair France did flourish,”}
\begin{scverse}
\vleftofline{“}In the days of old,\\
When fair France did flourish,” \&c.
\end{scverse}
Copies in the Roxburghe Collection, i. 102, the Bagford, the Pepys, Deloney’s
\textit{Garland of good-will}, and Percy’s \textit{Reliques}, series iii., book 2, 16.

The following is the ballad of “Constance of Cleveland.”

\musicinfo{Slow.}{}

\includemusic{chappellV1086.pdf}

\pagebreak
%180

\settowidth{\versewidth}{Thou shalt me in thy arms enclose;}
\begin{dcverse}
\indentpattern{01020102110110110110}
\begin{patverse}
His fair lady’s words\\
Nothing he regarded;\\
Wantonness affords,\\
To some, delightful sport;\\
While they dance and sing,\\
With great mirth prepared,\\
She her hands did wring\\
In most grievous sort.\\
Oh! what hap had I,\\
Thus to wail and cry,\\
Unrespected every day.\\
Living in disdain,\\
While that others gain\\
All the right I should enjoy!\\
I am left forsaken,\\
Others they are taken;\\
Ah! my love why dost thou so?\\
Her flatteries believe not,\\
Come to me and grieve not;\\
Wantons will thee overthrow.
\end{patverse}

\begin{patverse}
The knight, with his fair piece,\\
At length the lady spied,\\
Who did him daily fleece\\
Of his wealth and store;\\
Secretly she stood,\\
While she her fashions tryed\\
With a patient mind;\\
While deep the strumpet swore:\\
\vleftofline{“}O sir knight,” quoth she,\\
\vleftofline{“}So dearly I love thee,\\
My life doth rest at thy dispose.\\
By day, and eke by night,\\
For thy sweet delight\\
Thou shalt me in thy arms enclose;\\
I am thine for ever,\\
Still I will persever,\\
True to thee where’er I go.”\\
Her flatteries believe not,\\
Come to me and grieve not;\\
Wantons will thee overthrow.
\end{patverse}

\begin{patverse}
The virtuous lady mild\\
Enters then among them,\\
Being big with child\\
As ever she might be;\\
With distilling tears\\
She looked then upon them,\\
Filled full of fears,\\
Thus replyed she:\\
\vleftofline{“}Ah, my love and dear,\\
Wherefore stay you here,\\
Refusing me, your loving wife,\\
For an harlot’s sake,\\
Which each one will take;\\
Whose vile deeds provoke much strife.\\
Many can accuse her,\\
O, my love, refuse her,\\
With thy lady home return;\\
Her flatteries believe not,\\
Come to me and grieve not;\\
Wantons will thee overthrow.”
\end{patverse}

\begin{patverse}
All in a fury then\\
The angry knight upstarted,\\
Very furious when\\
He heard his lady’s speech;\\
With many bitter terms\\
His wife he ever thwarted,\\
Using hard extremes\\
While she did him beseech.\\
From her neck so white\\
He took away in spite\\
Her curious chain of purest gold:\\
Her jewels and her rings,\\
And all such costly things,\\
As he about her did behold;\\
The harlot, in her presence,\\
He did gently reverence,\\
And to her he gave them all.\\
He sent away his lady,\\
Full of woe as may be,\\
Who in a swound with grief did fall.
\end{patverse}

\begin{patverse}
At the lady’s wrong\\
The harlot fleer’d and laughed;\\
Enticements are so strong,\\
They overcome the wise:\\
The knight nothing regarded\\
To see the lady scoffed;\\
Thus she was rewarded\\
For her enterprise.\\
The harlot all this space\\
Did him oft embrace;\\
She flatters him, and thus doth say:\\
\vleftofline{“}For thee I’ll die and live,\\
For thee my faith I’ll give,\\
No woe shall work my love’s decay;\\
Thou shalt be my treasure,\\
Thou shalt be my pleasure,\\
Thou shalt be my heart’s delight;\\
I will be thy darling,\\
I will be thy worldling,\\
In despite of fortune’s spite.”
\end{patverse}
\end{dcverse}

\pagebreak
%181

\settowidth{\versewidth}{He bade from thence they should her take.}
\indentpattern{01010101220220220220}
\begin{dcverse}\begin{patverse}
Thus did he remain\\
In wasteful great expences,\\
Till it bred his pain,\\
And consum’d him quite.\\
When his lands were spent,\\
Troubled in his senses,\\
Then he did repent\\
Of his late lewd life;\\
For relief he hies,\\
For relief he flies\\
To them on whom he spent his gold;\\
They do him deny,\\
They do him defy,\\
They will not once his face behold.\\
Being thus distressed,\\
Being thus oppressed,\\
In the fields that night he lay;\\
Which the harlot knowing,\\
Through her malice growing,\\
Sought to take his life away.
\end{patverse}

\indentpattern{01020102110110110110}
\begin{patverse}
A young and proper lad\\
They had slain in secret\\
For the gold he had;\\
Whom they did convey,\\
By a ruffian lewd,\\
To that place directly,\\
Where the youthful knight\\
Fast a sleeping lay;\\
The bloody dagger, then,\\
Wherewith they kill’d the man,\\
Hard by the knight he likewise laid;\\
Sprinkling him with blood,\\
As he thought it good,\\
And then no longer there he stay’d.\\
The knight, being so abused,\\
Was forthwith accused\\
For this murder which was done;\\
And he was condemned\\
That had not offended,\\
Shameful death he might not shun.
\end{patverse}

\begin{patverse}
When the lady bright\\
Understood the matter,\\
That her wedded knight\\
Was condemned to die,\\
To the king she went\\
With all the speed that might be.\\
Where she did lament\\
Her hard destiny.\\
“Noble king,” quoth she,\\
“Pity take on me,\\
And pardon my poor husband’s life;\\
Else I am undone,\\
With my little son,\\
Let mercy mitigate this grief.”\\
\vleftofline{“}Lady fair, content thee,\\
Soon thou wouldst repent thee\\
If he should be saved so;\\
Sore he hath abus’d thee,\\
Sore he hath misus’d thee,\\
Therefore, lady, let him go.”
\end{patverse}

\begin{patverse}
\vleftofline{“}O, my liege,” quoth she,\\
\vleftofline{“}Grant your gracious favour;\\
Dear he is to me,\\
Though he did me wrong.”\\
The king replied again,\\
With a stern behaviour,\\
“A subject he hath slain,\\
Die, he shall, ere long:\\
Except thou canst find\\
Any one so kind\\
That will die and set him free.”\\
\vleftofline{“}Noble king,” she said,\\
\vleftofline{“}Glad am I apaid,\\
That same person will I be.\\
I will suffer duly,\\
I will suffer truly,\\
For my love and husband’s sake.”\\
The king thereat amazed,\\
Though he her beauty praised, \\
He bade from thence they should her take.
\end{patverse}

\begin{patverse}
It was the king’s command,\\
On the morrow after,\\
She should out of hand\\
To the scaffold go;\\
Her husband was\\
To bear the sword before her;\\
He must, eke alas!\\
Give the deadly blow.\\
He refus’d the deed,\\
She bade him to proceed\\
With a thousand kisses sweet.\\
In this woeful case\\
They did both embrace;\\
Which mov’d the ruffians in that place\\
Straight for to discover\\
This concealed murder;\\
Whereby the lady saved was.\\
The harlot then was hanged,\\
As she well deserved:\\
This did virtue bring to pass.
\end{patverse}

\end{dcverse}

\pagebreak
%182

\musictitle{Walking In A Country Town.}

The tune from Robinson’s \textit{Schoole of Musicke}, 1603, called \textit{Walking in a
country town}. In the Roxburghe Collection, i. 412, is a ballad beginning
“Walking in a meadow green,” and, from the similarity of the lines, and the
measure of the verse so exactly suiting the air, I infer this to be the tune of both.
The latter was printed by John Trundle, at the sign of the Nobody in Barbican,
rendered famous by Ben Jonson, who in his \textit{Every man in his Humour}, makes
Knowell say, “Well, if he read this with patience, I’ll ‘go,’ and troll ballads for
Master John Trundle yonder, the rest of my mortality.”

It is entitled “The two Leicestershire Lovers: to the tune of \textit{And yet methinks
I love thee}.” The first stanza is here printed to the music.

The last line of the verse is, “Upon the meadow brow,” and \textit{The meadow brow}
is often quoted as a tune. So in the Roxburghe Collection, i. 92, or Colliers’s
Roxburghe Ballads, p. 1, is “Death’s Dance” (beginning, “If Death would come
and shew his face”), “to be sung to a pleasant new tune called \textit{O no, no, no, not
yet}, or \textit{The meadow brow}.” And Bishop Corbet’s song, “Farewell, rewards and
fairies,” is “to be sung or whistled to the tune \textit{The meddow brow} by the learned:
by the unlearned, to the tune of \textit{Fortune}.”—(Percy, series iii., book 2.) All
might be sung to this tune.

\musicinfo{Slow.}{}

\includemusic{chappellV1087.pdf}

\vspace{-2\baselineskip}

\musictitle{Phillida Flouts Me.}

In \textit{The Crown Garland of Golden Roses}, 1612, is “A short and sweet sonnet
made by one of the Maides of Honor upon the death of Queene Elizabeth, which
she sowed upon a sampler, in red silke: to a new tune, or \textit{Phillida flouts me};”
beginning—
\settowidth{\versewidth}{Whom we have lov’d so dear,” }
\begin{scverse}
\vleftofline{“}Gone is Elizabeth,\\
Whom we have lov’d so dear,” \&c.
\end{scverse}

\pagebreak
%183

Patrick Carey also wrote a ballad to the tune of \textit{Phillida flouts me}; beginning—
\settowidth{\versewidth}{Ned! she that likes thee now,}
\begin{scverse}
\vleftofline{“}Ned! she that likes thee now,\\
Next week will leave thee!”
\end{scverse}
It is contained in his “Trivial Poems and Triolets, written in obedience to
Mrs. Tomkin’s commands, 20th August, 1651.” In Walton’s \textit{Angler}, 1653, the
Milkwoman asks, “What song was it, I pray? Was it \textit{Come, shepherds, deck
your heads}, or \textit{As at noon Dulcina rested}, or \textit{Phillida flouts me}?”

The ballad of \textit{Phillida flouts me} is in the Roxburghe Collection, ii. 142, and in
the same volume, p. 24, “The Bashful Virgin, or The Secret Lover: tune of
\textit{I am so deep in love}, or \textit{Little boy}, \&c.” It begins—
\settowidth{\versewidth}{O what a plague it is}
\begin{scverse}
\begin{altverse}
\vleftofline{“}O what a plague it is\\
To be a lover;\\
Being denied the bliss\\
For to discover,” \&c.
\end{altverse}
\end{scverse}

This appears to be also to the air of \textit{Phillida flouts me}, although the first line of
that ballad is “Oh! what a plague is love,” not “I am so deep in love.”

The words and music are in Watts’ \textit{Musical Miscellany}, ii. 132 (1729), and an
answer, beginning, “O where’s the plague in love.” The tune is also in many of
the ballad-operas, such as \textit{The Quaker’s Opera}, 1728; \textit{Love in a Riddle}, 1729;
\textit{Damon and Phillida}, 1734, \&c.

Ritson printed the words in his \textit{Ancient Songs}, from a copy in \textit{The Theatre of
Compliments, or New Academy}, 1689, but did not discover the tune.

\musicinfo{Slowly and gracefully.}{}

\includemusic{chappellV1088.pdf}

\pagebreak
184

\settowidth{\versewidth}{Thou shalt eat curds ond cream}
\indentpattern{010101010003}
\begin{dcverse}\begin{patverse}
At the fair t’other day,\\
As she pass’d by me,\\
She look’d another way,\\
And would not spy me.\\
I woo’d her for to dine,\\
But could not get her;\\
Dick had her to the Vine,\\
He might intreat her.\\
With Daniel she did dance,\\
On me she would not glance;\\
Oh, thrice unhappy chance!\\
Phillida flouts me.
\end{patverse}

\begin{patverse}
Fair maid, be not so coy,\\
Do not disdain me;\\
I am my mother’s joy;\\
Sweet, entertain me.\\
I shall have, when she dies,\\
All things that’s fitting;\\
Her poultry and her bees,\\
And her goose sitting;\\
A pair of mattrass beds,\\
A barrel full of shreds:\\
And yet, for all these goods,\\
Phillida flouts me.
\end{patverse}

\begin{patverse}
I often heard her say,\\
That she lov’d posies;\\
In the last month of May\\
I gave her roses,\\
Cowslips and gilly-flowers,\\
And the sweet lily,\\
I got to deck the bow’rs\\
Of my dear Philly.\\
She did them all disdain,\\
And threw them back again;\\
Therefore ’tis flat and plain\\
Phillida flouts me.
\end{patverse}

\begin{patverse}
Thou shalt eat curds ond cream\\
All the year lasting,\\
And drink the crystal stream,\\
Pleasant in tasting:\\
Swig whey until you burst,\\
Eat bramble-berries,\\
Pye-lid, and pastry-crust,\\
Pears, plums, and cherries;\\
Thy garments shall he thin,\\
Made of a wether’s skin;\\
Yet all’s not worth a pin:\\
Phillida flouts me.
\end{patverse}

\begin{patverse}
Which way soe’er I go,\\
She still torments me;\\
And, whatsoe’er I do,\\
Nothing contents me:\\
I fade, and pine away\\
With grief and sorrow;\\
I fall quite to decay,\\
Like any shadow;\\
I shall be dead, I fear,\\
Within a thousand year,\\
And all because my dear\\
Phillida flouts me.
\end{patverse}

\begin{patverse}
Fair maiden, have a care,\\
And in time take me;\\
I can have those as fair,\\
If you forsake me;\\
There’s Doll, the dairy-maid,\\
Smil’d on me lately,\\
And wanton Winifred\\
Favours me greatly;\\
One throws milk on my clothes,\\
T’other plays with my nose;\\
What pretty toys are those!\\
Phillida flouts me.
\end{patverse}

\begin{patverse}
She has a cloth of mine,\\
Wrought with blue Coventry,\\
Which she keeps as a sign\\
Of my fidelity:\\
But if she frowns on me,\\
She shall ne’er wear it;\\
I’ll give it my maid Joan,\\
And she shall tear it.\\
Since ’twill no better be,\\
I’ll bear it patiently;\\
Yet, all the world may see,\\
Phillida flouts me.
\end{patverse}
\end{dcverse}

\musictitle{Lady, Lie Near Me.}

This ballad is entitled “The longing Shepherdess, or Lady” [Laddy] “lie
near me.” Copies are in the Pepys Collection, iii., 59, and Douce, p. 119, \&c.
It is also in the list of ballads that were printed by W. Thackeray, at the Angel,
in Duck Lane.

The tune (which bears a strong resemblance to \textit{Phillida flouts me}) is in \textit{The
Dancing Master}, from the first edition in 1650, to the eighth in 1690.

\pagebreak
%185

In Ritson’s \textit{North Country Chorister} there is another ballad, called “Laddy, lie
near me” (beginning, “As I walked! over hills, dales, and high mountains”); and
in 1793 Mr. George Thomson gave Burns a tune of that name, to write words to,
which is now included in Scotch Collections. It differs wholly from this.

\musicinfo{Slowly and gracefully.}{}

\includemusic{chappellV1089.pdf}

\musictitle{Mill-Field.}

In the collection of ballads and proclamations in the library of the Society of
Antiquaries is one by W. Elderton, entitled “A new ballad, declaring the great
treason conspired against the young King of Scots, and how one Andrew Browne,
an Englishman, which was the King’s Chamberlaine, prevented the same. To the
tune of \textit{Milfield}, or els to \textit{Greene sleeves}.” It was printed by “Yarathe James,”
to whom it was licensed on 30th May, 1581.

The tune is in \textit{The Dancing Master} from 1650 to 1658. The ballad in Percy’s
\textit{Reliques}, series ii., book 2, No. 16. The first stanza is here with the music.

\pagebreak
%186

\musicinfo{Gracefully}{}

\includemusic{chappellV1090.pdf}

\vspace{-2\baselineskip}

\musictitle{The Spanish Lady.}

Dr. Percy says, “this beautiful old ballad most probably took its rise from one
of those descents made on the Spanish coasts in the time of Queen Elizabeth:
and, in all likelihood, from the taking of the city of Cadiz (called by our sailors,
corruptly, Cales), on June 21, 1596, under the command of the Lord Howard,
admiral, and of the Earl of Essex, general.”

The question as to who was the favored lover, has been fully discussed; it may
therefore be sufficient here to refer the reader to \textit{The Edinburgh Review} for April,
1846; \textit{The Times} newspapers of April 30, and May 1, 1846; and \textit{The Quarterly
Review} for October, 1846.

The ballad is quoted in \textit{Cupid’s Whirligig}, 1616, and parodied in Rowley’s
\textit{A Match at Midnight}, 1633. In the Douce Collection, ii. 210 and 212, there
are two copies, the one “to a pleasant new tune;” the other (which is of later
date) to the tune of \textit{Flying Fame}; but \textit{could no}t he sung to that air. In the
same volume, p. 254, is “The Westminster Wedding, or Carlton’s Epithalamium,”
(dated 1663): to the tune of \textit{The Spanish Lady}. It commences thus;

\settowidth{\versewidth}{Will you hear a German Princess,}
\begin{scverse}
\vleftofline{“}Will you hear a German Princess,\\
How she chous’d an English Lord,” \&c.
\end{scverse}

\pagebreak
%187

The tune is contained in the Skene MS., and in several of the ballad-operas,
such as \textit{The Quaker’s Opera}, 1728; \textit{The Jovial Crew}, 1731, \&c.

The words are found in \textit{The Garland of Good-will}, and in several of the celebrated
collections of ballads; also in Percy’s \textit{Reliques}, series ii., book 2.

\musicinfo{Slow.}{}

\includemusic{chappellV1091.pdf}

\musictitle{The Jovial Tinker, or Joan’s Ale Is New.}

On the 26th Oct., 1594, John Danter entered on the books of the Stationers’
Company, “for his copie, a ballet intituled Jone’s ale is newe;” and on the
15th Nov., of the same year, Edward White one called “The unthrifte’s adieu
to Jone’s ale is newe.”

In Ben Jonson’s \textit{Tale of a tub}, “old father Rosin, chief minstrel of Highgate,
and his two boys” play the dances called for by the company, which are “\textit{Tom
Tiler}; \textit{The jolly Joiner}; and \textit{The jovial Tinker}.” The burden of the song called
“The jovial Tinker” is “Joan’s ale is new.” ( “Tom Tiler” is one of the
country dances mentioned in Heywood’s \textit{A woman kill’d with kindness}.) In the
\textit{Mad Pranks and merry Jests of Robin Goodfellow}, 1628, there is a song to the
tune of \textit{The jovial Tinker}, which has a burden or chorus of four lines, unsuited to
this air, although the song itself could be sung to it. As tinkers were so famous
in song, there was probably another tune called \textit{The jovial Tinker}. “He that a
tinker, a tinker will be,” is one of the catches in the \textit{Antidote to Melancholy}, 1661;
“Tom Tinker lives a merry life,” is in Davenant’s play, \textit{The Benefice}; “Have
you any work for a tinker,” in \textit{Wit and Drollery}, 1661; and Ben Jonson says,
in \textit{Paris’ Anniversary}, “Here comes the tinker I told you of, with his kettledrum
before and after, a \textit{master of music}.”

\pagebreak
%188



The song of Joan’s ale is new is in the Douce Collection, p. 110. It is in the
list of those printed by W. Thackeray, at the Angel in Duck Lane, in the reign
of Charles II.; and is in both editions of \textit{Pills to purge Melancholy}, with the
tune.—(Ed. of 1707, iii. 133; or ed. of 1719, v. 61.)

The copy in the Douce Collection consists of thirteen stanzas, and has the
following lengthy title: “Joan’s ale is new; or a new merry medley, shewing
the power, the strength, the operation, and the virtue that remains in good ale,
which is accounted the mother-drink of England.”
\settowidth{\versewidth}{All you that do this merry ditty view}
\begin{scverse}
\vleftofline{“}All you that do this merry ditty view,\\
Taste of Joan’s ale, for it is strong and new, \&c.”
\end{scverse}

“To a pleasant new Northern tune.”

\musicinfo{Cheerfully.}{}

\includemusic{chappellV1092.pdf}

\pagebreak
%189

\settowidth{\versewidth}{Where they drank soundly for a space}
\indentpattern{00010001}
\begin{dcverse}\begin{patverse}
The tinker he did settle\\
Most like a man of mettle,\\
And vow’d to pawn his kettle;\\
Now mark what did ensue:\\
His neighbours they flock in apace,\\
To see Tom Tinker’s comely face,\\
Where they drank soundly for a space,\\
Whilst Joan’s ale, \&c.
\end{patverse}

\settowidth{\versewidth}{And said they would drink for boon, man,}
\begin{patverse}
The cobbler and the broom-man\\
Came up into the room, man,\\
And said they would drink for boon, man,\\
Let each one take his due!\\
But when the liquor good they found,\\
They cast their caps upon the ground,\\
And so the tinker he drank round,\\
Whilst Joan’s ale, \&c.
\end{patverse}
\end{dcverse}

In another volume in the Douce Collection, p. 180, is an answer to the
above, to the same tune. It is the “The poet’s new year’s gift; or a pleasant
poem in praise of sack: setting forth its admirable virtues and qualities, and how
much it is to be preferred before all other sorts of liquors, \&c. To the tune of
\textit{The jovial Tinker}, or \textit{Tom a Bedlam};” commencing—
\settowidth{\versewidth}{\vin Doth far exceed your fountain,” \&c.}
\begin{scverse}\begin{altverse}
\vleftofline{“}Come hither, learned sisters,\\
And leave Parnassus mountain;\\
I will you tell where is a well\\
Doth far exceed your fountain,” \&c.
\end{altverse}
\end{scverse}

\musictitle{Under And Over.}

This is the same air as the preceding, but in a minor instead of a major key.
It is in every edition of \textit{The Dancing Master}, under the name of \textit{Under and over};
but in a MS. volume of virginal music, formerly in the possession of Mr. Windsor,
of Bath, it is entitled \textit{A man had three sons}.

The ballad of \textit{Under and over} is in the Pepys Collection, i. 264, \textsc{b.l.}, as “A new
little Northern Song, called—
\settowidth{\versewidth}{Under and over, over and under,}
\begin{scverse}\vleftofline{“}Under and over, over and under,\\
Or a pretty new jest and yet no wonder;\\
\vleftofline{“}Or a maiden mistaken, as many now be,\\
View well this glass, and you may plainly see.”
\end{scverse}
“To a pretty new Northern tune.”

It is very long, full of typographical errors, and devoid of merit; I have
therefore only printed the first verse with the music.

In the same volume are the following: “Rocke the babie, Joane: to the tune
of \textit{Under and over},” p. 396; beginning—
\settowidth{\versewidth}{A young man in our parish,}
\begin{scverse}\vleftofline{“}A young man in our parish,\\
His wife was somewhat currish,” \&c.
\end{scverse}
And at p. 404, another, commencing—
\begin{scverse}\vleftofline{“}There was a country gallant,\\
That wasted had his talent,” \&c.
\end{scverse}
In the Roxburghe, iii. 176, “Rock the cradle, John:
\settowidth{\versewidth}{Let no man at this strange story wonder,}
\begin{scverse}Let no man at this strange story wonder,\\
It goes to the tune of \textit{Over and under}.”
\end{scverse}

And in the same Collection, i. 411, “The Times’ Abuses; to the tune of \textit{Over and
under}; commencing—
\begin{scverse}\vleftofline{“}Attend, my masters, and give ear,” \&c.
\end{scverse}
The last is also printed in Collier’s \textit{Roxburghe Ballads}, p. 281.

\pagebreak
%190


\musicinfo{Cheerfully.}{}

\includemusic{chappellV1093.pdf}

\vspace{-\baselineskip}

\musictitle{The Oxfordshire Tragedy.}

This is one of the old and simple chaunt-like ditties, which seem to have been
peculiarly suited to the lengthy narratives of the minstrels; and I am strongly
impressed with a belief that it was one of their tunes. It has very much the same
character as \textit{Sir Guy}, which I met with in another of the ballad operas, and
which—the entry at Stationers’ Hall proving to be earlier than 1592—may be
fairly supposed to he the air used, by the class of minstrel described by Puttenham,
in singing the adventures of Sir Guy, at feasts. See page 172.

I have seen no earlier copy of \textit{The Oxfordshire Tragedy}, than an edition
“printed and sold in Bow Church-Yard,” in which the name of the tune is not
mentioned. The ballad is in four parts, the third and fourth of which, being in
a different metre, must have been sung to another air.

“As I walk’d forth to take the air,” is the second line of the first part,
and a tune is often referred to under that title. As the measures agree, it may
be a second name for this air.

In the Douce Collection, \pagebreak 44, is a black-letter ballad of “Cupid’s Conquest, or
%191
Will the Shepherd and fair Kate of the Green, both united together in pure love:
to the tune, \textit{As I went forth to take the air};” commencing,—
\settowidth{\versewidth}{Now am I tost on waves of love;}
\begin{scverse}\vleftofline{“}Now am I tost on waves of love;\\
\vin Here like a ship that’s under sail,” \&c.
\end{scverse}
and in the Roxburgh ii. 149, “The faithful lovers of the West: tune, \textit{As I walkt
forth to take the air}.”

In Mr. Payne Collier’s Collection, is “The unfortunate Sailor’s Garland, with
an account how his parents murdered him for love of his gold.” It is in two
parts, and both to the tune of \textit{The Oxfordshire Tragedy}, After four lines of
exordium, it begins thus:—

\backskip{1}

\settowidth{\versewidth}{“Near Bristol liv’d a man of fame,}
\begin{scverse}\vleftofline{“}Near Bristol liv’d a man of fame,\\
But I’ll forbear to tell his name;\\
He had one son and daughter bright,\\
In whom he took a great delight,” \&c.
\end{scverse}

Another Garland, called “The cruel parents, or the two faithful lovers,” is to
the tune of \textit{The Oxfordshire Lady}, and in the same metre.

The tune of \textit{The Oxfordshire Tragedy} is in \textit{The Cobblers’ Opera}, 1729, \textit{The
Village Opera}, 1729, and \textit{Sylvia}, or \textit{The Country Burial}, 1731.

\musicinfo{Slow.}{}

\includemusic{chappellV1094.pdf}

\vspace{-2\baselineskip}

\settowidth{\versewidth}{Where a fair lady made great moan,}
\begin{dcverse}\footnotesize
Down by a crystal river side,\\
A gallant bower I espied,\\
Where a fair lady made great moan,\\
With many a bitter sigh and groan.

Alas! quoth she, my love's unkind,\\
My sighs and tears he will not mind;\\
But he is cruel unto me,\\
Which causes all my misery.

My father is a worthy knight,\\
My mother is a lady bright,\\
And I their only child and heir;\\
Yet love has brought me to despair.

A wealthy squire lived nigh,\\
Who on my beauty cast an eye;\\
He courted me, both day and night,\\
To be his jewel and delight.

To me these words he often said:\\
Fair, beauteous, handsome, comely maid,\\
Oh! pity me, I do implore,\\
For it is you I do adore.

He still did beg me to be kind,\\
And ease his love-tormented mind;\\
For if, said he, you should deny,\\
For love of you I soon shall die.

These words did pierce my tender heart,\\
I soon did yield, to ease his smart;\\
And unto him made this reply,—\\
For love of me you shall not die.

With that he flew into my arms,\\
And swore I had a thousand charms;\\
He call’d me angel, saint, and he\\
Did swear, for ever true to be.
\end{dcverse}

\pagebreak
%192

\begin{dcverse}\footnotesizerr
Soon after he had gain’d my heart,\\
He cruelly did from me part;\\
Another maid he does pursue,\\
And to his vows he bids adieu.

Tis he that makes my heart lament,\\
He causes all my discontent;\\
He hath caus’d my sad despair,\\
And now occasions this my care.

The lady round the meadow run,\\
And gather’d flowers as they sprung;\\
Of every sort she there did pull.\\
Until she got her apron full.

Now, there’s a flower, she did say,\\
Is named heart’s-ease; night and day,\\
I wish I could that flower find,\\
For to ease my love-sick mind.

But oh! alas! ’tis all in vain\\
For me to sigh, and to complain;\\
There’s nothing that can ease my smart,\\
For his disdain will break my heart.

The green ground served as a bed,\\
And flow’rs a pillow for her head;\\
She laid her down and nothing spoke,\\
Alas! for love her heart was broke.

But when I found her body cold,\\
I went to her false love, and told\\
What unto her had just befel;\\
I’m glad, said he, she is so well.

Did she think I so fond could be,\\
That I could fancy none but she?\\
Man was not made for one alone;\\
I take delight to hear her moan.

Oh! wicked man I find thou art,\\
Thus to break a lady's heart;\\
In Abraham’s bosom may she sleep,\\
While thy w'icked soul doth weep!

\vin\vin\vin\textsc{the answer.}

A second part, I bring you here,\\
Of the fair maid of Oxfordshire,\\
Who lately broke her heart for love\\
Of one, that did inconstant prove.

A youthful squire, most unjust,\\
When he beheld this lass at first,\\
A thousand solemn vows he made,\\
And so her yielding heart betray’d.

She mourning, broke her heart, and died,\\
Feeling the shades on every side;\\
With dying groans and grievous cries,\\
As tears were flowing through her eyes.

The beauty which did once appear,\\
On her sweet cheeks, so fair and clear,\\
Was waxed pale,—her life was fled;\\
He heard, at length, that she was dead.

He was not sorry in the least,\\
But cheerfully resolv’d to feast;\\
And quite forgot her beauty bright,\\
Whom he so basely ruin’d quite.

Now, when, alas! this youthful maid,\\
Within her silent tomb was laid,\\
The squire thought that all was well,\\
He should in peace and quiet dwell.

Soon after this he was possest\\
With various thoughts, that broke his rest;\\
Sometimes he thought her groans he heard,\\
Sometimes her ghastly ghost appear’d

With a sad visage, pale and grim,'\\
And ghastly looks she cast on him;\\
He often started back and cried,\\
Where shall I go myself to hide?

Here I am haunted, night and day,\\
Sometimes methinks I hear her say,\\
Perfidous man! false and unkind,\\
Henceforth you shall no comfort find.

If through the fields I chance to go,\\
Where she receiv’d her overthrow,\\
Methinks I see her in despair;\\
And, if at home, I meet her there.

No place is free of torment now;\\
Alas! I broke a solemn vow\\
Which once I made; but now, at last,\\
It does my worldly glory blast.

Since my unkindness did destroy\\
My dearest love and only joy,\\
My wretched life must ended be,\\
Now must I die and come to thee.

His rapier from his side he drew,\\
And pierced his body thro’ and thro’;\\
So he dropt down in purple gore\\
Just where she did some time before.

He buried was within the grave\\
Of his true love. And thus you have\\
A sad account of his hard fate,\\
Who died in Oxfordshire of late.
\end{dcverse}

The third and fourth parts present a similar story, in different metre; but
it is the lady’s cruelty which causes the first suicide.

\pagebreak
%193

\musictitle{Put On Thy Smock On Monday.}

This is mentioned as a country dance tune in Heywood’s \textit{A Woman kill’d with
Kindness}, act i., sc. 2; and alluded to in Fletcher’s \textit{Love’s Cure}, act ii., sc. 2.
It is contained in the fourth, fifth, and later editions of \textit{The Dancing Master}.

\musicinfo{Moderate time.}{}

\includemusic{chappellV1094.pdf}

\musictitle{Drive The Cold Winter Away.}

This is the burden of a song in praise of Christmas, copies of which are in the
Pepys (i. 186) and Roxburghe (i. 24) Collections. It is entitled “A pleasant
countrey new ditty: merrily shewing how to drive the cold winter away. To
the tune of \textit{When Phoebus did rest},”\dcfootnote{\textit{}
A song beginning “When Phoebus \textit{addrest} his course
to the West,” will be found in \textit{Merry Drollery Complete},
Part ii., 1661; also in \textit{Wit and Drollery, Jovial Poems}.
The burden is, “O do not, do not kill me yet, for I am
not prepared to die.” By that name it is quoted in J.
Starter’s \textit{Boertigheden}, quarto, Amsterdam, 1634, whete
the tune is also printed.}
\&c.; black-letter, printed by H[enry]
G[osson]. It is one of those parodied in Andro Hart’s \textit{Compendium of Godly
Songs}.
\settowidth{\versewidth}{The wind blawis cald, furious and bald,}
\begin{scverse}
\begin{altverse}
\vleftofline{“}The wind blawis cald, furious and bald,\\
This lang and mony a day;\\
But, Christ’s mercy, we mon all die,\\
Or keep the cald wind away.\\
This wind sa keine, that I of meine,\\
It is the vyce of auld;\\
Our faith is inclusit, and plainely abusit,\\
This wind he’s blawin too cald,” \&c.\\
\vin\vin\vin\textit{Scottish Poems of 16th Century}, ii. 177, 8vo., 1801.
\end{altverse}
\end{scverse}

The tune is in every edition of \textit{The Dancing Master}; in \textit{Musick’s Delight on
the Cithren}, 1666; and in Walsh’s \textit{Dancing Master}: also in both editions of
\textit{Pills to purge Melancholy}, with an abbreviated copy of the words.

In the Roxburghe Collection, i. 518, is a ballad entitled “Hang pinching; or
The good fellow’s observation ’mongst a jovial crew, of them that hate flinching, 
but are always true blue. To the tune of \textit{Drive the cold winter away};”
commencing—
\settowidth{\versewidth}{All you that lay claim to a good fellow’s name,}
\begin{scverse}
\begin{altverse}
\vleftofline{“}All you that lay claim to a good fellow’s name,\\
And yet do not prove yourselves so,\\
Give ear to this thing, the which I will sing,\\
Wherein I most plainly will shew
\end{altverse}
\end{scverse}

\pagebreak
%194

\settowidth{\versewidth}{With proof and good ground, those fellows profound,}
\begin{scverse}
\begin{altverse}
With proof and good ground, those fellows profound,\\
That unto the alewives are true,\\
In drinking their drink, and paying their chink,\\
\textit{O such a good fellow's true blue}.”
\end{altverse}
\end{scverse}
Sometimes a tune named \textit{True blue} is quoted, and perhaps from this ballad. It is
subscribed W. B., and printed for Thomas Lambert, at the sign of the Horse
Shoe, in Smithfield. Lambert was a printer of the reigns of James and Charles I.

In the Pepys Collection, i. 362, is another black-letter ballad, entitled “The
father hath beguil’d the son: Or a wonderful tragedy which lately befell in Wiltshire,
as many men know full well; to the tune of \textit{Drive the cold winter away};”
beginning—
\settowidth{\versewidth}{“I often have known, and experience hath shown,}
\begin{scverse}
\begin{altverse}
\vleftofline{“}I often have known, and experience hath shown,\\
That a spokesman hath wooed for himself,\\
And that one rich neighbour will, underhand, labour\\
To overthrow another with pelf,” \&c.
\end{altverse}
\end{scverse}
Other ballads to the tune will be found in the Roxburghe Collection (i. 150 and
160, \&c.); in the King’s Pamphlets, and the Collection of Songs against the
Rump Parliament; in Wright’s \textit{Political Songs}; in \textit{Mock Songs}, 1675; in Evans’
Collection, i. 349, \&c.

\musicinfo{Bold and not too fast.}{Song in praise of Christmas.}

\smallskip

\includemusic{chappellV1096.pdf}

\pagebreak
%195

\settowidth{\versewidth}{If wrath be to seek, do not lend her thy cheek,zz}
\begin{dcverse}\footnotesizerr
\begin{altverse}
Let Misery pack, with a whip at his back,\\
To the deep Tantalian flood;\\
In Lethe profound, let envy be drown’d,\\
That pines at another man’s good;\\
Let Sorrow’s expanse be banded from hence,\\
All payments of grief delay,\\
And wholly consort with mirth and with sport\\
To drive the cold winter away.
\end{altverse}

\begin{altverse}
{’}Tis ill for a mind to anger inclin{’d}\\
To think of old injuries now;\\
If wrath be to seek, do not lend her thy cheek,\\
Nor let her inhabit thy brow.\\
Cross out of thy books malevolent looks,\\
Both beauty and youth’s decay.\\
And spend the long nights in honest delights,\\
To drive the cold winter away.
\end{altverse}

\begin{altverse}
The court in all state now opens her gate,\\
And bids a free welcome to most;\\
The city likewise, tho’ somewhat precise,\\
Doth willingly part with her cost:\\
And yet by report, from city and court,\\
The country will gain the day;\\
More liquor is spent, and with better content,\\
To drive the cold winter away.
\end{altverse}

\begin{altverse}
Our good gentry there, for cost do not spare,\\
The yeomanry fast not till Lent;\dcfootnote{\textit{}
For the support and encouragement of the fishing
towns, in the time of Elizabeth, Wednesdays and Fridays
were constantly observed as fast days, or days of abstinence
from flesh. This was by the advice of her minister,
Cecil; and by the vulgar it was generally called Cecil’s
Fast. See Warburton's and Blakeway’s notes in Boswell’s
edition of Shakespeare, x. 49 and~50.}\\
The farmers, and such, think nothing too much,\\
If they keep but to pay for their rent.\\
The poorest of all do merrily call,\\
When at a fit place they can stay,\\
For a song or a tale, or a pot of good ale,\\
To drive the cold winter away.
\end{altverse}

\begin{altverse}
Thus none will allow of solitude now,\\
But merrily greets the time,\\
To make it appear, of all the whole year,\\
That this is accounted the prime:\\
December is seen apparel’d in green,\\
And January, fresh as May,\\
Comes dancing along, with a cup and a song,\\
To drive the cold winter away.
\end{altverse}

\begin{altverse}
\vin\vin\vin\textsc{the second part.}
\end{altverse}

\begin{altverse}
This time of the year is spent in good cheer,\\
And neighbours together do meet,\\
To sit by the fire, with friendly desire,\\
Each other in love to greet;\\
Old grudges forgot, are put in the pot,\\
All sorrows aside they lay,\\
The old and the young doth carol his song,\\
To drive the cold winter away.
\end{altverse}

\begin{altverse}
Sisley and Nanny, more jocund than any,\\
As blithe as the month of June,\\
Do carol and sing, like birds of the Spring,\\
(No nightingale sweeter in tune)\\
To bring in content, when summer is spent,\\
In pleasant delight and play,\\
With mirth and good cheer, to end the old year,\\
And drive the cold winter away.
\end{altverse}

\begin{altverse}
The shepherd and swain do highly disdain\\
To waste out their time in care,\\
And Clim of the Clough\dcfootnote{\textit{}
\textit{Clim of the Clough} means Clement of the Cleft. The
name is derived from a noted archer, once famous in the
north of England. See the old ballad, \textit{Adam Bell, Clim of
the Clough, and William of Cloudesly}, printed by Bp. Percy.
A \textit{Clough} is a sloping valley, breach, or \textit{Cleft}, from the
side of a hill, where trees or furze usually grow.}
hath plenty enough\\
If he but a penny can spare,\\
To spend at the night in joy and delight,\\
Now after his labours all day,\\
For better than lands is the help of his hands,\\
To drive the cold winter away.
\end{altverse}

\begin{altverse}
To mask and to mum kind neighbours will come\\
With wassails of nut-brown ale,\\
To drink and carouse to all in the house,\\
As merry as bucks in the dale;\\
Where cake, bread and cheese, Is brought for your fees,\\
To make you the longer stay;\\
At the fire to warm will do you no harm,\\
To drive the cold winter away.
\end{altverse}

\begin{altverse}
When Christmas’s tide comes in like a bride,\\
With holly and ivy clad,\\
Twelve days in the year, much mirth and good cheer, \\
In every household is had;\\
The country guise is then to devise\\
Some gambols of Christmas play,\\
Whereat the young men do best that they can,\\
To drive the cold winter away.
\end{altverse}

\begin{altverse}
When white-bearded frost hath threatened his worst\\
And fallen from branch and brier,,\\
Then time away calls, from husbandry halls\\
And from the good countryman’s fire,\\
Together to go to plough and to sow,\\
To get us both food and array;\\
And thus with content the time we have spent\\
To drive the cold winter away.
\end{altverse}
\end{dcverse}

\pagebreak
%196

\musictitle{Up, Tails All.}

This tune is in Queen Elizabeth’s Virginal Book, and in \textit{The Dancing Master}
from 1650 to 1690. It is alluded to in Sharpham’s \textit{Fleire}, 1610: “She every
day sings \textit{John for the King}, and at \textit{Up, tails all}, she’s perfect.” Also in Ben
Jonson’s \textit{Every man out of his humour}; in Beaumont and Fletcher’s \textit{Coxcomb};
Vanbrugh’s \textit{Provoked Wife}, \&c.

There are several political songs of the Cavaliers to this air, in the King’s
Pamphlets (Brit.. Mus.); in the Collection of Songs written against the Rump
Parliament; in \textit{Rats rhimed to Death}, 1660; and one in \textit{Merry Drollery complete},
1670: but party feeling was then so often expressed with more virulence than wit,
that few of them will bear republication. In both the editions of \textit{Pills to purge
Melancholy}, 1707 and 1719, the song of \textit{Up, tails all}, beginning “Fly, merry
news,” is printed by mistake with the title and tune of \textit{The Friar and the Nun}.

\musicinfo{Moderate time and lightly.}{}

\smallskip

\includemusic{chappellV1097.pdf}

\musictitle{Pescod Time.}

The tune of In \textit{Pescod Time} (\ie, peas-cod time, when the field peas are
gathered), was extremely popular towards the end of the sixteenth century. It is
contained in Queen Elizabeth’s and Lady Neville’s Virginal Books; in Anthony
Holborne’s \textit{Citharn Schoole} (1597); and in Sir John Hawkins’ transcripts; but
so disguised by point, augmentation, and other learned contrivances, that it was
only by scanning the whole arrangement (by Orlando Gibbons) that this simple
air could, be extracted. In Queen Elizabeth’s Virginal Book, the same air is
called \textit{The Hunt's up,} in another part of the book.

The words are in \textit{England's Helicon}, 1600 (or reprint in 1812, p. 206); in
Miss Cooper’s \textit{The Muses' Library}, 8vo, p. 281; and in Evans’ \textit{Old Ballads},
i. 332 (ed.~of~1810).

Two very important and popular ballads were sung to the tune: \textit{Chevy Chace},
and \textit{The Lady’s Fall}.

\textit{Chevy Chace} had also a separate air (see page 199); but the earlier printed
copies of the ballad direct it to be sung to “\textit{In Pescod Time}."
\pagebreak
%197

The “\textit{Lamentable ballad of the Lady’s Fall}, to the tune of \textit{In Pescod Time},”
will be found in the Douce, Pepys, and Bagford Collections, and has been reprinted
by Percy and Ritson. It commences thus:—
\settowidth{\versewidth}{Mark well my heavy dolefull tale,}
\begin{scverse}
\begin{altverse}
\vleftofline{“}Mark well my heavy dolefull tale,\\
You loyal lovers all;\\
And heedfully bear in your breast\\
A gallant lady’s fall.”
\end{altverse}
\end{scverse}

Among the ballads to the tune of \textit{The Lady’s Fall} are \textit{The Bride’s Burial},
and \textit{The Lady Isabella’s Tragedy}; both in Percy’s Reliques. \textit{The life and death
of Queen Elizabeth}, in the \textit{Crown Garland of Golden Roses}, 1612 (page 39 of the
reprint), and in Evans’ \textit{Old Ballads}, iii. 171. \textit{The Wandering Jew, or the Shoemaker
of Jerusalem, who lived when our Saviour Christ was crucified, and appointed
to live until his coming again}; two copies in the British Museum, and one in
Mr. Halliwell’s Collection; also reprinted by Washbourne. It has the burden,
“Repent, therefore, O England,” and is, perhaps, the ballad by Deloney, to which
Nashe refers in \textit{Have with you to Saffron-Walde}n (ante page 107). \textit{The Cruel
Black}; see Evans’ \textit{Old Ballads}, iii. 232. \textit{A Warning for Maidens, or young
Bateman}; Roxburghe Collection, i. 501. It begins, “You dainty dames so finely
framed.” And \textit{You dainty dames} is sometimes quoted as a tune; also \textit{Bateman},
as in a ballad entitled “\textit{A Warning for Married Women}, to a West-country tune
called \textit{The Fair Maid of Bristol}, or Bateman, or \textit{John True}; Roxburghe, i. 502.

The following Carol is from a Collection, printed in 1642, a copy of which is in
Wood’s Library, Oxford. I have not seen it elsewhere.

\qquad\qquad “A Carol for Twelfth Day, to the tune of \textit{The Lady’s Fall}.”
\vspace{-0.5\baselineskip}
\settowidth{\versewidth}{Plum porridge, roast beef, and minc’d pies,}

\begin{dcverse}\begin{altverse}
Mark well my heavy doleful tale,\\
For Twelfth Day now is come,\\
And now I must no longer stay,\\
And say no word but mum.\\
For I perforce must take my leave\\
Of all my dainty cheer—\\
Plum porridge, roast beef, and minc’d pies,\\
My strong ale and my beer.
\end{altverse}

\begin{altverse}
Kind-hearted Christmas, now adieu,\\
For I with thee must part;\\
But oh! to take my leave of thee\\
Doth grieve me at the heart.\\
Thou wert an ancient housekeeper,\\
And mirth with meat didst keep;\\
But thou art going out of town,\\
Which causes me to weep.
\end{altverse}

\begin{altverse}
God knoweth whether I again\\
Thy merry face shall see;\\
Which to good fellows and the poor\\
Was always frank and free.\\
Thou lovest pastime with thy heart,\\
And eke good company;\\
Pray hold me up for fear I swound [swoon],\\
For I am like to die.
\end{altverse}

\begin{altverse}
Come, butler, fill a brimmer full,\\
To cheer my fainting heart,\\
That to old Christmas I may drink\\
Before he does depart.\\
And let each one that’s in the room\\
With me likewise condole,\\
And now, to cheer their spirits sad,\\
Let each one drink a bowl.
\end{altverse}

\begin{altverse}
And when the same it hath gone round,\\
Then fall unto your cheer;\\
For you well know that Christmas time\\
It comes but once a year.\\
But this good draught which I have drank\\
Hath comforted my heart;\\
For I was very fearful that\\
My stomach would depart.
\end{altverse}

\begin{altverse}
Thanks to my master and my dame,\\
That do such cheer afford;\\
God bless them, that, each Christmas, they\\
May furnish so their board.\\
My stomach being come to me,\\
I mean to have a bout;\\
And now to eat most heartily,—\\
Good friends, I do not flout.
\end{altverse}
\end{dcverse}

\pagebreak
%198


\musicinfo{Rather slow and smoothly.}{}

\includemusic{chappellV1098.pdf}

\musictitle{Chevy Chace.}

Although sometimes sung to the tunes of \textit{Pescod Time} and \textit{The Children in the
Wood}, this is the air usually entitled \textit{Chevy Chace}. It bears that name in all the
editions of \textit{Pills to purge Melancholy}, and in the ballad operas, such as \textit{The
Beggars' Opera}, 1728, \textit{Trick for Trick}, 1735, \&c. Another name, and probably
an older, is \textit{Flying Fame}, or \textit{When flying Fame}, to which a large number of
ballads have been written. In \textit{Pills to purge Melancholy}, “King Alfred and the
Shepherd’s Wife,” which the old copies direct to be sung to the tune of \textit{Flying
Fame}, is printed to this air.

Much has been written on the subject of \textit{Chevy Chace}; but as both the ballads
are printed in Percy’s \textit{Reliques of Ancient Poetry} (and in many other collections), 
it maybe sufficient here to refer the reader to that work, and to \textit{The
British Bibliographer} (iv. 97). The latter contains an account of Richard Sheale,
the minstrel to whom we are indebted for the preservation of the more ancient
ballad, and of his productions. The manuscript containing them is in the Ashmolean
Library, Oxford (No. 48, 4to). His verses on being robbed on Dunsmore
Heath have been already quoted (pages 45 to 47).

The ballad of \textit{Chevy Chace}, in Latin Rhymes, by Henry Bold, will be found in
Dryden’s \textit{Miscellany Poems}, ii. 288. The translation was made at the request of
Dr. Compton, Bishop of London.

Bishop Corbet, in his \textit{Journey into Fraunce}, speaks of having sung \textit{Chevy
Chace} in his youth; the antiquated beau in Davenant’s play of \textit{The Wits}, also
prides himself on being able to sing it; and, in \textit{Wit’s Interpreter}, 1671, a man,
enumerating the good qualities of his wife, cites, after the beauties of her mind
and her patience, “her curious voice, wherewith she useth to sing \textit{Chevy Chace}.”
From these, and many similar allusions, it is evident that it was much sung in
the seventeenth century, despite its length.

Among the many ballads to the tune (either as \textit{Flying Fame} or \textit{Chevy Chace}),
the following require particular notice.

\pagebreak
%199

“A lamentable song of the Death of King Lear and his three Daughters: to
the tune of \textit{When flying Fame}.” See Percy’s \textit{Reliques}, series i., book 2.

“A mournefull dittie on the death of Faire Rosamond; tune of \textit{Flying Fame}:”
beginning, “When as King Henry rul’d this land;” and quoted in Rowley’s
\textit{A~Match at Midnight}. See \textit{Strange Histories}, 1607; \textit{The Garland of Good-wil}l;
and Percy, series~ii., book 2.

“The noble acts of Arthur of the Round Table, and of Sir Launcelot du Lake:
tune of \textit{Flying Fame}.” See \textit{The Garland of Good-will}, 1678, and Percy, series~i.,
book 2. The first line of this ballad (“When Arthur first in court began”) is
sung by Falstaff in Part II. of Shakespeare’s \textit{King Henry IV}.; also in Marston’s
\textit{The Malcontent}, 1604, and in Beaumont and Fletcher’s \textit{The Little French Lawyer}.

“King Alfred and the Shepherd’s Wife: to the tune of \textit{Flying Fame}.” See
\textit{Old Ballads}, 1727, i. 43; \textit{Pills to purge Melancholy}, 1719, v. 289; and Evans’
\textit{Old Ballads}, 1810, ii. 11.

“The Union of the Red Rose and the White, by a marriage between King
Henry VII. and Elizabeth Plantagenet, daughter of Edward IV: to the tune of
When flying Fame.” See \textit{Crown Garland}, 1612, and Evans, iii. 35.

“The Battle of Agincourt, between the Englishmen and the Frenchmen: tune,
\textit{Flying Fame}.” (Commencing, “A council grave our King did hold.”) See
\textit{Crown Garland}, 1659, and Evans, ii. 351.

“The King and the Bishop: tune of \textit{Chevy Chace}.” Roxburghe, iii. 170.

“Strange and true newes of an Ocean of Flies dropping out a cloud, upon the
town of Bodnam [Bodmin?] in Cornwall: tune of \textit{Chevy Chace}” (dated 1647).
See King’s Pamphlets, Brit. Mus., vol. v., and Wright’s \textit{Political Ballads}.

“The Fire on London Bridge” (from which the nursery rhyme, “Three
children sliding on the ice,” has been extracted), “to the tune of \textit{Chevy Chace}.”
\textit{Merry Drollery complete}, 1670, \textit{Pills to purge Melancholy}, ii. 6, 1707, and
Rimbault’s \textit{Little Book of Songs and Ballads}, 12mo., 1851. Dr. Rimbault quotes
other copies of the ballad, and especially one in the Pepys Collection (ii. 146),
to the tune of \textit{The Lady's Fall}; further proving the difficulty of distinguishing
between this tune and \textit{In Pescod Time}.

\musicinfo{Smoothly and rather slow.}{}

\includemusic{chappellV1099.pdf}

\pagebreak
%200

\musictitle{The Children In The Wood.}

In the Registers of the Stationers’ Company, under the date of 15th October,
1595, we find, “Thomas Millington entred for his copie under t’handes of bothe
the Wardens, a ballad intitutled ‘The Norfolk Gentleman, his Will and Testament,
and howe he commytted the keeping of his children to his owne brother, whoe delte
moste wickedly with them, and howe God plagued him for it.” This entry agrees,
almost verbatim, with the title of the ballad in the Pepys Collection (i.~518),
but which is of later date. Copies will also be found in the Roxburghe (i.~284),
and other Collections; in \textit{Old Ballads}, 1726, i. 222; and in Percy’s \textit{Reliques},
series~iii., book 2.

Sharon Turner says, “I have sometimes fancied that the popular ballad of
\textit{The Children in the Wood} may have been written at this time, on Richard [III.]
and his nephews, before it was quite safe to stigmatize him more openly.”—
(\textit{Hist.~Eng}., iii.~487, 4to). This theory has been ably advocated by Miss
Halsted, in the Appendix to her \textit{Richard III. as Duke of Gloucester and King of
England}. Her argument is based chiefly upon internal evidence, there being no
direct proof that the ballad is older than the date of the entry at Stationers’ Hall.

In Wager’s interlude, \textit{The longer thou livest the more fool thou art}, Moros says,
“I can sing a song of Robin Redbreast;” and in Webster’s \textit{The White Devil},
Cornelia says, “I’ll give you a saying which my grandmother was wont, when
she heard the bell toll, to sing, unto her lute:
\settowidth{\versewidth}{Call for the robin-redbreast and the wren.}
\begin{scverse}
Call for the robin-redbreast and the wren.\\
Since o’er the shady groves they hover,\\
And with leaves and flowers do cover\\
The friendless bodies of unburied men,” \&c.\\
\vin\vin\vin\vin \textit{Dodsley’s Old Plays}, vi. 312, 1825.
\end{scverse}
These \textit{may} be in allusion to the ballad.

In Anthony à Wood’s Collection, at Oxford, there is a ballad to the tune of
\textit{The two Children in the Wood}, entitled “The Devil’s Cruelty to Mankind,” \&c.

The history of the tune is somewhat perplexing. In the ballad-operas of
\textit{The Jovial Crew}, \textit{The Lottery}, \textit{An old man taught wisdom}, and \textit{The Beggars’
Opera}, it is printed under the title of \textit{Now ponder well}, which are the first words
of “\textit{The Children in the Wood}.”

The broadsides of \textit{Chevy Chace}, which were printed \textit{with music} about the commencement
of the last century, are also to this tune; and in the ballad-opera of
\textit{Penelope}, 1728, a parody on Chevy Chace to the same.

In \textit{Pills to purge Melancholy}, 1707 and 1719, the ballads of “Henry V. at the
battle of Agincourt,” “The Lady Isabella’s Tragedy,” and a song by Sir John
Birkenhead, are printed to it. The last seems to be a parody on “Some Christian
people all give ear,” or “The Fire on London Bridge.”

According to the old ballads, \textit{The Battle of Agincourt} should be to the tune of
\textit{Flying Fame}, \textit{The Lady Isabella's Tragedy} to \textit{In Pescod Time}, and \textit{The Fire on
London Bridge} to \textit{Chevy Chace}. I suppose the confusion to have arisen from.
\textit{Chevy Chace} being sung to all the three tunes.

The traditions of the stage also give this as the air of the Gravedigger’s Song
in \textit{Hamlet}, “A pick-axe and a spade.”

\pagebreak