\changefontsize{1.02\defaultfontsize}

\headingthree{REIGN OF ELIZABETH.}

During the long reign of Elizabeth, music seems to have been in universal
cultivation, as well as in universal esteem. Not only was it a necessary qualification
for ladies and gentlemen, but even the city of London advertised the musical
abilities of boys educated in Bridewell and Christ’s Hospital, as a mode of
recommending them as servants, apprentices, or husbandmen.\dcfootnote{
“That the preachers be moved at the sermons at the
Crosse” [St. Paul’s Cross] “and other convenient times,
and that all other good notorious meanes be used, to require
both citizens, artificers, and other, and also all
farmers and other for husbandry, and gentlemen and other
for their kitchens and other services, to take servants and
children both out of Bridewell and Christ’s Hospital at
their pleasures,\ldots with further declaration that many
of them be of toward qualities in readyng, wryting, grammer, 
and \textit{musike}.” This is the 66th and last of the
“Orders appointed to be executed in the cittie of London,
for setting rog[u]es and idle persons to worke, and for
releefe of the poore.” “At London, printed by Hugh
Singleton, dwelling in Smith Fielde, at the signe of the
Golden Tunne;” reprinted in \textit{The British Bibliographer}.
Edward VI. granted the charters of incorporation for
Bridewell and Christ’s Hospital, a few days before his
death. Bridewell is a foundation of a mixed and singular
nature, partaking of the hospital, prison, and work-house. 
Youths were sent to the Hospital as apprentices
to manufacturers, who resided there; and on leaving, received
a donation of 10\textit{l}., and their freedom of the city.
Pepys, in his Diary, 5th October, 1664, says, “To new
Bridewell, and there I did with great pleasure see the
many pretty works, and the little children employed,
every one to do something, which was a very fine sight,
and worthy encouragement.”
} In Deloney’s
\textit{History of the gentle Craft}, 1598, one who tried to pass for a shoemaker was
detected as an imposter, because he could neither “sing, sound the trumpet, play
upon the flute, nor reckon up his tools in rhyme.” Tinkers sang catches; milkmaids
sang ballads; carters whistled; each trade, and even the beggars, had
their special songs; the base-viol hung in the drawing room for the amusement of
waiting visitors; and the lute, cittern, and virginals, for the amusement of waiting
customers, were the necessary furniture of the barber’s shop. They had
music at dinner; music at supper; music at weddings; music at funerals; music
at night; music at dawn; music at work; and music at play.

He who felt not, in some degree, its soothing influences, was viewed as a
morose, unsocial being, whose converse ought to be shunned, and regarded with
suspicion and distrust.

\settowidth{\versewidth}{Nor is not mov’d with concord of sweet sounds,}
\begin{scverse}
“The man that hath no music in himself,\\
Nor is not mov’d with concord of sweet sounds,\\
Is fit for treasons, stratagems, and spoils;\\
The motions of his spirit are as dull as night,\\
And his affections dark as Erebus:\\
Let no such man be trusted.”\\
\vin\vin\vin\vin\textit{Merchant of Venice}, act v., sc. 1.

“Preposterous ass! that never read so far\\
To know the cause why music was ordain’d!\\
Was it not to refresh the mind of man\\
After his studies, or his usual pain?”\\
\vin\vin\vin\vin\textit{The Taming of the Shrew}, act ii., sc. 3.
\end{scverse}
\markright{reign of elizabeth.}
\pagebreak
%99


Steevens, in a note upon the above passage in \textit{The Merchant of Venice}, quotes the
authority of Lord Chesterfield against what he terms this “capricious sentiment”
of Shakespeare, and adds that Peacham requires of his gentleman \textit{only} to be able
“to sing his part \textit{sure, and at first sight}, and withall to play the same on a viol,
or lute.” But this sentiment, so far from being peculiar to Shakespeare, may be
said to have been the prevailing one of Europe. Nor was Peacham an exception,
for, although he says, “I dare not pass so rash a censure of these” (who love not
music) “as Pindar doth; or the Italian, having fitted a proverb to the same effect,
\textit{Whom God loves not, that man loves not music};” he adds, “but I am verily persuaded
that they are by nature very ill disposed, and of such a brutish stupidity
that scarce any thing else that is good and savoureth of virtue is to he found
in them.”\dcfootnote{\textit{99.a}
The Compleat Gentleman: fashioning him absolute in
the most necessary and commendable qualities, concerning
mind or bodie, that may be required in a noble gentleman,
By Henry Peacham, Master of Arts, \&c, 1622.}
Tusser, in his “Points of Huswifry united to the comfort of
Husbandry,”~1570, recommends the country huswife to select servants that sing
at their work, as being usually the most pains-taking, and the best. He says:

\begin{scverse}
“Such servants are oftenest painfull and good,\\
That sing in their labour, as birds in the wood;”
\end{scverse}
and old Merrythought says, “Never trust a \textit{tailor} that does not sing at
his work, for his mind is of nothing but filching.”—(\textit{Dyce’s Beaumont and
Fletcher}, vol. ii., p. 171.)


Byrd, in his \textit{Psalmes, Sonnets, and Songs}, \&c., 1588, gives the following eight
reasons why every one should learn to sing:—

1st.—“It is a knowledge easily taught, and quickly learned, where there is a good
master and an apt scholar.”

2nd.—“The exercise of singing is delightful to nature, and good to preserve the
health of man.”

3rd.—“It doth strengthen all parts of the breast, and doth open the pipes.”

4th.—“It is a singular good remedy for a stutting and stammering in the speech.”

5th.—“It is the best means to procure a perfect pronunciation, and to make a good
orator.”

6th.—“It is the only way to know where nature hath bestowed a good voice; \ldots
and in many that excellent gift is lost, because they want art to express nature.”

7th.—“There is not any music of instruments whatsoever, comparable to that which
is made of the voices of men; where the voices are good, and the same well sorted
and ordered.”

8th.—“The better the voice is, the meeter it is to honour and serve God therewith;
and the voice of man is chiefly to be employed to that end.”

\begin{scverse}
“Since singing is so good a thing,\\
I wish all men would learn to sing.”
\end{scverse}

Morley, in his \textit{Introduction to Pratical Musick}, 1597, written in dialogue,
introduces the pupil thus: “But supper being ended, and music books,
according to custom, being brought to the table, the mistress of the house presented
me with a part, earnestly requesting me to sing; but when, after many
excuses, I protested unfeignedly that \pagebreak \textit{I could not}, every one began to wonder; yea,
%100
some whispered to others, demanding how I was brought up, so that upon shame
of mine ignorance, I go now to seek out mine old friend, Master Gnorimus, to
make myself his scholar.”

\changefontsize{1.05\defaultfontsize}

Laneham, to whom we are indebted for the description of the pageants at Kenilworth
in 1575, thus describes his own evening amusements. “Sometimes I foot
it with dancing; now with my gittern, and else with my cittern, then at the
virginals (ye know nothing comes amiss to me): then carol I up a song withal;
that by and by they come flocking about me like bees to honey; and ever they
cry, ‘Another, good Laneham, another.’” He who thus speaks of his playing
upon three instruments and singing, had been promoted from a situation in the
royal stables, through the. favour of the Earl of Leicester, to the duty of keeping
eaves-droppers from the council-chamber door.

Dekker, in \textit{The Gull’s Horn-book}, tells us that the usual routine of a young
gentlewoman’s education was “to read and write; to play upon the virginals,
lute, and cittern; and to read prick-song (\ie, music written or pricked down) \textit{at
first sight}.” Whenever a lady was highly commended by a writer of that age,
her skill in music was sure to be included; as—

\settowidth{\versewidth}{Her own tongue speaks all tongues, and her own hand}
\begin{scverse}
\vleftofline{“}Her own tongue speaks all tongues, and her own hand\\
Can teach all strings to speak in their best grace.”\\
\vin\vin\vin\vin\vin\vin \textit{Heywood’s A Woman kill’d with kindness}.
\end{scverse}

“Observe,” says Lazarillo, who is instructing the ladies how to render themselves
most attractive, “it shall be your first and finest praise to sing the note of
every new fashion at first sight.—(\textit{Middleton’s Blurt, Master Constable}, 1602.)
Gosson, in his \textit{Schoole of Abuse}, 1579, alluding to the custom of serenading,
recommends young ladies to be careful not to “flee to inchaunting,” and says, “if
assaulted with music in the night, close up your eyes, stop your ears, tie up your
tongues; when they speak, answer them not; when they halloo, stoop not; when
they sigh, laugh at them; when they sue, scorn them.” He admits that “these are
hard lessons,” but advises them “nevertheless to drink up the potion, though it
like not [please not] your taste.” In those days, however, the “serenate, which
the starv’d lover sings to his proud fair,” was not quite so customary in England
as the Morning song or \textit{Hunt’s-up}; such as—

\settowidth{\versewidth}{Fain would I wake you, sweet, but fear}
\indentpattern{000031331}
\begin{scverse}
\begin{patverse}
\vleftofline{“}Fain would I wake you, sweet, but fear\\
I should invite you to worse cheer;\ldots \\
I’d wish my life no better play,\\
Your dream by night, your thought, by day:\\
Wake, gently wake,\\
Part softly from your dreams!\\
\textit{The Morning flies}\\
To your fair eyes,\\
To guide her special beams.”
\end{patverse}
\end{scverse}


As to the custom of having a base-viol (or viol da gamba) hanging up in drawing
rooms for visitors to play on, one quotation from Ben Jonson may suffice:
“In making love to her, never fear to be out, for\ldots  a base viol shall hang o’ the
wall, of purpose, shall put you in \pagebreak presently.—(\textit{Gifford’s Edit}. vol. ii., p. 162.)
%101
If more to the same purport be required, many similar allusions will be found in
the same volume. (See pages 125,126, 127, and 472, and Gifford’s Notes.)

The base-viol was also played upon by ladies (at least during the following
reign), although thought by some “an unmannerly instrument for a woman.”
The mode in which some ladies passed their time is described in the following
lines, and perhaps, even in the present day, instances not wholly unlike might be
found.

\settowidth{\versewidth}{Sit and answer them that woo;}
\begin{scverse}\vleftofline{“}This is all that women do,\\
Sit and answer them that woo;\\
Deck themselves in new attire,\\
To entangle fresh desire;\\
After dinner sing and play,\\
Or dancing, pass the time away.”
\end{scverse}

“England,” says a French writer of the seventeenth century, “is the paradise of
women, as Spain and Italy are their purgatory.”\dcfootnote{
Description of England by Jorevin de Rocheford.
Paris, 1672.}

The musical instruments principally in use in barbers’ shops, during the
sixteenth and seventeenth centuries, were the cittern, the gittern, the lute, and
the virginals. Of these the cittern was the most common, perhaps because most
easily played. It was in shape somewhat like the English guitar of the last
century, but had only four double strings of wire, \ie, two to each note.\dcfootnote{
Sir John Hawkins, in his \textit{History of Music}, vol.ii.,
p.~602, 8vo., copies the \textit{Cistrum} from Mersenne, as the
\textit{Cittern}, but it has six strings, and therefore more closely
resembles the English guitar.}
These
were tuned to the notes \textit{g}, \textit{b}, \textit{d}, and \textit{e} of the present treble staff, or to corresponding
intervals; for no rules are given concerning the pitch of these instruments,
unless they were to be used in concert. The instructions for tuning are generally
to draw up the treble string as high as possible, without breaking it, and to tune
the others from that. A particular feature of the cittern was the carved head,
which is frequently alluded to by the old writers.\dcfootnote{
In \textit{Love’s Labour Lost}, act v., sc. 2, Boyet compares
Holofernes’ countenance to that of a cittern head. In
Forde's \textit{Lovers’ Melancholy}, act ii., sc. 1, “Barbers shall
wear thee on their citterns;” and in Fletcher’s \textit{Love’s
Cure}, “Yon cittern head! you ill-countenanced cur!”
\&c., \&c.}
 Playford in his “\textit{Musick’s
Delight on the Cithren} restored and refined to a more easie and pleasant manner of
playing than formerly,” 1666, speaks of having revived the instrument, and restored
it to what it was in the reign of Queen Mary, and his tuning agrees with
that in Anthony Holborne’s \textit{Cittharn Schoole}, 1597, and in Thomas Robinson’s \textit{New
Citharen Lessons}, 1609. The peculiarity of the cittern, or cithren, was that the
third string was tuned lower than the fourth, so that if the first or highest string
were tuned to \textit{e}, the third would be the \textit{g} below, and the fourth the intermediate \textit{b}.
The cittern appears to have been an instrument of English invention.\dcfootnote{
The word \textit{Cetera}, as employed by Galilei (father of
the great astronomer, Galileo Galilei), I assume to mean
Cittern, because the word \textit{Liuto}, for Lute, was in common
use. He says, “Fu la \textit{Cetera} usata prima tra gli Inglesi
che da altre nazioni, nella quale Isola si lavoravano già
in eccellenza; quantunque hoggi le più riputate da loro
siano quelle che si lavorano in Brescia; con tutto questo
è adoperata ed apprezzata da nobili, e fu così detta dagli
autori di essa, per forse resuscitare l’antica Cithara; ma
la differenza che sia tra la nostra e quella, si è possuto
benissimo conoscere da quello che se n’è di sopra detto.”—
\textit{Dialogo di Vincenzo Galilei, nobile Fiorentino}, fol.~1581,
p. 147.}

Of the gittern or ghitterne, I can say but little, not having seen any instruction-book
for the instrument. Ritson says it \pagebreak differed chiefly from the cittern
%102
in being strung with gut instead of wire; and, from the various allusions to it,
I have no doubt of his correctness. Perhaps, also, it was somewhat less in size.
In the catalogue of musical instruments left in the charge of Philip van Wilder,
at the death of Henry VIII, we find “four Gitterons, which are called Spanish
vialles.” As Galilei says, in 1581, that “Viols are little used in Spain, and that
they do not make them,”\dcfootnote{
“La viola da gamba, e da braccio, nella Spagna non
se ne fanno, e poco vi si usano.”—\textit{Dialogo della Musica},
fol. 1581., p. 147.}
I assume Spanish viol to mean the guitarra, or guitar.
The gittern is ranked with string instruments in the following extract from the
old play of \textit{Lingua}, written in this reign:—

\settowidth{\versewidth}{’Tis true the finding of a dead horse-head}
\begin{scverse}
\vleftofline{“}’Tis true the finding of a dead horse-head\\
Was the first invention of \textit{string} instruments,\\
Whence rose the \textit{Gitterne, Viol, and the Lute};\\
Though others think the Lute was first devis’d\\
In imitation of a tortoise back,\\
Whose sinews, parched by Apollo’s beams,\\
Echo’d about the concave of the shell:\\
And seeing the shortest and smallest gave shrillest sound,\\
They found out \textit{Frets}, whose sweet diversity\\
(Well touched by the skilful learned fingers)\\
Raiseth so strange a multitude of \textit{Chords};\\
Which, their opinion, many do confirm,\\
Because \textit{Testudo} signifies a Lute.’’\\
\vin\vin\vin\vin\vin\vin\textit{Dodsley's Old Plays}, vol. v., p. 198.
\end{scverse}

\changefontsize{1.07\defaultfontsize}

Coles, in his Dictionary, describes gittern as a \textit{small} sort of cittern, and Playford
printed \textit{Cithren and Gittern Lessons, with plain and easie Instructions for Beginners
thereon}, together in one book, in 1659. Ritson may have gained his information
from this book, as he mentions it in the second edition of his \textit{Ancient Songs}, but
I have not succeeded in finding a copy.

The lute (derived from the Anglo-Saxon \textit{Hlud}, or \textit{Lud}, \ie, \textit{sounded}), was
once the most popular instrument in Europe, although now rarely to be seen,
except represented in old pictures. It has been superseded by the guitar, but
for what reason it is difficult to say, unless from the greater convenience of the
bent sides of the guitar for holding the instrument, when touching the higher notes
of the finger-board. The tone of the lute is decidedly superior to the guitar, being
larger, and having a convex back, somewhat like the vertical section of a gourd, or
more nearly resembling that of a pear. As it was used chiefly for accompanying
the voice, there were only eight frets, or divisions of the finger-board, and these
frets (so called from fretting, or stopping the strings) were made by tying
pieces of cord, dipped in glue, tightly round the neck of the lute, at intervals
of a semitone. It had virtually six strings, because, although the number
was eleven or twelve, five, at least, were doubled, the first, or treble, being
sometimes a single string.\dcfootnote{
I speak only of the usual English lute. There were
lutes of various sizes, from the mandura, or mandore,
to the theorbo and arch-lute; some with less, and others
with more strings.}
 The head, \pagebreak in which the pegs to turn the strings were
%103
inserted, receded almost at a right angle. The most usual mode of tuning it was
as follows: assuming \textit{c} in the third space of the treble clef to be the pitch of the
first string (\ie, \textit{cc} in the scale given at page 14), the base, or sixth string would
be \textit{C}; the tenor, or fifth, \textit{F}; the counter-tenor, or fourth, \textit{b} flat; the great
mean, or third, \textit{d}; the small mean, or second, \textit{g}; and the minikin, or treble, \textit{cc}.\dcfootnote{
The notes which these letters represent will be seen
by referring to the scale at p. 14.}

Lute strings\dcfootnote{
Mace, in his \textit{Musick’s Monument}, 1678, speaking of
lute-strings, says, “Chuse your trebles, seconds, and
thirds, and some of your small octaves, especially the
sixth, out of your \textit{Minikins}; the fourth and fifth, and
most of your octaves, of \textit{Venice Catlins}; your \textit{Pistoys} or
\textit{Lyons} only for the great bases.” In the list of CustomHouse
duties printed in 1545, the import duty on “lutestrings
called \textit{Mynikins}” was 22\textit{d}. the gross, but as no
other lute-strings are named, I assume that only the
smallest were then occasionally imported. Minikin is
one of the many words, derived from music or musical
instruments, which have puzzled the commentators on
the old dramatists. The first string of a violin was also
called a minikin.}
were a usual present to ladies as new-year’s gifts. From
Nichols’ \textit{Progresses} we learn that queen Elizabeth received a box of lute-strings,
as a new-year’s gift, from Innocent Corry, and at the same time, a box of lute-strings
and a glass of sweet water from Ambrose Lupo. When young men
in want of money went to usurers, it was their common practice to lend it
in the shape of goods which could only be re-sold at a great loss; and lute-strings
were then as commonly the medium employed as bad wine is now. In Lodge’s
\textit{Looking Glasse for London and Englande}, 1594, the usurer being very urgent
for the repayment of his loan, is thus answered, “I pray you, Sir, consider that
my loss was great by the commodity I took up; you know, Sir, I borrowed of you
forty pounds, whereof I had ten pounds in money, and thirty pounds in lute-strings, 
which, when I came to sell again, I could get but five pounds for them, so
had I, Sir, but fifteen pounds for my forty.” So in Dekker’s \textit{A Night’s Conjuring}, 
the spendthrift, speaking of his father, says, “He cozen’d young gentlemen
of their land, only for me, had acres mortgaged to him by wiseacres for three
hundred pounds, paid in hobby-horses, dogs, bells, and lute-strings, which, if they
had been sold by the drum, or at an out-rop (auction), with the ciy of ‘No man
better?’ would never have yielded \textit{£}50.” Nash alludes twice to the custom. In
\textit{Will Summer’s Last Will and Testament}, he says, “I know one that ran in debt,
in the space of four or five years, above fourteen thousand pounds in lute-strings
and grey paper;” and in \textit{Christ’s Tears over Jerusalem}, 1593; “In the first instance, 
spendthrifts and prodigals obtain what they desire, but at the second time
of their coming, it is doubtful to say whether they shall have money or no: the world
grows hard, and we are all mortal: let them make him any assurance before a
judge, and they shall have some hundred pounds (per consequence) in silks and
velvets. The third time, if they come, they have baser commodities. The fourth
time, lute-strings and grey paper; and then, I pray you pardon me, I am not for
you: pay me what you owe me, and you shall have anything.” (Dodsley, v.~9,
p. 22.)

The virginals (probably so called because chiefly played upon by young girls),
resembled in shape the “square” pianoforte of the present day, as the harpsichord
did the “grand.” The sound of the pianoforte is produced by a hammer \textit{striking}
the strings, but when the keys of the virginals or harpsichord were pressed, the
“jacks,” (slender pieces of wood, armed \pagebreak at the upper ends with quills) were
%104
raised to the strings, and acted as \textit{plectra}, by impinging, or twitching them.
These \textit{jacks} were the constant subject of simile and pun; for instance, in a play
of Dekker’s, where Matheo complains that his wife is never at home, Orlando says,
“No, for she’s like a pair of virginals, always with \textit{jacks} at her tail.”—(Dodsley’s
Old Plays, vol. iii., p. 398). And in Middleton’s \textit{Father Hubburd’s Tales}, describing
Charity as frozen, he says, “Her teeth chattered in her head, and leaped
up and down like virginal jacks.”

One branch of the barber’s occupation in former days was to draw teeth, to bind
up wounds, and to let blood. The parti-coloured pole, which was exhibited at the
doorway, painted after the fashion of a bandage, was his sign, and the teeth
he had drawn were suspended at the windows, tied upon lute strings. The lute,
the cittern, and the gittern hung from the walls, and the virginals stood in the
corner of his shop. “If idle,” says the author of \textit{The Trimming of Thomas
Nashe}, “barbers pass their time in life-delighting musique,” (1597). The
barber in Lyly’s \textit{Midas}, (1592), says to his apprentice, “Thou knowest I have
taught thee the knacking of the hands,\dcfootnote{
The knacking of the hands was a peculiar crack with
the fingers, by knocking them together, which every
barber was expected to make while shaving a customer.}
 like the tuning of a cittern,” and
Truewit, in Ben Jonson’s \textit{Silent Woman}, wishes the barber “may draw his own
teeth, and add them to the lute-string.” In the same play, Morose, who had
married the barber’s daughter, thinking her faithless, exclaims “That cursed
barber! I have married his \textit{cittern}, that is common to all men.” One of the
commentators not understanding this, altered it to “I have married his \textit{cistern},”~\&c. 
Dekker also speaks of “a barber’s cittern for every serving-man to play
upon.”

One of the \textit{Merrie-conceited jests of George Peele} is the stealing of a barber’s
lute, and in \textit{Lord Falkland’s Wedding Night}, we read “He has travelled
and speaks languages, as a barber’s boy plays o’th’ gittern.” Ben Jonson says,\dcfootnote{
Every man in his humour. Act iii., sc, 2.}
“I can compare him to nothing more happily than a barber’s virginals; for every
man may \textit{play} upon him,” and in \textit{The Staple of News}, “My barber Tom, one
Christmas, got into a Masque at court, by his wit and the good means of his
cittern, holding up thus for one of the music.” To the latter passage Gifford adds
another in a note. “For you know, says Tom Brown, that a cittern is as natural
to a barber, as milk to a calf, or dancing bears to a bagpiper.”

As to the music they played, we may assume it to have been, generally,
the common tunes of the day, and such as would be familiar to all. Morley, in
his \textit{Introduction to Music}, tells us that the tune called the \textit{Quadrant Pavan}, was
called \textit{Gregory Walker}, “because it walketh ’mongst barbers and fiddlers more
common than any other,” and says in derision, “Nay, you sing you know not
what; it should seem you came lately from a barber’s shop, where you had
\textit{Gregory Walker}, or a Coranto, played in the new proportions by them lately found
out.” Notwithstanding this, we find the \textit{Quadran Pavan} (so called, I suppose,
because it was a pavan for four to dance) was one of the tunes arranged for
queen Elizabeth in her Virginal Book; \pagebreak and Morley, himself, arranged it for
%105
several instruments in his \textit{Consort Lessons}. I have alluded to the custom of
introducing \textit{old} songs into plays, and playing \textit{old} tunes at the beginning and end
of the acts, at p. 72. Queen Elizabeth’s Virginal Book, and Lady Neville’s,
contain little else than old tunes, arranged with variations, or as then more
usually termed, with “division.” It is often difficult to extract the air accurately
from these arrangements, if there be no other copy as a guide. Occasionally
a mere skeleton of the tune is given, sometimes it is “in prolation,” \ie, with
every note drawn out to two, four, or eight times its proper duration, sometimes
the melody is in the base, at others it is to be found in an inner part.

The rage for popular tunes abroad had shewn itself in the Masses set to
music by the greatest composers. Baini, in his Life of Palestrina, gives, what
he terms, a \textit{short} list (“breve elenco”) of some of them. It contains the
names of eighty secular tunes upon which Masses had been composed, and sung
even in the Pope’s chapel. The tunes have principally French names, some
are of lascivious songs, others of dance tunes. He names fifty different authors
who composed them, and intimates that there is a much larger number than he
has cited in the library of the Vatican.\dcfootnote{
“Memorie storico-critiche della Vita, e delle Opere di
Giovanni Pierluigi da Palestrina.”—\textit{Roma}, 2 vols, 4to.,
1828. Vol. i., p. 136, et seq. This evil was checked by a
decree of the Council of Trent.}
 Even our island was not quite irreproachable
on this point. Shakespeare speaks of Puritans singing psalms to
hornpipes, and the Presbyterians sang their \textit{Divine Hymns} to the tunes of
popular songs, the titles of some of which the editor of \textit{Sacred Minstrel}sy (vol. i.,
p. 7) “would not allow to sully his pages.” Generally, however, the passion
for melody expended itself in singing old tunes about the country, in the streets,
and at the ends of plays, in playing them in barbers’ shops, or at home, when
arranged for chamber use with all the art and embellishment our musicians could
devise. The scholastic music of that age, great as it was, was so entirely devoted
to harmony, and that harmony so constructed upon old scales, that scarcely anything
like tune could be found in it—I mean such tune as the uncultivated ear
could carry away. Many would then, no doubt, say with Imperia, “I cannot abide
these dull and lumpish tunes; the musician stands longer a pricking them than
I would do to hear them: no, no, give me your light ones.”—(Middleton’s \textit{Blurt,
Master Constable.}) No line of demarcation could be more complete than that
between the music of the great composers of the time, and, what may be termed,
the music of the people. Perhaps the only instance of a tune by a well-known
musician of that age having been afterwards used as a ballad tune, is that of \textit{The
Frog Galliard}, composed by Dowland. Musicians held ballads in contempt, and
the great poets rarely wrote in ballad metre.

Dr. Drake, in his \textit{Shakespeare and his Times}, gives a list of two hundred and
thirty-three British poets\dcfootnote{
The word “Poet” is here too generally applied. “It
is already said (and, as I think, truly said) it is not
rhyming and versing that maketh poesy: one may he a
poet without versing, and a versifier without poetry.”—
\textit{Sir Philip Sidney’s Defence of Poesy}.}
 (forty major, and one hundred and ninety-three
minor), who were contemporaneous with Shakespeare, and even that list, large as
it is, might be greatly extended from miscellanies, and from ballads. Some idea
of the number of ballads that were printed \pagebreak in the early part of the reign of
%106
Elizabeth may be formed from the fact that seven hundred and ninety-six ballads,
left for entry at Stationers’ Hall, remained in the cupboard of the council chamber
of the company at the end of the year 1560, to be transferred to the new
Wardens, and only forty-four books.\dcfootnote{
See \textit{Collier's Extracts from the Registers of the Stationers’
Company}, vol. i., p. 28.}
 As to the latter part of her reign, see
Bishop Hall, 1597.

\settowidth{\versewidth}{Some drunken rhymer thinks his time well spent}
\begin{scverse}
\vleftofline{“}Some drunken rhymer thinks his time well spent\\
If he can live to see his name in print;\\
Who, when he once is fleshed to the press,\\
And sees his handsell have such fair success,\\
\textit{Sung to the wheel, and sung unto the pail},\dcfootnote{
“Sung to the wheel,” \ie, to the spinning wheel; and
“sung to the pail,” sung by milk-maids, of whose love of
ballads further proofs will be adduced.}\\
He sends forth \textit{thraves\dcfootnote{
“Thrave” signifies a number of sheaves of corn set
up together; metaphorically, an indefinite number of anything.—
\textit{Nares' Glossary}.}
 of ballads to the sale}.”
\end{scverse}

And to the same purport, in \textit{Martin Mar-sixtus}, 1592: “I lothe to speak it,
every red-nosed rhymester is an author; every drunken man’s dream is a book;
and he, whose talent of little wit is hardly worth a farthing, yet layeth about him
so outrageously as if all Helicon had run through his pen: in a word, scarce a cat
can look out of a gutter, but out starts a halfpenny chronicler, and presently a
proper new ballet of a strange sight is indited.”

Henry Chettle, in his pamphlet entitled \textit{Kind Hart’s Dream}, 1592, speaks of
idle youths singing and selling ballads in every corner of cities and market towns,
and especially at fairs, markets, and such like public meetings. Contrasting that
time with the simplicity of former days, he says, “What hath there not, contrary
to order, been printed? Now ballads are abusively chanted in every street; and
from London this evil has overspread Essex and the adjoining counties. There is
many a tradesman of a worshipful trade, yet no stationer, who after a little bringing
up apprentices to singing brokery, takes into his shop some fresh men, and
trusts his servants of two months’ standing with a dozen groatsworth of ballads.
In which, if they prove thrifty, he makes them pretty chapmen, able to spread
more pamphlets by the state forbidden, than all the booksellers in London.”
He particularly mentions the sons of one Barnes, most frequenting Bishop’s
Stortford, the one with a squeaking treble, the other with an ale-blown base, as
bragging that they earned twenty shillings a day; whilst others, horse and man,
the man with many a hard meal, and the horse pinched for want of provender,
have together hardly taken ten shillings in a week.

In a pamphlet intended to ridicule the follies of the times, printed in 1591, the
writer says, that if men that are studious would “read that which is good, a poor
man may be able”—not to obtain bread the cheaper, but as the most desirable of
all results, he would be able “to buy three ballets for a halfpenny.”\dcfootnote{
\textit{Fearefull and lamentable effects of two dangerous Comets
that shall appeare}, \&c., 4to, 1591.}

\settowidth{\versewidth}{And tell prose writers, stories are so stale}
\begin{scverse}
\vleftofline{“}And tell prose writers, stories are so stale,\\
That penny ballads make a better sale.”\\
\vin\vin\vin\vin\vin\vin\textit{Pasquill’s Madness}, 1600.
\end{scverse}

The words of the ballads were written by such men as Elderton, “with his ale-crammed 
nose,” and Thomas Deloney, “the balleting silk-weaver of Norwich.”
%107
The former is thus described in a MS. of the time of James I., in the possession
of Mr. Payne Collier:—

\settowidth{\versewidth}{Will. Elderton’a red nose is famous everywhere,}
\begin{scverse}
\vleftofline{“}Will. Elderton’a red nose is famous everywhere,\\
And many a ballet shows it cost him very dear;\\
In ale, and toast, and spice, he spent good store of coin,\\
You need not ask \textit{him} twice to take a cup of wine.\\
But though his nose was red, his hand was very white,\\
In work it never sped, nor took in it delight;\\
No marvel therefore ’tis, that white should be his hand,\\
That ballets writ a score, as you well understand.”
\end{scverse}

Nashe, in \textit{Have with you to Saffron Walden}, says of Deloney, “He hath rhyme
enough for all miracles, and wit to make a \textit{Garland of Good Will}, \&c., but
whereas his muse, from the first peeping forth, hath stood at livery at an ale-house
wisp, never exceeding a penny a quart, day or night—and this dear year,
together with the silencing of his looms, scarce that—he is constrained to betake
himself to carded ale” (i. e., ale mixed with small beer), “whence it proceedeth
that since Candlemas, or his jigg of \textit{John for the King}, not one \textit{merry} ditty will
come from him; nothing but \textit{The Thunderbolt against swearers, Repent, England,
repent}, and the \textit{Strange Judgments of God}.”

In 1581, Thomas Lovell, a zealous puritan, (one who objected to the word
Christmas, as savouring too much of popery, and calls it Chris\textit{tide}), published
“A Dialogue between Custom and Verity, concerning the use and abuse of
dauncinge and minstralsye.” From this, now rare book, Mr. Payne Collier has
printed various extracts. The object was to put down dancing and minstrelsy;
Custom defends and excuses them, and Verity, who is always allowed to have the
best of the argument, attacks and abuses them. It shows, however, that the old
race of minstrels was not quite extinct. Verity says:—

\settowidth{\versewidth}{But this do minstrels clean forget:}
\begin{scverse}
\begin{altverse}
\vleftofline{“}But this do minstrels clean forget:\\
Some godly songs they have,\\
Some wicked ballads and unmeet,\\
As companies do crave.\\
For filthies they have filthy songs;\\
For ‘some’ lascivious rhymes;\\
For honest, good; for sober, grave\\
Songs; so they watch their times.\\
Among the lovers of the truth,\\
Ditties of truth they sing;\\
Among the papists, such as of\\
Their godless legends spring\ldots\\
\textit{The minstrels do, with instruments,\\
With songs, or else with jest,\\
Maintain themselves}: but, as they use, [act]\\
Of these naught is the best.”\\
\vin\textit{Collier’s Extracts Reg. Stat. Comp}., vol. ii., pp.~144,~145.
\end{altverse}
\end{scverse}

\changefontsize{1.03\defaultfontsize}

Carew, in his \textit{Survey of Cornwall}, 1602, \pagebreak speaking of Tregarrick, then the
%108
 residence of Mr. Buller, the sheriff, says, “It was sometime the Wideslade’s
inheritance, until the father’s rebellion forfeited it,” and the “son then led
a walking life with his harp, to gentlemen’s houses, where-through, and by his
other active qualities, he was entitled Sir Tristram; neither wanted he (as some
say) a ‘\textit{belle Isound},’ the more aptly to resemble his pattern.”

So in the “Pleasant, plain, and pithy pathway, leading to a virtuous and honest
life” (about 1550),

\settowidth{\versewidth}{By which my minstrelsy and my fair speech and sport,}
\begin{scverse}\vleftofline{“}Very lusty I was, and pleasant withall,\\
To sing, dance, and play at the ball\ldots \\
And besides all this, I could then finely play\\
On the harp much better than now far away,\\
By which my minstrelsy and my fair speech and sport,\\
All the maids in the parish to me did resort.”
\end{scverse}

As minstrelsy declined, the harp became the common resource of the blind,
and towards the end of the reign of Elizabeth, harpers were proverbially blind:—
%\settowidth{\versewidth}{If thou’lt not have her look’d on by thy guests,}
\begin{scverse}\vleftofline{“}If thou’lt not have her look’d on by thy guests,\\
Bid none but harpers henceforth to thy feasts.”\\
\vin\vin\vin\vin\vin\vin\vin\vin\textit{Guilpins Skialetheia}, 1598.
\end{scverse}

There are many ballads about blind harpers, and many tricks were played upon
them, such as a rogue engaging a harper to perform at a tavern, and stealing the
plate “while the unseeing harper plays on.” As to the other street and tavern
musicians, Gosson tells us, in his \textit{Short Apologie of the Schoole of Abuse}, 1586,
that “London is so full of unprofitable pipers and fiddlers, that a man can no
sooner enter a tavern, than two or three cast (\ie, companies) of them, hang at
his heels, to give him a dance before he departs,” but they sang ballads and
catches as well as played dances. They also played at dinner,
%\settowidth{\versewidth}{But to the music, nor a drop of wine}
\begin{scverse}\vin\vin\vin “Not a dish removed\\
But to the music, nor a drop of wine\\
Mixt with the water, without harmony.”
\end{scverse}

“Thou need no more send for a fidler to a feast (says Lyly), than a beggar to~a~fair.”

Part-Singing, and especially the singing Rounds, or Roundelays, and Catches,
was general throughout England during the sixteenth and seventeenth centuries.
In the Moralities and the earliest plays, when part-music was sung instead of old
ballads, it was generally in Canon, for although neither Round, Catch, nor Canon
be specified, we find some direction from the one to the other to sing \textit{after} him.\dcfootnote{
Catch, Round or Roundelay, and Canon in unison, are,
in music, nearly the same thing. In all, the harmony is to
be sung by several persons; and is so contrived, that,
though each sings precisely the same notes as his fellows,
yet, by beginning at stated periods of time from each
other, there results a harmony of as many parts as there
are singers. The Catch differs only in that the words of
one part are made to answer, or catch the other; as, “Ah!
how, Sophia,” sung like “a house o' fire,” “Burney’s
History,” like “burn his history,” \&c.}
Thus, in the old Morality called \textit{New Custome} (Dodsley, vol. i.), Avarice says:—
%\settowidth{\versewidth}{If you he good fellows, let us depart with a song.”}

\begin{scverse}\vleftofline{“}But, Sirs, because we have tarried so long,\\
If you be good fellows, let us depart with a song.”
\end{scverse}
To which Cruelty answers:—
\begin{scverse}\vleftofline{“}I am pleased, and therefore let every man\\
\textit{Follow after in order} as well as he can.”
\end{scverse}
\pagebreak
%109
And in John Heywood’s \textit{The Four P’s}, one of our earliest plays, the Apothecary,
haying first asked the Pedler whether he can sing at sight, says, “Who that lyste
\textit{sing after me}.” In neither case are the words of the Round given.

Tinkers, tailors, blacksmiths, servants, clowns, and others, are so constantly
mentioned as singing music in parts, and by so many writers, as to leave no doubt
of the ability of at least many among them to do so.

Perhaps the form of Catch, or Round, was more generally in favour, because,
as each would sing the same notes, there would he but one part to remember, and
the tune would guide those who learnt by ear.

We find Roundelays generally termed “merry,” and cheerfulness was the
common attribute of country songs.

In Peele’s \textit{Arraignment of Paris}, 1584:—

\begin{scverse}
“Some Rounds, or merry Roundelays,—we sing no other songs;\\
Your \textit{melancholic notes not to our country mirth belongs}.”
\end{scverse}

And in his King Edward I., the Friar says:—

\begin{scverse}
“And let our lips and voices meet in a merry country song.”
\end{scverse}

In Shakespeare’s \textit{A Winter’s Tale}, when Autolycus says that the song is a
merry one, and that “there’s scarce a maid westward but she sings it,” Mopsa
answers, “We can both sing it: if thou wilt hear a part, thou shalt hear—’tis
in three parts.”

Tradesmen and artificers had evidently not retrograded in their love of music
since the time of Chaucer, whose admirable descriptions have been before quoted,
(p. 33, et seq.) Occleve, a somewhat later poet, has also remarked the different
effect produced by the labour of the hand and of the head. He says:—

\begin{scverse}
\vleftofline{“}These artificers see I, day by day,\\
In the hottest of all their business,\\
Talken and sing, and make game and play,\\
And forth their labour passeth with gladness;\\
But we labour in travailous stillness;\\
We stoop and stare upon the sheep-skin,\\
And keep most our song and our words in.”
\end{scverse}

From the numerous allusions to their singing in parts, I have selected the
following. Peele, in his \textit{Old Wive’s Tale}, 1595, says, “This \textit{smith} leads a life as
merry as a king. Sirrah Frolic, I am sure you are not without some Round or
other; no doubt but Clunch (the smith) can bear his part;” which he accordingly
does. In \textit{Damon and Pithias}, 1571, Grimme the \textit{collier} sings “a bussing base,”
and Jack and Will, two of his fellows, “quiddell upon it,” that is, they sing the
tune and words of the song whilst he buzzes the burden or under-song. In Ben
Jonson’s \textit{Silent Woman}, we find, “We got this cold sitting up late and singing
Catches with \textit{cloth-workers}.” In Shakespeare’s \textit{Twelfth Night}, Sir Toby says,
“Shall we rouse the night-owl in a Catch that will draw three souls out of one
\textit{weaver}?” and, in the same play, Malvolio says, “Do you make an ale-house of
my lady’s house that ye squeak out your \textit{cozier’s Catches}, without any mitigation
or remorse of voice?” \pagebreak Dr. Johnson says cozier  means a \textit{tailor}, from “coudre,”
%110
to sew; but Nares quotes four authorities to prove it to mean a \textit{cobbler}. In
Beaumont and Fletcher’s \textit{Coxcomb} we find—

\settowidth{\versewidth}{“Where were the \textit{Watch} the while? Good sober gentlemen,}
\begin{scverse}
“Where were \textit{the Watch} the while? Good sober gentlemen,\\
They were, like careful members of the city,\\
Drawing in diligent ale, and singing Catches.”
\end{scverse}

In \textit{A Declaration of egregious Impostures}, 1604, by Samuel Harsnet (afterwards
Archbishop of York), he speaks of “the master setter of Catches, or Rounds,
used to be sung by \textit{tinkers} as they sit by the fire, with a pot of good ale between
their legs.”

Sometimes the names of these Catches are given, as, for instance, “Three blue
beans in a blue bladder, rattle, bladder, rattle,” mentioned in Peele’s \textit{Old Wive’s
Tale}, in Ben Jonson’s \textit{Bartholomew Fair}, and in Dekker’s \textit{Old Fortunatus}; or
“Whoop, Barnaby,” which is also frequently named. But whoever will read the
words of those in \textit{Pammelia}, \textit{Deuteromelia}, Hilton’s \textit{Catch that catch can}, or Playford’s
\textit{Musical Companion}, will not doubt that many of the \textit{Catches} were intended for
the ale-house and its frequenters; but not so generally, the Rounds or Roundelays. 
Singing in parts was, by no means, confined to the meridian of London;
Carew, in his \textit{Survey of Cornwall}, 1602, says the same of Cornishmen: “Pastimes
to delight the mind, the Cornishmen have guary miracles [miracle plays] and
\textit{three-men’s songs}, cunningly contrived for the ditty, and pleasantly for the note.”

Catches seem to have increased in use towards the latter part of the seventeenth
century, for, although I cannot cite an instance of one composed by a
celebrated musician of Elizabeth’s reign, in that of Charles II. such cases were
abundant.

Some of the dances in favour in the reign of Elizabeth will be mentioned as
the tunes occur; the Queen herself danced galliards in her sixty-ninth year, and,
when given up by her physicians in her last illness, refusing to take medicine, she
sent for her band to play to her; upon which Beaumont, the French Ambassador,
remarks, in the despatch to his court, that he believed “she meant to die as
cheerfully as she had lived.” Her singing and playing upon the lute and
virginals have been so often mentioned, that I will not further allude to them
here.

\centerrule

\musictitle{All in a Garden Green.}

By the Registers of the Stationers’ Company we find that in 1565 William
Pickering had a license to print “A Ballett intituled \textit{All in a garden grene},
between two lovers;” and in 1568-9, William Griffith had a similar license. In
1584, “an excellent song of an outcast lover,” beginning “My fancie did I fire
in faithful form and frame,” to the tune of \textit{All in a garden grene}, appeared in
\textit{A Handeful of Pleasant Delites}.

In the rare tract called “Westward for smelts, or the Waterman’s fare of mad
merry Western Wenches,” quarto, 1603, the boatman, finding his fare sleeping,
sprinkles a little cool water on them with his oar, and, to “keep them from melancholy 
sleep,” promises “to strain the best voice \pagebreak he has, and not to cloy their ears
%111
with an \textit{old fiddler’s song}, as \textit{Riding to Rumford}, or \textit{All in a garden green}, but to
give them a new one of a serving man and his mistress, which neither fiddler nor
ballad-singer had ever polluted with their unsavoury breath.”

In the British Museum is a copy of “Psalmes, or Songs of Sion, turned into
the language, and set to the tunes of a strange land, by W[illiam] S[latyer],
intended for Christmas Carols, and fitted to divers of the most noted and common,
but solemne tunes, every where in this land familiarly used and knowne.” 1642.
Upon this copy, a former possessor has written the names of some of the tunes to
which the author designed them to be sung. \textit{One of these is All in a garden grene}.

The tune is in William Ballet’s Lute Book, from which this copy is taken, and
in \textit{The Dancing Masters} of 1651, 1670, 1686, 1690, \&c. The first part of the
air is the same as another in \textit{The Dancing Master}, called \textit{Gathering of Peascods}.
(See Index.)

The words are contained in a manuscript volume, in the possession of Mr.
Payne Collier.

\noindent\begin{minipage}{\textwidth}
\musicinfo{Moderate time.}{}
\bigskip
\smallskip

\includemusic{chappellV1040.pdf}
\end{minipage}

\pagebreak
%112

\settowidth{\versewidth}{And shines as bright and hot}
\begin{dcverse}\begin{altverse}
Quoth he, “Most lovely maid,\\
My troth shall aye endure;\\
And be not thou afraid,\\
But rest thee still secure,
\end{altverse}

\begin{altverse}
That I will love thee long\\
As life in me shall last;\\
Now I am strong and young,\\
And when my youth is past.
\end{altverse}

\begin{altverse}
When I am gray and old,\\
And then must stoop to age,\\
I’ll love thee twenty-fold,\\
My troth I here engage.”
\end{altverse}

\begin{altverse}
She heard with joy the youth,\\
When he thus far had gone;\\
She trusted in his truth,\\
And, loving, he went on:
\end{altverse}

\begin{altverse}
\vleftofline{“}Yonder thou seest the sun\\
Shine in the sky so bright,\\
And when this day is done,\\
And cometh the dark night,
\end{altverse}

\begin{altverse}
No sooner night is not,\\
But he returns alway,\\
And shines as bright and hot\\
As on this gladsome day.
\end{altverse}

\begin{altverse}
He is no older now\\
Than when he first was born;\\
Age cannot make him bow,\\
He laughs old Time to scorn.
\end{altverse}

\begin{altverse}
My love shall be the same,\\
It never shall decay,\\
But shine without all blame,\\
Though body turn to clay.”
\end{altverse}

\begin{altverse}
She listed to his song,\\
And heard it with a smile,\\
And, innocent as young,\\
She dreamèd not of guile.
\end{altverse}

\begin{altverse}
No guile he meant, I ween,\\
For he was true as steel,\\
As was thereafter seen\\
When she made him her weal.
\end{altverse}
\end{dcverse}

\settowidth{\versewidth}{Full soon both two were wed,}
\begin{scverse}
\begin{altverse}
Full soon both two were wed,\\
And these most faithful lovers\\
May serve at board at bed,\\
Example to all others.
\end{altverse}
\end{scverse}
\normalsize
\musictitle{Row Well, Ye Mariners.}

From the Registers of the Stationers’ Company, we find that in 1565-6,
William Pickering had a license to print a ballet entitled, \textit{Row well, ye mariners},
and in the following year, “Row well, ye mariners, moralized.” In 1566-7,
John Allde had a license to print “Stand fast, ye mariners,” which was, in all
probability, another moralization; and in the following year, two others; the one,
“Row well, ye mariners, moralized, with the story of Jonas,” the other, “Row
well, Christ’s mariners.” In 1567-8, Alexander Lacy took a license to print
“Row well, God’s mariners,” and in 1569-70, John Sampson to print “Row
well, ye mariners, for those that look big.” These numerous entries sufficiently
prove the popularity of the original, and I regret the not having succeeded in
finding a copy of any of these ballads.

Three others, to the tune of \textit{Row well, ye mariners}, have been reprinted by
Mr. Payne Collier, in his \textit{Old Ballads}, for the Percy Society. The first (dated
1570)—
\settowidth{\versewidth}{A lamentation from Rome, how the Pope doth bewail}
\begin{scverse}
\vleftofline{“}A lamentation from Rome, how the Pope doth bewail\\
That the rebels in England cannot prevail.”
\end{scverse}
The second, “The end and confession of John Felton, who suffred in Paules
Churcheyarde, in London, the 8th August [1570], for high treason.” Felton
placed the Bull of Pope Pius V., excommunicating Elizabeth, on the gate of the
palace of the Bishop of London, and was hung on a gallows set up expressly
before that spot. The third, “A warning to London by the fall of Antwerp.”
\origpage{113}%113

In \textit{A Handefull of Pleasant Delites}, 1584, there is “A proper sonet, wherein
the lover dolefully sheweth his grief to his love and requireth pity,” which is
also, to the tune of \textit{Row well, ye mariners}.

The tune is printed in Thomas Robinson’s \textit{Schoole of Musick}, fol., 1603, and
in every edition of \textit{The Dancing Master} that I have seen, from the first, dated
1651, to the eighteenth, 1725.

Not having the original words, a few verses from the “Lamentation from
Rome,” above mentioned, are given as a specimen of the merry political ballad of
those days. It is the Song of a fly buzzing about the Pope’s nose. The Pope and
his court are supposed to be greatly disconcerted at the news of the defeat of the
rebels in Northumberland.

\noindent\begin{minipage}{\textwidth}
\musicinfo{Moderate time and smoothly.\footnote{
I have added the old burden over the verse, feeling no doubt of its having
been sung to this part of the tune.}}{}

\includemusic{chappellV1041.pdf}
\end{minipage}


\pagebreak
%114

\indentpattern{010100001111}
\settowidth{\versewidth}{New news to him was brought that night,}
\begin{dcverse}\begin{patverse}
But as he was asleep,\\
Into the same again I got;\\
I crept therein so deep,\\
That I had almost burnt my coat.\\
New news to him was brought that night,\\
The rebels they were put to flight;\\
But, Lord, how then the Pope took on,\\
And called for a Mary-bone.\\
Up-ho!-make-haste,\\
My lovers all he like to waste;\\
Rise-Cardinal,-up-Priest,\\
Saint Peter he doth what he list.
\end{patverse}

\begin{patverse}
So then they fell to mess;\\
The friars on their heads did pray;\\
The Pope began to bless,\\
At last he wist not what to say.\\
It chanced so the next day morn,\\
A post came blowing of his horn,\\
Saying, Northumberland is take;\\
But then the Pope began to quake.\\
He-then-rubb’d-his-nose,\\
With pilgrim-salve he ’noint his hose;\\
Run-here,-run-there,\\
His nails, for anger, ’gan to pare.
\end{patverse}

\begin{patverse}
When he perceived well\\
The news was true to him was brought.\\
Upon his knees he fell,\\
And then Saint Peter he besought\\
That he would stand his friend in this,\\
To help to aid those servants his,\\
And he would do as much for him—\\
But Peter sent him to Saint Sim.\\
So-then-he-snuff’d,\\
The friars all about he cuff’d,\\
He-roar’d,-he-cried;\\
The priests they durst not once abide.
\end{patverse}

\begin{patverse}
The Cardinals then begin\\
To stay, and take him in their arms,\\
He spurn’d them on the shin.\\
Away they trudg’d, for fear of harms.\\
So then the Pope was left alone;\\
Good Lord! how he did make his moan!\\
The stools against the walls he threw,\\
And me, out of his nose he blew.\\
I-hopp’d,-I-skipp’d,\\
From place to place, about I whipp’d;\\
He-sware,-he-tare,\\
Till from his crown he pull’d the hair.
\end{patverse}
\end{dcverse}

\musictitle{Lord Willoughby.}

This tune is referred to under the names of Lord Willoughby; Lord Wil\-loughby’s
\textit{March}, and Lord Willoughby’s \textit{Welcome Home}. In Queen’s Elizabeth’s
Virginal Book, it is called Rowland.

In Lady Neville’s Virginal Book (MS., 1591), and in Robinson’s \textit{School of
Music}, 1603, it is called “Lord Willobie’s Welcome Home:” the ballad of The
Carman’s Whistle was to be sung to the tune of The Carman’s Whistle, or to
Lord Willoughby’s \textit{March}; and that of “Lord Willoughby—being a true relation
of a famous and bloody battel fought in Flanders, \&c., against the Spaniards;
where the English obtained a notable victory, to the glory and renown of our
nation”—was to the tune of “Lord Willoughby, \&c.” A copy of the last will
be found in the Bagford Collection of Ballads, British Museum.

Peregrine Bertie, Lord Willoughby of Eresby, one of the bravest and most
skilful soldiers of this reign, had distinguished himself in the Low Countries in
1586, and in the following year, on the recall of the Earl of Leicester, was
made commander of the English forces. The tune, with which his name was
associated, was as popular in the Netherlands as in England, and continued so, in
both countries, long after his death, which occurred in 1601. It was printed at
Haerlem, with other English tunes, in 1626, in Neder-landtsche Gedenck-clanck,
under the name of \textit{Soet Robbert}, and \textit{Soet, soet Robbertchen} [Sweet Robert, and
Sweet, sweet little Robert], which it probably derived from some other ballad
sung to the time.

As the ballad of “Brave Lord Willoughby” is printed in Percy’s \textit{Reliques of
Ancient Poetry}, a few verses, only, are subjoined.
\pagebreak
%115


\noindent\begin{minipage}{\textwidth}
\musicinfo{In Marching time.}{}

\includemusic{chappellV1042.pdf}
\end{minipage}

\settowidth{\versewidth}{For yonder comes Lord Willoughbey}
\begin{dcverse}
\begin{altverse}
Stand to it, noble pikemen,\\
And look you round about:\\
And shoot you right, you bowmen,\\
And we will keep them out:\\
You musquet and calìver men,\\
Do you prove true to me,\\
I’le be the foremost man in fight,\\
Says brave Lord Willoughbèy.
\end{altverse}

\begin{altverse}
The sharp steel-pointed arrows,\\
And bullets thick did fly,\\
Then did our valiant soldiers\\
Charge on most furiously;\\
Which made the Spaniards waver,\\
They thought it best to flee,\\
They fear’d the stout behaviour\\
Of brave Lord Willoughbèy.
\end{altverse}

\begin{altverse}
Then quoth the Spanish general,\\
Come let us march away,\\
I fear we shall be spoiled all\\
If here we longer stay;\\
For yonder comes Lord Willoughbey\\
With courage fierce and fell,\\
He will not give one inch of way\\
For all the devils in hell.
\end{altverse}

\begin{altverse}
And then the fearful enemy\\
Was quickly put to flight,\\
Our men pursued couragiously,\\
And caught their forces quite;\\
But at last they gave a shout,\\
Which ecchoed through the sky,\\
God, and St. George for England!\\
The conquerors did cry.
\end{altverse}
\end{dcverse}
\pagebreak
%116

\begin{dcverse}
\begin{altverse}
To the souldiers that were maimed,\\
And wounded in the fray,\\
The queen allowed a pension\\
Of fifteen pence a day;\\
And from all costs and charges\\
She quit and set them free:\\
And this she did all for the sake\\
Of brave Lord Willoughbèy.
\end{altverse}

\begin{altverse}
Then courage, noble Englishmen,\\
And never be dismaid;\\
If that we be but one to ten\\
We will not be afraid\\
To fight with foraign enemies,\\
And set our nation free.\\
And thus I end the bloody bout\\
Of brave Lord Willoughbèy.
\end{altverse}
\end{dcverse}

\musictitle{All Flowers of the Broom.}

This is mentioned as a dance tune by Nicholas Breton, in a passage already
quoted from his \textit{Works of a young Wit}, 1577 (ante p. 91); and by Nashe, in the
following, from his \textit{Have with you to Saffron-Walden}, 1596:—

“Or doo as Dick Harvey did, that having preacht and beat downe three pulpits in
inveighing against dauncing, one Sunday evening, when his wench or friskin was footing
it aloft on the greene, with foote out and foote in, and as busie as might be at
\textit{Rogero}, \textit{Basilino}, \textit{Turkelony}, \textit{All the flowers of the broom}, \textit{Pepper is black}, \textit{Greene
Sleeves}, \textit{Peggie Ramsey},\scfootnote{
All the tunes here mentioned will be found in this Collection, except \textit{Basilino}.}
 he came sneaking behind a tree, and lookt on; and though
hee was loth to be seene to countenance the sport, having laid God’s word against it so
dreadfully; yet to shew his good will to it in heart, hee sent her eighteen pence in
hugger-mugger (\ie, in secret), to pay the fiddlers.”

The tune is contained in William Ballet’s Lute Book, under the name of
\textit{All floures in broome}.

\includemusic{chappellV1043.pdf}

\pagebreak
%117

\musictitle{I am the Duke of Norfolk, or Paul’s Steeple.}

This tune is frequently mentioned under both names. In Playford’s \textit{Dancing
Master}, from 1650 to 1695, it is called Paul’s Steeple. In his \textit{Division Violin},
1685, at page 2, it is called \textit{The Duke of Norfolk, or Paul’s Steeple}; and at
page 18, \textit{Paul’s Steeple, or the Duke of Norfolk}.

The steeple of the \textit{old} Cathedral of St. Paul was proverbial for height. In the
\textit{Vulgaria}, printed by Wynkin de Worde, in 1530, we read: “Poule’s Steple is a
mighty great thing, and so hye that unneth [hardly] a man may discerne
the wether cocke,—the top is unneth perceived.” So in Lodge’s \textit{Wounds of
Civil War}, a clown talks of the \textit{Paul’s Steeple of honour}, as the highest point
that can be attained. The steeple was set on fire by lightning, and burnt
down on the 4th June, 1561; and within seven days, a ballad of “The true
report of the burning of the steeple and church of Paul’s, in London,” was
entered, and afterwards printed by William Seres, “at the west-ende of Pawles
church, at the sygne of the Hedghogge.” In 1564, a ballad was entered for
“the encouraging all kind of men to the re-edifying and building Paul’s steeple
again;” but the spire was never re-constructed. Mr. Payne Collier has printed
a ballad, written on the occasion of the fire, in his \textit{Extracts from the Registers of
the Stationers’ Company}, vol. i., p. 40; and it seems to have been intended for the
time. The first verse is as follows:—

\settowidth{\versewidth}{Lament each one the blazing fire,}
\indentpattern{0101221}
\begin{scverse}
\begin{patverse}
\vleftofline{“}Lament each one the blazing fire,\\
That down from heaven came,\\
And burnt S. Pawles his lofty spire\\
With lightning’s furious flame.\\
Lament, I say,\\
Both night and day,\\
Sith London’s sins did cause the same.”
\end{patverse}
\end{scverse}

In 1562-3, John Cherlewood had a license for printing another, called “When
young Paul’s steeple, old Paul’s steeple’s child.”

In Fletcher’s comedy, \textit{Monsieur Thomas}, act iii., sc. 3, a fiddler, being questioned
as to what ballads he is best versed in, replies:

\begin{scverse}
\vleftofline{“}Under your mastership’s correction, I can sing\\
\textit{The Duke of Norfolk}; or the merry ballad\\
\textit{Of Diverus and Lazarus; The Rose of England;\\
In Crete, when Dedimus first began;\\
Jonas, his crying out against Coventry;\\
Maudlin, the merchant's daughter;\\
The Devil and ye dainty dames;\\
The landing of the Spaniards at Bow;\\
With the bloody battle at Mile-End}.”\dcfootnote{
Of the ballads mentioned aboye, Diverus and Lazarus
seems to be an intentional corruption of Dives and Lazarus.
The Rose of England may be—
\begin{fnverse}
“The rose, the rose, the English rose,\\
It is the fairest flower that blows;”
\end{fnverse}
a copy of which is in Mr. Payne Collier's Manuscript; or,
perhaps, Deloney’s ballad of \textit{Fair Rosamond}, reprinted in
\textit{Percy's Reliques of Ancient Poetry}. \textit{In Crete} is often referred
to as a ballad tune; for instance, \textit{My mind to me a
kingdom is}, was to be sung to the tune of \textit{In Crete}, according
to a black-letter copy in the Pepysian Collection,
\textit{Maudlin, the merchant's daughter}, is \textit{The merchant's daughter
of Bristow} [Bristol], to the tune of The maiden's joy. (See
Roxburghe Collection, vol~i., 232, or Collier’s Roxburghe
Ballads, p. 104). \textit{Ye dainty dames}, are the first words of
\textit{A warning for maidens}, to the tune of \textit{The ladies’ fall}. (Sec
Roxburghe Collection, vol. i., 501). \textit{The landing of the
Spaniards}, \&c. (probably on some mock-fight of the train
hands, who exercised at Mile-end) seems to be referred to
in\textit{ The Knight of the Burning Pestle}, act ii., sc. 2.}
\end{scverse}

\pagebreak
%118
In the Pepysian Collection, vol. i., 146, and Roxburghe Collection, vol. i., 180,
is a black-letter ballad, called “A Lanthorne for Landlords” to the tune of
\textit{The Duke of Norfolk}, the initial lines of which are—

\settowidth{\versewidth}{With sobbing grief my heart will break}
\begin{scverse}
\vleftofline{“}With sobbing grief my heart will break\\
Asunder in my breast, \&c.”
\end{scverse}

In \textit{The Loyal Garland}, 1686, and in the Roxburghe Collection, vol. ii., 188 (or
Collier’s Roxburghe Ballads, p. 312), \textit{God speed the plough, and bless the corn-mow},
\&c., to the tune of \textit{I am the Duke of Norfolk}, beginning—

\settowidth{\versewidth}{My noble friends, give ear,}
\indentpattern{001001}
\begin{scverse}
\begin{patverse}
\vleftofline{“}My noble friends, give ear,\\
If mirth you love to hear,\\
I’ll tell you as fast as I can,\\
A story very true:\\
Then mark what doth ensue,\\
Concerning a husbandman.”
\end{patverse}
\end{scverse}
This ballad-dialogue, between a husbandman and a serving-man, has been orally
preserved in various parts of the country. One version will be found in Mr. Davies
Gilbert’s \textit{Christmas Carols}; a second in Mr. J. H. Dixon’s \textit{Ancient Poems and
Songs of the Peasantry} (printed for the Percy Society); and a third in “Old
English Songs, as now sung by the Peasantry of the Weald of Surrey and Sussex,”
\&c.,; “harmonized for the Collector” [the Rev. Mr. Broadwood] “in 1843, by
G. A. Dusart.”

In the \textit{Collection of Poems on Affairs of State}, vol. iii., 70, is “A new ballad
to an old tune, called \textit{I am the Duke of Norfolk}.” It is a satire on Charles II.,
and begins thus:—

\settowidth{\versewidth}{I am a senseless thing, with a hey, with a hey;}
\begin{scverse}
\indentpattern{00110}
\begin{patverse}
\vleftofline{“}I am a senseless thing, with a hey, with a hey;\\
Men call me a king, with a ho;\\
To my luxury and ease,\\
They brought me o’er the seas,\\
With a hey nonny, nonny, nonny no.”
\end{patverse}
\end{scverse}

In Shadwell’s \textit{Epsom Wells}, 1673, act iii., sc. 1, we find, “Could I not play
\textit{I am the Duke of Norfolk}, \textit{Green Sleeves}, and the fourth Psalm, upon the
virginals?” and in Wycherley’s \textit{Gentleman Dancing Master}, Ger. says, “Sing
him \textit{Arthur of Bradley}, or \textit{I am the Duke of Norfolk}.”

A curious custom still remains in parts of Suffolk, at the harvest suppers, to
sing the song “I am the Duke of Norfolk” (here printed with the music); one
of the company being crowned with an inverted pillow or cushion, and another
presenting to him a jug of ale, kneeling, as represented in the vignette of the
Horkey. [See \textit{Suffolk Garland}, 1818, p. 402.] The editor of the \textit{Suffolk
Garland} says, that “this custom has most probably some allusion to the homage
formerly paid to the Lords of Norfolk, the possessors of immense domains in the
county.” To “serve the Duke of Norfolk,” seems to have been equivalent to
making merry, as in the following speech of \textit{Mine host}, at the end of the play of
\textit{The merry Devil of Edmonton}, 1617:—

\pagebreak
%119

\settowidth{\versewidth}{Ha! ere’t be night, \textit{I’ll serve the good Duke of Norfolk}.}
\begin{scverse}
\vleftofline{“}Why, Sir George, send for Spendle’s noise\scfootnote{
Spindle’s noise, \ie, Spindle's band, or company of musicians.}
 presently;\\
Ha! ere’t be night, \textit{I’ll serve the good Duke of Norfolk}.”
\end{scverse}
To which Sir John rejoins:—
\begin{scverse}
\vleftofline{“}Grass and hay! mine host, let’s live till we die,\\
And be merry; and there’s an end.”\\
\vin\vin\vin\vin\vin\textit{Dodsley’s Old Plays}, vol. v., 271.
\end{scverse}

Dr. Letherland, in a note which Steevens has printed on King Henry IV.,
Part I., act ii., sc. 4 (where Falstaff says, “This chair shall be my state, this
dagger my sceptre, and this \textit{cushion my crown}”), observes that the country people
in Warwickshire also use a \textit{cushion for a crown}, at their harvest home diversions;
and in the play of King Edward IV., Part II., 1619, is the following passage:—
\settowidth{\versewidth}{Then comes a slave, one of those drunken sots,}
\begin{scverse}
\vleftofline{“}Then comes a slave, one of those drunken sots,\\
In with a tavern reck’ning for a supplication,\\
Disguised with a cushion on his head.”
\end{scverse}

In the Suffolk custom, he who is crowned with the pillow, is to take the ale, to
raise it to his lips, and to drink it off without spilling it, or allowing the cushion
to fall; but there was, also, another drinking custom connected with this tune.
In the first volume of \textit{Wit and Mirth, or\textit{ Pills to purge Melancholy}}, 1698 and
1707, and the third volume, 1719, is a song called \textit{Bacchus’ Health}, “to be sung
by all the company together, with directions to be observed.” They are as
follows: “First man stands up, with a glass in his hand, and sings—
\indentpattern{0202012002002002}
\begin{scverse}
\begin{patverse}
Here’s a health to jolly Bacchus, (\textit{sung three times})\\
I-ho, I-ho, I-ho;\\
For he doth make us merry, (\textit{three times})\\
I-ho, I-ho, I-ho.\\
\vleftofline{* }Come sit ye down together, (\textit{three times})\\
{\footnotesize (At this star all bow to each other and sit down.)}\\
I-ho, I-ho, I-ho;\\
And bring† more liquor hither (\textit{three times})\\
{\footnotesize (At this dagger all the company beckon to the drawer.)}\\
I-ho, I-ho, I-ho.\\
It goes into the * cranium, (\textit{three times})\\
{\footnotesize \vleftofline{(At this star the} first man drinks his glass, while the others sing and point at him.)}\\
I-ho, I-ho, I-ho;\\
And † thou’rt a boon companion (\textit{three times})\\
{\footnotesize \vleftofline{(At this dag}ger all sit down, each clapping the next man on the shoulder.)}\\
I-ho, I-ho, I-ho.
\end{patverse}
\end{scverse}
Every line of the above is to be sung three times, except I-ho, I-ho, I-ho. Then
the second man takes his glass and sings; and so round.

About 1728, after the success of \textit{The Beggars’ Opera}, a great number of other
ballad operas were printed. In the \textit{Cobblers’ Opera}, and some others, this tune is
called \textit{I am the Duke of Norfolk}; but in \textit{The Jovial Crew}, \textit{The Livery Rake}, and
\textit{The Lover his own Rival}, it is called \textit{There was a bonny blade}. It acquired that
name from the following song, which is still occasionally to be heard, and which
is also in \textit{Pills to purge Melancholy}, from 1698 to 1719:—
\pagebreak
%120

\indentpattern{010110}
\settowidth{\versewidth}{But ah! and alas! she was dumb, dumb, dumb.}
\begin{dcverse}
\begin{patverse}
\vin There was a bonny blade,\\
Had married a country maid,\\
And safely conducted her home, home, home;\\
She was neat in every part,\\
And she pleas’d him to the heart,\\
But ah! and alas! she was dumb, dumb, dumb.
\end{patverse}

\begin{patverse}
\vin She was bright as the day,\\
And brisk as the May,\\
And as round and as plump as a plum,\\
But still the silly swain\\
Could do nothing but complain\\
Because that his wife she was dumb.
\end{patverse}

\begin{patverse}
\vin She could brew, she could bake,\\
She could sew, and she could make,\\
She could sweep the house with a broom;\\
She could wash, and she could wring,\\
And do any kind of thing,\\
But ah! and alas! she was dumb.
\end{patverse}

\begin{patverse}
\vin To the doctor then he went,\\
For to give himself content,\\
And to cure his wife of the mum:\\
\vleftofline{“}Oh! it is the easiest part\\
That belongs unto my art\\
For to make a woman speak that is dumb.”
\end{patverse}

\begin{patverse}
\vin To the doctor he did her bring,\\
And he cut her chattering string,\\
And at liberty he set her tongue;\\
Her tongue began to walk,\\
And she began to talk\\
As though she never had been dumb.
\end{patverse}

\begin{patverse}
\vin Her faculty she tries,\\
And she fills the house with noise,\\
And she rattled in his ears like a drum;\\
She bred a deal of strife,\\
Made him weary of his life—\\
He’d give any thing again she was dumb.
\end{patverse}

\begin{patverse}
\vin To the doctor then he goes,\\
And thus he vents his woes:\\
\vleftofline{“}Oh! doctor, you’ve me undone;\\
For my wife she’s turn’d a scold,\\
And her tongue can never hold,\\
I’d give any kind of thing she was dumb.”
\end{patverse}

\begin{patverse}
\vin \vleftofline{“}When I did undertake\\
To make thy wife to speak,\\
It was a thing easily done,\\
But ’tis past the art of man,\\
Let him do whate’er he can,\\
For to make a scolding wife hold her tongue.”
\end{patverse}
\end{dcverse}

From the last line of the verses of this song, the tune also became known as
“Alack! and alas! she was dumb,” or “Dumb, dumb, dumb.”

\noindent\begin{minipage}{\textwidth}
\musicinfo{Rather slow.}{}
\smallskip

\includemusic{chappellV1044.pdf}
\end{minipage}

\pagebreak
