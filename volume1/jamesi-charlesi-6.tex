%361
\changefontsize{1.05\defaultfontsize}

\musictitle{The Court Lady.}

The first ballad in the \textit{Collection of Old Ballads}, 8vo., 1727, vol. i., is “The
unfortunate Concubine, or Rosamond’s Overthrow; occasioned by her brother’s
praising her beauty to two young knights of Salisbury, as they rid on the road.
To the tune of \textit{The Court Lady}.” I have not found the ballad of \textit{The Court Lady},
but the tune is contained in \textit{The Dancing Master}, from 1650 to 1698, under the
name of \textit{Confess}, or \textit{The Court Lady}.

This ballad of Fair Rosamond is so exceedingly long (twenty-six stanzas of
eight lines, and occupying ten pages in vol. ii. of Evans’ \textit{Old Ballads}, where it is
reprinted), that the first, third, and fourth stanzas only, are here subjoined.

\musicinfo{Moderate time.}{}

\includemusic{chappellV1193.pdf}

\settowidth{\versewidth}{As three young knights of Salisbury}
\begin{dcverse}\begin{altverse}
As three young knights of Salisbury\\
Were riding on the way,\\
One boasted of a fair lady,\\
Within her bower so gay:\\
I have a sister, Clifford swears,\\
But few men do her know;\\
Upon her face the skin appears\\
Like drops of blood on snow.
\end{altverse}

\begin{altverse}
My sister’s locks of curled hair\\
Outshine the golden ore;\\
Her skin for whiteness may compare\\
With the fine lily flow’r;\\
Her breasts are lovely to behold,\\
Like to the driven snow;\\
I would not, for her weight in gold,\\
King Henry should her know, \&c.
\end{altverse}
\end{dcverse}
\pagebreak
%362

\musictitle{Gather Your Rosebuds While You May.}

This song is in Playford’s \textit{Ayres and Dialogues}, 1659, p. 101; in Playford’s
\textit{Introduction to Music}, third edit., 1660; in \textit{Musick’s. Delight on the Cithren}, 1666;
and in \textit{The Musical Companion}, 1667. The music is the composition of William
Lawes; the poetry by Herrick. It became popular in ballad-form, and is in the
list of those printed by W. Thackeray, at the Angel in Duck Lane, as well as
in \textit{Merry Drollery Complete}, 1670. It has been reprinted (from a defective
copy) in Evans’ \textit{Old Ballads}, iii. 287, 1810. Herrick addresses it “To the
Virgins, to make much of time.” \textit{Hesperides}, i. 110,1846.

\includemusic{chappellV1194.pdf}

\settowidth{\versewidth}{The glorious lamp of heaven, the sun,}
\begin{dcverse}\begin{altverse}
The glorious lamp of heaven, the sun,\\
The higher he is getting,\\
The sooner will his race be run,\\
And nearer he’s to setting.
\end{altverse}

\begin{altverse}
That age is best which is the first,\\
When youth and blood are warmer;\\
But being spent, the worse and worst\\
Times still succeed the former.
\end{altverse}

\begin{altverse}
Then be not coy, but use your time,\\
And, while ye may, go marry;\\
For having once but lost your prime,\\
You may for ever tarry.
\end{altverse}
\end{dcverse}

\musictitle{Three Merry Boys Are We.}

This is properly a round, and composed by William Lawes, who was appointed
Gentleman of the Chapel Royal in 1602. He became afterwards one of Charles
the First’s Chamber Musicians, and was killed fighting for his cause in 1645.

It is to he found in Hilton’s \textit{Catch that catch can}, 1652; in Playford’s \textit{Musical
Companion}; in \textit{Musick’s Delight on the Cithren}; \&c. The words have been
adduced by Sir John Hawkins to illustrate the \textit{Three merry men are we} quoted
by Shakespeare. See note to \textit{Twelfth Night}, act ii., sc. 3.

In \textit{Merry Drollery Complete}, 1670, is a parody on this, entitled “The Cambridge
Droll”—
\begin{dcverse}\settowidth{\versewidth}{101100}
\begin{patverse}
\vin “The ’proctors are two and no more,\\
Then hang them, that makes them three;\\
The taverns are but four,\\
\vin I wish they were more for me:\\
For three merry boys, and three merry boys,\\
And three merry boys are we.”
\end{patverse}
\end{dcverse}

%\changefontsize{0.98\defaultfontsize}
\pagebreak%363

\musicinfo{Boldly.}{}

\includemusic{chappellV1195.pdf}

\settowidth{\versewidth}{And three merry girls, and three merry girls,}
\begin{dcverse}\begin{altverse}
The virtues they were seven,\\
And three the greater he;\\
The Caesars they were twelve,\\
And the fatal sisters three.\\
And three merry girls, and three merry girls,\\
And three merry girls are we.
\end{altverse}
\end{dcverse}

Another \textit{Three merry boys are we} has been already quoted (ante p. 216).

\musictitle{Cupid’s Courtesy.}

Copies of this ballad are in the Roxburghe Collection, ii. 58; and in the Douce
Collection, p. 27. It is also printed entire, with the tune, in \textit{Pills to purge
Melancholy}, vi. 43.

The copy in the Roxburghe Collection may be dated as of the reign of
Charles II., being “printed by and for W. O[nley], for A[lexander] M[ilbourne],
and sold by the booksellers;” but Mr. Payne Collier, who reprints it in his \textit{Book
of Roxburghe Ballads}, p. 80, mentions “a manuscript copy, dated 1595,” as still
extant. The words are in the same metre as \textit{Phillida flouts me}, and \textit{Lady lie near
me} (ante pages 183 and 185), but the stanzas are shorter, being of eight instead
of twelve lines. The ballad is entitled “\textit{Cupid’s Courtesie}; or The young Gallant
foil’d at his own weapon. To a most pleasant \textit{Northern tune}.”

In another volume of the Douce Collection (p. 264) is “The Young Man’s
Vindication against The Virgin’s Complaint. Tune of \textit{The Virgin’s Complaint},
or \textit{Cupid’s Courtesie};” commencing—
\settowidth{\versewidth}{Sweet virgin, hath disdain}
\begin{dcverse}\begin{altverse}
\vleftofline{“}Sweet virgin, hath disdain\\
Mov’d you to passion,—\\
Ne’er to love man again,\\
But for the fashion?” \&c.
\end{altverse}
\end{dcverse}

%\changefontsize{0.97\defaultfontsize}
\pagebreak
%364

This is also in eight-line stanzas (black-letter); and a former possessor has
penciled against the name of the tune, “\textit{I am so deep in love}” I have referred
to \textit{I am so deep in love} (ante p. 183) as probably another name for \textit{Phillida flouts
me}, but on this authority it should rather be to \textit{Cupid's Courtesy};

\musicinfo{Smoothly.}{}

\includemusic{chappellV1196.pdf}

\settowidth{\versewidth}{“Little boy, tell me why thou art here diving;}
\begin{dcverse}\footnotesizerr
“Little boy, tell me why thou art here diving;\\
Art thou some runaway, and hast no biding?”\\
“I am no runaway; Venus, my mother,\\
She gave me leave to play, when I came hither.”

“Little boy, go with me, and be my servant;\\
I will take care to see for thy preferment.” \\
“If I with thee should go, Venus would chide me,\\
And take away my bow, and never abide me.”

“Little boy, let me know what’s thy name termed,\\
That thou dost wear a bow, and go’st so armed?”\\
“You may perceive the same with often changing,\\
Cupid it is my name; I live by ranging.”

“If Cupid be thy name, that shoots at rovers,\\
I have heard of thy fame, by wounded lovers:\\
Should any languish that are set on fire\\
By such a naked brat, I much admire.”
%\columnbreak

“If thou dost but the least at my laws grumble,\\
I’ll pierce thy stubborn breast, and make thee humble:\\
If I with golden dart wound thee but surely,\\
There’s no physician’s art that e’er can cure thee.”

“Little boy, with thy bow why dost thou threaten?\\
It is not long ago since thou wast beaten.\\
Thy wanton mother, fair Venus, will chide thee:\\
When all thy arrows are gone, thou may’st go hide thee.”

“Of powerful shafts, you see, I am well stored,\\
Which makes my deity so much adored:\\
With one poor arrow now I’ll make thee shiver,\\
And bend unto my how, and fear my quiver.”
\end{dcverse}

\changefontsize{1.03\defaultfontsize}
\pagebreak
%365

\settowidth{\versewidth}{“Although thou call’st me blind, surely I’ll hit thee,}
\begin{dcverse}\footnotesizerr
“Dear little Cupid, be courteous and kindly:\\
I know thou canst not hit, but shootest blindly.” \\
“Although thou call’st me blind, surely I’ll hit thee,\\
That thou shalt quickly find; I’ll not forget thee.”

Then little Cupid caught his bow so nimble,\\
And shot a fatal shaft which made me tremble.\\
“Go, tell thy mistress dear thou canst discover\\
What all the passions are of a dying lover.”

And now his gallant heart sorely was bleeding,\\
And felt the greatest smart from love proceeding:\\
He did her help implore whom he affected,\\
But found that more and more him she rejected.

For Cupid with his craft quickly had chosen,\\
And with a leaden shaft her heart had frozen;\\
Which caus’d this lover more sadly to languish,\\
And Cupid’s aid implore to heal his anguish.

He humbly pardon crav’d for his offence past,\\
And vow’d himself a slave, and to love stedfast.\\
His pray’rs so ardent were, whilst his heart panted,\\
That Cupid lent an ear, and his suit granted.

For by his present plaint he was regarded,\\
And his adored saint his love rewarded.\\
And now they live in joy, sweetly embracing,\\
And left the little boy in the woods chasing.
\end{dcverse}

\musictitle{Have At Thy Coat, Old Woman.}

This tune is contained in every edition of \textit{The Dancing Master}, and in \textit{Musick’s
Delight on the Cithren}, 1666.

A copy of the ballad from which it derives the above name is in the Pepys
Collection, i. 284. It is—
\settowidth{\versewidth}{Who married a young man to her own undoing.}
\begin{scverse}
\vleftofline{“}A merry new song of a rich widow’s wooing,\\
Who married a young man to her own undoing.
\end{scverse}
To the tune of \textit{Stand thy ground, old Harry}.” It is a long ballad, in black-letter, 
“printed at London for T. Langley,” and commences thus:—
\settowidth{\versewidth}{Which makes me cry, with a love-sick sigh,}

\begin{dcverse}\begin{altverse}
“I am so sick for love,\\
As like was never no man, \\
Which makes me cry, with a love-sick sigh,\\
Have at thy coat, old woman.\\
\end{altverse}

Have at thy coat, old woman,\\
Have at thy coat, old woman,\\
Here and there, and everywhere,\\
Have at thy coat, old woman.”
\end{dcverse}


I have not found the ballad, \textit{Stand thy ground, old Harry}; but there is another
to the tune, under that name, in the same volume, i. 282—“A very pleasant
new ditty, to the tune of \textit{Stand thy ground, old Harry}; commencing, “Come,
hostess, fill the pot.” Printed at London for H. Gosson.

A song, commencing, “My name is honest Harry,” to the tune of \textit{Robin
Rowser}, which is in the same metre, is contained in \textit{Westminster Drollery}, 1671
and 1674; and in Dryden’s \textit{Miscellany Poems}, iv. 119. I imagine that \textit{Stand
thy ground, old Harry}, and \textit{My name is honest Harry}, are to the same tune,
although I cannot prove it. The words of the latter suit the air so exactly, that
I have here printed them with the music.

Whitlock, in his \textit{Zootomia; or Observations on the Present Manners of the
English}, 12mo., 1654, p. 45, commences his character of a female quack, with
the line, “\textit{And have at thy coat, old woman}.” In \textit{Vox Borealis}, 4to., 1641, we
find, “But all this sport was little to the court-ladies, who began to be very
melancholy for lack of company, till at last some young gentlemen revived an old
game, called \textit{Have at thy coat, old woman}.”

\changefontsize{1.05\defaultfontsize}
\pagebreak
%366

\musicinfo{Merrily.}{}
\smallskip

\includemusic{chappellV1197.pdf}

\settowidth{\versewidth}{Fresh and gay as the flowers in May,}
\begin{dcverse}\begin{altverse}
My love is blithe and buxom,\\
And sweet and fine as can be,\\
Fresh and gay as the flowers in May,\\
And looks like Jack-a-dandy.
\end{altverse}

\begin{altverse}
And if she will not have me.\\
That am so true a lover,\\
I’ll drink my wine, and ne’er repine,\\
And down the stairs I’ll shove her.
\end{altverse}

\begin{altverse}
But if that she will love me,\\
I’ll be as kind as may be;\\
I’ll give her rings and pretty things,\\
And deck her like a lady.
\end{altverse}

\begin{altverse}
Her petticoat of satin,\\
Her gown of crimson tabby,\\
Lac’d up before, and spangled o’er,\\
Just like a Bart’lemew baby.
\end{altverse}

\begin{altverse}
Her waistcoat shall be scarlet,\\
With ribbons tied together;\\
Her stockings of a Bow-dyed hue,\\
And her shoes of Spanish leather.
\end{altverse}

\begin{altverse}
Her smock o’ th’ finest holland,\\
And lac’d in every quarter;\\
Side and wide, and long enough,\\
To hang below her garter.
\end{altverse}

\begin{altverse}
Then to the church I’ll have her,\\
Where we will wed together;\\
And so come home when we have done,\\
In spite of wind and weather.
\end{altverse}

\begin{altverse}
The fiddlers shall attend us,\\
And first play \textit{John come kiss me};\\
And when that we have danc’d a round,\\
They shall play \textit{Hit or miss me}.
\end{altverse}

\begin{altverse}
Then hey for little Mary,\\
Tis she I love alone, sir;\\
Let any man do what he can,\\
I will have her or none, sir.
\end{altverse}
\end{dcverse}

\musictitle{A Health To Betty.}

This tune is contained in every edition of \textit{The Dancing Master}, and in \textit{Musick’s
Delight on the Cithren}.

D’Urfey prints “The Female Quarrel: Or a Lampoon upon Phillida and
Chloris, to the tune of a country dance, call’d \textit{A health to Betty},” \textit{Pills}
ii. 110, 1719.

In the Pepys Collection, i, 274, is a ballad—“Four-pence-half-penny-farthing;
or A woman will have the oddes;” signed M[artin] P[arker]. “Printed at
London for C. W. To the tune of \textit{Bessy Bell [she doth excell]}, or \textit{A health to
Betty}.” The first verse is here printed to the tune.

In the same Collection, ii. 372, is “The Northern Turtle:
\settowidth{\versewidth}{In being deprived of his sweet mate.}
\begin{scverse}
Wayling his unhappy fate,\\
In being deprived of his sweet mate.
\end{scverse}

\changefontsize{0.99\defaultfontsize}
\pagebreak
%367

To a new Northern tune, or \textit{A health to Betty}.” Printed at London for J. H.,
and beginning— “As I was walking all alone.”

In the Roxburghe Collection, i. 318, “The \textit{pair} of Northern Turtles—
\settowidth{\versewidth}{Deprived them both of life and breath.”}
\begin{scverse}
Whose love was firm, till cruel death\\
Deprived them both of life and breath.”
\end{scverse}

This is also “to a new Northern tune, or \textit{A health to Betty},” and commences—
\settowidth{\versewidth}{Farewell, farewell, my dearest dear}
\begin{scverse}
\begin{altverse}
\vleftofline{“}Farewell, farewell, my dearest dear,\\
All happiness wait on thee.”
\end{altverse}
\end{scverse}

\musicinfo{Gracefully.}{}

\includemusic{chappellV1198.pdf}

\backskip{1}

\musictitle{Shackley-Hay.}

The only copy I have found of this tune is in the Skene Manuscript, temp.
Charles~I.

It seems to derive its name from “A most excellent song of the love of young
Palmus and faire Sheldra, with their unfortunate love.” Copies of this, “to the
tune of \textit{Shackley-hay},” are in the Pepys Collection, i. 350; in the Roxburghe,
i.~436 and 472; the Bagford, fol. 75; and it is reprinted in Evans’ \textit{Old
Ballads}, i. 50.

In the Pepys Collection, i. 344, is a ballad of “Leander’s love to Hero. To
the tune of \textit{Shackley-hay}” beginning—
\settowidth{\versewidth}{“Two famous lovers once there was.”}
\begin{scverse}
“Two famous lovers once there was.”
\end{scverse}

In Westminster Drollery, 1671 and 1674, “A Song of the Declensions. The
tune is \textit{Shackle de hay},” and the same, with two others, in \textit{Grammatical Drollery},
by W.~H. (Captain Hicks), 1682.

In the Roxburghe Collection, ii. 244, and the Douce Collection, p. 109, is
“The Knitter’s Job: Or the earnest suitor of Walton town to a fair maid, with
her modest answers, and conclusion of their intents. To the tune of \textit{Shackley-hey}.”
It commences thus:—
\begin{dcverse}\indentpattern{0101005}
\begin{patverse}
“Within the town of Walton fair,\\
A lovely lass did dwell;\\
Both carding, spinning, knitting yarn,\\
She could do all full well.\\
This maid she many suitors had,\\
And some were good, and some were bad.\\
Fa, la la la la, \&c.
\end{patverse}
\end{dcverse}

\changefontsize{1.05\defaultfontsize}
\pagebreak%368

The Canaries (a dance “with sprightly fire and motion,” alluded to by
Shakespeare, and which, under that name, seems always to have had the same
tune) is called “The Canaries, or The \textit{Hay}” in \textit{Musick's Handmaid}, 1678.
The figure of \textit{The Hay} was also frequently danced in country-dances; but
\textit{Shackley-hay} is the name of a place in the ballad. It is very long—twenty-four
stanzas of eight lines—I have, therefore, selected nine from the first part. The
second recounts young Palmus’s going to sea in an open boat, through fair
Sheldra’s disdain; his being wrecked and drowned, and the sea-nymphs falling in
love with him.

\musicinfo{Smoothly.}{}

\includemusic{chappellV1199.pdf}

\indentpattern{01010000}
\settowidth{\versewidth}{But all in vain she did complain,}
\begin{dcverse}\begin{patverse}
But all in vain she did complain,\\
For nothing could him move,\\
Till wind did turn him back again,\\
And brought him to his love.\\
When she saw him thus turn’d by fate,\\
She turn’d her love to mortal hate;\\
Then weeping, to her he did say,\\
I’ll live with thee at Shackley-hay.
\end{patverse}

\begin{patverse}
No, no, quoth she, I thee deny,\\
My love thou once did scorn,\\
And my prayers wouldst not hear,\\
But left me here forlorn.\\
And now, being turn’d by fate of wind,\\
Thou thinkst to win me to thy mind;\\
Go, go, farewell! I thee deny,\\
Thou shalt not live at Shackley-hay.
\end{patverse}
\end{dcverse}
\changefontsize{1.01\defaultfontsize}
\pagebreak%369

\begin{dcverse}\begin{patverse}
If that thou dost my love disdain,\\
Because I live on seas;\\
Or that I am a ferry-man\\
My Sheldra doth displease,\\
I will no more in that estate\\
Be servile unto wind and fate,\\
But quite forsake boats, oars, and sea,\\
And live with thee at Shackley-hay.
\end{patverse}

\begin{patverse}
To strew my boat, for thy avail,\\
I’ll rob the flowery shores;\\
And whilst thou guid’st the silken sail,\\
I’ll row with silv’ry oars;\\
And as upon the streams we float,\\
A thousand swans shall guide our boat;\\
And to the shore still will I cry,\\
My Sheldra comes to Shackley-hay.
\end{patverse}

\begin{patverse}
And, walking lazily to the strand,\\
We’ll angle in the brook,\\
And fish with thy white lily hand,\\
Thou need’st no other hook;\\
To which the fish shall soon be brought,\\
And strive which shall the first be caught;\\
A thousand pleasures will we try,\\
As we do row to Shackley-hay.

\end{patverse}
\begin{patverse}
And if we be opprest with heat,\\
In mid-time of the day,\\
Under the willows tall and great\\
Shall be our quiet bay;\\
Where I will make thee fans of boughs,\\
From Phœbus’ beams to shade thy brows;\\
And cause them at the ferry cry,\\
A boat, a boat, to Shackley-hay!
\end{patverse}

\begin{patverse}
A troop of dainty neighbouring girls\\
Shall dance along the strand,\\
Upon the gravel all of pearls,\\
To wait when thou shalt land;\\
And cast themselves about thee round,\\
Whilst thou with garlands shalt be crown’d;\\
And all the shepherds with joy shall cry,\\
O Sheldra, come to Shackley-hay!
\end{patverse}

\begin{patverse}
Although I did myself absent,\\
’Twas but to try thy mind;\\
And now thou may’st thyself repent,\\
For being so unkind.—\\
No! now thou art turn’d by wind and fate,\\
Instead of love thou hast purchas’d hate,\\
Therefore return thee to the sea,\\
And bid farewell to Shackley-hay.
\end{patverse}
\end{dcverse}

\musictitle{Franklin Is Fled Away.}

Copies of this ballad are in the Pepys Collection, ii. 76; the Roxburghe,
ii. 348; the Bagford, 643, m. 10, p. 69; and the Douce, fol. 222.

In the same volume of the Bagford Collection, p. 139, is “The two faithful
Lovers. To the tune of \textit{Franklin is fled away};” commencing—
\settowidth{\versewidth}{Farewell, my heart’s delight,}
\indentpattern{0202}
\begin{dcverse}\begin{patverse}
\vleftofline{“}Farewell, my heart’s delight,\\
Ladies, adieu!
\end{patverse}

\begin{patverse}
I must now take my flight,\\
Whate’er ensue.”
\end{patverse}
\end{dcverse}

The tune is contained in \textit{Apollo’s Banquet for the Treble Violin}, 1669; in
\textit{180 Loyal Songs}, 1685 and 1694; and in \textit{ Pills to purge Melancholy}, iii. 208, 1707;
sometimes under the name of \textit{Franklin is fled away}, and at others as \textit{O hone,
O hone}, the burden of the ballad. This burden is derived from the Irish lamentation, 
to which there were many allusions in the sixteenth and seventeenth
centuries, as in Marston’s \textit{Eastward Hoe}, act v., sc. 1; or in Gayton’s \textit{Festivous
Notes upon Don Quixote}, 1654, p. 57,—“Who this night is to be rail’d upon by
the black-skins, in as lamentable noyse as the wild Irish make their \textit{O hones}.”
A different version of the tune will be found in the ballad opera of \textit{The Jovial
Crew}, 1731, under the name of \textit{You gallant ladies all}.

A variety of songs and ballads, which were sung to it, will be found in the
above-named collections of ballads; in the \textit{180 Loyal Songs}; in Patrick Carey’s
\textit{Trivial Poems}, 1651; and in \textit{ Pills to purge Melancholy}.

The tune is one of the many from which \textit{God save the King} has been said to be
derived.
\pagebreak%370
\changefontsize{1.05\defaultfontsize}

The title of the original ballad is “A mournful Caral: Or an Elegy lamenting
the tragical ends of two unfortunate faithful Lovers, Franklin and Cordeli\textit{us}; he
being slain, she slew herself with her dagger. To a new tune called \textit{Franklin is
fled away}.”

\musicinfo{Moderate time.}{}

\includemusic{chappellV1200.pdf}

\settowidth{\versewidth}{Franklin is fled and gone, O hone, O hone!}
\indentpattern{00220}
\begin{scverse}
\begin{patverse}
Franklin is fled and gone, O hone, O hone!\\
And left me here alone, O hone, O hone!\\
Franklin is fled away,\\
The glory of the May;\\
Who can but mourn and say, O hone, O hone!
\end{patverse}
\end{scverse}

There are six stanzas in the first, and eight in the second part. Black-letter.
Printed for M. Coles, W. Thackeray, \&c.

\musictitle{Queen Dido, Or Troy Town.}

“A ballett intituled \textit{The Wanderynge Prince}” was entered on the Registers
of the Stationers’ Company in 1564-5. This was, no doubt, the “Proper new
ballad, intituled \textit{The Wandering Prince of Troy}: to the tune of \textit{Queen Dido},” of
which there are two copies in the Pepys Collection (i. 84 and 548). Of these
copies, the first, being printed by John Wright, is probably not of earlier date
than 1620; and the second, by Clarke, Thackeray, and Passinger, after 1660.

The ballad has been reprinted in Percy’s \textit{Reliques of Ancient Poetry}, iii. 192,
1765; and in Ritson’s \textit{Ancient Songs}, ii. 141, 1829. Its extensive popularity
will be best shown by the following quotations:—“You ale-knights, you that
devour the marrow of the malt, and drink whole ale-tubs into consumptions; that
\pagebreak%371
sing \textit{Queen Dido} over a cup, and tell strange news over an ale-pot\ldots you shall be
awarded with this punishment, that the rot shall infect your purses, and eat out
the bottom before you are aware.”—\textit{The Penniless Parliament of threadbare
Poets}, 1608. (Percy Soc. reprint, p. 44.)
\settowidth{\versewidth}{With the old footman for singing \textit{Queen Dido}?”}
\begin{scverse}
\vleftofline{Frank.—“}These are your eyes!\\
Where were they, Clora, when you fell in love\\
With the old footman for singing \textit{Queen Dido}?”\\
\attribution Fletcher’s \textit{The Captain}, act iii., sc. 3.
\end{scverse}
Fletcher again mentions it in act i., sc. 2, of \textit{Bonduca}, where Petillius says of
Junius that he is “in love, indeed in love, most lamentably loving,—to the tune
of \textit{Queen Dido}.” At a later date, Sir Robert Howard (speaking of himself)
says, “In my younger time I have been delighted with a ballad for its sake;
and ’twas ten to one but my muse and I had so set up first: nay, I had almost
thought that \textit{Queen Dido}, sung that way, was some ornament to the pen of Virgil.
I was then a trifler with the lute and fiddle, and perhaps, being musical, might
have been willing that \textit{words} should have their tones, unisons, concords, and
diapasons, in order to a poetical gamuth.”—\textit{Poems and Essays}, 8vo., 1673.

A great number of ballads were sung to the tune, either under the name of
\textit{Queen Dido} or of \textit{Troy Town}. Of these I will only cite the following:—

“The most excellent History of the Duchess of Suffolk’s Calamity. To the
tune of \textit{Queen Dido};” commencing—
\settowidth{\versewidth}{That prudent prince, King Edward, away.”}
\begin{scverse}
\vleftofline{“}When God had taken for our sin\\
\vin That prudent prince, King Edward, away.”
\end{scverse}
Contained in \textit{Strange Histories, or Songes and Sonets}, \&c., 1607; in the \textit{Crown
Garland of Golden Roses}, 1659; in the Pepys Collection, i. 544; and reprinted
in Evans’ \textit{Old Ballads}, iii. 135.

“Of the Inconveniences by Marriage. To the tuno of \textit{When Troy towne};”
beginning—\qquad\qquad “Fond, wanton youth makes love a god.”

\noindent Contained in \textit{The Golden Garland of Princely Delights}, third edit., 1620; also
set to music by Robert Jones, and printed in his \textit{First Booke of Ayres}, fol., 1601.

“The lamentable song of the Lord Wigmore, Governor of Warwick Castle,
and the fayre Maid of Dunsmoore,” \&c.; beginning—
\settowidth{\versewidth}{In Warwickshire there stands a downe,}
\begin{scverse}
\vleftofline{“}In Warwickshire there stands a downe,\\
And Dunsmoore-heath it hath to name;”
\end{scverse}
which, in the \textit{Crown Garland of Golden Roses}, 1612, is to the tune of \textit{Diana [and
her darlings dear]}; but in the copy in the Bagford Collection is to the tune of
\textit{Troy Town}. (Reprinted by Evans, iii. 226.)

“The Spanish Tragedy: containing the lamentable murder of Horatio and
Belimperia; with the pitiful death of old Hieronimo. To the tune of \textit{Queen
Dido}; beginning—\qquad\qquad “You that have lost your former joys.”

\noindent Printed at the end of the play of \textit{The Spanish Tragedy}, in Dodsley’s \textit{Old Plays},
iii.~203, 1825; and by Evans, iii. 288.

“A Looking-glass for Ladies; or a Mirror for Married Women. Tune, \textit{Queen
Dido}, or \textit{Troy Town};” commencing—
\settowidth{\versewidth}{“When Greeks and Trojans fell at strife.”}
\begin{scverse}
“When Greeks and Trojans fell at strife.”
\end{scverse}
\pagebreak%372

Reprinted by Percy, under the name of \textit{Constant Penelope}, from a copy in the
Pepys Collection.

“The Pattern of True Love; or Bowes’ Tragedy,” written in 1717, and printed
in Ritson’s \textit{Yorkshire Garland}.

The last shows its popularity at a late period.

The only tune I can find for the ballad, \textit{The Wandering Prince of Troy}, is the
composition of Dr. Wilson. It is adopted in \textit{ Pills to purge Melancholy}, iii. 15,
1707, and iv. 266, 1719; and is the \textit{Troy Town} of the ballad-operas, such as
\textit{Polly}, 1729, \&c. The ballad was entered at Stationers’ Hall before Dr. Wilson
was born; therefore this cannot be the original tune,—unless he merely arranged
it for three voices, which we have no reason for supposing. It is printed in his
“\textit{Cheerful Ayres or Ballads}, first composed for one single voice, and since set
for three voices,” Oxford, 1660. Dr. Rimbault has recently identified Dr.
Wilson with the “Jack Wilson” who was a singer on the stage in Shakespeare’s
time. It is possible, therefore, that he may have sung the ballad on the stage,
according to the custom of those days. Wilson was created Doctor, at Oxford,
in 1644, and died in his seventy-ninth year, \ad 1673.

There is also a song of \textit{Queen Dido}, but, being in a different metre, it could
not be sung to the same air. (See Index.) In the following, I have adopted
Dr. Percy’s copy of the ballad, after the first stanza, which is printed with the
tune. It consists of twenty-three verses, of which eleven are subjoined; ending
with the first climax—Dido’s death.

\musicinfo{Moderate time.}{}

\includemusic{chappellV1201.pdf}

\pagebreak%373
\changefontsize{1.00\defaultfontsize}

\settowidth{\versewidth}{That thou, poor wandering prince, hast had.}
\indentpattern{010100}
\begin{dcverse}\begin{patverse}
Æneas, wandering prince of Troy,\\
When he for land long time had sought,\\
At length arriving with great joy,\\
To mighty Carthage walls was brought;\\
Where Dido queen, with sumptuous feast,\\
Did entertain that wandering guest.
\end{patverse}

\begin{patverse}
And, as in hall at meat they sate,\\
The queen, desirous news to hear,\\
Says, ‘Of thy Troy’s unhappy fate\\
Declare to me, thou Trojan dear:\\
The heavy hap and chance so bad,\\
That thou, poor wandering prince, hast had.
\end{patverse}

\begin{patverse}
And then anon this comely knight,\\
With words demure, as he could well,\\
Of his unhappy ten years’ fight,\\
So true a tale began to tell,\\
With words so sweet, and sighs so deep,\\
That oft he made them all to weep.
\end{patverse}

\begin{patverse}
And then a thousand sighs he fet,\\
And every sigh brought tears amain;\\
That where he sate the place was wet,\\
As though he had seen those wars again:\\
So that the queen, with ruth therefore,\\
Said, worthy prince, enough, no more.
\end{patverse}

\begin{patverse}
And then the darksome night drew on,\\
And twinkling stars the sky bespread;\\
When he his doleful tale had done,\\
And every one was laid in bed:\\
Where they full sweetly took their rest,\\
Save only Dido’s boiling breast.
\end{patverse}

\begin{patverse}
This silly woman never slept,\\
But in her chamber, all alone,\\
As one unhappy, always wept,\\
And to the walls she made her moan;\\
That she should still desire in vain\\
The thing she never must obtain.
\end{patverse}

\begin{patverse}
And thus in grief she spent the night,\\
Till twinkling stars the sky were fled,\\
And Phœbus, with his glistering light,\\
Through misty clouds appeared red;\\
Then tidings came to her anon,\\
That all the Trojan ships were gone.
\end{patverse}

\begin{patverse}
And then the queen, with bloody knife,\\
Did arm her heart as hard as stone,\\
Yet, something loth to loose her life.\\
In woful wise she made her moan;\\
And, rolling on her careful bed,\\
With sighs and sobs these words she said .
\end{patverse}

\begin{patverse}
O wretched Dido, queen! quoth she,\\
I see thy end approacheth near;\\
For he is fled away from thee,\\
Whom thou didst love and hold so dear:\\
What! is be gone, and passed by?\\
O heart, prepare thyself to die.
\end{patverse}

\begin{patverse}
Though reason says, thou shouldst forbear,\\
And stay thy hand from bloody stroke,\\
Yet fancy bids thee not to fear,\\
Which fetter’d thee in Cupid’s yoke.\\
Come death, quoth she, resolve thy smart!\\
And with those words she pierced her heart.
\end{patverse}
\end{dcverse}

\musictitle{Remember, O Thou Man.}

This Christmas Carol is the last of the “Country Pastimes” in “Melismata:
Musicall Phansies fitting the Court, Citie, and Countrey Humours,” edited by
Ravenscroft, 4to., 1611. It is paraphrased in “Ane compendious booke of
Godly and Spirituall Songs\ldots with sundrie\ldots ballates changed out of prophaine
Sanges,” \&c., printed by Andro Hart, in Edinburgh, in 1621.
\settowidth{\versewidth}{That I thy saull from Sathan wan,}
\begin{dcverse}\vleftofline{“}Remember, man, remember, man,\\
That I thy saull from Sathan wan,\\
\columnbreak
And hes done for thee what I can,\\
\vin Thow art full deir to me,” \&c.\\
\vleftofline{\textit{Scottish Poems of t}}\textit{he Sixteenth Century}, ii. 188, 1801.
\end{dcverse}

From \textit{Melismata} the carol was copied into Forbes’ \textit{Cantus}, and taught in the
Music School at Aberdeen. Some years ago, the latter work was sold for a
comparatively high price at public auctions in London (about 10\textit{l}.), and chiefly
on the reputation of containing, in this carol, the original of \textit{God save the King},
The report originated with Mr. Pinkerton, who asserted in his \textit{Recollections of
Paris}, ii. 4, that “the supposed national air is a mere transcript of a Scottish
Anthem” contained in a collection printed in 1682. Forbes’ \textit{Cantus} is comparatively
useless to a musician, since it contains \textit{only} the “cantus,” or treble voice
\pagebreak%374
part of English compositions, which were written, and should be, in three, four,
or five parts. There are, also, a few ballad tunes, such as “Satan, my foe,” to
\textit{Fortune, my foe}; “Shepherd, saw thou not,” to \textit{Crimson Velvet}, \&c.; and, in the
last edition, 1682, some Italian songs, and “new English Ayres,” in three parts
complete. The two former editions were printed at Aberdeen, in 1662 and 1666.
%\changefontsize{0.93\defaultfontsize}

\musicinfo{Moderate time.}{}

\includemusic{chappellV1202.pdf}

\settowidth{\versewidth}{Remember God’s goodness, how be sent his Son, doubtless,}
\begin{dcverse}\footnotesizerr
\begin{altverse}
Remember Adam’s fall, O thou man, \&c.,\\
Remember Adam’s fall, from heaven to hell;\\
Remember Adam’s fall, how we were condemned all\\
In hell perpetual there for to dwell.
\end{altverse}

\begin{altverse}
Remember God’s goodness, O thou man, \&c.,\\
Remember God’s goodness and his promise made;\\
Remember God’s goodness, how be sent his Son, doubtless,\\
Our sins for to redress;—Be not afraid.
\end{altverse}

\begin{altverse}
The angels all did sing, O thou man, \&c.,\\
The angels all did sing upon the shepherd’s hill;\\
The angels all did sing praises to our heavenly King,\\
And peace to man living, with a good will.
\end{altverse}

\begin{altverse}
The shepherds amazed were, O thou man, \&c.,\\
The shepherds amazed were, to hear the angels sing;\\
The shepherds amazed were, how it should  come to pass\\
That Christ, our Messias, should be our King.
\end{altverse}

\begin{altverse}
To Bethlem they did go, O thou man, \&c.,\\
To Bethlem they did go, the shepherds three;\\
To Bethlem they did go, to see wh’er it were so or no,\\
Whether Christ were born or no, to set man free.
\end{altverse}

\begin{altverse}
As the angels before did say, O thou man, \&c.,\\
As the angels before did say, so it came to pass;\\
As the angels before did say, they found a babe where it lay,\\
In a manger, wrapt in hay, so poor he was.
\end{altverse}

\begin{altverse}
In Bethlem he was born, O thou man, \&c.,\\
In Bethlem he was born for mankind’s sake;\\
In Bethlem he was born, for us that were forlorn,\\
And therefore took no scorn our flesh to take.
\end{altverse}

\begin{altverse}
Give thanks to God always, O thou man, \&c.,\\
Give thanks to God always with heart most joyfully;\\
Give thanks to God alway, for this our happy day—\\
Let all men sing and say, Holy, holy.
\end{altverse}
\end{dcverse}

\pagebreak%375

\changefontsize{1.05\defaultfontsize}

\musictitle{The Country Lass.}

This is the tune to which, with slight alteration, \textit{Sally in our Alley} is now sung.
Henry Carey, the author of that song, composed other music for it, which is
introduced four times in his \textit{Musical Century}. Carey’s tune is the \textit{Sally in our
Alley} of the ballad-operas that were printed from 1728 to 1760; but from the
latter period its popularity seems to have waned, and, at length, his music was
entirely superseded by this older ballad-tune.

\textit{The Countrey Lasse}, from which it derives its name, was to be sung to “a dainty
new note;” but, if unacquainted with that, the singer had the option of another
tune—\textit{The mother beguil’d the daughter}. In \textit{ Pills to purge Melancholy}, ii. 165,
1700 and 1707, it is printed (in an abbreviated form) to the one; and in \textit{The
Merry Musician, or a Cure for the Spleen},\footnote{\textit{}
The first volume of The Merry Musician is dated
1716; but the second, third, and fourth, being engraved,
not set up in type like the first, bear no dates.}
 iii. 9, to the other.

In \textit{The Devil to pay}, 8vo., 1731, where Carey’s tune is printed at p. 35, as
\textit{Charming Sally}, this will be found, as \textit{What tho’ I am a Country Lass}, at p. 50.
Being unfit for dancing, the air is not contained in \textit{The Dancing Master}.

I have quoted the full title of the ballad of \textit{The Country Lass} at p. 306. The
copy in the Roxburghe Collection, i. 52, being printed by the assigns of Thomas
Symcocke, would date in or after 1620, the year of that assignment. The copy in
the Pepys Collection, i. 268, is, perhaps, an original copy. It bears the initials
of Martin Parker, the famous ballad-writer, and is evidently more correctly
printed.

The versions in \textit{ Pills to purge Melancholy}, and in \textit{The Merry Musician}, have each
had “the rust of antiquity filed from them,” and, as usual, without any improvement. 
The two first stanzas are nearly the same as in the old ballad; but the
three remaining have been re-written. The older ballad is reprinted by Evans,
i. 41, from the Roxburghe copy.

The “a” at the end of each alternate line is a very old expedient of the
ballad-maker for fitting his words to music, when an extra syllable was required.
The reader may have observed it already in \textit{John Dory, Jog on the footpath way.
Good fellows must go learn to dance}, and others. The custom is thus reproved in
“\textit{A Discourse of English Poetrie}, by William Webbe, graduate,” 1586:—“If
I let passe the \textit{un-countable rabble of ryming ballet-makers}, and compylers of
sencelesse sonets (who be most busy to stuffe every stall full of grosse devises
and unlearned pamphlets), I trust I shall, with the best sort, be held excused.
For though many such can frame an alehouse song of five or six score verses,
hobbling uppon some tune of a \textit{Northern Jygge}, or \textit{Robyn Hoode}, or \textit{La
Lubber}, \&c.: and perhappes observe just number of sillables, eight in one
line, sixe in an other, and therewithall\textit{ an ‘a’ to make a jercke in the ende}: yet
if these might be accounted poets (as it is sayde some of them make meanes to
be promoted to the Lawrell), surely we shall shortly have whole swarmes of
poets; and every one that can frame a booke in ryme, though, for want of
matter, it be but in commendations of copper noses or bottle ale, wyll catch at \pagebreak
the garlande due to poets—whose \textit{potticall} (poeticall, I should say) \textit{heades},
%376
I would wyshe, at their worshipfull commencements, might, in steede of lawrell,
be gorgiously garnished with fayre greene \textit{barley}, in token of their good affection
to our Englishe malt.”
%\changefontsize{0.92\defaultfontsize}

The following verses are selected from the older copy of the ballad. In the
\textit{Pills}, and \textit{Merry Musician}, the burden, which requires the repetition of the first
part of the tune, is omitted:—

\musicinfo{Gracefully.}{}

\includemusic{chappellV1203.pdf}

\backskip{1.5}

\settowidth{\versewidth}{Close by a crystal fountain side,}
\begin{dcverse}\footnotesizer
\begin{altverse}
What, though I keep my father’s sheep,\\
A thing that must he done-a,\\
A garland of the fairest flow’rs\\
Shall shroud me from the sun-a;\\
And when I see them feeding by,\\
Where grass and flowers spring-a,\\
Close by a crystal fountain side,\\
I sit me down and sing-a.
\end{altverse}

\begin{altverse}
Dame Nature crowns us with delight\\
Surpassing court or city,\\
We pleasures take, from morn to night,\\
In sports and pastimes pretty:\\
Your city dames in coaches ride\\
Abroad for recreation.\\
We country lasses hate their pride,\\
And keep the country fashion.
\end{altverse}

\begin{altverse}
I care not for the fan or mask.\\
When Titan’s heat reflecteth,\\
A homely hat is all I ask,\\
Which well my face protecteth;\\
Yet am I, in my country guise,\\
Esteem’d a lass as pretty,\\
As those that every day devise\\
New shapes in court or city.
\end{altverse}

\begin{altverse}
Then do not scorn the country lass,\\
Though she go plain and meanly;\\
Who takes a country wench to wife\\
(That goeth neat and cleanly),\\
Is better sped, than if he wed\\
A fine one from the city,\\
For there they are so nicely bred,\\
They must not work for pity.
\end{altverse}
\end{dcverse}
\pagebreak%377
%\changefontsize{0.89\defaultfontsize}

\musictitle{Maying-Time.}

In \textit{The Golden Garland of Princely Delights}, 3rd edit., 1620, this is entitled
“The Shepherd’s Dialogue of Love between Willy and Cuddy: To the tune of
\textit{Maying-time}.” It is also in Dryden’s \textit{Miscellany Poems}, vi. 337, and in Percy’s
\textit{Reliques of Ancient Poetry}. Percy entitles it “The Willow Tree: a Pastoral
Dialogue.”

The tune is in a manuscript dated 1639, in the Advocates’ Library, Edinburgh;
in the Skene MS.; and in all the editions of Forbes’ \textit{Cantus}.

\includemusic{chappellV1204.pdf}

\backskip{1.1}

\begin{dcverse}\qquad\qquad\qquad \textsc{willy.}\\
Phillis! she that lov’d thee long?\\
Is she the lass hath done thee wrong?\\
She that lov’d thee long and best,\\
Is her love turned to a jest?

\qquad\qquad\qquad \textsc{cuddy}.\\
She that long true love profest,\\
She hath robb’d my heart of rest:\\
For she a new love loves, not me;\\
Which makes we wear the willow-tree.

\qquad\qquad\qquad \textsc{willy}.\\
Come then, shepherd, let us join,\\
Since thy hap is like to mine:\\
For the maid I thought most true\\
Me hath also bid adieu.

\qquad\qquad\qquad \textsc{cuddy}.\\
Thy hard hap doth mine appease,\\
Company doth sorrow ease:\\
Yet, Phillis, still I pine for thee,\\
And still must wear the willow-tree.

\qquad\qquad\qquad \textsc{willy}.\\
Shepherd, be advis’d by me,\\
Cast off grief and willow-tree:\\
For thy grief brings her content,\\
She is pleas’d if thou lament.

\qquad\qquad\qquad \textsc{cuddy}.\\
Herdsman, I’ll be rul’d by thee,\\
There lies grief and willow-tree:\\
Henceforth I will do as they,\\
And love a new love every day.
\end{dcverse}
\pagebreak%378
%\changefontsize{\defaultfontsize}

\musictitle{Never Love Thee More.}

This song, commencing, “My dear and only love, take heed,” is contained in a
manuscript volume of songs and ballads, with music, dated 1659, in the handwriting
of John Gamble, the composer. The MS. is now in the possession of
Dr. Rimbault.

Gamble published some of his own works in 1657 and 1659, but this seems to
have been his common-place book. It contains the songs Dr. Wilson composed
for Brome’s play, \textit{The Northern Lass}, and many compositions of H. and W.
Lawes, as well as common songs and ballads. The last are usually noted down
without bases; but, in some instances, the space intended for the tune is unfilled.

In the Pepys Collection, i. 256, is “The Faythfull Lover’s Resolution; being
forsaken of a coy and faythless dame. To the tune of \textit{My dear and only love, take
heed};” commencing, “Though booteles I must needs complain.” “Printed
at London for P. Birch.”

In the same volume, i. 280—“Good sir, you wrong your Britches;—pleasantly
discoursed by a witty youth and a wily wench. To the tune of \textit{O no, no, no, not
yet}, or \textit{Ile never love thee more};” commencing, “A young man and a lasse of
late.” “Printed at London for J[ohn] T[rundle].”

At p. 378—“Anything for a quiet life; or The Married Man’s Bondage,” \&c.
“To the tune of \textit{O no, no, no, not yet}, or \textit{Ile never love thee more}” Printed at
London by G. P.

And at p. 394—“’Tis not otherwise: Or The Praise of a Married Life. To
the tune of \textit{Ile never love thee more};” commencing, “A young man lately did
complaine.” Printed at London by G. B.

The above quotations tend to prove the tune to be of the time of James I.
Philip Birch, the publisher of the first ballad, had a “shop at the Guyldhall”
in 1618, when he published “Sir Walter Rauleigh his Lamentation,” to which
I have referred at p. 175. John Trundle, the publisher of the second, was dead
in 1628; the ballads were then printed by “M. T., widdow.” Trundle is
mentioned as a ballad-printer in Ben Jonson’s \textit{Every man in his humour}, 1598.

In the Roxburghe Collection, ii. 574, is “A proper new ballad, being the
regrate [regret] of a true Lover for his Mistris unkindness. To a new tune, \textit{Ile
ever love thee more}.” The rude orthography of this seems to mark it as an early
ballad; but, unfortunately, the printer’s name is cut away. It commences thus:
\settowidth{\versewidth}{Which could thy hands inshrine;}
\begin{dcverse}\begin{altverse}
\vleftofline{“}I wish I were those gloves, dear heart.\\
Which could thy hands inshrine;\\
Then should no sorrow, grief, or smart.\\
Molest this heart of mine,” \&c.;
\end{altverse}
\end{dcverse}
\noindent and consists of twenty-one stanzas of eight lines; thirteen in the first part, and
eight in the second.

In the same collection, and in Mr. Payne Collier’s \textit{Roxburghe Ballads}, p. 227, is
“The Tragedy of Hero and Leander. To \textit{a pleasant new tune}, or \textit{I will never love
thee more}.” The last was “printed for R. Burton, at the Horse-shoe in West-Smithfield,
neer the Hospital-gate;” and the copy would, therefore, date in the
reign of Charles I., or during the Commonwealth.
\pagebreak%379

James Graham, Marquis of Montrose, also wrote “Lines” to this tune,
retaining a part of the first line, and the burden of each verse, “\textit{I’ll never love
thee more}.” It is “An Address to his Mistress,” and commences—
\settowidth{\versewidth}{My dear and only love, I \textit{pray}}
\begin{scverse}
\begin{altverse}
\vleftofline{“}My dear and only love, I \textit{pray}\\
This noble world of thee, \&c.
\end{altverse}
\end{scverse}
Like “My dear and only love, \textit{take heed} it consists of five stanzas; and must
have been written after the establishment of the Committees and the Synod of
Divines at Westminster (1643), because he refers to both in the song.

Watson in his \textit{Collection of Scotch Poems}, part iii., 1711, printed one of the
extended versions of “My dear and only love, \textit{take heed},” as a “second part” to
the Marquis of Montrose’s song; but it cannot have been written by him, as he
was only born in 1612. Neither Ritson, Robert Chambers, nor Peter Cunningham,
have followed this error; but it has been reproduced in \textit{Memoirs of Montrose},
Edinburgh, 1819.

It was, no doubt, the Marquis of Montrose’s song that made the tune popular
in Scotland. It is found, under the name of \textit{Montrose Lyns}, in a manuscript of
lyra-viol music, dated 1695, recently in the possession of Mr. A. Blaikie. The
tune has, therefore, been included in collections of Scottish music; but “My dear
and only love, \textit{take heed}” continued to be the popular song in England, and from
that it derives its name. In English ballads it is called “A rare Northern
tune,”\footnote{\textit{}
In ballad-phrase, the terms “Northern” and ‘‘North-country”
were often applied to places within a hundred
miles of London. Percy describes the old ballad of \textit{Chevy
Chace} as written in “the coarsest and broadest Northern
dialect,” although Richard Sheale, the author of that version,
was a minstrel residing in Tamworth, and in the
service of the Earl of Derby. Puttenham thus notices
the difference of speech prevailing in his time beyond the
Trent:—“Our [writer] therefore at these days shall not
follow Piers Plowman, nor Gower, nor Lydgate, nor yet
Chaucer, for their language is now out of use with us:
neither shall he take the terms of North-men, such as they
use in dayly talke (whether they be noble men or gentlemen,
or of their best clarkes, all is a matter), nor in effect
any speach used beyond the river of Trent: though no man
can deny but theirs is the purer English Saxon at this day,
yet it is not so courtly nor so current as our Southerne
English is, no more is the far Western man’s speach:
ye shall therefore take the usuall speach of the Court,
and that of London and the shires lying about London,
within sixty miles, and not much above.” (\textit{Arte of
English Poesie}.) Many of the characters in plays of the
seventeenth century, such as Brome’s \textit{Northern Lass},
speak in a dialect which might often pass for Scotch with
those who are unacquainted with the language of the
time.}
 and I have never yet found that term applied to a Scotch air. Besides
Gamble’s manuscript, which contains both the words and air, the words will be
found in the first and second editions of \textit{Wit and Drollery}, 1656 and 1661,
(there entitled “A Song”); in \textit{ Pills to purge Melancholy}, 1700, 1707, and 1719.
The tune was first added to \textit{The Dancing Master} in 1686, and is contained in
every subsequent edition, in a form more appropriate to dancing than the
earlier copy.

Some of the ballads are of a later date than the Marquis of Montrose’s
song, such as “Teach me, Belissa, what to do:” to the tune of “\textit{My dear and
only love, take heed}” in \textit{Folly in print}, 1667; “A Dialogue between Tom and
Dick,” in \textit{Rats rhimed to death}, 1660; “The Swimming Lady,” in the Bagford,
others in Roxburghe and Pepys Collections; but I have already cited enough to
prove that it was a very popular air, and popular before the Marquis of Montrose’s
song can have been written.

A copy of the ballad, consisting of four verses in the first, and five in the
\pagebreak%380
second part, is contained in the Douce Collection, p. 102, entitled “\textit{Ile never love
thee more}: Being the Forsaken Lover’s Farewell to his fickle Mistress. To \textit{a rare
Northern tune}, or \textit{Ile never love thee more}.“It commences, “My dear and only
\textit{joy}, take heed;” and the second part, “Ile lock myself within a cell.” Having
been “Printed for W. Whitwood, at the Golden Lyon in Duck Lane,” this Copy
may be dated about 1670. It is also in the list of those printed by W.
Thackeray at the same period. The copies in \textit{Wit and Drollery}, and in Gamble’s
MS., consist only of five stanzas.

The following copy of the tune is taken from Gamble’s MS.; the words are the
first, second, and fourth stanzas, in the order in which they stand in \textit{Wit and
Drollery}; or first, third, and fourth, in the MS. All the old copies above cited
have verbal differences, as well as differences of arrangement.

\musicinfo{Rather slowly and smoothly.}{}

\includemusic{chappellV1205.pdf}


\pagebreak%381

\settowidth{\versewidth}{Nor smoothness of their language plot}
\begin{dcverse}\begin{altverse}
Let not their oaths, by volleys shot,\\
Make any breach at all,\\
Nor smoothness of their language plot\\
A way to scale the wall;\\
No balls of wild-fire-love consume\\
The shrine which I adore;\\
For, if such smoke about it fume,\\
I’ll never love thee more.
\end{altverse}

\begin{altverse}
Then if by fraud or by consent,\\
To ruin thou shouldst come,\\
I’ll sound no trumpet as of wont,\\
Nor march by beat of drum;\\
But fold my arms, like ensigns, up,\\
Thy falsehood to deplore,\\
And, after such a bitter cup,\\
I’ll never love thee more.
\end{altverse}
\end{dcverse}

\settowidth{\versewidth}{The Merchantman.}

The ballad of the \textit{Merchantman and the Fiddler’s Wife} is in the list of those
printed by Thackeray, in the reign of Charles II. It is also printed in \textit{Pills to
purge Melancholy}, iii. 153, 1707, to the following “pleasant Northern tune.”

It commences with the line, “It was a rich Merchantman,” and the ballad of
“George Barnwell” was to be sung to the tune of \textit{The rich Merchantman}. (See
Roxburghe Collection, iii. 26.) Percy prints it from another copy in the Ashmole
Collection, where the tune is entitled “\textit{The Merchant}.”

There must either be another tune called \textit{A rich Merchantman}, or else only
half the air is printed in \textit{ Pills to purge Melancholy}; for, although eight bars of
music suffice for the above-named, which are in short stanzas of four lines,
sixteen, at least, are required for other ballads, which are in stanzas of eight,
and have occasionally a burden of four more. It is not unusual to find only
the half of a tune printed in the \textit{Pills} (see, for instance, \textit{Tom of Bedlam, Green
Sleeves}, \&c.), but I know of no other version of this tune, and therefore have not
the means of testing it.

“A song of the strange Lives of two young Princes of England, who became
shepherds on Salisbury Plain, and were afterwards restored to their former
estates: To the tune of \textit{The Merchant Man}”—As contained in \textit{The Golden Garland
of Princely Delights}, 3rd edit., 1620, as well as in \textit{Old Ballads}, 2nd edit.,
iii. 5, 1738. It is in stanzas of eight lines (commencing, “In kingly Stephen’s
reign”), and reprinted, omitting the name of the tune, in Evans’ \textit{Old Ballads}
ii. 53, 1810.

“A most sweet song of an English Merchant, born at Chichester: To an
excellent new tune”—has the additional burden of four lines, and is probably
the earliest. It commences thus:—
\settowidth{\versewidth}{Did kill a man at Embden towne}
\indentpattern{010110102323}
\begin{dcverse}\begin{patverse}
\vleftofline{“}A rich merchant man there was,\\
That was both grave and wise,\\
Did kill a man at Embden towne\\
Through quarrels that did rise.\\
Through quarrels that did rise,\\
The German he was dead,\\
And for this fact the merchant man\\
Was judg’d to lose his head.\\
\textit{A sweet thing is love,\\
It rules both heart and mind,\\
There is no comfort in this world\\
‘Like’ women that are kind.”}
\end{patverse}
\end{dcverse}

Of this various copies are extant, and all apparently very corrupt. One in the
Roxburghe Collection, i. 104, is “Printed at London for Francis Coules;”
a second, in the Bagford Collection, printed for A. P.; a third, in the Pepys
Collection, by Clarke, Thackeray, and Passinger. Evans reprints from the last.
\pagebreak%382
(See \textit{Old Ballads} i.~28, 1810.) It is parodied in act iii. of Rowley’s comedy,
\textit{A new Wonder, a Woman never vext}, 1632; and quoted in\textit{ The Triumphant
Widow}, 1677:—
\settowidth{\versewidth}{There was a rich merchant man,}
\begin{dcverse}\begin{altverse}
\vleftofline{“}There was a rich merchant man,\\
That was both great and wise,\\
He kill’d a man in Athens town,\\
Great quarrels there did arise,” \&c.
\end{altverse}
\end{dcverse}


\textit{A rich Merchantman} is one of the tunes to a song in \textit{The Famous Historie of
Fryer Bacon}, \textsc{b.l.}, 4to, n.d.; and \textit{There was a rich Merchantman} to a ballad in
the Pepys Collection, ii. 190. Others (under the one name or the other) will
be found in the Roxburghe Collection, i. 286 and 444, ii. 242, \&c.

\includemusic{chappellV1206.pdf}

\musictitle{Fair Margaret And Sweet William.}

Copies of this ballad are in the Douce Collection, fol. 72, and in the Collection
of Mr. George Daniel; also in Percy’s \textit{Reliques of Ancient Poetry}.

Percy says, “This seems to be the old song quoted in Fletcher’s \textit{Knight of the
Burning Pestle}, acts ii. and iii.; although the six lines there preserved are somewhat
different from those in the ballad as it stands at present. The lines preserved
in the play are this distich:—
\settowidth{\versewidth}{You are no love for me, Margaret,}
\begin{scverse}
\vleftofline{“}You are no love for me, Margaret,\\
I am no love for you;”
\end{scverse}
and the following stanza:—
\settowidth{\versewidth}{In came Margaret’s grimly ghost,}
\begin{dcverse}\begin{altverse}
\vleftofline{“}When all was grown to dark midnight,\\
And all were fast asleep,\\
In came Margaret’s grimly ghost,\\
And stood at William’s feet.”
\end{altverse}
\end{dcverse}

Percy adds that “these lines have acquired an importance by giving birth to one
of the most beautiful ballads in our own or any other language”—“Margaret’s
Ghost” by Mallet.

Mallet’s ballad attained deserved popularity. It was printed in various forms
on half-sheets with music, and in Watts’ \textit{Musical Miscellany}, ii. 84, 1729. The
air became known by its name, and is so published in \textit{The Village Opera}, 1729,
and in \textit{The Devil to pay}, 1731.

It was not, however, printed exclusively to this tune. \pagebreak Thomson published it
%383
in his \textit{Orpheus Caledonius}, and described it, with his usual inaccuracy, as “an
old Scotch ballad, with the original Scotch tune;”—“old,” although (on the
authority of Dr. Johnson) it was first printed in Aaron Hill’s \textit{Plain Dealer},
No. 36, July 24, 1724, and Thomson’s \textit{Orpheus} was published within six months
of that time—viz., on January 5,1725. The “original Scotch tune” of Thomson
is a version of “Montrose’s lines,” or \textit{Never love thee more}.

Another point deserving notice in the old ballad, is that one part of it has
furnished the principal subject of the modern burlesque ballad, “Lord Lovel,” and
another that of T. Hood’s song, “Mary’s Ghost.”

The copy in the Douce Collection is entitled “Fair Margaret’s Misfortune; or
Sweet William’s frightful dreams on his wedding night: With the sudden death
and burial of those noble lovers. To an excellent new tune.”

The following version of the words is from \textit{Percy’s Reliques of Ancient
Poetry}:—

\musicinfo{With expression.}{}
\bigskip

\includemusic{chappellV1207.pdf}

\settowidth{\versewidth}{There she spied sweet William and his bride,}
\begin{dcverse}\begin{altverse}
I see no harm by you, Margaret,\\
And you see none by me;\\
Before to-morrow at eight o’ the clock\\
A rich wedding you shall see.
\end{altverse}

\begin{altverse}
Fair Margaret sat in her bower-window,\\
Combing her yellow hair;\\
There she spied sweet William and his bride,\\
As they were a riding near.
\end{altverse}

\begin{altverse}
Then down she laid her ivory comb,\\
And braided her hair in twain;\\
She went alive out of her bower,\\
But ne'er came alive in’t again.
\end{altverse}

\begin{altverse}
When day was gone, and night was come,\\
And all men fast asleep,\\
Then came the spirit of fair Marg’ret,\\
And stood at William’s feet.
\end{altverse}
\end{dcverse}
\pagebreak%384

\begin{dcverse}\begin{altverse}
Are you awake, sweet William? she said,\\
Or, sweet William, are you asleep?\\
God give you joy of your gay bride-bed,\\
And me of my winding-sheet.
\end{altverse}

\begin{altverse}
When day was come, and night was gone,\\
And all men wak’d from sleep,\\
Sweet William to his lady said,\\
My dear, I have cause to weep,
\end{altverse}

\begin{altverse}
I dreamt a dream, my dear lady,\\
Such dreams are never good;\\
I dreamt my bower was full of red wine,\\
And my bride-bed full of blood.
\end{altverse}

\begin{altverse}
Such dreams, such dreams, my honoured Sir,\\
They never do prove good;\\
To dream thy bower was full of red wine,\\
And thy bride-bed full of blood.
\end{altverse}

\begin{altverse}
He called up his merry men all,\\
By one, by two, and by three;\\
Saying, I’ll away to fair Marg’ret’s bower,\\
By leave of my lady.
\end{altverse}

\begin{altverse}
And when he came to fair Marg’ret’s bower,\\
He knocked at the ring;\\
And who so ready as her seven brethren.\\
To let sweet William in.
\end{altverse}

\begin{altverse}
Then he turned up the covering sheet,\\
Pray let me see the dead;\\
Methinks she looks all pale and wan,\\
She hath lost her cherry red.
\end{altverse}

\begin{altverse}
I'll do more for thee, Margaret,\\
Than any of thy kin;\\
For I will kiss thy pale wan lips,\\
Though a smile I cannot win.
\end{altverse}

\begin{altverse}
With that be spake the seven brethren,\\
Making most piteous moan:\\
You may go kiss your jolly brown bride,\\
And let our sister alone.
\end{altverse}

\begin{altverse}
If I do kiss my jolly brown bride,\\
I do but what is right;\\
I ne’er made a vow to yonder poor corpse\\
By day, nor yet by night.
\end{altverse}

\begin{altverse}
Deal on, deal on, my merry men all,\\
Deal on your cake and your wine;\\
For whatever is dealt at her funeral to-day,\\
Shall be dealt to-morrow at mine.
\end{altverse}

\begin{altverse}
Fair Margaret died to-day, to-day,\\
Sweet William died the morrow;\\
Fair Margaret died for pure true love,\\
Sweet William died for sorrow.
\end{altverse}

\begin{altverse}
Margaret was buried in the lower chancel,\\
And William in the higher;\\
Out of her breast there sprang a rose,\\
And out of his a brier.
\end{altverse}

\begin{altverse}
They grew till they grew unto the church-top,\\
And then they could grow no higher;\\
And there they tied in a true lover’s knot,\\
Which made all the people admire.
\end{altverse}

\begin{altverse}
Then came the clerk of the parish,\\
As you the truth shall hear,\\
And by misfortune cut them down,\\
Or they now had been there.
\end{altverse}
\end{dcverse}
\vfill
\center\textsc{end of volume the first.}
\vfill

%\newpage \begingroup \parindent 0pt \parskip 2ex \def\enotesize{\normalsize} \theendnotes \endgroup
