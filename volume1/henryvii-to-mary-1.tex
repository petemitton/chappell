%048
\thispagestyle{empty}
\changefontsize{1.09\defaultfontsize}
\headingthree{POPULAR MUSIC,
SONGS AND BALLADS,}

\headingfour{REIGNS OF HENRY VII., HENRY VIII., EDWARD VI.,
AND MARY.}

\centerrule

Little occurs about music and ballads during the short reigns of Edward V. and
Richard III.

Richard was very liberal to his musicians, giving annuities to some, and
gratuities to others. (See Harl. MS., No. 433.) But his chief anxiety seems to
have been to increase the already splendid choral establishment of the Chapel
Royal. For that purpose he empowered John Melynek, one of the gentlemen of
the chapel, “to take and seize for the king” not only children, but also “all
such singing men expert in the science of music, as he could find and think able
to do the king’s service, within all places of the realm, as well cathedral churches,
colleges, chapels, houses of religion, and all other \textit{franchised or exempt places}, or
elsewhere.” (Harl. MS., 433, p. 189.) But it is not my object to pursue the
subject of royal establishments further.

In the privy purse expenses of Henry VII., from the seventh to the twentieth
year of his reign, there are many payments relating to music and to popular
sports, from which the following are selected:—
\bigskip

\noindent\footnotesize
\begin{tabular}{lllrrr}
1492.&Feb. 4th,& To the childe that playeth on the records\\
&&\dent [recorder]\dotfill &£1&0&0\\
&April 6th,&To Gwyllim for flotes [flutes] with a case\dotfill&3&10&0\\
&May 8th,&For making a case for the kinges suerde, and a\\
&&\dent case for James Hide’s harp\dotfill&1&0&8\\
&July 8th,&To the maydens of Lambeth for a May\dotfill&0&10&0\\
&August 1st,&At Canterbury, To the children, for singing in	 the \\
&&\dent gardyn\dotfill&0&3&4\\
1493.&Jan 1st,&To the Queresters [choristers], at Paule’s and	\\
&&\dent St. Steven\dotfill&0&13&4\\
&Jan. 6th,&To Newark [William Newark, the composer] for\\
&&\dent making a song\dotfill&1&0&0\\
&Nov. 12th,&To one Cornysshe for a prophecy, in rewards\dotfill&0&13&4
\end{tabular}
\normalsize
\medskip

Probably William Cornish, jun., composer, who belonged to the king’s chapel,
and was the author of a poem, called “A Treatise between Trouth and Informacion.”
He was paid 13\textit{s}. 4\textit{d}. on Christmas day, 1502, for setting a carol.
\normalsize
\pagebreak
%49
\markright{henry vii.}

\noindent\footnotesize
\begin{tabular}{lllrrr}
           &Nov. 30th,&Delivered to a merchaunt,  for a pair\footnotemark of\\
           &&\dent Organnes\dotfill&30&0&0\\
&Dec. 1st,&To Basset, riding for th’ organ pleyer of\\
&&\dent Lichefelde\dotfill&0&13&4\\
1494.&Jan. 2,&For playing of the Mourice [Morris] Daunce\dotfill&2&0&0\\
&Nov. 29th,&To Burton, for making a Masse\dotfill&1&0&0\\
&\dent ”& To my Lorde Prince’s Luter, in rewarde\dotfill&1&0&0\\
1495.&Aug 2nd,&To the women that songe before the king and\\
&&\dent the quene, in rewarde\dotfill&0&6&8\\
&Nov. 2nd,&	To a woman that singeth with a fidell\dotfill&0&2&0\\
&Nov. 27th,&To Hampton of Wourcestre, for making of\\
&&\dent	Balades, in rewarde\dotfill&1&0&0\\
1496.&April 25th,&To Hugh Denes, for a lute\dotfill&0&13&4\\
&June 25th,&To Frensheman, player of the organes\dotfill&0&6&8\\
&Aug. 5th,&To a Preste that wrestelled at Ceceter\dotfill&0&6&8\\
&Aug. 17th,&To the quene’s fideler, in rewarde\dotfill&1&6&8\\
1499.&June 6th,&To the May-game at Greenwich\dotfill&0&4&0\\
1501.&May 21st,&For a lute for my lady Margaret [the king’s\\
&&\dent eldest daughter, then about twelve years\\
&&\dent  old, afterwards Queen of Scots]\dotfill&0&13&4\\
&Sept. 30th,&To theym that daunced the mer’[morris] daunce\dotfill&1&6&8\\
&Dec. 4th,&To the Princesse stryng mynstrels at \\
&&\dent Westminster\dotfill&2&0&0\\
1502.&Jan. 7th,&To one that sett the king’s cleyvecordes\dotfill&0&10&4\\
&Feb. 4th,&To one Lewes, for a morris daunee\dotfill&1&13&4\\
1504.&March 6th,&For a pair of Clavycordes\dotfill&0&13&0\\
&\dent”&To John Sudborough, for a songe\dotfill&1&0&0\\
1505.&July 25th,&To the gentylmen of the kinges chapell, for to\\
&&\dent drinke with a bucke\dotfill&2&0&0\\
&Aug. 1st,&For a lute for my Lady Mary\dotfill&0&13&4\\
\end{tabular}
\bigskip
\normalsize
\footnotetext{
A \textit{pair} of organs, means a \textit{set} of organs, \ie, an organ.
A pack of cards was formerly called a pair of cards, and
we still say, “a \textit{pair} of steps”—“up two \textit{pair} of stairs.”
                        }

There is also a great variety of payments to the musicians of different towns,
as the “"Waytes” of Dover, Canterbury, Dartford, Coventry, and Northampton;
the minstrels of Sandwich, the shawms of Maidstone; to bagpipers, the king’s
piper (repeatedly), the piper at Huntingdon, \&c.; to harpers, some of whom were
Welsh. And there are also several entries “To a Walsheman for a ryme;”
liberal presents to the poets, of his mother (the Countess of Richmond), of the
prince, and of the king; to “the rymer of Scotland,” who was in all probability the
Scotch poet, William Dunbar, who celebrated the nuptials of James IV. and the
princess Margaret, in his “Thistle and the Rose,” and to an Italian poet. All
these may be seen in Excerpta Historica (8vo., 1833), and, as the editor
remarks:—“To judge from the long catalogue of musicians and musical instruments, 
flutes, recorders, trumpets, sackbuts, harps, shalmes, bagpipes, organs, and
round organs, clavicords, lutes, horns, pipers, fiddlers, singers, and dancers, Henry’s
love of music must have been great, which is further established by the fact, that
in every town he entered, as well as on board the ship which conveyed him to
Calais, he was attended by minstrels and waits.”
\normalsize
\markboth{henry viii.}{venetian ambassadors---erasmus.} 
\pagebreak
%50

A manuscript, containing a large number of songs and carols, bas been recently
found in the library of Balliol Coll., Oxford, where it bad been accidently concealed,
behind a book-case, during a great number of years. It is in the handwriting
of Richard Hill, merchant of London, and contains entries from the year
1483 to 1535. Six or eight of the songs and carols are the same as in the book
printed by the Percy Society, to which I have referred at page 41, and especially
the carol, “Nowell Nowell,” but the volume does not contain music. The song
of the contention between Holly and Ivy, beginning “Holly beareth berries, berries
red enough,” which is printed in Ritson’s Ancient Songs, from a manuscript
of Henry the Sixth’s time, is there also, proving that some of the songs are
of a much earlier date than the manuscript, and that they were still in favor. At
fol. 210, v. is a copy of the “Nut-browne Mayde,” and at the end of it “Explicit
quod, Rich. Hill,” which was the usual mode of claiming authorship of a work.

In the Pepysian Library, Magdalene College, Cambridge, there is a manuscript
book of vocal music (No. 87), containing the compositions of the most eminent
masters, English and foreign, of the time of Henry VII., written for the then
Prince of Wales. It was the Prince’s book, is beautifully written on vellum, and
illuminated with his figure in miniature.

Henry VIII. was not only a great patron of music, but also a composer; and,
according to Lord Herbert of Cherbury, who wrote his life, he composed two
complete services, which were often sung in his chapel. Hollinshed, in speaking
of the removal of the court to Windsor, when Henry was beginning his progress,
tells us that he “exercised himselfe dailie in shooting, singing, dansing, wressling,
casting of the barre, plaieing at the recorders, flute, virginals, in setting of songs,
and making of ballades.” All accounts agree in describing him as an amiable and
accomplished prince in the early part of his reign; and the character given of him
to the Doge of Venice, by his three ambassadors at the English court, could
scarcely be expressed in more favorable terms. In their joint despatch of
May 3rd, 1515, they say: “He is so gifted and adorned with mental accomplishments
of every sort, that we believe him to have few equals in the world. He
speaks English, French, and Latin; understands Italian well; plays almost
on every instrument, and composes fairly (delegnamente); is prudent and sage,
and free from every vice.”\dcfootnote{
 Despatch written by Pasqualigo, Badoer, and Giustinian
conjointly. See four years at the Court of Henry
VIII., Selection of Despatches addressed to the Signory
of Venice, from January, 1515, to July 26, 1519. Translated
by Rawdon Brown. 8vo., 1854. vol. i., p. 76.
}

In the letter of Sagudino (Secretary to the Embassy), writen to Alvise Foscari,
at this same date, he says: “He is courageous, an excellent musician, plays the
virginals well, is learned for his age and station, and has many other endowments
and good parts.” On the 1st of May, 1515, after the celebration of May-day at
Greenwich, the ambassadors dined at the palace, and after dinner were taken into
certain chambers containing a number of organs, virginals (clavicimbani), flutes,
and other instruments; and having heard from the ambassadors that Sagudino 
was a proficient on some of them, he \pagebreak 
was asked by the nobles to play, which
%51
he did for a long while, both on the virginals and organ, and says that he bore
himself bravely, and was listened to with great attention. The prelates told him
that the king would certainly wish to hear him, for he practised on these instruments
day and night.

\changefontsize{1.05\defaultfontsize}
Pasqualigo, the ambassador-extraordinary, gives a similar account at the same
time. Of Henry he says: “He speaks French, English, and Latin, and a little
Italian, plays well on the lute and virginals, sings from book \textit{at sight}, draws the
bow with greater strength than any man in England, and jousts marvellously.
Believe me he is in every respect a most accomplished prince; and I, who have
now seen all the sovereigns in Christendom, and last of all these two of France
and England, might well rest content,” \&c. Of the chapel service, Pasqualigo
says: “We attended High Mass, which was chaunted by the bishop of Durham,
with a superb and noble descant choir”\dcfootnote{
Descant choir is not a proper term, because the Music
of the King’s Chapel was not extempore descant, but in
written counterpoint of four parts. Several of the manuscripts
in use about this period, are preserved in the
King’s Library, British Museum, and some were Henry’s
own books. They are beautifully written manuscripts
on parchment, bearing the King’s arms. In one a Canon
in eight parts is inserted on the words “Honi soit qui
mal y pense.” The references to these manuscripts
will be found in Mr. Oliphant’s Catalogue of Musical
MSS., British Museum, towards the commencement.
See Nos. 12, 13, 21, \&c.
} (Capella di Discanto); and Sagudino
says: “High Mass was chaunted, and it was sung by his majesty’s choristers,
whose voices are really rather divine than human; they did not chaunt, but sung
like angels (non cantavano, ma jubilavano); and as for the deep bass voices,
I don’t think they have their equals in the world.”\dcfootnote{
The florid character of the counterpoint in use in
churches in those days is slyly reproved in a dialogue between
Humanity and Ignorance, in the Interlude of \textit{The
Four Elements}, printed about 1510. (Prick-song meant
harmony written or pricked down, in opposition to plain-song, 
where the descant rested with the will of the singer.)

\settowidth{\versewidth}{Peace, man, prick-song may not be desp}
\begin{fnverse}\scriptsize
\vleftofline{ \textit{Hu}. -- “}Peace, man, prick-song may not be despised,\\
For therewith God is well pleased,\\
Honoured, praised, and served\\
In the Church oft-times among.”

\vleftofline{\textit{Ig}. —“}Is God well pleased, trow’st thou, thereby?\\
Nay, nay! for there is no reason why:\\
For is it not as good to say plainly\\
‘Give me a spade’\\
As ‘giveme a spa-ve-va, ve-va-ve-vade?’\\
But if thou wilt have a song that is good,\\
I have one of Robin Hood,” \&c.
\end{fnverse}\vspace{-1.5\baselineskip}} (Vol. i., p. 77.)

Upon these despatches the editor remarks: “As Pasqualigo had been ambassador
at the courts of Spain, Portugal, Hungary, France, and of the Emperor, he was
enabled to form comparisons between the state of the science in those kingdoms
and our own; and, indeed, it is the universal \textit{experience} of the Venetian ambassadors, 
and their peculiar freedom from prejudice or partiality (no jealousy or
rivalry existing between them and England), that makes their comments on our
country so valuable.” (Vol. 1, p. 89.)

Erasmus, speaking of the English, said that they challenge the prerogative of
having the most handsome women, of keeping the best tables, and of being most
accomplished in the skill of music of any people;\dcfootnote{
 “Britanni, præter alia, formam, musicam, et lautas
mensas propriè sibi vindicent.” — \textit{Erasmus Enconium
Moriæ}.
} and it is certain that the beginning
of the sixteenth century produced in England a race of musicians equal to
the best in foreign countries, and in point of \textit{secular} music decidedly in advance
of them. When Thomas Cromwell, afterwards Earl of Essex, went from Antwerp
to Rome, in 1510, to obtain from Pope Julius II. the renewal of the “greater and
lesser pardon”\dcfootnote{
These pardons, says Foxe, gave them the power to
receive full remission, “apæna et culpa;” also pardon
for souls in purgatory, on payment of 6\textit{s}. 8\textit{d}. for the first
year, and 12\textit{d}. for every year after, to the Church of St.
Botolph’s, Boston.
}
for the town of Boston, for the \pagebreak 
maintenance of their decayed port,
%52
“being loth,” says Foxe, “to spend much time, and more loth to spend his money,
among the greedy cormorants of the Pope’s court,” he devised to meet him on his
return from hunting; and “having knowledge how the Pope’s holy tooth greatly
delighted in new-fangled strange delicates and dainty dishes, it came into his
mind to prepare certain fine dishes of jelly, made after our country manner here
in England; which to them of Rome was not known nor seen before. This done,
Cromwell observing his time accordingly, as the Pope was newly come from
hunting into his pavilion, he, with his companions, approached with his English
presents, brought in with a \textit{three-man's song} (as we call it) in the English tongue,
and all after the English fashion. The Pope suddenly marvelling at the strangeness
of the song, and understanding that they were Englishmen, and that they
came not empty-handed, willed them to be called in; and seeing the strangeness of
the dishes, commanded by and by his Cardinal to make the assay; who in tasting
thereof, liked it so well, and so likewise the Pope after him, that knowing of them
what their suits were, and requiring them to make known the making of that meat,
he, incontinent, without any more ado, stamped both their pardons, as well the
greater as the lesser.” (Acts and Monuments.) The introduction of these songs
into Italy is also mentioned by Michael Drayton in his Legend of Thomas
Cromwell, Earl of Essex, which was first printed in quarto in 1609.

\settowidth{\versewidth}{Which won much licence for my countrymen.}
\begin{scverse}
“Not long it was ere Rome of me did ring.\\
Hardly shall Rome such full days see again;\\
Of \textit{Freemen's Catches} to the Pope I sing,\\
Which won much licence for my countrymen.\\
Thither the which I was the first did bring,\\
That were unknown in Italy till then,” \&c.
\end{scverse}

In the Life of Sir Peter Carew, by John Vowell, alias Hoker, of Exeter
(Archæo\-logia, vol. 28), Freemen’s Songs are again mentioned. “From this time
he (Sir Peter) continued for the most part in the court, spending his time in
all courtly exercises, to his great praise and commendation, and especially to the
good liking of the king (Henry VIII.), who had a great pleasure in him, as
well for his sundry noble qualities, as also for his singing. For the king himself
being much delighted to sing, and Sir Peter Carew having a pleasant voice, the
king would often use him to sing with him certain songs they call \textit{Freemen Songs},
as namely, ‘By the bancke as I lay,’ and ‘As I walked the wode so wylde,’” \&c.

To sing at sight was so usual an accomplishment of gentlemen in those days,
that to be deficient in that respect was considered a serious drawback to success in
life. Skelton, in his \textit{Bowge at Court}, introduces Harvy Hafter as one who cannot
sing “on the booke,” but he thus expresses his desire to learn:—
\settowidth{\versewidth}{“Wolde to God it wolde please you some day,}
\begin{scverse}
“Wolde to God it wolde please you some day,\\
A balade boke before me for to laye,\\
And lerne me for to synge \textit{re, mi, fa, sol},\\
And when I fayle, bobbe me on the noll.”\\
\hfill\textit{Skelton’s Works, Ed. Dyce}, vol. i., p. 40.\hspace*{4em}
\end{scverse}
\markboth{the english love of songs and ballads.}{henry viii.}
\pagebreak
%53

Barklay, in his fourth Eclogue, (about 1514) says—
\begin{scverse}
“When your fat dishes smoke hot upon your table,\\
Then laude ye songs, and ballades magnifie;\\
If they be merry, or written craftely,\\
Ye clap your handes and to the making harke,\\
And one say to another, Lo, here a proper warke!”
\end{scverse}
The interlude of “The Four Elements” was printed by Rastall about 1510;
and, in that, Sensual Appetite, one of the characters, recommends Humanity “to
comfort his lyf naturall” with “daunsing, laughyng, or plesaunt songe,” and
says—
\begin{scverse}
“Make room, sirs, and let us be merry,\\
With huff a galand, syng Tyrll on the berry,\\
\vin And let the wide world wynde.\\
Sing Frisk a jolly, with Hey trolly lolly,\\
For I see it is but folly for to have a sad mind.”\\
\hfill \textit{Percy Soc}., No. 74.\hspace*{4em}
\end{scverse}
“Hey, ho, frisca jolly, under the greenwood tree,” is the burden of one of the
songs in the musical volume of the reign of Henry VIII. (MS. Reg. Append. 58.)
from which I have extracted several specimens. It contains, also, some instrumental
pieces, such as “My Lady Carey’s Dompe,” and “My Lady Wynkfield’s
Rownde,” which when well played on the virginals, as recently, by an able lecturer,
are very effective and musical.

Some of Henry the Eighth’s own compositions are still extant. In a collection
of anthems, motets, and other church offices, in the handwriting of John Baldwin,
of Windsor, (who also transcribed that beautiful manuscript, Lady Neville’s
Virginal Book, in 1591), is a composition for three voices, “Quam pulchra es, et
quam decora.” It bears the name Henricus Octavus at the beginning, and “quod
Henricus Octavus” at the end of the cantus part. The anthem “O Lord, the
maker of all things,” which is attributed to him in Boyce’s Cathedral Music, is
the composition of William Mundy; the words only are taken from Henry the
Eighth’s primer. Some music for a mask, which Stafford Smith attributes to
him, will be found in the Arundel Collection of MS. (Brit. Mus.) or in Musica
Antiqua, vol. i.; and one of his ballads, “Pastime with good company,” is given
as a specimen in the following pages.

In 1533 a proclamation was issued to suppress “fond [foolish] books, ballads,
rhimes, and other lewd treatises in the English tongue;” and in 1537 a man of
the name of John Hogon was arrested for singing a political ballad to the tune of
“The hunt is up.” It was not only among the upper classes that songs and
ballads were then so general, although the allusions to the music of the lower
classes are less frequently to be met with at this period than a little later, when
plays, which give the best insight to the manners and customs of private life, had
become general. One passage, however, from Miles Coverdale’s “Address unto
the Christian reader” prefixed to his “Goastly Psalmes and Spirituall Songes,”
[1538] will suffice to prove it. “Wolde God that our Mynstrels had none other
%\origpage{}
thynge to play upon, neither our \textit{carters} \pagebreak 
and \textit{plowmen} other thynge to whistle
\markboth{henry viii.}{acts of parliament and proclamations against ballads.}
%54
upon, save psalmes, hymns, and such like godly songes... And if women at
the rockes,\dcfootnote{\textit{54.a}
Rock, a distaff: that is, the staff on which flax was
held, when spinning was performed without a wheel; or
the corresponding part of the spinning wheel.—Nares’
\textit{Glossary}.
} and spinnynge at the wheles, had none other songes to pass their tyme
withall, than such as Moses’ sister,... songe before them, they should be better
occupied than with \textit{Hey, nonny, nonny—Hey, trolly, lolly}, and such like fantasies.”
Despite the excellent intent with which this advice was given, it did not evidently
make much impression, either then or after. The traditional tunes of every
country seem as natural to the common people as warbling is to birds in a
state of nature; the carters and ploughmen continued to be celebrated for their
whistling, to the end of the eighteenth century, and the women thought rather with
Ophelia: “You must sing \textit{down, a-down}, an you call him a-down-a, Oh, \textit{how the
wheel becomes it}!”

Anthony à Wood says that Sternhold, who was Groom of the Chamber to
Henry VIII, versified fifty-one of the Psalms, and “caused musical notes to be
set to them, thinking thereby that the courtiers would sing them instead of their
sonnets, but did not, only some few excepted.” They were not, however, printed
till 1549. On the title page it is expressed that they were to be sung “in private
houses, for godly solace and comfort, and for the laying apart all ungodly songes
and ballads.”

Although Henry VIII. had given all possible encouragement to ballads and
songs in the early part of his reign, both in public and private,—and in proof
of their having been used on public occasions, I may mention the coronation of
Anne Boleyn, when a choir of men and boys stood on the leads of St. Martin’s
Church, and sang new ballads in praise of her majesty,—yet, when they were resorted
to as a weapon against the Reformation, or in opposition to any of his own
opinions and varying commands, he adopted the summary process of suppressing
them altogether. It is in some measure owing to that act, but principally to their
perishable nature, that we have no \textit{printed} ballads now remaining of an earlier
date than that on the downfall of his former favorite, Thomas, Lord Cromwell,
which is in the library of the Society of Antiquaries, at Somerset House. The
act, which was passed in 1543, is entitled “An act for the advancement of true
religion, and for the abolishment of the contrary” (Anno 34-35, c. i.), and recites
that “froward and malicious minds, intending to subvert the true exposition of
scripture, have taken upon them, by printed ballads, rhymes, etc., subtilly and
craftily to instruct his highness’ people, and specially the youth of this his realm,
untruly. For reformation whereof, his majesty considereth it most requisite to
purge his realm of all such books, ballads, rhymes, and songs, as be pestiferous
and noisome. Therefore, if any printer shall print, give, or deliver, any such, he
shall suffer for the first time imprisonment for three months, and forfeit for every
copy 10\textit{l}., and for the second time, forfeit all his goods and his body be committed
to perpetual prison.” Although the act only expresses “all such books, ballads,
rhymes, and songs as be pestiferous and noisome,” there is a list of exceptions 
to it, and no ballads of any description \pagebreak
 are excepted. “Provided, also, that
%055
all books printed before the year 1540, entituled Statutes, Chronicles, Canterbury
Tales, Chaucer’s books, Gower’s books, and stories of men’s lives, shall not be
comprehended in the prohibition of this act.” It was not, however, the first time
that ballads had been employed for controversy on religious subjects. The ballads
against the Lollards, and those against the old clergy, have been mentioned at
page 40; and there is a large number extant against monks and friars, many of
which were, and some still are, popular.

The first collection of songs in parts that was \textit{printed} in England, was in 1530;
but of that only a base part now remains.\dcfootnote{
 It contained compositions by Cornish, Pygot, Ashwell,
Taverner, Gwynneth, Jones, Dr. Cowper, and Dr.
Fairfax. See the Index in Ritson’s Ancient Songs,
p. xxiii., last edition, Stafford Smith’s are principally by
Fairfax, Newark, Heath, Turges, Sheringham, and Sir
Thomas Philipps; but this list of composers might be
increased greatly by including those in other manuscripts.
} There are, however, many such collections
in manuscript in public and private libraries. Stafford Smith’s printed
collection of songs in score, composed about the year 1500, is almost entirely
taken from one manuscript.

Henry VIII left a large number of musical instruments at his death, the inventory
of which may be seen in Harl. MSS. No. 1419, fol. 200; and, as might
be expected, all his children were well taught in music.

“Ballads,” says Mr. Collier, “seem to have multiplied after Edward VI. came
to the throne; no new proclamation was issued, nor statute passed on the subject,
while Edward continued to reign; but in less than a month after Mary became
queen, she published an edict against ‘books, ballads, rhymes, and treatises,’
which she complained had been ‘set out by printers and stationers, of an evil
zeal for lucre, and covetous of vile gain.’ There is little doubt, from the few
pieces remaining, that it was, in a considerable degree, effectual for the end
in view.”

\vfill
\centerrule
\vfill

The following tunes are occasionally classed rather under the dates to which
I consider them to belong, than by those of the copies from which they are derived;
but as the authorities are given in every case, the reader has the means before him
of forming his own opinion. Some, however, are classed rather for convenience of
subject, as songs of Robin Hood, songs or tunes mentioned by Shakespeare,~\&c.

After a few from manuscripts of the time of Henry VIII., there are specimens
of “King Henry’s Mirth, or Freemen’s Songs,” from a collection printed in 1609,
which contains many “fine vocal compositions of very great antiquity.”\dcfootnote{
In 1609, Thomas Ravenscroft, Mus. Bac., collected
and printed 100 old Catches, Rounds, and Canons, under
the title of “Pammelia: Musick’s Miscellanie, or mixed
varietie of pleasant Roundelayes and delightful Catches.”
It met with so much success, that in the same year he
published a second, called “Deuteromelia: or the second
part of Musick's Melodie, or melodious musicke of pleasant
Roundelayes, K. H. [\textit{King Henry's] Mirth, or Freemen’s
Songs},”~\&c.; and in 1611, a third collection, called
“Melismata: Musical Phansies, fitting the court, city,
and countrey humours.” Some of the Songs and Catches
in these collections are undoubtedly of the reign of Henry
VII., and it is to be presumed that the authors of all
were unknown to Ravenscroft, as, contrary to custom,
he does not mention them in any instance.
} But
of those, I have only selected such as were also used as song or ballad tunes,
sung by a single voice.

\markboth{english song and ballad music.}{from henry vii. to mary.}
\pagebreak
%056

\musictitle{Pastime with good Company.}

The words and music of this song are preserved in a manuscript of the time of
Henry VIII., formerly in Ritson’s possession, and now in the British Museum
(Add. MSS., 5665); in which it is entitled \textsc{The King’s Ballad}. Ritson
mentions it in a note to his Historical Essay on Scotish Song, and Stafford Smith
printed it in his \textit{Musica Antiqua} in score for three men’s voices. It is the first of
those mentioned in Wedderburn’s \textit{Complaint of Scotland}, which was published in
1549: “Now I will rehearse some of the sweet songs that I heard among them
(the shepherds) as after follows: in the first \textit{Pastance with good Company},” \&c.
The tune is also to be found arranged for the lute (without words) in the volume
among the king’s MSS. before cited (Append. 58), of which “Dominus Johannes
Bray” was at one time the possessor. This may be considered as another proof
of its former popularity.

\noindent\begin{minipage}{\textwidth}
\musictitle{Song by Henry VIII.}
\vspace{-2\baselineskip}

\musicinfo{In moderate time.}{}
\vspace{\baselineskip}

\includemusic{chappellV1007.pdf}
\end{minipage}

  \settowidth{\versewidth}{All thoughts and fantasies to digest,}
  \indentpattern{0000111221}

\begin{dcverse}\begin{patverse}
Youth will needs have dalliance,\\
Of good or ill some pastance;\\
Company me thinketh the best\\
All thoughts and fantasies to digest,\\
For idleness\\
Is chief mistress\\
Of vices all:\\
Then who can say\\
But pass the day\\
Is best of all?
\end{patverse}

\begin{patverse}
Company with honesty\\
Is virtue,—and vice to flee:\\
Company is good or ill,\\
But ev’ry man hath his free will,\\
The best I sue,\\
The worst eschew:\\
My mind shall be\\
Virtue to use\\
Vice to refuse\\
I shall use me.
\end{patverse}
\end{dcverse}

\pagebreak
%057


\musictitle{Ah! The sighs that come fro’ my heart.}

This little love-song is the first in MSS. Reg. Append. 58., of the time of
Henry VIII., and the air is both elegant and expressive. The cadence, or flourish
at the end, is characteristic of the period, and there is a pretty attempt at
musical expression on the words, “fro? my \textit{love} depart.”

\noindent\begin{minipage}{\textwidth}
\musicinfo{Smoothly and with expression.}{}

\includemusic{chappellV1008.pdf}
\end{minipage}

\settowidth{\versewidth}{Farewell my joy! and welcome pain!}
\begin{dcverse}\begin{altverse}
Ah! the sighs that come from my heart,\\
They grieve me passing sore,\\
Sith I must fro’ my love depart,\\
Farewell, my joye, for evermore.
\end{altverse}

\begin{altverse}
Oft to me, with her goodly face,\\
She was wont to cast an eye:\\
And now absence to me in place?\\
Alas! for woe I die, I die!
\end{altverse}

\begin{altverse}
I was wont her to behold,\\
And take in armès twain;\\
And now, with sighès manifold,\\
Farewell my joy! and welcome pain!
\end{altverse}

\begin{altverse}
Ah! me think that should I yet,\\
As would to God that I might!\\
There would no joys compare with it\\
Unto my heart, to make it light.
\end{altverse}
\end{dcverse}

\musictitle{Western wind, when wilt thou blow?}

This is also taken from MSS. Reg. Append. 58, time of Henry VIII. As the
tune appears to be in the ancient Dorian mode, it has been harmonized in that
mode, to preserve its peculiarity of character.

The writer of a quarto volume on ancient Scotish melodies has asserted that
\textit{all} the ancient English music in Ritson’s, or other collections, is of a heavy
drawling character. An assertion so at variance with fact must either have
proceeded from narrow-minded prejudice, or from his not having understood
ancient musical notation. That he could not discriminate between Scotch and
English music is evinced by the fact of his having appropriated some of the best
known English compositions as ancient Scotish melodies.\dcfootnote{
  This writer also cites the authority of Giraldus Cambrensis,
  \textit{who says nothing of the kind}; and in the same
sentence, appropriates what Giraldus says in favour of
\textit{Irish} music to Scotch.
}
\pagebreak
%58

The following song is one of those adduced by him in proof of the drawling of
English music; but I have restored the words to their proper places, and it is by
no means a drawling song. It should be borne in mind that these specimens of
English music are long anterior to any Scotish music that has been produced.

\medskip

\noindent\begin{minipage}{\textwidth}
\musicinfo{Moderate time.}{}

\includemusic{chappellV1009.pdf}
\end{minipage}

\musictitle{Blow thy horn, Hunter!}

This is also copied from MSS. Reg. Append. 58, time of Henry VIII. It is a
spirited tune, and should be sung more quickly in proportion than the others,
because in modernizing the notation, I have only made a crotchet into a quaver,
instead of into a semiquaver, as would have been more correct, considering the
date of the manuscript.

\noindent\begin{minipage}{\textwidth}
\musicinfo{Boldly and well marked.}{}
\vspace{\baselineskip}

\includemusic{chappellV1010.pdf}
\end{minipage}

\pagebreak
%59
\musictitle{The Three Ravens.}

This song is one of those included under the head of “Country Pastimes” in
Melismata, 1611. Ritson in his \textit{Ancient Songs}, remarks: “It will be obvious
that this ballad is much older, not only than the date of that book, but than most
of the other pieces contained in it.” It is nevertheless still so popular in some
parts of the country, that I have been favored with a variety of copies of it,
written down from memory; and all differing in some respects, both as to words
and tune, but with sufficient resemblance to prove a similar origin.

%\vspace{-\baselineskip}
\noindent\begin{minipage}{\textwidth}
\musicinfo{Slowly, smoothly, and with great expression.}{}
\medskip

\includemusic{chappellV1011.pdf}
\end{minipage}

\settowidth{\versewidth}{Down in yonder green field, Down a down, hey down, hey down,}
\begin{scverse}Down in yonder green field, Down a down, hey down, hey down,\\
There lies a knight slain, under his shield. With a down.\\
His hounds they lie down at his feet,\\
So well ‘do’ they their master keep. With a down, derry, \&c.

His hawks they fly so eagerly, Down a down, \&c.\\
There’s no fowl ‘that’ dare him come nigh. With a down.\\
Down there comes a fallow doe,\\
As great with young as she might go. With a down, derry, \&c.

She lifted up his bloody head, Down a down, \&c.\\
And kiss’d his wounds that were so red; With a down.\\
She got him up upon her back,\\
And carried him to earthen lake. With a down, \&c.

She buried him before the prime: With a down, \&c.\\
She was dead herself ere even-song time. With a down.\\
God send every gentleman\\
Such hawks, such hounds, and such a leman [lov’d one]. With a down, \&c.
\end{scverse}

\pagebreak
%60

\musictitle{The Hunt Is Up}

Among the favorites of Henry the Eighth, Puttenham notices “one Gray,
what good estimation did he grow unto with the same King Henry, and afterwards
with the Duke of Somerset, Protectour, for making certaine merry ballades,
whereof one chiefly was, \textit{The hunte is up, the hunte is up}.” Perhaps it was the
same William Gray who wrote a ballad on the downfall of Thomas Lord Cromwell
in 1540, to which there are several rejoinders in the library of the Society of
Antiquaries. The tune \textit{The Hunt is up} was known as early as 1537, when
information was sent to the Council against one John Hogon, who had offended
against the proclamation of 1533, which was issued to suppress “fond books,
ballads, rhimes, and other lewd treatises in the English tongue,” by singing,
“with a crowd or a fyddyll,” a political song to that tune. Some of the words
are inserted in the information, but they were taken down from recitation, and are
not given as verse (see Collier’s Shakespeare, i., p. cclxxxviii.) In the Complaint
of Scotland, 1549, \textit{The Hunt is up} is mentioned as a tune for dancing, for which,
from its lively character, it seems peculiarly suited; and Mr. Collier has a MS.
which contains a song called “The Kinges Hunt is upp,” which may be the very
one written by Gray, since “Harry our King” is twice mentioned in it, and a
religious parody as old as the reign of Henry VIII. is in precisely the same
measure. The following is the song:—

\noindent\begin{minipage}{\textwidth}
\musictitle{The Kinges Hunt is Upp}
\backskip{2}
\musicinfo{Merrily.}{}

\includemusic{chappellV1012.pdf}
\end{minipage}

\settowidth{\versewidth}{The east is bright with morning light,}
  \begin{dcverse}    \begin{altverse}
    The east is bright with morning light,\\
And darkness it is fled,\\
And the merie home wakes up the morne\\
To leave his idle bed.
\end{altverse}

\begin{altverse}{Beholde the skyes with golden dyes\\
Are glowing all around,\\
The grasse is greene, and so are the treene,\\
All laughing at the sound.}\end{altverse}

\begin{altverse}{The horses snort to be at the sport,\\
The dogges are running free,\\
The woddes rejoyce at the mery noise\\
Of hey tantara tee ree!}\end{altverse}

\begin{altverse}{The sunne is glad to see us clad\\
All in our lustie greene,\\
And smiles in the skye as he riseth hye,\\
To see and to he seene.}\end{altverse}

\end{dcverse}


\settowidth{\versewidth}{Awake, all men, I say agen,}
\begin{scverse}    \begin{altverse}{Awake, all men, I say agen,\\
Be mery as you maye,\\
For Harry our Kinge is gone hunting,\\
To bring his deere to baye.}\end{altverse}
\end{scverse}
\pagebreak
%61
\changefontsize{1.01\defaultfontsize}
The tune is taken from \textit{Musick’s delight on the Cithren}, edition of 1666, which
contains many very old and popular tunes, such as “Trip, and go,” and “Light
o’ Love” (both mentioned by Shakespeare), which I have not found in any other
printed collection. Ritson, in his Ancient Songs, quotes the following song of
one verse, which is in the same measure, and was therefore probably sung to the
same tune. It may be found in \textit{Merry Drollery Complete}, 1661, and the \textit{New
Academy of Complements}, 1694 and 1713.

\settowidth{\versewidth}{“The hunt is up, the hunt is up,}
\begin{scverse}\begin{altverse}
\vleftofline{“}The hunt is up, the hunt is up,\\
And now it is almost day;\\
And he that’s ‘at home, in bed with his wife,’\\
’Tis time to get him away.”
\end{altverse}
\end{scverse}

Any song intended to arouse in the morning—even a love-song—was formerly
called a \textit{hunt’s-up}. Shakespeare so employs it in \textit{Romeo and Juliet}, Act 3, Sc. 5;
and the name was of course derived from a tune or song employed by early
hunters. Butler, In his \textit{Principles of Musik}, 1636, defines a \textit{hunt’s-up} as
“morning music;” and Cotgrave defines “Resveil” as a hunt’s-up, or \textit{Morning
Song} for a new-married wife. In Barnfield’s \textit{Affectionate Shepherd}, 1594,—

\settowidth{\versewidth}{“And every morn by dawning of the day,}
\indentpattern{010100}
\begin{scverse}\begin{patverse}
\vleftofline{“}And every morn by dawning of the day,\\
When Phoebus riseth with a blushing face,\\
Silvanus’ chapel clerks shall chaunt a lay,\\
And play thee \textit{hunt’s-up} in thy resting place.\\
My cot thy chamber, my bosòm thy bed,\\
Shall be appointed for thy sleepy head.”
\end{patverse}
\end{scverse}

Again, in \textit{Wit's Bedlam}, 1617,—

%\settowidth{\versewidth}{“Maurus, last morne, at’s mistress’ window plaid}
\begin{scverse}\vleftofline{“}Maurus, last morne, at’s mistress’ window plaid\\
An hunt’s-up on his lute,” \&c.
\end{scverse}

The following song, which is also taken from Mr. Collier’s manuscript, is of
the character of a love-song:—
\musictitle{The New Hunt’s-up}
\vspace{-\baselineskip}

\settowidth{\versewidth}{The hunt is up, the hunt is up,}
\begin{dcverse}\begin{altverse}
The hunt is up, the hunt is up,\\
Awake, my lady free,\\
The sun hath risen, from out his prison,\\
Beneath the glistering sea.
\end{altverse}

\begin{altverse}
The hunt is up, the hunt is up,\\
Awake, my lady bright,\\
The morning lark is high, to mark\\
The coming of day-light.
\end{altverse}

\begin{altverse}
The hunt is up, the hunt is up,\\
Awake, my lady fair,\\
The kine and sheep, but now asleep,\\
Browse in the morning air.
\end{altverse}

\begin{altverse}
The hunt is up, the hunt is up,\\
Awake, my lady gay,\\
The stars are fled to the ocean bed,\\
And it is now broad day.
\end{altverse}

\begin{altverse}
The hunt is up, the hunt is up,\\
Awake, my lady sheen,\\
The hills look out, and the woods about,\\
Are drest in lovely green.
\end{altverse}

\begin{altverse}
The hunt is up, the hunt is up,\\
Awake, my lady dear,\\
A morn in spring is the sweetest thing\\
Cometh in all the year.
\end{altverse}
\end{dcverse}

\begin{scverse}\begin{altverse}
The hunt is up, the hunt is up,\\
Awake, my lady sweet,\\
I come to thy bower, at this lov'd hour,\\
My own true love to greet.
\end{altverse}
\end{scverse}
\normalsize
\pagebreak
%62

The religious parody of \textit{The Hunt is up}, which was written by John Thorne,
has been printed by Mr. Halliwell, at the end of the moral play of \textit{Wit and
Science}, together with other curious songs from the same manuscript (Addit. MS.,
No.~15,233, Brit. Mus.) There are seventeen verses; the first is as follows:—

\settowidth{\versewidth}{“The hunt ys up, the hunt ys up,}
\begin{scverse}
\begin{altverse}
\vleftofline{“}\vleftofline{“}The hunt ys up, the hunt ys up,\\
Loe! it is allmost daye;\\
For Christ our Kyng is cum a huntyng,\\
And browght his deare to staye,” \&c.
\end{altverse}
\end{scverse}
but a more lively performance is contained in “Ane compendious booke of Godly
and Spirituall Songs... with sundrie... ballates changed out of prophaine
Sanges,” \&c., printed by Andro Hart in Edinburgh in 1621. The writer is very
bitter against the Pope, who, he says, never ceased, “under dispence, to get our
pence,” and who sold “remission of sins in auld sheep skins;” and compares
him to the fox of the hunt. The original edition of that book was printed in 1590.

In Queen Elizabeth’s and Lady Neville’s Virginal Books, is a piece, with twelve
variations, by Byrde, called “The Hunt is up,” which is also called “Pescod
Time,” in another part of the former book. It bears no appearance of ever having
been intended for words; certainly the songs in question could not be sung
to it.

A tune called \textit{The Queene’s Majesties new Hunt is up}, is mentioned in Anthony
Munday’s \textit{Banquet of daintye conceits}, 1588; and the ditty he gives, to be sung
to it, called “Women are strongest, but truth overcometh all things,” is in the
same measure as the above, but I have not found any copy of the tune under that
name. In 1565, William Pickering paid 4\textit{d}. for a license to print “a ballett
intituled The Hunte ys up,” \&c. (see \textit{Registers of Stationers’ Company}, p. 129).

\musictitle{Yonder comes a Courteous Knight.}

This is one of King Henry’s Mirth or Freemen’s Songs, in Deutero\-melia, 1609,
and is to be found as a ballad in Wit and Mirth, or Pills to purge Melancholy,
vol.~i.~1698 and 1707, or in vol. iii. of the edition of 1719. The story seems to
have been particularly popular, as there are three ballads of later date upon the
same subject. It is of a young lady who, being alone and unprotected, finds the
too urgent addresses of a knight likely to prove troublesome; and, to escape
from that position, pretends to yield to him, and persuades him to escort her
home; but—

\settowidth{\versewidth}{“When she came to her father’s hall,}
\indentpattern{01014}
\begin{scverse}
\begin{patverse}
\vleftofline{“}When she came to her father’s hall,\\
It was well walled round about,\\
She yode in at the wicket gate,\\
And shut the four-ear’d fool without.\\
Then she sung down, a-down,” \&c.
\end{patverse}
\end{scverse}
The knight, regretting the lost opportunity, expresses himself in very uncourteous
terms on the deceit of women. The ballad is printed in Ritson’s Ancient Songs.
%\origpage{}%063
\origpage{63}%63

\noindent\begin{minipage}{\textwidth}
\musicinfo{Gracefully.}{}
\bigskip

\includemusic{chappellV1013.pdf}
\end{minipage}

\musictitle{Oft have I ridden upon my Grey Nag}

This is evidently a version of the tune called \textit{Dargason}. (See p. 65.) The latter
part differs, but that may be because this copy is taken from Pammelia, 1609,
where three old tunes, “Shall I go walk the woods so wild,” “Robin Hood, Robin
Hood, said Little John,” and this, are arranged to be sung together by three
persons at the same time. Perhaps, the two lines from the Isle of Gulls, which
are quoted at page 64, formed a portion of this song. Only one verse is given in
Pammelia, and I have not succeeded in finding any other copy.

\includemusic{chappellV1014.pdf}

\pagebreak